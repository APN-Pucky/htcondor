% Define a registered trademark sign for something
\newcommand{\Reg}[1]{{#1}\textsuperscript{\scriptsize{\textregistered}}}

%
%  Set up version, author and copyright notices
%
\newcommand{\AuthorNotice}{Condor Team, University of Wisconsin--Madison}
\newcommand{\VersionNotice}{Version 7.4.1}
\newcommand{\CondorR}{\Reg{Condor}}

\newcommand{\CopyrightNotice}{
  Copyright \copyright\ 1990-2009 Condor Team, Computer Sciences Department, 
  University of Wisconsin-Madison, Madison, WI.  All Rights Reserved.  
	Licensed under the Apache License, Version 2.0.

  See the \emph{Condor \VersionNotice\ Manual} or
	\URL{http://www.condorproject.org/license} for
  additional notices. 
}


%
%  Common motifs
%
\newcommand{\Prog}[1]{\textit{#1}}              	% Program name
\newcommand{\Term}[1]{\emph{#1}}			% Use this to introduce terminology
\newcommand{\Cmd}[2]{\textit{#1}(#2)}           	% Command w/ section number
\newcommand{\Sinful}[1]{$<$#1$>$}         	% Sinful string
\newcommand{\SinfulAny}{$<$a.b.c.d:port$>$}        	% Arbitrary Sinful string
\newcommand{\URL}[1]{\htmladdnormallink{#1}{#1}}	% a URL
\newcommand{\Email}[1]{\htmladdnormallink{#1}{mailto:#1}}
\newcommand{\File}[1]{\texttt{#1}}        	    	% File name
\newcommand{\Bs}{$\mathtt{\backslash}$}         	% Backslash
% If we don't use math mode, latex2html can use the aux files generated for
% the PDF.  Doing so also eliminates a symlink.
\newcommand{\Bar}{{\tt |}}
\newcommand{\Dots}{\dots}
	% This brings in \Bar |, \Dots ...
\newcommand{\Tilde}{\~{}}                 	% tilde
\newcommand{\Circum}{\^{}}                      	% ^
\newcommand{\Lbr}{[}                            	% [
\newcommand{\Rbr}{]}                            	% ]
\newcommand{\Percent}{\%}                       	% %
\newcommand{\Opt}[1]{\mbox{\textbf{#1}}}            % Option
\newcommand{\Arg}[1]{\mbox{\textit{#1}}}            % Argument
\newcommand{\OptArg}[2]{\mbox{\Opt{#1\ }\Arg{#2}}}  % Option with Argument
\newcommand{\oArg}[1]{\mbox{[\Arg{#1}]}}            % optional Argument
\newcommand{\oOpt}[1]{\mbox{[\Opt{#1}]}}            % optional Option
\newcommand{\oOptArg}[2]{\mbox{[\OptArg{#1\ }{#2}]}}% optional Option w/ Arg
\newcommand{\Optnm}[1]{\textbf{#1}}            % Option w/o mbox
\newcommand{\Argnm}[1]{\textit{#1}}            % Argument w/o mbox
\newcommand{\OptArgnm}[2]{\Optnm{#1\ }\Argnm{#2}}% Option with Argument w/o mbox
\newcommand{\oArgnm}[1]{[\Argnm{#1}]}            % optional Argument w/o mbox
\newcommand{\oOptnm}[1]{[\Optnm{#1}]}            % optional Option w/o mbox
\newcommand{\oOptArgnm}[2]{[\OptArgnm{#1\ }{#2}]}% optional Option w/Arg w/o mbox
\newcommand{\Env}[1]{\texttt{#1}}		% Environment variable
\newcommand{\DCPerm}[1]{\texttt{#1}}		% DC Permission
\newcommand{\ShortExpr}[1]{\mbox{\texttt{#1}}}		% A pithy config file expression
\newcommand{\Expr}[1]{\texttt{#1}}		% Config file expression

\newcommand{\MacroNI}[1]{\texttt{#1}}		% Config file macro, NO index
\newcommand{\Macro}[1]{\texttt{#1} \index{#1 macro@\texttt{#1} macro} \index{configuration macro!\texttt{#1}}}		% Config file macro and index
\newcommand{\MacroIndex}[1]{\index{#1 macro@\texttt{#1} macro} \index{configuration macro!\texttt{#1}}}		% index only of config file macro
\newcommand{\MacroB}[1]{\texttt{#1} \index{configuration macro!\texttt{#1}}}		% bracketted Config file macro and index

\newcommand{\MacroUNI}[1]{\texttt{\$(#1)}}	% Config file macro's use, NO index
\newcommand{\MacroU}[1]{\texttt{\$(#1)}\index{#1 macro@\texttt{#1} macro}\index{configuration macro!\texttt{#1}}}	% Config file macro's use and index
\newcommand{\AdAttr}[1]{\texttt{#1}}	% ClassAd Attribute name
\newcommand{\Attr}[1]{\texttt{#1}}	% ClassAd Attribute name again
\newcommand{\AdStr}[1]{\texttt{"#1"}}	% ClassAd Attribute string
\newcommand{\Dflag}[1]{\texttt{D\_{#1}}}		% Debug flag name
\newcommand{\Bold}[1]{\textbf{#1}}		% Something you want bold
\input{fontsize.tex}
\newcommand{\Note}{\underline{NOTE}: }          % NOTE:
\newcommand{\Warn}{\underline{WARNING}: }       % WARNING:
\newcommand{\Security}{\underline{\textit{Security Item}}: } % To identify release notes that pertain to security items
\newcommand{\Todo}{\begin{center} \fbox{This section has not yet been written} \end{center}}
\newcommand{\MoreTodo}{\begin{center} \fbox{This section has not yet been completed} \end{center}}
\newcommand{\Syscall}[1]{\texttt{#1()}}		% The name of a syscall
\newcommand{\Login}[1] {{\tt #1}}       % login name like root or nobody
\newcommand{\Username}[1]{``\textbf{#1}''}		% A username for an account

% A keyword in a meta-language
\newcommand{\Keyword}[1] {{\tt #1}}

% The name of a procedure
\newcommand{\Procedure}[1] {{\tt #1()}}

% ==== C/C++ functions & methods ====

% \BaseFunctionDef {1:name} {2:return type} {3:return description}
%   {4:static/{}} {5:const/{}}
%   {6:parameter list} {7:synopsys} {8:Method/Function}
\newcommand{\BaseFunctionDef}[8]
 { {\tt{#4}{#2}} {\tt{#1}({#6})} {#5}
  \\ {\textbf{Synopsis:} {#7}}
  \\ {\textbf{Returns:} {#2}{#3}}
  \\ {\textbf{#8} parameters:} }
% \MethodDef {1:class} {2:method} {3:return type} {4:return description}
%   {5:parameter list} {6:synopsys}
\newcommand{\MethodDef}[6]
 { \BaseFunctionDef{{#1}::{#2}} {#3} {; #4} {}   {}      {#5} {#6} {Method}  }
\newcommand{\ConstMethodDef}[6]
 { \BaseFunctionDef{{#1}::{#2}} {#3} {; #4} {} {\Type{const}}   {#5} {#6} {Method}  }
\newcommand{\StaticMethodDef}[6]
 { \BaseFunctionDef{{#1}::{#2}} {#3} {; #4} {static } {} {#5} {#6} {Method}  }
% \FunctionDef {1:function} {2:return type} {3:return description}
%   {4:static/{}} {5:parameter list} {6:synopsys}
\newcommand{\FunctionDef}[6]
 { \BaseFunctionDef{#1} {#2} {; #3} {#4} {#5} {} {#6} {Function} }
\newcommand{\MethodRef}[2]   {{\tt{{#1}::{#2}}}}
\newcommand{\FunctionRef}[1] {{\tt{#1}}}

% \Constructor {class} {param list} {synopsis}
\newcommand{\Constructor}[3]
 { \BaseFunctionDef {{#1}::{#1}}     {} {None} {} {} {#2}
    {Constructor {#3}} {Constructor} }
% \Destructor {class}
\newcommand{\Destructor}[1]
 { \BaseFunctionDef {{#1}::{\~{}#1}} {} {None} {} {} {void}
    {Destructor} {Destructor}
    \begin{itemize}\item None. \end{itemize} }

% ==== Function parameters ====

% Optional param
\newcommand{\ParamDefOpt}[3]
 { \texttt{#2} \texttt{#1} (\textit{Optional with default} = {#3}) \\ }
% Required parameter
\newcommand{\ParamDef}[2]    { \texttt{#2} \texttt{#1} \\ }
% Parameter reference
\newcommand{\Param}[1]       {\texttt{#1}}
% Constant expression
\newcommand{\Const}[1]       {\texttt{#1}}
% Custom type
\newcommand{\Type}[1]        {\texttt{#1}}


% A program code snippet
%\newcommand{\Code}[1] {{\tt #1}}
\newcommand{\Code}[1]{\texttt{#1}}		% code in courier font

% A command that is in a submit description file 
\newcommand{\SubmitCmd}[1]{\textbf{#1}}		% Submit command

% Make a nice box with a header to point out a tricky feature
\newcommand{\Notice}[1] {\noindent {\bf Notice: }\\ \fbox{\parbox[t]{\textwidth}{#1}}}

% Release directory entry
\newcommand{\Release}[1]{\texttt{$<$release\_dir$>$/#1}}

%
% Talking about SQL entities
%
% Name of a Table
\newcommand{\SQLTable}[1] {\Bold{#1}}
% Defining a Table
\newcommand{\SQLTableDef}[2] {\SQLTable{#1} \Bold{(}#2\Bold{)}}
% Name of a View
\newcommand{\SQLView}[1] {\Bold{#1}}
% Defining a View
\newcommand{\SQLViewDef}[2] {\Keyword{CREATE VIEW} \SQLView{#1} \Keyword{as} #2\Bold{;}}



%
% This sets the BODY tag when converted to HTML.
% It has no effect on the DVI file.
\bodytext{ BGCOLOR=#FFFFFF }

%
%  To help with typing in names of condor commands
%    e.g., condor_submit == \Condor{submit}
\newcommand{\Condor}[1]{\Prog{condor\_{#1}}}
\newcommand{\condor}[1]{condor\_{#1}}

%
%  To help with typing in names of stork commands
%    e.g., stork_submit == \Stork{submit}
%\newcommand{\Stork}[1]{\Prog{stork\_{#1}}}
%\newcommand{\stork}[1]{stork\_{#1}}
