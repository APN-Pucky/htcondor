%
%  Set up version, author and copyright notices
%
\newcommand{\AuthorNotice}{Condor Team, University of Wisconsin--Madison}
\newcommand{\VersionNotice}{Version 6.5.0}
\newcommand{\CopyrightNotice}{Copyright \copyright\ 1990-2003 Condor Team, Computer Sciences Department, 
  University of Wisconsin-Madison, Madison, WI.  All Rights Reserved.  
  No use of the Condor Software Program is authorized 
  without the express consent of the Condor Team.  For more information 
  contact: Condor Team, Attention: Professor Miron Livny, 
  7367 Computer Sciences, 1210 W. Dayton St., Madison, WI 53706-1685, 
  (608) 262-0856 or miron@cs.wisc.edu.
 
  U.S. Government Rights Restrictions: Use, duplication, or disclosure 
  by the U.S. Government is subject to restrictions as set forth in 
  subparagraph (c)(1)(ii) of The Rights in Technical Data and Computer 
  Software clause at DFARS 252.227-7013 or subparagraphs (c)(1) and 
  (2) of Commercial Computer Software-Restricted Rights at 48 CFR 
  52.227-19, as applicable, Condor Team, Attention: Professor Miron 
  Livny, 7367 Computer Sciences, 1210 W. Dayton St., Madison, 
  WI 53706-1685, (608) 262-0856 or miron@cs.wisc.edu. 

  See the \emph{Condor \VersionNotice\ Manual} for
  additional notices. }

%
%  Common motifs
%
\newcommand{\Prog}[1]{\textit{#1}}              	% Program name
\newcommand{\Term}[1]{\emph{#1}}			% Use this to introduce terminology
\newcommand{\Cmd}[2]{\textit{#1}(#2)}           	% Command w/ section number
\newcommand{\Sinful}[1]{$<$#1$>$}         	% Sinful string
\newcommand{\SinfulAny}{$<$a.b.c.d:port$>$}        	% Arbitrary Sinful string
\newcommand{\URL}[1]{\htmladdnormallink{#1}{#1}}	% a URL
\newcommand{\Email}[1]{\htmladdnormallink{#1}{mailto:#1}}
\newcommand{\File}[1]{\texttt{#1}}        	    	% File name
\newcommand{\Bs}{$\mathtt{\backslash}$}         	% Backslash
% If we don't use math mode, latex2html can use the aux files generated for
% the PDF.  Doing so also eliminates a symlink.
\newcommand{\Bar}{{\tt |}}
\newcommand{\Dots}{\dots}
	% This brings in \Bar |, \Dots ...
\newcommand{\Tilde}{\~{}}                 	% tilde
\newcommand{\Circum}{\^{}}                      	% ^
\newcommand{\Lbr}{[}                            	% [
\newcommand{\Rbr}{]}                            	% ]
\newcommand{\Percent}{\%}                       	% %
\newcommand{\Opt}[1]{\mbox{\textbf{#1}}}            % Option
\newcommand{\Arg}[1]{\mbox{\textit{#1}}}            % Argument
\newcommand{\OptArg}[2]{\mbox{\Opt{#1\ }\Arg{#2}}}  % Option with Argument
\newcommand{\oArg}[1]{\mbox{[\Arg{#1}]}}            % optional Argument
\newcommand{\oOpt}[1]{\mbox{[\Opt{#1}]}}            % optional Option
\newcommand{\oOptArg}[2]{\mbox{[\OptArg{#1\ }{#2}]}}% optional Option w/ Arg
\newcommand{\Optnm}[1]{\textbf{#1}}            % Option w/o mbox
\newcommand{\Argnm}[1]{\textit{#1}}            % Argument w/o mbox
\newcommand{\OptArgnm}[2]{\Optnm{#1\ }\Argnm{#2}}% Option with Argument w/o mbox
\newcommand{\oArgnm}[1]{[\Argnm{#1}]}            % optional Argument w/o mbox
\newcommand{\oOptnm}[1]{[\Optnm{#1}]}            % optional Option w/o mbox
\newcommand{\oOptArgnm}[2]{[\OptArgnm{#1\ }{#2}]}% optional Option w/Arg w/o mbox
\newcommand{\Env}[1]{\texttt{#1}}		% Environment variable
\newcommand{\DCPerm}[1]{\texttt{#1}}		% DC Permission
\newcommand{\ShortExpr}[1]{\mbox{\texttt{#1}}}		% A pithy config file expression
\newcommand{\Expr}[1]{\texttt{#1}}		% Config file expression
\newcommand{\MacroNI}[1]{\texttt{#1}}		% Config file macro, NO index
\newcommand{\Macro}[1]{\texttt{#1} \index{#1 macro@\texttt{#1} macro} \index{configuration macro!\texttt{#1}}}		% Config file macro and index
\newcommand{\MacroUNI}[1]{\texttt{\$(#1)}}	% Config file macro's use, NO index
\newcommand{\MacroU}[1]{\texttt{\$(#1)}\index{#1 macro@\texttt{#1} macro}\index{configuration macro!\texttt{#1}}}	% Config file macro's use and index
\newcommand{\AdAttr}[1]{\texttt{#1}}	% ClassAd Attribute name
\newcommand{\Attr}[1]{\texttt{#1}}	% ClassAd Attribute name again
\newcommand{\AdStr}[1]{\texttt{"#1"}}	% ClassAd Attribute string
\newcommand{\Dflag}[1]{\texttt{D\_#1}}		% Debug flag name
\newcommand{\Bold}[1]{\textbf{#1}}		% Something you want bold
\input{fontsize.tex}
\newcommand{\Note}{\underline{NOTE}: }          % NOTE:
\newcommand{\Warn}{\underline{WARNING}: }       % WARNING:
\newcommand{\Todo}{\begin{center} \fbox{This section has not yet been written} \end{center}}
\newcommand{\Syscall}[1]{\texttt{#1()}}		% The name of a syscall

% A keyword in a meta-language
\newcommand{\Keyword}[1] {{\tt #1}}

% The name of a procedure
\newcommand{\Procedure}[1] {{\tt #1\(\)}}

% A program code snippet
\newcommand{\Code}[1] {{\tt #1}}

% Make a nice box with a header to point out a tricky feature
\newcommand{\Notice}[1] {\noindent {\bf Notice: }\\ \fbox{\parbox[t]{\textwidth}{#1}}}

% Release directory entry
\newcommand{\Release}[1]{\texttt{$<$release\_dir$>$/#1}}


%
% This sets the BODY tag when converted to HTML.
% It has no effect on the DVI file.
\bodytext{ BGCOLOR=#FFFFFF }

%
%  To help with typing in names of condor commands
%    e.g., condor_submit == \Condor{submit}
\newcommand{\Condor}[1]{\Prog{condor\_#1}}
\newcommand{\condor}[1]{condor\_#1}

