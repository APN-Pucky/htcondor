\begin{ManPage}{\label{man-condor-load-history}\Condor{load\_history}}{1}
{Read a Condor history file into a Quill database}
\Synopsis \SynProg{\Condor{load\_history}}
\OptArg{-f}{historyfilename}
\oOptArg{-name}{schedd-name jobqueue-birthdate}

\index{Condor commands!condor\_load\_history}
\index{condor\_load\_history command}

\Description
\Condor{load\_history} reads a Condor history file,
adding its information to a Quill database.
The Quill database is located via configuration variables.
The history file to read is defined by the required
\OptArg{-f}{historyfilename} argument.

The combination of a \Condor{schedd} daemon's name together
with its process creation date
(the job queue's birthdate) define a unique identifier that
may be attached to the Quill database with the
\Opt{-name} option.
The format of birthdate expected is exactly the first
line of the \File{job\_queue.log} file.
The location of this file may be determined using
\footnotesize
\begin{verbatim}
% condor_config_val spool
\end{verbatim}
\normalsize

Be aware and expect that the reading and processing of a sizable
history file may take a large amount of time.

\begin{Options}
  \OptItem{\OptArg{-name}{schedd-name jobqueue-birthdate} }
    {The \Arg{schedd-name} and \Arg{jobqueue-birthdate} combine to form
    a unique name for the database.  The expected values are the
    name of the \Condor{schedd} daemon and the first line
     of the \File{job\_queue.log} file, which gives a job queue
     creation time. }
\end{Options}


\ExitStatus

\Condor{load\_history} will exit with a status value of 0 (zero) upon success,
and it will exit with the value 1 (one) upon failure.

\end{ManPage}
