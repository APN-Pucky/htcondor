\begin{ManPage}{\label{man-condor-ping}\Condor{ping}}{1}
{Attempt a security negotiation to determine if it succeeds}
\index{HTCondor commands!condor\_ping}
\index{condor\_ping command}

\Synopsis \SynProg{\Condor{ping}}
\ToolArgsBase

\SynProg{\Condor{ping}}
\oOpt{-debug}
\oOptArg{-address}{<a.b.c.d:port>}
\oOptArg{-pool}{host name}
\oOptArg{-name}{daemon name}
\oOptArg{-type}{subsystem}
\oOptArg{-config}{filename}
\oOpt{-quiet | -table | -verbose}
\Arg{token}
\oArg{token [\Dots]}

\Description
\Condor{ping} attempts a security negotiation to discover whether the
configuration is set such that the negotiation succeeds.
The target of the negotiation is defined by one or a combination of 
the \Opt{address}, \Opt{pool}, \Opt{name}, or \Opt{type} options.
If no target is specified,
the default target is the \Condor{schedd} daemon on the local machine.

One or more \Arg{token}s may be listed,
thereby specifying one or more authorization level to impersonate in
security negotiation.
A token is the value \Expr{ALL}, an authorization level,
a command name, or the integer value of a command.
The many command names and their associated integer values will more
likely be used by experts,
and they are defined in the file \File{condor\_includes/condor\_commands.h}. 

An authorization level may be one of the following strings.
If \Expr{ALL} is listed, 
then negotiation is attempted for each of these possible authorization levels.
\begin{description}
\item[READ]
\item[WRITE]
\item[ADMINISTRATOR]
\item[SOAP]
\item[CONFIG]
\item[OWNER]
\item[DAEMON]
\item[NEGOTIATOR]
\item[ADVERTISE\_MASTER]
\item[ADVERTISE\_STARTD]
\item[ADVERTISE\_SCHEDD]
\item[CLIENT]
\end{description}

\begin{Options}
  \ToolArgsBaseDesc
  \OptItem{\Opt{-debug}}{Print extra debugging information as the command
    executes.}
  \OptItem{\OptArgnm{-config}{filename}}
    {Attempt the negotiation based on the contents of the configuration file
    contents in file \Arg{filename}.  }
  \OptItem{\OptArgnm{-address}{<a.b.c.d:port>}}
    {Target the given IP address with the negotiation attempt.  }
  \OptItem{\OptArgnm{-pool}{hostname}}
    {Target the given \Arg{host} with the negotiation attempt. 
    May be combined with specifications defined by \Opt{name} 
    and \Opt{type} options. }
  \OptItem{\OptArgnm{-name}{daemonname}}
    {Target the daemon given by \Arg{daemonname} with the negotiation attempt. 
    }
  \OptItem{\OptArgnm{-type}{subsystem}}
    {Target the daemon identified by \Arg{subsystem}, 
    one of the values of the predefined \MacroUNI{SUBSYSTEM} macro. }
  \OptItem{\Opt{-quiet}}{Set exit status only; no output displayed.}
  \OptItem{\Opt{-table}}{Output is displayed with one result per line,
    in a table format.}
  \OptItem{\Opt{-verbose}}{Display all available output.}
\end{Options}

\Examples

The example Unix command
\begin{verbatim}
condor_ping  -address "<127.0.0.1:9618>" -table READ WRITE DAEMON
\end{verbatim}
places double quote marks around the sinful string to prevent the
less than and the greater than characters from causing redirect of
input and output.
The given IP address is targeted with 3 attempts to negotiate:
one at the \Expr{READ} authorization level, 
one at the \Expr{WRITE} authorization level, 
and one at the \Expr{DAEMON} authorization level.

\ExitStatus

\Condor{ping} will exit with the status value of the negotiation
it attempted,
where 0 (zero) indicates success, and 1 (one) indicates failure.
If multiple security negotiations were attempted, 
the exit status will be the logical OR of all values.

\end{ManPage}
