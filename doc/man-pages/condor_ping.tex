\begin{ManPage}{\label{man-condor-ping}\Condor{ping}}{1}
{Attempt a security negotiation to determine if it succeeds}
\Synopsis \SynProg{\Condor{ping}}
\ToolArgsBase

\SynProg{\Condor{ping}}
\oOpt{-debug}
\oOptArg{-address}{<a.b.c.d:port>}
\oOptArg{-pool}{host name}
\oOptArg{-name}{daemon name}
\oOptArg{-type}{subsystem}
\oOptArg{-config}{filename}
\oOpt{-quiet | -table | -verbose}
\Arg{token}
\oArg{token [\Dots]}

\index{HTCondor commands!condor\_ping}
\index{condor\_ping command}

\Description
\Condor{ping} attempts a security negotiation to discover whether the
configuration is set such that the negotiation succeeds.
The target of the negotiation is defined by one of 
the blah blah blah options.
If no target is specified,
the default target is the \Condor{schedd} daemon on the local machine.

One or more \Arg{token}s may be listed,
thereby specifying one or more authorization level to impersonate in
security negotiation.
A token is the string \Expr{ALL}, an authorization level,
a command name, or the integer value of a command.
The many command names and their associated integer values will more
likely be used by experts,
and they are defined in the file \File{condor\_includes/condor\_commands.h}. 

An authorization level may be one of the following strings:
\begin{description}
\item[READ]
\item[WRITE]
\item[ADMINISTRATOR]
\item[SOAP]
\item[CONFIG]
\item[OWNER]
\item[DAEMON]
\item[NEGOTIATOR]
\item[ADVERTISE\_MASTER]
\item[ADVERTISE\_STARTD]
\item[ADVERTISE\_SCHEDD]
\item[CLIENT]
\end{description}



\begin{Options}
    \ToolArgsBaseDesc
    \OptItem{\Opt{-debug}}{Print extra debugging information as the command
            executes.}
    \OptItem{\OptArgnm{-config}{file name}}
            {Definition not yet written.
	    }
    \OptItem{\OptArgnm{-address}{<a.b.c.d:port>}}
            {Definition not yet written.
	    }
    \OptItem{\OptArgnm{-pool}{host name}}
            {Definition not yet written.
	    }
    \OptItem{\OptArgnm{-name}{daemon name}}
            {Definition not yet written.
	    }
    \OptItem{\OptArgnm{-type}{subsystem}}
            {Definition not yet written.
	    }
    \OptItem{\Opt{-quiet}}{Set exit status only; no output displayed.}
    \OptItem{\Opt{-table}}{Output is displayed with one result per line,
            in a table format.}
    \OptItem{\Opt{-verbose}}{Display all available output.}
\end{Options}

\Examples

What it does with the Unix formatted
\begin{verbatim}
condor_ping  -address "<127.0.0.1:9618>" -table READ WRITE DAEMON
\end{verbatim}
Double quote marks surround the sinful string to prevent the
less than and the greater than characters from causing redirect of
input and output.

\ExitStatus

\Condor{ping} will exit with the status value of the negotiation
it attempted,
where 0 (zero) indicates success, and 1 (one) indicates failure.
If multiple security negotiations were attempted, 
the exit status will be the logical OR of all values.

\end{ManPage}
