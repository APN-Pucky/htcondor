\begin{ManPage}{\label{man-condor-cold-start}\Condor{cold\_start}}{1}
{installs and starts condor on this machine}

\Synopsis \SynProg{\Condor{cold\_start}}
\oOpt{-help}
\OptArg{-configfile}{file}
\oOptArg{-basedir}{directory}
\oOpt{-force}
%\oOpt{-dyn}
\oOpt{\Opt{-setuponly} \Bar \Opt{-runonly}}
\oOptArg{-arch}{architecture}
\oOptArg{-site}{repository}
\oOptArg{-localdir}{directory}
\oOptArg{-runlocalconfig}{file}
\oOptArg{-logarchive}{archive}
\oOptArg{-spoolarchive}{archive}
\oOptArg{-execarchive}{archive}
\oOpt{-flock}
\oOpt{-pid}
\oOptArg{-artifact}{filename}
\oOpt{-wget}
\oOptArg{-globuslocation}{directory}

\index{Condor commands!condor\_cold\_start}
\index{Deployment commands!condor\_cold\_start}
\index{condor\_cold\_start}

\Description

\Condor{cold\_start} installs and starts Condor on this machine.
Essentially, \Condor{cold\_start} binds \Prog{install\_release} and
\Condor{local\_start} together, passing information from the install
process to the start process.  In addition, it has the functionality
to determine the local architecture if one is not specified.
Additionally, this program can install pre-made log, execute, and/or
spool directories by specifying the archived versions.

\Examples

To start a Condor installation on the current machine, using
\texttt{http://www.example.com/Condor/deployment} as the installation
site: 
\begin{verbatim}
% condor_cold_start \
  -configfile http://www.example.com/Condor/deployment/condor_config.mobile \
  -site http://www.example.com/Condor/deployment \
\end{verbatim}

Optionally if this instance of Condor requires a local configuration
file \File{condor\_config.local}:
\begin{verbatim}
% condor_cold_start \
  -configfile http://www.example.com/Condor/deployment/condor_config.mobile \
  -site http://www.example.com/Condor/deployment \
  -runlocalconfig condor_config.local
\end{verbatim}
  
\begin{Options}
  \OptItem{\OptArg{-arch}{architecture}}{
    Uses the given architecture string to fetch the installation
    package.  The string is in the format: \textless condor\_version
    \textgreater-\textless machine\_arch\textgreater -\textless
    os\_name\textgreater -\textless os\_version\textgreater (e.g.,
    6.6.7-i686-Linux-2.4).  \textless condor\_version\textgreater may
    be replaced with ``latest'' (e.g., latest-i686-Linux-2.4).
  }
  \OptItem{\OptArg{-artifact}{filename}}{
    Uses \Arg{filename} for name of the artifact file that is used to
    determine whether the master is still alive.
  }
  \OptItem{\OptArg{-basedir}{directory}}{
    The directory to install/find the Condor executables and
    libraries.  Defaults to the current working directory.
  }
  \OptItem{\OptArg{-configfile}{file}}{
    Specifies the Condor configuration file to use for this
    installation.  This file can be located on a http, ftp, or gsiftp
    site, or alternatively on a mounted file system.
  }
%  \OptItem{\Opt{-dyn}}{
%    Use dynamic names for the log, spool, and execute directories, as
%    well as the binding configuration file.  This option can be used
%    to run multiple instances of condor in the same local directory.
%    This option cannot be used with \Opt{-*archive} options.  The
%    dynamic names are created by appending the IP address and process
%    id of the master to the file names.
%  }
  \OptItem{\OptArg{-execarchive}{archive}}{
    Creates the Condor execute directory from the given archive file.
  }
  \OptItem{\Opt{-flock}}{
    Specifies that this program should use a POSIX file lock midwife
    program to create an artifact of the birth of a Condor Master.  A
    flock undertaker can later be used to determine whether the
    Master has exited.  This is the preferred option if the user wants
    to check the status of the Master from another machine that
    shares a distributed file system that supports POSIX file locking,
    e.g., AFS.
  }
  \OptItem{\Opt{-force}}{
    Overwrite previously installed files if neccessary.
  }
  \OptItem{\OptArg{-globuslocation}{directory}}{
    The location of the globus installation on this machine. Defaults
    to \File{/opt/globus}. This option is only needed if the
    \Opt{-*archive} options are used.
  }
  \OptItem{\Opt{-help}}{
    Display brief usage information and exit.
  }
  \OptItem{\OptArg{-localdir}{directory}}{
    The directory where the Condor log, spool, and execute directories
    will be installed.  Each running instance of Condo must have its
    own local directory. %or the dynamic naming option must be enabled.
  }
  \OptItem{\OptArg{-logarchive}{archive}}{
    Creates the Condor log directory from the given archive file.
  }
  \OptItem{\Opt{-pid}}{
    Specifies that this program should use a unique process id midwife
    program to create an artifact of the birth of a Condor Master.  A
    unique pid undertaker can later be used to determine whether the
    Master has exited.  This is the default option and the preferred method
    to check the status of the Master from the same machine it was
    started on.
  }
  \OptItem{\OptArg{-runlocalconfig}{file}}{
    A special local configuration file bound into the Condor
    configuration at runtime.  This file will only affect the instance
    of Condor started by this command.  No other Condor instance
    sharing the same global configuration file will be affected.
  }
  \OptItem{\Opt{-runonly}}{
    Runs condor from the specified installation directory without
    installing it.  It is possible to run several instantiations of
    Condor from a single installation.
  }
  \OptItem{\Opt{-setuponly}}{
    Installs Condor without running it.
  }
  \OptItem{\OptArg{-site}{repository}}{
    The ftp, http, gsiftp, or mounted file system directory where the
    installation packages can be found (e.g.,
    www.cs.example.edu/packages/coldstart).
  }
  \OptItem{\OptArg{-spoolarchive}{archive}}{
    Creates the Condor spool directory from the given archive file.
  }
  \OptItem{\Opt{-wget}}{
    Uses \Prog{wget} to fetch the log, spool, and execute directories,
    if specified with the \Opt{-*archive} options.  \Prog{wget} must
    be installed on the machine and in the user's path.
  }
\end{Options}

\ExitStatus
\Condor{cold\_start} will exit with a status value of 0 (zero) upon
success, and non-zero otherwise.

\SeeAlso
\Condor{cold\_stop} (on page~\pageref{man-condor-cold-stop}), 
\Prog{flock\_midwife} (on page~\pageref{man-flock-midwife}), 
\Prog{uniq\_pid\_midwife} (on page~\pageref{man-uniq-pid-midwife}).

\end{ManPage}
