\begin{ManPage}{\label{man-condor-cold-start}\Condor{cold\_start}}{1}
{install and start Condor on this machine}

\Synopsis \SynProg{\Condor{cold\_start}}
\Opt{-help}

\SynProg{\Condor{cold\_start}}
\oOptArg{-basedir}{directory}
\oOpt{-force}
%\oOpt{-dyn}
\oOpt{\Opt{-setuponly} \Bar \Opt{-runonly}}
\oOptArg{-arch}{architecture}
\oOptArg{-site}{repository}
\oOptArg{-localdir}{directory}
\oOptArg{-runlocalconfig}{file}
\oOptArg{-logarchive}{archive}
\oOptArg{-spoolarchive}{archive}
\oOptArg{-execarchive}{archive}
\oOpt{-filelock}
\oOpt{-pid}
\oOptArg{-artifact}{filename}
\oOpt{-wget}
\oOptArg{-globuslocation}{directory}
\OptArg{-configfile}{file}

\index{Condor commands!condor\_cold\_start}
\index{Deployment commands!condor\_cold\_start}
\index{condor\_cold\_start}

\Description

\Condor{cold\_start} installs and starts Condor on this machine,
setting up or using a predefined configuration.
In addition, it has the functionality
to determine the local architecture if one is not specified.
Additionally, this program can install pre-made \File{log},
\File{execute}, and/or
\File{spool} directories by specifying the archived versions.

\begin{Options}
  \OptItem{\OptArg{-arch}{architecturestr}}{
    Use the given \Arg{architecturestr} to fetch the installation
    package.  The string is in the format:

    \Sinful{condor\_version}-\Sinful{machine\_arch}-\Sinful{os\_name}-\Sinful{os\_version}

    (for example 6.6.7-i686-Linux-2.4).
    The portion of this string
    \Sinful{condor\_version}
    may be replaced with the string "latest" 
    (for example, latest-i686-Linux-2.4)
    to substitute the most recent version of Condor.
  }
  \OptItem{\OptArg{-artifact}{filename}}{
    Use \Arg{filename} for name of the artifact file used to
    determine whether the \Condor{master} daemon is still alive.
  }
  \OptItem{\OptArg{-basedir}{directory}}{
    The directory to install or find the Condor executables and
    libraries.  When not specified, the current working directory
    is assumed.
  }
%  \OptItem{\Opt{-dyn}}{
%    Use dynamic names for the log, spool, and execute directories, as
%    well as the binding configuration file.  This option can be used
%    to run multiple instances of condor in the same local directory.
%    This option cannot be used with \Opt{-*archive} options.  The
%    dynamic names are created by appending the IP address and process
%    id of the master to the file names.
%  }
  \OptItem{\OptArg{-execarchive}{archive}}{
    Create the Condor \File{execute} directory from the given
    \Arg{archive} file.
  }
  \OptItem{\Opt{-filelock}}{
    Specifies that this program should use a POSIX file lock midwife
    program to create an artifact of the birth of a \Condor{master} daemon.
    A file lock undertaker can later be used to determine whether the
    \Condor{master} daemon has exited.
    This is the preferred option when the user wants
    to check the status of the \Condor{master} daemon from another machine that
    shares a distributed file system that supports POSIX file locking,
    for example, AFS.
  }
  \OptItem{\Opt{-force}}{
    Overwrite previously installed files, if necessary.
  }
  \OptItem{\OptArg{-globuslocation}{directory}}{
    The location of the globus installation on this machine. 
    When not specified \File{/opt/globus} is the directory used.
    This option is only necessary when other options of the
    form \Opt{-*archive} are specified.
  }
  \OptItem{\Opt{-help}}{
    Display brief usage information and exit.
  }
  \OptItem{\OptArg{-localdir}{directory}}{
    The directory where the Condor \File{log}, \File{spool},
    and \File{execute} directories
    will be installed.  Each running instance of Condor must have its
    own local directory. 
    % or the dynamic naming option must be enabled.
  }
  \OptItem{\OptArg{-logarchive}{archive}}{
    Create the Condor log directory from the given \Arg{archive} file.
  }
  \OptItem{\Opt{-pid}}{
    This program is to use a unique process id midwife
    program to create an artifact of the birth of a \Condor{master} daemon.
    A unique pid undertaker can later be used to determine whether the
    \Condor{master} daemon has exited.
    This is the default option and the preferred method
    to check the status of the \Condor{master} daemon from
    the same machine it was started on.
  }
  \OptItem{\OptArg{-runlocalconfig}{file}}{
    A special local configuration file bound into the Condor
    configuration at runtime.  This file only affects the instance
    of Condor started by this command.  No other Condor instance
    sharing the same global configuration file will be affected.
  }
  \OptItem{\Opt{-runonly}}{
    Run Condor from the specified installation directory without
    installing it.  It is possible to run several instantiations of
    Condor from a single installation.
  }
  \OptItem{\Opt{-setuponly}}{
    Install Condor without running it.
  }
  \OptItem{\OptArg{-site}{repository}}{
    The ftp, http, gsiftp, or mounted file system directory where the
    installation packages can be found (for example,
    \File{www.cs.example.edu/packages/coldstart}).
  }
  \OptItem{\OptArg{-spoolarchive}{archive}}{
    Create the Condor spool directory from the given \Arg{archive} file.
  }
  \OptItem{\Opt{-wget}}{
    Use \Prog{wget} to fetch the \File{log}, \File{spool},
    and \File{execute} directories,
    if other options of the form \Opt{-*archive} are specified.
    \Prog{wget} must be installed on the machine and in the user's path.
  }
  \OptItem{\OptArg{-configfile}{file}}{
    A required option to specify the Condor configuration file to use for this
    installation.  This file can be located on an http, ftp, or gsiftp
    site, or alternatively on a mounted file system.
  }
\end{Options}

\ExitStatus
\Condor{cold\_start} will exit with a status value of 0 (zero) upon
success, and non-zero otherwise.

\Examples

To start a Condor installation on the current machine, using
\texttt{http://www.example.com/Condor/deployment} as the installation
site: 
\footnotesize
\begin{verbatim}
% condor_cold_start \
  -configfile http://www.example.com/Condor/deployment/condor_config.mobile \
  -site http://www.example.com/Condor/deployment 
\end{verbatim}
\normalsize

Optionally if this instance of Condor requires a local configuration
file \File{condor\_config.local}:
\footnotesize
\begin{verbatim}
% condor_cold_start \
  -configfile http://www.example.com/Condor/deployment/condor_config.mobile \
  -site http://www.example.com/Condor/deployment \
  -runlocalconfig condor_config.local
\end{verbatim}
\normalsize
  
\SeeAlso
\Condor{cold\_stop} (on page~\pageref{man-condor-cold-stop}), 
\Prog{filelock\_midwife} (on page~\pageref{man-filelock-midwife}), 
\Prog{uniq\_pid\_midwife} (on page~\pageref{man-uniq-pid-midwife}).

\end{ManPage}
