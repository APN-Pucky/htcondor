\begin{ManPage}{\label{man-stork-rm}\Stork{rm}}{1}
{remove a Stork job}
\Synopsis \SynProg{\Stork{rm}}
\ToolArgsBase

\SynProg{\Stork{rm}}
\oOpt{-debug}
\Storkname
\Arg{job-id}


\index{Condor commands!stork\_rm}
\index{Stork commands!stork\_rm}
\index{stork\_rm command}

\Description 

\Stork{rm} removes a Stork job from the queue,
using the required command-line argument to identify the Stork job.
\Stork{rm} removes a single job from the Stork job queue.  
If the \Opt{-name} option is specified, the named \Stork{server} is targeted
for processing.  Otherwise, the local \Stork{server} is targeted.
The job to be removed is identified by the job id returned by  \Stork{submit}.

\begin{Options}
	\ToolArgsBaseDesc
	\OptItem{\Opt{-debug}}{Show extra debugging information.}
	\OptItem{\OptArg{-name}{server\_name}}{Name of \Stork{server}.}
	\StorknameDesc
\end{Options}

\ExitStatus

\Stork{rm} will exit with a status value of 0 (zero) upon success,
and it will exit with the value 1 (one) upon failure.

\end{ManPage}
