\begin{ManPage}{\label{man-condor-prio}\Condor{prio}}{1}
{change priority of jobs in the condor queue} 
\Synopsis \SynProg{\Condor{prio}}
\oOptArg{-p}{priority}
\oOptArg{+ \Bar -}{value}
\oOptArg{-n}{schedd\_name}
\oOptArg{-pool}{pool\_name}
\Arg{cluster \Bar cluster.process \Bar username \Bar -a}

\index{Condor commands!condor\_prio}
\index{condor\_prio command}

\Description

\Condor{prio} changes the priority of one or more jobs in the condor
queue.
If a cluster\_id and a process\_id are both specified, \Condor{prio}
attempts to change the priority of the specified process.
If a cluster\_id is specified without a process\_id, \Condor{prio}
attempts to change priority for all processes belonging to the
specified cluster.
If a username is specified, \Condor{prio} attempts to change priority
of all jobs belonging to that user.
If the -a flag is set, \Condor{prio} attempts to change priority of
all jobs in the condor queue.
The user must specify a priority adjustment or new priority. 
If the -p option is specified, the priority of the job(s) are set to
the next argument.
The user can also adjust the priority by supplying a + or -
immediately followed by a digit.
The priority of a job can be any integer, with higher numbers
corresponding to greater priority.
Only the owner of a job or the super user can change the priority for
it.

The priority changed by \Condor{prio} is only compared to the priority
of other jobs owned by the same user and submitted from the same
machine.
See the "Condor Users and Administrators Manual" for further details
on Condor's priority scheme.

\begin{Options}
    \OptItem{\OptArg{-p}{priority}}{Set priority to the specified value}
    \OptItem{\OptArg{+ \Bar -}{value}}{Change priority by the specified value}
    \OptItem{\OptArg{-n}{schedd\_name}}{Change priority of jobs queued
		at the specified schedd in the local pool}
    \OptItem{\OptArg{-pool}{pool\_name}}{Change priority of jobs queued at 
		the specified schedd in the specified pool} 
    \OptItem{\Opt{-a}}{Change priority of all the jobs in the queue}
\end{Options}

\ExitStatus

\Condor{prio} will exit with a status value of 0 (zero) upon success,
and it will exit with the value 1 (one) upon failure.

\end{ManPage}
