\begin{ManPage}{\label{man-condor-glidein}\Condor{glidein}}{1}
{add a Globus resource to a Condor pool}
\Synopsis \SynProg{\Condor{glidein}}
\oOpt{-help}
\oOptArg{-basedir}{basedir}
\oOptArg{-archdir}{dir}
\oOptArg{-localdir}{dir}
\oOpt{-setuponly}
\oOpt{-runonly}
\oOptArg{-scheduler}{name}
\oOptArg{-queue}{name}
\oOptArg{-project}{name}
\oOptArg{-memory}{MBytes}
\oOptArg{-count}{CPUcount}
\oOptArg{-vms}{VMcount}
\oOptArg{-idletime}{minutes}
\oOptArg{-runtime}{minutes}
\oOpt{-anybody}
\oOptArg{-admin}{address}
\oOpt{-genconfig}
\oOptArg{-useconfig}{filename}
\OptArg{\{-contactfile}{filename} \} \Bar \Opt{Globus contact string}

\index{Condor commands!condor\_glidein}
\index{condor\_glidein command}

\Description

\Condor{glidein} allows the temporary addition of a Globus resource to
a local Condor pool.
The addition is accomplished by installing and executing some of the Condor
daemons on the Globus resource.
A condor\_shadow.globus job appears in the queue of the local
Condor pool for each glidein request.
To remove the Globus resource from the local Condor pool,
use \Condor{rm} to remove the condor\_shadow.globus job from
the job queue.

You must have an X.509 certificate and access
to the Globus resource to use \Condor{glidein}.
The Globus software must also be installed.

Globus is a software system that provides uniform access to
different high-performance computing resources.
When specifying a machine to use with Globus,
you provide a \Opt{Globus contact string}.
Often, the contact string can be just the hostname of the machine.
Sometimes, a more complicated contact string is required.
For example, if a machine has multiple schedulers (ways to run a job),
the contact string may need to specify which to use.
See the Globus home page, \Url{www.globus.org} for more
information about Globus.

\Condor{glidein} works in two steps: set up and execution.
During set up, a configuration file and the Condor daemons
master, startd and starter are installed on the Globus
resource.
Binaries for the correct architecture are copied from a central server.
To obtain access to the server,
send
e-mail to \Email{condor-admin@cs.wisc.edu} with your X.509
certificate name.
Globus software version 1.1.3 does not yet include
the Globus program \Prog{gsincftp},
the Globus secure version of \Prog{ftp}.
\Condor{glidein} needs this program.
Install \Prog{gsincftp}, obtained from
\Url{http://www.globus.org/datagrid/deliverables/gsiftp-tools.html}.
Set up need only be done once per machine and version of Condor.
The execution step starts the Condor daemons running through
the resource's Globus interface.

By default, all files placed on the remote machine are placed in
\File{\MacroUNI{HOME}/Condor\_glidein}.
Each use of \Condor{glidein} will generate spool and log
files on the Globus resource.
These files should be occasionally removed.

\begin{Options}
    \OptItem{\Opt{-help}}{Display brief usage information and exit}
    \OptItem{\OptArg{-basedir}{basedir}}{
	Specifies the base directory on the Globus resource
	used for placing files.
	The default file is \File{\MacroUNI{HOME}/Condor\_glidein}
	on the Globus resource.
	}
    \OptItem{\OptArg{-archdir}{dir}}{
	Specifies the directory on the Globus resource for placement
	of the executables. 
	The default value for \OptArg{-archdir},
	given according to version information on the Globus resource, is 
	\Arg{basedir}\File{/\Sinful{condor-version}-\Sinful{Globus canonicalsystemname}} 
	An example of the directory
	(without the base directory on the Globus resource)
	for Condor version 6.1.13
	running on a Sun Sparc machine with Solaris 2.6 is
	\File{6.1.13-sparc-sun-solaris-2.6} }
    \OptItem{\OptArg{-localdir}{dir}}{
	Specifies the directory on the Globus resource
	in which to create log and execution 
	subdirectories needed by Condor.
	If limited disk quota in the home or base directory
	on the Globus resource is a problem,
	set \Opt{-localdir} to a large temporary space,
	such as \File{/tmp} or \File{/scratch}
	}
    \OptItem{\OptArg{-contactfile}{filename}}{
	Allows the use of a file of Globus contact strings,
	rather than the single
	Globus contact string given in the command line.
	For each of the contacts listed in the file,
	the Globus resource is added to the local Condor pool.
	}
    \OptItem{\Opt{-setuponly}}{
        Performs only the placement of files on the Globus resource.
        This option cannot be run simultaneously with \Opt{-runonly} }
    \OptItem{\Opt{-runonly}}{
	Starts execution of the Condor daemons on the Globus
	resource.
        If any of the files are missing, exits with an error code.
        This option cannot be run simultaneously with \Opt{-setuponly} }
    \OptItem{\OptArg{-scheduler}{name}}{
	Selects the Globus job scheduler type.
	Defaults to \Arg{fork}.
	NOTE: Contact strings which already contain the scheduler
	type will \emph{not} be overridden by this option.
	}
    \OptItem{\OptArg{-queue}{name}}{
	The argument \Arg{name} is a string which specifies which
	job queue is to be used for submission on the Globus resource.
	}
    \OptItem{\OptArg{-project}{name}}{
	The argument \Arg{name} is a string which specifies which
	project is to be used for submission on the Globus resource.
	}
    \OptItem{\OptArg{-memory}{MBytes}}{
	The maximum memory size to request from the Globus resource
	(in megabytes).
	}
    \OptItem{\OptArg{-count}{CPUcount}}{
	Number of CPUs to request, default is 1. }
    \OptItem{\OptArg{-vms}{VMcount}}{
        For machines with multiple CPUs, the CPUs maybe divided
	up into virtual machines. \Arg{VMcount} is the number
	of virtual machines that results.
	By default, Condor divides multiple-CPU resources such that
	each CPU is a virtual machine, each with an equal share of RAM,
	disk, and swap space.
	This option configures the number of virtual machines, so that
	multi-threaded jobs can run in a virtual machine with multiple
	CPUs.
	For example, if 4 CPUs are requested and 
	\Opt{-vms} is not specified, Condor will
	divide the request up into 4 virtual machines with 1 CPU each.
	However, if \OptArg{-vms}{2} is specified,
	Condor will divide the request up into 2 virtual machines with
	2 CPUs each, and if \OptArg{-vms}{1} is
	specified, Condor will put all 4 CPUs into one virtual
	machine. }
    \OptItem{\OptArg{-idletime}{minutes}}{
	How long the Condor daemons on the Globus resource
	can remain idle before the resource reverts back to its
	former state of not being part of the local Condor pool.
	If the value is 0 (zero), the resource will not
	revert back to its former state.
	In this case,
	the Condor daemons will run until the \Arg{runtime} time expires,
	or they are killed by the resource or
	with \Condor{rm}.
	The default value is 20 minutes.  }
    \OptItem{\OptArg{-runtime}{minutes}}{
	How long the Condor daemons on the Globus resource
	will run before shutting themselves down. This option is useful
	for resources with enforced maximum run times. Setting
	\Arg{runtime} to be a few minutes shorter than the allowable
	limit gives the daemons time to perform a graceful shutdown.
	}
    \OptItem{\Arg{-anybody}}{
	Sets the Condor \Expr{START} expression to TRUE
	to allow any user job which meets the job's requirements to run on
	the Globus resource added to the local Condor pool.
	Without this option, only jobs owned by the user executing
	\Condor{glidein} can execute on the Globus resource. WARNING:
	Using this option may violate the usage policies of many
	institutions.
	}
    \OptItem{\OptArg{-admin}{address}}{
	Where to send e-mail with problems.
	The defaults is the login of the user running
	\Condor{glidein} at UID domain of the local Condor pool.  }
    \OptItem{\Opt{-genconfig}}{
	This option creates a local copy of the configuration file used
	on the Globus resource.
	The file is called \File{\condor{config.glidein}}.
        }
    \OptItem{\OptArg{-useconfig}{filename}}{
	Install \Arg{filename} as the configuration file on the Globus resource
	instead of the default configuration file during the set up phase.
	}

\end{Options}

\ExitStatus

\Condor{glidein} will exit with a status value of 0 (zero) upon 
complete success.
The script exits with non-zero values upon failure.
The status value will be 1 (one) if 
\Condor{glidein} encountered an error making a directory,
was unable to copy a tar file,
encountered an error in parsing the command line,
or was not able to gather required information.
The status value will be 2 (two) if 
there was an error in the remote set up.
The status value will be 3 (three) if 
there was an error in remote submission.
The status value will be -1 (negative one) if 
no resource was specified in the command line.

\end{ManPage}
