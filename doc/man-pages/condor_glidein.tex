\begin{ManPage}{\label{man-condor-glidein}\Condor{glidein}}{1}
{add a Globus resource to a Condor pool}
\Synopsis \SynProg{\Condor{glidein}}
\oOpt{-help}
\oOptArg{-admin}{address}
\oOpt{-anybody}
\oOptArg{-archdir}{dir}
\oOptArg{-basedir}{basedir}
\oOptArg{-count}{CPUcount}
\oOpt{-genconfig}
\oOpt{-genstartup}
\oOpt{-gensubmit}
\oOptArg{-idletime}{minutes}
\oOptArg{-localdir}{dir}
\oOptArg{-memory}{MBytes}
\oOptArg{-project}{name}
\oOptArg{-queue}{name}
\oOptArg{-runtime}{minutes}
\oOpt{-runonly}
\oOpt{-setuponly}
\oOpt{-setup\_here}
\oOptArg{-setup\_jobmanager}{jobmanager}
\oOptArg{-scheduler}{name}
\oOptArg{-suffix}{suffix}
\oOptArg{-useconfig}{filename}
\oOptArg{-usestartup}{filename}
\oOptArg{-vms}{VMcount}
\OptArg{\{-contactfile}{filename} \} \Bar \Opt{Globus contact string}

\index{Condor commands!condor\_glidein}
\index{condor\_glidein command}

\Description

\Condor{glidein} allows the temporary addition of a Globus resource to
a local Condor pool.
The addition is accomplished by installing and executing some of the Condor
daemons on the Globus resource.
A glidein\_startup job appears in the queue of the local
Condor pool for each glidein request.
To remove the Globus resource from the local Condor pool,
use \Condor{rm} to remove the glidein\_startup job from
the job queue.

You must have an X.509 certificate and access
to the Globus resource to use \Condor{glidein}.
The Globus software must also be installed.

Globus is a software system that provides uniform access to
different high-performance computing resources.
When specifying a machine to use with Globus,
you provide a \Opt{Globus contact string}.
Often, the contact string can be just the hostname of the machine.
Sometimes, a more complicated contact string is required.
For example, if a machine has multiple schedulers (ways to run a job),
the contact string may need to specify which to use.
See the Globus home page, \URL{http://www.globus.org/} for more
information about Globus.

\Condor{glidein} works in two steps: set up and execution.
During set up, a configuration file, startup file, and the Condor daemons
master, startd and starter are installed on the Globus
resource.
Binaries for the correct architecture are copied from a central server.
To obtain access to the server,
or to set up your own server, follow
instructions on the Glidein Server Setup page,
at \URL{http://www.cs.wisc.edu/condor/glidein}.
Set up need only be done once per site.
The execution step starts the Condor daemons running through
the resource's Globus interface.

By default, all files placed on the remote machine are placed in
\File{\MacroUNI{HOME}/Condor\_glidein} (or whatever \Opt{-basedir} is
defined to be).  It is assumed that this directory is shared by all of
the machines that will be running the glidein Condor daemons.  By
default, the daemon log files will also be written into this area, but
you are encouraged to change this (e.g. with \Opt{-localdir}) to make
them write to local scratch space on the execution machine.  However,
for debugging initial problems, it may be convenient to have the log
files in a more accessible place.  If you do leave the default
setting alone, you should at least occasionally clean up old
log and execute directories left behind by glideins or you may eventually
run out of space.

\Examples

To setup and run 10 glideins under PBS on a grid site with a
gatekeeper named \textit{gatekeeper.site.edu}:
\begin{verbatim}
% condor_glidein -count 10 gatekeeper.site.edu/jobmanager-pbs
\end{verbatim}

If you try something like the above and \Condor{glidein} is not able to
automatically determine everything it needs to know about the remote site,
it will ask you to provide more information.  A typical result of this
process is something like the following command:
\begin{verbatim}
% condor_glidein \
    -count 10 \
    -arch 6.6.7-i686-pc-Linux-2.4 \
    -setup_jobmanager jobmanager-fork \
    gatekeeper.site.edu/jobmanager-pbs
\end{verbatim}

You may use \Condor{q} to see the glidein jobs that have been
submitted.  Once they successfully run, you may see them join your
Condor pool by using \Condor{status}.

See the list of common problems and solutions near the end of this section
if you have trouble getting the system to work.

\begin{Options}
    \OptItem{\Opt{-help}}{Display brief usage information and exit}
    \OptItem{\OptArg{-basedir}{basedir}}{
	Specifies the base directory on the Globus resource
	used for placing files.
	The default file is \File{\MacroUNI{HOME}/Condor\_glidein}
	on the Globus resource.
	}
    \OptItem{\OptArg{-archdir}{dir}}{
	Specifies the directory on the Globus resource for placement
	of the executables. 
	The default value for \OptArg{-archdir},
	given according to version information on the Globus resource, is 
	\Arg{basedir}\File{/\Sinful{condor-version}-\Sinful{Globus canonicalsystemname}} 
	An example of the directory
	(without the base directory on the Globus resource)
	for Condor version 6.1.13
	running on a Sun Sparc machine with Solaris 2.6 is
	\File{6.1.13-sparc-sun-solaris-2.6} }
    \OptItem{\OptArg{-localdir}{dir}}{
	Specifies the directory on the Globus resource
	in which to create log and execution 
	subdirectories needed by Condor.
	If limited disk quota in the home or base directory
	on the Globus resource is a problem,
	set \Opt{-localdir} to a large temporary space,
	such as \File{/tmp} or \File{/scratch}.  If the batchsystem
        makes glidein start up in a temporary scratch directory,
        you can use `.' for \Opt{-localdir}.
	}
    \OptItem{\OptArg{-contactfile}{filename}}{
	Allows the use of a file of Globus contact strings,
	rather than the single
	Globus contact string given in the command line.
	For each of the contacts listed in the file,
	the Globus resource is added to the local Condor pool.
	}
    \OptItem{\Opt{-runonly}}{
	Starts execution of the Condor daemons on the Globus
	resource.
        If any of the files are missing, exits with an error code.
        This option cannot be run simultaneously with \Opt{-setuponly} }
    \OptItem{\Opt{-run\_here}}{
        Runs \Condor{master} directly rather than submitting it to
        Condor-G for remote execution.  To instead generate a script that
        does this, use \Opt{-run\_here} in combination with
        \Opt{-gensubmit}.  This may be useful for running Glidein on
        resources that are not directly accessible to Condor-G.}
    \OptItem{\Opt{-setuponly}}{
        Performs only the placement of files on the Globus resource.
        This option cannot be run simultaneously with \Opt{-runonly} }
    \OptItem{\Opt{-setup\_here}}{
        Runs the setup process directly instead of submitting a setup
	job to the remote Globus resource.  For example, this may
        be used to install glidein in an AFS area that is read-only
        from the remote Globus resource. }
    \OptItem{\OptArg{-setup\_jobmanager}{jobmanager-fork}}{
        Jobmanager to use for running Glidein setup process.  If a
	readonable default can be discovered through MDS, this is optional. }
    \OptItem{\OptArg{-arch}{architecture}}{
        Identifies the glidein tarball to download and install.  If a
	readonable default can be discovered through MDS, this is
	optional.  A list of possible values may be found here:
	\URL{http://www.cs.wisc.edu/condor/glidein/binaries}.  The architecture
        name is the same as the tarball name minus the tar.gz.  For
	example: 6.6.5-i686-pc-Linux-2.4 }

    \OptItem{\OptArg{-scheduler}{name}}{
	Selects the Globus job scheduler type.
	Defaults to \Arg{fork}.
	NOTE: Contact strings which already contain the scheduler
	type will \emph{not} be overridden by this option.
	}
    \OptItem{\OptArg{-queue}{name}}{
	The argument \Arg{name} is a string which specifies which
	job queue is to be used for submission on the Globus resource.
	}
    \OptItem{\OptArg{-project}{name}}{
	The argument \Arg{name} is a string which specifies which
	project is to be used for submission on the Globus resource.
	}
    \OptItem{\OptArg{-memory}{MBytes}}{
	The maximum memory size to request from the Globus resource
	(in megabytes).
	}
    \OptItem{\OptArg{-count}{CPUcount}}{
	Number of CPUs to request, default is 1. }
    \OptItem{\OptArg{-vms}{VMcount}}{
        For machines with multiple CPUs, the CPUs maybe divided
	up into virtual machines. \Arg{VMcount} is the number
	of virtual machines that results.
	By default, Condor divides multiple-CPU resources such that
	each CPU is a virtual machine, each with an equal share of RAM,
	disk, and swap space.
	This option configures the number of virtual machines, so that
	multi-threaded jobs can run in a virtual machine with multiple
	CPUs.
	For example, if 4 CPUs are requested and 
	\Opt{-vms} is not specified, Condor will
	divide the request up into 4 virtual machines with 1 CPU each.
	However, if \OptArg{-vms}{2} is specified,
	Condor will divide the request up into 2 virtual machines with
	2 CPUs each, and if \OptArg{-vms}{1} is
	specified, Condor will put all 4 CPUs into one virtual
	machine. }
    \OptItem{\OptArg{-idletime}{minutes}}{
	How long the Condor daemons on the Globus resource
	can remain idle before the resource reverts back to its
	former state of not being part of the local Condor pool.
	If the value is 0 (zero), the resource will not
	revert back to its former state.
	In this case,
	the Condor daemons will run until the \Arg{runtime} time expires,
	or they are killed by the resource or
	with \Condor{rm}.
	The default value is 20 minutes.  }
    \OptItem{\OptArg{-runtime}{minutes}}{
	How long the Condor daemons on the Globus resource
	will run before shutting themselves down. This option is useful
	for resources with enforced maximum run times. Setting
	\Arg{runtime} to be a few minutes shorter than the allowable
	limit gives the daemons time to perform a graceful shutdown.
	}
    \OptItem{\Arg{-anybody}}{
	Sets the Condor \Expr{START} expression to TRUE
	to allow any user job which meets the job's requirements to run on
	the Globus resource added to the local Condor pool.
	Without this option, only jobs owned by the user executing
	\Condor{glidein} can execute on the Globus resource. WARNING:
	Using this option may violate the usage policies of many
	institutions.
	}
    \OptItem{\OptArg{-admin}{address}}{
	Where to send e-mail with problems.
	The defaults is the login of the user running
	\Condor{glidein} at UID domain of the local Condor pool.  }
    \OptItem{\Opt{-genconfig}}{
	This option creates a local copy of the configuration file used
	on the Globus resource.
	The file is called \File{glidein\_condor\_config}\{\emph{suffix}\}.
        You may edit this file and use \Opt{-useconfig} to install your
        modified config.
        }
    \OptItem{\OptArg{-useconfig}{config\_file}}{
        This option makes the setup process copy the config file
        you specify, rather than generating one from scratch.
        }
    \OptItem{\Opt{-genstartup}}{
	This option creates a local copy of the startup script used
	on the Globus resource when \Condor{master} runs.
	The file is called \File{glidein\_startup}\{\emph{suffix}\}.
        You may edit this file and use \Opt{-usestartup} to install your
        modified config.
        }
    \OptItem{\OptArg{-usestartup}{startup\_file}}{
        This option makes the setup process copy the startup script
        you specify, rather than generating one from scratch.
        }
    \OptItem{\Opt{-suffix}{X}}{
        Suffix to use when generating files.  Default is process id. }
    \OptItem{\OptArg{-gsi\_daemon\_name}{cert\_name}}{
        Using this option turns on GSI authentication in the glidein
        configuration.  The argument to this option is the GSI
        certificate name that the glidein daemons will use to authenticate
        themselves.  It should be set to whatever certificate name
        you will use to execute the glideins.}
    \OptItem{\OptArg{-install\_gsi\_trusted\_ca\_dir}{path}}{
        Using this option turns on GSI authentication in the glidein
        configuration.  The argument to this option is the path
        to the trusted CA certificates that you wish the glidein
        daemons to use (e.g. /etc/grid-security/certificates).  The
        contents of this directory will be installed at the remote
        site in \Opt{-basedir}/grid-security. }
    \OptItem{\OptArg{-install\_gsi\_gridmap}{file}}{
        Using this option turns on GSI authentication in the glidein
        configuration.  The argument to this option is the filename
        of the grid-mapfile that you wish the glidein daemons to
        use.  The file will be installed at the remote site
        in \Opt{-basedir}/grid-security.  The file should contain
        entries mapping grid-certificates to user names.  At the
        very least, it must contain an entry for the certificate
        \OptArg{-gsi\_daemon\_name}.  If your other Condor
        daemons use different certificates, then this file should
        also mention any certificates that the glidein daemons
        will encounter (schedd, collector, and negotiator).  See
        section~\ref{sec:GSI-Authentication} for more information. }


\end{Options}

\ExitStatus

\Condor{glidein} will exit with a status value of 0 (zero) upon 
complete success.
The script exits with non-zero values upon failure.
The status value will be 1 (one) if 
\Condor{glidein} encountered an error making a directory,
was unable to copy a tar file,
encountered an error in parsing the command line,
or was not able to gather required information.
The status value will be 2 (two) if 
there was an error in the remote set up.
The status value will be 3 (three) if 
there was an error in remote submission.
The status value will be -1 (negative one) if 
no resource was specified in the command line.

Common problems are listed below.  Many of these are best discovered by
looking in the remove StartLog in the glidein ``localdir''.

\begin{description}

\item[WARNING: The file xxx is not writable by condor]  This happens if you
run \Condor{glidein} from a directory that does not have the right
permissions for Condor to access files.  If you are in an AFS directory,
keep in mind that Condor does not have your AFS ACLs.

\item[Glideins fail to run due to GLIBC errors] Check the list of
available glidein binaries
(\URL{http://www.cs.wisc.edu/condor/glidein/binaries}) and try setting
up glidein with an architecture name that includes the correct glibc
version for the remote site.

\item[Glideins join pool but no jobs run on them] One common
cause of this problem is that the glidein machines are in a different
filesystem domain and your jobs have been submitted with an implicit
requirement that they must run in the same filesystem domain.  If this
is your problem, see section~\ref{sec:file-transfer} for details on
using Condor's file-transfer capabilities.  Another cause of this
problem is a communication failure.  For example, a firewall may be
preventing \Condor{negotiator} or \Condor{schedd} from connecting to
the glidein \Condor{startd}.  Although work is being done to remove
this requirement in the future, it is currently necessary to have full
bi-directional connectivity, at least over a restricted range of
ports.  See page~\pageref{param:HighPort} for more information on
configuring a port range.

\item[Glideins run but fail to join the pool] This may be caused by
your pool's security settings or by a communication failure.  Check
that the security settings in your pool's Condor config file allow
write access to the glidein machines.  If you do not wish to modify
the security settings for the pool, you can run a separate pool
specifically for the glideins and use flocking to balance jobs across
the two pools of resources.  If instead the glidein daemon log files
indicate a communication failure, then see the next item.

\item[The startd cannot connect to the collector] This may be caused
by several things.  One is a firewall.  Another is when the compute
nodes do not have even outgoing network access.  Configuring Glidein
to work without full network access to and from the compute nodes is
still in the experimental stages, so for now, the short answer is that
you must at least have a range of open (bi-directional) ports and set
up the glidein config file as described on
page~\pageref{param:HighPort}.  (Use \Opt{-genconfig}, edit the file,
and then use \Opt{-useconfig}.)

Another possible cause of connectivity problems may be the use of UDP by
the \Condor{startd} to register itself with the collector.  You can
force it to use TCP as described on
page~\pageref{param:UpdateCollectorWithTcp}.

Yet another possible cause of connectivity problems is when the glidein
machines have more than one network interface and the default one chosen
by Condor is not the correct one.  One way to fix this is to modify
the glidein startup script (using \Opt{-genstartup} and \Opt{-usestartup}).
The script simply needs to determine the IP address associated with
the correct network interface and assign this to the environment
variable \MacroNI{\_condor\_NETWORK\_INTERFACE}.

\item[NFS file locking problems]  If you have the \Opt{-localdir}
configured to be on NFS (not recommended, but sometimes convenient
for testing), the Condor daemons may have trouble manipulating file
locks.  You may insert the following into the Glidein config file:

\begin{verbatim}
IGNORE_NFS_LOCK_ERRORS = True
\end{verbatim}

\end{description}

\end{ManPage}
