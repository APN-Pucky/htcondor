\begin{ManPage}{\label{man-bosco-cluster}\Bosco{cluster}}{1}
{Manage and configure the clusters to be accessed. }

\index{Bosco commands!bosco\_cluster}
\index{bosco\_cluster command}

\Synopsis \SynProg{\Bosco{cluster}}
[\verb@-@\textbf{h} || \verb@--@\textbf{help}]

\SynProg{\Bosco{cluster}}
[\verb@-@\textbf{l} || \verb@--@\textbf{list}]
[\verb@-@\textbf{a} || \verb@--@\textbf{add \lt{}host\gt{} $[$schedd$]$}]
[\verb@-@\textbf{r} || \verb@--@\textbf{remove \lt{}host\gt{}}]
[\verb@-@\textbf{s} || \verb@--@\textbf{status \lt{}host\gt{}}]
[\verb@-@\textbf{t} || \verb@--@\textbf{test \lt{}host\gt{}}]

\Description

\Bosco{cluster} is part of the Bosco system for accessing high
throughput computing resources from a local desktop.
For detailed information, please see the Bosco web site:
\URL{https://osg-bosco.github.io/docs/}

\Bosco{cluster} enables management and configuration of the computing resources
the Bosco tools access; these are called clusters.

A \Opt{<host>} is of the form \Expr{user@fqdn.example.com}.

\begin{Options}
  \OptItem{\Opt{---help}} {Print usage information and exit.}
  \OptItem{\Opt{---list}} {List all installed clusters.}
  \OptItem{\OptArg{---remove}{<host>}} {Remove an already installed cluster,
    where the cluster is identified by \Arg{<host>}.}
  \OptItem{\OptArg{---add}{<host> [scheduler]}} 
    {Install and add a cluster defined by \Arg{<host>}.
    The optional \Arg{scheduler} specifies the scheduler on the cluster.
    Valid values are \Expr{pbs}, \Expr{lsf}, \Expr{condor}, \Expr{sge} or
    \Expr{slurm}.  If not given, the default will be \Expr{pbs}. }
  \OptItem{\OptArg{---status}{<host>}} {Query and print the status of
    an already installed cluster,
    where the cluster is identified by \Arg{<host>}.}
  \OptItem{\OptArg{---test}{<host>}} {Attempt to submit a test job to 
    an already installed cluster,
    where the cluster is identified by \Arg{<host>}.}

\end{Options}

%\ExitStatus

%\Bosco{cluster} will exit \Dots.

\end{ManPage}
