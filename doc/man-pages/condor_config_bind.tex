\begin{ManPage}{\label{man-condor-config-bind}\Condor{config\_bind}}{1}
{bind together a set of configuration files}

\Synopsis \SynProg{\Condor{config\_bind}}
\oOpt{-help}
\OptArg{-o}{outputfile}
\Arg{configfile1}
\Arg{configfile2}
\oArg{configfile3\ldots}

\index{Condor commands!condor\_config\_bind}
\index{condor\_config\_bind command}

\Description

\Condor{config\_bind} dynamically binds two or more condor
configuration files through the use of a new configuration file.  The
purpose of this tool is to allow the user to dynamically bind a local
configuration file into an already created, and possible immutable,
configuration file.  This is particularly useful we the user wants to
modify a configuration but can't actually make any changes to the
global configuration file (even changing the list of local config
files).  This program does not modify the given configuration files,
rather it creates a new configuration file that specifies the given
configuration files as local configuration files.  Condor evaluates
the configuration files left to right.  That is, if a value is defined
in two or more of the configuration files, the rightmost configuration
file that contains this value overrides the others.  If a user wants to
bind a new local configuration into a global configuration, the user
should specify the local configuration second.

\begin{Options}
  \OptItem{\Arg{configfile1}}{First configuration file to
    bind.}
  \OptItem{\Arg{configfile2}}{Second configuration file to
    bind.} 
  \OptItem{\Arg{configfile3\ldots}}{An optional list of other
    configuration files to bind.}
  \OptItem{\Opt{-help}}{Display brief usage information and exit}
  \OptItem{\OptArg{-o}{output\_file}} {
    Specifies the file name where this program should output the
    binding configuration. 
  }
\end{Options}

\ExitStatus

\Condor{config\_bind} will exit with a status value of 0 (zero) upon
success, and non-zero on error.

\end{ManPage}
