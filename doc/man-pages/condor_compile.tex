\begin{ManPage}{\Condor{compile}}{man-condor-compile}{1}
{create a relinked executable for use as a standard universe job}

\index{HTCondor commands!condor\_compile}
\index{condor\_compile command}

\Synopsis \SynProg{\Condor{compile}}
\Arg{cc \Bar{}\ CC \Bar{}\ gcc \Bar{}\ f77 \Bar{}\ g++ \Bar{}\ ld \Bar{}\ make \Bar{}\ \Dots }

\Description

Use \Condor{compile} to relink a program with the HTCondor libraries for
submission as a standard universe job.
The HTCondor libraries provide the program with additional support, such
as the capability to produce checkpoints, 
which facilitate the standard universe mode of operation.
\Condor{compile} requires access to the source or object code of the
program to be submitted; if source or object code for the program is
not available,
then the program must use another universe, such as vanilla.
Source or object code may not be available if there is 
only an executable binary, or if a shell script is to be executed as
an HTCondor job. 

To use \Condor{compile}, issue the command
\Condor{compile} with command line arguments that form
the normally entered command to compile or link the application.
Resulting executables will have the HTCondor libraries linked in.
For example, 
\footnotesize
\begin{verbatim}
  condor_compile cc -O -o myprogram.condor file1.c file2.c ... 
\end{verbatim}
\normalsize
will produce the binary \File{myprogram.condor}, 
which is relinked for HTCondor,
capable of checkpoint/migration/remote system calls, and ready to
submit as a standard universe job.  

If the HTCondor administrator has opted to fully install
\Condor{compile}, then \Condor{compile} can be followed by practically
any command or program, including make or shell script programs.
For example, the following would all work:
\footnotesize
\begin{verbatim}
  condor_compile make 

  condor_compile make install 

  condor_compile f77 -O mysolver.f 

  condor_compile /bin/csh compile-me-shellscript 
\end{verbatim}
\normalsize

If the HTCondor administrator has opted to only do a partial install of
\Condor{compile}, then you are restricted to following \Condor{compile}
with one of these programs:  
\footnotesize
\begin{verbatim}
  cc (the system C compiler) 

  c89 (POSIX compliant C compiler, on some systems) 

  CC (the system C++ compiler) 

  f77 (the system FORTRAN compiler) 

  gcc (the GNU C compiler) 

  g++ (the GNU C++ compiler) 

  g77 (the GNU FORTRAN compiler) 

  ld (the system linker) 
\end{verbatim}
\normalsize

\Note If you explicitly call \Prog{ld} when you normally create
your binary, instead use:
\footnotesize
\begin{verbatim}
  condor_compile ld <ld arguments and options>
\end{verbatim}
\normalsize

\ExitStatus

\Condor{compile} is a script that executes specified compilers and/or linkers.
If an error is encountered before calling these other programs,
\Condor{compile} will exit with a status value of 1 (one).
Otherwise, the exit status will be that given by the executed program.

\end{ManPage}
