\begin{ManPage}{\label{man-condor-compile}\Condor{compile}}{1}
{create a relinked executable for submission to the Standard Universe}

\Synopsis \SynProg{\Condor{compile}}
\Arg{cc \Bar\ CC \Bar\ gcc \Bar\ f77 \Bar\ g++ \Bar\ ld \Bar\ make \Bar\ \Dots } 

\Description

\index{HTCondor commands!condor\_compile}
\index{condor\_compile command}

Use \Condor{compile} to relink a program with the HTCondor libraries for
submission into HTCondor's Standard Universe.
The HTCondor libraries provide the program with additional support, such
as the capability to checkpoint, which is required in HTCondor's
Standard Universe mode of operation.
\Condor{compile} requires access to the source or object code of the
program to be submitted; if source or object code for the program is
not available (i.e. only an executable binary, or if it is a shell
script), then the program must submitted into HTCondor's Vanilla
Universe.
See the reference page for \Condor{submit} and/or consult the "HTCondor
Users and Administrators Manual" for further information.

To use \Condor{compile}, simply enter "condor\_compile" followed by
whatever you would normally enter to compile or link your
application.
Any resulting executables will have the HTCondor libraries linked in.
For example: 
\begin{verbatim}
        condor_compile cc -O -o myprogram.condor file1.c file2.c ... 
\end{verbatim}
will produce a binary "myprogram.condor" which is relinked for HTCondor,
capable of checkpoint/migration/remote-system-calls, and ready to
submit to the Standard Universe.  

If the HTCondor administrator has opted to fully install
\Condor{compile}, then \Condor{compile} can be followed by practically
any command or program, including make or shell-script programs.
For example, the following would all work:
\begin{verbatim}
        condor_compile make 

        condor_compile make install 

        condor_compile f77 -O mysolver.f 

        condor_compile /bin/csh compile-me-shellscript 
\end{verbatim}

If the HTCondor administrator has opted to only do a partial install of
\Condor{compile}, the you are restricted to following \Condor{compile}
with one of these programs:  
\begin{verbatim}
        cc (the system C compiler) 

        c89 (POSIX compliant C compiler, on some systems) 

        CC (the system C++ compiler) 

        f77 (the system FORTRAN compiler) 

        gcc (the GNU C compiler) 

        g++ (the GNU C++ compiler) 

        g77 (the GNU FORTRAN compiler) 

        ld (the system linker) 
\end{verbatim}

\Note If you use explicitly call ``ld'' when you normally create
your binary, simply use:
\begin{verbatim}
        condor_compile ld <ld arguments and options>
\end{verbatim}
instead.  

\ExitStatus

\Condor{compile} is a script that executes specified compilers and/or linkers.
If an error is encountered before calling these other programs,
\Condor{compile} will exit with a status value of 1 (one).
Otherwise, the exit status will be that given by the executed program.

\end{ManPage}
