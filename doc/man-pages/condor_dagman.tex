\begin{ManPage}{\label{man-condor-dagman}\Condor{dagman}}{3}
{metascheduler of the jobs submitted as the nodes of a DAG}
\Synopsis \SynProg{\Condor{dagman}}
\oOptArg{-debug}{level}
\oOptArg{-rescue}{filename}
\oOptArg{-maxjobs}{numberofjobs}
\oOptArg{-maxpre}{NumberOfPREscripts}
\oOptArg{-maxpost}{NumberOfPOSTscripts}
\oOpt{-nopostfail}
\oOpt{-noeventchecks}
\oOpt{-allowlogerror}
(\OptArg{-condorlog}{filename} \Bar \OptArg{-storklog}{filename})
\OptArg{-lockfile}{filename}
\OptArg{-dag}{filename}

\index{Condor commands!condor\_dagman}
\index{condor\_dagman command}

\Description
\Condor{dagman} is a metascheduler for the Condor jobs within
a DAG (directed acyclic graph).
In typical usage,
a submitter of jobs that are organized into a DAG submits the
DAG using \Condor{submit\_dag}.
\Condor{submit\_dag} does error checking on aspects of the DAG
and then submits \Condor{dagman} as a Condor job.
\Condor{dagman} uses log files to coordinate the further 
submission of the jobs within the DAG.

As part of \Prog{daemoncore}, the set of command-line arguments
given in
section~\ref{sec:DaemonCore-Arguments}
work for \Condor{dagman}.

Arguments to \Condor{dagman} are either automatically set
by \Condor{submit\_dag} 
or they are specified as command-line arguments to \Condor{submit\_dag}
and passed on to \Condor{dagman}.
The method by which the arguments are set is
given in their description below.


\begin{Options}
    \OptItem{\OptArg{-debug}{level}}{An integer level of debugging output.
       \Arg{level} is an integer, with values of 0-7 inclusive,
       where 7 is the most verbose output.
       This command-line option to \Condor{submit\_dag}
       is passed to \Condor{dagman} or
       defaults to the value 3, as set by \Condor{submit\_dag}.  }
    \OptItem{\OptArg{-rescue}{filename}}{Sets the file name of
       the rescue DAG to write in the case of a failure.
       As passed by \Condor{submit\_dag}, the name of the file
       will be the name of the DAG input file concatenated with
       the string \File{.rescue}. }
    \OptItem{\OptArg{-maxjobs}{numberofjobs}}{Sets the maximum number of jobs
       within the DAG that will be submitted to Condor at one time.
       \Arg{numberofjobs} is a positive integer.
       This command-line option to \Condor{submit\_dag} is passed to
       \Condor{dagman}.
       If not specified, the default number of jobs is unlimited.  }
    \OptItem{\OptArg{-maxpre}{NumberOfPREscripts}}{Sets the maximum number
       of PRE
       scripts within the DAG that may be running at one time.
       \Arg{NumberOfPREScripts} is a positive integer. 
        This command-line option to \Condor{submit\_dag} is passed to
        \Condor{dagman}.
       If not specified,
       the default number of PRE scripts is unlimited.}
    \OptItem{\OptArg{-maxpost}{NumberOfPOSTscripts}}{Sets the maximum number of
       POST scripts within the DAG that may be running at one time.
       \Arg{NumberOfPOSTScripts} is a positive integer. 
        This command-line option to \Condor{submit\_dag} is passed to
        \Condor{dagman}.
       If not specified,
       the default number of POST scripts is unlimited.}
    \OptItem{\Opt{-nopostfail}}{A command-line option to
       \Condor{submit\_dag}, further passed to \Condor{dagman},
       is applied to all nodes within
       the DAG and prevents the POST script within a node
       from running in the case that the job within the node fails.
       Without this option, POST scripts always run when jobs fail.}
    \OptItem{\Opt{-noeventchecks}}{This argument is deprecated and
       is now ignored.  Its functionality is now implemented by
       the \MacroNI{DAGMAN\_ALLOW\_EVENTS} configuration macro
       (see section~\ref{param:DAGManAllowEvents}).}
    \OptItem{\Opt{-allowlogerror}}{This optional argument has 
       \Condor{dagman} try to run the specified DAG, even in the case
       of detected errors in the user log specification. }
    \OptItem{\OptArg{-storklog}{filename}}{Sets the file name of
       the Stork log for data placement jobs.  }
    \OptItem{\OptArg{-condorlog}{filename}}{Sets the file name of
       the file used in conjunction with the \OptArg{-lockfile}{filename}
       in determining whether to run in recovery mode.  }
    \OptItem{\OptArg{-lockfile}{filename}}{Names the file
       created and used as a lock file.
       The lock file prevents execution of two of the 
       same DAG, as defined by a DAG input file.
       A default lock file ending with the suffix \File{.dag.lock}
       is passed to \Condor{dagman} by \Condor{submit\_dag}.  }
    \OptItem{\OptArg{-dag}{filename}}{\Arg{filename} is the name of the
       DAG input file that is set as an argument to \Condor{submit\_dag},
       and passed to \Condor{dagman}.}
\end{Options}


\ExitStatus

\Condor{dagman} will exit with a status value of 0 (zero) upon success,
and it will exit with the value 1 (one) upon failure.

\end{ManPage}
