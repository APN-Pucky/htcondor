\begin{ManPage}{\label{man-condor-dagman}\Condor{dagman}}{3}
{meta scheduler of the jobs submitted as the nodes of a DAG or DAGs}
\Synopsis \SynProg{\Condor{dagman}}
\oOptArg{-debug}{level}
\oOptArg{-rescue}{filename}
\oOptArg{-maxidle}{numberOfJobs}
\oOptArg{-maxjobs}{numberOfJobs}
\oOptArg{-maxpre}{NumberOfPREscripts}
\oOptArg{-maxpost}{NumberOfPOSTscripts}
\oOpt{-noeventchecks}
\oOpt{-allowlogerror}
\oOpt{-usedagdir}
(\OptArg{-condorlog}{filename} \Bar \OptArg{-storklog}{filename})
\OptArg{-lockfile}{filename}
\oOpt{-waitfordebug}
\oOptArg{-autorescue}{0|1}
\oOptArg{-dorescuefrom}{number}
\OptArg{-csdversion}{version\_string}
\OptArg{-dag}{dag\_file}
\oArg{\OptArg{-dag}{dag\_file\_2} \Dots \OptArg{-dag}{dag\_file\_n} }

\index{Condor commands!condor\_dagman}
\index{condor\_dagman command}

\Description
\Condor{dagman} is a meta scheduler for the Condor jobs within
a DAG (directed acyclic graph) (or multiple DAGs).
In typical usage,
a submitter of jobs that are organized into a DAG submits the
DAG using \Condor{submit\_dag}.
\Condor{submit\_dag} does error checking on aspects of the DAG
and then submits \Condor{dagman} as a Condor job.
\Condor{dagman} uses log files to coordinate the further 
submission of the jobs within the DAG.

As part of \Prog{daemoncore}, the set of command-line arguments
given in
section~\ref{sec:DaemonCore-Arguments}
work for \Condor{dagman}.

Arguments to \Condor{dagman} are either automatically set
by \Condor{submit\_dag} 
or they are specified as command-line arguments to \Condor{submit\_dag}
and passed on to \Condor{dagman}.
The method by which the arguments are set is
given in their description below.

\Condor{dagman} can run multiple, independent DAGs.  This is done
by specifying multiple \OptArg{-dag} arguments.
Pass multiple
DAG input files as command-line arguments to \Condor{submit\_dag}.

Debugging output may be obtained by using the
\OptArg{-debug}{level} option.
Level values and what they produce is described as
\begin{itemize}
  \item level = 0; never produce output, 
        except for usage info 
  \item level = 1; very quiet, output severe errors 
  \item level = 2; normal output, errors and warnings
  \item level = 3; output errors, as well as all warnings
  \item level = 4; internal debugging output
  \item level = 5; internal debugging output; outer loop debugging
  \item level = 6; internal debugging output; inner loop debugging
  \item level = 7; internal debugging output; rarely used
\end{itemize}


\begin{Options}
    \OptItem{\OptArg{-debug}{level}}{An integer level of debugging output.
       \Arg{level} is an integer, with values of 0-7 inclusive,
       where 7 is the most verbose output.
       This command-line option to \Condor{submit\_dag}
       is passed to \Condor{dagman} or
       defaults to the value 3, as set by \Condor{submit\_dag}.}
    \OptItem{\OptArg{-rescue}{filename}}{Sets the file name of
       the rescue DAG to write in the case of a failure.
       As passed by \Condor{submit\_dag}, the name of the file
       will be the name of the DAG input file concatenated with
       the string \File{.rescue}.  This argument is now optional,
       and in general it is preferred to not specify it.  This allows
       \Condor{dagman} to automatically generate an appropriate
       rescue DAG name.}
    \OptItem{\OptArg{-maxidle}{NumberOfJobs}}{Sets the maximum number of idle
       jobs allowed before \Condor{dagman} stops submitting more jobs.  Once
       idle jobs start to run, \Condor{dagman} will resume submitting jobs.
       \Arg{NumberOfJobs} is a positive integer.
       This command-line option to \Condor{submit\_dag} is passed to
       \Condor{dagman}.
       If not specified, the number of idle jobs is unlimited.  }
    \OptItem{\OptArg{-maxjobs}{numberOfJobs}}{Sets the maximum number of jobs
       within the DAG that will be submitted to Condor at one time.
       \Arg{numberOfJobs} is a positive integer.
       This command-line option to \Condor{submit\_dag} is passed to
       \Condor{dagman}.
       If not specified, the default number of jobs is unlimited.  }
    \OptItem{\OptArg{-maxpre}{NumberOfPREscripts}}{Sets the maximum number
       of PRE
       scripts within the DAG that may be running at one time.
       \Arg{NumberOfPREScripts} is a positive integer. 
        This command-line option to \Condor{submit\_dag} is passed to
        \Condor{dagman}.
       If not specified,
       the default number of PRE scripts is unlimited.}
    \OptItem{\OptArg{-maxpost}{NumberOfPOSTscripts}}{Sets the maximum number of
       POST scripts within the DAG that may be running at one time.
       \Arg{NumberOfPOSTScripts} is a positive integer. 
        This command-line option to \Condor{submit\_dag} is passed to
        \Condor{dagman}.
       If not specified,
       the default number of POST scripts is unlimited.}
    \OptItem{\Opt{-noeventchecks}}{This argument is no longer used;
       it is now ignored.  Its functionality is now implemented by
       the \MacroNI{DAGMAN\_ALLOW\_EVENTS} configuration macro
       (see section~\ref{param:DAGManAllowEvents}).}
    \OptItem{\Opt{-allowlogerror}}{This optional argument has 
       \Condor{dagman} try to run the specified DAG, even in the case
       of detected errors in the user log specification. }
    \OptItem{\Opt{-usedagdir}}{This optional argument has causes
       \Condor{dagman} to run each specified DAG as if the directory
       containing that DAG file was the current working directory.  This
       option is most useful when running multiple DAGs in a single
       \Condor{dagman}.}
    \OptItem{\OptArg{-storklog}{filename}}{Sets the file name of
       the Stork log for data placement jobs.  }
    \OptItem{\OptArg{-condorlog}{filename}}{Sets the file name of
       the file used in conjunction with the \OptArg{-lockfile}{filename}
       in determining whether to run in recovery mode.  }
    \OptItem{\OptArg{-lockfile}{filename}}{Names the file
       created and used as a lock file.
       The lock file prevents execution of two of the 
       same DAG, as defined by a DAG input file.
       A default lock file ending with the suffix \File{.dag.lock}
       is passed to \Condor{dagman} by \Condor{submit\_dag}.  }
    \OptItem{\Opt{-waitfordebug}}{This optional argument causes
       \Condor{dagman} to wait at startup until someone attaches to
       the process with a debugger and sets the wait\_for\_debug
       variable in main\_init() to false.}
    \OptItem{\OptArg{-autorescue}{0|1}}{Whether to automatically run
       the newest rescue DAG for the given DAG file, if one exists
       (0 = \Expr{false}, 1 = \Expr{true}).}
    \OptItem{\OptArg{-dorescuefrom}{number}}{Forces \Condor{dagman} to
       run the specified rescue DAG number for the given DAG.  A value
       of 0 is the same as not specifying this option.  Specifying a
       non-existant rescue DAG is a fatal error.}
    \OptItem{\OptArg{-csdversion}{version\_string}}{\Arg{version\_string}
       is the version of the \Condor{submit\_dag} program.  At startup,
       \Condor{dagman} checks for a version mismatch with the
       \Condor{submit\_dag} version in this argument.}
    \OptItem{\OptArg{-dag}{filename}}{\Arg{filename} is the name of the
       DAG input file that is set as an argument to \Condor{submit\_dag},
       and passed to \Condor{dagman}.}
\end{Options}


\ExitStatus

\Condor{dagman} will exit with a status value of 0 (zero) upon success,
and it will exit with the value 1 (one) upon failure.

\Examples

\Condor{dagman} is normally not run directly, but submitted as a Condor
job by running \condor{submit\_dag}.  See the \condor{submit\_dag} manual
page~\pageref{man-condor-submit-dag} for examples.

\end{ManPage}
