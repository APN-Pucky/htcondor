\begin{ManPage}{\label{man-condor-chirp}\Condor{chirp}}{1}
{Access files or job ClassAd from an executing job}
\Synopsis
\SynProg{\Condor{chirp}}
\oOpt{-help} 

\SynProg{\Condor{chirp}}
\Opt{fetch} \Arg{RemoteFileName} \Arg{LocalFileName}

\SynProg{\Condor{chirp}}
\Opt{put} \oOptArg{-mode}{mode} \oOptArg{-perm}{UnixPerm} \Arg{LocalFileName} \Arg{RemoteFileName}

\SynProg{\Condor{chirp}}
\Opt{remove} \Arg{RemoteFileName}

\SynProg{\Condor{chirp}}
\Opt{get\_job\_attr} \Arg{JobAttributeName}

\SynProg{\Condor{chirp}}
\Opt{set\_job\_attr} \Arg{JobAttributeName} \Arg{AttributeValue}

\Description 
\index{Condor commands!condor\_chirp}
\index{condor\_chirp command}
\Condor{chirp} is run run from a user job while executing.
It accesses files or job ClassAd attributes on the submit machine.
Files can be read, written or removed.
Job attributes can be read, and most attributes can be updated.

A description using of the terms \Term{local} and \Term{remote}
are given from the point of view of the executing program.

If the input file name for \Opt{put} is a dash,
\Condor{chirp} uses standard input as the source.
If the output file name for \Opt{fetch} is a dash,
\Condor{chirp} writes to standard output instead of a local file.

Jobs that use \Condor{chirp} must have the attribute
\Attr{WantIOProxy} set to \Expr{True} in the job ad.
To do this, place
\begin{verbatim}
+WantIOProxy = true
\end{verbatim}
in the submit description file for the job.

\Condor{chirp} only works for jobs run in the
vanilla, mpi, parallel and java universes.

The optional \OptArg{-mode}{mode} argument
is one or more of the following characters describing the
\Arg{RemoteFileName} file.
  \begin{itemize}
    \item{w:  open for writing}
    \item{a:  force all writes to append}
    \item{t:  truncate before use}
    \item{c:  create the file, if it does not exist}
    \item{x:  fail if 'c' is given, and the file already exists}
  \end{itemize}

The optional \OptArg{-perm}{UnixPerm} argument
describes the file access permissions in a Unix format
(for example, 660).

\begin{Options}
  \OptItem{\Opt{-help}}{Display usage information and exit.}
  \OptItem{fetch}{Copy the \Arg{RemoteFileName} from the submit machine
    to the execute machine.}
  \OptItem{remove}{Remove the \Arg{RemoteFileName} file from the 
    submit machine.}
  \OptItem{put}{Copy the \Arg{LocalFileName} from the execute machine
    to the submit machine.  Perm is the unix permission to open the file with.}
  \OptItem{get\_job\_attr}{Prints the named job ClassAd attribute to
    standard output.}
  \OptItem{set\_job\_attr}{Sets the named job ClassAd attribute with
    the given attribute value.}
\end{Options}

\Examples

To copy a file from the submit machine to the execute machine while the 
user job is running, run

\footnotesize
\begin{verbatim}
% condor_chirp fetch remotefile localfile
\end{verbatim}
\normalsize

To print to standard output the value of the \Attr{Requirements}
expression from within a running job, run

\footnotesize
\begin{verbatim}
% condor_chirp get_job_attr Requirements
\end{verbatim}
\normalsize

Note that the remote (submit-side) directory path is relative to the
submit directory, and the local (execute-side) directory is relative to the
current directory of the running program.

To append the word "foo" to a file on the submit machine, run

\footnotesize
\begin{verbatim}
% echo foo | condor_chirp put -mode wat - RemoteFile
\end{verbatim}
\normalsize


\ExitStatus

\Condor{chirp} will exit with a status value of 0 (zero) upon success,
and it will exit with the value 1 (one) upon failure.

\end{ManPage}
