\begin{ManPage}{\label{man-filelock-undertaker}\Prog{filelock\_undertaker}}{1}
{determine whether a process has exited}

\Synopsis \SynProg{\Prog{filelock\_undertaker}}
\Opt{-help}

\SynProg{\Prog{filelock\_undertaker}}
\oOptArg{--file}{filename}
\oOpt{--block}

\index{Deployment commands!filelock\_undertaker}
\index{filelock\_undertaker}

\Description
\Prog{filelock\_undertaker} can examine an artifact file created by
\Prog{filelock\_midwife} and determine whether the program started by
the \Prog{midwife} has exited.  It does this by attempting to acquire
a file lock.

Be warned that this will not work on NFS unless the separate
file lock server is running.

\begin{Options}
  \OptItem{\Opt{--block}}{
    If the process has not exited, block until it does.
  }
  \OptItem{\OptArg{--file}{filename}}{
    The name of the artifact file.
    created by \Prog{filelock\_midwife}.
    The file \File{lock.file} is the default file used when
    this option is not specified.
  }
\end{Options}

\ExitStatus 
\Prog{filelock\_undertaker} will exit with a status of 0 (zero) if the
monitored process has exited, with a status of 1 (one) if the
monitored process has definitely not exited, with a status of 2 if it
is uncertain whether the process has exited (this is generally due to
a failure by the \Prog{filelock\_midwife}), or with any other value
for program failure.

\SeeAlso
\Prog{uniq\_pid\_undertaker} (on page~\pageref{man-uniq-pid-undertaker}),
\Prog{filelock\_midwife} (on page~\pageref{man-filelock-midwife}).

\end{ManPage}
