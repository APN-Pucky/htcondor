\begin{ManPage}{\label{man-condor-rm}\Condor{rm}}{1}
{remove jobs from the condor queue}
\Synopsis \SynProg{\Condor{rm}}
\oOptArg{-n}{schedd\_name}
\oOpt{-help}
\oOpt{-version}
\oOpt{job identifiers}

\index{Condor commands!condor\_rm}
\index{condor\_rm command}

\Description

\Condor{rm} removes one or more jobs from the Condor job queue.  
If the \Opt{-n} option is specified, the named \Condor{schedd} is targeted
for processing.  
Otherwise, the local \Condor{schedd} is targeted.
The jobs to be removed are identified by one or more job identifiers, as
described below.
For any given job, only the owner of the job or one of the queue super users
(defined by the \MacroNI{QUEUE\_SUPER\_USERS} macro) can remove the job.

\begin{Options}
    \OptItem{\Opt{-help}}{Display usage information and exit}
    \OptItem{\Opt{-version}}{Display version information and exit}
    \OptItem{\OptArg{-n}{schedd\_name}}{Remove jobs in the queue of the 
		specified schedd}
	\OptItem{\Arg{cluster}}{(Job identifier.) Remove all jobs in the specified 
		cluster}
	\OptItem{\Arg{cluster.process}}{(Job identifier.) Remove the specific job 
		in the cluster}
	\OptItem{\Arg{name}}{(Job identifier.) Remove jobs belonging to specified 
		user}
    \OptItem{\Opt{-a}}{(Job identifier.) Remove all the jobs in the queue}
	\OptItem{\Arg{-constraint \emph{constraint}}}{(Job identifier.) Remove 
		jobs matching specified constraint}
\end{Options}

\GenRem

When removing a PVM universe job, you should always remove the entire
job cluster.  (In the PVM universe, each PVM job is assigned its own
cluster number, and each machine class is assigned a ``process''
number in the job's cluster.)  Removing a subset of the machine
classes for a PVM job is not supported.

\end{ManPage}
