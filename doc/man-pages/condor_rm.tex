\begin{ManPage}{\label{man-condor-rm}\Condor{rm}}{1}
{remove jobs from the Condor queue}
\Synopsis \SynProg{\Condor{rm}}
\ToolArgsBase

\SynProg{\Condor{rm}}
\ToolLocate
\ToolJobs
$|$ \OptArg{-constraint}{expression} \Dots

\SynProg{\Condor{rm}}
\ToolLocate
\ToolAll

\index{Condor commands!condor\_rm}
\index{condor\_rm command}

\Description

\Condor{rm} removes one or more jobs from the Condor job queue.  
If the \Opt{-name} option is specified, the named \Condor{schedd} is targeted
for processing.  
Otherwise, the local \Condor{schedd} is targeted.
The jobs to be removed are identified by one or more job identifiers, as
described below.
For any given job, only the owner of the job or one of the queue super users
(defined by the \MacroNI{QUEUE\_SUPER\_USERS} macro) can remove the job.

\begin{Options}
	\ToolArgsBaseDesc
	\ToolLocateDesc
	\OptItem{\Arg{cluster}}{Remove all jobs in the specified cluster}
	\OptItem{\Arg{cluster.process}}{Remove the specific job in the cluster}
	\OptItem{\Arg{user}}{Remove jobs belonging to specified user}
	\OptItem{\OptArg{-constraint}{expression}} {Remove all jobs which match
	                the job ClassAd expression constraint}
        \OptItem{\Arg{-all}}{Remove all the jobs in the queue}
	\OptItem{\Arg{-forcex}}{Force the immediate local removal of
	jobs in the 'X' state (only affects jobs already being removed)
\end{Options}

\GenRem

When removing a PVM universe job, you should always remove the entire
job cluster.  (In the PVM universe, each PVM job is assigned its own
cluster number, and each machine class is assigned a ``process''
number in the job's cluster.)  Removing a subset of the machine
classes for a PVM job is not supported.

Use the \Arg{-forcex} argument with caution, as it will remove jobs
from the local queue immediately, but can ``orphan'' parts of the job
that are running remotely and haven't yet been stopped or removed.

\Examples
To remove all jobs of a user named Mary that are not currently running:
\footnotesize
\begin{verbatim}
% condor_rm Mary -constraint Activity!=\"Busy\"
\end{verbatim}
\normalsize
Note that quotation marks must be escaped with the backslash characters
for most shells.

\ExitStatus

\Condor{rm} will exit with a status value of 0 (zero) upon success,
and it will exit with the value 1 (one) upon failure.

\end{ManPage}
