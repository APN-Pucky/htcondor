\begin{ManPage}{\label{man-condor-rm}\Condor{rm}}{1}
{remove jobs from the Condor queue}
\Synopsis \SynProg{\Condor{rm}}
\ToolArgsBase
\ToolLocate
\ToolJobs

\SynProg{\Condor{rm}}
\ToolArgsBase
\ToolLocate
\ToolAll

\index{Condor commands!condor\_rm}
\index{condor\_rm command}

\Description

\Condor{rm} removes one or more jobs from the Condor job queue.  
If the \Opt{-name} option is specified, the named \Condor{schedd} is targeted
for processing.  
Otherwise, the local \Condor{schedd} is targeted.
The jobs to be removed are identified by one or more job identifiers, as
described below.
For any given job, only the owner of the job or one of the queue super users
(defined by the \MacroNI{QUEUE\_SUPER\_USERS} macro) can remove the job.

\begin{Options}
	\ToolArgsBaseDesc
	\ToolLocateDesc
	\OptItem{\Arg{cluster}}{Remove all jobs in the specified cluster}
	\OptItem{\Arg{cluster.process}}{Remove the specific job in the cluster}
	\OptItem{\Arg{user}}{Remove jobs belonging to specified user}
        \OptItem{\Arg{-all}}{Remove all the jobs in the queue}
\end{Options}

\GenRem

When removing a PVM universe job, you should always remove the entire
job cluster.  (In the PVM universe, each PVM job is assigned its own
cluster number, and each machine class is assigned a ``process''
number in the job's cluster.)  Removing a subset of the machine
classes for a PVM job is not supported.

\ExitStatus

\Condor{rm} will exit with a status value of 0 (zero) upon success,
and it will exit with the value 1 (one) upon failure.

\end{ManPage}
