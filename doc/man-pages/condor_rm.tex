\begin{ManPage}{\label{man-condor-rm}\Condor{rm}}{1}
{remove jobs from the HTCondor queue}
\index{HTCondor commands!condor\_rm}
\index{condor\_rm command}

\Synopsis \SynProg{\Condor{rm}}
\ToolArgsBase

\SynProg{\Condor{rm}}
\ToolDebugOption
\oOpt{-forcex}
\ToolLocate
\ToolJobs
\Bar{} \OptArg{-constraint}{expression} \Dots

\SynProg{\Condor{rm}}
\ToolDebugOption
\ToolLocate
\ToolAll

\Description

\Condor{rm} removes one or more jobs from the HTCondor job queue.  
If the \Opt{-name} option is specified, the named \Condor{schedd} is targeted
for processing.  
Otherwise, the local \Condor{schedd} is targeted.
The jobs to be removed are identified by one or more job identifiers, as
described below.
For any given job, only the owner of the job or one of the queue super users
(defined by the \MacroNI{QUEUE\_SUPER\_USERS} macro) can remove the job.

When removing a grid job, the job may remain in
the ``X'' state for a very long time. 
This is normal, as HTCondor is attempting to communicate with the
remote scheduling system, 
ensuring that the job has been properly cleaned up.
If it takes too long, or in rare circumstances is never removed,
the job may be forced to
leave the job queue by using the \Opt{-forcex} option.
This forcibly removes jobs that are in the ``X'' state without attempting
to finish any clean up at the remote scheduler.

\begin{Options}
	\ToolArgsBaseDesc
	\ToolLocateDesc
        \ToolDebugDesc
	\OptItem{\Opt{-forcex}}{Force the immediate local removal of
	jobs in the 'X' state (only affects jobs already being removed)}
	\OptItem{\Arg{cluster}}{Remove all jobs in the specified cluster}
	\OptItem{\Arg{cluster.process}}{Remove the specific job in the cluster}
	\OptItem{\Arg{user}}{Remove jobs belonging to specified user}
	\OptItem{\OptArg{-constraint}{expression}} {Remove all jobs which match
	                the job ClassAd expression constraint}
        \OptItem{\Opt{-all}}{Remove all the jobs in the queue}
\end{Options}

\GenRem

Use the \Arg{-forcex} argument with caution, as it will remove jobs
from the local queue immediately, but can orphan parts of the job
that are running remotely and have not yet been stopped or removed.

\Examples
For a user to remove all their jobs that are not currently running:
\footnotesize
\begin{verbatim}
% condor_rm -constraint 'JobStatus =!= 2'
\end{verbatim}
\normalsize

\ExitStatus

\Condor{rm} will exit with a status value of 0 (zero) upon success,
and it will exit with the value 1 (one) upon failure.

\end{ManPage}
