\begin{ManPage}{\label{man-condor-rmdir}\Condor{rmdir}}{1}
{Windows-only no-fail deletion of directories}
\index{HTCondor commands!condor\_rmdir}
\index{condor\_rmdir command}

\Synopsis
\SynProg{\Condor{rmdir}}
\oOpt{/HELP \Bar /?}

\SynProg{\Condor{rmdir}}
\Arg{@filename}

\SynProg{\Condor{rmdir}}
\oOpt{/VERBOSE}
\oOpt{/DIAGNOSTIC}
\oOpt{/PATH:<path>}
\oOpt{/S}
\oOpt{/C}
\oOpt{/Q}
\oOpt{/NODEL}
\Arg{directory}

\Description 

\Condor{rmdir} can delete a specified \Arg{directory},
and will not fail if the directory contains files that have ACLs 
that deny the SYSTEM process delete access,
unlike the built-in Windows \Prog{rmdir} command. 

The directory to be removed together with other command line arguments
may be specified within a file named \Arg{filename},
prefixing this argument with an \Expr{@} character.

The \Condor{rmdir.exe} executable is is intended to be used  
by HTCondor with the \Opt{/S} \Opt{/C}  options, 
which cause it to recurse into subdirectories and continue on errors.

\begin{Options}

  \OptItem{\Opt{/HELP}}{Print usage information.  }

  \OptItem{\Opt{/?}}{Print usage information.  }

  \OptItem{\Opt{/VERBOSE}}{Print detailed output.  }

  \OptItem{\Opt{/DIAGNOSTIC}}{Print out the internal flow of control
   information.  }

  \OptItem{\Opt{/PATH:<path>}}{Remove the directory given by \Opt{<path>}.  }

  \OptItem{\Opt{/S}}{Include subdirectories in those removed.  }

  \OptItem{\Opt{/C}}{Continue even if access is denied.  }

  \OptItem{\Opt{/Q}}{Print error output only.  }

  \OptItem{\Opt{/NODEL}}{Do not remove directories.  ACLs may still be
   changed. }

\end{Options}

\ExitStatus

\Condor{rmdir} will exit with a status value of 0 (zero) upon success,
and it will exit with the standard HRESULT error code upon failure.

\end{ManPage}
