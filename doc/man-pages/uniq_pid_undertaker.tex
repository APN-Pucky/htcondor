\begin{ManPage}{\label{man-uniq-pid-undertaker}\Prog{uniq\_pid\_undertaker}}{1}
{determine whether a process has exited}

\Synopsis \SynProg{\Prog{uniq\_pid\_undertaker}}
\oOpt{-\,-block}
\oOptArg{-\,-file}{file}
\oOptArg{-\,-precision}{seconds}

\index{Deployment commands!uniq\_pid\_undertaker}
\index{uniq\_pid\_undertaker}

\Description
\Prog{uniq\_pid\_undertaker} can examine an artifact file created by
\Prog{uniq\_pid\_midwife} and determine whether the program started by
the \Prog{midwife} has exited.

\begin{Options}
  \OptItem{\Opt{-\,-block}}{
    If the process has not exited, block until it does.
  }
  \OptItem{\OptArg{-\,-file}{file}}{
    The name of the \Prog{uniq\_pid\_midwife} created artifact file.
    Defaults to \File{pid.file}.
  }
  \OptItem{\OptArg{-\,-precision}{seconds}}{
    Uses \Arg{seconds} as the precision range within which the
    operating system will provide a process's birthday.  Defaults to
    an operating system specific value.  Only use this option if the
    same \Arg{seconds} value was provided to
    \Prog{uniq\_pid\_midwife}.
  }
\end{Options}

\ExitStatus 
\Prog{uniq\_pid\_undertaker} will exit with a status of 0 (zero) if
the monitored process has exited, with a status of 1 (one) if the
monitored process has definitely not exited, with a status of 2 if it
is uncertain whether the process has exited (this is generally due to
a failure by the \Prog{uniq\_pid\_midwife}), or with any other value
for program failure.

\SeeAlso
\Prog{uniq\_pid\_midwife} (on page~\pageref{man-uniq-pid-midwife}),
\Prog{filelock\_undertaker} (on page~\pageref{man-filelock-undertaker}).

\end{ManPage}
