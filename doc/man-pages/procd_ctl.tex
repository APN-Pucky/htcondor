\begin{ManPage}{procd\_ctl}{man-procd-ctl}{}{1}
{command line interface to the \Condor{procd}}
\index{HTCondor commands!procd\_ctl}
\index{procd\_ctl command}

\Synopsis \SynProg{procd\_ctl} 
\Opt{-h}

\SynProg{procd\_ctl} 
\OptArg{-A}{address-file}
\oOpt{command}

\Description 

This is a programmatic interface to the \Condor{procd} daemon. 
It may be used to 
cause the \Condor{procd} to do anything that the \Condor{procd}
is capable of doing,
such as tracking and managing process families.

This is a program only available for the Linux ports of HTCondor.

The \Opt{-h} option prints out usage information and exits.
The  \Arg{address-file} specification within the \Opt{-A} argument
specifies the path and file name of the address file
which the named pipe clients must use to speak with the \Condor{procd}.

One command is given to the \Condor{procd}. 
The choices for the command are defined by the Options. 

\begin{Options}

  \OptItem{\Opt{TRACK\_BY\_ASSOCIATED\_GID} \Arg{GID} \oArg{PID}}
  {Use the specified \Arg{GID} to track the specified family rooted at 
  \Arg{PID}.  
  If the optional \Arg{PID} is not specified, 
  then the PID used is the one given or assumed by \Condor{procd}.}

  \OptItem{\Opt{GET\_USAGE} \oArg{PID}}
  {Get the total usage information about the PID family at \Arg{PID}.
  If the optional \Arg{PID} is not specified, 
  then the PID used is the one given or assumed by \Condor{procd}.}

  \OptItem{\Opt{DUMP} \oArg{PID}}
  {Print out information about both the root \Arg{PID} being watched 
  and the tree of processes under this root \Arg{PID}.
  If the optional \Arg{PID} is not specified, 
  then the PID used is the one given or assumed by \Condor{procd}.}

  \OptItem{\Opt{LIST} \oArg{PID}}
  {With no \Arg{PID} given, print out information about all 
  the watched processes.  
  If the optional \Arg{PID} is specified,
  print out information about the process specified by \Arg{PID} 
  and all its child processes.}

  \OptItem{\Opt{SIGNAL\_PROCESS} \Arg{signal} \oArg{PID}}
  {Send the \Arg{signal} to the process specified by \Arg{PID}.
  If the optional \Arg{PID} is not specified, 
  then the PID used is the one given or assumed by \Condor{procd}.}

  \OptItem{\Opt{SUSPEND\_FAMILY} \Arg{PID}}
  {Suspend the process family rooted at \Arg{PID}.}

  \OptItem{\Opt{CONTINUE\_FAMILY} \Arg{PID}}
  {Continue execution of the process family rooted at \Arg{PID}.}

  \OptItem{\Opt{KILL\_FAMILY} \Arg{PID}}
  {Kill the process family rooted at \Arg{PID}.}

  \OptItem{\Opt{UNREGISTER\_FAMILY} \Arg{PID}}
  {Stop tracking the process family rooted at \Arg{PID}.}

  \OptItem{\Opt{SNAPSHOT}}
  {Perform a snapshot of the tracked family tree.}

  \OptItem{\Opt{QUIT}}
  {Disconnect from the \Condor{procd} and exit.}

\end{Options}
	

\GenRem

This program may be used in a standalone mode, independent of
HTCondor, to track process families. The programs \Prog{procd\_ctl} and
\Prog{gidd\_alloc} are used with the \Condor{procd} in standalone mode
to interact with the daemon and inquire about certain state of running
processes on the machine, respectively.

\ExitStatus

\Prog{procd\_ctl} will exit with a status value of 0 (zero) upon success,
and it will exit with the value 1 (one) upon failure.

\end{ManPage}
