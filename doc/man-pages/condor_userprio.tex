\begin{ManPage}{\label{man-condor-userprio}\Condor{userprio}}{1}
{Manage user priorities} 
\index{HTCondor commands!condor\_userprio}
\index{condor\_userprio command}

\Synopsis \SynProg{\Condor{userprio}}
\Opt{-help}

\SynProg{\Condor{userprio}}
\oOptArg{-pool}{centralmanagerhostname[:portnumber]}
\oOpt{Edit option} $|$ \oOpt{Display options}
\oOptArg{-inputfile}{filename}

\Description 
\Condor{userprio} either modifies priority-related information
or displays priority-related information.
Displayed information comes from the accountant log,
where the \Condor{negotiator} daemon stores historical usage information
in the file at \File{\MacroUNI{SPOOL}/Accountantnew.log}.
Which fields are displayed changes based on command line arguments.
\Condor{userprio} with no arguments, 
lists the active users along with their priorities,
in increasing priority order. 
The \Opt{-all} option can be used to display
more detailed information about each user, 
resulting in a rather wide display,
and includes the following columns:

\begin{description}
\item[Effective Priority] The effective priority value of the user, 
which is used to calculate the user's share when allocating resources. 
A lower value means a higher priority, 
and the minimum value (highest priority) is 0.5. 
The effective priority is
calculated by multiplying the real priority by the priority factor.
\item[Real Priority] The value of the real priority of the user. 
This value follows the user's resource usage.
\item[Priority Factor] The system administrator can set this value 
for each user, 
thus controlling a user's effective priority relative to other users. 
This can be used to create different classes of users.
\item[Res Used] The number of resources currently used. 
\item[Accumulated Usage] The accumulated number of resource-hours 
used by the user since the usage start time.
\item[Usage Start Time] The time since when usage has been recorded 
for the user. 
This time is set when a user job runs for the first time. 
It is reset to the present time when the
usage for the user is reset.
\item[Last Usage Time] The most recent time a resource usage has been recorded 
for the user.
\end{description}

By default only users for whom usage was recorded in the last 24 hours, 
or whose priority is greater than the minimum are listed.

The \Opt{-pool} option can be used to contact a different central manager 
than the local one (the default).


For security purposes of authentication and authorization,
specifying an Edit Option requires the ADMINISTRATOR level of access.

\begin{Options}

\OptItem{\Opt{-help}}
  {Display usage information and exit.}

\OptItem{\OptArg{-pool}{centralmanagerhostname[:portnumber]}}
  {Contact the specified \Arg{centralmanagerhostname} with an
  optional port number, instead of the local central manager. 
  This can be used to check other pools.  
  NOTE: The host name (and optional port) specified refer to the host name
  (and port) of the \Condor{negotiator} to query for user priorities.
  This is slightly different than most HTCondor tools
  that support a \Opt{-pool} option,
  and instead expect the host name (and port) of the \Condor{collector}.}

\OptItem{\OptArg{-inputfile}{filename}}
  {Introduced for debugging purposes,
   read priority information from \Arg{filename}.
   The contents of \Arg{filename} are expected to be the same as captured
   output from running a \texttt{condor\_userprio -long} command.}

\OptItem{\OptArg{-delete}{username}}
  {(Edit option) Remove the specified \Arg{username} from HTCondor's accounting.}

\OptItem{\Opt{-resetall}}
  {(Edit option) Reset the accumulated usage of all the users to zero.}

\OptItem{\OptArg{-resetusage}{username}}
  {(Edit option) Reset the accumulated usage of the user specified by
  \Arg{username} to zero. }

\OptItem{\OptArg{-setaccum}{username value}}
  {(Edit option) Set the accumulated usage of the user specified by
  \Arg{username} to the specified floating point \Arg{value}. }

\OptItem{\OptArg{-setbegin}{username value}}
  {(Edit option) Set the begin usage time of the user specified by
  \Arg{username} to the specified \Arg{value}. }

\OptItem{\OptArg{-setfactor}{username value}}
  {(Edit option) Set the priority factor of the user specified by
  \Arg{username} to the specified \Arg{value}.  }

\OptItem{\OptArg{-setlast}{username value}}
  {(Edit option) Set the last usage time of the user specified by
  \Arg{username} to the specified \Arg{value}.  }

\OptItem{\OptArg{-setprio}{username value}}
  {(Edit option) Set the real priority of the user specified by
  \Arg{username} to the specified \Arg{value}.  }

\OptItem{\OptArg{-activefrom}{month day year}}
  {(Display option) Display information for users who have 
  some recorded accumulated usage since the specified date.  }

\OptItem{\Opt{-all}}
  {(Display option) Display all available fields about each group or user.}

\OptItem{\Opt{-allusers}}
  {(Display option) Display information for all the users 
  who have some recorded accumulated usage.}

\OptItem{\Opt{-debug[:<opts>]}}
  {(Display option) Without \Opt{:<opts>} specified, 
  use configured debug level to send debugging output to \File{stderr}.
  With \Opt{:<opts>} specified, these options are debug levels that
  override any configured debug levels for this command's execution
  to send debugging output to \File{stderr}.}

\OptItem{\Opt{-flat}}
  {(Display option) Display information such that users within hierarchical
  groups are \emph{not} listed with their group. }

\OptItem{\OptArg{-getreslist}{username}}
  {(Display option) Display all the resources currently allocated to the 
  user specified by \Arg{username}.  }

\OptItem{\Opt{-grouporder}}
  {(Display option) Display submitter information with accounting group
   entries at the top of the list, 
   and in breadth-first order within the group hierarchy tree.}

\OptItem{\Opt{-grouprollup}}
  {(Display option) For hierarchical groups,
  the display shows sums as computed for groups, 
  and these sums include sub groups.  }

\OptItem{\Opt{-hierarchical}}
  {(Display option) Display information such that users within hierarchical
  groups are listed with their group. }

\OptItem{\Opt{-long}}
  {(Display option) A verbose output which displays entire ClassAds.  }

\OptItem{\Opt{-most}}
  {(Display option) Display fields considered to be the most useful.
  This is the default set of fields displayed.   }

\OptItem{\Opt{-priority}}
  {(Display option) Display fields with user priority information.  }

\OptItem{\Opt{-quotas}}
  {(Display option) Display fields relevant to hierarchical group quotas.  }

\OptItem{\Opt{-usage}}
  {(Display option) Display usage information for each group or user.}

\end{Options}

\Examples

\underline{Example 1} Since the output varies due to command line arguments, 
here is an example of the default output for a pool that does not use 
Hierarchical Group Quotas.
This default output is the same as given with the \Opt{-most} Display option.
\footnotesize
\begin{verbatim}
Last Priority Update:  1/19 13:14
                        Effective   Priority   Res   Total Usage  Time Since
User Name                Priority    Factor   In Use (wghted-hrs) Last Usage
---------------------- ------------ --------- ------ ------------ ----------
www-cndr@cs.wisc.edu           0.56      1.00      0    591998.44    0+16:30
joey@cs.wisc.edu               1.00      1.00      1       990.15 <now>
suzy@cs.wisc.edu               1.53      1.00      0       261.78    0+09:31
leon@cs.wisc.edu               1.63      1.00      2     12597.82 <now>
raj@cs.wisc.edu                3.34      1.00      0      8049.48    0+01:39
jose@cs.wisc.edu               3.62      1.00      4     58137.63 <now>
betsy@cs.wisc.edu             13.47      1.00      0      1475.31    0+22:46
petra@cs.wisc.edu            266.02    500.00      1    288082.03 <now>
carmen@cs.wisc.edu           329.87     10.00    634   2685305.25 <now>
carlos@cs.wisc.edu           687.36     10.00      0     76555.13    0+14:31
ali@proj1.wisc.edu          5000.00  10000.00      0      1315.56    0+03:33
apu@nnland.edu              5000.00  10000.00      0       482.63    0+09:56
pop@proj1.wisc.edu         26688.11  10000.00      1     49560.88 <now>
franz@cs.wisc.edu          29352.06    500.00    109    600277.88 <now>
martha@nnland.edu          58030.94  10000.00      0     48212.79    0+12:32
izzi@nnland.edu            62106.40  10000.00      0      6569.75    0+02:26
marta@cs.wisc.edu          62577.84    500.00     29    193706.30 <now>
kris@proj1.wisc.edu       100597.94  10000.00      0     20814.24    0+04:26
boss@proj1.wisc.edu       318229.25  10000.00      3    324680.47 <now>
---------------------- ------------ --------- ------ ------------ ----------
Number of users: 19                              784   4969073.00    0+23:59 
\end{verbatim}
\normalsize

\underline{Example 2} This is an example of the default output for a pool
that uses hierarchical groups, and the groups accept surplus.
This leads to a very wide display.
\footnotesize
\begin{verbatim}
% condor_userprio -pool crane.cs.wisc.edu -allusers
Last Priority Update:  1/19 13:18
Group                                 Config     Use    Effective   Priority   Res   Total Usage  Time Since
  User Name                            Quota   Surplus   Priority    Factor   In Use (wghted-hrs) Last Usage
------------------------------------ --------- ------- ------------ --------- ------ ------------ ----------
<none>                                    0.00     yes                   1.00      0         6.78    9+03:52
  johnsm@crane.cs.wisc.edu                                     0.50      1.00      0         6.62    9+19:42
  John.Smith@crane.cs.wisc.edu                                 0.50      1.00      0         0.02    9+03:52
  Sedge@crane.cs.wisc.edu                                      0.50      1.00      0         0.05   13+03:03
  Duck@crane.cs.wisc.edu                                       0.50      1.00      0         0.02   31+00:28
  other@crane.cs.wisc.edu                                      0.50      1.00      0         0.04   16+03:42
Duck                                      2.00      no                   1.00      0         0.02   13+02:57
  goose@crane.cs.wisc.edu                                      0.50      1.00      0         0.02   13+02:57
Sedge                                     4.00      no                   1.00      0         0.17    9+03:07
  johnsm@crane.cs.wisc.edu                                     0.50      1.00      0         0.13    9+03:08
  Half@crane.cs.wisc.edu                                       0.50      1.00      0         0.02   31+00:02
  John.Smith@crane.cs.wisc.edu                                 0.50      1.00      0         0.05    9+03:07
  other@crane.cs.wisc.edu                                      0.50      1.00      0         0.01   28+19:34
------------------------------------ --------- ------- ------------ --------- ------ ------------ ----------
Number of users: 10                            ByQuota                             0         6.97 
\end{verbatim}
\normalsize

\ExitStatus

\Condor{userprio} will exit with a status value of 0 (zero) upon success,
and it will exit with the value 1 (one) upon failure.

\end{ManPage}
