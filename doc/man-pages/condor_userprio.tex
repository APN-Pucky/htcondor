\begin{ManPage}{\label{man-condor-userprio}\Condor{userprio}}{1}
{Manage user priorities} 
\Synopsis \SynProg{\Condor{userprio}}
\oOptArg{-pool}{hostname}
\oOpt{-all}
\oOpt{-usage}
\oOptArg{-setprio}{username value}
\oOptArg{-setfactor}{username value}
\oOptArg{-resetusage}{username}
\oOpt{-resetall}
\oOptArg{-getreslist}{username}
\oOpt{-allusers}
\oOptArg{-activefrom}{month day year}
\oOpt{-l}

\index{Condor commands!condor\_userprio}
\index{condor\_userprio command}

\Description 
\Condor{userprio} with no arguments, lists the active users (see below) along with their
priorities, in increasing priority order. The -all option can be used to display
more detailed information about each user, which includes the following columns:

\begin{description}
\item[Effective Priority] The effective priority value of the user, which is used to
calculate the user's share when allocating resources. A lower value means a higher
priority, and the minimum value (highest priority) is 0.5. The effective priority is
calculated by multiplying the real priority by the priority factor.
\item[Real Priority] The value of the real priority of the user. This value follows the
user's resource usage.
\item[Priority Factor] The system administrator can set this value for each user, thus 
controlling a user's effective priority relative to other users. This can be used to
create different classes of users.
\item[Res Used] The number of resources currently used (e.g. the number of running jobs
for that user).
\item[Accumulated Usage] The accumulated number of resource-hours used by the user since
the usage start time.
\item[Usage Start Time] The time since when usage has been recorded for the user. This time
is set when a user job runs for the first time. It is reset to the present time when the
usage for the user is reset (with the -resetusage or -resetall options).
\item[Last Usage Time] The most recent time a resource usage has been recorded for the user.
\end{description}

The -usage option displays the username, accumulated usage, usage start time and last usage time
for each user, sorted on accumulated usage.

The -setprio, -setfactor options are used to change a user's real priority and priority factor.
The -resetusage and -resetall options are used to reset the accumulated usage for users. The
usage start time is set to the current time when the accumulated usage is reset. These
options require administrator privilages.

By default only users for whom usage was recorded in the last 24 hours or whose priority is
greater than the minimum are listed. The -activefrom and -allusers options can be used
to display users who had some usage since a specified date, or ever. The summary line for 
last usage time will show this date.

The -getreslist option is used to display the resources currently used by a user. The
output includes the start time (the time the resource was allocated to the user), and
the match time (how long has the resource been allocated to the user).

Note that when specifying usernames on the command line, the name must include the
uid domain (e.g. user@uid-domain - exactly the same way usernames are listed by the
userprio command).

The -pool option can be used to contact a different central-manager instead of
the local one (the default).

\begin{Options}

\OptItem{\OptArg{-pool}{hostname}}
	{Contact the specified hostname instead of the local central manager. This
can be used to check other pools.}

\OptItem{\Opt{-all}}
	{Display detailed information about each user.}

\OptItem{\Opt{-usage}}
	{Display usage information for each user.}

\OptItem{\OptArg{-setprio}{username value}}
	{Set the real priority of the specified user to the specified value.}

\OptItem{\OptArg{-setfactor}{username value}}
	{Set the priority factor of the specified user to the specified value.}

\OptItem{\OptArg{-resetusage}{username}}
	{Reset the accumulated usage of the specified user to zero.}

\OptItem{\Opt{-resetall}}
	{Reset the accumulated usage of all the users to zero.}

\OptItem{\OptArg{-getreslist}{username}}
	{Display all the resources currently allocated to the specified user.}

\OptItem{\Opt{-allusers}}
	{Display information for all the users who have some recorded accumulated usage.}

\OptItem{\OptArg{-activefrom}{month day year}}
	{Display information for users who have some recorded accumulated usage since
the specified date.}

\OptItem{\Opt{-l}}
	{Show the class-ad which was received from the central-manager in long format.}

\end{Options}

\ExitStatus

\Condor{userprio} will exit with a status value of 0 (zero) upon success,
and it will exit with the value 1 (one) upon failure.

\end{ManPage}
