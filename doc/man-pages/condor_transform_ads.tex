\begin{ManPage}{\Condor{transform\_ads}}{man-condor-transform-ads}{1}
{Transform ClassAds according to specified rules, and output the transformed
ClassAds.}
\index{HTCondor commands!condor\_transform\_ads}
\index{condor\_transform\_ads command}

\Synopsis
\SynProg{\Condor{transform\_ads}}
\oOpt{-help [rules]}

\SynProg{\Condor{transform\_ads}}
\OptArg{-rules}{rules-file}
\oOptArg{-in[:<form>]}{infile}
\oOptArg{-out[:<form>[, nosort]]}{outfile}
\oArg{<key>=<value>}
\oOpt{-long}
\oOpt{-json}
\oOpt{-xml}
\oOpt{-verbose}
\oOpt{-terse}
\oOpt{-debug}
\oOpt{-unit-test}
\oOpt{-testing}
\oOpt{-convertoldroutes}
\oArg{infile1 \Dots infileN}

Note that exactly one rules file, and at least one input file, must
be specified.  If no output file is specified, output will be written
to \File{stdout}.

\Description

\Condor{transform\_ads} reads ClassAds from a set of input files,
transforms them according to rules defined in a rules file, and outputs
the resulting transformed ClassAds.

See
\URL{https://htcondor-wiki.cs.wisc.edu/index.cgi/wiki?p=TjsAdTransformLanguage}
for a description of the transform language.

\begin{Options}
  \OptItem{\Opt{-help [rules]}}{Display usage information and exit.
  \Opt{-help rules} displays information about the available transformation
  rules.}
  \OptItem{\OptArg{-rules}{rules-file}}{Specifies the file containing
    definitions of the transformation rules.}
  \OptItem{\OptArg{-in[:<form>]}{infile}}{Specifies an input file
    containing ClassAd(s) to be transformed.
    \Opt{<form>}, if specified, is one of:
    \begin{itemize} 
    \item \Opt{long}: traditional long form (default)
    \item \Opt{xml}: XML form
    \item \Opt{json}: JSON ClassAd form
    \item \Opt{new}: "new" ClassAd form without newlines
    \item \Opt{auto}: guess format by reading the input
    \end{itemize}
    If \File{-} is specified for \Arg{infile},
    input is read from \File{stdin}.
    }
  \OptItem{\OptArg{-out[:<form>[, nosort]}{outfile}}{Specifies an
    output file to receive the transformed ClassAd(s).  \Opt{<form>}, if
    specified, is one of:
    \begin{itemize} 
    \item \Opt{long}: traditional long form (default)
    \item \Opt{xml}: XML form
    \item \Opt{json}: JSON ClassAd form
    \item \Opt{new}: "new" ClassAd form without newlines
    \item \Opt{auto}: use the same format as the first input
    \end{itemize} 
    ClassAds are storted by attribute unless \Opt{nosort} is specified.}
  \OptItem{\oArg{<key>=<value>}}{Assign key/value pairs before rules
    file is parsed; can be used to pass arguments to rules.  (More
    detail needed here.)}
  \OptItem{\Opt{-long}}{Use long form for both input and output ClassAd(s).
    (This is the default.)}
  \OptItem{\Opt{-json}}{Use JSON form for both input and output ClassAd(s).}
  \OptItem{\Opt{-xml}}{Use XML form for both input and output ClassAd(s).}
  \OptItem{\Opt{-verbose}}{Verbose mode, echo transform rules as they are executed.}
  \OptItem{\Opt{-terse}}{Disable the \Opt{-verbose} option.}
  \OptItem{\Opt{-debug}}{More information needed here.}
  \OptItem{\Opt{-unit-test}}{More information needed here.}
  \OptItem{\Opt{-testing}}{More information needed here.}
  \OptItem{\Opt{-convertoldroutes}}{More information needed here.}
\end{Options}

\ExitStatus

\Condor{transform\_ads} will exit with a status value of 0 (zero) upon success,
and it will exit with the value 1 (one) upon failure.

\Examples

Here's a simple example that transforms the given input ClassAds according
to the given rules:

\begin{verbatim}
  # File: my_input
  ResidentSetSize = 500
  DiskUsage = 2500000
  NumCkpts = 0
  TransferrErr = false
  Err = "/dev/null"

  # File: my_rules
  EVALSET MemoryUsage ( ResidentSetSize / 100 )
  EVALMACRO WantDisk = ( DiskUsage * 2 )
  SET RequestDisk ( $(WantDisk) / 1024 )
  RENAME NumCkpts NumCheckPoints
  DELETE /(.+)Err/

  # Command:
  condor_transform_ads -rules my_rules -in my_input

  # Output:
  DiskUsage = 2500000
  Err = "/dev/null"
  MemoryUsage = 5
  NumCheckPoints = 0
  RequestDisk = ( 5000000 / 1024 )
  ResidentSetSize = 500
\end{verbatim}

\end{ManPage}

