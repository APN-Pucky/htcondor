\begin{ManPage}{\label{man-condor-wait}\Condor{wait}}{1}
{Wait for jobs to finish}

\Synopsis
\SynProg{\Condor{wait}}
\ToolArgsBase

\SynProg{\Condor{wait}}
\oOpt{-debug}
\oOptArg{-wait}{seconds}
\oOptArg{-num}{number-of-jobs}
\Arg{log-file}
\oOpt{job ID}


\index{Condor commands!condor\_wait}
\index{condor\_wait command}

\Description

\Condor{wait} watches a user log file
(created with the \Opt{log} command within a submit description file)
and returns when one or more jobs from the
log have completed or aborted.

Because \Condor{wait} expects to find at least one job submitted
event in the log file, at least one job must have been successfully submitted
with \Condor{submit} before \Condor{wait} is executed.

\Condor{wait} will wait forever for jobs to finish, unless
a shorter wait time is specified.

\begin{Options}
    \ToolArgsBaseDesc
    \OptItem{\Opt{-debug}}{Show extra debugging information.}
    \OptItem{\OptArgnm{-wait}{seconds}}{Wait no more than the
    integer number of \Arg{seconds}.
    The default is unlimited time.} 
    \OptItem{\OptArgnm{-num}{number-of-jobs}}{Wait for the integer
    \Arg{number-of-jobs} jobs to end. The default is all jobs in the
    log file.} 
    \OptItem{log file}{The name of the log file to watch for
    information about the job.}
    \OptItem{job ID}{A specific job or set of jobs to watch.
    \index{job ID!use in \Condor{wait}}
    If the \Opt{job ID} is only the job ClassAd attribute \Attr{ClusterId},
    then \Condor{wait} waits for all jobs with the given 
    \Attr{ClusterId}.
    If the \Opt{job ID} is a pair of the 
    job ClassAd attributes, given by \Attr{ClusterId}.\Attr{ProcId},
    then \Condor{wait} waits for the specific job with this \Opt{job ID}.
    If this option is not specified, all jobs that exist in the log file
    when \Condor{wait} is invoked will be watched.}

\end{Options}

\GenRem 

\Condor{wait} is an inexpensive way to test or wait for the completion
of a job or a whole cluster, if you are trying to get a process
outside of Condor to synchronize with a job or set of jobs.

It can also be used to wait for the completion of a limited subset of
jobs, via the \Opt{-num} option.

\Examples

\begin{verbatim}
condor_wait logfile
\end{verbatim}
This command waits for all jobs that exist in \File{logfile} to complete.

\begin{verbatim}
condor_wait logfile 40
\end{verbatim}
This command waits for all jobs that exist in \File{logfile} 
with a job ClassAd attribute \Attr{ClusterId} of
40 to complete.

\begin{verbatim}
condor_wait -num 2 logfile
\end{verbatim}
This command waits for any two jobs that exist in \File{logfile} to
complete.

\begin{verbatim}
condor_wait logfile 40.1
\end{verbatim}
This command waits for job 40.1 that exists in \File{logfile} to
complete.

\begin{verbatim}
condor_wait -wait 3600 logfile 40.1
\end{verbatim}
This waits for job 40.1 to
complete by watching \File{logfile}, but it will not wait more than one
hour (3600 seconds).

\ExitStatus

\Condor{wait} exits with 0 if and only if the specified job or jobs
have completed or
aborted. \Condor{wait} returns 1 if unrecoverable errors occur, such
as a missing log file, if the job does not exist in the log file, or
the user-specified waiting time has expired.

\end{ManPage}
