\begin{ManPage}{\Condor{master}}{man-condor-master}{1}
{The master HTCondor Daemon}

\index{HTCondor commands!condor\_master}
\index{daemon!condor\_master@\Condor{master}}
\index{condor\_master daemon}
\index{HTCondor daemon!condor\_master@\Condor{master}}
\Synopsis \SynProg{\Condor{master}}

\Description 

This daemon is responsible for keeping all the
rest of the HTCondor daemons running on each machine in your pool.  It  
spawns the other daemons, and periodically checks to see if there are
new binaries installed for any of them.  If there are,
the \Condor{master} will restart the affected daemons.
In addition, if any daemon crashes, the
\Condor{master} will send e-mail to the HTCondor Administrator of your pool and 
restart the daemon.  The \Condor{master} also supports various
administrative commands that let you start, stop or reconfigure
daemons remotely.  The \Condor{master} will run on every machine in 
your HTCondor pool, regardless of what functions each machine are
performing.  Additionally, on Linux platforms, if you start the
\Condor{master} as root, it will tune (but never decrease) certain
kernel parameters important to HTCondor's performance.


The \Macro{DAEMON\_LIST} configuration macro is used by the
\Condor{master} to provide a per-machine list of daemons that
should be started and kept running.
For daemons that are specified in the \MacroNI{DC\_DAEMON\_LIST}
configuration macro,
the \Condor{master} daemon will spawn them automatically
appending a \Arg{-f} argument.
For those listed in \MacroNI{DAEMON\_LIST}, but not in \MacroNI{DC\_DAEMON\_LIST},
there will be no \Arg{-f} argument.

\begin{Options}
    \OptItem{\OptArgnm{-n}{name}}
            {Provides an alternate name for the \Condor{master}
            to override that given by the \Macro{MASTER\_NAME}
	    configuration variable.
            }
\end{Options}

\end{ManPage}
