\begin{ManPage}{\label{man-condor-annex}\Condor{annex}}{1}
{Add cloud resources to the pool.}
\index{HTCondor commands!condor\_annex}
\index{condor\_annex command}

\Synopsis

\SynProg{\Condor{annex}} \Opt{-help}

\SynProg{\Condor{annex}} \oOptArg{-aws-region}{<region>} \OptArg{-setup}{[/full/path/to/access/key/file [/full/path/to/secret/key/file]]}

\SynProg{\Condor{annex}} \oOpt{-aws-on-demand}
  \OptArg{-annex-name}{<name of the annex>}
  \OptArg{-count}{<integer number of instances>}
  \oOpt{-aws-on-demand-*}
  \oOpt{common options}

\SynProg{\Condor{annex}} \oOpt{-aws-spot-fleet}
  \OptArg{-annex-name}{<name of the annex>}
  \OptArg{-slots}{<integer weight>}
  \oOpt{-aws-spot-fleet-*}
  \oOpt{common options}

\SynProg{\Condor{annex}}
  \OptArg{-annex-name}{<name of the annex>}
  \OptArg{-duration}{hours}

\SynProg{\Condor{annex}}
  \oOptArg{-annex-name}{<name of the annex>}
  \Opt{-status}
  \oOpt{-classad}

\SynProg{\Condor{annex}} \Opt{-check-setup}

\SynProg{\Condor{annex}} \emph{<condor\_annex options>} \Opt{status} \emph{<condor\_status options>}

\Description

\Condor{annex} adds clouds resources to the pool.  (``The pool'' is determined
in the usual manner for HTCondor daemons and tools.)  This version supports
only Amazon Web Services (`AWS').  To add ``on-demand'' instances, use
the third form listed above; to add ``spot'' instances, use the fourth.  For an
explanation of terms, consult either the HTCondor manual
(chapter \ref{cloud-computing}) or the AWS documentation.

Using \Condor{annex} with AWS requires a one-time setup procedure
performed by invoking \Condor{annex} with the \Opt{-setup} flag
(the second form listed above).  You may check if this procedure has been
performed with the \Opt{-check-setup} flag (the seventh form listed above).

To reset the lease on an existing annex, invoke \Condor{annex} with
only the \Opt{-annex-name} option and \Opt{-duration} flag (the fifth form
listed above).

To determine which of the instances previously requested for a
particular annex are not currently in the pool, invoke \Condor{annex}
with the \Opt{-status} flag and the \Opt{-annex-name} option (the sixth
form listed above).  The output of this command is intended to be
human-readable; specifying the \Opt{-classad} flag will produce the
same information in ClassAd format.  If you omit \Opt{-annex-name},
information for all annexes will be returned.

Starting in 8.7.3, you may instead invoke \Condor{annex} with \Opt{status}
as a command argument (the eighth form listed above).  This will cause \Condor{annex} to use \Condor{status}
to present annex instance data.  Arguments and options on the command line
after \Opt{status} will be passed unmodified to \Condor{status}, but not
all arguments and options will behave as expected.  (See below.)
\Condor{annex} will construct an ad for each annex instance and pass that
information to \Condor{status}; \Condor{status} will (unless you specify
otherwise using its command line) query the collector for more information
about the instances.  Information from the collector will be presented as
usual; instances which did not have ads in the collector will be presented
last, in their own table.  These instances can not be presented in the
usual way because the annex instance ads generated by \Condor{annex} do not
(and can not) have the same information in them as ads generated by a
\Condor{startd} running in the instance.  See the \Condor{status}
documentation (section~\ref{man-condor-status}) for details about the ``merge'' mode of
\Condor{status} used by this command argument.  Note that both \Condor{annex}
and \Condor{status} have \Opt{-annex-name} options; if you're interested
in a particular annex, put this flag on the command line \emph{before}
the \Opt{status} command argument to avoid confusing results.

Common options are listed first, followed by options specific to AWS,
followed by options specific to AWS' on-demand instances, followed by
options specific to AWS' spot instances, followed by options intended
for use by experts.

\begin{Options}
%% COMMON OPTIONS
	\OptItem {\Opt{-help}}
		{Print a usage reminder.}

	\OptItem {\OptArg{-setup}{[/full/path/to/access/key/file /full/path/to/secret/key/file]}}
		{Do the first-time setup.}

	\OptItem {\OptArg{-duration}{hours}}
		{Set the maximum lease duration in decimal \Arg{hours}.  After this amount of time, all instances will terminated, regardless of their idleness.  Defaults to 50 minutes.}
	\OptItem {\OptArg{-idle}{hours}}
		{Set the maximum idle duration in decimal \Arg{hours}.  An instance idle for longer than this duration will terminate itself.  Defaults to 15 minutes.}
	\OptItem {\OptArg{-config-dir}{/full/path/to/directory}}
		{Copy the contents of \Arg{/full/path/to/directory} to each instance's configuration directory.}
	\OptItem {\OptArg{-owner}{owner[, owner]*}}
		{Configure the annex so that only \Arg{owner} may start jobs there.  By default, configure the annex so that only the user running \Condor{annex} may start jobs there.}
	\OptItem {\Opt{-no-owner}}
		{Configure the annex so that anyone in the pool may use the annex.}

%% AWS-SPECIFIC OPTIONS
	\OptItem {\OptArg{-aws-region}{region}}
		{Specify the region in which to create the annex.}
	\OptItem {\OptArg{-aws-user-data}{user-data}}
		{Set the instance user data to \Arg{user-data}.}
	\OptItem {\OptArg{-aws-user-data-file}{/full/path/to/file}}
		{Set the instance user data to the contents of the file \Arg{/full/path/to/file}.}
	\OptItem {\OptArg{-aws-default-user-data}{user-data}}
		{Set the instance user data to \Arg{user-data}, if it's not already set.  Only applies to spot fleet requests.}
	\OptItem {\OptArg{-aws-default-user-data-file}{/full/path/to/file}}
		{Set the instance user data to the contents of the file \Arg{/full/path/to/file}, if it's not already set.  Only applies to spot fleet requests.}

%% AWS ON-DEMAND OPTIONS
	\OptItem {\OptArg{-aws-on-demand-instance-type}{instance-type}}
		{This annex will requests instances of type \Arg{instance-type}.  The default for v8.7.1 is `m4.large'.}
	\OptItem {\OptArg{-aws-on-demand-ami-id}{ami-id}}
		{This annex will start instances of the AMI \Arg{ami-id}.  The default for v8.7.1 is `ami-35b13223', a GPU-compatible Amazon Linux image with HTCondor pre-installed.}
	\OptItem {\OptArg{-aws-on-demand-security-group-ids}{group-id[,group-id]}}
		{This annex will start instances with the listed security group IDs.  The default is the security group created by \Opt{-setup}.}
	\OptItem {\OptArg{-aws-on-demand-key-name}{key-name}}
		{This annex will start instances with the key pair named \Arg{key-name}.  The default is the key pair created by \Opt{-setup}.}

%% AWS SPOT FLEET OPTIONS
	\OptItem {\OptArg{-aws-spot-fleet-config-file}{/full/path/to/file}}
		{Use the JSON blob in \Arg{/full/path/to/file} for the spot fleet request.}

%% AWS EXPERTS' OPTIONS
	\OptItem {\OptArg{-aws-access-key-file}{/full/path/to/access-key-file}} {Experts only.}
	\OptItem {\OptArg{-aws-secret-key-file}{/full/path/to/secret-key-file}} {Experts only.}
	\OptItem {\OptArg{-aws-ec2-url}{https://ec2.<region>.amazonaws.com}} {Experts only.}
	\OptItem {\OptArg{-aws-events-url}{https://events.<region>.amazonaws.com}} {Experts only.}
	\OptItem {\OptArg{-aws-lambda-url}{https://lambda.<region>.amazonaws.com}} {Experts only.}
	\OptItem {\OptArg{-aws-s3-url}{https://s3.<region>.amazonaws.com}} {Experts only.}
	\OptItem {\OptArg{-aws-spot-fleet-lease-function-arn}{sfr-lease-function-arn}} {Developers only.}
	\OptItem {\OptArg{-aws-on-demand-lease-function-arn}{odi-lease-function-arn}} {Developers only.}
	\OptItem {\OptArg{-aws-on-demand-instance-profile-arn}{instance-profile-arn}} {Developers only.}
\end{Options}

\GenRem

Currently, only AWS is supported.  The AMI configured by setup runs
HTCondor v8.6.10 on Amazon Linux 2016.09, and the default instance type
is ``m4.large''.  The default AMI has the appropriate drivers for AWS'
GPU instance types.

\Examples

To start an on-demand annex named `MyFirstAnnex' with one core,
using the default AMI and instance type, run

\begin{verbatim}
  condor_annex -count 1 -annex-name MyFirstAnnex
\end{verbatim}

You will be asked to confirm that the defaults are what you want.

As of 2017-04-17, the following example will cost a minimum of \$90.

To start an on-demand annex with 100 GPUs that job owners `big' and `little'
may use (be sure to include yourself!), run

\begin{verbatim}
  condor_annex -count 100 -annex-name MySecondAnnex \
    -aws-on-demand-instance-type p2.xlarge -owner "big, little"
\end{verbatim}

\ExitStatus

\Condor{annex} will exit with a status value of 0 (zero) on success.

\end{ManPage}
