\begin{ManPage}{\label{man-condor-convert-history}\Condor{convert\_history}}{1}
{Convert the history file to the new format}
\Synopsis 

\SynProg{\Condor{convert\_history}}
\oOpt{-help}
\Optnm{files}

\index{Condor commands!condor\_convert\_history}
\index{condor\_convert\_history command}

\Description

Beginning with Condor 6.7.19, the Condor history file was changed to a
new format to allow fast searches backwards through the file. Not all
queries can take advantage of the speed increase, but the ones that
can are significantly faster. 

New job ClassAds saved in the history file will automatically be saved
in the new format. In order to search these new ClassAds, no changes
are necessary. However, to be able to search the entire history, the
history file must be converted, and \Condor{convert\_history} will do
this for you. 

To run this command, simply give it a list of history files to
convert. The history file is normally in the Condor spool directory
and is named ``history''. Note that the history file is rotated, so
there may be multiple history files, and all of them should be
converted. On Unix variants, the easiest way to do this is:

\begin{verbatim}
cd `condor_config_val SPOOL`
condor_convert_history history*
\end{verbatim}

Note that this will back up the original history files in case of a
problem. It is best not to keep these in the spool directory, but to
copy them elsewhere. If you keep them in the spool directory,
\Condor{history} will find them and you will appear to have duplicate
jobs. 

\ExitStatus

\Condor{convert\_history} will exit with a status value of 0 (zero)
upon success, and it will exit with the value 1 (one) upon failure.

\end{ManPage}
