\begin{ManPage}{\label{man-condor-convert-history}\Condor{convert\_history}}{1}
{Convert the history file to the new format}
\Synopsis 

\SynProg{\Condor{convert\_history}}
\oOpt{-help}

\SynProg{\Condor{convert\_history}}
\Arg{history-file1}
\oArg{history-file2\ldots}

\index{Condor commands!condor\_convert\_history}
\index{condor\_convert\_history command}

\Description

As of Condor version 6.7.19,
the Condor history file has a
new format to allow fast searches backwards through the file.
Not all queries can take advantage of the speed increase,
but the ones that can are significantly faster. 

Entries placed in the history file after upgrade
to Condor 6.7.19 will automatically be saved
in the new format.
The new format adds information to the string which
distinguishes and separates job entries.
In order to search within this new format,
no changes are necessary. 
However, to be able to search the entire history,
the history file must be converted to the updated format.
\Condor{convert\_history} does this.

Turn the \Condor{schedd} daemon off while converting 
history files.
Turn it back on after conversion is completed.

Arguments to \Condor{convert\_history} are the
history files to convert.
The history file is normally in the Condor spool directory;
it is named \File{history}.
Since the history file is rotated,
there may be multiple history files, and all of them should be
converted. On Unix platform variants, the easiest way to do this is:

\begin{verbatim}
cd `condor_config_val SPOOL`
condor_convert_history history*
\end{verbatim}

\Condor{convert\_history} makes a
back up of each original history files in case of a problem.
The names of these back up files are listed;
names are formed by appending the suffix \verb@.oldver@
to the original file name.
Move these back up files to a directory other than
the spool directory.
If kept in the spool directory,
\Condor{history} will find the back ups,
and will appear to have duplicate jobs. 

\ExitStatus

\Condor{convert\_history} will exit with a status value of 0 (zero)
upon success, and it will exit with the value 1 (one) upon failure.

\end{ManPage}
