\begin{ManPage}{\label{man-condor-gather-info}\Condor{gather\_info}}{1}
{Gather information about a Condor installation and/or a queued job}
\Synopsis

\SynProg{\Condor{gather\_info}}
\oOptnm{OPTION}...

% \index{Condor commands!condor\_gather\_info}
% \index{condor\_gather\_info command}

\Description
\Condor{gather\_info} will collect information about the machine it is run
upon, the condor installation local to the machine, and optionally a specified
jobid. If the jobid is not specified, then only the machine information is
written to a file called \File{condor-profile.txt} in the current working
directory. If the jobid is specified a tarball containing information about
the machine, the Condor installation, and as much job information as can be 
ascertained will be written into the current working directory.

The information gathered by this tool is most often used as a debugging aid
for the developers of Condor.

\begin{Options}
    \OptItem{\OptArg{--jobid}{CLUSTER.PROC}}{Datamine information about this job id from the local Condor installation and condor\_schedd.}
    \OptItem{\OptArg{--scratch}{path}}{A path to temporary space needed when building the output tarball. Defaults to /tmp/cbd-<PID>.}
\end{Options}

\GenRem

The information gathered by this tool is:

\begin{enumerate}
	\item Identity
	\begin{itemize}
          \item User name who generated the report
          \item Script location and Machine Name
          \item Date of Report Creation
          \item \Shell{uname -a}
          \item Contents of \File{/etc/issue}
          \item Contents of \File{/etc/redhat-release}
          \item Contents of \File{/etc/debian\_version}
          \item \Shell{ps -auxww --forest}
          \item \Shell{df -h}
          \item \Shell{iptables -L}
          \item \Shell{ls `condor\_config\_val LOG`}
          \item \Shell{ldd `condor\_config\_val SBIN`/condor\_schedd}
          \item Contents of \File{/etc/hosts}
          \item Contents of \File{/etc/nsswitch.conf}
          \item \Shell{ulimit -a}
          \item Network Interface Configuration (\Shell{ifconfig})
          \item Condor Version
          \item Location of Condor Config Files
          \item Condor Config Variables
		  \begin{itemize}
                \item All variables and values
                \item Definition locations for each config variable 
		  \end{itemize}
	\end{itemize}
	\item Job Information
	\begin{itemize}
    	\item \Shell{condor\_q jobid}
    	\item \Shell{condor\_q -l jobid}
    	\item \Shell{condor\_q -better-analyze jobid}
    	\item Job User log, if exists...
		\begin{itemize}
          	\item Only events pertaining to jobid 
		\end{itemize}
	\end{itemize}
\end{enumerate}

\Files

\begin{description}
	\item{\File{condor-profile.txt}} Just the identity portion of the information gathered when \Condor{gather\_info} is run without arguments.
	\item{\File{cgi-<username>-jid<cluster>.<proc>-<year>-<month>-<day>-<hour>\_<minute>\_<second>-<TZ>.tar.gz}} The output file which contains all of the information produced by this tool.
\end{description}

\ExitStatus

\Condor{gather\_info} will exit with a status value of 0 (zero) upon success,
and it will exit with the value 1 (one) upon failure.

\end{ManPage}
