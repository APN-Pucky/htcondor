\begin{ManPage}{\label{man-condor-run}\Condor{run}}{1}
{Submit a shell command-line as an HTCondor job}
\index{HTCondor commands!condor\_run}
\index{condor\_run command}

\Synopsis \SynProg{\Condor{run}}
\oOptArg{-u}{universe}
\Arg{"shell command"}

\Description
\Condor{run} bundles a shell command line into an HTCondor
job and submits the job.
The \Condor{run} command waits for the HTCondor job to complete,
writes the job's output to the terminal,
and exits with the exit status of the HTCondor job.
No output appears until the job completes.

Enclose the shell command line in double quote marks,
so it may be passed to
\Condor{run} without modification.
\Condor{run} will not read input from the terminal while the job
executes.  If the shell command line requires input,
redirect the input from a file, as illustrated by the example
\begin{verbatim}
% condor_run "myprog < input.data"
\end{verbatim}

\Condor{run} jobs rely on a shared file system for
access to any necessary input files.
The current working directory of the job must be
accessible to the machine within the HTCondor pool where the job runs.

Specialized environment variables may be used to specify
requirements for the machine where the job may run.

\begin{description}
\item[CONDOR\_ARCH] Specifies the architecture of the required
  platform. Values will be the same as the \Attr{Arch}
  machine ClassAd attribute.
\item[CONDOR\_OPSYS] Specifies the operating system of the required
  platform. Values will be the same as the \Attr{OpSys}
  machine ClassAd attribute.
\item[CONDOR\_REQUIREMENTS] Specifies any additional requirements for
  the HTCondor job.  It is recommended that the value defined for
  \Env{CONDOR\_REQUIREMENTS} be enclosed in parenthesis.
\end{description}

When one or more of these environment variables is specified, the job is
submitted with:

\begin{verbatim}
Requirements = $CONDOR_REQUIREMENTS && Arch == $CONDOR_ARCH && \
   OpSys == $CONDOR_OPSYS
\end{verbatim}

Without these environment variables, 
the job receives the default requirements expression, which
requests a machine of the same platform as
the machine on which \Condor{run} is executed.

All environment variables set when \Condor{run} is executed will be
included in the environment of the HTCondor job.

\Condor{run} removes the HTCondor job from the queue and
deletes its temporary files, if \Condor{run} is killed before the 
HTCondor job completes.

\begin{Options}

\OptItem{\OptArg{-u}{universe}}
  {Submit the job under the specified universe. The default is vanilla.
  While any universe may be specified, only the vanilla, standard,
  scheduler, and local universes result in a submit description
  file that may work properly.}

\end{Options}

\Examples

\Condor{run} may be used to compile an executable on
a different platform.
As an example, first set the environment variables for
the required platform:

\begin{verbatim}
% setenv CONDOR_ARCH "SUN4u"
% setenv CONDOR_OPSYS "SOLARIS28"
\end{verbatim}

Then, use \Condor{run} to submit the compilation as in the
following three examples.
\begin{verbatim}
% condor_run "f77 -O -o myprog myprog.f"
\end{verbatim}
or
\begin{verbatim}
% condor_run "make"
\end{verbatim}
or
\begin{verbatim}
% condor_run "condor_compile cc -o myprog.condor myprog.c"
\end{verbatim}


\Files

\Condor{run} creates the following temporary files in the user's
working directory.
The placeholder <pid> is replaced by the process id
of \Condor{run}.
\begin{description}
\item[\File{.condor\_run.<pid>}] A shell script containing the shell
  command line.
\item[\File{.condor\_submit.<pid>}] The submit description file for
  the job.
\item[\File{.condor\_log.<pid>}] The HTCondor job's log file; it is
  monitored by \Condor{run}, to determine when the job exits.
\item[\File{.condor\_out.<pid>}] The output of the HTCondor
  job before it is output to the terminal.
\item[\File{.condor\_error.<pid>}] Any error messages for the HTCondor
  job  before they are output to the terminal.
\end{description}
\Condor{run} removes these files when the job completes.  However, if
\Condor{run} fails, it is possible that these files will remain in the
user's working directory, and the HTCondor job may remain in the queue.

\GenRem

\Condor{run} is intended for submitting simple shell command lines to
HTCondor.  It does not provide the full functionality of
\Condor{submit}.  Therefore, some \Condor{submit} errors and
system failures may not be handled correctly. 

All processes specified within the single shell command line
will be executed on the single machine
matched with the job.  HTCondor will not distribute multiple
processes of a command line pipe across multiple machines.

\Condor{run} will use the shell specified in the \Macro{SHELL} environment
variable, if one exists.  Otherwise, it will use \Prog{/bin/sh} to execute
the shell command-line.

By default, \Condor{run} expects Perl to be installed in
\File{/usr/bin/perl}.  If Perl is installed in another path, 
ask the Condor administrator to edit the path in the \Condor{run}
script, or explicitly call Perl from the command line:

\begin{verbatim}
% perl path-to-condor/bin/condor_run "shell-cmd"
\end{verbatim}


\ExitStatus

\Condor{run} exits with a status value of 0 (zero) upon complete success.
The exit status of \Condor{run} will be non-zero upon failure.
The exit status in the case of a single error due to a system call
will be the error number (\Code{errno}) of the failed call.

\end{ManPage}
