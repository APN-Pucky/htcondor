\begin{ManPage}{\label{man-condor-history}\Condor{history}}{1}
{View log of condor jobs completed to date}
\Synopsis \SynProg{\Condor{history}}
\oOpt{-help} \oOpt{-l} \oOptArg{-f}{filename}
\oArg{cluster \Bar cluster.process \Bar owner}

\Description
\Condor{history} displays a summary of all condor jobs listed in the
specified history files.  If no history files are specified (with the \Opt{-f} option), the local
history file as specified in Condor's configuration file
(~condor/spool/history by default) is read.  The default listing
summarizes each job on a single line, and  contains the following items:

\begin{description}
\item[ID] The cluster/process id of the condor job. 
\item[OWNER] The owner of the job. 
\item[SUBMITTED] The month, day, hour, and minute the job was submitted to the queue. 
\item[CPU\_USAGE] Remote CPU time accumulated by the job to date in days, hours, minutes, and seconds.
\item[ST] Completion status of the job (C = completed and X = removed).
\item[COMPLETED] The time the job was completed.
\item[PRI] User specified priority of the job, ranges from -20 to +20, with higher numbers corresponding to greater priority. 
\item[SIZE] The virtual image size of the executable in megabytes. 
\item[CMD] The name of the executable. 
\end{description}

If a job ID (in the form of cluster\_id or cluster\_id.proc\_id) or an owner is provided, 
output will be restricted only to jobs with the specified IDs and/or submitted by the specified owner.

\begin{Options}
    \OptItem{\Opt{-help}}{Get a brief description of the supported options}
    \OptItem{\OptArg{-f}{filename}}{Use the specified file instead of the default history file}
    \OptItem{\Opt{-l}}{Display job ads in long format}
\end{Options}

\end{ManPage}
