\begin{ManPage}{\label{man-condor-history}\Condor{history}}{1}
{View log of Condor jobs completed to date}
\Synopsis
\SynProg{\Condor{history}}
\oOpt{-help} 

\SynProg{\Condor{history}}
\oOpt{-backwards}
\oOpt{-forwards}
\oOptArg{-constraint}{expr}
\oOptArg{-f}{filename}
\oOptArg{-format}{formatString AttributeName}
\oOpt{-l \Bar -long \Bar -xml} 
\oOptArg{-match}{number}
\oOpt{cluster \Bar cluster.process \Bar owner}

\Description
\Condor{history} displays a summary of all Condor jobs listed in the
specified history files.
If no history files are specified with the \Opt{-f} option, 
the local history file as specified in Condor's configuration file
(\File{\MacroUNI{SPOOL}/history} by default) is read.  
The default listing summarizes in reverse chronological order
each job on a single line, and  contains the following items:

\index{Condor commands!condor\_history}
\index{condor\_history command}

\begin{description}
\item[ID] The cluster/process id of the job. 
\item[OWNER] The owner of the job. 
\item[SUBMITTED] The month, day, hour, and minute the job was submitted to the queue. 
\item[RUN\_TIME] Remote wall clock time accumulated by the job to date in days, hours, minutes, and seconds.  See the definition of
\AdAttr{RemoteWallClockTime} on page~\pageref{RemoteWallClockTime}.
\item[ST] Completion status of the job (C = completed and X = removed).
\item[COMPLETED] The time the job was completed.
\item[CMD] The name of the executable. 
\end{description}

If a job ID (in the form of \Argnm{cluster\_id} or \Argnm{cluster\_id.proc\_id}) or an
\Argnm{owner} is provided, output will be restricted to jobs with the
specified IDs and/or submitted by the specified owner.  
The \Argnm{-constraint} option can be used to display jobs that satisfy a
specified boolean expression.

The history file is kept in chronological order,
implying that new entries are appended at the end of the
file.
As of Condor version 6.7.19,
the format of the history file is altered to enable faster
reading of the history file backwards (most recent job first).
History files written with earlier versions of Condor,
as well as those that have entries of both
the older and newer format
need to be converted to the new format.
See the \Condor{convert\_history} manual page 
on page~\pageref{man-condor-convert-history}
for details on converting history files to the new format.

\begin{Options}
  \OptItem{\Opt{-help}}{Display usage information and exit.}
  \OptItem{\Opt{-backwards}}{List jobs in reverse chronological
    order. The job most recently added to the history file is first.
    This is the default ordering.}
  \OptItem{\Opt{-forwards}}{List jobs in chronological
    order. The job most recently added to the history file is last.
    At least 4 characters must be given to distinguish this option
    from the \Opt{-f} and \Opt{-format} options. }
  \OptItem{\OptArg{-constraint}{expr}}{Display jobs that satisfy the
    expression.}
  \OptItem{\OptArg{-f}{filename}}{Use the specified file instead of the
    default history file.  }
  \OptItem{\OptArg{-format}{formatString}{AttributeName}}{Display jobs
    with a custom format. See the \Condor{q} man page \Opt{-format} option
    for details.}
  \OptItem{\Opt{-l} or \Opt{-long}}{Display job ClassAds in long format.}
  \OptItem{\OptArg{-match}{number}}{Limit the number of jobs displayed
    to \Arg{number}.}
  \OptItem{\Opt{-xml}}{Display job ClassAds in XML format.
      The XML format is fully defined at
      \URL{http://www.cs.wisc.edu/condor/classad/refman/}.}
\end{Options}

\ExitStatus

\Condor{history} will exit with a status value of 0 (zero) upon success,
and it will exit with the value 1 (one) upon failure.

\end{ManPage}
