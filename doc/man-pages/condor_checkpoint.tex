\begin{ManPage}{\label{man-condor-checkpoint}\Condor{checkpoint}}{1}
{checkpoint jobs running on the specified hosts}
\Synopsis \SynProg{\Condor{checkpoint}}
\oOpt{-help}
\oOpt{-version}
\oOpt{hostname ...}

\Description
\Condor{checkpoint} causes the startd's on the specified hosts to perform a 
checkpoint on any running jobs. The jobs continue to run once
they are done checkpointing. If no host is specified, only the current host is sent
the checkpoint command.

A periodic checkpoint means that the job will checkpoint itself, but then it will immediately continue running
after the checkpoint has completed. \Condor{vacate}, on the other hand, will result in the job exiting (vacating) after it
checkpoints. 

If the job being checkpointed is running in the Standard Universe, the job is checkpointed and then just continues running
on the same machine. If the job is running in the Vanilla Universe, or there is currently no Condor job
running on that host, then \Condor{checkpoint} has no effect. 

Normally there is no need for the user or administrator to explicitly run \Condor{checkpoint}. Checkpointing a running condor
job is normally handled automatically by Condor by following the policies stated in Condor's configuration files. 

\begin{Options}
    \OptItem{\Opt{-help}}{Display usage information}
    \OptItem{\Opt{-version}}{Display version information}
\end{Options}

\end{ManPage}
