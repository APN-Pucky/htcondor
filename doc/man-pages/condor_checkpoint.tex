\begin{ManPage}{\label{man-condor-checkpoint}\Condor{checkpoint}}{1}
{send a checkpoint command to jobs running on specified hosts}
\Synopsis \SynProg{\Condor{checkpoint}}
\ToolArgsBase \ToolArgsLocate

\index{Condor commands!condor\_checkpoint}
\index{condor\_checkpoint command}

\Description
\Condor{checkpoint} sends a checkpoint command to a specified
machine or set of machines within a single pool.
This causes the startd daemon on each of the specified hosts to take a 
checkpoint of any running jobs. The job is temporarily
stopped, a checkpoint is taken, and then the job continues.
If no host is specified, then the checkpoint command
is sent to the host that issued the
\Condor{checkpoint} command.

The command sent is a
periodic checkpoint.
The job will take a checkpoint, but then the job will immediately
continue running
after the checkpoint is completed.
\Condor{vacate}, on the other hand, will result in the job exiting
(vacating) after it produces a checkpoint. 

If the job being checkpointed is running under the standard universe,
the job produces a checkpoint and then continues running
on the same machine.
If the job is running under another universe,
or if there is currently no Condor job
running on that host, then \Condor{checkpoint} has no effect. 

There is no need for the user or administrator to explicitly
run \Condor{checkpoint} under normal circumstances.
Taking checkpoints of running Condor jobs is
handled automatically by Condor by following the policies
stated in Condor's configuration files. 

\begin{Options}
	\ToolArgsBaseDesc
	\ToolArgsLocateDesc
\end{Options}

\ExitStatus

\Condor{checkpoint} will exit with a status value of 0 (zero) upon success,
and it will exit with the value 1 (one) upon failure.

\end{ManPage}
