\begin{ManPage}{\label{man-condor-vacate}\Condor{vacate}}{1}
{Vacate jobs that are running on the specified hosts}
\Synopsis \SynProg{\Condor{vacate}}
\oOpt{-help}
\oOpt{-version}
\oOpt{-fast}
\oOpt{hostname ...}

\index{Condor commands!condor\_vacate}
\index{condor\_vacate command}

\Description
\Condor{vacate} causes the condor startd to checkpoint any running jobs
and make them vacate the machine. The job remains in
the submitting machine's job queue, however. 

If the job on the specified host is running in the Standard Universe, then the
job is first checkpointed and then killed (and will then restart somewhere
else where it left off). If the job on the specified host is running in the
Vanilla Universe, then the job is not checkpointed but is simply killed (and
will then restart somewhere else from the beginning). If there is currently
no Condor job running on that host, then \Condor{vacate} has no effect. 
 Normally there is no need for the user or administrator to explicitly run
\Condor{vacate}. Vacating a running condor job off of a machine is handled
automatically by Condor by following the policies stated in Condor's
configuration files.   
\begin{Options}
    \OptItem{\Opt{-help}}{Display usage information}
    \OptItem{\Opt{-version}}{Display version information}
    \OptItem{\Opt{-fast}}{Hard-kill jobs instead of checkpointing them}
\end{Options}

\end{ManPage}
