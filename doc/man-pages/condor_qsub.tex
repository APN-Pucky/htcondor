\begin{ManPage}{\label{man-condor-qsub}\Condor{qsub}}{1}
{Queue jobs that use PBS/SGE-style submission}
\Synopsis \SynProg{\Condor{qsub}}
[\verb@--@\textbf{version}]

\SynProg{\Condor{qsub}}
\oOpt{Specific options}
\oOpt{Directory options}
\oOpt{Environmental options}
\oOpt{File options}
\oOpt{Notification options}
\oOpt{Resource options}
\oOpt{Status options}
\oOpt{Submission options}
\Arg{commandfile}

\Description

\Condor{qsub} submits an HTCondor job.
This job is specified in a PBS/Torque style or an SGE style.
\Condor{qsub} permits the 
submission of dependent jobs without the need to specify the full
dependency graph at submission time.
Doing things this way is neither as efficient
as HTCondor's DAGMan, nor as functional as SGE's \Prog{qsub} or \Prog{qalter}.
\Condor{qsub} serves as a minimal translator to be able to use 
software originally written to interact 
with PBS, Torque, and SGE in an HTCondor pool. 

\Condor{qsub} attempts to behave like \Prog{qsub}. 
Less than half of the \Prog{qsub}
functionality is implemented. 
Option descriptions 
describe the differences between the behavior of \Prog{qsub} and 
\Condor{qsub}.
\Prog{qsub} options not listed here are not supported.
Some concepts present in PBS and SGE do not apply to HTCondor,
and so these options are not implemented.

For a full listing of \Prog{qsub} options, please see
\begin{description}
\item[POSIX]: \URL{http://pubs.opengroup.org/onlinepubs/9699919799/utilities/qsub.html}
\item[SGE]: \URL{http://gridscheduler.sourceforge.net/htmlman/htmlman1/qsub.html}
\item[PBS/Torque]: \URL{http://docs.adaptivecomputing.com/torque/4-1-3/Content/topics/commands/qsub.htm}
\end{description}

\Condor{qsub} accepts
either command line options or the single file, \Arg{commandfile},
that contains all of the commands. 

\Condor{qsub} does the opposite of job submission within the 
\SubmitCmd{grid} universe 
\SubmitCmd{batch} grid type (section~\ref{sec:batch}),
which takes HTCondor jobs submitted with HTCondor
syntax and submits them to PBS, SGE, or LSF.

\begin{Options}
\OptItem{\OptArg{-a}{date\_time}}{(Submission option)
  Specify a deferred execution date and time.
  The PBS/Torque syntax of \Arg{date\_time} 
  is a string in the form \Arg{[[[[CC]YY]MM]DD]hhmm[.SS]}.  
  The portions of this string which are optional are
  \Arg{CC}, \Arg{YY}, \Arg{MM}, \Arg{DD}, and \Arg{SS}.
  For SGE, \Arg{MM} and \Arg{DD} are \emph{not} optional. 
  For PBS, \Arg{MM} and \Arg{DD} are optional. 
  \Condor{qsub} follows the PBS style. }
\OptItem{\OptArg{-A}{account\_string}}{(Status option)
  Uses group accounting (see section~\ref{sec:group-accounting})
  where the string \Arg{account\_string} 
  is the accounting group associated with this job. 
  Unlike SGE, there is no default group of \texttt{"sge"}. }
\OptItem{\OptArg{-b}{y|n}}{(Submission option)
  Using the SGE definition of its \Arg{-b} option,
  a value of \Arg{y} causes \Condor{qsub} to \emph{not}
  parse the file for additional \Condor{qsub} commands.
  The default value is \Arg{n}.
  If the command line argument \OptArg{-f}{filename} is also specified,
  it negates a value of \Arg{y}.  }
\OptItem{\OptArg{-c}{checkpoint\_option}}{(Submission option)  
  For standard universe jobs only, 
  controls the how HTCondor produces checkpoints.  \Arg{checkpoint\_options}
  may be one of
  \begin{description}
  \item[n or N] Do not produce checkpoints.
  \item[s or S] Do not produce periodic checkpoints.  A job will only produce a checkpoint when the job is evicted.
  \end{description}
  More options may be implemented in the future. }
\OptItem{\Opt{---condor-keep-files}}{(Specific option)
  Directs HTCondor to \emph{not} remove temporary files
  generated by \Condor{qsub},
  such as HTCondor submit files and sentinel jobs.
  These temporary files may be important for debugging. }
\OptItem{\Opt{-cwd}}{(Directory option)
  Specifies the initial directory in which the job will run to be the current 
  directory from which the job was submitted.
  This sets \SubmitCmd{initialdir} for \Condor{submit}.  }
\OptItem{\OptArg{-d}{path} or \OptArg{-wd}{path}}{(Directory option)
  Specifies the initial directory in which the job will run to be \Arg{path}.
  This sets \SubmitCmd{initialdir} for \Condor{submit}.  }
\OptItem{\OptArg{-e}{filename}}{(File option)
  Specifies the \Condor{submit} command \SubmitCmd{error}, the file 
  where \File{stderr} is written. 
  If not specified, set to the default name of 
  \Expr{ <commandfile>.e<ClusterId>},
  where \Expr{<commandfile>} is the \Condor{qsub} argument,
  and \Expr{ <ClusterId>} is the job attribute \Attr{ClusterId} assigned
  for the job.  }
\OptItem{\OptArg{---f}{qsub\_file}}{(Specific option)
  Parse \Arg{qsub\_file} to search for and set additional \Condor{submit}
  commands.
  Within the file, 
  commands will appear as \Expr{\#PBS} or \Expr{\#SGE}. 
  \Condor{qsub} will parse the batch file listed as \Arg{qsub\_file}. }
  %This command may be replaced with \Opt{-@} in the future to emulate SGE's 
  %implementation of dealing with these kinds of files.}
\OptItem{\Opt{-h}}{(Status option)
  Placed submitted job directly into the hold state.  }
\OptItem{\Opt{---help}}{(Specific option)
  Print usage information and exit.  }
\OptItem{\OptArg{-hold\_jid}{<jid>}}{(Status option)
  If given, the job will be submitted in the hold state. Along with the actual 
  job, a `sentinel' job will be submitted to HTCondor's local universe. This 
  sentinel then watches the specified job and releases the submitted job whenever
  the job has completed. The sentinel observes HTCondor's queue and history
  to detect a job exiting with code 100 and not start the dependent job in this 
  case. If a cluster id of an array job is given, the dependent job will only be 
  released after all individual jobs of a cluster have completed.}
\OptItem{\OptArg{-i}{ [hostname:]filename }}{(File option)
  Specifies the \Condor{submit} command \SubmitCmd{input}, the file 
  from which \File{stdin} is read.  }
\OptItem{\OptArg{-j}{characters}}{(File option)
  Acceptable characters for this option are \Expr{e}, \Expr{o}, and \Expr{n}.
  The only sequence that is relevant is \Expr{eo}; it specifies that both
  standard output and standard error are to be sent to the same file.
  The file will be the one specified by the \Opt{-o} option,
  if both the \Opt{-o} and \Opt{-e} options exist.  
  The file will be the one specified by the \Opt{-e} option, 
  if only the \Opt{-e} option is provided.
  If neither the \Opt{-o} nor the \Opt{-e} options are provided,
  the file will be the default used for the \Opt{-o} option.  }
\OptItem{\OptArg{-l}{resource\_spec}}{(Resource option)
  Specifies requirements for the job, such as the amount of RAM and
  the number of CPUs.
  Only PBS-style resource requests are supported.
  \Arg{resource\_spec} is a comma separated list of key/value pairs.
  Each pair is of the form
  \Expr{resource\_name=value}.
\Expr{resource\_name} and \Expr{value} may be

\begin{tabular*}{5in}{|c|p{1in}|p{3in}|} \hline
\Expr{resource\_name} & \Expr{value} & Description \\ \hline
arch  & string & Sets \Attr{Arch} machine attribute. Enclose in double quotes. \\ \hline
file  & size & Disk space requested. \\ \hline
host  & string & Host machine on which the job must run. \\ \hline
mem   & size & Amount of memory requested.\\ \hline
nodes & 
%\begin{verbatim}
%\ShortExpr{\{<node\_count> | <hostname>\} 
%[:ppn=<ppn>][:gpus=<gpu>]\\
%[:<property>[:<property>]\ldots] \\
%[+ \ldots]}
%\end{verbatim}
\Shell{\{<node\_count> | <hostname>\} [:ppn=<ppn>] [:gpus=<gpu>] [:<property> [:<property>] \ldots] [+ \ldots]}
%\{<node_count> | <hostname>\} [:ppn=<ppn>][:gpus=<gpu>][:<property>[:<property>]\ldots] [+ \ldots]}
& Number and/or properties of nodes to be used. 
%We are working on having arbitrary properties be supported. 
For examples, please see
\parbox{2in}{ 
\URL{http://docs.adaptivecomputing.com/torque/4-1-3/Content/topics/2-jobs/requestingRes.htm\#qsub}
}
\\ \hline
opsys & string & Sets \Attr{OpSys} machine attribute. Enclose in double quotes.
\\ \hline
procs & integer  & Number of CPUs requested. \\ \hline  
\end{tabular*}
A size value is an integer specified in bytes, 
following the PBS/Torque default.
Append \Expr{Kb}, \Expr{Mb}, \Expr{Gb}, or \Expr{Tb} 
to specify the value in quantities greater than bytes.
%We do not currently support the use of words (w, kw, mw, gw, and tw).
}
\OptItem{\OptArg{-m}{a|e|n}}{(Notification option)
Identify when HTCondor sends notification e-mail. 
If \Arg{a}, send e-mail when the job terminates abnormally.
If \Arg{e}, send e-mail when the job terminates.
If \Arg{n}, never send e-mail.
%There are two additional options that may become supported in the future:
%\begin{description}
%\item[b] UNSUPPORTED. Mail is sent when the job commences execution.
%\item[s] UNSUPPORTED. Mail is sent when the job is suspended.
%\end{description}
%We hope to implement these options.
}
\OptItem{\OptArg{-M}{e-mail\_address}}{(Notification option)
Sets the destination address for HTCondor e-mail.  }
\OptItem{\OptArg{-o}{filename}}{(File option)
  Specifies the \Condor{submit} command \SubmitCmd{output}, the file 
  where \File{stdout} is written. 
  If not specified, set to the default name of 
  \Expr{ <commandfile>.o<ClusterId>},
  where \Expr{<commandfile>} is the \Condor{qsub} argument,
  and \Expr{ <ClusterId>} is the job attribute \Attr{ClusterId} assigned
  for the job.  }
\OptItem{\OptArg{-p}{integer}}{(Status option)
Sets the priority, with 0 being the default. Jobs with higher numerical priority will
run before jobs with lower numerical priority.
See section~\ref{man-condor-submit-priority} on 
page~\pageref{man-condor-submit-priority}
for more information.}
\OptItem{\Opt{---print}}{(Specific option) 
Send to \File{stdout} the contents of the HTCondor submit description
file that \Condor{qsub} generates.}
\OptItem{\OptArg{-r}{y|n}}{(Status option)
The default value of \Arg{y} implements the default HTCondor policy 
of assuming that jobs that do not complete are placed back in the queue 
to be run again. 
When \Arg{n},
job submission is restricted to only running the job if the job
ClassAd attribute \Attr{NumJobStarts} is currently 0. 
This identifies the job as not re-runnable, limiting it to start once.  }
\OptItem{\OptArg{-S}{shell}}{(Submission option)
Specifies the path and executable name of a shell.  
Alters the HTCondor submit description file produced, such that
the executable becomes a wrapper script.
Within the submit description file will be
\Expr{executable = <shell>} and \Expr{arguments = <commandfile>}.
}
% html is bad; nothing tried so far corrects it
%\OptItem{\OptArg{-t}{start[-stop[:step]] }} {(Submission option)
%\OptItem{\OptArg{-t}{ start[-stop[:step]] }} {(Submission option)
%\OptItem{\OptArg{-t}{ start[-stop[:step]] }}{(Submission option)
\OptItem{\OptArg{-t}{start [-stop:step] }}{(Submission option)
Queues a set of nearly identical jobs.
The SGE-style syntax is supported. \Arg{start}, 
\Arg{stop}, and \Arg{step} are all integers.
\Arg{start} is the starting index of the jobs,
\Arg{stop} is the ending index (inclusive) of the jobs,
and \Arg{step} is the step size through the indices.
Note that using more than one processor or 
node in a job will not work with this option. }
\OptItem{\Opt{---test}}{(Specific option)
With the intention of testing a potential job submission,
parse files and commands to generate error output.
Produces, but then removes the HTCondor submit description file.
Never submits the job, even if no errors are encountered. }
\OptItem{\OptArg{-v}{variable list}}{(Environmental option)
\Arg{variable list} is a comma separated string with the format of
\Env{variable[=value][,variable[=value],...]}. Since \Env{[=value]} can be optional
and HTCondor does not accept a variable without a value, \Condor{qsub} appends 
\Env{=True} to the variable listed. Note that this option does not handle white space
well as they must be contained by quotes. Also, all single and double quotes need to
be escaped by backslashes.
}
\OptItem{\Opt{-V}}{(Environmental option)
Set \Expr{getenv = True} in the submit description file.
See section~\ref{man-condor-submit-getenv} for a description of 
\SubmitCmd{getenv}.  }
\OptItem{\OptArg{-W}{attr\_name=attr\_value[,attr\_name=attr\_value\ldots]}}{
(File option)
PBS/Torque supports a number of attributes. However, \Condor{qsub} only supports
the names \Arg{stagein} and \Arg{stageout} for \Arg{attr\_name}.
The format of \Arg{attr\_value} for \Arg{stagein} and \Arg{stageout} is
\Env{local\_file@hostname:remote\_file[,\ldots]} and we strip it to 
\Env{remote\_file[,\ldots]}. HTCondor's file transfer mechanism is then used if needed.}
\OptItem{\Opt{---version}}{(Specific option)
  Print version information for the \Condor{qsub} program and exit.
  Note that \Condor{qsub} has its own version numbers which are
  separate from those of HTCondor.  } 
\end{Options}

\ExitStatus

\Condor{qsub} will exit with a status value of 0 (zero) upon success,
and it will exit with the value 1 (one) upon failure to submit a job. 

\end{ManPage}


















