\begin{ManPage}{\label{man-uniq-pid-midwife}\Prog{uniq\_pid\_midwife}}{1}
{create an artifact of the creation of a process}

\Synopsis \SynProg{\Prog{uniq\_pid\_midwife}}
\oOpt{-\,-noblock}
\oOptArg{-\,-file}{filename}
\oOptArg{-\,-precision}{seconds}
\Arg{program}
\oArg{programargs}

\index{Deployment commands!uniq\_pid\_midwife}
\index{uniq\_pid\_midwife}

\Description
\Prog{uniq\_pid\_midwife} starts a given program, while creating an
artifact of the program's birth.  At a later time the
\Prog{uniq\_pid\_undertaker} can examine the artifact to determine
whether the program is still running or whether it has exited.  
\Prog{uniq\_pid\_midwife}
accomplishes this by recording an enforced unique process identifier to
the artifact.

\begin{Options}
  \OptItem{\OptArg{-\,-file}{filename}}{
    The \Arg{filename} to use for the artifact file.  Defaults to
    \File{pid.file}. 
  }
  \OptItem{\OptArg{-\,-precision}{seconds}}{
    The precision the operating system is expected to have in regards
    to process creation times.  Defaults to an operating system specific
    value.  The default is the best choice in most cases.
  }
  \OptItem{\Opt{-\,-noblock}}{
    Exit after the program has been confirmed, typically 3 times the
    precision.  Defaults to block until the program exits.
  }
  \OptItem{\Arg{program} \oArg{programargs}}{
    Forks a process and executes \Arg{program} with
    \Arg{programargs} as command-line arguments (when specified).
  }
\end{Options}

\ExitStatus
\Prog{uniq\_pid\_midwife} will exit with a status of 0 (zero) upon
success, and non-zero otherwise.

\SeeAlso
\Prog{uniq\_pid\_undertaker} (on page~\pageref{man-uniq-pid-undertaker}),
\Prog{filelock\_midwife} (on page~\pageref{man-filelock-midwife}).

\end{ManPage}
