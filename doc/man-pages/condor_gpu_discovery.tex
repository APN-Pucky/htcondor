\begin{ManPage}{\label{man-condor-gpu-discovery}\Condor{gpu\_discovery}}{1}
{Output GPU-related ClassAd attributes}
\index{HTCondor commands!condor\_gpu\_discovery}
\index{condor\_gpu\_discovery command}

\Synopsis \SynProg{\Condor{gpu\_discovery}}
\Opt{-help}

\SynProg{\Condor{gpu\_discovery}}
\oOpt{<options>}

\Description

\Condor{gpu\_discovery} outputs ClassAd attributes corresponding to a
host's GPU capabilities.  It can presently report CUDA and OpenCL
devices; which type(s) of device(s) it reports is determined by which
libraries, if any, it can find when it runs; this reflects what GPU jobs
will find on that host when they run.  (Note that some HTCondor
configuration settings may cause the environment to differ between jobs
and the HTCondor daemons in ways that change library discovery.)

If \Expr{CUDA\_VISIBLE\_DEVICES} or \Expr{GPU\_DEVICE\_ORDINAL} is set
in the environment when \Condor{gpu\_discovery} is run, it will report
only devices present in the those lists.

This tool is not available for MAC OS platforms.

With no command line options, the single ClassAd attribute
\Attr{DetectedGPUs} is printed.
If the value is 0, no GPUs were detected.
If one or more GPUS were detected, the value is
a string, presented as a comma and space separated list of the GPUs discovered,
where each is given a name further used as the \Arg{prefix string} in other
attribute names.
Where there is more than one GPU of a particular type,
the \Arg{prefix string} includes an integer value numbering the device;
these integer values monotonically increase from 0 (unless otherwise
specified in the environment; see above).
For example, a discovery of two GPUs may output
\begin{verbatim}
DetectedGPUs="CUDA0, CUDA1"
\end{verbatim}
Further command line options use \AdStr{CUDA} either with or without one of
the integer values 0 or 1 as the \Arg{prefix string} in
attribute names.

\begin{Options}
  \OptItem{\Opt{-help}} {
    Print usage information and exit.
  }
  \OptItem{\Opt{-properties}} {
    In addition to the \Attr{DetectedGPUs} attribute, display
    some of the attributes of the GPUs.
    Each of these attributes will have a \Arg{prefix string}
    at the beginning of its name.
    The displayed CUDA attributes are
    \Attr{Capability}, \Attr{DeviceName}, \Attr{DriverVersion},
    \Attr{ECCEnabled}, \Attr{GlobalMemoryMb}, and \Attr{RuntimeVersion}.
    The displayed Open CL attributes are
    \Attr{DeviceName}, \Attr{ECCEnabled}, \Attr{OpenCLVersion}, and
    \Attr{GlobalMemoryMb}.
  }
  \OptItem{\Opt{-extra}} {
    Display more attributes of the GPUs.
    Each of these attribute names will have a \Arg{prefix string}
    at the beginning of its name.
    The additional CUDA attributes are
    \Attr{ClockMhz}, \Attr{ComputeUnits}, and \Attr{CoresPerCU}.
    The additional Open CL attributes are
    \Attr{ClockMhz} and \Attr{ComputeUnits}.
  }
  \OptItem{\Opt{-dynamic}} {
    Display attributes of NVIDIA devices that change values as the GPU
    is working.
    Each of these attribute names will have a \Arg{prefix string}
    at the beginning of its name.
    These are \Attr{FanSpeedPct}, \Attr{BoardTempC}, \Attr{DieTempC},
    \Attr{EccErrorsSingleBit}, and \Attr{EccErrorsDoubleBit}.
  }
  \OptItem{\Opt{-mixed}} {
    When displaying attribute values, assume that the machine has a
    heterogeneous set of GPUs,
    so always include the integer value in the \Arg{prefix string}.
  }
  \OptItem{\OptArg{-device}{<N>}}{
    Display properties only for GPU device \Arg{<N>},
    where \Arg{<N>} is the integer value defined for the \Arg{prefix string}.
    This option may be specified more than once; additional \Arg{<N>} are
    listed along with the first.  This option adds to the devices(s)
    specified by the environment variables \Expr{CUDA\_VISIBLE\_DEVICES}
    and \Expr{GPU\_DEVICE\_ORDINAL}, if any.
  }
  \OptItem{\OptArg{-tag}{string}}{
    Set the resource tag portion of the intended machine ClassAd attribute
    \Attr{Detected<ResourceTag>} to be \Arg{string}.
    If this option is not specified, the resource tag is \AdStr{GPUs},
    resulting in attribute name \Attr{DetectedGPUs}.
  }
  \OptItem{\OptArg{-prefix}{str}}{
    When naming attributes, use \Arg{str} as the \Arg{prefix string}.
    When this option is not specified, 
    the \Arg{prefix string} is either \Expr{CUDA} or \Expr{OCL}.
  }
  \OptItem{\Opt{-simulate:D,N}} {
    For testing purposes, assume that N devices of type D were detected.
    No discovery software is invoked.
    If D is 0, it refers to GeForce GT 330, and a default value for N is 1.
    If D is 1, it refers to GeForce GTX 480, and a default value for N is 2.
  }
  \OptItem{\Opt{-opencl}} {
    Prefer detection via OpenCL rather than CUDA.
    Without this option, CUDA detection software is invoked first,
    and no further Open CL software is invoked if CUDA devices are detected.
  }
  \OptItem{\Opt{-cuda}} {
    Do only CUDA detection.
  }
  \OptItem{\Opt{-nvcuda}} {
    For Windows platforms only, use a CUDA driver rather than the
    CUDA run time.
  }
  \OptItem{\Opt{-config}} {
    Output in the syntax of
    HTCondor configuration, instead of ClassAd language.
    An additional attribute is produced \Attr{NUM\_DETECTED\_GPUs} which
    is set to the number of GPUs detected.
  }
  \OptItem{\Opt{-cron}} {
    This option suppresses the \Attr{DetectedGpus} attribute so that the output
    is suitable for use with \Condor{startd} cron.
    Combine this option with the \Opt{-dynamic} option to periodically refresh the dynamic
    Gpu information such as temperature. For example, to refresh GPU temperatures every 5 minutes
  }
  \begin{verbatim}
  use FEATURE : StartdCronPeriodic(DYNGPUS, 5*60, $(LIBEXEC)/condor_gpu_discovery, -dynamic -cron)
  \end{verbatim}
  \OptItem{\Opt{-verbose}} {
    For interactive use of the tool, output extra information to show 
    detection while in progress.
  }
  \OptItem{\Opt{-diagnostic}} {
    Show diagnostic information, to aid in tool development.
  }
\end{Options}

\ExitStatus

\Condor{gpu\_discovery} will exit with a status value of 0 (zero) upon success,
and it will exit with the value 1 (one) upon failure.

%\Examples

\end{ManPage}
