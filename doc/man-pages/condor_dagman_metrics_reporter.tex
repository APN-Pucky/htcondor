\begin{ManPage}{\label{man-condor-dagman-metrics-reporter}\Condor{dagman\_metrics\_reporter}}{1}
{Report the statistics of a DAGMan run to a central HTTP server}

\index{HTCondor commands!condor\_dagman\_metrics\_reporter}
\index{condor\_dagman\_metrics\_reporter command}

\Synopsis \SynProg{\Condor{dagman\_metrics\_reporter}}
\oOpt{-s}
\oOptArg{-u}{URL}
\oOptArg{-t}{maxtime}
\OptArg{-f}{/path/to/metrics/file}

\Description

\Condor{dagman\_metrics\_reporter} reports statistics of a DAGMan run to a
central server.  This behavior is needed for running DAGMan under Pegasus.  
See \URL{http://pegasus.isi.edu/wms/docs/latest/funding\_citing\_usage.php\#usage\_statistics}
and \URL{https://confluence.pegasus.isi.edu/display/pegasus/DAGMan+Metrics+Reporting}
for the requirements to collect this data.

The data sent to the server is in JSON format.  
Here is an example of what is sent:
\begin{verbatim}
{
    "client": "pegasus-plan",
    "version": "4.2.2",
    "type": "metrics",
    "wf_uuid": "efc309d4-0ca8-4492-be02-3bf931451234",
    "root_wf_uuid": "efc309d4-0ca8-4492-be02-3bf931451234",
    "start_time": 1371570219.048,
    "end_time": 1371571617.634,
    "duration": 1398.586,
    "exitcode": 1,
    "dagman_version": "8.0.0",
    "dagman_id": "manta.cs.wisc.edu:1234:99",
    "parent_dagman_id": "",
    "rescue_dag_number": 0,
    "jobs": 10,
    "jobs_failed": 1,
    "jobs_succeeded": 5,
    "dag_jobs": 2,
    "dag_jobs_failed": 0,
    "dag_jobs_succeeded": 2,
    "total_jobs": 12,
    "total_jobs_run": 8,
    "total_job_time": 1968.000,
    "dag_status": 4
}
\end{verbatim}

Metrics are sent only if the \Condor{dagman} process has
\Env{PEGASUS\_METRICS} set to \Expr{True} in its environment,
and the \Macro{CONDOR\_DEVELOPERS} configuration variable does \emph{not}
have the value \Expr{NONE}.

Ordinarily, this program will be run by \Condor{dagman}, 
and users do not need to interact with it.
This program uses the following environment variables:
\begin{description}
  \item[\Env{PEGASUS\_USER\_METRICS\_SERVER}]
    A comma separated list of URLs that will receive the data. 
  \item[\Env{PEGASUS\_USER\_METRICS\_DEFAULT\_SERVER}]
    The default server to send the data to.
    It defaults to \File{http://metrics.pegasus.isi.edu/metrics}.
    It can be overridden at the command line with the \Opt{-u} option.
\end{description}

The \Arg{-f} argument specifies the metrics file to be sent
to the HTTP server.

\begin{Options}
  \OptItem{\Opt{-s}}{Sleep for a random number of seconds 
    between 1 and 10, before attempting to send data.}
  \OptItem{\OptArg{-u}{URL}}
     {Overrides setting of the environment variable
     \Env{PEGASUS\_USER\_METRICS\_DEFAULT\_SERVER}. 
     This option is unused by \Condor{dagman}; it is for testing by developers.}
  \OptItem{\OptArg{-t}{maxtime}}
    {A maximum time in seconds that defaults to 100 seconds,
    setting a limit on the amount of time this program will wait for
    communication from the server.  
    A setting of zero will result in a single attempt per server. 
    \Condor{dagman} retrieves this value from the 
    \Macro{DAGMAN\_PEGASUS\_REPORT\_TIMEOUT} configuration variable.}
\end{Options}

\ExitStatus

\Condor{dagman\_metrics\_reporter} will exit with a status value of 0 (zero)
 upon success,
and it will exit with a value of 1 (one) upon failure.

\end{ManPage}

