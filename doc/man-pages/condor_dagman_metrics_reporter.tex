\begin{ManPage}{\label{man-condor-dagman-metrics-reporter}\Condor{dagman\_metrics\_reporter}}{1}
{Report the statistics of a DAGMan run to a central HTTP server}

\index{HTCondor commands!condor\_dagman\_metrics\_reporter}
\index{condor\_dagman\_metrics\_reporter command}

\Synopsis \SynProg{\Condor{dagman\_metrics\_reporter}}
\oOpt{-s}
\oOptArg{-u}{URL}
\oOptArg{-t}{maxtime}
\OptArg{-f}{/path/to/metrics/file}

\Description

\Condor{dagman\_metrics\_reporter} anonymously reports metrics from
a DAGMan workflow to a central server.  The reporting of workflow metrics
is only enabled for DAGMan workflows run under Pegasus; metrics reporting
has been requested by Pegasus' funding sources:
see \URL{http://pegasus.isi.edu/wms/docs/latest/funding\_citing\_usage.php\#usage\_statistics}
and \URL{https://confluence.pegasus.isi.edu/display/pegasus/DAGMan+Metrics+Reporting}
for the requirements to collect this data.

The data sent to the server is in JSON format.  
Here is an example of what is sent:
\begin{verbatim}
{
    "client":"condor_dagman",
    "version":"8.1.0",
    "planner":"/lfs1/devel/Pegasus/pegasus/bin/pegasus-plan",
    "planner_version":"4.3.0cvs",
    "type":"metrics",
    "wf_uuid":"htcondor-test-job_dagman_metrics-A-subdag",
    "root_wf_uuid":"htcondor-test-job_dagman_metrics-A",
    "start_time":1375313459.603,
    "end_time":1375313491.498,
    "duration":31.895,
    "exitcode":1,
    "dagman_id":"26",
    "parent_dagman_id":"11",
    "rescue_dag_number":0,
    "jobs":4,
    "jobs_failed":1,
    "jobs_succeeded":3,
    "dag_jobs":0,
    "dag_jobs_failed":0,
    "dag_jobs_succeeded":0,
    "total_jobs":4,
    "total_jobs_run":4,
    "total_job_time":0.000,
    "dag_status":2
}
\end{verbatim}

Metrics are sent only if the \Condor{dagman} process has
\Env{PEGASUS\_METRICS} set to \Expr{True} in its environment,
and the \Macro{CONDOR\_DEVELOPERS} configuration variable does \emph{not}
have the value \Expr{NONE}.

Ordinarily, this program will be run by \Condor{dagman}, 
and users do not need to interact with it.
This program uses the following environment variables:
\begin{description}
  \item[\Env{PEGASUS\_USER\_METRICS\_DEFAULT\_SERVER}]
    The URL of the default server to which to send the data.
    It defaults to \File{http://metrics.pegasus.isi.edu/metrics}.
    It can be overridden at the command line with the \Opt{-u} option.
  \item[\Env{PEGASUS\_USER\_METRICS\_SERVER}]
    A comma separated list of URLs of servers that will receive the
    data, in addition to the default server. 
\end{description}

The \Arg{-f} argument specifies the metrics file to be sent
to the HTTP server.

\begin{Options}
  \OptItem{\Opt{-s}}{Sleep for a random number of seconds 
    between 1 and 10, before attempting to send data.  This option
    is used to space out the reporting from any sub-DAGs when a
    DAG is removed.}
  \OptItem{\OptArg{-u}{URL}}
     {Overrides setting of the environment variable
     \Env{PEGASUS\_USER\_METRICS\_DEFAULT\_SERVER}. 
     This option is unused by \Condor{dagman}; it is for testing by developers.}
  \OptItem{\OptArg{-t}{maxtime}}
    {A maximum time in seconds that defaults to 100 seconds,
    setting a limit on the amount of time this program will wait for
    communication from the server.  
    A setting of zero will result in a single attempt per server. 
    \Condor{dagman} retrieves this value from the 
    \Macro{DAGMAN\_PEGASUS\_REPORT\_TIMEOUT} configuration variable.}
  \OptItem{\OptArg{-f}{metrics\_file}}
    {The name of the file containing the metrics values to be reported.}
\end{Options}

\ExitStatus

\Condor{dagman\_metrics\_reporter} will exit with a status value of 0 (zero)
 upon success,
and it will exit with a value of 1 (one) upon failure.

\end{ManPage}

