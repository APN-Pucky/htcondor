\begin{ManPage}{\label{man-condor-dagman-metrics-reporter}\Condor{dagman\_metrics\_reporter}}{1}
{Report the statistics of a DAGMan run to a central HTTP server}
\Synopsis \SynProg{\Condor{dagman\_metrics\_reporter}}
\oOpt{-s}
\oOptArg{-f}{/path/to/metrics/file}
\oOptArg{-u}{url of a HTTP server to receive metrics}
\oOptArg{-t}{maximum time to run}

\index{HTCondor commands!condor\_dagman\_metrics\_reporter}
\index{condor\_dagman\_metrics\_reporter command}

\Description

\Condor{dagman\_metrics\_reporter} reports statistics of a DAGMan run to a
central server.  This behavior is needed for running DAGMan under Pegasus.  See
http://pegasus.isi.edu/wms/docs/latest/funding\_citing\_usage.php\#usage\_statistics
and https://confluence.pegasus.isi.edu/display/pegasus/DAGMan+Metrics+Reporting
for the requirement to collect this data.

The data sent to the server is in JSON format.  Here is an example of what is sent:
\begin{verbatim}
{
    "client": "pegasus-plan",
    "version": "4.2.2",
    "type": "metrics",
    "wf_uuid": "efc309d4-0ca8-4492-be02-3bf931451234",
    "root_wf_uuid": "efc309d4-0ca8-4492-be02-3bf931451234",
    "start_time": 1371570219.048,
    "end_time": 1371571617.634,
    "duration": 1398.586,
    "exitcode": 1,
    "dagman_version": "8.0.0",
    "dagman_id": "manta.cs.wisc.edu:1234:99",
    "parent_dagman_id": "",
    "rescue_dag_number": 0,
    "jobs": 10,
    "jobs_failed": 1,
    "jobs_succeeded": 5,
    "dag_jobs": 2,
    "dag_jobs_failed": 0,
    "dag_jobs_succeeded": 2,
    "total_jobs": 12,
    "total_jobs_run": 8,
    "total_job_time": 1968.000,
    "dag_status": 4
}
\end{verbatim}

Metrics are sent only if the \Condor{dagman} process has
\Macro{PEGASUS\_METRICS} set to ``true'' in its environment and the
\Macro{CONDOR\_DEVELOPERS} configuration variable does not have value NONE

Ordinarily, this program will be run by \Condor{dagman}, and users do not need to interact with it.
This program uses the following environment variables:
\begin{Options}
	\OptItem{\Macro{PEGASUS\_USER\_METRICS\_SERVER}}{A comma separated list of URLs that will receive the
data.} 
	\OptItem{\Macro{PEGASUS\_USER\_METRICS\_DEFAULT\_SERVER}}{The default server to send the data to.  It defaults to http://metrics.pegasus.isi.edu/metrics. It can be overridden at the command line with the -u option.}
\end{Options}

The -f command line option that specifies the metrics file is required.

\begin{Options}
	\OptItem{\Opt{-s}}{Sleep for a random number of seconds (between 1 and 10), before attempting to
send data.}
	\OptItem{\OptArg{-f}{The path of the metrics file}}{The metrics file that contains the data to be sent to the HTTP server.}
	\OptItem{\OptArg{-u}{URL to send metrics to}}{This command line option will override \Macro{PEGASUS\_USER\_METRICS\_DEFAULT\_SERVER}. Unused by \Condor{dagman}, it is for testing by developers.}
	\OptItem{\OptArg{-t}{Number of seconds to attempt communication.}}{This defaults to 100 seconds.  A setting of zero will result in a single attempt per server. \Condor{dagman} retrieves this value from the \Macro{DAGMAN\_PEGASUS\_REPORT\_TIMEOUT} configuration variable.}
\end{Options}

\ExitStatus

The exit status of \Condor{dagman\_metrics\_reporter} is zero on success, and one on failure.

\end{ManPage}

