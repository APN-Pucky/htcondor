\begin{ManPage}{\label{man-condor-configure}\Condor{configure}}{1}
{Configure or install Condor}

\Synopsis \SynProg{\Condor{configure}}
[\verb@--@\textbf{help}] $|$ [\verb@--@\textbf{version}]

\SynProg{\Condor{configure}}
[\verb@--@\textbf{install}]
[\verb@--@\textbf{install-dir=$<$path$>$}]
[\verb@--@\textbf{local-dir=$<$path$>$}]
[\verb@--@\textbf{make-personal-condor}]
[\verb@--@\textbf{type = $<$ submit, execute, manager $>$}]
[\verb@--@\textbf{central-manager = $<$ hostname$>$}]
[\verb@--@\textbf{owner = $<$ ownername $>$}]
[\verb@--@\textbf{verbose}]

% 2 hyphens in a row, without any spaces inbetween poses a problem
% for LaTeX.  Things tried, without success.
%   --      results in a single hyphen
%   - -     results in a 2 hyphens, with too much space inbetween
%   -\--    results in a 2 hyphens, with too much space inbetween
%  $--$     results in 2 long, very high up dashes, with nice spacing
%  $-$$-$   results in 2 long, very high up dashes, with too much spacing

% Alain's fix
%   {\tt--}          makes one nice dash in html
%                    makes 2 nice dashes in postscript
%   {\tt--}{\tt--}   makes 2 nice dashes that collide with following letter
%                       in html
%                    makes 4  nice dashes in postscript
%   {\small{\tt--}}{\small{\tt--}}  2 nice dashes in html
%                                   4 nice dashes in postscript

% Peter's fix
%   ---            makes 2 dashes exactly as required in the html
%                  makes 1 very long dash in postscript

% Karen's final fix:  do everything manually (no LaTeX macros),
%  and use the verbatim.

\index{Condor commands!condor\_configure}
\index{condor\_configure command}

\Description 

\Condor{configure} is a Perl script that installs and/or configures
Condor.
It will run with Perl 5 or more recent versions.

\Condor{configure} is designed to be run more than one time
where required.
It can install Condor (with a correct configuration),
or it can change the configuration files.
Note that changes in the configuration files do not result
in changes while Condor is running.
To effect changes while Condor is running,
it is necessary to further do \Condor{reconfig} or \Condor{restart}.
\Condor{reconfig}  is required where the currently executing
daemons need to be informed of configuration changes.
\Condor{restart} is required where the options
\verb@--@\textbf{make-personal-condor} or
\verb@--@\textbf{type}
are used, since these affect which daemons are running.

Running \Condor{configure} with no options results in the
help screen being printed.

\begin{Options}
	\OptItem{\Opt{---help}} {Print help screen and exit}
	\OptItem{\Opt{---version}} {Print Condor version number and exit}
	\OptItem{\Opt{---install}} {Perform installation, assuming that
	the current working directory contains the \File{release.tar}
	file.  Without further options, the configuration is that of
	a Personal Condor, i.e a complete one-machine pool.  If used as an 
	upgrade within an existing installation directory, existing 
	configuration files and local directory are preserved.}
	\OptItem{\Opt{---install-dir=$<$path$>$}} {Specifies the path
	where Condor should be installed (or the path where it already is
	installed). The default is the current working directory.}
	\OptItem{\Opt{---local-dir=$<$path$>$}} {Specifies the
	location of the local directory, which is the directory that generally 
	contains the local (machine-specific) configuration file as well as the
	directories where Condor daemons write their run-time information 
	(spool, log, execute). This location is indicated  by the \MacroNI{LOCAL\_DIR} 
	setting in the configuration file. 
	When installing (i.e. if ---install is specified) \Condor{configure} 
	will properly create the local directory in the location specified. If
	none is specified, the default value is 
\Arg{\$(RELEASE\_DIR)/local.\$(HOSTNAME)}.

	During subsequent invocations of \Condor{configure} (i.e. w/o the 
	---install option), if the ---local-dir is specified, the new directory
	will be created and the \File{log}, \File{spoool} and \File{execute} 
	directories will be moved there from their current location.}
	\OptItem{\Opt{---make-personal-condor}} {Installs and configures for 
	  Personal Condor (i.e. a fully-functional one-machine pool)}
	\OptItem{\Opt{---type= $<$ submit, execute, manager $>$}} {One
	or more of the types may be listed.
	This determines the roles that a machine may play in a pool (in general any
	machine can be a submit and/or execute machine, and there is one central manager
	per pool). In case of a Personal Condor, the machine fulfills all three of those
	roles.}
	\OptItem{\Opt{---central-manager=$<$hostname$>$}} {Instructs the current
	Condor installation to use the specified machine as the central manager. 
	(This effectively modifies the configuration variables \MacroNI{COLLECTOR\_HOST}
	and \MacroNI{NEGOTIATOR\_HOST} to point to the given host name). The central \
	manager machine's Condor configuration needs to be independently configured to 
	act as a manager (using --type=manager). }
	\OptItem{\Opt{---owner=$<$ownername$>$}} {Set configuration
	such that Condor daemons will be executed as the given owner. This modifies the 
	ownership on the \File{log}, \File{spool} and \File{execute} directories and sets
	\MacroNI{CONDOR\_IDS} value in the config file, to ensure that Condor daemons start up as the specified effective user. See section~\ref{sec:uids} on
UIDs in Condor on page~\pageref{sec:uids} in the Administrator's
Manual for details.
	This is only applicable when \Condor{configure} is run by root.
	If not run as root, the owner is the user running the \Condor{configure} 
	command.
	}
	\OptItem{\Opt{---verbose}} {Print information about changes
	to configuration variables as they occur.}
\end{Options}

\ExitStatus

\Condor{configure} will exit with a status value of 0 (zero) upon success,
and it will exit with a nonzero value upon failure.

\Examples
To install Condor on the machine  (e.g. \Arg{machine1@cs.wisc.edu}) that is to be the pool's central manager, do (on \Arg{machine1}, from the directory where you've unzipped the Condor distribution .tar.gz file):
\begin{verbatim}
% condor_configure --install --type=submit,execute,manager
\end{verbatim}
This will alow the machine to submit and execute Condor jobs (in addition to being the central manager of the pool).


To make an execute-only machine (i.e. dedicate computing node) join a pool with central manager on \Arg{machine1@cs.wisc.edu}, do (on that machine from the directory where Condor is installed):
\begin{verbatim}
% condor_configure --central-manager=machine1@cs.wisc.edu --type=execute
\end{verbatim}



To change the location of the \MacroNI{LOCAL\_DIR} directory
in the configuration file, do (from the directory where Condor is installed):
\begin{verbatim}
% condor_configure --local-dir=/path/to/new/local/directory
\end{verbatim}
This will move the \File{log},\File{spool},\File{execute} directories to \File{/path/to/new/local/directory} from the current local directory.



\end{ManPage}
