\begin{ManPage}{\label{man-condor-submit}\Condor{submit}}{1}
{Queue jobs for execution under Condor}
\Synopsis \SynProg{\Condor{submit}}
\oOpt{-verbose}
\oOpt{-unused}
\oOptArg{-name}{schedd\_name}
\oOptArg{-remote}{schedd\_name}
\oOptArg{-pool}{pool\_name}
\oOpt{-disable}
\oOptArg{-password}{passphrase}
\ToolDebugOption
% this option needs the dots in boldface, so using the macro oOptArg won't work
\Lbr\Opt{-append} \Arg{command} \Opt{\Dots}\Rbr 
\oOpt{-spool}
\oOptArg{-dump}{filename}
\oArg{submit description file}

\index{Condor commands!condor\_submit}
\index{condor\_submit command}

\Description

\Condor{submit} is the program for submitting jobs for execution
under Condor.
\Condor{submit} requires a submit description file which contains commands
to direct the queuing of jobs.
One submit description file may contain
specifications for the queuing of many Condor jobs at once.
A single invocation of \Condor{submit} may cause one or
more clusters.
A cluster is a set of jobs
specified in the submit description file
between \SubmitCmd{queue} commands for which the executable is not changed.
It is advantageous to submit
multiple jobs as a single cluster because:
\begin{itemize}
\item Only one copy of the checkpoint file is needed to 
represent all jobs in a cluster until they begin execution.
\item There is much less overhead involved for Condor to start the next
job in a cluster than for Condor to start a new cluster.  This can make
a big difference when submitting lots of short jobs.
\end{itemize}

Multiple clusters may be specified within a single
submit description file.
Each cluster must specify a single executable.

The job ClassAd attribute \Attr{ClusterId} identifies a cluster.
See specifics for this attribute
in the Appendix on page~\pageref{sec:Job-ClassAd-Attributes}.

Note that submission of jobs from a Windows machine requires
a stashed password to allow Condor to impersonate the user submitting
the job.
To stash a password, use the \Condor{store\_cred} command.
See the manual page at
page~\pageref{man-condor-store-cred} for details.

See section~\ref{sec:submit-cmds}
for the commands that may be placed in the submit description file
to direct the submission of a job.

\begin{Options}

\OptItem{\Opt{-verbose}}{Verbose output - display the created job ClassAd}

\OptItem{\Opt{-unused}}{As a default, causes no warnings to be issued about
  user-defined macros not being used within the submit description file.
  The meaning reverses (toggles) when
  the configuration variable \Macro{WARN\_ON\_UNUSED\_SUBMIT\_FILE\_MACROS}
  is set to the nondefault value of \Expr{False}.
  Printing the warnings can help identify spelling
  errors of submit description file commands.  The warnings are sent
  to stderr. }

\OptItem{\OptArg{-name}{schedd\_name}}{Submit to the specified \Condor{schedd}.
Use this option to submit to a \Condor{schedd} other than the default local one.
\Arg{schedd\_name} is the value of the \Attr{Name} ClassAd attribute on
the machine where the \Condor{schedd} daemon runs.}


\OptItem{\OptArg{-remote}{schedd\_name}}{Submit to the specified
\Condor{schedd}, spooling all required input files over the network connection.
\Arg{schedd\_name} is the value of the \Attr{Name} ClassAd attribute
on the machine where the \Condor{schedd} daemon runs.
This option is equivalent to using both \Opt{-name} and \Opt{-spool}.}

\OptItem{\OptArg{-pool}{pool\_name}}{Look in the specified pool for the 
\Condor{schedd} to submit to.
This option is used with \Opt{-name} or \Opt{-remote}.}

\OptItem{\Opt{-disable}}{Disable file permission checks.}

\OptItem{\OptArg{-password}{passphrase}}{Specify a password to the 
\Prog{MyProxy} server. }

\OptItem{\Opt{-debug}}{Cause debugging information to be sent to \File{stderr},
based on the value of the configuration variable \MacroNI{SUBMIT\_DEBUG}.}

\OptItem{\OptArg{-append}{command}}{Augment the commands in the submit 
description file with the given command.
This command will be considered to immediately precede the Queue
command within the submit description file, and come after all other
previous commands.
The submit description file is not modified.
Multiple commands are specified by using the \Opt{-append} option multiple times.
Each new command is given in a separate \Opt{-append} option.
Commands with spaces in them will need to be enclosed in double quote
marks.}

\OptItem{\Opt{-spool}}{Spool all required input files, user log, and
proxy over the connection to the \Condor{schedd}.
After submission, modify
local copies of the files without affecting your jobs. Any output
files for completed jobs need to be retrieved with \Condor{transfer\_data}.}

\OptItem{\OptArg{-dump}{filename}}{Sends all ClassAds to the specified
  file, instead of to the \Condor{schedd}.}

\OptItem{\Arg{submit description file}}{The pathname to the submit description
file. If this optional argument is missing or equal to ``-'', then the commands
are taken from standard input.}

\end{Options}

\ExitStatus

\Condor{submit} will exit with a status value of 0 (zero) upon success, and a
non-zero value upon failure.

\label{condor-submit-examples}
\Examples

\begin{itemize} 
\item{Submit Description File Example 1:} This example queues three jobs for
execution by Condor. The first will be given command line arguments of
\Arg{15} and \Arg{2000}, and it will write its standard output
to \File{foo.out1}.
The second will be given command line arguments of 
\Arg{30} and \Arg{2000}, and it will
write its standard output to \File{foo.out2}.
Similarly the third will have
arguments of 
\Arg{45} and \Arg{6000}, and it will use \File{foo.out3} for its standard
output. Standard error output (if any) from all three programs will
appear in \File{foo.error}.

\footnotesize
\begin{verbatim}
      ####################
      #
      # submit description file
      # Example 1: queuing multiple jobs with differing
      # command line arguments and output files.
      #                                                                      
      ####################                                                   
                                                                         
      Executable     = foo                                                   
      Universe       = standard
                                                                         
      Arguments      = 15 2000                                               
      Output  = foo.out1                                                     
      Error   = foo.err1
      Queue                                                                  
                                                                         
      Arguments      = 30 2000                                               
      Output  = foo.out2                                                     
      Error   = foo.err2
      Queue                                                                  
                                                                         
      Arguments      = 45 6000                                               
      Output  = foo.out3                                                     
      Error   = foo.err3
      Queue                   
\end{verbatim}
\normalsize

\item{Submit Description File Example 2:} This submit description file
example queues 150
runs of program \Prog{foo} which must have been compiled and linked for
Sun workstations running Solaris 8.
Condor will not attempt
to run the processes on machines which have less than 32 Megabytes of
physical memory, and it will run them on machines which have at least 64
Megabytes, if such machines are available.
Stdin, stdout, and stderr will
refer to \File{in.0}, \File{out.0}, and \File{err.0} for the first run
of this program (process 0).
Stdin, stdout, and stderr will refer to
\File{in.1}, \File{out.1}, and \File{err.1} for process 1, and so forth.
A log file containing entries
about where and when Condor runs, takes checkpoints, and migrates processes
in this cluster will be written into file \File{foo.log}.

\footnotesize
\begin{verbatim}
      ####################                                                    
      #                                                                       
      # Example 2: Show off some fancy features including
      # use of pre-defined macros and logging.                                
      #                                                                       
      ####################                                                    
                                                                          
      Executable     = foo                                                    
      Universe       = standard
      Requirements   = Memory >= 32 && OpSys == "SOLARIS28" && Arch =="SUN4u"     
      Rank           = Memory >= 64
      Image_Size     = 28 Meg                                                 
                                                                          
      Error   = err.$(Process)                                                
      Input   = in.$(Process)                                                 
      Output  = out.$(Process)                                                
      Log = foo.log                                                                       
                                                                          
      Queue 150
\end{verbatim}
\normalsize

\item{Command Line example:} The following command uses the
\Opt{-append} option to add two commands before the job(s) is queued.
A log file and an error log file are specified.
The submit description file is unchanged.
\footnotesize
\begin{verbatim}
condor_submit -a "log = out.log" -a "error = error.log" mysubmitfile
\end{verbatim}
\normalsize
Note that each of the added commands is contained within quote marks
because there are space characters within the command.

\item{\AdAttr{periodic\_remove} example:}
A job should be removed from the queue,
if the total suspension time of the job
is more than half of the run time of the job.

Including the command
\footnotesize
\begin{verbatim}
   periodic_remove = CumulativeSuspensionTime > 
                     ((RemoteWallClockTime - CumulativeSuspensionTime) / 2.0)
\end{verbatim}
\normalsize
in the submit description file causes this to happen.

\end{itemize} 


\GenRem
\begin{itemize}

\item For security reasons, Condor will refuse to run any jobs submitted
by user root (UID = 0) or by a user whose default group is group wheel
(GID = 0). Jobs submitted by user root or a user with a default group of
wheel will appear to sit forever in the queue in an idle state. 

\item All path names specified in the submit description file must be
less than 256 characters in length, and command line arguments must be
less than 4096 characters in length; otherwise, \Condor{submit} gives a
warning message but the jobs will not execute properly. 

\item Somewhat understandably, behavior gets bizarre if the user makes
the mistake of requesting multiple Condor jobs to write to the
same file, and/or if the user alters any files that need to be accessed
by a Condor job which is still in the queue.
For example, the compressing of data or
output files before a Condor job has completed is a common mistake.

\item To disable checkpointing for Standard Universe jobs, include the
line:
\begin{verbatim}
      +WantCheckpoint = False
\end{verbatim}
in the submit description file before the queue command(s).
\end{itemize}

\SeeAlso
Condor User Manual

\end{ManPage}

