\begin{ManPage}{\label{man-condor-submit}\Condor{submit}}{1}
{Queue jobs for execution on remote machines}
\Synopsis \SynProg{\Condor{submit}}
\oOpt{--}
\oOpt{-v}
\oOptArg{-n}{schedd\_name}
\oOptArg{-r}{schedd\_name}
\oOpt{-d}
\Arg{submit-description file}

\index{Condor commands!condor\_submit}
\index{condor\_submit command}

\Description

\Condor{submit} is the program for submitting jobs to Condor.
\Condor{submit} requires a submit-description file which contains commands
to direct the queuing of jobs. One description file may contain
specifications for the queuing of many condor jobs at once. All jobs queued by a
single invocation of \Condor{submit} must share the same executable, and
are referred to as a ``job cluster''. It is advantageous to submit
multiple jobs as a single cluster because:
\begin{itemize}
\item Only one copy of the checkpoint file is needed to 
represent all jobs in a cluster until they begin execution.
\item There is much less overhead involved for Condor to start the next
job in a cluster than for Condor to start a new cluster.  This can make
a big difference if you are submitting lots of short running jobs.
\end{itemize}

\emph{SUBMIT DESCRIPTION FILE COMMANDS}

Each condor job description file describes one cluster of jobs to be
placed in the condor execution pool. All jobs in a cluster must share
the same executable, but they may have different input and output files,
and different program arguments, etc. The submit-description file is then
used as the only command-line argument to \Condor{submit}. 

The submit-description file must contain one \Arg{executable} command and at least one
\Arg{queue} command.  All of the other commands have default actions.

The commands which can appear in the submit-description file are:

\begin{description} 

%%%%%%%%%%%%%%%%%%%
%% executable
%%%%%%%%%%%%%%%%%%%

\item[executable = $<$name$>$]The name of the executable file for this
job cluster. Only one executable command may be present in a description
file. If submitting into the Standard Universe, which is the default,
then the named executable must have been re-linked with the Condor
libraries (such as via the \Condor{compile} command). If submitting into
the Vanilla Universe, then the named executable need not be re-linked and
can be any process which can run in the background (shell scripts work
fine as well). 

%%%%%%%%%%%%%%%%%%%
%% input
%%%%%%%%%%%%%%%%%%%

\item[input = $<$pathname$>$] Condor assumes that its jobs are
long-running, and that the user will not wait at the terminal for their
completion. Because of this, the standard files which normally access
the terminal, (stdin, stdout, and stderr), must refer to files. Thus,
the filename specified with \Opt{input} should contain any keyboard
input the program requires (i.e. this file becomes stdin). If not
specified, the default value of /dev/null is used. 

%%%%%%%%%%%%%%%%%%%
%% output
%%%%%%%%%%%%%%%%%%%

\item[output = $<$pathname$>$] The \Opt{output} filename will capture
any information the program would normally write to the screen (i.e.
this file becomes stdout). If not specified, the default value of
/dev/null is used. More than one job should not use the same output
file, since this will cause one job to overwrite the output of
another.

%%%%%%%%%%%%%%%%%%%
%% error
%%%%%%%%%%%%%%%%%%%

\item[error = $<$pathname$>$] The \Opt{error} filename will capture any
error messages the program would normally write to the screen (i.e. this
file becomes stderr). If not specified, the default value of /dev/null
is used. More than one job should not use the same error file, since
this will cause one job to overwrite the errors of another.

%%%%%%%%%%%%%%%%%%%
%% arguments
%%%%%%%%%%%%%%%%%%%

\item[arguments = $<$argument\_list$>$] List of arguments to be supplied
to the program on the command line. 


%%%%%%%%%%%%%%%%%%%
%% initialdir
%%%%%%%%%%%%%%%%%%%

\item[initialdir = $<$directory-path$>$] Used to specify the current
working directory for the Condor job. Should be a path to a preexisting
directory. If not specified, \Condor{submit} will automatically insert
the user's current working directory at the time \Condor{submit} was run
as the value for \Opt{initialdir}. 


%%%%%%%%%%%%%%%%%%%
%% requirements
%%%%%%%%%%%%%%%%%%%

\item[requirements = $<$ClassAd Boolean Expression$>$] The requirements
command is a boolean ClassAd expression which uses C-like operators. In
order for any job in this cluster to run on a given machine, this
requirements expression must evaluate to true on the given machine. For
example, to require that whatever machine executes your program has a
least 64 Meg of RAM and has a MIPS performance rating greater than 45,
use: 
\begin{verbatim}
        requirements = Memory >= 64 && Mips > 45
\end{verbatim}
Only one requirements command may be present in a
description file. By default, \Condor{submit} 
appends the following clauses to the requirements expression:
\begin{enumerate}
	\item Arch and OpSys are set equal to the Arch and OpSys of the
submit machine.  In other words: unless you request otherwise, Condor will give your
job machines with the same architecture and operating system version as
the machine running \Condor{submit}.
	\item Disk $>$ ExecutableSize.  To ensure there is enough disk space on the 
target machine for Condor to copy over your executable.
	\item VirtualMemory $>=$ ImageSize.  To ensure the target machine
has enough virtual memory to run your job.
	\item If Universe is set to Vanilla, FileSystemDomain is set equal to
the submit machine's FileSystemDomain.
\end{enumerate}
You can view the requirements of a job
which has already been submitted (along with everything else about the
job ClassAd) with the command \Condor{q -l}; see the command reference for
\Condor{q} on page~\pageref{man-condor-q}.  Also, see the Condor Users
Manual for complete information on the syntax and available attributes
that can be used in the ClassAd expression.

%%%%%%%%%%%%%%%%%%%
%% rank
%%%%%%%%%%%%%%%%%%%

\item[rank = $<$ClassAd Float Expression$>$] A ClassAd Floating-Point 
expression that states how to rank machines which have already met the requirements
expression. Essentially, rank expresses preference.  A higher numeric value 
equals better rank. Condor will give the job the machine with the 
highest rank.  For example,
\begin{verbatim}
        requirements = Memory > 60
        rank = Memory
\end{verbatim}
asks Condor to find all available machines with more than 60 megabytes of memory
and give the job the one with the most amount of memory.  See the Condor Users
Manual for complete information on the syntax and available attributes
that can be used in the ClassAd expression.

%%%%%%%%%%%%%%%%%%%
%% priority
%%%%%%%%%%%%%%%%%%%

\item[priority = $<$priority$>$] Condor job priorities range from -20 to
+20, with 0 being the default. Jobs with higher numerical priority will
run before jobs with lower numerical priority. Note that this priority
is on a per user basis; setting the priority will determine the order in
which your own jobs are executed, but will have no effect on whether or
not your jobs will run ahead of another user's jobs. 

%%%%%%%%%%%%%%%%%%%
%% notification
%%%%%%%%%%%%%%%%%%%

\item[notification = $<$when$>$]\label{man-condor-submit-notification} Owners of condor jobs are notified by
email when certain events occur.
If \Arg{when} is set to \mbox{Always}, the owner will be notified
whenever the job is checkpointed, and when it completes.
If \Arg{when} is set to \mbox{Complete} (the default), the owner will
be notified when the job terminates.
If \Arg{when} is set to \mbox{Error}, the owner will only be notified
if the job terminates abnormally.
If \Arg{when} is set to \mbox{Never}, the owner will not be mailed,
regardless what happens to the job.
The statistics included in the email are documented in
section~\ref{sec:job-completion} on
page~\pageref{sec:job-completion}.

%%%%%%%%%%%%%%%%%%%
%% notify_user
%%%%%%%%%%%%%%%%%%%

\item[notify\_user = $<$email-address$>$]\label{man-condor-submit-notify-user} Used to specify the email
address to use when Condor sends email about a job.  If not specified,
Condor will default to using :
\begin{verbatim}
        job-owner@UID_DOMAIN
\end{verbatim}
where \Macro{UID\_DOMAIN} is specified by the Condor site administrator.  If 
\Macro{UID\_DOMAIN} has not been specified, Condor will send the email
to :
\begin{verbatim}
        job-owner@submit-machine-name
\end{verbatim}

%%%%%%%%%%%%%%%%%%%
%% copy_to_spool 
%%%%%%%%%%%%%%%%%%%

\item[copy\_to\_spool = $<$True \Bar\ False$>$] If \Opt{copy\_to\_spool} is set to
\Arg{True}, then \Condor{submit} will copy the executable to the local spool 
directory before running it on a remote host. Oftentimes this can be quite
time consuming and unnecessary. By setting it to \Arg{False}, \Condor{submit}
will skip this step.  Defaults to \Arg{True}.

%%%%%%%%%%%%%%%%%%%
%% getenv
%%%%%%%%%%%%%%%%%%%

\item[getenv = $<$True \Bar\ False$>$] If \Opt{getenv} is set to
\Arg{True}, then \Condor{submit} will copy all of the user's current
shell environment variables at the time of job submission into the job
ClassAd. The job will therefore execute with the same set of environment
variables that the user had at submit time. Defaults to \Arg{False}.

%%%%%%%%%%%%%%%%%%%
%% hold
%%%%%%%%%%%%%%%%%%%

\item[hold = $<$True \Bar\ False$>$] If \Opt{hold} is set to
\Arg{True}, then the job will be submitted in the hold state.  Jobs in
the hold state will not run until released by \Condor{release}.

%%%%%%%%%%%%%%%%%%%
%% environment
%%%%%%%%%%%%%%%%%%%

\item[environment = $<$parameter\_list$>$] List of environment variables
of the form :
\begin{verbatim}
        <parameter> = <value>
\end{verbatim}
Multiple environment variables can be specified by separating them with a
semicolon (`` ; ''). These environment variables will be placed into the
job's environment before execution. The length of all characters
specified in the environment is currently limited to 4096 characters. 

%%%%%%%%%%%%%%%%%%%
%% log
%%%%%%%%%%%%%%%%%%%

\item[log = $<$pathname$>$] Use \Opt{log} to specify a filename where
Condor will write a log file of what is happening with this job cluster.
For example, Condor will log into this file when and where the job
begins running, when the job is checkpointed and/or migrated, when the
job completes, etc. Most users find specifying a \Opt{log} file to be very
handy; its use is recommended. If no \Opt{log} entry is specified, 
Condor does not create a log for this cluster.

%%%%%%%%%%%%%%%%%%%
%% universe
%%%%%%%%%%%%%%%%%%%

\item[universe = $<$vanilla \Bar\ standard \Bar\ pvm \Bar\ scheduler
\Bar\ globus \Bar\ mpi$>$] 
Specifies which Condor Universe to use when running this job.  The Condor 
Universe specifies a Condor execution environment.  The \Arg{standard} 
Universe is the default, and tells Condor that this job has been re-linked 
via \Condor{compile} with the Condor libraries and therefore supports
checkpointing and remote system calls.  The \Arg{vanilla} Universe is an
execution environment for jobs which have not been linked with the
Condor libraries.  \textit{Note:} use the \Arg{vanilla} Universe to
submit shell scripts to Condor.  The \Arg{pvm} Universe is for a
parallel job written with PVM 3.4. The \Arg{scheduler} is for a job that
should act as a metascheduler. The \Arg{globus} universe translates the
submit description file to a Globus RSL string and passes it to 
the \Prog{globusrun} program for execution.  The \Arg{mpi} universe is
for running mpi jobs made with the MPICH package.
See the Condor User's Manual for more information about using Universe.

%%%%%%%%%%%%%%%%%%%
%% image_size
%%%%%%%%%%%%%%%%%%%

\item[image\_size = $<$size$>$] This command tells Condor the maximum
virtual image size to which you believe your program will grow during
its execution. Condor will then execute your job only on machines which
have enough resources, (such as virtual memory), to support executing
your job. If you do not specify the image size of your job in the
description file, Condor will automatically make a (reasonably accurate)
estimate about its size and adjust this estimate as your program runs.
If the image size of your job is underestimated, it may crash due to
inability to acquire more address space, e.g. malloc() fails. If the image
size is overestimated, Condor may have difficulty finding machines which
have the required resources. \Arg{size} must be in kbytes, e.g. for
an image size of 8 megabytes, use a \Arg{size} of 8000.

%%%%%%%%%%%%%%%%%%%
%% machine_count
%%%%%%%%%%%%%%%%%%%

\item[machine\_count = $<$min..max$>$ \Bar\ $<$max$>$] 
If \Opt{machine\_count} is
specified, Condor will not start the job until it can simultaneously
supply the job with \Arg{min} machines.  Condor will continue to try 
to provide up
to \Arg{max} machines, but will not delay starting of the job to do so.
If the job is started with fewer than \Arg{max} machines, the job
will be notified via a usual PvmHostAdd notification as additional
hosts come on line.
\textbf{Important:} only use \Opt{machine\_count} if an only if
submitting into the PVM or MPI Universes.  Use min..max for the PVM
universe, and just max for the MPI universe.

%%%%%%%%%%%%%%%%%%%
%% coresize
%%%%%%%%%%%%%%%%%%%

\item[coresize = $<$size$>$] Should the user's program abort and produce
a core file, \Opt{coresize} specifies the maximum size in bytes of the
core file which the user wishes to keep. If \Opt{coresize} is not
specified in the command file, the system's user resource limit
\mbox{``coredumpsize''} is used (except on HP-UX). 

%%%%%%%%%%%%%%%%%%%
%% nice_user
%%%%%%%%%%%%%%%%%%%

\item[nice\_user = $<$True \Bar\ False$>$] \label{man-condor-submit-nice}Normally, when a machine
becomes available to Condor, Condor decides which job to run based upon
user and job priorities. Setting \Opt{nice\_user} equal to \Arg{True}
tells Condor not to use your regular user priority, but that this job
should have last priority amongst all users and all jobs. So jobs
submitted in this fashion run only on machines which no other
non-nice\_user job wants --- a true ``bottom-feeder'' job! This is very
handy if a user has some jobs they wish to run, but do not wish to use
resources that could instead be used to run other people's Condor jobs. Jobs
submitted in this fashion have ``nice-user.'' pre-appended in front of
the owner name when viewed from \Condor{q} or \Condor{userprio}.  The
default value is \Opt{False}.

%%%%%%%%%%%%%%%%%%%
%% kill_sig
%%%%%%%%%%%%%%%%%%%

\item[kill\_sig = $<$signal-number$>$] When Condor needs to kick a job
off of a machine, it will send the job the signal specified by
\Arg{signal-number}.  \Arg{signal-number} needs to be an integer which
represents a valid signal on the execution machine.  For jobs submitted
to the Standard Universe, the default value is the number for
\verb@SIGTSTP@ which tells the Condor libraries to initiate a checkpoint
of the process.  For jobs submitted to the Vanilla Universe, the default 
is \verb@SIGTERM@ which is the standard way to terminate a program in UNIX.  

%%%%%%%%%%%%%%%%%%%
%% buffer_size
%%%%%%%%%%%%%%%%%%%

\item[buffer\_size $=$ $<$bytes-in-buffer$>$]
Condor keeps a buffer of recently-used data for each file an
application opens.  This option specifies the maximum
number of bytes to be buffered for each open file at the executing machine.

The buffer size and its effect on throughput may be viewed with
the \verb@-io@ option to \Condor{status}. 
In this version of Condor, the default buffer size is 512 KB, unless
the configuration file macro \Macro{DEFAULT\_IO\_BUFFER\_SIZE} has
been set to a different default by your administrator on your
submission machine.

This option only applies to standard-universe jobs.

%%%%%%%%%%%%%%%%%%%
%% buffer_block_size
%%%%%%%%%%%%%%%%%%%

\item[buffer\_block\_size $=$ $<$bytes-in-block$>$]
When buffering is enabled, Condor will attempt to consolidate small read
and write operations into large blocks.  This option specifies the block
size Condor will use.  A very small block size may actually decrease I/O performance.
The block size should definitely be larger than any of the I/O operations
your program performs.

The buffer block size and its effect on throughput may be viewed with
the \verb@-io@ option to \Condor{status}. 
In this version of Condor, the default buffer block size is 32 KB,
unless the configuration file \Macro{DEFAULT\_IO\_BUFFER\_BLOCK\_SIZE}
has been set to a different default by your administrator on your
submission machine.

This option only applies to standard-universe jobs.

%%%%%%%%%%%%%%%%%%%
%% file_remaps
%%%%%%%%%%%%%%%%%%%

\item[file\_remaps $=$ $<$ `` name $=$ newname ; name2 $=$ newname2 ... ''$>$ ]
Directs Condor to use a new filename in place of an old one.  \Arg{name}
describes a filename that your job may attempt to open, and \Arg{newname}
describes the filename it should be replaced with.
\Arg{newname} may include an optional leading
access specifier, \verb@local:@ or \verb@remote:@.  If left unspecified,
the default access specifier is \verb@remote:@.  Multiple remaps can be 
specified by separating each with a semicolon.

If you wish to remap file names that contain equals signs or semicolons,
these special chracaters may be escaped with a backslash.

This option only applies to standard-universe jobs.

\begin{description}
\item[Example One:]
Suppose that your job reads a file named \verb@dataset.1@.  To instruct Condor
to force your job to read \verb@other.dataset@ instead, 
add this to the submit file:
\begin{verbatim}
file_remaps = "dataset.1=other.dataset"
\end{verbatim}
\item[Example Two:]
Suppose that your run many jobs which all read in the same large file,
called \verb@very.big@.  If this file can be found in the same place on
a local disk in every machine in the pool,
(say \verb@/bigdisk/bigfile@,) you can
instruct Condor of this fact by remapping \verb@very.big@ to
\verb@/bigdisk/bigfile@ and specifying that the file is to be read locally,
which will be much faster than reading over the network.
\begin{verbatim}
file_remaps = "very.big = local:/bigdisk/bigfile"
\end{verbatim}
\item[Example Three:]
Several remaps can be applied at once by separating each with a semicolon.
\begin{verbatim}
file_remaps = "very.big = local:/bigdisk/bigfile ; dataset.1 = other.dataset"
\end{verbatim}
\end{description}

%%%%%%%%%%%%%%%%%%%
%% rendezvousdir
%%%%%%%%%%%%%%%%%%%

\item[rendezvousdir = $<$directory-path$>$] Used to specify the 
shared-filesystem directory to be used for filesystem authentication
when submitting to a remote scheduler.  Should be a path to a preexisting 
directory.  

%%%%%%%%%%%%%%%%%%%
%% x509directory
%%%%%%%%%%%%%%%%%%%

\item[x509directory = $<$directory-path$>$] Used to specify the directory 
which contains the certificate, private key, and trusted certificate directory
for GSS authentication. If this attribute is set, the environment variables 
X509\_USER\_KEY, X509\_USER\_CERT, and X509\_CERT\_DIR are exported with 
default values. See section~\ref{sec:X509-Authentication} for more info.

%%%%%%%%%%%%%%%%%%%
%% x509userproxy
%%%%%%%%%%%%%%%%%%%

\item[x509userproxy = $<$full-pathname$>$] Used to override the default
pathname for X509 user certificates. The default location for X509 proxies
is the /tmp directory, which is generally a local filesystem. Setting
this value would allow Condor to access the proxy in a shared filesystem
(e.g., AFS).  See section~\ref{sec:X509-Authentication} for more info.

%%%%%%%%%%%%%%%%%%%
%% globusscheduler
%%%%%%%%%%%%%%%%%%%

\item[globusscheduler = $<$scheduler-name$>$] Used to specify the 
Globus resource to which the job should be submitted. More than one scheduler
can be submitted to, simply place a \Arg{queue} command after each instance
of globusscheduler. Each instance should be a valid Globus scheduler, using
either the full Globus contact string or the host/scheduler format shown below:
\begin{description}
\item[Example:]
To submit to the LSF scheduler of the Globus gatekeeper on lego at 
Boston University:
\begin{verbatim}
...
GlobusScheduler = lego.bu.edu/jobmanager-lsf
queue
\end{verbatim}
\end{description}

%%%%%%%%%%%%%%%%%%%
%% globusrsl
%%%%%%%%%%%%%%%%%%%

\item[globusrsl = $<$RSL-string$>$] Used to provide any additional Globus RSL
string attributes which are not covered by regular submit file parameters.

%%%%%%%%%%%%%%%%%%%
% +
%%%%%%%%%%%%%%%%%%%

\item[+$<$attribute$>$ = $<$value$>$] A line which begins with a '+'
(plus) character instructs \Condor{submit} to simply insert the
following \Arg{attribute} into the job ClasssAd with the given 
\Arg{value}. 

%%%%%%%%%%%%%%%%%%%
%% queue
%%%%%%%%%%%%%%%%%%%

\item[queue [number-of-procs] ] Places one or more copies of the job into
the Condor queue. If desired, new \Opt{input}, \Opt{output},
\Opt{error}, \Opt{initialdir}, \Opt{arguments}, \Opt{nice\_user},
\Opt{priority}, \Opt{kill\_sig}, \Opt{coresize}, or \Opt{image\_size}
commands may be issued between \Opt{queue} commands. This is very handy
when submitting multiple runs into one cluster with one submit file; for
example, by issuing an \Opt{initialdir} between each \Opt{queue}
command, each run can work in its own subdirectory. The optional
argument \Arg{number-of-procs} specifies how many times to submit the
job to the queue, and defaults to 1.

\end{description}

In addition to commands, the submit-description file can contain macros
and comments:

\begin{description}

\item[Macros] Parameterless macros in the form of \MacroU{macro\_name}
may be inserted anywhere in condor description files. Macros can be
defined by lines in the form of 
\begin{verbatim} 
        <macro_name> = <string> 
\end{verbatim} 
Two pre-defined macros are supplied by the description file parser. The
\MacroU{Cluster} macro supplies the number of the job cluster, and the
\MacroU{Process} macro supplies the number of the job. These macros are
intended to aid in the specification of input/output files, arguments,
etc., for clusters with lots of jobs, and/or could be used to supply a
Condor process with its own cluster and process numbers on the command
line.  The \MacroU{Process} macro should not be used for PVM jobs.

\item[Comments] Blank lines and lines beginning with a '\#' (pound-sign)
character are ignored by the submit-description file parser. 

\end{description}


\begin{Options}

	\OptItem{\Opt{--}}{Accept the command file from stdin.}
    \OptItem{\Opt{-v}}{Verbose output - display the created job class-ad}

    \OptItem{\OptArg{-n}{schedd\_name}}{Submit to the specified schedd. This option is used when there is more than one schedd running on the submitting machine}

	\OptItem{\OptArg{-r}{schedd\_name}}{Submit to a remote schedd. The jobs
	will be submitted to the schedd on the specified remote host. On Unix
	systems, the Condor administrator for you site must override the default 
	AUTHENTICATION\_METHODS configuration setting to enable remote filesystem 
	(FS\_REMOTE) authentication.}

	\OptItem{\Opt{-d}}{Disable file permission checks.}

\end{Options}

\ExitStatus

\Condor{submit} will exit with a status value of 0 (zero) upon success, and a
non-zero value upon failure.

\Examples

\underline{Example 1}: The below example queues three jobs for
execution by Condor. The first will be given command line arguments of
'15' and '2000', and will write its standard output to 'foo.out1'. The
second will be given command line arguments of '30' and '2000', and will
write its standard output to 'foo.out2'. Similarly the third will have
arguments of '45' and '6000', and will use 'foo.out3' for its standard
output. Standard error output, (if any), from all three programs will
appear in 'foo.error'.

\begin{verbatim}
      ####################
      #
      # Example 1: queueing multiple jobs with differing
      # command line arguments and output files.
      #                                                                      
      ####################                                                   
                                                                         
      Executable     = foo                                                   
                                                                         
      Arguments      = 15 2000                                               
      Output  = foo.out1                                                     
      Error   = foo.err1
      Queue                                                                  
                                                                         
      Arguments      = 30 2000                                               
      Output  = foo.out2                                                     
      Error   = foo.err2
      Queue                                                                  
                                                                         
      Arguments      = 45 6000                                               
      Output  = foo.out3                                                     
      Error   = foo.err3
      Queue                   
\end{verbatim}

\underline{Example 2}: This submit-description file example queues 150
runs of program 'foo' which must have been compiled and linked for
Silicon Graphics workstations running IRIX 6.x. Condor will not attempt
to run the processes on machines which have less than 32 megabytes of
physical memory, and will run them on machines which have at least 64
megabytes if such machines are available. Stdin, stdout, and stderr will
refer to ``in.0'', ``out.0'', and ``err.0'' for the first run of this program
(process 0). Stdin, stdout, and stderr will refer to ``in.1'', ``out.1'',
and ``err.1'' for process 1, and so forth. A log file containing entries
about where/when Condor runs, checkpoints, and migrates processes in this
cluster will be written into file ``foo.log''.

\begin{verbatim}
      ####################                                                    
      #                                                                       
      # Example 2: Show off some fancy features including                            
      # use of pre-defined macros and logging.                                
      #                                                                       
      ####################                                                    
                                                                          
      Executable     = foo                                                    
      Requirements   = Memory >= 32 && OpSys == "IRIX6" && Arch =="SGI"     
      Rank           = Memory >= 64
      Image_Size     = 28 Meg                                                 
                                                                          
      Error   = err.$(Process)                                                
      Input   = in.$(Process)                                                 
      Output  = out.$(Process)                                                
      Log = foo.log                                                                       
                                                                          
      Queue 150
\end{verbatim}

\GenRem
\begin{itemize}

\item For security reasons, Condor will refuse to run any jobs submitted
by user root (UID = 0) or by a user whose default group is group wheel
(GID = 0). Jobs submitted by user root or a user with a default group of
wheel will appear to sit forever in the queue in an idle state. 

\item All pathnames specified in the submit-description file must be
less than 256 characters in length, and command line arguments must be
less than 4096 characters in length; otherwise, \Condor{submit} gives a
warning message but the jobs will not execute properly. 

\item Somewhat understandably, behavior gets bizzare if the user makes
the mistake of requesting multiple Condor jobs to write to the
same file, and/or if the user alters any files that need to be accessed
by a Condor job which is still in the queue (i.e. compressing of data or
output files before a Condor job has completed is a common mistake).

\item To disable checkpointing for Standard Universe jobs, include the
line:
\begin{verbatim}
      +WantCheckpoint = False
\end{verbatim}
in the submit-description file before the queue command(s).
\end{itemize}

\SeeAlso
Condor User Manual

\end{ManPage}

