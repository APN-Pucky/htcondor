\begin{ManPage}{\label{man-condor-cod}\Condor{cod}}{1}
{manage COD machines and jobs}

\Synopsis
\SynProg{\Condor{cod}}
\ToolArgsBase

\SynProg{\Condor{cod}}
\Arg{request}
\ToolLocate
\Lbr \oOpt{-help $|$ -version} $|$ \oOpt{-debug $|$ -timeout N $|$ -classad file} \Rbr
\oOpt{-requirements expr} \oOpt{-lease N}

\SynProg{\Condor{cod}}
\Arg{release} \OptArg{-id}{ClaimID}
\Lbr \oOpt{-help $|$ -version} $|$ \oOpt{-debug $|$ -timeout N $|$ -classad file} \Rbr
\oOpt{-fast}

\SynProg{\Condor{cod}}
\Arg{activate} \OptArg{-id}{ClaimID}
\Lbr \oOpt{-help $|$ -version} $|$ \oOpt{-debug $|$ -timeout N $|$ -classad file} \Rbr
\oOpt{-keyword string $|$ -jobad filename $|$ -cluster N $|$ -proc N $|$ -requirements expr}

\SynProg{\Condor{cod}}
\Arg{deactivate} \OptArg{-id}{ClaimID}
\Lbr \oOpt{-help $|$ -version} $|$ \oOpt{-debug $|$ -timeout N $|$ -classad file} \Rbr
\oOpt{-fast}

\SynProg{\Condor{cod}}
\Arg{suspend} \OptArg{-id}{ClaimID}
\Lbr \oOpt{-help $|$ -version} $|$ \oOpt{-debug $|$ -timeout N $|$ -classad file} \Rbr

\SynProg{\Condor{cod}}
\Arg{renew} \OptArg{-id}{ClaimID}
\Lbr \oOpt{-help $|$ -version} $|$ \oOpt{-debug $|$ -timeout N $|$ -classad file} \Rbr

\SynProg{\Condor{cod}}
\Arg{resume} \OptArg{-id}{ClaimID}
\Lbr \oOpt{-help $|$ -version} $|$ \oOpt{-debug $|$ -timeout N $|$ -classad file} \Rbr

\SynProg{\Condor{cod}}
\Arg{delegate\_proxy} \OptArg{-id}{ClaimID}
\Lbr \oOpt{-help $|$ -version} $|$ \oOpt{-debug $|$ -timeout N $|$ -classad file} \Rbr
\oOptArg{-x509proxy}{ProxyFile}

\index{HTCondor commands!condor\_cod}
\index{condor\_cod command}

\Description

\Condor{cod} issues commands that manage and use COD claims on
machines, given proper authorization.

Instead of specifying an argument of
\Arg{request}, \Arg{release}, \Arg{activate}, \Arg{deactivate}, \Arg{suspend},
\Arg{renew}, or \Arg{resume},
the user may invoke the \Condor{cod} tool by appending an
underscore followed by one of these arguments.
As an example, the following two commands are equivalent:
\begin{verbatim}
    condor_cod release -id "<128.105.121.21:49973>#1073352104#4"
\end{verbatim}
\begin{verbatim}
    condor_cod_release -id "<128.105.121.21:49973>#1073352104#4"
\end{verbatim}
To make these extended-name commands work,
hard link  the extended name to the \Condor{cod} executable.
For example on a Unix machine:
\begin{verbatim}
ln condor_cod_request condor_cod
\end{verbatim}

The \Arg{request} argument gives a claim ID, and the other 
commands (\Arg{release}, \Arg{activate}, \Arg{deactivate}, \Arg{suspend},
and \Arg{resume}) use the claim ID.
The claim ID is given as the last line of output for a \Arg{request},
and the output appears of the form:
\footnotesize
\begin{verbatim}
ID of new claim is: "<a.b.c.d:portnumber>#x#y"
\end{verbatim}
\normalsize
An actual example of this line of output is 
\footnotesize
\begin{verbatim}
ID of new claim is: "<128.105.121.21:49973>#1073352104#4"
\end{verbatim}
\normalsize

Also see
section~\ref{sec:cod}
for more a complete description of COD.

\begin{Options}
  \ToolArgsBaseDesc
  \ToolLocateDesc
  \OptItem{\OptArg{-lease}{N}}{For the \Opt{request} of a new claim,
    automatically release the claim after \Arg{N} seconds.}
  \OptItem{\Opt{request}}{Create a new COD claim}
  \OptItem{\Opt{release}}{Relinquish a claim and kill any running job}
  \OptItem{\Opt{activate}}{Start a job on a given claim}
  \OptItem{\Opt{deactivate}}{Kill the current job, but keep the claim}
  \OptItem{\Opt{suspend}}{Suspend the job on a given claim}
  \OptItem{\Opt{renew}}{Renew the lease to the COD claim}
  \OptItem{\Opt{resume}}{Resume the job on a given claim}
  \OptItem{\Opt{delegate\_proxy}}{Delegate an X509 proxy for the given claim}
\end{Options}

\GenRem

\Examples

\ExitStatus

\Condor{cod} will exit with a status value of 0 (zero) upon success,
and it will exit with the value 1 (one) upon failure.

\end{ManPage}
