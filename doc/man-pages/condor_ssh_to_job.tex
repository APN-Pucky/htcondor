\begin{ManPage}
{\label{man-condor-ssh-to-job}\Condor{ssh\_to\_job}}{1}
{create an ssh session to a running job}
\Synopsis \SynProg{\Condor{ssh\_to\_job}}
\oOpt{-help}

\SynProg{\Condor{ssh\_to\_job}}
\ToolDebugOption
\oOptArg{-name}{schedd-name}
\oOptArg{-pool}{pool-name}
\oOptArg{-ssh}{ssh-command}
\oOptArg{-keygen-options}{ssh-keygen-options}
\oOptArg{-shells}{shell1,shell2,...}
\oOpt{-auto-retry}
\oOpt{-remove-on-interrupt}
\Arg{cluster \Bar cluster.process \Bar cluster.process.node }
\oArg{remote-command}

\index{HTCondor commands!condor\_ssh\_to\_job}
\index{condor\_ssh\_to\_job command}

\Description

\Condor{ssh\_to\_job} creates an \Prog{ssh} session to a running job.
The job is specified with the argument.
If only the job \Arg{cluster} id is given,
then the job \Arg{process} id defaults to the value 0.

It is available in Unix HTCondor distributions, 
and it works for vanilla, vm, java, local, and parallel universe jobs.
The user must be the owner of the job or must be a queue super user, 
and both the \Condor{schedd} and \Condor{starter} daemons
must allow \Condor{ssh\_to\_job} access.
If no \Arg{remote-command} is specified, an interactive shell is created.
An alternate \Prog{ssh} program such as \Prog{sftp} may be specified,
using the \Opt{-ssh} option for uploading and downloading files.

The remote command or shell runs with the same user id as the running job,
and it is initialized with the same working directory.
The environment is initialized to be the same as that of the job,
plus any changes made by the shell setup scripts
and any environment variables passed by the \Prog{ssh} client.
In addition, the environment variable
\Env{\_CONDOR\_JOB\_PIDS} is defined.  
It is a space-separated list of PIDs associated with the job.
At a minimum, the list will contain the
PID of the process started when the job was launched,
and it will be the first item in the list.
It may contain additional PIDs of other processes that the job has created.

The \Prog{ssh} session and all processes it creates are treated by HTCondor as
though they are processes belonging to the job.
If the slot is preempted or suspended,
the \Prog{ssh} session is killed or suspended along with the job.
If the job exits before the \Prog{ssh} session finishes,
the slot remains in the Claimed Busy state and is treated as though not
all job processes have exited until all \Prog{ssh} sessions are closed.
Multiple \Prog{ssh} sessions may be created to the same job at the
same time.  Resource consumption of the \Prog{sshd} process and all processes
spawned by it are monitored by the \Condor{starter} as though these
processes belong to the job, so any policies such as \MacroNI{PREEMPT} that
enforce a limit on resource consumption also take into account resources
consumed by the \Prog{ssh} session.

\Condor{ssh\_to\_job} stores ssh keys in temporary files within a newly
created and uniquely named directory.
The newly created directory will be within the directory defined
by the environment variable \Env{TMPDIR}. 
When the ssh session is finished, this directory and the ssh keys
contained within it are removed.

See section~\ref{sec:Config-ssh-to-job}
for details of the configuration variables related to \Condor{ssh\_to\_job}.

An \Prog{ssh} session works by first authenticating and authorizing
a secure connection between \Condor{ssh\_to\_job} and the \Condor{starter}
daemon, using HTCondor protocols.
The \Condor{starter} generates an ssh key
pair and sends it securely to \Condor{ssh\_to\_job}.
Then the \Condor{starter} spawns \Prog{sshd} in inetd mode with its
stdin and stdout attached to the TCP connection from \Condor{ssh\_to\_job}.
\Condor{ssh\_to\_job} acts as a proxy
for the \Prog{ssh} client to communicate with \Prog{sshd},
using the existing connection authorized by HTCondor.
\emph{At no point is \Prog{sshd}
listening on the network for connections or running with any privileges
other than that of the user identity running the job.}
If CCB is being used to enable connectivity to the execute node from
outside of a firewall or private network, \Condor{ssh\_to\_job} is
able to make use of CCB in order to form the \Prog{ssh} connection.

The login shell of the user id running the job is used to run the
requested command, \Prog{sshd} subsystem, or interactive shell.  This
is hard-coded behavior in \Prog{OpenSSH} and cannot be overridden by
configuration.  This means that \Condor{ssh\_to\_job} access is
effectively disabled if the login shell disables access,
as in the example programs
\Prog{/bin/true} and \Prog{/sbin/nologin}.

\Condor{ssh\_to\_job} is intended to work with \Prog{OpenSSH}
as installed in typical environments.
It does not work on Windows platforms.
If the \Prog{ssh} programs are installed in non-standard locations,
then the paths to
these programs will need to be customized within the HTCondor configuration.
Versions of \Prog{ssh} other than \Prog{OpenSSH} may work,
but they will likely
require additional configuration of command-line arguments,
changes to the \Prog{sshd} configuration template file,
and possibly modification of the
\verb@$(LIBEXEC)/condor_ssh_to_job_sshd_setup@ script 
used by the \Condor{starter} to set up \Prog{sshd}.


\begin{Options}
  \OptItem{\Opt{-help}}{ Display brief usage information and exit. }
  \ToolDebugDesc
  \OptItem{\OptArg{-name}{schedd-name}}{Specify an alternate \Condor{schedd},
    if the default (local) one is not desired.}
  \OptItem{\OptArg{-pool}{pool-name}}{Specify an alternate HTCondor pool,
    if the default one is not desired.}
  \OptItem{\OptArg{-ssh}{ssh-command}}{Specify an alternate \Prog{ssh} program
    to run in place of \Prog{ssh}, for example \Prog{sftp} or \Prog{scp}.
    Additional arguments are specified as \Arg{ssh-command}.
    Since the arguments are delimited by spaces,
    place double quote marks around the whole command,
    to prevent the shell from splitting it into multiple arguments
    to \Condor{ssh\_to\_job}.
    If any arguments must contain spaces, enclose them within single quotes.}
  \OptItem{\OptArg{-keygen-options}{ssh-keygen-options}}{Specify additional
    arguments to the \Prog{ssh\_keygen} program,
    for creating the ssh key that is used for the duration of the session.
    For example, a different number of bits could be used,
    or a different key type than the default.}
  \OptItem{\OptArg{-shells}{shell1,shell2,...}}{Specify a comma-separated list
    of shells to attempt to launch.
    If the first shell does not exist on the remote machine,
    then the following ones in the list will be tried.
    If none of the specified shells can be found, \Prog{/bin/sh} is used
    by default.
    If this option is not specified, it defaults to the environment variable
    \Env{SHELL} from within the \Condor{ssh\_to\_job} environment.}
  \OptItem{\Opt{-auto-retry}}{Specifies that if the job is not yet running,
    \Condor{ssh\_to\_job} should keep trying periodically until it succeeds 
    or encounters some other error.}
  \OptItem{\Opt{-remove-on-interrupt}}{If specified, attempt to remove the job
    from the queue if \Condor{ssh\_to\_job} is interrupted via a ctrl-c or
    otherwise terminated abnormally.}
  \OptItem{\Opt{-X}}{Enable X11 forwarding.}
\end{Options}

\Examples
\footnotesize
\begin{verbatim}
% condor_ssh_to_job 32.0
Welcome to slot2@tonic.cs.wisc.edu!
Your condor job is running with pid(s) 65881.
% gdb -p 65881
(gdb) where
...
% logout
Connection to condor-job.tonic.cs.wisc.edu closed.
\end{verbatim}
\normalsize

To upload or download files interactively with \Prog{sftp}:
\footnotesize
\begin{verbatim}
% condor_ssh_to_job -ssh sftp 32.0
Connecting to condor-job.tonic.cs.wisc.edu...
sftp> ls
...
sftp> get outputfile.dat
\end{verbatim}
\normalsize

This example shows downloading a file from the job with \Prog{scp}.
The string "remote" is used in place of a host name in this example.
It is not necessary to insert the correct remote host name, or even
a valid one, because the connection to the job is created automatically.
Therefore, the placeholder string "remote" is perfectly fine.
\footnotesize
\begin{verbatim}
% condor_ssh_to_job -ssh scp 32 remote:outputfile.dat .
\end{verbatim}
\normalsize

This example uses \Condor{ssh\_to\_job} to accomplish the
task of running \Prog{rsync} to synchronize a local file with
a remote file in the job's working directory.
Job id 32.0 is used in place of a host name in this example.
This causes \Prog{rsync} to insert the expected job id in the
arguments to \Condor{ssh\_to\_job}.
\footnotesize
\begin{verbatim}
% rsync -v -e "condor_ssh_to_job" 32.0:outputfile.dat .
\end{verbatim}
\normalsize

Note that \Condor{ssh\_to\_job} was added to HTCondor in version 7.3.  
If  one uses \Condor{ssh\_to\_job} to connect to a job
on an execute machine running a version of HTCondor older than the 7.3 series,
the command will fail with the error message
\begin{verbatim}
Failed to send CREATE_JOB_OWNER_SEC_SESSION to starter
\end{verbatim}

\ExitStatus

\Condor{ssh\_to\_job} will exit with a non-zero status value if it fails
to set up an ssh session.  If it succeeds, it will exit with the
status value of the remote command or shell.

\end{ManPage}
