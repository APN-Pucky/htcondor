\begin{ManPage}{\label{man-condor-preen}\Condor{preen}}{1}
{remove extraneous files from Condor directories}
\Synopsis \SynProg{\Condor{preen}}
\oOpt{-mail}
\oOpt{-remove}
\oOpt{-verbose}

\index{Condor commands!condor\_preen}
\index{condor\_preen command}

\Description 

\Condor{preen} examines the directories belonging to Condor, 
and removes extraneous files and directories which may be left over from
Condor processes which terminated abnormally either due to internal errors or
a system crash. The directories checked are the LOG, EXECUTE, and SPOOL
directories as defined in the Condor configuration files. \Condor{preen} is
intended to be run as user root (or user condor) periodically as a backup
method to ensure reasonable file system cleanliness in the face of
errors. This is done automatically by default by the \Condor{master}. 
It may also be explicitly invoked on an as needed basis.

When \Condor{preen} cleans the SPOOL directory, it always leaves
behind the files specified in the \Macro{VALID\_SPOOL\_FILES} list in
your config file.
For the log directory, the only files removed or reported are those
listed in the  \Macro{INVALID\_LOG\_FILES} list.
The reason for this difference is that, in general, you want to leave
all files in the LOG directory alone, with a few exceptions (namely,
core files).
\Condor{preen} still works if you supply a \Macro{VALID\_LOG\_FILES}
list instead, but this usage is deprecated.
There are new log files for different things introduced all the time,
and you wouldn't want to have to keep updating the list of files to
leave alone in the LOG directory.
For example, the SMP startd can spawn an arbitrary number of
\Condor{starter} processes, each with its own log file.  
On the other hand, there are only a small, fixed number of files in
the SPOOL directory that the \Condor{schedd} needs to keep around, so
it is easier to specify the files you want to keep instead of the ones
you want to get rid of.

\begin{Options}

	\OptItem{\Opt{-mail}}{Send mail to the \Macro{PREEN\_ADMIN} as
	defined in the Condor configuration files instead of writing
	to the standard output}

	\OptItem{\Opt{-remove}}{Remove the offending files and directories
	rather than just reporting on them}

	\OptItem{\Opt{-verbose}}{List all files found in the Condor
	directories, even those which are not considered extraneous}

\end{Options}

\ExitStatus

\Condor{preen} will exit with a status value of 0 (zero) upon success,
and it will exit with the value 1 (one) upon failure.

\end{ManPage}
