\begin{ManPage}{\label{man-condor-config-val}\Condor{config\_val}}{1}
{Query or set a given HTCondor configuration variable}
\index{HTCondor commands!condor\_config\_val}
\index{condor\_config\_val command}

\Synopsis \SynProg{\Condor{config\_val}}
\Arg{<help option>}

\SynProg{\Condor{config\_val}}
\oOpt{<location options>}
\Arg{<edit option>}

\SynProg{\Condor{config\_val}}
\oOpt{<location options>}
\oOpt{<view options>}
\Arg{vars}

\Description

\Condor{config\_val} can be used to quickly see what the current
HTCondor configuration is on any given machine.  
Given a space separated set of
configuration variables with the \Arg{vars} argument,
\Condor{config\_val} will report what each of these
variables is currently set to.  If a given variable is not defined,
\Condor{config\_val} will halt on that variable, and report that it is
not defined.  By default, \Condor{config\_val} looks in the local
machine's configuration files in order to evaluate the variables.
Variables and values may instead be queried from a daemon specified
using a \Opt{location option}.

\Term{Raw} output of \Condor{config\_val} displays the string used to
define the configuration variable.
This is what is on the right hand side of the equals sign (\Expr{=})
in a configuration file for a variable.
The default output is an \Term{expanded} one.
Expanded output recursively replaces any macros within the raw definition
of a variable with the macro's raw definition.

Each daemon remembers settings made by a successful invocation
of \Condor{config\_val}.  
The configuration \emph{file} is not modified.  

\Condor{config\_val} can be used to persistently set or unset 
configuration variables for a specific daemon on a given machine
using a \Arg{-set} or \Arg{-unset} \Opt{edit option}.
Persistent settings remain when the daemon is restarted.  
Configuration variables for a specific daemon on a given machine
may be set or unset for the time period that the daemon continues to run
using a \Arg{-rset} or \Arg{-runset} \Opt{edit option}.
These runtime settings are lost when the daemon is restarted.  
Any changes made will not take effect until \Condor{reconfig} is invoked.

In general, modifying a host's configuration with
\Condor{config\_val}  
requires the \DCPerm{CONFIG} access level, which is disabled on all
hosts by default.
Administrators have more
fine-grained control over which access levels can modify which
settings.
See section~\ref{sec:Config-Security} on
page~\pageref{sec:Config-Security} for more details on security settings.
Further, security considerations require proper settings of
configuration variables
\Macro{SETTABLE\_ATTRS\Dots} (see \ref{param:SettableAttrs}),
\Macro{ENABLE\_PERSISTENT\_CONFIG} (see \ref{param:EnablePersistentConfig}),
and \Macro{HOSTALLOW\Dots} (see \ref{param:HostAllow})
in order to use \Condor{config\_val} to change any configuration variable.

It is generally wise to test a new configuration on a single
machine to ensure that no syntax or other errors in the
configuration have been made before the reconfiguration of many machines.  
Having bad syntax or invalid configuration settings is a fatal error
for HTCondor daemons, and they will exit.
It is far better to discover such a problem on a single machine than to
cause all the HTCondor daemons in the pool to exit.
\Condor{config\_val} can help with this type of testing.

\begin{Options}
  \OptItem{\Opt{-help}} {(help option) 
    Print usage information and exit.
   }
  \OptItem{\Opt{-version}} {(help option) 
    Print the HTCondor version information and exit.
   }
  \OptItem{\OptArg{-set}{"var = value"}}{(edit option) 
    Sets one or more persistent configuration file 
    variables.  The new value remains if the daemon is restarted.
    One or more variables can be set; the syntax requires double quote marks
    to identify the pairing of variable name to value, and to permit spaces. 
   }
  \OptItem{\OptArg{-unset}{var}}{(edit option)
    Each of the persistent configuration variables listed reverts to
    its previous value.
   }
  \OptItem{\OptArg{-rset}{"var = value"}}{(edit option) 
    Sets one or more configuration file variables.  
    The new value remains as long as the daemon continues running.
    One or more variables can be set; the syntax requires double quote marks
    to identify the pairing of variable name to value, and to permit spaces. 
   }
  \OptItem{\OptArg{-runset}{var}}{(edit option) 
    Each of the configuration variables listed reverts to
    its previous value as long as the daemon continues running.
   }
  \OptItem{\Opt{-dump}} {(view option)
    Display the raw value of all \Arg{vars} listed.  
    If no \Arg{vars} are listed, then print all configuration variables and
    their values.
    The \Opt{-expand}, \Opt{-default}, and \Opt{-evaluate} options take
    precedence over this \Opt{-dump} option, such that the output will 
    not be raw.
   }
  \OptItem{\Opt{-default}} {(view option)
    Default values are displayed.
   }
  \OptItem{\Opt{-expand}} {(view option)
    Expanded values are displayed.  This is the default.
   }
  \OptItem{\Opt{-raw}} {(view option)
    Raw values are displayed.
   }
  \OptItem{\Opt{-verbose}} {(view option)
    Display configuration file name and line number where the variable is
    set, along with the raw, expanded, and default values of the variable.
   }
  \OptItem{\Opt{-debug[:<opts>]}} {(view option)
    Send output to \File{stderr},
    overriding a set value of \MacroNI{TOOL\_DEBUG}. 
   }
  \OptItem{\Opt{-evaluate}} {(view option)
    Applied only when a \Opt{location option} specifies a daemon.
    The value of the requested parameter will be evaluated with 
    respect to the ClassAd of that daemon.  
   }
  \OptItem{\Opt{-used}} {(view option)
    Applied only when a \Opt{location option} specifies a daemon.
    Modifies which variables are displayed to only those 
    used by the specified daemon.
   }
  \OptItem{\Opt{-unused}} {(view option)
    Applied only when a \Opt{location option} specifies a daemon.
    Modifies which variables are displayed to only those \emph{not}
    used by the specified daemon.
   }
  \OptItem{\Opt{-config}} {(view option)
    Applied only when the configuration is read from files (the default),
    and \emph{not} when applied to a specific daemon.
    Display the current configuration file that set the variable.
   }
  \OptItem{\OptArg{-writeconfig[:upgrade]}{filename}} {(view option)
    For the configuration read from files (the default),
    write to file \Arg{filename} all configuration variables. Values that are
    the same as internal, compile-time defaults will be preceded by the comment character.
    If the \OptArg{:upgrade} option is specified, then values that are the same as
    the internal, compile-time defaults are omitted. Variables are in the same
    order as the they were read from the original configuration files.
   }
  \OptItem{\Opt{-mixedcase}} {(view option)
    Applied only when the configuration is read from files (the default),
    and \emph{not} when applied to a specific daemon.
    Print variable names with the same letter case used in the 
    variable's definition.
   }
  \OptItem{\OptArg{-local-name}{<name>}} {(view option) 
    Applied only when the configuration is read from files (the default),
    and \emph{not} when applied to a specific daemon.
    Inspect the values of attributes that use local names,
    which is useful to distinguish which daemon when there is more than
    one of the particular daemon running.}
  \OptItem{\OptArg{-subsystem}{<daemon>}} {(view option) 
    Applied only when the configuration is read from files (the default),
    and \emph{not} when applied to a specific daemon.
    Specifies the subsystem or daemon name to query, 
    with a default value of the \Expr{TOOL} subsystem.
   }
  \OptItem{\OptArg{-address}{\Sinful{ip:port}}} {(location option)
    Connect to the given IP address and port number. }
  \OptItem{\OptArg{-pool}{centralmanagerhostname[:portnumber]}}
    {(location option) Use the given central manager and an optional 
    port number to find daemons. }
  \OptItem{\OptArg{-name}{<machine\_name>}} { (location option)
    Query the specified
    machine's \Condor{master} daemon for its configuration. 
    Does not function together with any of the options:
    \Opt{-dump}, \Opt{-config}, or \Opt{-verbose}. }
  \OptItem{\Opt{-master \Bar -schedd \Bar -startd \Bar -collector 
           \Bar -negotiator}}
    {(location option) The specific daemon to query. }
\end{Options}

\ExitStatus

\Condor{config\_val} will exit with a status value of 0 (zero) upon success,
and it will exit with the value 1 (one) upon failure.

\Examples

Here is a set of examples to show a sequence of operations using 
\Condor{config\_val}.
To request the \Condor{schedd} daemon on host perdita
to display the value of the \MacroNI{MAX\_JOBS\_RUNNING} configuration variable:
\footnotesize
\begin{verbatim}
   % condor_config_val -name perdita -schedd MAX_JOBS_RUNNING
   500
\end{verbatim}
\normalsize

To request the \Condor{schedd} daemon on host perdita
to set the value of the \MacroNI{MAX\_JOBS\_RUNNING} configuration variable
to the value 10.
\footnotesize
\begin{verbatim}
   % condor_config_val -name perdita -schedd -set "MAX_JOBS_RUNNING = 10"
   Successfully set configuration "MAX_JOBS_RUNNING = 10" on 
   schedd perdita.cs.wisc.edu <128.105.73.32:52067>.
\end{verbatim}
\normalsize

A command that will implement the change just set in the previous
example.
\footnotesize
\begin{verbatim}
   % condor_reconfig -schedd perdita
   Sent "Reconfig" command to schedd perdita.cs.wisc.edu
\end{verbatim}
\normalsize

A re-check of the configuration variable reflects the change implemented:
\footnotesize
\begin{verbatim}
   % condor_config_val -name perdita -schedd MAX_JOBS_RUNNING
   10
\end{verbatim}
\normalsize

To set the configuration variable \MacroNI{MAX\_JOBS\_RUNNING}
back to what it was before the command to set it to 10:
\footnotesize
\begin{verbatim}
   % condor_config_val -name perdita -schedd -unset MAX_JOBS_RUNNING
   Successfully unset configuration "MAX_JOBS_RUNNING" on 
   schedd perdita.cs.wisc.edu <128.105.73.32:52067>.
\end{verbatim}
\normalsize

A command that will implement the change just set in the previous
example.
\footnotesize
\begin{verbatim}
   % condor_reconfig -schedd perdita
   Sent "Reconfig" command to schedd perdita.cs.wisc.edu
\end{verbatim}
\normalsize

A re-check of the configuration variable reflects that variable
has gone back to is value before initial set of the variable:
\footnotesize
\begin{verbatim}
   % condor_config_val -name perdita -schedd MAX_JOBS_RUNNING
   500
\end{verbatim}
\normalsize

\end{ManPage}
