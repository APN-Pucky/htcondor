\begin{ManPage}{\label{man-condor-config-val}\Condor{config\_val}}{1}
{Query or set a given condor configuration variable}
\Synopsis \SynProg{\Condor{config\_val}}
\oArg{options}
\Opt{variable}
\oOpt{variable \Dots}

\SynProg{\Condor{config\_val}}
\oArg{options}
\OptArg{-set}{string}
\oOpt{string \Dots}

\SynProg{\Condor{config\_val}}
\oArg{options}
\OptArg{-rset}{string}
\oOpt{string \Dots}

\SynProg{\Condor{config\_val}}
\oArg{options}
\OptArg{-unset}{variable}
\oOpt{variable \Dots}

\SynProg{\Condor{config\_val}}
\oArg{options}
\OptArg{-runset}{variable}
\oOpt{variable \Dots}

\SynProg{\Condor{config\_val}}
\oArg{options}
\Opt{-tilde}

\SynProg{\Condor{config\_val}}
\oArg{options}
\Opt{-owner}

\index{Condor commands!condor\_config\_val}
\index{condor\_config\_val command}

\Description

\Condor{config\_val} can be used to quickly see what the current
condor configuration is on any given machine.  Given a list of
variables, \Condor{config\_val} will report what each of these
variables is currently set to.  If a given variable is not defined,
\Condor{config\_val} will halt on that variable, and report that it is
not defined.  By default, \Condor{config\_val} looks in the local
machine's configuration files in order to evaluate the variables.

\Condor{config\_val} can also be used to quickly set configuration
variables for a specific daemon on a given machine.  Each daemon
remembers settings made by \Condor{config\_val}.  The configuration
file is not modified by this command.  Persistent settings remain when
the daemon is restarted.  Runtime settings are lost when the daemon is
restarted.  

\Note The changes will not take effect until you perform a
\Condor{reconfig}.

\Note It is generally wise to test a new configuration on a single
machine to ensure you have no syntax or other errors in the
configuration before you reconfigure many machines.  
Having bad syntax or invalid configuration settings is a fatal error
for Condor daemons, and they will exit.
Far better to discover such a problem on a single machine than to
cause all the Condor daemons in your pool to exit.

\begin{Options}
	\OptItem{\OptArg{-name}{daemon\_name}}{ Query the specified
	daemon for its configuration. }
	\OptItem{\OptArg{-pool}{hostname}}{ Use the given central
	manager to find daemons. }
	\OptItem{\OptArg{-address}{\Sinful{ip:port}}}{ Connect to the
	given ip/port. }
	\OptItem{\Opt{-master \Bar -schedd \Bar -startd \Bar
	-collector \Bar -negotiator}}{The daemon to query (if not
	specified, master is default). }
	\OptItem{\OptArg{-set}{string}} { Set a persistent
	config file entry.
	The string must be a single argument, so you should enclose it
	in double quotes.
	The string must be of the form ``variable = value''.
	\OptItem{\OptArg{-rset}{string}} { Set a runtime
	config file entry.
	See the description for \Opt{-set} for details about the
	string to use. } 
	\OptItem{\OptArg{-unset}{variable}} { Unset a persistent
	config file variable. }
	\OptItem{\OptArg{-runset}{variable}} { Unset a runtime
	config file variable. }
	\OptItem{\Opt{-tilde}}{ Return the path to the Condor home
	directory. }
	\OptItem{\Opt{-owner}}{ Return the owner of the
	\Condor{config\_val} process. }
	\OptItem{\Opt{variable \Dots}}{The variables to query. }
\end{Options}

\Examples

\begin{verbatim}
% condor_config_val -name perdita -schedd MAX_JOBS_RUNNING
500
% condor_config_val -name perdita -schedd -set ``MAX_JOBS_RUNNING = 10''
Successfully set configuration "MAX_JOBS_RUNNING = 10" on 
schedd perdita.cs.wisc.edu <128.105.73.32:52067>.
% condor_reconfig -schedd perdita
Sent "Reconfig" command to schedd perdita.cs.wisc.edu
% condor_config_val -name perdita -schedd MAX_JOBS_RUNNING
10
% condor_config_val -name perdita -schedd -unset MAX_JOBS_RUNNING
Successfully unset configuration "MAX_JOBS_RUNNING" on 
schedd perdita.cs.wisc.edu <128.105.73.32:52067>.
% condor_reconfig -schedd perdita
Sent "Reconfig" command to schedd perdita.cs.wisc.edu
% condor_config_val -name perdita -schedd MAX_JOBS_RUNNING
500
\end{verbatim}

\end{ManPage}
