\begin{ManPage}{\label{man-condor-config-val}\Condor{config\_val}}{1}
{Query or set a given Condor configuration variable}
\Synopsis \SynProg{\Condor{config\_val}}
\oArg{options}
\Opt{variable}
\oOpt{variable \Dots}

\SynProg{\Condor{config\_val}}
\oArg{options}
\OptArg{-set}{string}
\oOpt{string \Dots}

\SynProg{\Condor{config\_val}}
\oArg{options}
\OptArg{-rset}{string}
\oOpt{string \Dots}

\SynProg{\Condor{config\_val}}
\oArg{options}
\OptArg{-unset}{variable}
\oOpt{variable \Dots}

\SynProg{\Condor{config\_val}}
\oArg{options}
\OptArg{-runset}{variable}
\oOpt{variable \Dots}

\SynProg{\Condor{config\_val}}
\oArg{options}
\Opt{-tilde}

\SynProg{\Condor{config\_val}}
\oArg{options}
\Opt{-owner}

\index{Condor commands!condor\_config\_val}
\index{condor\_config\_val command}

\Description

\Condor{config\_val} can be used to quickly see what the current
condor configuration is on any given machine.  Given a list of
variables, \Condor{config\_val} will report what each of these
variables is currently set to.  If a given variable is not defined,
\Condor{config\_val} will halt on that variable, and report that it is
not defined.  By default, \Condor{config\_val} looks in the local
machine's configuration files in order to evaluate the variables.

\Condor{config\_val} can also be used to quickly set configuration
variables for a specific daemon on a given machine.  Each daemon
remembers settings made by \Condor{config\_val}.  The configuration
file is not modified by this command.  Persistent settings remain when
the daemon is restarted.  Runtime settings are lost when the daemon is
restarted.  In general, modifying a host's configuration with
\Condor{config\_val}  
requires the \DCPerm{CONFIG} access level, which is disabled on all
hosts by default.  See section~\ref{sec:Security-Access-Levels} on
page~\pageref{sec:Security-Access-Levels} for more details.
Begining with Condor version 6.3.2, administrators have more
fine-grained control over which access levels can modify which
settings.
See section~\ref{sec:Config-Val-Security} on
page~\pageref{sec:Config-Val-Security} for more details.

\Note The changes will not take effect until you perform a
\Condor{reconfig}.

\Note It is generally wise to test a new configuration on a single
machine to ensure you have no syntax or other errors in the
configuration before you reconfigure many machines.  
Having bad syntax or invalid configuration settings is a fatal error
for Condor daemons, and they will exit.
Far better to discover such a problem on a single machine than to
cause all the Condor daemons in your pool to exit.

\begin{Options}
	\OptItem{\OptArg{-name}{daemon\_name}}{ Query the specified
	daemon for its configuration. }
	\OptItem{\OptArg{-pool}{hostname}}{ Use the given central
	manager to find daemons. }
	\OptItem{\OptArg{-address}{\Sinful{ip:port}}}{ Connect to the
	given ip/port. }
	\OptItem{\Opt{-master \Bar -schedd \Bar -startd \Bar
	-collector \Bar -negotiator}}{The daemon to query (if not
	specified, master is default). }
	\OptItem{\OptArg{-set}{string}} { Set a persistent
	config file entry.
	The string must be a single argument, so you should enclose it
	in double quotes.
	The string must be of the form ``variable = value''. }
	\OptItem{\OptArg{-rset}{string}} { Set a runtime
	config file entry.
	See the description for \Opt{-set} for details about the
	string to use. } 
	\OptItem{\OptArg{-unset}{variable}} { Unset a persistent
	config file variable. }
	\OptItem{\OptArg{-runset}{variable}} { Unset a runtime
	config file variable. }
	\OptItem{\Opt{-tilde}}{ Return the path to the Condor home
	directory. }
	\OptItem{\Opt{-owner}}{ Return the owner of the
	\Condor{config\_val} process. }
	\OptItem{\Opt{variable \Dots}}{The variables to query. }
\end{Options}

\ExitStatus

\Condor{config\_val} will exit with a status value of 0 (zero) upon success,
and it will exit with the value 1 (one) upon failure.

\Examples

To request the schedd daemon on host perdita
to give the value of the
\MacroNI{MAX\_JOBS\_RUNNING}
configuration variable:
\begin{verbatim}
   % condor_config_val -name perdita -schedd MAX_JOBS_RUNNING
   500
\end{verbatim}

To request the schedd daemon on host perdita
to set the value of the 
\MacroNI{MAX\_JOBS\_RUNNING}
configuration variable
to the value 10.
\begin{verbatim}
   % condor_config_val -name perdita -schedd -set "MAX_JOBS_RUNNING = 10"
   Successfully set configuration "MAX_JOBS_RUNNING = 10" on 
   schedd perdita.cs.wisc.edu <128.105.73.32:52067>.
\end{verbatim}

A command that will implement the change just set in the previous
example.
\begin{verbatim}
   % condor_reconfig -schedd perdita
   Sent "Reconfig" command to schedd perdita.cs.wisc.edu
\end{verbatim}

A re-check of the configuration variable reflects the change implemented:
\begin{verbatim}
   % condor_config_val -name perdita -schedd MAX_JOBS_RUNNING
   10
\end{verbatim}

To set the configuration variable \MacroNI{MAX\_JOBS\_RUNNING}
back to what it was before the command to set it to 10:
\begin{verbatim}
   % condor_config_val -name perdita -schedd -unset MAX_JOBS_RUNNING
   Successfully unset configuration "MAX_JOBS_RUNNING" on 
   schedd perdita.cs.wisc.edu <128.105.73.32:52067>.
\end{verbatim}

A command that will implement the change just set in the previous
example.
\begin{verbatim}
   % condor_reconfig -schedd perdita
   Sent "Reconfig" command to schedd perdita.cs.wisc.edu
\end{verbatim}

A re-check of the configuration variable reflects that variable
has gone back to is value before initial set of the variable:
\begin{verbatim}
   % condor_config_val -name perdita -schedd MAX_JOBS_RUNNING
   500
\end{verbatim}

\end{ManPage}
