\begin{ManPage}{\label{man-condor-who}\Condor{who}}{1}
{Display information about owners of jobs and jobs running on an execute machine}
\index{HTCondor commands!condor\_who}
\index{condor\_who command}

\Synopsis \SynProg{\Condor{who}}
\oArg{help options}
\oArg{address options}
\oArg{display options}

\Description
\Condor{who} queries and displays information about the user that
owns the jobs running on a machine.
It is intended to be run on an execute machine.

The options that may be supplied to \Condor{who} belong to three groups:
\begin{itemize}
  \item \textbf{Help options} provide information about the \Condor{who}
  tool.
  \item \textbf{Address options} allow destination specification for query.
  \item \textbf{Display options} control the formatting and which of the 
  queried information to display.
\end{itemize}

At any time, only one \textbf{help option} and one \textbf{address option}
may be specified.  
Any number of \textbf{display options} may be specified.

\Condor{who} obtains its information about jobs by talking to 
one or more \Condor{startd} daemons.
So, \Condor{who} must identify the command port of any \Condor{startd}
daemons.
An \textbf{address option} provides this information.
If \emph{no} \textbf{address option} is given on the command line,
then \Condor{who} searches using this ordering:
\begin{enumerate}
  \item A defined value of the environment variable \Env{CONDOR\_CONFIG}
  specifies the directory where log and address files are to be scanned
  for needed information.
  \item With the aim of finding all \Condor{startd} daemons,
  \Condor{who} utilizes the same algorithm it would using the
  \Opt{-allpids} option.
  The Linux \Prog{ps} or the Windows \Prog{tasklist}
  program obtains all PIDs.
  As Linux \Login{root} or Windows \Login{administrator},
  the Linux \Prog{lsof} or the Windows \Prog{netstat} identifies open
  sockets and from there the PIDs of listen sockets.
  Correlating the two lists of PIDs results in identifying the command ports of
  all \Condor{startd} daemons. 
\end{enumerate}


\begin{Options}
    \OptItem{\Opt{-help}}{(help option) Display usage information}
    \OptItem{\Opt{-daemons}}{(help option) Display information about 
     the daemons running on the specified machine, 
     including the daemon's PID, IP address and command port}
    \OptItem{\Opt{-diagnostic}}{(help option) Display extra information 
      helpful for debugging }
    \OptItem{\Opt{-verbose}}{(help option) Display PIDs and 
      addresses of daemons }
    \OptItem{\OptArg{-address}{hostaddress}} {(address option) 
      Identify the \Condor{startd} host address to query }
    \OptItem{\Opt{-allpids}}{(address option) Query all local
      \Condor{startd} daemons }
    \OptItem{\OptArg{-logdir}{directoryname}} {(address option) 
      Specifies the directory containing log and address files
      that \Condor{who} will scan to search
      for command ports of \Condor{start} daemons to query }
    \OptItem{\OptArg{-pid}{PID}} {(address option) Use the given \Arg{PID}
      to identify the \Condor{startd} daemon to query }
    \OptItem{\Opt{-long}}{(display option) Display entire ClassAds }
    \OptItem{\Opt{-wide}}{(display option) Displays fields without
      truncating them in order to fit screen width }
    \OptItem{\OptArg{-format}{fmt attr}}{(display option) Display attribute
      \Arg{attr} in format \Arg{fmt}.
      To display the attribute or expression the format must contain a single
      \Code{printf(3)}-style conversion specifier.
      Attributes must be from the resource ClassAd.
      Expressions are ClassAd expressions and may
      refer to attributes in the resource ClassAd.
      If the attribute is not present in a given ClassAd and cannot
      be parsed as an expression, then the
      format option will be silently skipped.
      \%r prints the unevaluated, or raw values.
      The conversion specifier must match the type of the
      attribute or expression.
      \%s is suitable for strings such as \Attr{Name},
      \%d for integers such as \Attr{LastHeardFrom},
      and \%f for floating point numbers such as \Attr{LoadAvg}.
      \%v identifies the type of the attribute,
      and then prints the value in an appropriate format.
      \%V identifies the type of the attribute,
      and then prints the value in an appropriate format as it would
      appear in the \Opt{-long} format.
      As an example, strings used with \%V will have quote marks.
      An incorrect format will result in undefined behavior.
      Do not use more than one conversion specifier in a given
      format.  More than one conversion specifier will result
      in undefined behavior.  To output multiple attributes
      repeat the \Opt{-format} option once for each desired
      attribute.
      Like \Code{printf(3)}-style formats, one may include other
      text that will be reproduced directly.
      A format without any conversion specifiers may be specified,
      but an attribute is still required.
      Include a backslash followed by an `n' to specify a line break. }
  \OptItem{\OptArg{-autoformat[:lhVr,tng]}{attr1 [attr2 ...]} 
  or \OptArg{-af[:lhVr,tng]}{attr1 [attr2 ...]}} {
    (display option) Display attribute(s) or expression(s)
    formatted in a default way according to attribute types.  
    This option takes an arbitrary number of attribute names as arguments,
    and prints out their values, 
    with a space between each value and a newline character after 
    the last value.  
    It is like the \Opt{-format} option without format strings.

    It is assumed that no attribute names begin with a dash character,
    so that the next word that begins with dash is the 
    start of the next option.
    The \Opt{autoformat} option may be followed by a colon character
    and formatting qualifiers to deviate the output formatting from
    the default:

    \textbf{l} label each field,

    \textbf{h} print column headings before the first line of output,

    \textbf{V} use \%V rather than \%v for formatting (string values
    are quoted),

    \textbf{r} print "raw", or unevaluated values,

    \textbf{,} add a comma character after each field,

    \textbf{t} add a tab character before each field instead of 
    the default space character,

    \textbf{n} add a newline character after each field,

    \textbf{g} add a newline character between ClassAds, and
    suppress spaces before each field.

    Use \textbf{-af:h} to get tabular values with headings.

    Use \textbf{-af:lrng} to get -long equivalent format.

    The newline and comma characters may \emph{not} be used together.
    The \textbf{l} and \textbf{h} characters may \emph{not} be used
    together.
    }
\end{Options}

\Examples

\underline{Example 1} Sample output from the local machine,
which is running a single HTCondor job.
Note that the output of the \Expr{PROGRAM} field will be truncated
to fit the display, similar to the artificial truncation shown in this 
example output.
\footnotesize
\begin{verbatim}
% condor_who 

OWNER                    CLIENT            SLOT JOB RUNTIME    PID    PROGRAM
smith1@crane.cs.wisc.edu crane.cs.wisc.edu    2 320.0 0+00:00:08 7776 D:\scratch\condor\execut
\end{verbatim}
\normalsize

\underline{Example 2} Verbose sample output.
\footnotesize
\begin{verbatim}
% condor_who -verbose

LOG directory "D:\scratch\condor\master\test/log"

Daemon       PID      Exit       Addr                     Log, Log.Old
------       ---      ----       ----                     ---, -------
Collector    6788                <128.105.136.32:7977> CollectorLog, CollectorLog.old
Credd        8148                <128.105.136.32:9620> CredLog, CredLog.old
Master       5976                <128.105.136.32:64980> MasterLog,
Match MatchLog, MatchLog.old
Negotiator   6600 NegotiatorLog, NegotiatorLog.old
Schedd       6336                <128.105.136.32:64985> SchedLog, SchedLog.old
Shadow ShadowLog,
Slot1 StarterLog.slot1,
Slot2        7272                <128.105.136.32:65026> StarterLog.slot2,
Slot3 StarterLog.slot3,
Slot4 StarterLog.slot4,
SoftKill SoftKillLog,
Startd       7416                <128.105.136.32:64984> StartLog, StartLog.old
Starter StarterLog,
TOOL                                                      TOOLLog,

OWNER                    CLIENT            SLOT JOB RUNTIME    PID    PROGRAM
smith1@crane.cs.wisc.edu crane.cs.wisc.edu    2 320.0 0+00:01:28 7776 D:\scratch\condor\execut
\end{verbatim}
\normalsize

\ExitStatus

\Condor{who} will exit with a status value of 0 (zero) upon success,
and it will exit with the value 1 (one) upon failure.

\end{ManPage}

