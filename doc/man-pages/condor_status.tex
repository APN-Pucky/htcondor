\begin{ManPage}{\Condor{status}}{1}{Display status of the Condor pool}
\label{man-condor-status}
\Synopsis \SynProg{\Condor{status}}
\oArg{help options}
\oArg{query options}
\oArg{display options}
\oArg{custom options}
\oArg{hostname \Dots}

\Description
\Condor{status} is a versatile tool that may be used to monitor and query the 
Condor pool.  The \Condor{status} tool can be used to query resource 
information, submitter information, checkpoint server information, and daemon
master information.  The specific query sent and the resulting information 
display is controlled by the query options supplied.  Queries and display 
formats can also be customized.

The options that may be supplied to \Condor{status} belong to five groups:
\begin{itemize}
	\item \textbf{Help options} provide information about the \Condor{status}
		tool.
	\item \textbf{Query options} control the content and presentation of status
		information.
	\item \textbf{Display options} control the display of the queried 
		information.
	\item \textbf{Custom options} allow the user to customize query and
		display information.
	\item \textbf{Host options} specify specific machines to be queried
\end{itemize}

At any time, only one \Arg{help option}, one \Arg{query option} and one
\Arg{custom option} may be specified.  Any number of \Arg{custom} and \Arg{host
options} may be specified.

\begin{Options}
    \OptItem{\Opt{-help}}{(Help option) Display usage information}
    \OptItem{\Opt{-diagnose}}{(Help option) Print out query ad without 
		performing query}
    \OptItem{\Opt{-avail}}{(Query option) Query \Condor{startd} ads and identify
		resources which are available}
    \OptItem{\Opt{-claimed}}{(Query option) Query \Condor{startd} ads and print 
		information about claimed resources}
    \OptItem{\Opt{-ckptsrvr}}{(Query option) Query \Condor{ckpt\_server} ads
		and display checkpoint server attributes}
    \OptItem{\Opt{-master}}{(Query option) Query \Condor{master} ads and display
		daemon master attributes}
    \OptItem{\OptArg{-pool}{hostname}}{Query the specified central manager. 
		(\Condor{status} queries \Macro{COLLECTOR\_HOST} by default)}
    \OptItem{\Opt{-schedd}}{(Query option) Query \Condor{schedd} ads and display
		attributes}
    \OptItem{\Opt{-server}}{(Query option) Query \Condor{startd} ads and 
		display resource attributes}
    \OptItem{\Opt{-startd}}{(Query option) Query \Condor{startd} ads} 
    \OptItem{\Opt{-state}}{(Query option) Query \Condor{startd} ads and display 
		resource state information}
    \OptItem{\Opt{-submitters}}{(Query option) Query ads sent by submitters and
		display important submitter attributes}
    \OptItem{\Opt{-verbose}}{(Display option) Display entire classads.  Implies
		that totals will not be displayed.}
    \OptItem{\Opt{-long}}{(Display option) Display entire classads 
		(same as \Opt{-verbose})}
    \OptItem{\Opt{-total}}{(Display option) Display totals only}
    \OptItem{\OptArg{-constraint}{const}}{(Custom option) Add constraint 
		expression}
    \OptItem{\OptArg{-format}{fmt attr}}{(Custom option) Register display 
		format and attribute name.  The \Arg{fmt} string has the same format as 
		\texttt{printf(3)}, and \Arg{attr} is the name of the attribute that 
		should be displayed in the specified format.}
\end{Options}

\GenRem
\begin{itemize}
	\item The information obtained from \Condor{startds} and \Condor{schedds} 
	may sometimes appear to be inconsistent.  This is normal since startds and 
	schedds update the Condor manager at different rates, and since there is a
	delay as information propagates through the network and the system.

	\item Note that the \texttt{ActivityTime} in the \texttt{Idle} state is
	\emph{not} the amount of time that the machine has been idle.  See the
	section on \Condor{startd} states in the \emph{Administrator's Manual}
	for more information.
\end{itemize}

\end{ManPage}
