% This is the description of Daemoncore flags that will be included in
% certain programs as options.

%% UNDER CONSTRUCTION and NOT in the man pages, but in cvs! When I get
%% back to it, I'll make it look more like tool_common.tex. -psilord

These next options are for the common daemon subsystem that most Condor
programs use for ease of administration. Unless you know what you are doing
and have a specific need for these command line flags, do not use them.

\begin{Options}

\OptItem{\Opt{-b}}{Causes the daemon to start up in the background. When a
DaemonCore process starts up with this option, disassociates itself from
the terminal and forks itself so that it runs in the background. This is
the default behavior for Condor daemons, and what you get if you specify
no options at all.}

\OptItem{\Opt{-f}}{Causes the daemon to start up in the
foreground. Instead of forking, the daemon just runs in the foreground.

\Note: when the condor_master starts up daemons, it does so with the -f
option since it has already forked a process for the new daemon. That
is why you will see -f in the argument list of all Condor daemons that
the master spawns.}

\OptItem{\Opt{-c filename}}{Causes the daemon to use the specified
filename (you must use a full path) as its global config file. This
overrides the CONDOR_CONFIG environment variable, and the regular
locations that Condor checks for its config file: the condor user's home
directory and \File{/etc/condor/condor_config}.}

\OptItem{\Opt{-p port}}{Causes the daemon to bind to the specified port
for its command socket. The master uses this option to make sure the
\Condor{collector} and \Condor{negotiator} start up on the well-known ports
that the rest of Condor depends on them using.}

\OptItem{\Opt{-t}}{Causes the daemon to print out its error message
to stderr instead of its specified log file. This option forces the -f
option described above.}

\OptItem{\Opt{-v}}{Causes the daemon to print out version information
and exit.}

\OptItem{\Opt{-l directory}}{Overrides the value of LOG as specified
in your config files. Primarily, this option would be used with the
\Condor{kbdd} when it needs to run as the individual user logged into the
machine, instead of running as root. Regular users would not normally have
permission to write files into Condor's log directory. Using this option,
they can override the value of LOG and have the \Condor{kbdd} write its
log file into a directory that the user has permission to write to.}

\OptItem{\Opt{-a string}}{Whatever string you specify is automatically
appended (with a ``.'') to the filename of the log for this daemon, as
specified in your config file.}

\OptItem{\Opt{-pidfile filename}}{Causes the daemon to write out its
PID, or process id number, to the specified file. This file can be used
to help shutdown the daemon without searching through the output of the
\Cmd{ps} command.

Since daemons run with their current working directory set to the
value of LOG, if you don't specify a full path (with a ``/'' to
begin), the file will be left in the log directory.

\OptItem{\Opt{-k filename}}{Causes the daemon to read out a pid from
the specified filename and send a SIGTERM to that process. The daemon
that you start up with ``-k'' will wait until the daemon it is trying
to kill has exited.}

\OptItem{\Opt{-r minutes}}{Causes the daemon to set a timer, upon
expiration of which, sends itself a SIGTERM for graceful shutdown.}

\end{Options}
