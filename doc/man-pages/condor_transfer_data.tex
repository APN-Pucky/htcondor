\begin{ManPage}{\label{man-condor-transfer-data}\Condor{transfer\_data}}{1}
{transfer spooled data}
\Synopsis \SynProg{\Condor{transfer\_data}}
\ToolArgsBase

\SynProg{\Condor{transfer\_data}}
\ToolLocate
\ToolJobs
$|$ \OptArg{-constraint}{expression} \Dots

\SynProg{\Condor{transfer\_data}}
\ToolLocate
\ToolAll

\index{Condor commands!condor\_transfer\_data}
\index{condor\_transfer\_data command}

\Description
\Condor{transfer\_data} causes Condor to transfer spooled
data.
It is meant to be used in conjunction with the \Opt{-spool}
option of \Condor{submit}, as in
\footnotesize
\begin{verbatim}
condor_submit -spool mysubmitfile
\end{verbatim}
\normalsize
Submission of a job with the \Opt{-spool} option causes Condor
to spool all input files, the user log, and any proxy across
a connection to the machine where the \Condor{schedd} daemon
is running.
After spooling these files,
the machine from which the job is submitted may
disconnect from the network
or modify its local copies of the spooled files.

When the job finishes,
the job has \Attr{JobStatus} = 4, meaning that the job has
completed.
The output of the job is spooled,
and
\Condor{transfer\_data} retrieves the output of the completed
job.



\begin{Options}
  \ToolArgsBaseDesc
  \ToolLocateDesc
  \OptItem{\Arg{cluster}}{Transfer spooled data belonging to the
  specified cluster}
  \OptItem{\Arg{cluster.process}}{Transfer spooled data belonging to a
  specific job in the cluster}
  \OptItem{\Arg{user}}{Transfer spooled data belonging to the specified user}
  \OptItem{\OptArg{-constraint}{expression}} {Transfer spooled data for
  jobs which match the job ClassAd expression constraint}
  \OptItem{\Arg{-all}}{Transfer all spooled data}
\end{Options}

\ExitStatus

\Condor{transfer\_data} will exit with a status value of 0 (zero) upon success,
and it will exit with the value 1 (one) upon failure.

\end{ManPage}
