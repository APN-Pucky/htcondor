\begin{ManPage}{\label{man-condor-hold}\Condor{hold}}{1}
{put jobs in the queue in hold state}
\Synopsis \SynProg{\Condor{hold}}
\oOptArg{-n}{schedd\_name}
\oArg{-help}
\oArg{-version}
\oArg{job identifiers}

\index{Condor commands!condor\_hold}
\index{condor\_hold command}

\Description

\Condor{hold} places one or more jobs from the Condor job queue in hold state.
If the \Opt{-n} option is specified, the named \Condor{schedd} is targeted
for processing.  
Otherwise, the local \Condor{schedd} is targeted.
The jobs to be held are identified by one or more job identifiers, as
described below.
For any given job, only the owner of the job or one of the queue super users
(defined by the \MacroNI{QUEUE\_SUPER\_USERS} macro) can place the job on hold.

\begin{Options}
    \OptItem{\Arg{-help}}{Display usage information and exit}
    \OptItem{\Arg{-version}}{Display version information and exit}
    \OptItem{\OptArg{-n}{schedd\_name}}{Target jobs in the queue of the named 
		schedd}
	\OptItem{\Arg{cluster}}{(Job identifier.) Hold all jobs in the specified 
		cluster}
	\OptItem{\Arg{cluster.process}}{(Job identifier.) Hold the specific job 
		in the cluster}
	\OptItem{\Arg{name}}{(Job identifier.) Hold jobs belonging to specified 
		user}
    \OptItem{\Opt{-a}}{(Job identifier.) Hold all the jobs in the queue}
	\OptItem{\Arg{-constraint \emph{constraint}}}{(Job identifier.) Hold 
		jobs matching specified constraint}
\end{Options}

\SeeAlso
\Condor{release} (on page~\pageref{man-condor-release})

\GenRem

To put a PVM universe job on hold, you must put each ``process'' in
the PVM job cluster on hold.  (In the PVM universe, each PVM job is
assigned its own cluster number, and each machine class is assigned a
``process'' number in the job's cluster.)  Putting a subset of the
machine classes for a PVM job on hold is not supported.

\end{ManPage}
