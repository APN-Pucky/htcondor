\begin{ManPage}{\Condor{hold}}{1}{Put jobs in the condor queue in hold state}
\label{man-condor-hold}
\Synopsis \SynProg{\Condor{hold}}
\oOptArg{-n}{schedd\_name}
\Arg{cluster \Bar cluster.process \Bar username \Bar -a}

\index{Condor commands!condor\_hold}
\index{condor\_hold command}

\Description

\Condor{hold} puts one or more jobs in the condor job queue in hold state.  If a
cluster\_id and a process\_id are both specified, \Condor{hold} attempts to
put the specified job on hold. If a cluster\_id is specified
without a process\_id, \Condor{hold} attempts to hold all processes belonging
to the specified cluster. If a username is specified, \Condor{hold} attempts to
hold all jobs belonging to that user. Only the owner of a job, user condor,
or user root can put any given job on hold.

When a job is put on hold, it will not be scheduled to run until it is released.
If the job is runnning when \Condor{hold} is invoked, it will be vacated from the
machine it was running on.

\begin{Options}
    \OptItem{\OptArg{-n}{schedd\_name}}{Hold jobs in the queue of the specified schedd}
    \OptItem{\Opt{-a}}{Hold all the jobs in the queue}
\end{Options}

\SeeAlso
\Condor{release} (on page~\pageref{man-condor-release})

\end{ManPage}
