\begin{ManPage}{\Condor{hold}}{man-condor-hold}{1}
{put jobs in the queue into the hold state}
\index{HTCondor commands!condor\_hold}
\index{condor\_hold command}

\Synopsis \SynProg{\Condor{hold}}
\ToolArgsBase

\SynProg{\Condor{hold}}
\ToolDebugOption
\oOptArg{-reason}{reasonstring}
\oOptArg{-subcode}{number}
\ToolLocate 
\ToolJobs
\Bar{} \OptArg{-constraint}{expression} \Dots

\SynProg{\Condor{hold}}
\ToolDebugOption
\oOptArg{-reason}{reasonstring}
\oOptArg{-subcode}{number}
\ToolLocate 
\ToolAll

\Description

\Condor{hold} places jobs from the HTCondor job queue in
the hold state.
If the \Opt{-name} option is specified, the named \Condor{schedd} is targeted
for processing.  
Otherwise, the local \Condor{schedd} is targeted.
The jobs to be held are identified by one or more job identifiers, as
described below.
For any given job, only the owner of the job or one of the queue super users
(defined by the \MacroNI{QUEUE\_SUPER\_USERS} macro) can place the job on hold.

A job in the hold state remains in the job queue,
but the job will not run until released with \Condor{release}.

A currently running job that is placed in the hold state by \Condor{hold}
is sent a hard kill signal.
For a standard universe job,
this means that the job is removed from the machine without
allowing a checkpoint to be produced first.

\begin{Options}

  \ToolArgsBaseDesc
  \ToolLocateDesc
  \ToolDebugDesc
  \OptItem{\OptArg{-reason}{reasonstring}}{Sets the job ClassAd attribute
    \Attr{HoldReason} to the value given by \Arg{reasonstring}.
    \Arg{reasonstring} will be delimited by double quote marks on the
    command line, if it contains space characters.} 
  \OptItem{\OptArg{-subcode}{number}}{Sets the job ClassAd attribute
    \Attr{HoldReasonSubCode} to the integer value given by \Arg{number}.}
  \OptItem{\Arg{cluster}}{Hold all jobs in the specified cluster}
  \OptItem{\Arg{cluster.process}}{Hold the specific job in the cluster}
  \OptItem{\Arg{user}}{Hold all jobs belonging to specified user}
  \OptItem{\OptArg{-constraint}{expression}} {Hold all jobs which match
    the job ClassAd expression constraint (within quotation
    marks).
    Note that quotation marks must be escaped with the
    backslash characters for most shells.  }
  \OptItem{\Arg{-all}}{Hold all the jobs in the queue}

\end{Options}

\SeeAlso
\Condor{release}

\Examples
To place on hold all jobs
(of the user that issued the \Condor{hold} command)
that are not currently running:
\footnotesize
\begin{verbatim}
% condor_hold -constraint "JobStatus!=2"
\end{verbatim}
\normalsize

Multiple options within the same command cause the union of 
all jobs that meet either (or both) of the options to be placed in the
hold state.
Therefore, the command
\footnotesize
\begin{verbatim}
% condor_hold Mary -constraint "JobStatus!=2"
\end{verbatim}
\normalsize
places all of Mary's queued jobs into the hold state, 
and the constraint holds all queued jobs not currently running.
It also sends a hard kill signal to any of Mary's jobs that are
currently running.
Note that the jobs specified by the constraint will also be
Mary's jobs, if it is Mary that issues this
example \Condor{hold} command.

\ExitStatus

\Condor{hold} will exit with a status value of 0 (zero) upon success,
and it will exit with the value 1 (one) upon failure.

\end{ManPage}
