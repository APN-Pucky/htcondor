\begin{ManPage}{\label{man-condor-hold}\Condor{hold}}{1}
{put jobs in the queue into the hold state}
\Synopsis \SynProg{\Condor{hold}}
\ToolArgsBase
\ToolLocate 
\ToolJobs

\SynProg{\Condor{hold}}
\ToolArgsBase
\ToolLocate 
\ToolAll

\index{Condor commands!condor\_hold}
\index{condor\_hold command}

\Description

\Condor{hold} places one or more jobs from the Condor job queue in hold state.
If the \Opt{-name} option is specified, the named \Condor{schedd} is targeted
for processing.  
Otherwise, the local \Condor{schedd} is targeted.
The jobs to be held are identified by one or more job identifiers, as
described below.
For any given job, only the owner of the job or one of the queue super users
(defined by the \MacroNI{QUEUE\_SUPER\_USERS} macro) can place the job on hold.

\begin{Options}

	\ToolArgsBaseDesc
	\ToolLocateDesc
        \OptItem{\Arg{cluster}}{Hold all jobs in the specified cluster}
        \OptItem{\Arg{cluster.process}}{Hold the specific job in the cluster}
        \OptItem{\Arg{user}}{Hold jobs belonging to specified user}
        \OptItem{\Arg{-all}}{Hold all the jobs in the queue}
        \OptItem{\Arg{-constraint constraint}}{Hold
        jobs matching specified constraint}

\end{Options}

\SeeAlso
\Condor{release} (on page~\pageref{man-condor-release})

\GenRem

To put a PVM universe job on hold, you must put each ``process'' in
the PVM job cluster on hold.  (In the PVM universe, each PVM job is
assigned its own cluster number, and each machine class is assigned a
``process'' number in the job's cluster.)  Putting a subset of the
machine classes for a PVM job on hold is not supported.

\ExitStatus

\Condor{hold} will exit with a status value of 0 (zero) upon success,
and it will exit with the value 1 (one) upon failure.

\end{ManPage}
