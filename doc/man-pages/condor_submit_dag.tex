\begin{ManPage}{\label{man-condor-submit-dag}\Condor{submit\_dag}}{1}
{Manage and queue jobs within a specified DAG for execution on remote machines}
\Synopsis
\SynProg{\Condor{submit\_dag}}
\oOpt{-help | -version}

\SynProg{\Condor{submit\_dag}}
\oOpt{-no\_submit}
\oOpt{-verbose}
\oOpt{-force}
\oOptArg{-maxidle}{NumberOfJobs}
\oOptArg{-maxjobs}{NumberOfJobs}
\oOptArg{-dagman}{DagmanExecutable}
\oOptArg{-maxpre}{NumberOfPREscripts}
\oOptArg{-maxpost}{NumberOfPOSTscripts}
\oOptArg{-notification}{value}
\oOpt{-noeventchecks}
\oOpt{-allowlogerror}
\oOptArg{-r}{schedd\_name}
\oOptArg{-debug}{level}
\oOpt{-usedagdir}
\oOptArg{-outfile\_dir}{directory}
\oOptArg{-config}{ConfigFileName}
\oOptArg{-insert\_sub\_file}{FileName}
\oOptArg{-append}{Command}
\oOptArg{-oldrescue}{0|1}
\oOptArg{-autorescue}{0|1}
\oOptArg{-dorescuefrom}{number}
\oOpt{-allowversionmismatch}
\oOpt{-no\_recurse}
\oOpt{-do\_recurse}
\oOpt{-update\_submit}
\oOpt{-import\_env}
\oOpt{-DumpRescue}
\oOpt{-valgrind}
\Arg{DAGInputFile1}
\oArg{{DAGInputFile2} \Dots {DAGInputFileN} }
\oOptArg{-schedd-daemon-ad-file}{FileName}
\oOptArg{-schedd-address-file}{FileName}

\index{Condor commands!condor\_submit\_dag}
\index{condor\_submit\_dag command}

\Description

\Condor{submit\_dag} is the program for submitting a DAG (directed
acyclic graph) of jobs for execution under Condor.
The program enforces the job dependencies defined
in one or more \Arg{DAGInputFile}s.
Each \Arg{DAGInputFile} contains commands
to direct the submission of jobs implied by the nodes
of a DAG to Condor.
See the Condor User Manual, section~\ref{sec:DAGMan} for a complete
description.

\begin{Options}
  \OptItem{\Opt{-help}}{Display usage information.}
  \OptItem{\Opt{-version}}{Display version information.}
  \OptItem{\Opt{-no\_submit}}{Produce the Condor submit description file
     for DAGMan, but do not submit DAGMan as a Condor job.}
  \OptItem{\Opt{-verbose}}{Cause \Condor{submit\_dag}
     to give verbose error messages.}
  \OptItem{\Opt{-force}}{Require \Condor{submit\_dag} to overwrite
     the files that it produces, if the files already exist.  Note that
     \File{dagman.out} will be appended to, not overwritten.  If
     new-style rescue DAG mode
     is in effect, and any new-style rescue DAGs exist, the \Opt{-force} flag
     will cause them to be renamed, and the original DAG will be run.
     If old-style rescue DAG mode
     is in effect, any existing old-style rescue DAGs will be deleted,
     and the original DAG will be run. Section~\ref{sec:DAGMan-rescue}
     details rescue DAGs.}
  \OptItem{\OptArg{-maxidle}{NumberOfJobs}}{Sets the maximum number of idle
     jobs allowed before \Condor{dagman} stops submitting more jobs.  Once
     idle jobs start to run, \Condor{dagman} will resume submitting jobs.
     \Arg{NumberOfJobs} is a positive integer. If the option is
     omitted, the number of idle jobs is unlimited.   Note that for
     this argument, each individual process within a cluster counts as a 
     job, which is inconsistent with \OptArg{-maxjobs}.}
  \OptItem{\OptArg{-maxjobs}{NumberOfJobs}}{Sets the maximum number of jobs
     within the DAG that will be submitted to Condor at one time.
     \Arg{NumberOfJobs} is a positive integer. If the option is
     omitted, the default number of jobs is unlimited.  Note that for
     this argument, each cluster counts as one job, no matter how many
     individual processes are in the cluster.}
  \OptItem{\OptArg{-dagman}{DagmanExecutable}}{Allows the specification of an 
     alternate \Condor{dagman} executable to be used instead of the one found 
     in the user's path. This must be a fully qualified path.}
  \OptItem{\OptArg{-maxpre}{NumberOfPREscripts}}{Sets the maximum number of PRE
     scripts within the DAG that may be running at one time.
     \Arg{NumberOfPREScripts} is a positive integer. If this option is
     omitted, the default number of PRE scripts is unlimited.}
  \OptItem{\OptArg{-maxpost}{NumberOfPOSTscripts}}{Sets the maximum number of 
     POST scripts within the DAG that may be running at one time.
     \Arg{NumberOfPOSTScripts} is a positive integer. If this option is
     omitted, the default number of POST scripts is unlimited.}
  \OptItem{\OptArg{-notification}{value}}{Sets the e-mail notification
     for DAGMan itself. This information will be used within the Condor
     submit description file for DAGMan. This file is produced by
     \Condor{submit\_dag}. See \Opt{notification}
     within the section of submit description file commands in the
     \Condor{submit} manual page on page~\pageref{man-condor-submit}
     for specification of \Arg{value}.
     }
  \OptItem{\Opt{-noeventchecks}}{This argument is no longer used; it
     is now ignored.  Its functionality is now implemented by
     the \MacroNI{DAGMAN\_ALLOW\_EVENTS} configuration macro
     (see section~\ref{param:DAGManAllowEvents}).}
  \OptItem{\Opt{-allowlogerror}}{This optional argument has
     \Condor{dagman} try to run the specified DAG, even in the case
     of detected errors in the user log specification.
     As of version 7.3.2, this argument has an effect only on
     DAGs containing Stork job nodes.}
  \OptItem{\OptArg{-r}{schedd\_name}}{
     Submit \Condor{dagman} to a remote machine, specifically the
     \Condor{schedd} daemon on that machine.
     The \Condor{dagman} job will not run on the local \Condor{schedd}
     (the submit machine), but on the specified one.
     This is implemented using the \Opt{-remote} option to \Condor{submit}.
     Note that this option does not currently specify input files
     for \Condor{dagman}, nor the individual nodes to be taken along!
     It is assumed that any necessary files will be present on 
     the remote computer,
     possibly via a shared file system between the local computer and 
     the remote computer. 
     It is also necessary that the user has appropriate permissions 
     to submit a job to the remote machine;
     the permissions are the same as those required to use \Condor{submit}'s
     \Opt{-remote} option.
     If other options are desired,
     including transfer of other input files,
     consider using the \Opt{-no\_submit} option,
     modifying the resulting submit file for specific needs,
     and then using \Condor{submit} on that.  }
  \OptItem{\OptArg{-debug}{level}}{
     Passes the the \Arg{level} of debugging output desired to
     \Condor{dagman}.  \Arg{level} is an integer, with values of
     0-7 inclusive, where 7 is the most verbose output.
     A default value of 3 is passed to \Condor{dagman} when not
     specified with this option.
     See the 
     \Condor{dagman} manual page on page~\pageref{man-condor-dagman}
     for detailed descriptions of these values.
     }
  \OptItem{\Opt{-usedagdir}}{This optional argument causes
     \Condor{dagman} to run each specified DAG as if \Condor{submit\_dag}
     had been run in the directory containing that DAG file.  This option
     is most useful when running multiple DAGs in a single \Condor{dagman}.
     Note that the \Opt{-usedagdir} flag must not be used when running
     an old-style rescue DAG (see section~\ref{sec:DAGMan-rescue}).
     }
  \OptItem{\OptArg{-outfile\_dir}{directory}}{
     Specifies the directory in which the \File{.dagman.out} file will
     be written.  The \Argnm{directory} may be specified relative to
     the current working directory as \Condor{submit\_dag} is executed,
     or specified with an absolute path.
     Without this option, the \File{.dagman.out} file is placed in the
     same directory as the first DAG input file listed on the command line.
     }
  \OptItem{\OptArg{-config}{ConfigFileName}}{
     Specifies a configuration file to be used for this DAGMan run.
     Note that the options specified in the configuration file apply to all
     DAGs if multiple DAGs are specified.  Further note that it is a
     fatal error if the configuration file specified by this option
     conflicts with a configuration file specified in any of the DAG
     files, if they specify one.  For more information about how
     \Condor{dagman} configuration files work, see
     section~\ref{sec:DAG-configuration}.
     }
  \OptItem{\OptArg{-insert\_sub\_file}{FileName}}{
    Specifies a file to insert into the \File{.condor.sub} file created
    by \Condor{submit\_dag}.  The specified file must contain only legal
    submit file commands.  Only one file can be inserted.  (If both
    the DAGMAN\_INSERT\_SUB\_FILE configuration variable and
    \Opt{-insert\_sub\_file} are specified, \Opt{-insert\_sub\_file}
    overrides DAGMAN\_INSERT\_SUB\_FILE.)  The specified file is inserted
    into the \File{.condor.sub} file before the Queue command and before
    any commands specified with the \Opt{-append} option.
  }
  \OptItem{\OptArg{-append}{Command}}{
    Specifies a command to append to the \File{.condor.sub} file created
    by \Condor{submit\_dag}.  The specified command is appended to the
    \File{.condor.sub} file immediately before the Queue command.
    Multiple commands are specified by using the \Opt{-append} option
    multiple times. Each new command is given in a separate \Opt{-append}
    option. Commands with spaces in them must be enclosed in double quotes. 
    Commands specified with the \Opt{-append} option are appended to
    the \File{.condor.sub} file \emph{after} commands inserted from a file
    specified by the \Opt{-insert\_sub\_file} option or the
    DAGMAN\_INSERT\_SUB\_FILE configuration variable, so the \Opt{-append}
    command(s) will override commands from the inserted file if the commands
    conflict.
  }
  \OptItem{\OptArg{-oldrescue}{0|1}}{Whether to use "old-style" rescue
    DAG naming (see section~\ref{sec:DAGMan-rescue}) when creating a
    rescue DAG (0 = \Expr{false}, 1 = \Expr{true}).
  }
  \OptItem{\OptArg{-autorescue}{0|1}}{Whether to automatically run the
    newest rescue DAG for the given DAG file, if one exists
    (0 = \Expr{false}, 1 = \Expr{true}).
  }
  \OptItem{\OptArg{-dorescuefrom}{number}}{Forces \Condor{dagman} to
    run the specified rescue DAG number for the given DAG.  A value
    of 0 is the same as not specifying this option.  Specifying a
    non-existent rescue DAG is a fatal error.
  }
  \OptItem{\Opt{-allowversionmismatch}}{This optional argument causes
     \Condor{dagman} to allow a version mismatch between
     \Condor{dagman} itself and the \File{.condor.sub} file produced
     by \Condor{submit\_dag} (or, in other words, between
     \Condor{submit\_dag} and \Condor{dagman}).  WARNING!  This option
     should be used only if absolutely necessary.  Allowing version
     mismatches can cause subtle problems when running DAGs.
     (Note that, starting with version 7.4.0, \Condor{dagman} no longer
     requires an exact version match between itself and the
     \File{.condor.sub} file.  Instead, a "minimum compatible version"
     is defined, and any \File{.condor.sub} file of that version or
     newer is accepted.)
  }
  \OptItem{\Opt{-no\_recurse}}{This optional argument causes
     \Condor{submit\_dag} to \emph{not} run itself recursively on
     nested DAGs (this is now the default; this flag has been kept
     mainly for backwards compatibility).
  }
  \OptItem{\Opt{-do\_recurse}}{This optional argument causes
     \Condor{submit\_dag} to run itself recursively on nested DAGs
     (the default is now that it does \emph{not} run itself recursively;
     instead the \File{.condor.sub} files for nested DAGs are generated
     "lazily" by \Condor{dagman} itself).  (DAG nodes specified with
     the \MacroNI{SUBDAG EXTERNAL} keyword or with submit file names
     ending in \File{.condor.sub} are considered nested DAGs.)
     (See also the
     \MacroNI{DAGMAN\_GENERATE\_SUBDAG\_SUBMITS} configuration
     variable in section~\ref{param:DAGManGenerateSubDagSubmits} for
     more information.)
  }
  \OptItem{\Opt{-update\_submit}}{This optional argument causes an existing
     \File{.condor.sub} file to not be treated as an error; rather, the
     \File{.condor.sub} file will be overwritten, but the existing values
     of \Opt{-maxjobs}, \Opt{-maxidle}, \Opt{-maxpre}, and \Opt{-maxpost}
     will be preserved.
  }
  \OptItem{\Opt{-import\_env}}{This optional argument causes
     \Condor{submit\_dag} to import the current environment into the
     \Opt {environment} command of the \File{.condor.sub} file it
     generates.
  }
  \OptItem{\Opt{-DumpRescue}}{This optional argument tells \Condor{dagman}
     to immediately dump a rescue DAG and then exit, as opposed to
     actually running the DAG.  This feature is mainly intended for
     testing.  The Rescue DAG file is produced whether or not
       there are parse errors reading the original DAG input file.
       The name of the file differs if there was a parse error.
  }
  \OptItem{\Opt{-valgrind}}{This optional argument causes the submit
     description file generated for the submission of \Condor{dagman}
     to be modified. 
     The executable becomes \Prog{valgrind} run on \Condor{dagman},
     with a specific set of arguments intended for testing \Condor{dagman}.
     Note that this argument is intended for testing purposes only.
     Using the \Opt{-valgrind} option without the necessary
     \Prog{valgrind} software installed will cause the DAG to fail.
     If the DAG does run, it will run much more slowly than usual.
  }
  \OptItem{\OptArg{-schedd-daemon-ad-file}{FileName}}{
    Specifies a full path to a daemon ad file dropped by a \Condor{schedd}.
    Therefore this allows submission to a specific scheduler if several
    are available without repeatedly querying the \Condor{collector}.
    The value for this argument defaults to the configuration attribute
    \MacroNI{SCHEDD\_DAEMON\_AD\_FILE}.
  }
   \OptItem{\OptArg{-schedd-address-file}{FileName}}{
    Specifies a full path to an address file dropped by a \Condor{schedd}.
    Therefore this allows submission to a specific scheduler if several
    are available without repeatedly querying the \Condor{collector}.
    The value for this argument defaults to the configuration attribute
    \MacroNI{SCHEDD\_ADDRESS\_FILE}.
  }
  \end{Options}

\SeeAlso
Condor User Manual

\ExitStatus

\Condor{submit\_dag} will exit with a status value of 0 (zero) upon success,
and it will exit with the value 1 (one) upon failure.

\Examples

To run a single DAG:
\begin{verbatim}
% condor_submit_dag diamond.dag
\end{verbatim}

To run a DAG when it has already been run and the output files exist:
\begin{verbatim}
% condor_submit_dag -force diamond.dag
\end{verbatim}

To run a DAG, limiting the number of idle node jobs in the DAG to a
maximum of five:
\begin{verbatim}
% condor_submit_dag -maxidle 5 diamond.dag
\end{verbatim}

To run a DAG, limiting the number of concurrent PRE scripts to 10
and the number of concurrent POST scripts to five:
\begin{verbatim}
% condor_submit_dag -maxpre 10 -maxpost 5 diamond.dag
\end{verbatim}

To run two DAGs, each of which is set up to run in its own directory:
\begin{verbatim}
% condor_submit_dag -usedagdir dag1/diamond1.dag dag2/diamond2.dag
\end{verbatim}

\end{ManPage}

