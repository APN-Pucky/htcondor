\begin{ManPage}{\label{man-condor-submit-dag}\Condor{submit\_dag}}{1}
{Manage and queue jobs within a specified DAG for execution on remote machines}
\Synopsis \SynProg{\Condor{submit\_dag}}
\oOpt{-no\_submit}
\oOpt{-verbose}
\oOpt{-force}
\oOptArg{-maxjobs}{NumberOfJobs}
\oOptArg{-dagman}{DagmanExecutable}
\oOptArg{-maxpre}{NumberOfPREscripts}
\oOptArg{-maxpost}{NumberOfPOSTscripts}
\oOpt{-nopostfail}
\oOptArg{-storklog}{LogFileName}
\oOptArg{-notification}{value}
\oOpt{-noeventchecks}
\oOpt{-allowlogerror}
\oOptArg{-r}{schedd\_name}
\oOptArg{-debug}{level}
\Arg{DAGInputFile1}
\oArg{{DAGInputFile2} \Dots {DAGInputFileN} }


\index{Condor commands!condor\_submit\_dag}
\index{condor\_submit\_dag command}

\Description

\Condor{submit\_dag} is the program for submitting a DAG (directed
acyclic graph) of jobs for execution under Condor.
The program enforces the job dependencies defined
in one or more \Arg{DAGInputFile}s.
Each \Arg{DAGInputFile} contains commands
to direct the submission of jobs implied by the nodes
of a DAG to Condor.
See the Condor User Manual, section~\ref{sec:DAGMan} for a complete
description.

\begin{Options}
  \OptItem{\Opt{-no\_submit}}{Produce the Condor submit description file
     for DAGMan, but do not submit DAGMan as a Condor job.}
  \OptItem{\Opt{-verbose}}{Cause \Condor{submit\_dag}
     to give verbose error messages.}
  \OptItem{\Opt{-force}}{Require \Condor{submit\_dag} to overwrite
     the files that it produces, if the files already exist.}
  \OptItem{\OptArg{-maxjobs}{NumberOfJobs}}{Sets the maximum number of jobs
     within the DAG that will be submitted to Condor at one time.
     \Arg{NumberOfJobs} is a positive integer. If the option is
     omitted, the default number of jobs is unlimited.}
  \OptItem{\OptArg{-dagman}{DagmanExecutable}}{Allows the specification of an 
     alternate \Condor{dagman} executable to be used instead of the one found 
     in the user's path. This must be a fully qualified path.}
  \OptItem{\OptArg{-maxpre}{NumberOfPREscripts}}{Sets the maximum number of PRE
     scripts within the DAG that may be running at one time.
     \Arg{NumberOfPREScripts} is a positive integer. If this option is
     omitted, the default number of PRE scripts is unlimited.}
  \OptItem{\OptArg{-maxpost}{NumberOfPOSTscripts}}{Sets the maximum number of 
     POST scripts within the DAG that may be running at one time.
     \Arg{NumberOfPOSTScripts} is a positive integer. If this option is
     omitted, the default number of POST scripts is unlimited.}
  \OptItem{\Opt{-nopostfail}}{An option applied to all nodes within
     the DAG that prevents the POST script within a node
     from running in the case that the job within the node fails.
     Without this option, POST scripts always run when jobs fail.}
  \OptItem{\OptArg{-log}{LogFileName}}{Deprecated option; do not use.}
  \OptItem{\OptArg{-storklog}{LogFileName}}{Sets the file name for the
     Stork log for data placement jobs.}
  \OptItem{\OptArg{-notification}{value}}{Sets the e-mail notification
     for DAGMan itself. This information will be used within the Condor
     submit description file for DAGMan. This file is produced by
     \Condor{submit\_dag}. See \Opt{notification}
     within the section of submit description file commands in the
     \Condor{submit} manual page on page~\pageref{man-condor-submit}
     for specification of \Arg{value}.
     }
  \OptItem{\Opt{-noeventchecks}}{This argument is deprecated and
     is now ignored.  Its functionality is now implemented by
     the \MacroNI{DAGMAN\_ALLOW\_EVENTS} configuration macro
     (see section~\ref{param:DAGManAllowEvents}).}
  \OptItem{\Opt{-allowlogerror}}{This optional argument has
     \Condor{dagman} try to run the specified DAG, even in the case
     of detected errors in the user log specification. }
  \OptItem{\OptArg{-r}{schedd\_name}}{Submit to a remote schedd. The jobs
    will be submitted to the schedd on the specified remote host. On Unix
    systems, the Condor administrator for you site must override the default
    AUTHENTICATION\_METHODS configuration setting to enable remote file system
    (FS\_REMOTE) authentication.}
  \OptItem{\OptArg{-debug}{level}}{
     Passes the the \Arg{level} of debugging output desired to
     \Condor{dagman}.  \Arg{level} is an integer, with values of
     0-7 inclusive, where 7 is the most verbose output.
     A default value of 3 is passed to \Condor{dagman} when not
     specified with this option.
     }
  \end{Options}

\SeeAlso
Condor User Manual

\ExitStatus

\Condor{submit\_dag} will exit with a status value of 0 (zero) upon success,
and it will exit with the value 1 (one) upon failure.

\end{ManPage}

