\begin{ManPage}{\label{man-condor-advertise}\Condor{advertise}}{1}
{Send a ClassAd to the \Condor{collector} daemon}
\index{HTCondor commands!condor\_advertise}
\index{condor\_advertise command}

\Synopsis \SynProg{\Condor{advertise}}
\ToolArgsBase

\SynProg{\Condor{advertise}}
\oOptArg{-pool}{centralmanagerhostname[:portname]}
\oOpt{-debug}
\oOpt{-tcp}
\oOpt{-udp}
\oOpt{-multiple}
\oArg{update-command \oArg{classad-filename}}

\Description
\Condor{advertise} sends one or more ClassAds to the \Condor{collector}
daemon on the central manager machine.
The optional argument \Arg{update-command} says what daemon type's ClassAd
is to be updated; if it is absent, it assumed to be the update command
corresponding to the type of the (first) ClassAd.
The optional argument \Arg{classad-filename} is the file from which the
ClassAd(s) should be read.
If \Arg{classad-filename} is omitted or is the dash character ('-'),
then the ClassAd(s) are read from standard input.
You must specify \Arg{update-command} if you do not want to read from
standard input.

When \Opt{-multiple} is specified, multiple ClassAds may be published.
Publishing many ClassAds in a single invocation of \Condor{advertise} is
more efficient than invoking \Condor{advertise} once per ClassAd.
The ClassAds are expected to be separated by one or more blank lines.
When \Opt{-multiple} is not specified, blank lines are ignored (for
backward compatibility).  
It is best not to rely on blank lines being ignored,
as this may change in the future.

The \Arg{update-command} may be one of the following strings:
\begin{description}
\item[UPDATE\_STARTD\_AD]
\item[UPDATE\_SCHEDD\_AD]
\item[UPDATE\_MASTER\_AD]
\item[UPDATE\_GATEWAY\_AD]
\item[UPDATE\_CKPT\_SRVR\_AD]
\item[UPDATE\_NEGOTIATOR\_AD]
\item[UPDATE\_HAD\_AD]
\item[UPDATE\_AD\_GENERIC]
\item[UPDATE\_SUBMITTOR\_AD]
\item[UPDATE\_COLLECTOR\_AD]
\item[UPDATE\_LICENSE\_AD]
\item[UPDATE\_STORAGE\_AD]
\end{description}

\Condor{advertise} can also be used to invalidate and delete
ClassAds currently held by the \Condor{collector} daemon.  In this case
the \Arg{update-command} will be one of the following strings:
\begin{description}
\item[INVALIDATE\_STARTD\_ADS]
\item[INVALIDATE\_SCHEDD\_ADS]
\item[INVALIDATE\_MASTER\_ADS]
\item[INVALIDATE\_GATEWAY\_ADS]
\item[INVALIDATE\_CKPT\_SRVR\_ADS]
\item[INVALIDATE\_NEGOTIATOR\_ADS]
\item[INVALIDATE\_HAD\_ADS]
\item[INVALIDATE\_ADS\_GENERIC]
\item[INVALIDATE\_SUBMITTOR\_ADS]
\item[INVALIDATE\_COLLECTOR\_ADS]
\item[INVALIDATE\_LICENSE\_ADS]
\item[INVALIDATE\_STORAGE\_ADS]
\end{description}

For any of these INVALIDATE commands, the ClassAd in the required file
consists of three entries.
The file contents will be similar to:
\begin{verbatim}
MyType = "Query"
TargetType = "Machine"
Requirements = Name == "condor.example.com"
\end{verbatim}
The definition for \Attr{MyType} is always \Expr{Query}.
\Attr{TargetType} is set to the \Attr{MyType} of the ad to be deleted.
This \Attr{MyType} is
\Expr{DaemonMaster} for the \Condor{master} ClassAd,
\Expr{Machine} for the \Condor{startd} ClassAd,
\Expr{Scheduler} for the \Condor{schedd} ClassAd, and
\Expr{Negotiator} for the \Condor{negotiator} ClassAd.
\Attr{Requirements} is an expression evaluated within the context
of ads of \Attr{TargetType}.
When \Attr{Requirements} evaluates to \Expr{True},
the matching ad is invalidated.  
A full example is given below.


\begin{Options}
    \ToolArgsBaseDesc
    \OptItem{\Opt{-debug}}{Print debugging information as the command
            executes.}
    \OptItem{\Opt{-multiple}}{Send more than one ClassAd, where the boundary
        between ClassAds is one or more blank lines.}
    \OptItem{\OptArgnm{-pool}{centralmanagerhostname[:portname]}}
            {Specify a pool by
            giving the central manager's host name and an optional port
	    number.  The default is the
	    \MacroNI{COLLECTOR\_HOST} specified in the configuration file.}
    \OptItem{\Opt{-tcp}}
            {Use TCP for communication. Used by default if
	    \MacroNI{UPDATE\_COLLECTOR\_WITH\_TCP} is true.}
    \OptItem{\Opt{-udp}}{Use UDP for communication.}
\end{Options}

\GenRem
The job and machine ClassAds are regularly updated.
Therefore, the result of \Condor{advertise} is likely to be
overwritten in a very short time.
It is unlikely that either HTCondor users (those who submit jobs)
or administrators will ever have a use for this command.
If it is desired to update or set a ClassAd attribute, the
\Condor{config\_val} command is the proper command to use.

Attributes are defined in Appendix A of the HTCondor manual.


For those administrators who do need \Condor{advertise}, the following
attributes may be included:

\begin{description}
\item[\Attr{DaemonStartTime}]
\item[\Attr{UpdateSequenceNumber}]
\end{description}

If both of the above are included, the \Condor{collector} will
automatically include the following attributes:

\begin{description}
\item[\Attr{UpdatesTotal}]
\item[\Attr{UpdatesLost}]
\item[\Attr{UpdatesSequenced}]
\item[\Attr{UpdatesHistory}] Affected by \Macro{COLLECTOR\_DAEMON\_HISTORY\_SIZE}.
\end{description}

\Examples

Assume that a machine called condor.example.com is turned off,
yet its \Condor{startd} ClassAd does not expire for another 20 minutes.
To avoid this machine being matched, an administrator chooses
to delete the machine's \Condor{startd} ClassAd.
Create a file (called \File{remove\_file} in this example)
with the three required attributes:
\begin{verbatim}
MyType = "Query"
TargetType = "Machine"
Requirements = Name == "condor.example.com"
\end{verbatim}

This file is used with the command:
\begin{verbatim}
% condor_advertise INVALIDATE_STARTD_ADS remove_file 
\end{verbatim}

\ExitStatus

\Condor{advertise} will exit with a status value of 0 (zero) upon
success, and it will exit with the value 1 (one) upon failure.
Success means that all ClassAds were successfully sent to all 
\Condor{collector} daemons.
When there are multiple ClassAds or multiple \Condor{collector} daemons,
it is possible that some but not all publications succeed;
in this case, the exit status is 1, indicating failure.

\end{ManPage}
