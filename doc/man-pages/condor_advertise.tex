\begin{ManPage}{\label{man-condor-advertise}\Condor{advertise}}{1}
{Send a classad to the collector daemon}
\Synopsis \SynProg{\Condor{advertise}}
\ToolArgsBase
\oOptArg{-pool}{centralmanagerhostname[:portname]}
\oOpt{-debug}
\oOpt{-tcp}
\Arg{update-command}
\oArg{classad-filename}

\index{Condor commands!condor\_advertise}
\index{condor\_advertise command}

\Description
\Condor{advertise} sends a classad to the collector daemon on
the central manager machine.
The classad may be contained in a file,
whose name is given as the final argument. If the file
argument is missing or equal to ``-'', then the classad is read
from standard input.
Which daemon's classad to update is specified by the first
required argument.
The \Arg{update-command} may be one of the following strings:
\begin{description}
\item[UPDATE\_STARTD\_AD]
\item[UPDATE\_SCHEDD\_AD]
\item[UPDATE\_MASTER\_AD]
\item[UPDATE\_GATEWAY\_AD]
\item[UPDATE\_CKPT\_SRVR\_AD]
\item[UPDATE\_NEGOTIATOR\_AD]
\item[UPDATE\_HAD\_AD]
\item[UPDATE\_AD\_GENERIC]
\item[UPDATE\_SUBMITTOR\_AD]
\item[UPDATE\_COLLECTOR\_AD]
\item[UPDATE\_LICENSE\_AD]
\item[UPDATE\_STORAGE\_AD]
\end{description}

\Condor{advertise} can also be used to invalidate and delete
classads currently held by the collector daemon.  In this case
the \Arg{update-command} will be one of the following strings:
\begin{description}
\item[INVALIDATE\_STARTD\_ADS]
\item[INVALIDATE\_SCHEDD\_ADS]
\item[INVALIDATE\_MASTER\_ADS]
\item[INVALIDATE\_GATEWAY\_ADS]
\item[INVALIDATE\_CKPT\_SRVR\_ADS]
\item[INVALIDATE\_NEGOTIATOR\_ADS]
\item[INVALIDATE\_HAD\_ADS]
\item[INVALIDATE\_ADS\_GENERIC]
\item[INVALIDATE\_SUBMITTOR\_ADS]
\item[INVALIDATE\_COLLECTOR\_ADS]
\item[INVALIDATE\_LICENSE\_ADS]
\item[INVALIDATE\_STORAGE\_ADS]
\end{description}

When an INVALIDATE command is given the provided classad consists
of only three entries.  "MyType" should always be set to "Query."
"TargetType" should be set to the MyType of the ad you want to
delete.  "Requirements" is an expression evaluated in the context
of ads of TargetType.  When Requirements evaluates to TRUE, the
matching ad is invalidated.  For example, the following ad, when
used with INVALIDATE\_STARTD\_ADS will delete the classad for the
machine named condor.example.com.
\begin{verbatim}
MyType = "Query"
TargetType = "Machine"
Requirements = Name == "condor.example.com"
\end{verbatim}


\begin{Options}
    \ToolArgsBaseDesc
    \OptItem{\OptArgnm{-pool}{centralmanagerhostname[:portname]}}
            {Specify a pool by
            giving the central manager's hostname and an optional port
	    number.  The default is the
	    \MacroNI{COLLECTOR\_HOST} specified in the configuration file.}
    \OptItem{\Opt{-tcp}}{Use TCP for communication. Without this
        option, UDP is used.}
    \OptItem{\Opt{-debug}}{Print debugging information as the command
            executes.}
\end{Options}

\GenRem
The job and machine classads are regularly updated.
Therefore, the result of \Condor{advertise} is likely to be
overwritten in a very short time.
It is unlikely that either Condor users (those who submit jobs)
or administrators will ever have a use for this command.
If it is desired to update or set a classad attribute, the
\Condor{config\_val} command is the proper command to use.

For those administrators who do need \Condor{advertise}, you can 
optionally include these attributes:

DaemonStartTime - The time the service you are advertising
started running.  Measured in seconds since the Unix epoch.

UpdateSequenceNumber - An integer that begins at 0 and increments by
one each time you re-advertise the same ad.

If both of the above are included, the \Condor{collector} will
automatically include the following attributes:

UpdatesTotal - The actual number of advertisements for this
daemon that the collector has seen.

UpdatesLost - The number of advertisements that for this daemon
that the collector expected to see, but didn't.

UpdatesSequenced - The total of UpdatesTotal and UpdatesLost.

UpdatesHistory - See \Macro{COLLECTOR\_DAEMON\_HISTORY\_SIZE} in
section~\ref{sec:Collector-Config-File-Entries}.

\ExitStatus

\Condor{advertise} will exit with a status value of 0 (zero) upon success,
and it will exit with the value 1 (one) upon failure.

\end{ManPage}
