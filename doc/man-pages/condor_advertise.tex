\begin{ManPage}{\label{man-condor-advertise}\Condor{advertise}}{1}
{Send a classad to the collector daemon}
\Synopsis \SynProg{\Condor{advertise}}
\ToolArgsBase
\oOptArg{-pool}{centralmanagerhostname[:portname]}
\oOpt{-debug}
\oOpt{-tcp}
\Arg{update-command}
\Arg{classad-filename}

\index{Condor commands!condor\_advertise}
\index{condor\_advertise command}

\Description
\Condor{advertise} sends a classad to the collector daemon on
the central manager machine.
The classad is contained in a file,
which is specified by the second required argument.
Which daemon's classad to update is specified by the first
required argument.
The \Arg{update-command} may be one of the following strings:
\begin{description}
\item[UPDATE\_STARTD\_AD]
\item[UPDATE\_SCHEDD\_AD]
\item[UPDATE\_MASTER\_AD]
\item[UPDATE\_GATEWAY\_AD]
\item[UPDATE\_CKPT\_SRVR\_AD]
\item[UPDATE\_NEGOTIATOR\_AD]
\item[UPDATE\_HAD\_AD]
\item[UPDATE\_AD\_GENERIC]
\item[UPDATE\_SUBMITTOR\_AD]
\item[UPDATE\_COLLECTOR\_AD]
\item[UPDATE\_LICENSE\_AD]
\item[UPDATE\_STORAGE\_AD]
\end{description}

\begin{Options}
    \ToolArgsBaseDesc
    \OptItem{\OptArgnm{-pool}{centralmanagerhostname[:portname]}}
            {Specify a pool by
            giving the central manager's hostname and an optional port
	    number.  The default is the
	    \MacroNI{COLLECTOR\_HOST} specified in the configuration file.}
    \OptItem{\Opt{-tcp}}{Use TCP for communication. Without this
        option, UDP is used.}
    \OptItem{\Opt{-debug}}{Print debugging information as the command
            executes.}
\end{Options}

\GenRem
The job and machine classads are regularly updated.
Therefore, the result of \Condor{advertise} is likely to be
overwritten in a very short time.
It is unlikely that either Condor users (those who submit jobs)
or administrators will ever have a use for this command.
If it is desired to update or set a classad attribute, the
\Condor{config\_val} command is the proper command to use.

For those administrators who do need \Condor{advertise}, you can 
optionally include these attributes:

DaemonStartTime - The time the service you are advertising
started running.  Measured in seconds since the Unix epoch.

UpdateSequenceNumber - An integer that begins at 0 and increments by
one each time you re-advertise the same ad.

If both of the above are included, the \Condor{collector} will
automatically include the following attributes:

UpdatesTotal - The actual number of advertisements for this
daemon that the collector has seen.

UpdatesLost - The number of advertisements that for this daemon
that the collector expected to see, but didn't.

UpdatesSequenced - The total of UpdatesTotal and UpdatesLost.

UpdatesHistory - See \Macro{COLLECTOR\_DAEMON\_HISTORY\_SIZE} in
section~\ref{sec:Collector-Config-File-Entries}.

\ExitStatus

\Condor{advertise} will exit with a status value of 0 (zero) upon success,
and it will exit with the value 1 (one) upon failure.

\end{ManPage}
