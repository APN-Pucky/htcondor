\begin{ManPage}{\label{man-condor-sos}\Condor{sos}}{1}
{Issue a command that will be service with a higher priority}
\index{HTCondor commands!condor\_sos}
\index{condor\_sos command}

\Synopsis \SynProg{\Condor{sos}}
\ToolArgsBase

\SynProg{\Condor{sos}}
\oOpt{-debug}
\oOptArg{-timeoutmult}{value}
\Arg{condor\_command}

\Description
\Condor{sos} sends the \Arg{condor\_command} in such a way 
that the command is serviced ahead of other waiting commands.
It appears to have a higher priority than other waiting commands.

\Condor{sos} is intended to give administrators a way to query the
\Condor{schedd} and \Condor{collector} daemons when they are under
such a heavy load that they are not responsive. 

There must be a special command port configured, 
in order for a command to be serviced with priority.
The \Condor{schedd} and \Condor{collector} always have the special
command port.
Other daemons require configuration by setting
configuration variable \MacroB{<SUBSYS>\_SUPER\_ADDRESS\_FILE}.

\begin{Options}
  \ToolArgsBaseDesc
  \OptItem{\Opt{-debug}}{Print extra debugging information as the command
    executes.}
  \OptItem{\OptArgnm{-timeoutmult}{value}}
    {Multiply any timeouts set for the command by the integer \Arg{value}.  }
\end{Options}

\Examples

The example command
\begin{verbatim}
  condor_sos -timeoutmult 5 condor_hold -all
\end{verbatim}
causes the \Code{condor\_hold -all} command to be handled by the
\Condor{schedd} with priority over any other commands that the \Condor{schedd}
has waiting to be serviced.
It also extends any set timeouts by a factor of 5.

\ExitStatus

\Condor{sos} will exit with the value 1 on error and with the
exit value of the invoked command when the command is successfully
invoked. 

\end{ManPage}
