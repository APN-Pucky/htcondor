\begin{ManPage}{\label{man-stork-submit}\Stork{submit}}{1}
{submit a Stork job}
\Synopsis \SynProg{\Stork{submit}}
\ToolArgsBase

\SynProg{\Stork{submit}}
\oOpt{-debug}
\oOpt{-stdin}
\Storkname
\oOptArg{-lognotes}{"prose"}
\Arg{submit-descpription-file}


\index{Condor commands!stork\_submit}
\index{Stork commands!stork\_submit}
\index{stork\_submit command}

\Description 

\Stork{submit} is used to submit a Stork data placement job.
Upon job submission, an integer identifier is assigned to the
submission,
	and it is printed to standard output.
This job identifier is required by other commands
that manage Stork jobs.

The name of the Stork submit description file is the single,
required, command-line argument.
Stork places no constraints on the submit
description file name.
See section \ref{sec:Stork}, for a more 
complete description of Stork.

\emph{STORK SUBMIT DESCRIPTION FILE COMMANDS}
Stork submit description files use ClassAd syntax, different from Condor submit
file syntax.  See \URL{http://www.cs.wisc.edu/condor/classad/}.

\begin{description}

%%%%%%%%%%%%%%%%%%%
%% dap\_type
%%%%%%%%%%%%%%%%%%%
\item[dap\_type = transfer;]
% only "transfer" for now
%\item[dap\_type = $<$reserve \Bar\ release \Bar\ transfer \Bar\ stage \Bar\ remove $>$]
\index{Stork submit commands!dap\_type}
Required command identifying that there will be
a transfer from source to destination.

%%%%%%%%%%%%%%%%%%%
%% arguments
%%%%%%%%%%%%%%%%%%%

\item[arguments = "$<$argument\_list$>$";]
List of arguments to be supplied
to the module on the command line.
Arguments are delimited (separated) by space characters.

%%%%%%%%%%%%%%%%%%%
%% input
%%%%%%%%%%%%%%%%%%%

\item[input = "$<$pathname$>$";]
Stork assumes that its jobs are
long-running, and that the user will not wait at the terminal for their
completion. Because of this, the standard files which normally access
the terminal, (\File{stdin}, \File{stdout}, and \File{stderr}),
must refer to files. Thus,
the file name specified with \SubmitCmd{input} should contain any keyboard
input the program requires (that is, this file becomes \File{stdin}).
If not specified, the default value
of \File{/dev/null} is used for submission to a Unix machine.
%If not specified, input is ignored
%for submission to a Windows machine.

\Stork{submit} will prepend the current working directory if the pathname is
relative (does not start with a / character).  This implies that the submit
directory must be shared between \Stork{submit} and the Stork server host, when
using relative paths.  All local file paths passed to Stork must be valid on
the Stork server host.

Note that this command does \emph{not} refer to the command-line
arguments of the program.  The command-line arguments are specified by
the \SubmitCmd{arguments} command.

%%%%%%%%%%%%%%%%%%%
%% output
%%%%%%%%%%%%%%%%%%%

\item[output = "$<$pathname$>$";]
The \SubmitCmd{output} file name will capture
any information the program would normally write to the screen
(that is, this file becomes \File{stdout}).
If not specified, the default value of
\File{/dev/null} is used for submission to a Unix machine.
%If not specified, output is ignored
%for submission to a Windows machine.
Multiple jobs should not use the same output
file, since this will cause one job to overwrite the output of
another.
The output file and the error file should not be the same file
as the outputs will overwrite each other or be lost.

Note that if your program explicitly opens and writes to a file,
that file should \emph{not} be specified as the output file.

\Stork{submit} will prepend the current working directory if the pathname is
relative (does not start with a / character).  This implies that the submit
directory must be shared between \Stork{submit} and the Stork server host, when
using relative paths.  All local file paths passed to Stork must be valid on
the Stork server host.

%%%%%%%%%%%%%%%%%%%
%% error
%%%%%%%%%%%%%%%%%%%

\item[err = "$<$pathname$>$";]
The \SubmitCmd{err} file name will capture any
error messages the program would normally write to the screen
(that is, this file becomes \File{stderr}).
If not specified, the default value of
\File{/dev/null} is used for submission to a Unix machine.
%If not specified, error messages are ignored
%for submission to a Windows machine.
More than one job should not use the same error file, since
this will cause one job to overwrite the errors of another.
The error file and the output file should not be the same file
as the outputs will overwrite each other or be lost.

\Stork{submit} will prepend the current working directory if the pathname is
relative (does not start with a / character).  This implies that the submit
directory must be shared between \Stork{submit} and the Stork server host, when
using relative paths.  All local file paths passed to Stork must be valid on
the Stork server host.

%%%%%%%%%%%%%%%%%%%
%% log
%%%%%%%%%%%%%%%%%%%

\item[log = "$<$pathname$>$";]
Use \SubmitCmd{log} to specify a file name where
\index{submit commands!log}
Stork will write a log file of what is happening with this job cluster.
For example, Stork will log into this file when and where the job
begins running, when the job is checkpointed and/or migrated, when the
job completes, etc. Most users find specifying a \SubmitCmd{log} file to be very
handy; its use is recommended. If no \SubmitCmd{log} entry is specified, 
Stork does not create a log for this job.

\Stork{submit} will prepend the current working directory if the pathname is
relative (does not start with a / character).  This implies that the submit
directory must be shared between \Stork{submit} and the Stork server host, when
using relative paths.  All local file paths passed to Stork must be valid on
the Stork server host.
\SubmitCmd{log} file paths should not use NFS file systems.

\item[log\_xml = "True"; \Bar\ "False";]
If \SubmitCmd{log\_xml} is true, 
then the log file will be written in ClassAd XML. If it isn't
specified, XML is not used. Note that it's an XML fragment, and is
missing the file header and footer. Also note that you should never
mix XML and non-XML in a single file: if multiple jobs write to a
single log file, it is up to you to make sure that all of them specify
(or don't specify) this option in the same way.
%%%%%%%%%%%%%%%%%%%
% src_url
%%%%%%%%%%%%%%%%%%%
\item[src\_url = $<$protocol-name:URL$>$]
\index{Stork submit commands!src\_url}
A (required) URL to identify the data source, 
as well as the protocol to be used at the source.
\File{file:///} URLs must refer to valid paths on the Stork server host.

%%%%%%%%%%%%%%%%%%%
% dest_url
%%%%%%%%%%%%%%%%%%%
\item[dest\_url = $<$protocol-name:URL$>$]
\index{Stork submit commands!dest\_url}
A (required) URL to identify the data destination, 
as well as the protocol to be used at the destination.
\File{file:///} URLs must refer to valid paths on the Stork server host.

%%%%%%%%%%%%%%%%%%%
% reserve_id
%%%%%%%%%%%%%%%%%%%
%\item[reserve\_id = $<$ID$>$]
%\index{Stork submit commands!reserve\_id}
% Description of this command, including possible default if not present
% but necessary.

%%%%%%%%%%%%%%%%%%%
% reserve_size
%%%%%%%%%%%%%%%%%%%
%\item[reserve\_size = $<$size-in-bytes$>$]
%\index{Stork submit commands!reserve\_size}
% Description of this command, including possible default if not present
% but necessary.

%%%%%%%%%%%%%%%%%%%
% x509proxy
%%%%%%%%%%%%%%%%%%%
\item[x509proxy = $<$path-to-proxy$>$]
\index{Stork submit commands!x509proxy}
The path to and file name of an X.509 proxy when needed
for GSI authentication.  A value of \SubmitCmd{"default"} directs Stork to
search in the standard Globus GSI proxy locations.

%%%%%%%%%%%%%%%%%%%
% cred_name
%%%%%%%%%%%%%%%%%%%
\item[cred\_name = $<$credential-handle$>$]
\index{Stork submit commands!cred\_name}
Alternatively, a X.509 proxy may be managed via the Stork credential manager.
The \SubmitCmd{cred\_name} specifies the name by which the credential was
stored in the credential manager.  See the 
\Prog{stork\_store\_cred} manual on page \pageref{man-stork-store-cred}.

%%%%%%%%%%%%%%%%%%%
% alt_protocols
%%%%%%%%%%%%%%%%%%%
\item[alt\_protocols = $<$sourceprotocol-desinationprotocol, sourceprotocol-desinationprotocol, ...$>$]
\index{Stork submit commands!alt\_protocols}
A comma separated list of alternative protocol pairings to be used
when a data transfer fails.
For each pair, the protocol to use at the source of the transfer
is followed by a \verb@-@ (dash) and the protocol to be used
at the destination of the transfer.
The list is used
(together with the original \SubmitCmd{src\_url}
and \SubmitCmd{dest\_url} protocols)
in a round robin fashion.
The source and destination URLs are unchanged;
only the protocols to be used are changed.


\end{description}


\begin{Options}
  \ToolArgsBaseDesc
  \OptItem{\Opt{-debug}}{Show extra debugging information.}
  \OptItem{\Opt{-stdin}}{Read commands from \File{stdin} instead of from a file.}
  \StorknameDesc
  \OptItem{\OptArg{-lognotes}{"prose"}}{The string given within quote
  marks is appended to the data placement ClassAd before
  the job is submitted.}
\end{Options}

\ExitStatus

\Stork{submit} will exit with a status value of 0 (zero) upon success,
and it will exit with the value 1 (one) upon failure.

\end{ManPage}
