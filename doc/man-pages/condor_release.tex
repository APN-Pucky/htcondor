\begin{ManPage}{\Condor{release}}{1}{Release jobs in the condor queue from hold state}
\label{man-condor-release}
\Synopsis \SynProg{\Condor{release}}
\oOptArg{-n}{schedd\_name}
\Arg{cluster \Bar cluster.process \Bar username}

\index{Condor commands!condor\_release}
\index{condor\_release command}

\Description

\Condor{release} releases one or more jobs in the condor job queue from hold state.  If a
cluster\_id and a process\_id are both specified, \Condor{release} attempts to
release the specified job. If a cluster\_id is specified
without a process\_id, \Condor{release} attempts to release all processes belonging
to the specified cluster. If a username is specified, \Condor{release} attempts to
release all jobs belonging to that user. Only the owner of a job, user condor,
or user root can release any given job.

When a job is released from hold state, it is returned to idle state, and will be scheduled
to run when possible. Only jobs that are on hold can be released.

\begin{Options}
    \OptItem{\OptArg{-n}{schedd\_name}}{Release jobs in the queue of the specified schedd}
\end{Options}

\SeeAlso
\Condor{hold} (on page~\pageref{man-condor-hold})

\end{ManPage}
