\begin{ManPage}{\label{man-condor-release}\Condor{release}}{1}
{release held jobs in the condor queue}
\Synopsis \SynProg{\Condor{release}}
\ToolArgs
\oArg{cluster $|$ cluster.process $|$ user $|$ -all}
\oOpt{-constraint \emph{constraint}}

\index{Condor commands!condor\_release}
\index{condor\_release command}

\Description

\Condor{release} releases one or more jobs from the Condor job queue that were 
previously placed in hold state.  
If the \Opt{-name} option is specified, the named \Condor{schedd} is targeted
for processing.  
Otherwise, the local \Condor{schedd} is targeted.
The jobs to be released are identified by one or more job identifiers, as
described below.
For any given job, only the owner of the job or one of the queue super users
(defined by the \MacroNI{QUEUE\_SUPER\_USERS} macro) can release the job.

\begin{Options}
	\ToolArgsDesc
	\OptItem{\Arg{cluster}}{Release all jobs in the specified cluster}
	\OptItem{\Arg{cluster.process}}{Release the specific job in the cluster}
	\OptItem{\Arg{user}}{Release jobs belonging to specified user}
    \OptItem{\Opt{-all}}{Release all the jobs in the queue}
	\OptItem{\Arg{-constraint \emph{constraint}}}{Release 
		jobs matching specified constraint}
\end{Options}

\SeeAlso
\Condor{hold} (on page~\pageref{man-condor-hold})

\GenRem

When releasing a held PVM universe job, you must release the entire
job cluster.  (In the PVM universe, each PVM job is assigned its own
cluster number, and each machine class is assigned a ``process''
number in the job's cluster.)  Releasing a subset of the machine
classes for a PVM job is not supported.

\end{ManPage}
