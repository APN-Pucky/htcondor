\begin{ManPage}{\label{man-condor-release}\Condor{release}}{1}
{release held jobs in the Condor queue}
\Synopsis \SynProg{\Condor{release}}
\ToolArgsBase 

\SynProg{\Condor{release}}
\ToolDebugOption
\ToolLocate
\ToolJobs
$|$ \OptArg{-constraint}{expression} \Dots

\SynProg{\Condor{release}}
\ToolDebugOption
\ToolLocate
\ToolAll

\index{Condor commands!condor\_release}
\index{condor\_release command}

\Description

\Condor{release} releases jobs from the Condor job queue that were 
previously placed in hold state.  
If the \Opt{-name} option is specified, the named \Condor{schedd} is targeted
for processing.  
Otherwise, the local \Condor{schedd} is targeted.
The jobs to be released are identified by one or more job identifiers, as
described below.
For any given job, only the owner of the job or one of the queue super users
(defined by the \MacroNI{QUEUE\_SUPER\_USERS} macro) can release the job.

\begin{Options}
	\ToolArgsBaseDesc
	\ToolLocateDesc
	\ToolDebugDesc
	\OptItem{\Arg{cluster}}{Release all jobs in the specified cluster}
	\OptItem{\Arg{cluster.process}}{Release the specific job in the cluster}
	\OptItem{\Arg{user}}{Release jobs belonging to specified user}
	\OptItem{\OptArg{-constraint}{expression}} {Release all jobs which match
	                the job ClassAd expression constraint}
        \OptItem{\Arg{-all}}{Release all the jobs in the queue}
\end{Options}

\SeeAlso
\Condor{hold} (on page~\pageref{man-condor-hold})


\Examples
To release all of the jobs of a user named Mary:
\footnotesize
\begin{verbatim}
% condor_release Mary 
\end{verbatim}
\normalsize

\ExitStatus

\Condor{release} will exit with a status value of 0 (zero) upon success,
and it will exit with the value 1 (one) upon failure.

\end{ManPage}
