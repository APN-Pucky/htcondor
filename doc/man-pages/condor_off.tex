\begin{ManPage}{\label{man-condor-off}\Condor{off}}{1}
{Shutdown HTCondor daemons}
\index{HTCondor commands!condor\_off}
\index{condor\_off command}

\Synopsis \SynProg{\Condor{off}}
\ToolArgsBase

\SynProg{\Condor{off}}
\oOptnm{-graceful \Bar{} -fast  \Bar{} -peaceful \Bar{} -force-graceful}
\oOptArg{-annex}{name}
\ToolDebugOption
\ToolWhere
\ToolArgsAffect

\Description 

\Condor{off} shuts down a set of the HTCondor daemons running on a set of
one or more machines.
It does this cleanly so that checkpointable jobs may gracefully exit with
minimal loss of work.

The command \Condor{off} without any arguments will shut down
all daemons except \Condor{master}, unless \OptArg{-annex}{name} is
specified.
The \Condor{master} can then handle both local and remote
requests to restart the other HTCondor daemons if need be.  To restart
HTCondor running on a machine, see the \Condor{on} command.

With the \OptArg{-daemon}{master} option, \Condor{off} will shut down
all daemons including the \Condor{master}.
Specification using the \Opt{-daemon} option
will shut down
only the specified daemon.

For security reasons of authentication and authorization,
this command requires \Expr{ADMINISTRATOR} level of access.

\begin{Options}
	\ToolArgsBaseDesc
	\OptItem{\Opt{-graceful}}{Gracefully shutdown daemons (the default)}
	\OptItem{\Opt{-fast}}{Quickly shutdown daemons. 
A minimum of the first two characters of this option must be specified,
to distinguish it from the \Opt{-force-graceful} command.}
	\OptItem{\Opt{-peaceful}}{Wait indefinitely for jobs to finish}
	\OptItem{\Opt{-force-graceful}}{Force a graceful shutdown, even after issuing a \Opt{-peaceful} command.
A minimum of the first two characters of this option must be specified,
to distinguish it from the \Opt{-fast} command.}
	\OptItem{\OptArg{-annex}{name}}{Turn off master daemons in the specified annex.  By default this will result in the corresponding instances shutting down.}
	\ToolDebugDesc
	\ToolArgsLocateDesc
	\ToolArgsAffectDesc
\end{Options}

\ExitStatus
\Condor{off} will exit with a status value of 0 (zero) upon success,
and it will exit with the value 1 (one) upon failure.


\Examples
To shut down all daemons (other than \Condor{master}) on the
local host:
\begin{verbatim}
% condor_off
\end{verbatim}

To shut down only the \Condor{collector} on three named machines:
\begin{verbatim}
% condor_off  cinnamon cloves vanilla -daemon collector
\end{verbatim}

To shut down daemons within a pool of machines other than the
local pool, use the \Opt{-pool} option.
The argument is the name of the central manager for the pool.
Note that one or more machines within the pool must be
specified as the targets for the command.
This command shuts down all daemons except the \Condor{master}
on the single machine named \Opt{cae17} within the
pool of machines that has \Opt{condor.cae.wisc.edu} as
its central manager:
\begin{verbatim}
% condor_off  -pool condor.cae.wisc.edu -name cae17
\end{verbatim}

\end{ManPage}
