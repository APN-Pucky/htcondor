\begin{ManPage}{\label{man-condor-vacate-job}\Condor{vacate\_job}}{1}
{vacate jobs in the Condor queue from the hosts where they are running}
\Synopsis \SynProg{\Condor{vacate\_job}}
\ToolArgsBase

\SynProg{\Condor{vacate\_job}}
\ToolLocate
\oOptnm{-fast}
\ToolJobs
$|$ \OptArg{-constraint}{expression} \Dots

\SynProg{\Condor{vacate\_job}}
\ToolLocate
\oOptnm{-fast}
\ToolAll

\index{Condor commands!condor\_vacate\_job}
\index{condor\_vacate\_job command}

\Description

\Condor{vacate\_job} finds one or more jobs from the Condor job queue
and vacates them from the host(s) where they are currently running.  
The jobs remain in the job queue and return to the idle state.

A job running under the standard universe will first produce a
checkpoint and then the job will be killed.
Condor will then restart the job somewhere else, using the checkpoint
to continue from where it left off.
A job running under any other universe will be sent a soft kill signal
(SIGTERM by default, or whatever is defined as the \Attr{SoftKillSig}
in the job ClassAd), and Condor will restart the job from the
beginning somewhere else. 

If the \Opt{-fast} option is used, the job(s) will be immediately killed,
meaning that standard universe jobs will not be allowed to checkpoint,
and the job will have to revert to the last checkpoint or start over
from the beginning.

If the \Opt{-name} option is specified, the named \Condor{schedd} is
targeted for processing.  
If the \Opt{-addr} option is used, the \Condor{schedd} at the given
address is targeted for processing.  
Otherwise, the local \Condor{schedd} is targeted.
The jobs to be vacated are identified by one or more job identifiers, as
described below.
For any given job, only the owner of the job or one of the queue super users
(defined by the \MacroNI{QUEUE\_SUPER\_USERS} macro) can vacate the job.

Using \Condor{vacate\_job} on jobs which are not currently running has
no effect.

\begin{Options}
	\ToolArgsBaseDesc
	\ToolLocateDesc
	\OptItem{\Arg{cluster}}{Vacate all jobs in the specified cluster}
	\OptItem{\Arg{cluster.process}}{Vacate the specific job in the cluster}
	\OptItem{\Arg{user}}{Vacate jobs belonging to specified user}
	\OptItem{\OptArg{-constraint}{expression}} {Vacate all jobs which match
	                the job ClassAd expression constraint}
        \OptItem{\Arg{-all}}{Vacate all the jobs in the queue}
        \OptItem{\Arg{-fast}}{Perform a fast vacate and hard kill the jobs}
\end{Options}

\GenRem

Do not confuse \Condor{vacate\_job} with \Condor{vacate}.
\Condor{vacate} is given a list of hosts to vacate, regardless of what
jobs happen to be running on them.
Only machine owners and administrators have permission to use
\Condor{vacate} to evict jobs from a given host.
\Condor{vacate\_job} is given a list of job to vacate, regardless of
which hosts they happen to be running on.
Only the owner of the jobs or queue super users have permission to use
\Condor{vacate\_job}.

When vacating a PVM universe job, you should always vacate the entire
job cluster.  (In the PVM universe, each PVM job is assigned its own
cluster number, and each machine class is assigned a ``process''
number in the job's cluster.)  Vacating a subset of the machine
classes for a PVM job is not supported using \Condor{vacate\_job}.
To vacate individual nodes in a PVM computation, you must use
\Condor{vacate} and target the specific hosts.  

\Examples

To vacate job 23.0:
\begin{verbatim}
% condor_vacate_job 23.0
\end{verbatim}

To vacate all jobs of a user named Mary:
\begin{verbatim}
% condor_vacate_job mary 
\end{verbatim}

To vacate all standard universe jobs owned by Mary:
\footnotesize
\begin{verbatim}
% condor_vacate_job -constraint 'JobUniverse == 1 && Owner == "mary"'
\end{verbatim}
\normalsize
Note that the entire constraint, including the quotation marks, must
be enclosed in single quote marks for most shells.

\ExitStatus

\Condor{vacate\_job} will exit with a status value of 0 (zero) upon success,
and it will exit with the value 1 (one) upon failure.

\end{ManPage}
