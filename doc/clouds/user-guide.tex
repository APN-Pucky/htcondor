\section{\label{sec:clouds-annex}HTCondor Annex User's Guide}

A user of \Condor{annex} may be a regular job submitter, or she may be an
HTCondor pool administrator.  This guide will cover basic \Condor{annex} usage
first, followed by advanced usage that may be of less interest to the
submitter.  Users interested in customizing \Condor{annex} should consult
section \ref{sec:clouds-services}.

\subsection{Considerations and Limitations}

When you run \Condor{annex}, you are adding (virtual) machines to an HTCondor
pool.  As a submitter, you probably don't have permission to add machines to
the HTCondor pool you're already using; generally speaking, security concerns
will forbid this.  If you're a pool administrator, you can of course add
machines to your pool as you see fit.  By default, however, \Condor{annex}
instances will only start jobs submitted by the user who started the annex,
so pool administrators using \Condor{annex} on their users' behalf will
probably want to use the \Opt{-owners} option or \Opt{-no-owner} flag;
see the man page (section \ref{man-condor-annex}).  Once the new machines
join the pool, they will run jobs as normal.

Submitters, however, will have to set up their own personal HTCondor pool,
so that \Condor{annex} has a pool to join, and then work with their pool
administrator if they want to move their existing jobs to their new pool.
Otherwise, jobs will have to be manually divided (removed from one and
resubmitted to the other) between the pools.  For instructions on creating a
personal condor pool, configuring \Condor{annex} to use a particular AWS
account, and then setting up that account for use with \Condor{annex}, see

%% This Wiki page should be frozen and migrated into the manual for
%% the stable release (v8.8.0); ideally, this would also be a reasonable
%% thing to do for the next development series (v8.9) as well.

\URL{https://htcondor-wiki.cs.wisc.edu/index.cgi/wiki?p=UsingCondorAnnexForTheFirstTimeEightSevenOne}.

As of v8.7.1, \Condor{annex} does not support joining pools which do not
allow inbound access to port 9618.  \Condor{annex} will therefore check
for this connectivity before starting an annex.  When checking connectivity
from AWS, the IP(s) used by the AWS Lambda function implementing this check
may not be in the same range(s) as those used by AWS instance; please
consult AWS's list of all their IP
ranges\footnote{\URL{https://ip-ranges.amazonaws.com/ip-ranges.json}}
when configuring your firewall.

\subsection{Basic Usage}

A quick-start guide with basic usage information is kept on our Wiki:

%% This Wiki page should be frozen and migrated into the manual for
%% the stable release (v8.8.0); ideally, this would also be a reasonable
%% thing to do for the next development series (v8.9) as well.

\URL{https://htcondor-wiki.cs.wisc.edu/index.cgi/wiki?p=HowToUseCondorAnnexWithOnDemandInstances}.

\subsection{Advanced Usage}

The basic usage guide on the Wiki covered using what AWS calls ``on-demand''
instances.  (An ``instance'' is ``a single occurrence of something,'' in
this case, a virtual machine.  The intent is to distinguish between the
active process that's pretending to be a real piece of hardware --
the ``instance'' -- and the template it used to start it up, which may also
be called a virtual machine.)  An on-demand instance has a price fixed by AWS;
once acquired, AWS will let you keep it running as long as you continue to
pay for it.

In constrast, a ``Spot'' instance has a price determined by an (automated)
auction; when you request a ``Spot'' instance, you specify the most (per hour)
you're willing to pay for that instance.  If you get an instance, however,
you pay only what the spot price is for that instance; in effect, AWS
determines the spot price by lowering it until they run out of instances
to rent.  AWS advertises savings of up to 90\% over on-demand instances.

There are two drawbacks to this cheaper type of instance: first,
you may have to wait (indefinitely) for instances to become available at
your preferred price-point; the second is that your instances may be taken
away from you before you're done with them because somebody else will pay
more for them.  (You won't be charged for the hour in which AWS kicks
you off an instance, but you will still owe them for all of that instance's
previous hours.)  Both drawbacks can be mitigated (but not eliminated) by
bidding the on-demand price for an instance; of course, this also minimizes
your savings.

Determining an appropriate bidding strategy is outside the purview of
this manual.

\subsubsection{Using AWS Spot Fleet}

\Condor{annex} supports Spot instances via an AWS technology called
``Spot Fleet''.  Normally, when you request instances, you request a specific
type of instance (the default on-demand instance is, for instance, `m4.large'.)
However, in many cases, you don't care too much about how many cores an
intance has -- HTCondor will automatically advertise the right number and
schedule jobs appropriately, so why would you?  In such cases -- or in
other cases where your jobs will run acceptably on more than one type of
instance -- you can make a Spot Fleet request which says something like
``give me a thousand cores as cheaply as possible'', and specify that
an `m4.large' instance has two cores, while `m4.xlarge' has four, and so
on.  (The interface actually allows you to assign arbitrary values --
like HTCondor slot weights -- to each instance
type\footnote{Strictly speaking, to each ``launch specification''; see
the explanation below, in \emph{AWS Instance User Data}.},
but the default value is core count.)  AWS will then divide the current price for each
instance type by its core count and request spot instances at the cheapest
per-core rate until the number of cores (not the number of instances!) has
reached a thousand, or that instance type is exhausted, at which point it will
request the next-cheapest instance type.

(At present, a Spot Fleet only chooses the cheapest price within each
AWS region; you would have to start a Spot Fleet in each AWS region you
were willing to use to make sure you got the cheapest possible price.  For
fault tolerance, each AWS region is split into independent zones, but each
zone has its own price.  Spot Fleet takes care of that detail for you.)

In order to create an annex via a Spot Fleet, you'll need a file containing
a JSON blob which describes the Spot Fleet request you'd like to make.  (It's
too complicated for a reasonable command-line interface.)  The AWS web
console can be used to create such a file; the button to download that
file is (currently) in the upper-right corner of the last page before
you submit the Spot Fleet request; it is labeled `JSON config'.  You
may need to create an IAM role the first time you make a Spot Fleet
request; please do so before running \Condor{annex}.

You \emph{must} select the instance role profile used by your on-demand
instances for \Condor{annex} to work.  This value will be stored in the
configuration macro \Macro{ANNEX\_DEFAULT\_ODI\_INSTANCE\_PROFILE\_ARN}
by the setup procedure.

%% I should really fix it so that the ODI InstanceProfileARN is preferred
%% over the IamInstanceProfile specified in the JSON configuration file.

Specify the JSON configuration file using \Opt{-aws-spot-fleet-config-file},
or set the configuration macro \Macro{ANNEX\_DEFAULT\_SFR\_CONFIG\_FILE} to
the full path of the file you just downloaded, if you'd like it to become
your default configuration for Spot annexes.  Be aware that \Condor{annex}
does \emph{not} alter the validity period if one is set in the Spot
Fleet configuration file.  You should remove the references to `ValidFrom'
and `ValidTo' in the JSON file to avoid confusing surprises later.

Additionally, be aware that \Condor{annex} uses the Spot Fleet API in
its ``request'' mode, which means that an annex created with Spot
Fleet has the same semantics with respect to replacement as it would
otherwise: if an instance terminates for any reason, including AWS
taking it away to give to someone else, it is not replaced.

You must specify the number of cores (total instance weight; see above) using
\Opt{-slots}.  You may also specify \Opt{-aws-spot-fleet}, if you wish;
doing so may make this \Condor{annex} invocation more self-documenting.
You may use other options as normal, excepting those which begin with
\Opt{-aws-on-demand}, which indicates an option specific to on-demand
instances.

\subsubsection{Custom HTCondor Configuration}

When you specify a custom configuration, you specify the full path to a
configuration directory which will be copied to the instance.  The customizations
performed by \Condor{annex} will be applied to a temporary copy of this
directory before it is uploaded to the instance.  Those customizations
consist of creating two files: {\tt password\_file.pl} (named that way to ensure
that it isn't ever accidentally treated as configuration), and
{\tt 00ec2-dynamic.config}.  The former is a password file for use by the pool
password security method, which if configured, will be used by \Condor{annex}
automatically.  The latter is an HTCondor configuration file; it is named
so as to sort first and make it easier to over-ride with whatever configuration
you see fit.

\subsubsection{AWS Instance User Data}

HTCondor doesn't interfere with this in anyway, but if you'd like to set
an instance's user data, you may do so.  However, as of v8.7.1, the
\Opt{-user-data} options don't work for on-demand instances (the default
type).  If you'd like to specify user data for your Spot Fleet -driven
annex, you may do so in four different ways: on the command-line or
from a file, and for all launch specifications or for only those launch
specifications which don't already include user data.  These two choices
correspond to the absence or presence of a trailing \Opt{-file} and the
absence or presence of \Opt{-default} immediately preceding \Opt{-user-data}.

A ``launch specification,'' in this context, means one of the virtual machine
templates you told Spot Fleet would be an acceptable way to accomodate your
resource request.  This usually corresponds one-to-one with instance types,
but this is not required.

\subsubsection{Expert Mode}

The man page (in section \ref{man-condor-annex}) lists the ``expert
mode'' options.

Four of the ``expert mode'' options set the URLs used to access AWS services,
not including the CloudFormation URL needed by the \Opt{-setup} flag.  (To
change the region for the \Opt{-setup} flag, you must change the HTCondor
configuration macro \Macro{ANNEX\_DEFAULT\_CF\_URL}, which you may of
course change temporarily via the environment on the command line.)  This
can be useful if you'd like to temporarily use a different region than the
one in your configuration, which defaults to {\tt us-east-1}.  Be sure to
change all of the URLs -- Lambda functions and CloudWatch events in one
region don't work with instances in another region.

%% At some point, we may should add a \Opt{-aws-region} option to make this
%% easy enough for submitters, as well.  Or at least an option to set the
%% CF URL?

You may also temporarily specify a different AWS account by using the
access (\Opt{-aws-access-key-file}) and
secret key (\Opt{-aws-secret-key-file}) options.  Regular users may have
an accounting reason to do this.

The options labeled ``implementation details'' are probably best left
unused unless you're a developer.

%% Developers should but don't have an option to set the connectivity-checking
%% Lambda function's name (or ARN).
