\subsection{\label{sec:clouds-services-aws}Amazon Web Services}

Requirements for an Annex-compatible AMI are driven by how \Condor{annex}
securely transports HTCondor configuration and security tokens to the
instances; we will discuss that implementation briefly, to help you
understand the requirements, even though it will hopefully never matter
to you.

\subsubsection{Resource Requests}

For on-demand or Spot instances, we begin by making a single resource request
whose client token is the annex name concatenated with an underscore and then
a newly-generated GUID.  This construction allows us to terminate on-demand
instances belonging to a particular annex (by its name), as well as discover
the annex name from inside an instance.

An on-demand instance may obtain its instance ID directly from the AWS
metadata server, and then ask another AWS API for that instance ID's
client token.  Since GUIDs do not contain underscores, we can be certain
that anything to the left of the last underscore is the annex's name.

An instance started by a Spot Fleet has a client token generated by the
Spot Fleet.  Instead of performing a direct lookup, a Spot Fleet instance
must therefore determine which Spot Fleet started it, and then obtain that
Spot Fleet's client token.  A Spot Fleet will tag an instance with the
Spot Fleet's identity after the instance starts up.  This usually only
takes a few minutes, but the default image waits for up to 50 minutes,
since you're already paying for the first hour anyway.

\subsubsection{Secure Transport}

At this point, the instance knows its annex's name.  This allows the
instance to construct the name of the tarball it should download
({\tt config-AnnexName.tar.gz}), but does not tell it from where a file
with that name should be downloaded.

(Because the user data associated with resource request is not secure,
and because we want to leave the user data available for its normal
usage, we can't just encode the tarball or its location in the user data.)

The instance determines from which S3 bucket to download by asking the
metadata server which role the instance is playing.  (An instance without
a role is unable to make use of any AWS services without acquiring valid
AWS tokens through some other method.)  The instance role created by
the setup procedure includes permission to read files matching the pattern
{\tt config-*.tar.gz} from a particular private S3 bucket.  If the instance
finds permissions matching that pattern, it assumes that the corresponding
S3 bucket is the one from which it should download, and does so; if
successful, it untars the file in {\tt /etc/condor/config.d}.

In v8.7.1, the script executing these steps is named {\tt 49ec2-instance.sh},
and is called during configuration when HTCondor first starts up.

In v8.7.2, the script executing these steps is named {\tt condor-annex-ec2},
and is called during system start-up.

The HTCondor configuration and security tokens are at this point protected
on the instance's disk by the usual filesystem permissions.  To prevent
HTCondor jobs from using the instance's permissions to do anything, but
in particular download their own copy of the security tokens, the last
thing the script does is use the Linux kernel firewall to forbid any
non-root process from accessing the metadata server.

\subsubsection{Image Requirements}

Thus, to work with \Condor{annex}, an AWS AMI must:

\begin{itemize}
\item  Fetch the HTCondor configuration and security tokens from S3;
\item  configure HTCondor to turn off after it's been idle for too long;
\item  and turn off the instance when the HTCondor master daemon exits.
\end{itemize}

The second item could be construed as optional, but if left unimplemented,
will disable the \Opt{-idle} command-line option.

The default disk image implements the above as follows:

\begin{itemize}
\item  with a configuration script ({\tt /etc/condor/49ec2-instance.sh});
\item  with a single configuration item (\Macro{STARTD\_NOCLAIM\_SHUTDOWN});
\item  with a configuration item (\Macro{DEFAULT\_MASTER\_SHUTDOWN\_SCRIPT})
and the corresponding script ({\tt /etc/condor/master\_shutdown.sh}), which
just turns around and runs {\tt shutdown -h now}.
\end{itemize}

We also strongly recommend that every \Condor{annex} disk image:

\begin{itemize}
\item  Advertise, in the master and startd, the instance ID.
\item  Use the instance's public IP, by setting \Macro{TCP\_FORWARDING\_HOST}.
\item  Turn on communications integrity and encryption.
\item  Encrypt the run directories.
\end{itemize}

The default disk image is configured to do all of this.
