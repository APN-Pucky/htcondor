%%%%%%%%%%%%%%%%%%%%%%%%%%%%%%%%%%%%
\section{\label{API-Python} Python Bindings}
%%%%%%%%%%%%%%%%%%%%%%%%%%%%%%%%%%%%
\index{API!Python bindings}
\index{Python bindings}

The Python module provides bindings to the client-side APIs for HTCondor.  
It tries to provide functionality similar to the HTCondor command line tools.
Further improvements are planned for the difficult to use submission API.

%%%%%%%%%%%%%%%%%%%%%%%%%%%%%%%%%%%%
\subsection{\label{Python-OtherModule} Other Module}
%%%%%%%%%%%%%%%%%%%%%%%%%%%%%%%%%%%%
This module attempts to wrap a C++ API that is
far less formal when compared to the ClassAd language.

\textbf{Module functions:}
\begin{flushleft}
\begin{tabular}{|p{14cm}|} \hline
\texttt{platform( )}

Returns the platform of HTCondor this module is running on.
\\ \hline
\texttt{version( )}

Returns the version of HTCondor this module is linked against. 
\\ \hline
\texttt{reload\_config( )}

Reload the HTCondor configuration from disk. 
\\ \hline
\texttt{send\_command( ad, (DaemonCommands)dc, (str)target = None) }

Send a command to an HTCondor daemon specified by a location ClassAd. 

\texttt{ad} is a ClassAd specifying the location of the daemon; 
typically, found by using \texttt{Collector.locate(...)}.

\texttt{dc} is a command type; must be a member of the enum 
\texttt{DaemonCommands}. 

\texttt{target} is an optional parameter, representing an additional command
to send to a daemon.   Some commands require additional arguments; 
for example, sending \texttt{DaemonOff} to a \Condor{master} requires 
one to specify which subsystem to turn off. 
\\ \hline

\end{tabular}
\end{flushleft}

\textbf{The module object, \texttt{param}}, is
a dictionary-like object providing access to the configuration variables
in the current HTCondor configuration.

\textbf{The \Condor{schedd} class:}
\Todo

\textbf{The \Condor{collector} class:}
\Todo

\textbf{The class to access the internal security object:}
\Todo

%%%%%%%%%%%%%%%%%%%%%%%%%%%%%%%%%%%%
\subsection{\label{Python-ClassAd} ClassAd Module}
%%%%%%%%%%%%%%%%%%%%%%%%%%%%%%%%%%%%

The ClassAd module class provides a dictionary-like mechanism for interacting
with the ClassAd language. 
ClassAd objects implement the iterator interface to iterate 
through the ClassAd's attributes.

\textbf{Module functions:}
\begin{flushleft}
\begin{tabular}{|p{14cm}|} \hline
\texttt{parse( input )}

Parse input into a ClassAd.  
Returns a ClassAd object.

Parameter \texttt{input} is a string-like object or a file pointer.
\\ \hline

\texttt{parseOld}

Parse old ClassAd format input into a ClassAd.
\\ \hline
\texttt{version}

Return the version of the linked ClassAd library.
\\ \hline

\end{tabular}
\end{flushleft}


\textbf{Standard Python object methods for the \Expr{ClassAd} class:}
\begin{flushleft}
\begin{tabular}{|p{14cm}|} \hline

\texttt{\_\_init\_\_( str )}

Create a ClassAd object from string, \texttt{str}, passed as a parameter.
The string must be formatted in the new ClassAd format.
\\ \hline
\texttt{\_\_len\_\_( )}

Returns the number of attributes in the ClassAd; 
allows \texttt{len(object)} semantics for ClassAds.
\\ \hline
\texttt{\_\_str\_\_( )}

Converts the ClassAd to a string and returns the string;
the formatting style is new ClassAd,
with square brackets and semicolons.
For example, \Expr{[ Foo = "bar"; ]} may be returned.

\\ \hline
\end{tabular}
\end{flushleft}


\textbf{The ClassAd object has the following dictionary-like methods:}
\begin{flushleft}
\begin{tabular}{|p{14cm}|} \hline
\texttt{items( )}

Returns an iterator of tuples. Each item returned by the iterator 
is a tuple representing a pair (attribute,value) in the ClassAd object.
\\ \hline
\texttt{values( )}

Returns an iterator of objects. 
Each item returned by the iterator is a value in the ClassAd. 

If the value is a literal, 
it will be cast to a native Python object, 
so a ClassAd string will be returned as a Python string.
\\ \hline
\texttt{keys( )}

Returns an iterator of strings. Each item returned by the iterator 
is an attribute string in the ClassAd.
\\ \hline
\texttt{\_\_getitem( attr )}

Returns (as an object) the value corresponding to the attribute
\texttt{attr} passed as a parameter.

ClassAd values will be returned as Python objects; 
ClassAd expressions will be returned as \texttt{ExprTree} objects.
\\ \hline
\texttt{\_\_setitem\_\_( attr, value )}

Sets the ClassAd attribute \texttt{attr} to the \texttt{value}.

ClassAd values will be returned as Python objects; 
ClassAd expressions will be returned as \texttt{ExprTree} objects.
\\ \hline

\end{tabular}
\end{flushleft}


\textbf{Additional methods:}
\begin{flushleft}
\begin{tabular}{|p{14cm}|} \hline
\texttt{eval( attr )}

Evaluate the value given a ClassAd attribute \texttt{attr}.
Throws \texttt{ValueError} if unable to evaluate the object.

Returns the Python object corresponding to the evaluated ClassAd attribute.
\\ \hline
\texttt{lookup( attr )}

Look up the \texttt{ExprTree} object associated with attribute \texttt{attr}.
No attempt will be made to convert to a Python object.

Returns an \texttt{ExprTree} object.
\\ \hline
\texttt{printOld( )}

Print the ClassAd in the old ClassAd format. 

Returns a string.
\\ \hline
\end{tabular}
\end{flushleft}


\textbf{The \Expr{ExprTree} class} object
represents an expression in the ClassAd language.

\textbf{\texttt{ExprTree} class methods:}
\begin{flushleft}
\begin{tabular}{|p{14cm}|} \hline
\texttt{\_\_init\_\_( str )}

Parse the string \texttt{str} to create an \texttt{ExprTree}.
\\ \hline
\texttt{\_\_str\_\_( )}

Represent and return the ClassAd expression as a string.
\\ \hline
\texttt{eval()}

Evaluate the expression and return as a ClassAd value, 
typically a Python object.
\\ \hline
\end{tabular}
\end{flushleft}

%%%%%%%%%%%%%%%%%%%%%%%%%%%%%%%%%%%%
\subsection{\label{Python-ClassAd-Example} Sample Code using the ClassAd Module}
%%%%%%%%%%%%%%%%%%%%%%%%%%%%%%%%%%%%
This sample Python code illustrates interactions with the ClassAd Module. 

\footnotesize
\begin{verbatim}
$ python
Python 2.6.6 (r266:84292, Jun 18 2012, 09:57:52) 
[GCC 4.4.6 20110731 (Red Hat 4.4.6-3)] on linux2
Type "help", "copyright", "credits" or "license" for more information.
>>> import classad
>>> ad = classad.ClassAd()
>>> expr = classad.ExprTree("2+2")
>>> ad["foo"] = expr
>>> print ad["foo"].eval()
4
>>> ad["bar"] = 2.1
>>> ad["baz"] = classad.ExprTree("time() + 4")
>>> print list(ad)
['bar', 'foo', 'baz']
>>> print dict(ad.items())
{'baz': time() + 4, 'foo': 2 + 2, 'bar': 2.100000000000000E+00}
>>> print ad
    [
        bar = 2.100000000000000E+00; 
        foo = 2 + 2; 
        baz = time() + 4
    ]
>>> ad2=classad.parse(open("test_ad", "r"));
>>> ad2["error"] = classad.Value.Error
>>> ad2["undefined"] = classad.Value.Undefined
>>> print ad2
    [
        error = error; 
        bar = 2.100000000000000E+00; 
        foo = 2 + 2; 
        undefined = undefined; 
        baz = time() + 4
    ]
>>> ad2["undefined"]
classad.Value.Undefined

\end{verbatim}
\normalsize
