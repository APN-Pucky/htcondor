%%%%%%%%%%%%%%%%%%%%%%%%%%%%%%%%%%%%%%%%%%%%%%%%%%%%%%%%%%%%%%%%%%%%%%
\section{\label{sec:interp}Running Interpreted Programs in Condor}
%%%%%%%%%%%%%%%%%%%%%%%%%%%%%%%%%%%%%%%%%%%%%%%%%%%%%%%%%%%%%%%%%%%%%%

\index{Interpreted languages}
\index{Languages!interpreted}
\index{Java}
\index{Perl}
\index{Python}

Condor can manage programs that are interpreted, such as those written
in Java, Perl, Python, or almost any other language.  The advantage of
using interpreted languages is that they may run on any sort of machine,
regardless of its CPU or operating system.  With some help from the local
system administrator, Condor will find the correct path to the components
needed to execute the user's program.

Before getting started, the machines on which you wish to run must make it
known what languages they support.  This must be done by the \emph{owners}
of the machines you wish to run on.   This is done by setting the \Macro{INTERPRETERS}
value in the local config file.  Because interpreted languages come in many
versions and implementations, this config value must describe four very
specific things about the available interpreters.  Each line must contain
the language name, language version, vendor's name, vendor's version, and
the path to the interpreter.

For example, at our site, we advertise several versions of Java and Perl.
(Please carefully note the backslashes at the end of the lines.)

\begin{verbatim}
INTERPRETERS = \
        java 2 sun 1.2.2 /s/jdk/1.2.2/bin/java   \
        java 1 sun 1.1.8 /s/jdk/1.1.8/bin/java   \
        perl 5 perl 5.005 /usr/bin/perl          \
        perl 5 perl 5.001 /usr/bin/old-perl
\end{verbatim}

With these values set, the user may execute interpreted programs
in any of the available languages without knowing exactly where
they are installed.  Below we will give examples using
Java.  Other languages are possible by using the obvious substitutions.

Let's suppose that you have written a simple Java program called Hello.
First, you must compile the source into a class file:

\begin{verbatim}
 % javac Hello.java
\end{verbatim}

If you were to run the program at your workstation, you would
execute this:

\begin{verbatim}
 % java Hello
\end{verbatim}

To instead submit Hello as a Condor job, write a submit file that selects the
vanilla universe, specifies the Java language, and gives the class
file as the executable:

\begin{verbatim}
universe = vanilla
language = java
executable = Hello.class
arguments = Hello
output = Hello.out
\end{verbatim}

When this program is submitted, Condor will choose a machine that has
Java installed and will run the appropriate interpreter with ``Hello''
in the arguments.  More precisely, it will select a machine that
publishes the attribute \AdStr{javaInterpreter}.  You can list the
machines that make a Java interpreter available like this:

\index{javaInterpreter}
\begin{verbatim}
 % condor_status -constraint '(javaInterpreter!="")'
\end{verbatim}

Some programs are very sensitive to the exact interpreter used.
For example, some programs may require version 2 of the Java
language.  Some may even require a particular vendor's
interpreter.

If you simply specify \AdStr{language=java}, then your program
will be executed on \emph{any} available interpreter of any version
and any vendor.  If you require a particular version of the language,
then give the language version in the submit file:

\begin{verbatim}
language = java2
\end{verbatim}

If you require a particular vendor's interpreter:

\begin{verbatim}
language = java2sun
\end{verbatim}

Or even an exact software version:

\begin{verbatim}
language = java2sun1.2.2
\end{verbatim}

Although these jobs technically execute in the Vanilla universe,
they may be executed on machines that do not share a common file
system.  Condor's file transfer mechanism (described in section
\ref{sec:file-transfer}) may be used to move the necessary data
files in and out.  For example:

\index{File transfer}
\index{transfer\_input\_files}
\index{transfer\_output\_files}
\begin{verbatim}
requirements = (FileSystemDomain=="cs.wisc.edu") || (FileSystemDomain=="hep.wisc.edu")
transfer_input_files = Hello.input Helper.class
transfer_output_files = Hello.output
\end{verbatim}



