
%%%%%%%%%%%%%%%%%%%%%%%%%%%%%%%%%%%%%%%%%%%%%%%%%%%%%%%%%%%%%%%%%%%%%%
\section{\label{sec:Java}Java Applications}
%%%%%%%%%%%%%%%%%%%%%%%%%%%%%%%%%%%%%%%%%%%%%%%%%%%%%%%%%%%%%%%%%%%%%%

Condor allows users to access a wide variety of
machines distributed around the world.
The Java Virtual Machine (JVM)
\index{Java}
\index{Java Virtual Machine}
\index{JVM}
provides a uniform platform on any machine, regardless of the
machine's architecture or operating system.
The Condor Java universe brings together these
two features to create a distributed, homogeneous computing environment.

Compiled Java programs can be submitted to Condor, and Condor
can execute the programs on any machine in the pool that will run
the Java Virtual Machine.

The \Condor{status} command can be used to see a list of
machines in the pool for which Condor can use the Java Virtual
Machine.

\begin{verbatim}
% condor_status -java

Name          JavaVendor  Ver    State      Activity   LoadAv Mem   ActvtyTime

coral.cs.wisc Sun Microsy 1.2.2  Unclaimed  Idle       0.000   511  0+02:28:04
doc.cs.wisc.e Sun Microsy 1.2.2  Unclaimed  Idle       0.000   511  0+01:05:04
dsonokwa.cs.w Sun Microsy 1.2.2  Unclaimed  Idle       0.000   511  0+01:05:04
...
\end{verbatim}

If there is no output from the
\Condor{status} command,
then Condor does not know the location details of the Java Virtual
Machine on machines in the pool,
or no machines have Java correctly installed.
In this case,
contact your system administrator or see section \ref{sec:java-install}
for more information on getting Condor to work together
with Java.

Here is a complete, if simple, example.
Start with a simple Java program, \File{Hello.java}:

\begin{verbatim}
public class Hello {
        public static void main( String [] args ) {
                System.out.println("Hello, world!\n");
        }
}
\end{verbatim}

Build this program using your Java compiler.
On most platforms, this is
accomplished with the command
\begin{verbatim}
javac Hello.java
\end{verbatim}

Submission to Condor requires a submit description file.
This is a simple one that works:

\begin{verbatim}
  ####################
  #
  # Example 1
  # Execute a single Java class
  #
  ####################

  universe       = java
  executable     = Hello.class
  arguments      = Hello
  output         = Hello.output
  error          = Hello.error
  queue
\end{verbatim}

The Java universe must be explicitly selected.

The main class of the program is given in the \Code{executable} statement.
This is a file name which contains the entry point of the program.
The name of the main class (not a file name) must
be specified as the first argument to the program.

To submit the job, where the submit description file
is named \File{Hello.cmd}, 
execute 
\begin{verbatim}
condor_submit Hello.cmd
\end{verbatim}

To monitor the job, the commands \Condor{q} and \Condor{rm}
are used as with all jobs.

For programs that 
consist of more than one \Code{.class} file,
one option that tells Condor of the additional files adds
this line to the submit description file:

\begin{verbatim}
transfer_input_files = Larry.class Curly.class Moe.class
\end{verbatim}

If the program consists of a large number of class files,
it may be easier to collect them all together into
a single Java Archive (JAR) file.
A JAR can be created with:

\begin{verbatim}
% jar cvf Library.jar Larry.class Curly.class Moe.class
\end{verbatim}

Condor must then be told where to find the JAR by
adding the 
following to the submit description file:

\begin{verbatim}
jar_files = Library.jar
\end{verbatim}

Note that the JVM must know whether it is receiving JAR files
or class files.
Therefore, Condor must also be informed, in order to pass the
information on to the JVM.
That is why there is a difference in submit description file commands
for the two ways of specifying files (\verb@transfer_input_files@
and \verb@jar_files@).

If the program uses Java features found only in certain
JVMs, then inform Condor by adding a \Code{requirements}
statement to the submit description file.
For example, to require version 3.2, add to the submit description
file:

\begin{verbatim}
requirements = (JavaVersion=="3.2")
\end{verbatim}

Each machine with Java capability in a Condor pool
will execute a benchmark to determine its speed.
The benchmark is taken when Condor is started on
the machine, and it uses the SciMark2
(\Url{http://math.nist.gov/scimark2}) benchmark.
The result of the benchmark is held as an attribute
within the 
machine ClassAd.
The attribute is called \AdAttr{JavaMFlops}.
Jobs that are run under the Java universe (as all other Condor jobs)
may prefer or require a machine of a specific speed
by setting \AdAttr{rank} or \AdAttr{requirements} in
the submit description file.
As an example, to execute only on machines of a minimum speed:

\begin{verbatim}
requirements = (JavaMFlops>4.5)
\end{verbatim}

By default, Condor moves all input files
to each execution site before the job runs, and it moves
all output files back when the job completes.

\index{Chirp}
If a job has more sophisticated I/O requirements, then
you may use a facility called Chirp.
Chirp has two advantages over simple whole-file transfers.
First, it permits the input files to be decided upon at run-time
rather than submit time, and second,
it permits partial-file I/O with results than can be seen as the
program executes.
However, you must make small changes to the program
in order to take advantage of Chirp.
Depending on the style of the program, use either Chirp I/O streams
or UNIX-like I/O functions.

\index{Chirp!ChirpInputStream}
\index{Chirp!ChirpOutputStream}
Chirp I/O streams are the easiest way to get started.
Modify the program to use the objects \Code{ChirpInputStream}
and \Code{ChirpOutputStream} instead of \Code{FileInputStream} and
\Code{FileOutputStream}.
These classes are completely documented
\index{Software Developer's Kit!Chirp}
\index{SDK!Chirp}
in the Condor Software Developer's Kit (SDK).
Here is a simple code example:

\begin{verbatim}
import java.io.*;
import edu.wisc.cs.condor.chirp.*;

public class TestChirp {

   public static void main( String args[] ) {

      try {
         BufferedReader in = new BufferedReader(
            new InputStreamReader(
               new ChirpInputStream("input")));

         PrintWriter out = new PrintWriter(
            new OutputStreamWriter(
               new ChirpOutputStream("output")));

         while(true) {
            String line = in.readLine();
            if(line==null) break;
            out.println(line);
         }
      } catch( IOException e ) {
         System.out.println(e);
      }
   }
}
\end{verbatim}

\index{Chirp!ChirpClient}
To perform UNIX-like I/O with Chirp,
create a \Code{ChirpClient} object.
This object supports familiar operations such as \Code{open}, \Code{read},
\Code{write}, and \Code{close}.
Exhaustive detail of the methods may be found in the Condor 
SDK, but here is a brief example:

\begin{verbatim}
import java.io.*;
import edu.wisc.cs.condor.chirp.*;

public class TestChirp {

   public static void main( String args[] ) {

      try {
         ChirpClient client = new ChirpClient();
         String message = "Hello, world!\n";
         byte [] buffer = message.getBytes();

         // Note that we should check that actual==length.
         // However, skip it for clarity.

         int fd = client.open("output","wct",0777);
         int actual = client.write(fd,buffer,0,buffer.length);
         client.close(fd);

         client.rename("output","output.new");
         client.unlink("output.new");

      } catch( IOException e ) {
         System.out.println(e);
      }
   }
}
\end{verbatim}

\index{Chirp!Chirp.jar}
Regardless of which I/O style, 
the Chirp library must be specified and included with the job.
The Chirp JAR (\Code{Chirp.jar})
is found in the \File{lib} directory of the Condor installation.
Copy it into your working directory in order to
compile the program after modification to use Chirp I/O.

\begin{verbatim}
% condor_config_val LIB
/usr/local/condor/lib
% cp /usr/local/condor/lib/Chirp.jar .
\end{verbatim}

Rebuild the program with the Chirp JAR file in the class path.

\begin{verbatim}
% javac -classpath Chirp.jar:. TestChirp.java
\end{verbatim}

The Chirp JAR file must be specified in the submit description file.
Here is an example submit description file that works for both
of the given test programs:

\begin{verbatim}
universe = java
executable = TestChirp.class
arguments = TestChirp
jar_files = Chirp.jar
queue
\end{verbatim}
