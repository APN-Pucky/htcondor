%%%%%%%%%%%%%%%%%%%%%%%%%%%%%%%%%%%%%%%%%%%%%%%%%%%%%%%%%%%%%%%%%%%%%%
\subsection{\label{sec:Submit-Interactive}Interactive Jobs}
%%%%%%%%%%%%%%%%%%%%%%%%%%%%%%%%%%%%%%%%%%%%%%%%%%%%%%%%%%%%%%%%%%%%%%
\index{job!interactive}
\index{interactive jobs}

An \Term{interactive job} is a Condor job that is provisioned and
scheduled like any other vanilla universe Condor job 
onto an execute machine within the pool.
The result of a running interactive job is a shell prompt 
issued on the execute machine where the job runs.
The user that submitted the interactive job may then use the
shell as desired,
perhaps to interactively run an instance of what is to become
a Condor job.
This might aid in checking that the set up and execution
environment are correct,
or it might provide information on the RAM or disk space needed.
This job (shell) continues until the user logs out or any other
policy implementation causes the job to stop running.
A useful feature of the interactive job is that the users and jobs
are accounted for within Condor's scheduling and priority system.

Neither the submit nor the execute host for interactive
jobs may be on Windows platforms. 

The current working directory of the shell will be the
initial working directory of the running job.
The shell type will be the default for the user that submits
the job.
At the shell prompt, X11 forwarding is enabled.

Each interactive job will have a job ClassAd attribute of 
\begin{verbatim}
  InteractiveJob = True
\end{verbatim}

Submission of an interactive job specifies the option \Opt{-interactive}
on the \Condor{submit} command line.

A submit description file may be specified for this interactive job.
Within this submit description file, 
a specification of these 5 commands will be either ignored or altered:
\begin{enumerate}
\item \SubmitCmd{executable}
\item \SubmitCmd{transfer\_executable}
\item \SubmitCmd{arguments}
\item \SubmitCmd{universe}.  The interactive job is a vanilla universe job. 
\item \SubmitCmd{queue <n>}.  In this case the value of \SubmitCmdNI{<n>} is
ignored; exactly one interactive job is queued.
\end{enumerate}
The submit description file may specify anything else needed for
the interactive job, such as files to transfer.

If \emph{no} submit description file is specified for the job,
a default one is utilized as identified by the value of the
configuration variable \Macro{INTERACTIVE\_SUBMIT\_FILE}.

Here are examples of situations where interactive jobs may be
of benefit.
\begin{itemize}
\item An application that cannot be batch processed might be run
as an interactive job.
Where input or output cannot be captured in a file and the
executable may not be modified,
the interactive nature of the job may still be run on a pool
machine, and within the purview of Condor.
\item A pool machine with specialized hardware that requires
interactive handling can be scheduled with an interactive
job that utilizes the hardware.
\item The debugging and set up of complex jobs or environments
may benefit from an interactive session.
This interactive session provides the opportunity to run scripts 
or applications, 
and as errors are identified, 
they can be corrected on the spot.
\item Development may have an interactive nature,
and proceed more quickly when done on a pool machine.
It may also be that the development platforms required
reside within Condor's purview as execute hosts. 
\end{itemize}
