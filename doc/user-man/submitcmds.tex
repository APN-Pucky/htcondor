%%%%%%%%%%%%%%%%%%%%
\subsection{\label{sec:submit-cmds}Submit Description File Commands}  
%%%%%%%%%%%%%%%%%%%%
\index{submit commands}

Each submit description file describes one cluster of jobs to be
placed in the Condor execution pool. All jobs in a cluster must share
the same executable, but they may have different input and output files,
and different program arguments. The submit description file is
the only command-line argument to \Condor{submit}. If the submit description
file argument is omitted, \Condor{submit} will read the submit description
from standard input.

The submit description file must contain 
one \Arg{executable} command and at least one \Arg{queue} command.
All of the other commands have default actions.

The commands which can appear in the submit description file are
numerous.  They are listed here in alphabetical order by category.

%%%%%%%%%%%%%%%%%%%%%%%%%%%%%%%
\emph{BASIC COMMANDS}
%%%%%%%%%%%%%%%%%%%%%%%%%%%%%%%
\begin{description} 

%%%%%%%%%%%%%%%%%%%
%% arguments
%%%%%%%%%%%%%%%%%%%

\index{submit commands!arguments}
\item[arguments = $<$argument\_list$>$]
\label{man-condor-submit-arguments}
List of arguments to be supplied
to the program on the command line.   In the Java Universe, the first
argument must be the name of the class containing \Code{main}.

There are two permissible formats for specifying arguments.  The new
syntax supports uniform quoting of spaces within arguments; the old
syntax supports spaces in arguments only in special circumstances.

In the old syntax, arguments are delimited (separated) by space
characters.  Double-quotes must be escaped with a backslash (i.e. put
a backslash in front of each double-quote).

Further interpretation of the argument string differs depending on the
operating system.  On Windows, your argument string is
simply passed verbatim (other than the backslash in front of
double-quotes) to the Windows application.  Most Windows applications
will allow you to put spaces within an argument value by surrounding
the argument with double-quotes.
In all
other cases, there is no further interpretation of the arguments.

Example:

\begin{verbatim}
arguments = one \"two\" 'three'
\end{verbatim}

Produces in Unix vanilla universe:

\begin{verbatim}
argument 1: one
argument 2: "two"
argument 3: 'three'
\end{verbatim}

Here are the rules for using the new syntax:

\begin{enumerate}

\item Put double quotes around the entire argument string.  This
distinguishes the new syntax from the old, because these double-quotes
are not escaped with backslashes, as required in the old syntax.  Any
literal double-quotes within the string must be escaped by repeating
them.

\item Use white space (e.g. spaces or tabs) to separate arguments.

\item To put any white space in an argument, you must surround the
space and as much of the surrounding argument as you like with
single-quotes.

\item To insert a literal single-quote, you must repeat it anywhere
inside of a single-quoted section.

\end{enumerate}

Example:

\begin{verbatim}
arguments = "one ""two"" 'spacey ''quoted'' argument'"
\end{verbatim}

Produces:

\begin{verbatim}
argument 1: one
argument 2: "two"
argument 3: spacey 'quoted' argument
\end{verbatim}

Notice that in the new syntax, backslash has no special meaning.  This
is for the convenience of Windows users.

%%%%%%%%%%%%%%%%%%%
%% environment
%%%%%%%%%%%%%%%%%%%

\index{submit commands!environment}
\item[environment = $<$parameter\_list$>$] List of environment
\index{environment variables!setting, for a job}
\label{man-condor-submit-environment}
variables.

There are two different formats for specifying the environment
variables: the old format and the new format.  The old format is
retained for backward-compatibility.  It suffers from a
platform-dependent syntax and the inability to insert some special
characters into the environment.

The new syntax for specifying environment values:

\begin{enumerate}

\item Put double quote marks around the entire argument string.  This
distinguishes the new syntax from the old.
The old syntax does not have double quote marks around it.
Any literal double quote marks within the string
must be escaped by repeating the double quote mark.

\item Each environment entry has the form

\begin{verbatim}
<name>=<value>
\end{verbatim}

\item Use white space (space or tab characters) to separate environment entries.

\item To put any white space in an environment entry, surround
the space and as much of the surrounding entry as desired with
single quote marks.

\item To insert a literal single quote mark, repeat the single quote mark
anywhere inside of a section surrounded by single quote marks.

\end{enumerate}

Example:

\begin{verbatim}
environment = "one=1 two=""2"" three='spacey ''quoted'' value'"
\end{verbatim}

Produces the following environment entries:

\begin{verbatim}
one=1
two="2"
three=spacey 'quoted' value
\end{verbatim}

Under the old syntax, there are no double quote marks surrounding the
environment specification.  Each environment entry remains of the form
\begin{verbatim}
<name>=<value>
\end{verbatim}
Under Unix, list multiple environment entries by separating them with
a semicolon (;).  Under Windows, separate multiple entries with a
vertical bar (\Bar).  There is no way to insert a literal semicolon
under Unix or a literal vertical bar under Windows.  Note that spaces
are accepted, but rarely desired, characters within parameter names
and values, because they are treated as literal characters, not
separators or ignored white space.  Place spaces within the parameter
list only if required.

A Unix example:

\begin{verbatim}
environment = one=1;two=2;three="quotes have no 'special' meaning"
\end{verbatim}

This produces the following:

\begin{verbatim}
one=1
two=2
three="quotes have no 'special' meaning"
\end{verbatim}

If the environment is set with the \Opt{environment} command \emph{and}
\Opt{getenv} is also set to true, values specified with
\Opt{environment} override values in the submittor's environment
(regardless of the order of the \Opt{environment} and \Opt{getenv}
commands).


%%%%%%%%%%%%%%%%%%%
%% error
%%%%%%%%%%%%%%%%%%%

\index{submit commands!error}
\item[error = $<$pathname$>$]
A path and file name used by Condor to capture any
error messages the program would normally write to the screen
(that is, this file becomes \File{stderr}).
If not specified, the default value of
\File{/dev/null} is used for submission to a Unix machine.
If not specified, error messages are ignored
for submission to a Windows machine.
More than one job should not use the same error file, since
this will cause one job to overwrite the errors of another.
The error file and the output file should not be the same file
as the outputs will overwrite each other or be lost.
For grid universe jobs, \SubmitCmd{error} may be a URL that the Globus
tool \Prog{globus\_url\_copy} understands.


%%%%%%%%%%%%%%%%%%%
%% executable
%%%%%%%%%%%%%%%%%%%

\index{submit commands!executable}
\item[executable = $<$pathname$>$]
An optional path and a required file name of the executable file for this
job cluster. Only one \SubmitCmd{executable} command within a
submit description file is guaranteed to work properly.
More than one often works.

If no path or a relative path is used, then the executable file
is presumed to be relative
to the current working directory of the user as the
\Condor{submit} command is issued.

If submitting into the standard universe, then the named executable
must have been re-linked with the Condor libraries (such as via the
\Condor{compile} command). If submitting into the vanilla universe
(the default), then the named executable need not be re-linked and can
be any process which can run in the background (shell scripts work
fine as well).  If submitting into the Java universe, then the
argument must be a compiled \File{.class} file.

%%%%%%%%%%%%%%%%%%%
%% getenv
%%%%%%%%%%%%%%%%%%%

\index{submit commands!getenv}
\item[getenv = $<$True \Bar\ False$>$] If \SubmitCmd{getenv} is set to
\index{environment variables!copying current environment}
\Expr{True}, then \Condor{submit} will copy all of the user's current
shell environment variables at the time of job submission into the job
ClassAd. The job will therefore execute with the same set of environment
variables that the user had at submit time. Defaults to \Expr{False}.
%You must be careful when using this feature, since the maximum allowed
%size of the environment in Condor is 10240 characters.
%If your environment is larger than that, Condor will not allow you to
%submit your job, and you will have to use the ``Environment'' setting
%described below, instead.

If the environment is set with the \Opt{environment} command \emph{and}
\Opt{getenv} is also set to true, values specified with
\Opt{environment} override values in the submittor's environment
(regardless of the order of the \Opt{environment} and \Opt{getenv}
commands).


%%%%%%%%%%%%%%%%%%%
%% input
%%%%%%%%%%%%%%%%%%%

\index{submit commands!input}
\item[input = $<$pathname$>$]
Condor assumes that its jobs are
long-running, and that the user will not wait at the terminal for their
completion. Because of this, the standard files which normally access
the terminal, (\File{stdin}, \File{stdout}, and \File{stderr}),
must refer to files. Thus,
the file name specified with \SubmitCmd{input} should contain any keyboard
input the program requires (that is, this file becomes \File{stdin}).
If not specified, the default value
of \File{/dev/null} is used for submission to a Unix machine.
If not specified, input is ignored
for submission to a Windows machine.
For grid universe jobs, \SubmitCmd{input} may be a URL that the Globus
tool \Prog{globus\_url\_copy} understands.

Note that this command does \emph{not} refer to the command-line
arguments of the program.  The command-line arguments are specified by
the \SubmitCmd{arguments} command.

%%%%%%%%%%%%%%%%%%%
%% log
%%%%%%%%%%%%%%%%%%%

\index{submit commands!log}
\item[log = $<$pathname$>$] Use \SubmitCmd{log} to specify a file name where
Condor will write a log file of what is happening with this job cluster.
For example, Condor will place a log entry into this file
when and where the job begins running,
when the job produces a checkpoint, or moves (migrates) to another machine,
and when the job completes.
Most users find specifying a \SubmitCmd{log} file to be handy;
its use is recommended. If no \SubmitCmd{log} entry is specified, 
Condor does not create a log for this cluster.

%%%%%%%%%%%%%%%%%%%
%% log_xml
%%%%%%%%%%%%%%%%%%%

\index{submit commands!log\_xml}
\item[log\_xml = $<$True \Bar\ False$>$]
If \SubmitCmd{log\_xml} is \Expr{True}, 
then the log file will be written in ClassAd XML.
If not specified, XML is not used.
Note that the file is an XML fragment; it is
missing the file header and footer.
Do not mix XML and non-XML within a single file.
If multiple jobs write to a
single log file, ensure that all of the jobs specify
this option in the same way.

%%%%%%%%%%%%%%%%%%%
%% notification
%%%%%%%%%%%%%%%%%%%

\label{man-condor-submit-notification}
\item[notification = $<$Always \Bar\ Complete \Bar\ Error \Bar\ Never$>$]
\index{submit commands!notification}
Owners of Condor jobs are notified by
e-mail when certain events occur.
If defined by \Arg{Always}, the owner will be notified
whenever the job produces a checkpoint, as well as when the job completes.
If defined by \Arg{Complete} (the default), the owner will
be notified when the job terminates.
If defined by \Arg{Error}, the owner will only be notified
if the job terminates abnormally.
If defined by \Arg{Never}, the owner will not receive e-mail,
regardless to what happens to the job.
The statistics included in the e-mail are documented in
section~\ref{sec:job-completion} on
page~\pageref{sec:job-completion}.

%%%%%%%%%%%%%%%%%%%
%% notify_user
%%%%%%%%%%%%%%%%%%%

\index{submit commands!notify\_user}
\item[notify\_user = $<$email-address$>$]
\label{man-condor-submit-notify-user}
Used to specify the e-mail
address to use when Condor sends e-mail about a job.  If not specified,
Condor defaults to using the e-mail address defined by
\begin{verbatim}
job-owner@UID_DOMAIN
\end{verbatim}
where the configuration variable \Macro{UID\_DOMAIN}
is specified by the Condor site administrator.
If \Macro{UID\_DOMAIN} has not been specified,
Condor sends the e-mail to:
\begin{verbatim}
job-owner@submit-machine-name
\end{verbatim}

%%%%%%%%%%%%%%%%%%%
%% output
%%%%%%%%%%%%%%%%%%%

\index{submit commands!output}
\item[output = $<$pathname$>$]
\label{man-condor-submit-output}
The \SubmitCmd{output} file captures
any information the program would ordinarily write to the screen
(that is, this file becomes \File{stdout}).
If not specified, the default value of
\File{/dev/null} is used for submission to a Unix machine.
If not specified, output is ignored
for submission to a Windows machine.
Multiple jobs should not use the same output
file, since this will cause one job to overwrite the output of
another.
The output file and the error file should not be the same file
as the outputs will overwrite each other or be lost.
For grid universe jobs, \SubmitCmd{output} may be a URL that the Globus
tool \Prog{globus\_url\_copy} understands.

Note that if a program explicitly opens and writes to a file,
that file should \emph{not} be specified as the \SubmitCmd{output} file.

%%%%%%%%%%%%%%%%%%%
%% priority
%%%%%%%%%%%%%%%%%%%

\index{submit commands!priority}
\item[priority = $<$integer$>$] 
\label{man-condor-submit-priority}
A Condor job priority 
can be any integer, with 0 being the default.
Jobs with higher numerical priority will
run before jobs with lower numerical priority. Note that this priority
is on a per user basis.
One user with many jobs may use this command
to order his/her own jobs,
and this will have no effect on whether or
not these jobs will run ahead of another user's jobs.

%%%%%%%%%%%%%%%%%%%
%% queue
%%%%%%%%%%%%%%%%%%%

% these do not put the right bracket on the same line
% \item[queue \Lbr number-of-procs \Rbr] Places one or more
% \item[queue $[$number-of-procs$]$] Places one or more
% this also doesn't work, but Karen doesn't know what else to try,
% and it isn't worth the time.
\index{submit commands!queue}
\item[queue \oOptnm{number-of-procs}] Places one or more
\label{man-condor-submit-queue}
copies of the job into the Condor queue.
The optional
argument \Arg{number-of-procs} specifies how many times to submit the
job to the queue, and it defaults to 1.
If desired, any commands may be placed
between subsequent \SubmitCmd{queue} commands, such as new \SubmitCmd{input},
\SubmitCmd{output}, \SubmitCmd{error}, \SubmitCmd{initialdir},
or \SubmitCmd{arguments} commands.
This is handy when submitting multiple runs into one cluster with
one submit description file.

%%%%%%%%%%%%%%%%%%%
%% universe
%%%%%%%%%%%%%%%%%%%

\index{submit commands!universe}
\item[universe = $<$vanilla \Bar\ standard \Bar\ scheduler
\Bar\ local \Bar\ grid \Bar\ java \Bar\ vm$>$]
\label{man-condor-submit-universe}
Specifies which Condor Universe to use when running this job.  The Condor
Universe specifies a Condor execution environment.
The \SubmitCmd{standard} Universe tells Condor that this job has been
re-linked via \Condor{compile} with the Condor libraries and therefore
supports checkpointing and remote system calls.
The \SubmitCmd{vanilla} Universe is the default (except where the
configuration variable \Macro{DEFAULT\_UNIVERSE} defines it
otherwise), and is an execution environment for jobs which have not
been linked with the Condor libraries.  \emph{Note:} Use the
\SubmitCmd{vanilla} Universe to submit shell scripts to Condor.
The \SubmitCmd{scheduler} is for a job that
should act as a metascheduler.
The \SubmitCmd{grid} universe forwards the job to an external job
management system.
Further specification of the \SubmitCmd{grid} universe is done with the
\SubmitCmd{grid\_resource} command.
The \SubmitCmd{java} universe is for programs written to the Java Virtual Machine.
The \SubmitCmd{vm} universe facilitates the execution of a virtual
machine.

\end{description} 


%%%%%%%%%%%%%%%%%%%%%%%%%%%%%%%
\emph{COMMANDS FOR MATCHMAKING}
%%%%%%%%%%%%%%%%%%%%%%%%%%%%%%%
\begin{description} 

%%%%%%%%%%%%%%%%%%%
%% rank
%%%%%%%%%%%%%%%%%%%

\index{submit commands!rank}
\item[rank = $<$ClassAd Float Expression$>$]
A ClassAd Floating-Point
expression that states how to rank machines which have already met the requirements
expression. Essentially, rank expresses preference.  A higher numeric value
equals better rank. Condor will give the job the machine with the
highest rank.  For example,
\begin{verbatim}
        requirements = Memory > 60
        rank = Memory
\end{verbatim}
asks Condor to find all available machines with more than 60 megabytes of memory
and give to the job the machine with the most amount of memory.
See section~\ref{sec:user-man-req-and-rank} 
within the Condor Users
Manual for complete information on the syntax and available attributes
that can be used in the ClassAd expression.

%%%%%%%%%%%%%%%%%%%
%% requirements
%%%%%%%%%%%%%%%%%%%

\index{submit commands!requirements}
\item[requirements = $<$ClassAd Boolean Expression$>$]
The requirements
command is a boolean ClassAd expression which uses C-like operators. In
order for any job in this cluster to run on a given machine, this
requirements expression must evaluate to true on the given machine. For
example, to require that whatever machine executes a Condor job has a
least 64 Meg of RAM and has a MIPS performance rating greater than 45,
use:
\begin{verbatim}
requirements = Memory >= 64 && Mips > 45
\end{verbatim}
For scheduler and local universe jobs, the requirements expression is
evaluated against
the \Attr{Scheduler} ClassAd which represents the 
the \Condor{schedd} daemon running on the submit machine,
rather than a remote machine.
Like all commands in the submit description file, if multiple requirements
commands are present, all but the last one are ignored.
By default, \Condor{submit} appends the following clauses to
the requirements expression:
\begin{enumerate}
        \item Arch and OpSys are set equal to the Arch and OpSys of the
submit machine.  In other words: unless you request otherwise, Condor will give your
job machines with the same architecture and operating system version as
the machine running \Condor{submit}.
        \item Disk $>=$ DiskUsage.
The \AdAttr{DiskUsage} attribute is initialized to the size of the
executable plus the size of any files specified in a
\SubmitCmd{transfer\_input\_files} command.
It exists to ensure there is enough disk space on the
target machine for Condor to copy over both the executable
and needed input files.
The \AdAttr{DiskUsage} attribute represents the maximum amount of
total disk space required by the job in kilobytes.
Condor automatically updates the \AdAttr{DiskUsage} attribute
approximately every 20 minutes while the job runs with the
amount of space being used by the job on the execute machine.
        \item (Memory * 1024) $>=$ ImageSize.  To ensure the target machine
has enough memory to run your job.
        \item If Universe is set to Vanilla, FileSystemDomain is set equal to
the submit machine's FileSystemDomain.
\end{enumerate}
View the requirements of a job
which has already been submitted (along with everything else about the
job ClassAd) with the command \Condor{q -l}; see the command reference for
\Condor{q} on page~\pageref{man-condor-q}.  Also, see the Condor Users
Manual for complete information on the syntax and available attributes
that can be used in the ClassAd expression.


\end{description} 

%%%%%%%%%%%%%%%%%%%%%%%%%%%%%%%
\emph{FILE TRANSFER COMMANDS}
%%%%%%%%%%%%%%%%%%%%%%%%%%%%%%%
\begin{description} 

%%%%%%%%%%%%%%%%%%%
%% should_transfer_files
%%%%%%%%%%%%%%%%%%%

\index{submit commands!should\_transfer\_files}
\item[should\_transfer\_files = $<$YES \Bar\ NO \Bar\ IF\_NEEDED $>$] 
\index{file transfer mechanism}
The \SubmitCmd{should\_transfer\_files} setting is used to define if Condor
should transfer files to and from the remote machine where the job
runs.
The file transfer mechanism is used to run jobs which are not in the
standard universe (and can therefore use remote system calls for file
access) on machines which do not have a shared file system with the
submit machine.
\SubmitCmd{should\_transfer\_files} equal to \Arg{YES} will cause Condor to
always transfer files for the job.
\Arg{NO} disables Condor's file transfer mechanism.
\Arg{IF\_NEEDED} will not transfer files for the job if it is matched
with a resource in the same \Attr{FileSystemDomain} as the submit
machine (and therefore, on a machine with the same shared file
system).
If the job is matched with a remote resource in a different 
\Attr{FileSystemDomain}, Condor will transfer the necessary files. 

If defining \SubmitCmd{should\_transfer\_files} you \emph{must} also
define \SubmitCmd{when\_to\_transfer\_output} (described below).
For more information about this and other settings related to
transferring files, see section~\ref{sec:file-transfer} on
page~\pageref{sec:file-transfer}.

Note that \SubmitCmd{should\_transfer\_files} is not supported
for jobs submitted to the grid universe.

%%%%%%%%%%%%%%%%%%%
%% stream_error
%%%%%%%%%%%%%%%%%%%
\index{submit commands!stream\_error}
\item[stream\_error = $<$True \Bar\ False$>$]
If \Expr{True}, then \File{stderr} is streamed back to
the machine from which the job was submitted.
If \Expr{False}, \File{stderr} is stored locally
and transferred back when the job completes.
This command is ignored if the job ClassAd attribute
\Attr{TransferErr} is
\Expr{False}.
The default value is \Expr{True} in the grid
universe and \Expr{False} otherwise.
This command must be used in conjunction with 
\SubmitCmd{error}, otherwise \File{stderr} will
sent to \File{/dev/null} on Unix machines and
ignored on Windows machines.

%%%%%%%%%%%%%%%%%%%
%% stream_input
%%%%%%%%%%%%%%%%%%%
\index{submit commands!stream\_input}
\item[stream\_input = $<$True \Bar\ False$>$]
If \Expr{True}, then \File{stdin} is streamed from the
machine on which the job was submitted.
The default value is \Expr{False}.
The command is only relevant for jobs submitted to
the vanilla or java universes, and
it is ignored by the grid
universe.
This command must be used in conjunction with 
\SubmitCmd{input}, otherwise \File{stdin} will
be \File{/dev/null} on Unix machines and
ignored on Windows machines.

%%%%%%%%%%%%%%%%%%%
%% stream_output
%%%%%%%%%%%%%%%%%%%
\index{submit commands!stream\_output}
\item[stream\_output = $<$True \Bar\ False$>$]
If \Expr{True}, then \File{stdout} is streamed back to
the machine from which the job was submitted.
If \Expr{False}, \File{stdout} is stored locally
and transferred back when the job completes.
This command is ignored if the job ClassAd attribute
\Attr{TransferOut} is
\Expr{False}.
The default value is \Expr{True} in the grid
universe and \Expr{False} otherwise.
This command must be used in conjunction with 
\SubmitCmd{output}, otherwise \File{stdout} will
sent to \File{/dev/null} on Unix machines and
ignored on Windows machines.

%%%%%%%%%%%%%%%%%%%
%% transfer_executable
%%%%%%%%%%%%%%%%%%%

\index{submit commands!transfer\_executable}
\item[transfer\_executable = $<$True \Bar\ False$>$]
This command is applicable to jobs submitted to the grid,
and vanilla universes.
If \SubmitCmd{transfer\_executable} is set to
\Expr{False}, then Condor looks for the executable on the remote machine, and
does not transfer the executable over.
This is useful for an already pre-staged 
executable; Condor behaves more like rsh.
The default value is \Expr{True}.

%%%%%%%%%%%%%%%%%%%
%% transfer_input_files
%%%%%%%%%%%%%%%%%%%

\index{submit commands!transfer\_input\_files}
\item[transfer\_input\_files = $<$ file1,file2,file... $>$]
A comma-delimited list of all the files to be transferred into the
working directory for the job before the job is started.
By default, the file specified in the
\SubmitCmd{executable} command and any file specified in the \SubmitCmd{input}
command (for example, \File{stdin}) are transferred.

Only the transfer of files is available; the transfer of
subdirectories is not supported.

For more information about this and other settings related to
transferring files, see section~\ref{sec:file-transfer} on
page~\pageref{sec:file-transfer}.

%%%%%%%%%%%%%%%%%%%
%% transfer_output_files
%%%%%%%%%%%%%%%%%%%

\index{submit commands!transfer\_output\_files}
\item[transfer\_output\_files = $<$ file1,file2,file... $>$]
This command forms an explicit list of output files to be transferred
back from the temporary working directory on the execute machine to
the submit machine.
Most of the time, there is no need to use this command.
For grid universe jobs, other than where the grid type is \SubmitCmd{condor},
the use of \SubmitCmd{transfer\_output\_files} is useful.
For Condor-C jobs and all other non-grid universe jobs,
if \SubmitCmd{transfer\_output\_files} is not specified,
Condor will automatically transfer back all files in the job's
temporary working directory which have been
modified or created by the job.
This is usually the desired behavior.
Explicitly listing output files is typically only done when the job creates
many files, and the user wants to keep a subset of
those files.
If there are multiple files, they must be delimited with commas.
\Warn Do not specify \SubmitCmd{transfer\_output\_files} in the
submit description file unless there is a really good reason -- it is
best to let Condor figure things out by itself based upon what
the job produces.

For grid universe jobs other than with grid type \SubmitCmd{condor},
to have files other than standard output and standard error transferred
from the execute machine back to the submit machine,
do use \SubmitCmd{transfer\_output\_files}, listing
all files to be transferred.
These files are found on the execute machine in the
working directory of the job.

For more information about this and other settings related to
transferring files, see section~\ref{sec:file-transfer} on
page~\pageref{sec:file-transfer}.

%%%%%%%%%%%%%%%%%%%
%% transfer_output_remaps
%%%%%%%%%%%%%%%%%%%

\index{submit commands!transfer\_output\_remaps}
\item[transfer\_output\_remaps $=$ $<$ `` name $=$ newname ; name2 $=$ newname2 ... ''$>$ ]
This specifies the name (and optionally path) to use when downloading output
files from the completed job.  Normally, output files are transferred back
to the initial working directory with the same name they had in the execution
directory.  This gives you the option to save them with a different path
or name.  If you specify a relative path, the final path will be relative
to the job's initial working directory.

\Arg{name} describes an output file name produced by your job, and
\Arg{newname} describes the file name it should be downloaded to.
Multiple remaps can be specified by separating each with a semicolon.
If you wish to remap file names that contain equals signs or
semicolons, these special characters may be escaped with a backslash.


%%%%%%%%%%%%%%%%%%%
%% when_to_transfer_output
%%%%%%%%%%%%%%%%%%%

\index{submit commands!when\_to\_transfer\_output}
\item[when\_to\_transfer\_output = $<$ ON\_EXIT \Bar\ ON\_EXIT\_OR\_EVICT $>$] 

Setting \SubmitCmd{when\_to\_transfer\_output} equal to \Arg{ON\_EXIT} will
cause Condor to transfer the job's output files back to the submitting
machine only when the job completes (exits on its own).

The \Arg{ON\_EXIT\_OR\_EVICT} option is intended for fault tolerant
jobs which periodically save their own state and can restart where
they left off.
In this case, files are spooled to the submit machine any time the
job leaves a remote site, either because it exited on its own, or was
evicted by the Condor system for any reason prior to job completion.
The files spooled back are placed in a directory defined by
the value of the \MacroNI{SPOOL} configuration variable.
Any output files transferred back to the submit machine are
automatically sent back out again as input files if the job restarts.

For more information about this and other settings related to
transferring files, see section~\ref{sec:file-transfer} on
page~\pageref{sec:file-transfer}.

\end{description} 

%%%%%%%%%%%%%%%%%%%%%%%%%%%%%%%
\emph{POLICY COMMANDS}
%%%%%%%%%%%%%%%%%%%%%%%%%%%%%%%
\begin{description} 

%%%%%%%%%%%%%%%%%%%
%% hold
%%%%%%%%%%%%%%%%%%%

\index{submit commands!hold}
\item[hold = $<$True \Bar\ False$>$] If \SubmitCmd{hold} is set to
\Expr{True}, then the submitted job will be placed into the Hold state.
Jobs in the Hold state will not run until released by \Condor{release}.
Defaults to \Expr{False}.

%%%%%%%%%%%%%%%%%%%
%% leave_in_queue
%%%%%%%%%%%%%%%%%%%

\index{submit commands!leave\_in\_queue}
\item[leave\_in\_queue = $<$ClassAd Boolean Expression$>$]
When the ClassAd Expression evaluates to \Expr{True}, the job is
not removed from the queue upon completion.
This allows the user of a remotely spooled job to retrieve output
files in cases where Condor would have removed them as part of
the cleanup associated with completion. The job will only exit
the queue once it has been marked for removal (via \Condor{rm},
for example) and the \SubmitCmd{leave\_in\_queue} expression has
become \Expr{False}.
\SubmitCmd{leave\_in\_queue} defaults to \Expr{False}.

%%%%%%%%%%%%%%%%%%%
%% on_exit_hold
%%%%%%%%%%%%%%%%%%%

\index{submit commands!on\_exit\_hold}
\item[on\_exit\_hold = $<$ClassAd Boolean Expression$>$] 
The ClassAd expression is checked when the job exits, and if \Expr{True},
places the job into the Hold state.
If \Expr{False} (the default value when not defined),
then nothing happens and the \AdAttr{on\_exit\_remove} expression is
checked to determine if that needs to be applied.

For example:
Suppose a job is known to run for a minimum of an hour.
If the job exits after less than an hour, the job should be placed on
hold and an e-mail notification sent,
instead of being allowed to leave the queue.

\footnotesize
\begin{verbatim}
  on_exit_hold = (CurrentTime - JobStartDate) < (60 * $(MINUTE))
\end{verbatim}
\normalsize

This expression places the job on hold if it exits for any reason
before running for an hour. An e-mail will be sent to the user explaining
that the job was placed on hold because this expression became \Expr{True}.

\AdAttr{periodic\_*} expressions take
precedence over \AdAttr{on\_exit\_*} expressions,
and \AdAttr{*\_hold} expressions take
precedence over a \AdAttr{*\_remove} expressions.

Only job ClassAd attributes will be defined for use by this ClassAd expression.
This expression is available for the vanilla, java, parallel, grid,
local and scheduler universes.
It is additionally available, when submitted from a Unix machine,
for the standard universe.

%%%%%%%%%%%%%%%%%%%
%% on_exit_remove
%%%%%%%%%%%%%%%%%%%

\index{submit commands!on\_exit\_remove}
\item[on\_exit\_remove = $<$ClassAd Boolean Expression$>$] 
The ClassAd expression is checked when the job exits, and if \Expr{True}
(the default value when undefined),
then it allows the job to leave the queue normally.
If \Expr{False}, then the job is placed back into the Idle state.
If the user job runs under the vanilla universe,
then the job restarts from the beginning.
If the user job runs under the standard universe,
then it continues from where it left off, using the last checkpoint.

For example,
suppose a job occasionally segfaults,
but chances are that the job will finish successfully
if the job is run again with the same data.
The \SubmitCmd{on\_exit\_remove} expression can cause the job
to run again with the following command.
Assume that the signal identifier for the segmentation fault is 11 on the
platform where the job will be running.
\footnotesize
\begin{verbatim}
  on_exit_remove = (ExitBySignal == False) || (ExitSignal != 11)
\end{verbatim}
\normalsize
This expression lets the job leave the queue if the job was
not killed by a signal or if it was
killed by a signal other than 11, representing segmentation fault in
this example.
So, if the exited due to signal 11, it will stay in the job queue.
In any other case of the job exiting,
the job will leave the queue as it normally would have done.

As another example,
if the job should only leave the queue if it exited on its own with
status 0,
this \SubmitCmd{on\_exit\_remove} expression works well:

\footnotesize
\begin{verbatim}
  on_exit_remove = (ExitBySignal == False) && (ExitCode == 0)
\end{verbatim}
\normalsize
If the job was killed by a signal or exited with a non-zero exit
status, Condor would leave the job in the queue to run again.

\AdAttr{periodic\_*} expressions take
precedence over \AdAttr{on\_exit\_*} expressions,
and \AdAttr{*\_hold} expressions take
precedence over a \AdAttr{*\_remove} expressions.

Only job ClassAd attributes will be defined for use by this ClassAd expression.
This expression is available for the vanilla, java, parallel, grid,
local and scheduler universes.
It is additionally available, when submitted from a Unix machine, for the
standard universe.  Note that the \Condor{schedd} daemon,
by default, only checks
these periodic expressions once every 300 seconds.  The period of
these evaluations can be adjusted by setting the
\MacroNI{PERIODIC\_EXPR\_INTERVAL} configuration macro.


%%%%%%%%%%%%%%%%%%%
%% periodic_hold
%%%%%%%%%%%%%%%%%%%

\index{submit commands!periodic\_hold}
\item[periodic\_hold = $<$ClassAd Boolean Expression$>$]
This expression is checked periodically at an interval of
the number of seconds set by
the configuration variable \MacroNI{PERIODIC\_EXPR\_INTERVAL}.
If it becomes \Expr{True}, the job will be placed on hold.
If unspecified, the default value is \Expr{False}.

\AdAttr{periodic\_*} expressions take
precedence over \AdAttr{on\_exit\_*} expressions,
and \AdAttr{*\_hold} expressions take
precedence over a \AdAttr{*\_remove} expressions.

Only job ClassAd attributes will be defined for use by this ClassAd expression.
This expression is available for the vanilla, java, parallel, grid,
local and scheduler universes.
It is additionally available, when submitted from a Unix machine,
for the standard universe.  Note that the \Condor{schedd} daemon,
by default, only checks
these periodic expressions once every 300 seconds.  The period of
these evaluations can be adjusted by setting the
\MacroNI{PERIODIC\_EXPR\_INTERVAL} configuration macro.

%%%%%%%%%%%%%%%%%%%
%% periodic_release
%%%%%%%%%%%%%%%%%%%

\index{submit commands!periodic\_release}
\item[periodic\_release = $<$ClassAd Boolean Expression$>$]
This expression is checked periodically at an interval of
the number of seconds set by
the configuration variable \MacroNI{PERIODIC\_EXPR\_INTERVAL}
while the job is in the Hold state.
If the expression becomes \Expr{True}, the job will be released.

% The expression is evaluated:
%- every time there is a condor_reconfig
%- every time condor_reschedule happens
%- every time condor_submit happens
%- BUT never more often than condor_config SCHEDD_MIN_INTERVAL seconds
%(default of 5 seconds)
%- AND never less often than condor_config SCHEDD_INTERVAL seconds (defaults
%to 5 minutes).
%
%So the short answer: by default, every SCHEDD_INTERVAL (5 minutes by
%default) or more often as needed.

Only job ClassAd attributes will be defined for use by this ClassAd expression.
This expression is available for the vanilla, java, parallel, grid,
local and scheduler universes.
It is additionally available, when submitted from a Unix machine,
for the standard universe.  Note that the \Condor{schedd} daemon,
by default, only checks
periodic expressions once every 300 seconds.  The period of
these evaluations can be adjusted by setting the
\MacroNI{PERIODIC\_EXPR\_INTERVAL} configuration macro.

%%%%%%%%%%%%%%%%%%%
%% periodic_remove
%%%%%%%%%%%%%%%%%%%

\index{submit commands!periodic\_remove}
\item[periodic\_remove = $<$ClassAd Boolean Expression$>$]
This expression is checked periodically at an interval of
the number of seconds set by
the configuration variable \MacroNI{PERIODIC\_EXPR\_INTERVAL}.
If it becomes \Expr{True}, the job is removed from the queue.
If unspecified, the default value is \Expr{False}.

See section~\ref{condor-submit-examples},
the Examples section of the \Condor{submit} manual page,
for an example of a \SubmitCmd{periodic\_remove} expression.

\AdAttr{periodic\_*} expressions take
precedence over \AdAttr{on\_exit\_*} expressions,
and \AdAttr{*\_hold} expressions take
precedence over a \AdAttr{*\_remove} expressions.
So, the \AdAttr{periodic\_remove} expression takes precedent over
the \AdAttr{on\_exit\_remove} expression,
if the two describe conflicting actions.

Only job ClassAd attributes will be defined for use by this ClassAd expression.
This expression is available for the vanilla, java, parallel, grid,
local and scheduler universes.
It is additionally available, when submitted from a Unix machine,
for the standard universe.  Note that the \Condor{schedd} daemon,
by default, only checks
periodic expressions once every 300 seconds.  The period of
these evaluations can be adjusted by setting the
\MacroNI{PERIODIC\_EXPR\_INTERVAL} configuration macro.

%%%%%%%%%%%%%%%%%%%
%% next_job_start_delay
%%%%%%%%%%%%%%%%%%%

\index{submit commands!next\_job\_start\_delay}
\item[next\_job\_start\_delay = $<$ClassAd Boolean Expression$>$]
This expression specifies the number of seconds to delay after starting up
this job before the next job is started.  The maximum
allowed delay is specified by the Condor configuration variable
\Macro{MAX\_NEXT\_JOB\_START\_DELAY}, which defaults to 10 minutes.
This command does not apply to \SubmitCmd{scheduler}
or \SubmitCmd{local} universe jobs.

This command has been historically used to implement a form
of job start throttling from the job submitter's perspective.
It was effective for the case of multiple job submission where
the transfer of extremely large input data sets to the execute
machine caused machine performance to suffer.
This command is no longer useful, as throttling should be
accomplished through configuration of the \Condor{schedd} daemon.

\end{description} 



%%%%%%%%%%%%%%%%%%%%%%%%%%%%%%%
\emph{COMMANDS SPECIFIC TO THE STANDARD UNIVERSE}
%%%%%%%%%%%%%%%%%%%%%%%%%%%%%%%

\begin{description} 

%%%%%%%%%%%%%%%%%%%
%% allow_startup_script
%%%%%%%%%%%%%%%%%%%
\index{submit commands!allow\_startup\_script}
\item[allow\_startup\_script = $<$True \Bar\ False$>$]
If True, a standard universe job will execute a script
instead of submitting the job,
and the consistency check to see if the executable has
been linked using \Condor{compile} is omitted.
The \SubmitCmd{executable} command within the submit description
file specifies the name of the script.
The script is used to do preprocessing before the
job is submitted.
The shell script ends with an \Prog{exec} of the
job executable, such that the process id of the executable is the
same as that of the shell script.
Here is an example script that gets a copy of a machine-specific
executable before the \Prog{exec}.
\footnotesize
\begin{verbatim} 
   #! /bin/sh

   # get the host name of the machine
   $host=`uname -n`

   # grab a standard universe executable designed specifically
   # for this host
   scp elsewhere@cs.wisc.edu:${host} executable

   # The PID MUST stay the same, so exec the new standard universe process.
   exec executable ${1+"$@"}
\end{verbatim} 
\normalsize
If this command is not present (defined), then the value
defaults to false.

%%%%%%%%%%%%%%%%%%%
%% append_files
%%%%%%%%%%%%%%%%%%%

\index{submit commands!append\_files}
\item[append\_files = file1, file2, ...]

If your job attempts to access a file mentioned in this list,
Condor will force all writes to that file to be appended to the end.
Furthermore, condor\_submit will not truncate it.
This list uses the same syntax as compress\_files, shown above.

This option may yield some surprising results.  If several
jobs attempt to write to the same file, their output may be intermixed.
If a job is evicted from one or more machines during the course of its
lifetime, such an output file might contain several copies of the results.
This option should be only be used when you wish a certain file to be
treated as a running log instead of a precise result.

This option only applies to standard-universe jobs.

%%%%%%%%%%%%%%%%%%%
%% buffer_files, buffer_size, buffer_block_size
%%%%%%%%%%%%%%%%%%%

\index{submit commands!buffer\_files}
\index{submit commands!buffer\_size}
\index{submit commands!buffer\_block\_size}
\item[buffer\_files $=$ $<$ `` name $=$ (size,block-size) ; name2 $=$ (size,block-size) ... '' $>$ ]
\item[buffer\_size $=$ $<$bytes-in-buffer$>$]
\item[buffer\_block\_size $=$ $<$bytes-in-block$>$]
Condor keeps a buffer of recently-used data for each file a job accesses.
This buffer is used both to cache commonly-used data and to consolidate small
reads and writes into larger operations that get better throughput.
The default settings should produce reasonable results for most programs.

These options only apply to standard-universe jobs.

If needed, you may set the buffer controls individually for each file using
the buffer\_files option. For example, to set the buffer size to 1 Mbyte and
the block size to 256 Kbytes for the file \File{input.data}, use this command:

\begin{verbatim}
buffer_files = "input.data=(1000000,256000)"
\end{verbatim}

Alternatively, you may use these two options to set
the default sizes for all files used by your job:

\begin{verbatim}
buffer_size = 1000000
buffer_block_size = 256000
\end{verbatim}

If you do not set these, Condor will use the values given by these
two configuration file macros:

\begin{verbatim}
DEFAULT_IO_BUFFER_SIZE = 1000000
DEFAULT_IO_BUFFER_BLOCK_SIZE = 256000
\end{verbatim}

Finally, if no other settings are present, Condor will use
a buffer of 512 Kbytes
and a block size of 32 Kbytes.

%%%%%%%%%%%%%%%%%%%
%% compress_files
%%%%%%%%%%%%%%%%%%%

\index{submit commands!compress\_files}
\item[compress\_files = file1, file2, ...]

If your job attempts to access any of the files mentioned in this list,
Condor will automatically compress them (if writing) or decompress them (if reading).
The compress format is the same as used by GNU gzip.

The files given in this list may be simple file names or complete paths and may
include $*$ as a wild card.  For example, this list causes the file /tmp/data.gz,
any file named event.gz, and any file ending in .gzip to be automatically
compressed or decompressed as needed:

\begin{verbatim}
compress_files = /tmp/data.gz, event.gz, *.gzip
\end{verbatim}
Due to the nature of the compression format, compressed files must only
be accessed sequentially.  Random access reading is allowed but is very slow,
while random access writing is simply not possible.  This restriction may be
avoided by using both compress\_files and fetch\_files at the same time.  When
this is done, a file is kept in the decompressed state at the execution
machine, but is compressed for transfer to its original location.

This option only applies to standard universe jobs.

%%%%%%%%%%%%%%%%%%%
%% fetch_files
%%%%%%%%%%%%%%%%%%%

\index{submit commands!fetch\_files}
\item[fetch\_files = file1, file2, ...]
If your job attempts to access a file mentioned in this list,
Condor will automatically copy the whole file to the executing machine,
where it can be accessed quickly.  When your job closes the file,
it will be copied back to its original location.
This list uses the same syntax as compress\_files, shown above.

This option only applies to standard universe jobs.

%%%%%%%%%%%%%%%%%%%
%% file_remaps
%%%%%%%%%%%%%%%%%%%

\index{submit commands!file\_remaps}
\item[file\_remaps $=$ $<$ `` name $=$ newname ; name2 $=$ newname2 ... ''$>$ ]

Directs Condor to use a new file name in place of an old one.  \Arg{name}
describes a file name that your job may attempt to open, and \Arg{newname}
describes the file name it should be replaced with.
\Arg{newname} may include an optional leading
access specifier, \verb@local:@ or \verb@remote:@.  If left unspecified,
the default access specifier is \verb@remote:@.  Multiple remaps can be
specified by separating each with a semicolon.

This option only applies to standard universe jobs.

If you wish to remap file names that contain equals signs or semicolons,
these special characters may be escaped with a backslash.

\begin{description}
\item[Example One:]
Suppose that your job reads a file named \File{dataset.1}.
To instruct Condor
to force your job to read \File{other.dataset} instead,
add this to the submit file:
\begin{verbatim}
file_remaps = "dataset.1=other.dataset"
\end{verbatim}
\item[Example Two:]
Suppose that your run many jobs which all read in the same large file,
called \File{very.big}.
If this file can be found in the same place on
a local disk in every machine in the pool,
(say \File{/bigdisk/bigfile},) you can
instruct Condor of this fact by remapping \File{very.big} to
\File{/bigdisk/bigfile} and specifying that the file is to be read locally,
which will be much faster than reading over the network.
\begin{verbatim}
file_remaps = "very.big = local:/bigdisk/bigfile"
\end{verbatim}
\item[Example Three:]
Several remaps can be applied at once by separating each with a semicolon.
\footnotesize
\begin{verbatim}
file_remaps = "very.big = local:/bigdisk/bigfile ; dataset.1 = other.dataset"
\end{verbatim}
\normalsize
\end{description}


%%%%%%%%%%%%%%%%%%%
%% local_files
%%%%%%%%%%%%%%%%%%%

\index{submit commands!local\_files}
\item[local\_files = file1, file2, ...]

If your job attempts to access a file mentioned in this list,
Condor will cause it to be read or written at the execution machine.
This is most useful for temporary files not used for input or output.
This list uses the same syntax as compress\_files, shown above.

\begin{verbatim}
local_files = /tmp/*
\end{verbatim}

This option only applies to standard universe jobs.

%%%%%%%%%%%%%%%%%%%
%% want_remote_io
%%%%%%%%%%%%%%%%%%%

\index{submit commands!want\_remote\_io}
\item[want\_remote\_io = $<$True \Bar\ False$>$]
This option controls how a file is opened and manipulated in a standard
universe job.
If this option is true, which is the default, then the \Condor{shadow}
makes all decisions about how each and every file should be opened by
the executing job.
This entails a network round trip (or more) from the job to the
\Condor{shadow} and back again for every single \Syscall{open}
in addition to other needed information about the file.
If set to false, then when the job queries the \Condor{shadow} for the
first time about how to open a file, the \Condor{shadow} will inform the
job to automatically perform all of its file manipulation on the local
file system on the execute machine and any file remapping will be ignored.
This means that there \Bold{must} be a shared file system (such
as NFS or AFS) between the execute machine and the submit machine and that
\Bold{ALL} paths that the job could open on the execute machine must be valid.
The ability of the standard universe job to checkpoint, possibly to a
checkpoint server, is not affected by this attribute.
However, when the job resumes it will be expecting the same file system
conditions that were present when the job checkpointed.

\end{description} 

%%%%%%%%%%%%%%%%%%%%%%%%%%%%%%%
\emph{COMMANDS FOR THE GRID}
%%%%%%%%%%%%%%%%%%%%%%%%%%%%%%%
\begin{description} 

%%%%%%%%%%%%%%%%%%%
%% amazon_ami_id
%%%%%%%%%%%%%%%%%%%

\index{submit commands!amazon\_ami\_id}
\item[amazon\_ami\_id = $<$Amazon EC2 AMI ID$>$]
AMI identifier of the VM image to run for \SubmitCmd{amazon} jobs.

%%%%%%%%%%%%%%%%%%%
%% amazon_instance_type
%%%%%%%%%%%%%%%%%%%

\index{submit commands!amazon\_instance\_type}
\item[amazon\_instance\_type = $<$VM Type$>$]
Identifier for the type of VM desired for an \SubmitCmd{amazon} job.
The default value is ``m1.small''.

%%%%%%%%%%%%%%%%%%%
%% amazon_keypair_file
%%%%%%%%%%%%%%%%%%%

\index{submit commands!amazon\_keypair\_file}
\item[amazon\_keypair\_file = $<$pathname$>$]
The complete path and filename of a file into which Condor will write an
ssh key for use with \SubmitCmd{amazon} jobs. The key can be used to
ssh into the virtual machine once it is running.

%%%%%%%%%%%%%%%%%%%
%% amazon_private_key
%%%%%%%%%%%%%%%%%%%

\index{submit commands!amazon\_private\_key}
\item[amazon\_private\_key = $<$pathname$>$]
Used for \SubmitCmd{amazon} jobs. Path and filename of a file containing
the private key to be used to authenticate with Amazon's EC2 service
via SOAP.

%%%%%%%%%%%%%%%%%%%
%% amazon_public_key
%%%%%%%%%%%%%%%%%%%

\index{submit commands!amazon\_public\_key}
\item[amazon\_public\_key = $<$pathname$>$]
Used for \SubmitCmd{amazon} jobs. Path and filename of a file containing
the public X509 certificate to be used to authenticate with Amazon's EC2
service via SOAP.

%%%%%%%%%%%%%%%%%%%
%% amazon_security_groups
%%%%%%%%%%%%%%%%%%%

\index{submit commands!amazon\_security\_groups}
\item[amazon\_security\_groups = group1, group2, ...]
Used for \SubmitCmd{amazon} jobs. A list of Amazon EC2 security group
names, which should be associated with the job.

%%%%%%%%%%%%%%%%%%%
%% amazon_user_data
%%%%%%%%%%%%%%%%%%%

\index{submit commands!amazon\_user\_data}
\item[amazon\_user\_data = $<$data$>$]
Used for \SubmitCmd{amazon} jobs. A block of data that can be accessed
by the virtual machine job inside Amazon EC2.

%%%%%%%%%%%%%%%%%%%
%% amazon_user_data_file
%%%%%%%%%%%%%%%%%%%

\index{submit commands!amazon\_user\_data\_file}
\item[amazon\_user\_data\_file = $<$pathname$>$]
Used for \SubmitCmd{amazon} jobs.
A file containing data that can be accessed
by the virtual machine job inside Amazon EC2.

%%%%%%%%%%%%%%%%%%%
%% globus_rematch
%%%%%%%%%%%%%%%%%%%

\label{condor-submit-globus-rematch}
\index{submit commands!globus\_rematch}
\item[globus\_rematch = $<$ClassAd Boolean Expression$>$]
This expression is evaluated by the \Condor{gridmanager} whenever:
\begin{enumerate}
\item
   the \SubmitCmd{globus\_resubmit} expression evaluates to \Expr{True}
\item
   the \Condor{gridmanager} decides it needs to retry a submission
   (as when a previous submission failed to commit)
\end{enumerate}
If \SubmitCmd{globus\_rematch} evaluates to \Expr{True},
then \emph{before} the job is submitted again to globus,
the \Condor{gridmanager} will request that the \Condor{schedd}
daemon renegotiate
with the matchmaker (the \Condor{negotiator}).
The result is this job will be matched again.

%%%%%%%%%%%%%%%%%%%
%% globus_resubmit
%%%%%%%%%%%%%%%%%%%

\label{condor-submit-globus-resubmit}
\index{submit commands!globus\_resubmit}
\item[globus\_resubmit = $<$ClassAd Boolean Expression$>$]
The expression is evaluated by the \Condor{gridmanager} each time
the \Condor{gridmanager} gets a job ad to manage.
Therefore, the expression is evaluated:
\begin{enumerate}
\item
   when a grid universe job is first submitted to Condor-G
\item
   when a grid universe job is released from the hold state
\item
   when Condor-G is restarted (specifically, whenever the \Condor{gridmanager}
   is restarted)
\end{enumerate}
If the expression evaluates to \Expr{True},
then any previous submission to the grid universe will be
forgotten and this job will be submitted again as a fresh submission to
the grid universe.
This may be useful if there is a desire to give up on a
previous submission and try again.
Note that this may result in the same job running more than
once.  Do not treat this operation lightly.

%%%%%%%%%%%%%%%%%%%
%% globus_rsl
%%%%%%%%%%%%%%%%%%%

\index{submit commands!globus\_rsl}
\item[globus\_rsl = $<$RSL-string$>$]
Used to provide any additional Globus RSL
string attributes which are not covered by other submit description
file commands or job attributes. Used for \SubmitCmd{grid} \SubmitCmd{universe}
jobs, where the grid resource has a \SubmitCmd{grid-type-string} of
\SubmitCmd{gt2}.

%%%%%%%%%%%%%%%%%%%
%% globus_xml
%%%%%%%%%%%%%%%%%%%

\index{submit commands!globus\_xml}
\item[globus\_xml = $<$XML-string$>$]
Used to provide any additional attributes in the GRAM XML job description
that Condor writes which are not covered by regular submit description
file parameters. Used for grid type \SubmitCmd{gt4} jobs.

%%%%%%%%%%%%%%%%%%%
%% grid_resource
%%%%%%%%%%%%%%%%%%%
\index{submit commands!grid\_resource}
\item[grid\_resource = $<$grid-type-string$>$ $<$grid-specific-parameter-list$>$ ]
For each \SubmitCmd{grid-type-string} value, 
there are further type-specific values that must specified.
This submit description file command allows each to
be given in a space-separated list.
Allowable \SubmitCmd{grid-type-string} values are
\SubmitCmd{amazon}, \SubmitCmd{condor}, \SubmitCmd{cream}, 
\SubmitCmd{gt2}, \SubmitCmd{gt4}, \SubmitCmd{gt5},
\SubmitCmd{lsf}, \SubmitCmd{nordugrid}, \SubmitCmd{pbs},
and \SubmitCmd{unicore}.
See section~\ref{sec:GridUniverse} for details on the variety of 
grid types.

For a \SubmitCmd{grid-type-string} of \SubmitCmd{amazon}, no additional
parameters are used.
See section~\ref{sec:Amazon} for details.

For a \SubmitCmd{grid-type-string} of \SubmitCmd{condor},
the first parameter is the name of the remote \Condor{schedd}
daemon.
The second parameter is the name of the pool to which the remote
\Condor{schedd} daemon belongs.
See section~\ref{sec:Condor-C-Submit} for details.

For a \SubmitCmd{grid-type-string} of \SubmitCmd{cream},
there are three parameters.
The first parameter is the web services address of the CREAM server.
The second parameter is the 
name of the batch system that sits behind the CREAM server.
The third parameter identifies a site-specific queue
within the batch system.
See section~\ref{sec:CREAM} for details.

For a \SubmitCmd{grid-type-string} of \SubmitCmd{gt2},
the single parameter is the name of the pre-WS GRAM resource to be used.
See section~\ref{sec:Using-gt2} for details.

For a \SubmitCmd{grid-type-string} of \SubmitCmd{gt4},
the first parameter is the name of the WS GRAM service to be used.
The second parameter is the name of WS resource to be used (usually the
name of the back-end scheduler).
See section~\ref{sec:Using-gt4} for details.

For a \SubmitCmd{grid-type-string} of \SubmitCmd{gt5},
the single parameter is the name of the pre-WS GRAM resource to be used,
which is the same as for the \SubmitCmd{grid-type-string} of \SubmitCmd{gt2}.
See section~\ref{sec:Using-gt5} for details.

For a \SubmitCmd{grid-type-string} of \SubmitCmd{lsf}, no additional
parameters are used.
See section~\ref{sec:LSF} for details.

For a \SubmitCmd{grid-type-string} of \SubmitCmd{nordugrid},
the single parameter is the name of the NorduGrid resource to be used.
See section~\ref{sec:NorduGrid} for details.

For a \SubmitCmd{grid-type-string} of \SubmitCmd{pbs}, no additional
parameters are used.
See section~\ref{sec:PBS} for details.

For a \SubmitCmd{grid-type-string} of \SubmitCmd{unicore},
the first parameter is the name of the Unicore Usite to be used.
The second parameter is the name of the Unicore Vsite to be used.
See section~\ref{sec:Unicore} for details.


%%%%%%%%%%%%%%%%%%%
%% keystore_alias
%%%%%%%%%%%%%%%%%%%

\index{submit commands!keystore\_alias}
\item[keystore\_alias = $<$name$>$]
A string to locate the certificate in a Java keystore file,
as used for a \SubmitCmd{unicore} job.

%%%%%%%%%%%%%%%%%%%
%% keystore_file
%%%%%%%%%%%%%%%%%%%

\index{submit commands!keystore\_file}
\item[keystore\_file = $<$pathname$>$]
The complete path and file name of the Java keystore file
containing the certificate to be used for a \SubmitCmd{unicore} job.

%%%%%%%%%%%%%%%%%%%
%% keystore_passphrase_file
%%%%%%%%%%%%%%%%%%%
\index{submit commands!keystore\_passphrase\_file}
\item[keystore\_passphrase\_file = $<$pathname$>$]
The complete path and file name
to the file containing the passphrase protecting a Java keystore
file containing the certificate.
Relevant for a \SubmitCmd{unicore} job.

%%%%%%%%%%%%%%%%%%%
%% MyProxyCredentialName
%%%%%%%%%%%%%%%%%%%

\index{submit commands!MyProxyCredentialName}
\item[MyProxyCredentialName = $<$symbolic name$>$]
The symbolic name that identifies a credential to the \Prog{MyProxy} server.
This symbolic name is set as the credential is
initially stored on the server (using \Prog{myproxy-init}).


%%%%%%%%%%%%%%%%%%%
%% MyProxyHost
%%%%%%%%%%%%%%%%%%%

\index{submit commands!MyProxyHost}
\item[MyProxyHost = $<$host$>$:$<$port$>$]
The Internet address of the host that is the \Prog{MyProxy} server.
The \SubmitCmd{host} may be specified by either a host name
(as in \File{head.example.com}) or an IP address
(of the form 123.456.7.8).
The \SubmitCmd{port} number is an integer.

%%%%%%%%%%%%%%%%%%%
%% MyProxyNewProxyLifetime
%%%%%%%%%%%%%%%%%%%

\index{submit commands!MyProxyNewProxyLifetime}
\item[MyProxyNewProxyLifetime = $<$number-of-minutes$>$]
The new lifetime (in minutes) of the proxy after it is refreshed.

%%%%%%%%%%%%%%%%%%%
%% MyProxyPassword
%%%%%%%%%%%%%%%%%%%

\index{submit commands!MyProxyPassword}
\item[MyProxyPassword = $<$password$>$]
The password needed to refresh a credential on the \Prog{MyProxy} server.
This password is set when the user initially stores
credentials on the server (using \Prog{myproxy-init}).
As an alternative to using \SubmitCmd{MyProxyPassword} in the
submit description file,
the password may be specified as a command line argument to \Condor{submit}
with the \Arg{-password} argument.

%%%%%%%%%%%%%%%%%%%
%% MyProxyRefreshThreshold
%%%%%%%%%%%%%%%%%%%

\index{submit commands!MyProxyRefreshThreshold}
\item[MyProxyRefreshThreshold = $<$number-of-seconds$>$]
The time (in seconds) before the expiration of a proxy 
that the proxy should be refreshed.
For example, if \SubmitCmd{MyProxyRefreshThreshold} is set to the
value 600, the proxy will be refreshed 10 minutes before
it expires.

%%%%%%%%%%%%%%%%%%%
%% MyProxyServerDN
%%%%%%%%%%%%%%%%%%%

\index{submit commands!MyProxyServerDN}
\item[MyProxyServerDN = $<$credential subject$>$]
A string that specifies the expected Distinguished Name (credential subject,
abbreviated DN)
of the \Prog{MyProxy} server.
It must be specified when the \Prog{MyProxy} server
DN does not follow the
conventional naming scheme of a host credential.
This occurs, for
example, when the  \Prog{MyProxy} server DN begins with a user credential.


%%%%%%%%%%%%%%%%%%%
%% nordugrid_rsl
%%%%%%%%%%%%%%%%%%%

\index{submit commands!nordugrid\_rsl}
\item[nordugrid\_rsl = $<$RSL-string$>$]
Used to provide any additional RSL
string attributes which are not covered by regular submit description
file parameters. Used when the \SubmitCmd{universe} is \SubmitCmd{grid},
and the type of grid system is \SubmitCmd{nordugrid}.


%%%%%%%%%%%%%%%%%%%
%% transfer_error
%%%%%%%%%%%%%%%%%%%
\index{submit commands!transfer\_error}
\item[transfer\_error = $<$True \Bar\ False$>$]
For jobs submitted to the grid universe only.
If \Expr{True}, then the error output (from \File{stderr}) from the job
is transferred from the remote machine back to the submit machine.
The name of the file after transfer is given
by the \SubmitCmd{error} command.
If \Expr{False}, no transfer takes place (from the remote machine
to submit machine),
and the name of the file is given
by the \SubmitCmd{error} command.
The default value is \Expr{True}.

%%%%%%%%%%%%%%%%%%%
%% transfer_input
%%%%%%%%%%%%%%%%%%%
\index{submit commands!transfer\_input}
\item[transfer\_input = $<$True \Bar\ False$>$]
For jobs submitted to the grid universe only.
If \Expr{True}, then the job input (\File{stdin}) is transferred
from the machine where the job was submitted to the remote machine.
The name of the file that is transferred is given by the
\SubmitCmd{input} command.
If \Expr{False}, then the job's input is taken from a pre-staged
file on the remote machine, and
the name of the file is given by the \SubmitCmd{input} command.
The default value is \Expr{True}.

For transferring files other than \File{stdin},
see \SubmitCmd{transfer\_input\_files}.

%%%%%%%%%%%%%%%%%%%
%% transfer_output
%%%%%%%%%%%%%%%%%%%
\index{submit commands!transfer\_output}
\item[transfer\_output = $<$True \Bar\ False$>$]
For jobs submitted to the grid universe only.
If \Expr{True}, then the output (from \File{stdout}) from the job
is transferred from the remote machine back to the submit machine.
The name of the file after transfer is given
by the \SubmitCmd{output} command.
If \Expr{False}, no transfer takes place (from the remote machine
to submit machine),
and the name of the file is given
by the \SubmitCmd{output} command.
The default value is \Expr{True}.

For transferring files other than \File{stdout},
see \SubmitCmd{transfer\_output\_files}.

%%%%%%%%%%%%%%%%%%%
%% x509userproxy
%%%%%%%%%%%%%%%%%%%

\index{submit commands!x509userproxy}
\item[x509userproxy = $<$full-pathname$>$] Used to override the default
path name for X.509 user certificates. The default location for X.509 proxies
is the \File{/tmp} directory,
which is generally a local file system.
Setting
this value would allow Condor to access the proxy in a shared file system
(for example, AFS).
Condor will use the proxy specified in the submit description file first.
If nothing is specified in the submit description file,
it will use the environment variable X509\_USER\_CERT.
If that variable is not present,
it will search in the default location.

\SubmitCmd{x509userproxy} is relevant when
the \SubmitCmd{universe} is \SubmitCmd{grid},
and the type of grid system is one of \SubmitCmd{gt2},
\SubmitCmd{gt4}, or \SubmitCmd{nordugrid}.

\end{description} 


%%%%%%%%%%%%%%%%%%%%%%%%%%%%%%%
\emph{COMMANDS FOR PARALLEL, JAVA, and SCHEDULER UNIVERSES}
%%%%%%%%%%%%%%%%%%%%%%%%%%%%%%%
\begin{description} 

%%%%%%%%%%%%%%%%%%%
%% hold_kill_sig
%%%%%%%%%%%%%%%%%%%

\index{submit commands!hold\_kill\_sig}
\item[hold\_kill\_sig = $<$signal-number$>$] For the scheduler universe only,
\SubmitCmd{signal-number} is the signal delivered
to the job when the job is put on hold
with \Condor{hold}.
\SubmitCmd{signal-number} may be either the platform-specific name or value
of the signal.
If this command is not present,
the value of \SubmitCmd{kill\_sig} is used.


%%%%%%%%%%%%%%%%%%%
%% jar_files
%%%%%%%%%%%%%%%%%%%

\index{submit commands!jar\_files}
\item[jar\_files = $<$file\_list$>$]
Specifies a list of additional JAR files to include when using
the Java universe.  JAR files will be transferred along with
the executable and automatically added to the classpath.

%%%%%%%%%%%%%%%%%%%
%% java_vm_args
%%%%%%%%%%%%%%%%%%%

\index{submit commands!java\_vm\_args}
\item[java\_vm\_args = $<$argument\_list$>$]
Specifies a list of additional arguments to the Java VM itself,
When Condor runs the Java program, these are the arguments that 
go before the class name.  This can be used to set VM-specific 
arguments like stack size, garbage-collector arguments 
and initial property values.

%%%%%%%%%%%%%%%%%%%
%% machine_count
%%%%%%%%%%%%%%%%%%%

\index{submit commands!machine\_count}
\item[machine\_count = $<$max$>$] 
For the parallel universe,
a single value (\Arg{max}) is required.
It is neither a maximum or minimum, but 
the number of machines to be dedicated toward running the job.

%%%%%%%%%%%%%%%%%%%
%% remove_kill_sig
%%%%%%%%%%%%%%%%%%%

\index{submit commands!remove\_kill\_sig}
\item[remove\_kill\_sig = $<$signal-number$>$] For the scheduler universe only,
\SubmitCmd{signal-number} is the signal delivered
to the job when the job is removed
with \Condor{rm}.
\SubmitCmd{signal-number} may be either the platform-specific name or value
of the signal.
This example shows it both ways for a Linux signal:
\begin{verbatim}
remove_kill_sig = SIGUSR1
remove_kill_sig = 10
\end{verbatim}
If this command is not present,
the value of \SubmitCmd{kill\_sig} is used.

\end{description} 

%%%%%%%%%%%%%%%%%%%%%%%%%%%%%%%
\emph{COMMANDS FOR THE VM UNIVERSE}
%%%%%%%%%%%%%%%%%%%%%%%%%%%%%%%
\begin{description} 

%%%%%%%%%%%%%%%%%%%
%% vm_cdrom_files
%%%%%%%%%%%%%%%%%%%

\index{submit commands!vm\_cdrom\_files}
\item[vm\_cdrom\_files = file1, file2, \Dots]
A comma-separated list of input CD-ROM files.

%%%%%%%%%%%%%%%%%%%
%% vm_checkpoint
%%%%%%%%%%%%%%%%%%%

\index{submit commands!vm\_checkpoint}
\item[vm\_checkpoint = $<$True \Bar\ False$>$]
A boolean value specifying whether or not to take checkpoints.
If not specified, the default value is \Expr{False}.
In the current implementation, setting both
\SubmitCmd{vm\_checkpoint} and \SubmitCmd{vm\_networking}
to \Expr{True} does not yet work in all cases.
Networking cannot be used if a vm universe job uses a
checkpoint in order to continue execution after migration
to another machine.

%%%%%%%%%%%%%%%%%%%
%% vm_macaddr
%%%%%%%%%%%%%%%%%%%
\index{submit commands!vm\_macaddr}
\item[vm\_macaddr = $<$MACAddr$>$]
Defines that MAC address that the virtual machine's network interface
should have,
in the standard format of six groups of
two hexadecimal digits separated by colons.

%%%%%%%%%%%%%%%%%%%
%% vm_memory
%%%%%%%%%%%%%%%%%%%

\index{submit commands!vm\_memory}
\item[vm\_memory = $<$MBytes-of-memory$>$]
The amount of memory in MBytes that a vm universe job
requires.

%%%%%%%%%%%%%%%%%%%
%% vm_networking
%%%%%%%%%%%%%%%%%%%

\index{submit commands!vm\_networking}
\item[vm\_networking = $<$True \Bar\ False$>$]
Specifies whether to use networking or not.
In the current implementation, setting both
\SubmitCmd{vm\_checkpoint} and \SubmitCmd{vm\_networking}
to \Expr{True} does not yet work in all cases.
Networking cannot be used if a vm universe job uses a
checkpoint in order to continue execution after migration
to another machine.


%%%%%%%%%%%%%%%%%%%
%% vm_networking_type
%%%%%%%%%%%%%%%%%%%

\index{submit commands!vm\_networking\_type}
\item[vm\_networking\_type = $<$nat \Bar\ bridge $>$]
When \SubmitCmd{vm\_networking} is \Expr{True},
this definition augments the job's requirements to match
only machines with the specified networking.
If not specified, then either networking type matches.

%%%%%%%%%%%%%%%%%%%
%% vm_no_output_vm
%%%%%%%%%%%%%%%%%%%

\index{submit commands!vm\_no\_output\_vm}
\item[vm\_no\_output\_vm = $<$True \Bar\ False$>$]
When \Expr{True}, prevents Condor from transferring output
files back to the machine from which the vm universe job
was submitted.
If not specified, the default value is \Expr{False}.

%%%%%%%%%%%%%%%%%%%
%% vm_should_transfer_cdrom_files
%%%%%%%%%%%%%%%%%%%

\index{submit commands!vm\_should\_transfer\_cdrom\_files}
\item[vm\_should\_transfer\_cdrom\_files = $<$True \Bar\ False$>$]
Specifies whether Condor will transfer CD-ROM files to
the execute machine (\Expr{True}) or rely on access through 
a shared file system (\Expr{False}).

%%%%%%%%%%%%%%%%%%%
%% vm_type
%%%%%%%%%%%%%%%%%%%

\index{submit commands!vm\_type}
\item[vm\_type = $<$vmware \Bar\ xen \Bar\ kvm$>$]
Specifies the underlying virtual machine software that this
job expects.

%%%%%%%%%%%%%%%%%%%
%% vmware_dir
%%%%%%%%%%%%%%%%%%%

\index{submit commands!vmware\_dir}
\item[vmware\_dir = $<$pathname$>$]
The complete path and name of the directory where VMware-specific
files and applications such as the VMDK (Virtual Machine Disk Format) and
VMX (Virtual Machine Configuration) reside.

%%%%%%%%%%%%%%%%%%%
%% vmware_should_transfer_files
%%%%%%%%%%%%%%%%%%%

\index{submit commands!vmware\_should\_transfer\_files}
\item[vmware\_should\_transfer\_files = $<$True \Bar\ False$>$]
Specifies whether Condor will transfer VMware-specific files 
located as specified by \SubmitCmd{vmware\_dir} to
the execute machine (\Expr{True}) or rely on access through 
a shared file system (\Expr{False}).
Omission of this required command (for VMware vm universe jobs)
results in an error message from \Condor{submit}, and the
job will not be submitted.

%%%%%%%%%%%%%%%%%%%
%% vmware_snapshot_disk
%%%%%%%%%%%%%%%%%%%

\index{submit commands!vmware\_snapshot\_disk}
\item[vmware\_snapshot\_disk = $<$True \Bar\ False$>$]
When \Expr{True}, causes Condor to utilize
a VMware snapshot disk for new or modified files.
If not specified, the default value is \Expr{True}.

%%%%%%%%%%%%%%%%%%%
%% xen_cdrom_device
%%%%%%%%%%%%%%%%%%%

\index{submit commands!xen\_cdrom\_device}
\item[xen\_cdrom\_device = $<$device$>$]
Describes the Xen CD-ROM device when
\SubmitCmd{vm\_cdrom\_files} is defined.

%%%%%%%%%%%%%%%%%%%
%% xen_disk
%%%%%%%%%%%%%%%%%%%

\index{submit commands!xen\_disk}
\item[xen\_disk = file1:device1:permission1, file2:device2:permission2, \Dots]
A list of comma separated disk files.
Each disk file is specified by 3 colon separated fields.
The first field is the path and file name of the disk file.
The second field specifies the device, and
the third field specifies permissions.

An example that specifies two disk files:
\footnotesize
\begin{verbatim}
xen_disk = /myxen/diskfile.img:sda1:w,/myxen/swap.img:sda2:w
\end{verbatim}
\normalsize

%%%%%%%%%%%%%%%%%%%
%% xen_initrd
%%%%%%%%%%%%%%%%%%%

\index{submit commands!xen\_initrd}
\item[xen\_initrd = $<$image-file$>$]
When \SubmitCmd{xen\_kernel} gives a path and file name for
the kernel image to use, 
this optional command may specify a path to and ramdisk 
(\File{initrd}) image file.

%%%%%%%%%%%%%%%%%%%
%% xen_kernel
%%%%%%%%%%%%%%%%%%%

\index{submit commands!xen\_kernel}
\item[xen\_kernel = $<$included \Bar\  path-to-kernel$>$]
A value of \SubmitCmd{included} specifies that the kernel is
included in the disk file.
If not one of these values, then the value is a path and
file name of the kernel to be used.

%%%%%%%%%%%%%%%%%%%
%% xen_kernel_params
%%%%%%%%%%%%%%%%%%%

\index{submit commands!xen\_kernel\_params}
\item[xen\_kernel\_params = $<$string$>$]
A string that is appended to the Xen kernel command line.

%%%%%%%%%%%%%%%%%%%
%% xen_root
%%%%%%%%%%%%%%%%%%%

\index{submit commands!xen\_root}
\item[xen\_root = $<$string$>$]
A string that is appended to the Xen kernel command line
to specify the root device. This string is required when
\SubmitCmd{xen\_kernel} gives a path to a kernel.  Omission 
for this required case results in an error message during 
submission.

%%%%%%%%%%%%%%%%%%%
%% xen_transfer_files
%%%%%%%%%%%%%%%%%%%

\index{submit commands!xen\_transfer\_files}
\item[xen\_transfer\_files = $<$list-of-files$>$]
A comma separated list of all files that Condor is to transfer
to the execute machine.

%%%%%%%%%%%%%%%%%%%
%% kvm_disk
%%%%%%%%%%%%%%%%%%%

\index{submit commands!kvm\_disk}
\item[kvm\_disk = file1:device1:permission1, file2:device2:permission2, \Dots]
A list of comma separated disk files.
Each disk file is specified by 3 colon separated fields.
The first field is the path and file name of the disk file.
The second field specifies the device, and
the third field specifies permissions.

An example that specifies two disk files:
\footnotesize
\begin{verbatim}
kvm_disk = /myxen/diskfile.img:sda1:w,/myxen/swap.img:sda2:w
\end{verbatim}
\normalsize

%%%%%%%%%%%%%%%%%%%
%% kvm_cdrom_device
%%%%%%%%%%%%%%%%%%%

\index{submit commands!kvm\_cdrom\_device}
\item[kvm\_cdrom\_device = $<$device$>$]
Describes the KVM CD-ROM device when
\SubmitCmd{vm\_cdrom\_files} is defined.

%%%%%%%%%%%%%%%%%%%
%% kvm_transfer_files
%%%%%%%%%%%%%%%%%%%

\index{submit commands!kvm\_transfer\_files}
\item[kvm\_transfer\_files = $<$list-of-files$>$]
A comma separated list of all files that Condor is to transfer
to the execute machine.

\end{description} 

%%%%%%%%%%%%%%%%%%%%%%%%%%%%%%%
\emph{ADVANCED COMMANDS}
%%%%%%%%%%%%%%%%%%%%%%%%%%%%%%%
\begin{description} 

%%%%%%%%%%%%%%%%%%%
%% concurrency_limits
%%%%%%%%%%%%%%%%%%%
\index{submit commands!concurrency\_limits}
\item[concurrency\_limits = $<$string-list$>$]
A list of resources that this job needs.
The resources are presumed to have concurrency limits placed upon them,
thereby limiting the number of concurrent jobs in execution which
need the named resource.
Commas and space characters delimit the items in the list.
Each item in the list may specify a numerical value identifying the integer
number of resources required for the job.
The syntax follows the resource name by a colon character (\verb@:@)
and the numerical value.
See section~\ref{sec:Concurrency-Limits} for details on concurrency limits.

%%%%%%%%%%%%%%%%%%%
%% copy_to_spool 
%%%%%%%%%%%%%%%%%%%

\index{submit commands!copy\_to\_spool}
\item[copy\_to\_spool = $<$True \Bar\ False$>$]
If \SubmitCmd{copy\_to\_spool} is \Expr{True},
then \Condor{submit} copies the executable to the local spool 
directory before running it on a remote host. 
As copying can be quite time consuming and unnecessary,
the default value is \Expr{False} for all job universes
other than the standard universe. 
When \Expr{False}, \Condor{submit} does not copy the executable
to a local spool directory.
The default is \Expr{True} in standard universe, because
resuming execution from a checkpoint can only be guaranteed to work using
precisely the same executable that created the checkpoint.

%%%%%%%%%%%%%%%%%%%
%% coresize
%%%%%%%%%%%%%%%%%%%

\index{submit commands!coresize}
\item[coresize = $<$size$>$] Should the user's program abort and produce
a core file, \SubmitCmd{coresize} specifies the maximum size in bytes of the
core file which the user wishes to keep. If \SubmitCmd{coresize} is not
specified in the command file, the system's user resource limit
\mbox{``coredumpsize''} is used.
This limit is not used in HP-UX and DUX operating systems. 

%%%%%%%%%%%%%%%%%%
% cron_day_of_month
%%%%%%%%%%%%%%%%%%
\index{submit commands!cron\_day\_of\_month}
\item[cron\_day\_of\_month = $<$Cron-evaluated Day$>$]
The set of days of the month for which a deferral time applies.
See section~\ref{sec:CronTab} for further details and examples.

%%%%%%%%%%%%%%%%%%
% cron_day_of_week
%%%%%%%%%%%%%%%%%%
\index{submit commands!cron\_day\_of\_week}
\item[cron\_day\_of\_week = $<$Cron-evaluated Day$>$]
The set of days of the week for which a deferral time applies.
See section~\ref{sec:CronTab} for details, semantics, and examples.

%%%%%%%%%%%%%%%%%%
% cron_hour
%%%%%%%%%%%%%%%%%%
\index{submit commands!cron\_hour}
\item[cron\_hour = $<$Cron-evaluated Hour$>$]
The set of hours of the day for which a deferral time applies.
See section~\ref{sec:CronTab} for details, semantics, and examples.

%%%%%%%%%%%%%%%%%%
% cron_minute
%%%%%%%%%%%%%%%%%%
\index{submit commands!cron\_minute}
\item[cron\_minute = $<$Cron-evaluated Minute$>$]
The set of minutes within an hour for which a deferral time applies.
See section~\ref{sec:CronTab} for details, semantics, and examples.

%%%%%%%%%%%%%%%%%%
% cron_month
%%%%%%%%%%%%%%%%%%
\index{submit commands!cron\_month}
\item[cron\_month = $<$Cron-evaluated Month$>$]
The set of months within a year for which a deferral time applies.
See section~\ref{sec:CronTab} for details, semantics, and examples.

%%%%%%%%%%%%%%%%%%
% cron_prep_time
%%%%%%%%%%%%%%%%%%
\index{submit commands!cron\_prep\_time}
\item[cron\_prep\_time = $<$ClassAd Integer Expression$>$]
Analogous to \SubmitCmd{deferral\_prep\_time}.
The number of seconds prior to a job's deferral time that
the job may be matched and sent to an execution machine.

%%%%%%%%%%%%%%%%%%%
%% cron_window
%%%%%%%%%%%%%%%%%%%

\index{submit commands!cron\_window}
\item[cron\_window = $<$ClassAd Integer Expression$>$]
Analogous to the submit command \SubmitCmd{deferral\_window}.
It allows cron jobs that
miss their deferral time to begin execution.

See section~\ref{sec:JobDeferral} for further details and examples.

%%%%%%%%%%%%%%%%%%%
%% deferral_prep_time
%%%%%%%%%%%%%%%%%%%

\index{submit commands!deferral\_prep\_time}
\item[deferral\_prep\_time = $<$ClassAd Integer Expression$>$]
The number of seconds prior to a job's deferral time that
the job may be matched and sent to an execution machine.

See section~\ref{sec:JobDeferral} for further details.

%%%%%%%%%%%%%%%%%%%
%% deferral_time
%%%%%%%%%%%%%%%%%%%

\index{submit commands!deferral\_time}
\item[deferral\_time = $<$ClassAd Integer Expression$>$]
Allows a job to specify the time at which its execution
is to begin,
instead of beginning execution as soon as it arrives at the execution
machine. The deferral time is an expression that 
evaluates to a Unix Epoch timestamp (the number of
seconds elapsed since 00:00:00 on January 1, 1970, Coordinated
Universal Time). 
Deferral time is evaluated with respect to the execution machine.
This option delays the start of
execution, but not the matching and claiming of
a machine for the job.
If the job is not available and ready to begin
execution at the deferral time, it has missed its deferral time.
A job that misses its deferral time will be put on hold
in the queue. 

See section~\ref{sec:JobDeferral} for further details and examples.

Due to implementation details,
a deferral time may not be used for scheduler universe jobs.

%%%%%%%%%%%%%%%%%%%
%% deferral_window
%%%%%%%%%%%%%%%%%%%

\index{submit commands!deferral\_window}
\item[deferral\_window = $<$ClassAd Integer Expression$>$]
The deferral window is used in conjunction with the
\SubmitCmd{deferral\_time} command to allow jobs that
miss their deferral time to begin execution.

See section~\ref{sec:JobDeferral} for further details and examples.

%%%%%%%%%%%%%%%%%%%
%% email_attributes
%%%%%%%%%%%%%%%%%%%

\index{submit commands!email\_attributes}
\item[email\_attributes = $<$list-of-job-ad-attributes$>$] 
A comma-separated list of attributes from the job ClassAd. These
attributes and their values will be included in the e-mail notification
of job completion.

%%%%%%%%%%%%%%%%%%%
%% image_size
%%%%%%%%%%%%%%%%%%%

\index{submit commands!image\_size}
\item[image\_size = $<$size$>$] This command tells Condor the maximum
virtual image size to which you believe your program will grow during
its execution. Condor will then execute your job only on machines which
have enough resources, (such as virtual memory), to support executing
your job. If you do not specify the image size of your job in the
description file, Condor will automatically make a (reasonably accurate)
estimate about its size and adjust this estimate as your program runs.
If the image size of your job is underestimated, it may crash due to
inability to acquire more address space, e.g. malloc() fails. If the image
size is overestimated, Condor may have difficulty finding machines which
have the required resources. \Arg{size} must be in Kbytes, e.g. for
an image size of 8 megabytes, use a \Arg{size} of 8000.

%%%%%%%%%%%%%%%%%%%
%% initialdir
%%%%%%%%%%%%%%%%%%%

\index{submit commands!initialdir}
\item[initialdir = $<$directory-path$>$] 
Used to give jobs a directory with respect to file input and output.
Also provides a directory 
(on the machine from which the job is submitted)
for the user log, when a full path is not specified. 

For vanilla universe jobs where there is a shared file system,
it is the current working directory on the machine where the
job is executed.

For vanilla or grid universe jobs where file transfer mechanisms are
utilized (there is \emph{not} a shared file system),
it is the directory on the machine from which the job is submitted
where the input files come from, and where the job's output
files go to.

For standard universe jobs,
it is the directory on the machine from which the job is submitted
where the \Condor{shadow} daemon runs;
the current working directory for file input and output accomplished
through remote system calls.

For scheduler universe jobs,
it is the directory on the machine from which the job is submitted
where the job runs;
the current working directory for file input and output with
respect to relative path names.

Note that the path to the executable is \emph{not} relative to
\SubmitCmd{initialdir}; if it is a relative path, it is relative to the
directory in which the \Condor{submit} command is run.


%%%%%%%%%%%%%%%%%%%
%% job_lease_duration
%%%%%%%%%%%%%%%%%%%
\index{submit commands!job\_lease\_duration}
\item[job\_lease\_duration = $<$number-of-seconds$>$] For vanilla
and java universe jobs only, the duration (in seconds) of a
job lease.
The default value is twenty minutes for universes that support it.
If a job lease is not desired, the value can be explicitly set to 0 to
disable the job lease semantics.
See section~\ref{sec:Job-Lease} for details of job leases.


%%%%%%%%%%%%%%%%%%%
%% kill_sig
%%%%%%%%%%%%%%%%%%%

\index{submit commands!kill\_sig}
\item[kill\_sig = $<$signal-number$>$] When Condor needs to kick a job
off of a machine, it will send the job the signal specified by
\SubmitCmd{signal-number}.  \SubmitCmd{signal-number} needs to be an integer which
represents a valid signal on the execution machine.  For jobs submitted
to the standard universe, the default value is the number for
\verb@SIGTSTP@ which tells the Condor libraries to initiate a checkpoint
of the process.  For jobs submitted to the vanilla universe,
the default 
is \verb@SIGTERM@ which is the standard way to terminate a program in Unix.  

%%%%%%%%%%%%%%%%%%%
%% load_profile
%%%%%%%%%%%%%%%%%%%

\index{submit commands!load\_profile}
\item[load\_profile = $<$True \Bar\ False$>$]
When \Expr{True}, loads the account profile of the dedicated run account for
Windows jobs.
May not be used with \SubmitCmd{run\_as\_owner}.

%%%%%%%%%%%%%%%%%%%
%% match_list_length
%%%%%%%%%%%%%%%%%%%

\index{submit commands!match\_list\_length}
\item[match\_list\_length = $<$integer value$>$]
Defaults to the value zero (0).
When \SubmitCmd{match\_list\_length} is defined with an integer value
greater than zero (0),
attributes are inserted into the job ClassAd.
The maximum number of attributes defined is given by the integer
value.
The job ClassAds introduced are given as
\begin{verbatim}
LastMatchName0 = "most-recent-Name"
LastMatchName1 = "next-most-recent-Name"
\end{verbatim}

The value for each introduced ClassAd is given by the
value of the \Attr{Name} attribute
from the machine ClassAd of a previous execution (match).
As a job is matched, the definitions for these attributes
will roll,
with \verb@LastMatchName1@ becoming \verb@LastMatchName2@,
\verb@LastMatchName0@ becoming \verb@LastMatchName1@,
and \verb@LastMatchName0@ being set by the most recent
value of the \Attr{Name} attribute.

An intended use of
these job attributes is in the requirements expression.
The requirements can allow a job to prefer a match with either the same
or a different resource than a previous match.


%%%%%%%%%%%%%%%%%%%
%% max_job_retirement_time
%%%%%%%%%%%%%%%%%%%

\index{max\_job\_retirement\_time}
\index{submit commands!max\_job\_retirement\_time}
\item[max\_job\_retirement\_time = $<$integer expression$>$]
An integer-valued expression (in seconds) that
does nothing unless the machine that runs the job has been configured
to provide retirement time
(see section~\ref{sec:State-Expression-Summary}).
Retirement time is a
grace period given to a job to finish naturally
when a resource claim is about to be preempted.
No kill signals are sent during a retirement time.
The default behavior in many cases is to take as much
retirement time as the machine offers,
so this command will rarely appear in a submit description file.

When a resource claim is to be preempted, this expression in the
submit file specifies the maximum run time of the job (in seconds, since
the job started).
This expression has no effect,
if it is greater than the maximum retirement time provided
by the machine policy.
If the resource claim is \emph{not} preempted,
this expression and the machine retirement policy are irrelevant. 
If the resource claim \emph{is} preempted
and the job finishes sooner than the maximum time,
the claim closes gracefully and all is well.
If the resource claim is preempted
and the job does \emph{not} finish in time,
the usual preemption
procedure is followed (typically a soft kill signal, followed by some
time to gracefully shut down, followed by a hard kill signal).

Standard universe jobs and any jobs running with \SubmitCmd{nice\_user}
priority have a default \SubmitCmd{max\_job\_retirement\_time} of 0,
so no retirement time is utilized by default.
In all other cases,
no default value is provided,
so the maximum amount of retirement time is utilized by default.

Setting this expression does not affect the job's resource
requirements or preferences.  
For a job to only run on
a machine with a minimum \AdAttr{},
or to preferentially run on such machines, explicitly
specify this in the requirements and/or rank expressions.

%%%%%%%%%%%%%%%%%%%
%% nice_user
%%%%%%%%%%%%%%%%%%%

\index{submit commands!nice\_user}
\item[nice\_user = $<$True \Bar\ False$>$] \label{man-condor-submit-nice}Normally, when a machine
becomes available to Condor, Condor decides which job to run based upon
user and job priorities. Setting \SubmitCmd{nice\_user} equal to \Expr{True}
tells Condor not to use your regular user priority, but that this job
should have last priority among all users and all jobs. So jobs
submitted in this fashion run only on machines which no other
non-nice\_user job wants --- a true ``bottom-feeder'' job! This is very
handy if a user has some jobs they wish to run, but do not wish to use
resources that could instead be used to run other people's Condor jobs. Jobs
submitted in this fashion have ``nice-user.'' pre-appended in front of
the owner name when viewed from \Condor{q} or \Condor{userprio}.  The
default value is \SubmitCmd{False}.

%%%%%%%%%%%%%%%%%%%
%% noop_job
%%%%%%%%%%%%%%%%%%%

\index{submit commands!noop\_job}
\item[noop\_job = $<$ClassAd Boolean Expression$>$]
When this boolean expression is \Expr{True},
the job is immediately removed from the queue,
and Condor makes no attempt at running the job.
The log file for the job will show a
job submitted event and a job terminated event,
along with an exit code of 0,
unless the user specifies a different signal or exit code.

%%%%%%%%%%%%%%%%%%%
%% noop_job_exit_code
%%%%%%%%%%%%%%%%%%%

\index{submit commands!noop\_job\_exit\_code}
\item[noop\_job\_exit\_code = $<$return value$>$]
When \SubmitCmd{noop\_job} is in the submit description file
and evaluates to \Expr{True},
this command allows the job
to specify the return value as shown in the job's log file
job terminated event.
If not specified, the job will show as having terminated with status 0.
This overrides any value specified with \SubmitCmd{noop\_job\_exit\_signal}.

%%%%%%%%%%%%%%%%%%%
%% noop_job_exit_signal
%%%%%%%%%%%%%%%%%%%

\index{submit commands!noop\_job\_exit\_signal}
\item[noop\_job\_exit\_signal = $<$signal number$>$]
When \SubmitCmd{noop\_job} is in the submit description file
and evaluates to \Expr{True},
this command allows the job
to specify the signal number that the job's log event will show
the job having terminated with.


%%%%%%%%%%%%%%%%%%%
%% remote_initialdir
%%%%%%%%%%%%%%%%%%%
\index{submit commands!remote\_initialdir}
\item[remote\_initialdir = $<$directory-path$>$]
The path specifies the directory in which the job is to be
executed on the remote machine.  This is currently supported in all
universes except for the standard universe.


%%%%%%%%%%%%%%%%%%%
%% rendezvousdir
%%%%%%%%%%%%%%%%%%%
\index{submit commands!rendezvousdir}
\item[rendezvousdir = $<$directory-path$>$] Used to specify the
shared file system directory to be used for file system authentication
when submitting to a remote scheduler.  Should be a path to a preexisting
directory.

%%%%%%%%%%%%%%%%%%%
%% request_cpus
%%%%%%%%%%%%%%%%%%%
\index{submit commands!request\_cpus}
\item[request\_cpus = $<$num-cpus$>$] For pools that enable
dynamic \Condor{startd} provisioning
(see section~\ref{sec:SMP-dynamicprovisioning}),
the number of CPUs requested for this job.

%%%%%%%%%%%%%%%%%%%
%% request_disk
%%%%%%%%%%%%%%%%%%%
\index{submit commands!request\_disk}
\item[request\_disk = $<$quantity$>$] For pools that enable
dynamic \Condor{startd} provisioning
(see section~\ref{sec:SMP-dynamicprovisioning}),
the amount of disk space requested for this job.

%%%%%%%%%%%%%%%%%%%
%% request_memory
%%%%%%%%%%%%%%%%%%%
\index{submit commands!request\_memory}
\item[request\_memory = $<$quantity$>$] For pools that enable
dynamic \Condor{startd} provisioning
(see section~\ref{sec:SMP-dynamicprovisioning}),
the amount of memory space requested for this job.

%%%%%%%%%%%%%%%%%%%
%% run_as_owner
%%%%%%%%%%%%%%%%%%%

\index{submit commands!run\_as\_owner}
\item[run\_as\_owner = $<$True \Bar\ False$>$]
A boolean value that causes the job to be run under the login of 
the submitter,
if supported by the joint configuration of the submit and execute machines.
On Unix platforms, this defaults to \Expr{True},
and on Windows platforms, it defaults to \Expr{False}.
May not be used with \SubmitCmd{load\_profile}.
See section~\ref{sec:windows-run-as-owner} for administrative details on
configuring Windows to support this option, as well as 
section~\ref{param:StarterAllowRunAsOwner}
on page~\pageref{param:StarterAllowRunAsOwner} for a definition of
\MacroNI{STARTER\_ALLOW\_RUNAS\_OWNER}.


%%%%%%%%%%%%%%%%%%%
% +
% This should always be at the end of the list
%%%%%%%%%%%%%%%%%%%

\item[+$<$attribute$>$ = $<$value$>$] A line which begins with a '+'
(plus) character instructs \Condor{submit} to insert the
following \Arg{attribute} into the job ClassAd with the given 
\Arg{value}. 

\end{description} 


\index{macro!in submit description file}
In addition to commands, the submit description file can contain macros
and comments:

\begin{description}

\item[Macros] Parameterless macros in the form of \MacroUNI{macro\_name}
may be inserted anywhere in Condor submit description files. Macros can be
defined by lines in the form of 
\begin{verbatim} 
        <macro_name> = <string> 
\end{verbatim} 
Three pre-defined macros are supplied by the submit description file parser.
The third of the pre-defined macros is only relevant to MPI applications
under the parallel universe.
The
\MacroU{Cluster} macro supplies the value of the
\index{ClassAd job attribute!ClusterId}
\index{ClusterId!job ClassAd attribute}
\index{job ID!cluster identifier}
\Attr{ClusterId} job
ClassAd attribute, and the
\MacroU{Process} macro supplies the value of the
\Attr{ProcId} job
ClassAd attribute.
These macros are
intended to aid in the specification of input/output files, arguments,
etc., for clusters with lots of jobs, and/or could be used to supply a
Condor process with its own cluster and process numbers on the command
line.
The 
\MacroU{Node} macro is defined for MPI applications run
as parallel universe jobs.
It is a unique value assigned for the duration of the job
that essentially identifies the machine on which a program is
executing.

\index{\$!as a literal character in a submit description file}
To use the dollar sign character (\verb@$@) as a literal,
without macro expansion, use
\begin{verbatim}
$(DOLLAR)
\end{verbatim}

In addition to the normal macro, there is also a special kind of macro
called a \Term{substitution macro}
\index{substitution macro!in submit description file}
that allows the substitution of
a ClassAd attribute value defined on the resource machine itself
(gotten after a match to the machine has been made)
into specific commands within the
submit description file. The substitution macro is of the form:
\begin{verbatim} 
$$(attribute)
\end{verbatim}

A common use of this macro is for the heterogeneous
submission of an executable:
\begin{verbatim}
executable = povray.$$(opsys).$$(arch)
\end{verbatim}
Values for the \AdAttr{opsys} and \AdAttr{arch} attributes are substituted at
match time for any given resource. This allows Condor to automatically
choose the correct executable for the matched machine.

An extension to the syntax of the substitution macro provides an
alternative string to use if the machine attribute within the
substitution macro is undefined.
The syntax appears as:
\begin{verbatim} 
$$(attribute:string_if_attribute_undefined)
\end{verbatim}

An example using this extended syntax provides a path name to a
required input file.
Since the file can be placed in different locations on
different machines, the file's path name is given as an argument
to the program.
\begin{verbatim} 
argument = $$(input_file_path:/usr/foo)
\end{verbatim}
On the machine, if the attribute \Attr{input\_file\_path} is not
defined, then the path \File{/usr/foo} is used instead.

A further extension to the syntax of the substitution macro allows the
evaluation of a ClassAd expression to define the value.
As all substitution macros, the expression is evaluated after
a match has been made.
Therefore, the expression may refer to machine attributes
by prefacing them with the scope resolution prefix \Attr{TARGET.},
as specified in section~\ref{ClassAd:evaluation}.
To place a ClassAd expression into the substitution macro,
square brackets are added to delimit the expression.
The syntax appears as:
\begin{verbatim} 
$$([ClassAd expression])
\end{verbatim}
An example of a job that uses this syntax may be one that
wants to know how much memory it can use. 
The application cannot detect this itself, as it would potentially use
all of the memory on a multi-slot machine.
So the job determines the memory per slot, 
reducing it by 10\Percent\
to account for miscellaneous overhead,
and passes this as a command line argument to the application.
In the submit description file will be
\begin{verbatim} 
arguments=--memory $$([TARGET.Memory * 0.9])
\end{verbatim}

\index{\$\$!as literal characters in a submit description file}
To insert two dollar sign characters (\verb@$$@) as literals
into a ClassAd string, use
\begin{verbatim}
$$(DOLLARDOLLAR)
\end{verbatim}

\index{\$ENV!in submit description file}
\index{submit commands!\$ENV macro}
\index{environment variables!in submit description file}
The environment macro, \$ENV, allows the evaluation of an environment
variable to be used in setting a submit description file command.
The syntax used is
\begin{verbatim} 
$ENV(variable)
\end{verbatim}
An example submit description file command that uses this functionality
evaluates the submittor's home directory in order to set the
path and file name of a log file:
\begin{verbatim} 
log = $ENV(HOME)/jobs/logfile
\end{verbatim}
The environment variable is evaluated when the submit description
file is processed.

\index{\$RANDOM\_CHOICE()!in submit description file}
\index{submit commands!\$RANDOM\_CHOICE() macro}
\index{RANDOM\_CHOICE() macro!use in submit description file}
The \$RANDOM\_CHOICE macro allows a random choice to be made
from a given list of parameters at submission time.
For an expression, if some randomness needs to be generated,
the macro may appear as
\begin{verbatim} 
    $RANDOM_CHOICE(0,1,2,3,4,5,6)
\end{verbatim}
When evaluated, one of the parameters values will be chosen. 

\item[Comments] Blank lines and lines beginning with a 
pound sign
('\#')
character are ignored by the submit description file parser. 

\end{description}

