
\section{Job Monitor}
\index{Job monitor}
\index{viewing!log files}

The Condor Job Monitor is a Java application designed to allow users to view user log files. 

To view a user log file, select it using the open file command in the File menu.  After the file is parsed, it will be visually represented.  Each horizontal line represents an individual job.  The x-axis
is time.  Whether a job is running at a particular time is represented by its color at that time -- white for running, black for idle.  For example, a job which appears predominantly white has made
efficient progress, whereas a job which appears predominantly black has received an inordinately small proportion of computational time. 


\subsection{\label{sec:transition-states}Transition States}

A transition state is the state of a job at any time.  It is called a "transition" because it is defined by the two events which bookmark it.  There are two basic transition states: running and idle. 
An idle job typically is a job which has just been submitted into the Condor pool and is waiting to be matched with an appropriate machine or a job which has vacated from a machine and has been
returned to the pool.  A running job, by contrast, is a job which is making active progress. 

Advanced users may want a visual distinction between two types of running transitions: "goodput" or "badput".  Goodput is the transition state preceding an eventual job completion or
checkpoint.  Badput is the transition state preceding a non-checkpointed eviction event.  Note that "badput" is potentially a misleading nomenclature; a job which is not checkpointed by the
Condor program may checkpoint itself or make progress in some other way.  To view these two transition as distinct transitions, select the appropriate option from the "View" menu. 


\subsection{\label{sec:events}Events}

There are two basic kinds of events: checkpoint events and error events.   Plus advanced users can ask to see more events. 


\subsection{\label{sec:job-selector}Selecting Jobs}

To view any arbitrary selection of jobs in a job file, use the job selector tool.  Jobs appear visually by order of appearance within the actual text log file.  For example, the log file might contain jobs
775.1, 775.2, 775.3, 775.4, and 775.5, which appear in that order.  A user who wishes to see only jobs 775.2 and 775.5 can select only these two jobs in the job selector tool and click the "Ok" or
"Apply" button.  The job selector supports double clicking; double
click on any single job to see it drawn in isolation. 

\subsection{\label{sec:zooming}Zooming}

To view a small area of the log file, zoom in on the area which you would like to see in greater detail. You can zoom in, out and do a full zoom. A full zoom redraws the log file in its entirety. For
example, if you have zoomed in very close and would like to go all the way back out, you could do so with a succession of zoom outs or with one full zoom. 

There is a difference between using the menu driven zooming and the mouse driven zooming. The menu driven zooming will recenter itself around the current center, whereas mouse driven
zooming will recenter itself (as much as possible) around the mouse click. To help you re-find the clicked area, a box will flash after the zoom. This is called the "zoom finder" and it can be turned
off in the zoom menu if you prefer. 

\subsection{\label{sec:k-m-shortcuts}Keyboard and Mouse Shortcuts}

\begin{enumerate}
\item The Keyboard shortcuts: 

\begin{itemize}
\item Arrows - an approximate ten percent scrollbar movement
\item PageUp and PageDown - an approximate one hundred percent scrollbar movement 
\item Control + Left or Right - approximate one hundred percent scrollbar movement 
\item End and Home - scrollbar movement to the vertical extreme 
\item Others - as seen beside menu items
\end{itemize}

\item The mouse shortcuts: 

\begin{itemize}
\item Control + Left click - zoom in 
\item Control + Right click - zoom out
\item Shift + left click - re-center
\end{itemize}
\end{enumerate}
 

