%%%%%%%%%%%%%%%%%%%%%%%%%%%%%%%%%%%%%%%%%%%%%%%%%%%%%%%%%%%%%%%%%%%%%%
\section{\label{sec:PVM}PVM Applications}
%%%%%%%%%%%%%%%%%%%%%%%%%%%%%%%%%%%%%%%%%%%%%%%%%%%%%%%%%%%%%%%%%%%%%%

\newcommand{\func}[1]{\texttt{#1}}

\index{PVM (Parallel Virtual Machine)|(}
\index{Parallel Virtual Machine (PVM)}
\index{Condor-PVM}
Applications that use PVM (Parallel Virtual Machine) may use
Condor.
PVM offers a set of message passing primitives for use in
C and C++ language programs.
The primitives, together with the PVM environment
allow parallelism at the program level.
Multiple processes
may run on multiple machines,
while communicating with each other.
More information about PVM is available at 
\URL{http://www.epm.ornl.gov/pvm/}.

Condor-PVM provides a framework to run PVM applications
in Condor's opportunistic environment.
Where PVM needs dedicated machines
to run PVM applications, Condor does not.
Condor can be used to dynamically 
construct PVM virtual machines from a Condor pool of machines.

\index{Condor!PVM applications}
In Condor-PVM, Condor acts as the
resource manager for the PVM daemon.  Whenever a PVM program asks
for nodes (machines), the request is forwarded to Condor.
Condor finds a machine in the Condor pool using usual mechanisms,
and adds it to the virtual machine.
If a machine needs to leave the pool, the
PVM program is notified by normal PVM mechanisms.

% This manual section
% discusses the differences between running under normal PVM
% and running PVM within the Condor
% environment.  The most effective way to use Condor-PVM is
% presented, along with an example program.
% A sample Condor submit description file for a PVM job illustrates
% how to submit a Condor-PVM job.

\Note Condor-PVM is an optional Condor module.  It is not
automatically installed with Condor. To check and see if
it has been installed at your site, enter the command:
\begin{verbatim}
        ls -l `condor_config_val PVMD`
\end{verbatim}
Please note the use of back ticks in the above command.
They specify to run the \Condor{config\_val} program.
If the result of this program shows the
file \File{condor\_pvmd} on your system, then the Condor-PVM module
is installed.
If not,
ask your site administrator to download and install Condor-PVM from
\URL{http://www.cs.wisc.edu/condor/downloads/}.

%%%%%%%%%%%%%%%%%%%%%%%%%%%%%%%%%%%%%%%%%%%%%%%%%%%%%%%%%%%%%%%%%%%%%%
\subsection{Effective Usage: the Master-Worker Paradigm}
%%%%%%%%%%%%%%%%%%%%%%%%%%%%%%%%%%%%%%%%%%%%%%%%%%%%%%%%%%%%%%%%%%%%%%

\index{master-worker paradigm}
\index{PVM!master-worker paradigm}
There are several different parallel programming paradigms.  One of the
more common is the \Term{master-worker} (or \Term{pool of tasks})
arrangement.  In a master-worker program model, one node acts as the
controlling master for the parallel application and sends pieces
of work out to worker nodes.
The worker node does some computation, and it sends the result back to the
master node.  The master has a pool of work that needs to be
done, so it assigns the next piece of work out to the next worker
that becomes available.  

Condor-PVM is designed to run PVM applications which follow the
master-worker paradigm.  Condor runs the master application on the
machine where the job was submitted and will not preempt it.  Workers
are pulled in from the Condor pool as they become available.

Not all parallel programming paradigms lend themselves to Condor's
opportunistic environment. In such an environment, any of the nodes
could be preempted and disappear at any moment.
The master-worker model does work well in this environment.
The master keeps track of which piece of work it
sends to each worker. The master node is informed of
the addition and disappearance of nodes.
If the master node is informed that a worker node
has disappeared, the master places the unfinished work it had assigned
to the disappearing node
back into the pool of tasks.
This work is sent again to the next
available worker node.
If the master notices that the number of workers has
dropped below an acceptable level, it requests more workers
(using \func{pvm\_addhosts()}).
Alternatively, the master requests
a replacement node every time it is notified that a worker has
gone away. The benefit of this paradigm is that the number of workers is
not important and changes in the size of
the virtual machine are easily handled.

A tool called \Term{MW} has been developed to assist in the
development of master-worker style applications for
distributed, opportunistic environments like Condor.
MW provides a C++ API which hides the complexities of managing a
master-worker Condor-PVM application.
We suggest that you consider modifying your PVM application to use MW
instead of developing your own dynamic PVM master from scratch.
Additional information about MW is available at 
\URL{http://www.cs.wisc.edu/condor/mw/}.

%%%%%%%%%%%%%%%%%%%%%%%%%%%%%%%%%%%%%%%%%%%%%%%%%%%%%%%%%%%%%%%%%%%%%%
\subsection{Binary Compatibility and Runtime Differences}
%%%%%%%%%%%%%%%%%%%%%%%%%%%%%%%%%%%%%%%%%%%%%%%%%%%%%%%%%%%%%%%%%%%%%%

Condor-PVM does not define a new API (application program interface);
programs use the existing resource management PVM calls such
as \func{pvm\_addhosts()} and \func{pvm\_notify()}.  Because of this, some
master-worker PVM applications are ready to run under Condor-PVM with no
changes at all.  Regardless of using Condor-PVM or not, it is good
master-worker design to handle the case of a disappearing worker node,
and therefore many programmers have already constructed their master program
with all the necessary fault tolerant logic.

Regular PVM and Condor-PVM are \emph{binary compatible}.
The same binary which runs under regular PVM will run
under Condor, and vice-versa.  There is no need to re-link for Condor-PVM.
This permits easy application development
(develop your PVM application interactively with the regular PVM console, XPVM,
etc.) as well as binary sharing between Condor and some dedicated MPP systems.

This release of Condor-PVM is based on PVM 3.4.2.  PVM versions 
3.4.0 through 3.4.2 are all supported.  The vast majority of the PVM
library functions under Condor maintain the same semantics as in
PVM 3.4.2, including messaging operations, group operations, and 
\func{pvm\_catchout()}.

The following list
is a summary of the changes and new features of PVM running within the
Condor environment:

\begin{itemize}

\item Condor introduces the concept of machine class.
  A pool of machines is likely to contain machines of more than
  one platform.
  Under Condor-PVM, machines of
  different architectures belong to different machine classes.
  With the concept machine class,
\index{PVM!machine class}
  Condor can be told what type
  of machine to allocate.
  Machine classes are assigned integer values, starting with 0.
  A machine class is
  specified in a submit description file when the job
  is submitted to Condor.

\item \func{pvm\_addhosts()}.  When an application
  adds a host machine, it calls \func{pvm\_addhosts()}.
  The first argument to 
  \func{pvm\_addhosts()}
  is a string that specifies the machine class.
  For example, to specify class 0, a pointer to the string ``0''
  is the first argument.  Condor finds a machine
  that satisfies the requirements of class 0 and adds it to the PVM
  virtual machine.

  The function \func{pvm\_addhosts()} does not block.  It
  returns immediately, before hosts are added to the virtual
  machine.  
  In a non-dedicated environment the amount of time it takes until
  a machine becomes available is not bound.
  A program should call 
  \func{pvm\_notify()} before calling
  \func{pvm\_addhosts()}. When a host is added later, the program
  will be notified in the usual PVM 
  fashion (with a \func{PvmHostAdd} notification message).

  After receiving a \func{PvmHostAdd} notification, the PVM master can
  unpack the following information about the added host: an integer
  specifying the TID of the new host, a string specifying the name of
  the new host, followed by a string specifying the machine class of
  the new host.  The PVM master can then call \func{pvm\_spawn()} to
  start a worker process on the new host, specifying this machine
  class as the architecture and using the appropriate executable path
  for this machine class.  Note that the name of the host is given by
  the startd and may be of the form ``vmN@hostname'' on SMP machines.
    

\item \func{pvm\_notify()}.  Under Condor, there are two additional 
  possible notification types
  to the function \func{pvm\_notify()}.
  They are \func{PvmHostSuspend} and
  \func{PvmHostResume}.
\index{PVM!PvmHostSuspend and PvmHostResume notifications}
  The program calls \func{pvm\_notify()}
  with a host tid and \func{PvmHostSuspend} (or \func{PvmHostResume})
  as arguments, and the program will receive
  a notification for the event of a host being suspended.
  Note that a notification occurs only once for each request.
  As an example,
  a \func{PvmHostSuspend} 
  notification request for tid 4 results in a single \func{PvmHostSuspend}
  message for tid 4. 
  There will not be another \func{PvmHostSuspend} message for
  that tid without another notification request.

  The easiest way to handle this is the following:  When a worker
  node starts up, set up a notification for \func{PvmHostSuspend} on
  its tid.  When that node gets suspended, set up a \func{PvmHostResume}
  notification.  When it resumes, set up a \func{PvmHostSuspend}
  notification.

  If your application uses the \func{PvmHostSuspend} and
  \func{PvmHostResume} notification types, you will need to modify
  your PVM distribution to support them as follows.  First, go to your
  \$(PVM\_ROOT). In \File{include/pvm3.h}, add
\begin{verbatim}
#define PvmHostSuspend  6   /* condor suspension */
#define PvmHostResume   7   /* condor resumption */
\end{verbatim}

  to the list of "pvm\_notify kinds".
  In \File{src/lpvmgen.c}, in \func{pvm\_notify()}, change 

\begin{verbatim}
} else {
        switch (what) {
        case PvmHostDelete:
        ....
\end{verbatim}
to 
\begin{verbatim}
} else {
        switch (what) {
        case PvmHostSuspend:  /* for condor */
        case PvmHostResume:   /* for condor */
        case PvmHostDelete:
        ....
\end{verbatim}
And that's it. Re-compile, and you're done. 


\item \func{pvm\_spawn()}.  If the flag in \func{pvm\_spawn()} is 
  \func{PvmTaskArch}, then a machine class string 
  should be used.  If there is only one machine class
  in a virtual machine, ``0'' is the string for the desired architecture.

  Under Condor, only one
  PVM task spawned per node is currently allowed,
  due to Condor's machine load checks.
  Most Condor 
  sites will suspend or vacate
  a job if the load on its machine is higher than a specified
  threshold.
  Having more than one PVM task per node pushes the load
  higher than the threshold.

  Also, Condor only supports starting one copy of the executable with
  each call to \func{pvm\_spawn()} (i.e., the fifth argument must
  always be equal to one).  To spawn multiple copies of an executable
  in Condor, you must call \func{pvm\_spawn()} once for each copy.

  A good fault tolerant program will be able to deal with
  \func{pvm\_spawn()} failing.  It happens more often in opportunistic 
  environments like Condor than in dedicated ones.

\item \func{pvm\_exit()}.  If a PVM task calls \func{pvm\_catchout()}
  during its run to catch the output of child tasks,
  \func{pvm\_exit()} will attempt to gather the output of all child
  tasks before returning.  Due to the dynamic nature of the virtual
  machine in Condor, this cleanup procedure (in the PVM library and
  daemon) is error-prone and should be avoided.  So, any PVM tasks
  which call \func{pvm\_catchout()} should be sure to call it again
  with a NULL argument to disable output collection before calling
  \func{pvm\_exit()}.

\end{itemize}

%%%%%%%%%%%%%%%%%%%%%%%%%%%%%%%%%%%%%%%%%%%%%%%%%%%%%%%%%%%%%%%%%%%%%%
\subsection{\label{sec:PVM-Submit}Sample PVM submit file}
%%%%%%%%%%%%%%%%%%%%%%%%%%%%%%%%%%%%%%%%%%%%%%%%%%%%%%%%%%%%%%%%%%%%%%

\index{PVM!submit description file}
\index{submit description file!for PVM application}
PVM jobs are submitted to the PVM universe.
The following is an
example of a submit description file for a PVM job.  
This job has a master PVM program called \func{master.exe}.

\footnotesize
\begin{verbatim}
###########################################################
# sample_submit
# Sample submit file for PVM jobs. 
###########################################################

# The job is a PVM universe job.
universe = PVM  

# The executable of the master PVM program is ``master.exe''.
executable = master.exe

input = "in.dat"
output = "out.dat"
error = "err.dat"

###################  Machine class 0  ##################

Requirements = (Arch == "INTEL") && (OpSys == "LINUX") 

# We want at least 2 machines in class 0 before starting the 
# program.  We can use up to 4 machines.
machine_count = 2..4  
queue

###################  Machine class 1  ##################

Requirements = (Arch == "SUN4x") && (OpSys == "SOLARIS26") 

# We need at least 1 machine in class 1 before starting the 
# executable.  We can use up to 3 to start with.
machine_count = 1..3
queue

###################  Machine class 2  ##################

Requirements = (Arch == "INTEL") && (OpSys == "SOLARIS26") 

# We don't need any machines in this class at startup, but we can use
# up to 3. 
machine_count = 0..3
queue

###############################################################
# note: the program will not be started until the least 
#       requirements in all classes are satisfied.
###############################################################
\end{verbatim}
\normalsize

In this sample submit file, the command \verb@universe = PVM@
specifies that the jobs should be submitted into PVM universe.

The command \verb@executable = master.exe@ tells Condor that the PVM
master program is \Prog{master.exe}.  This program will be started on
the submitting machine.  The workers should be spawned by this master
program during execution.

The \verb@input@, \verb@output@, and \verb@error@ commands specify
files that should be redirected to the standard in, out, and error of
the PVM master program.  Note that these files will not include output
from worker processes unless the master calls \func{pvm\_catchout()}.

This submit file also tells Condor that the virtual machine
consists of three different classes of machine.  Class
0 contains machines with INTEL processors running LINUX; class
1 contains machines with SUN4x (SPARC) processors running SOLARIS26;
class 2 contains machines with INTEL processors running SOLARIS26.

By using \verb@machine_count = <min>..<max>@, the submit file tells
Condor that before the PVM master is started, there should be at least
\verb@<min>@ 
number of machines of the current class.  It also asks Condor to give
it as many as \verb@<max>@ machines.  During the execution of the program,
the application may request more machines of each of the class by calling
\func{pvm\_addhosts()} with a string specifying the machine
class.
It is often useful to specify \verb@<min>@ of 0 for each
class, so the PVM master will be started immediately when the first
host from any machine class is allocated.

The \func{queue} command should be inserted after the specifications of
each class.

\index{PVM (Parallel Virtual Machine)|)}
