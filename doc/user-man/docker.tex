%%%%%%%%%%%%%%%%%%%%%%%%%%%%%%%%%%%%%%%
\section{\label{sec:dockeruniverse}Docker Universe Applications}
%%%%%%%%%%%%%%%%%%%%%%%%%%%%%%%%%%%%%%%
\index{docker universe|(}
\index{universe!docker}
A docker universe job instantiates a Docker container
from a Docker image, and HTCondor manages the running
of that container as an HTCondor job, on an execute machine.
This running container can then be managed as any HTCondor job.
For example, it can be scheduled, removed, put on hold, 
or be part of a workflow managed by DAGMan.

The docker universe job will only be matched with an execute host
that advertises its capability to run docker universe jobs.
When an execute machine with docker support starts, 
the machine checks to
see if the \Prog{docker} command is available and has the correct
settings for HTCondor.  
Docker support is advertised if available and if it has the correct settings.

The image from which the container is instantiated is
defined by specifying a Docker image with the submit command
\SubmitCmd{docker\_image}.  
This image must be pre-staged on a docker
hub that the execute machine can access.

After submission, the job is treated much the same way as a vanilla 
universe job.  
Details of file transfer are the same as applied to 
the vanilla universe.  
One of the benefits of Docker containers is 
the file system isolation they provide.  
Each container has a distinct file system, 
from the root on down, and this file
system is completely independent of the file system on the host machine.
The container does not share a file system with either the execute
host or the submit host, with the exception of the scratch directory,
which is volume mounted to the host, and is the initial working
directory of the job.

Therefore,
the submit description file should contain the submit command
\begin{verbatim}
  should_transfer_files = YES
\end{verbatim}
With this command,  all input and output files will be transferred
as required to and from the scratch directory mounted as a
Docker volume.

If no \SubmitCmd{executable} is specified in the submit description file,
it is presumed that the Docker container has a default command to run.

When the job completes, is held, evicted, 
or is otherwise removed from the machine, the container will be removed.

Here is a complete submit description file for a sample docker universe job:
\begin{verbatim}
  universe                = docker
  docker_image            = debian
  executable              = /bin/cat
  arguments               = /etc/hosts
  should_transfer_files   = YES
  when_to_transfer_output = ON_EXIT
  output                  = out.$(Process)
  error                   = err.$(Process)
  log                     = log.$(Process)
  request_memory          = 100M
  queue 1
\end{verbatim}

A debian container is the HTCondor job,
and it runs the \Prog{/bin/cat} program on the \File{/etc/hosts} file
before exiting.

\index{docker universe|)}
