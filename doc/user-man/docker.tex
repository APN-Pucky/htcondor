%%%%%%%%%%%%%%%%%%%%%%%%%%%%%%%%%%%%%%%
\section{\label{sec:dockeruniverse}Docker Universe Applications}
%%%%%%%%%%%%%%%%%%%%%%%%%%%%%%%%%%%%%%%
\index{docker universe|(}
\index{universe!docker}
A docker universe job causes an instantiation of a container,
and the job is run within that container,
providing all of the benefits of Docker containers to the job.  

The docker universe job will only be matched with an execute host
that advertises its capability to run docker universe jobs.
The details of which container is instantiated are defined by
specifying a Docker image with the submit command
\SubmitCmd{docker\_image}.

After submission, the job is treated as a vanilla universe job.
Details of file transfer are the same as applied to the vanilla universe.
One of the benefits of Docker containers is the isolation they provide.
This extends to the file system within a container.
The container will not share a file system with either the execute
host or the submit host.
Therefore,
the submit description file should contain the submit command
\begin{verbatim}
  should_transfer_files = YES
\end{verbatim}
With this command,  all input and output files will be transferred
as required to and from the scratch directory mounted as a
Docker volume.

If no \SubmitCmd{executable} is specified in the submit description file,
it is presumed that the Docker container has a default command to run.

When the job completes, or is otherwise evicted,
the container will be removed.

\index{docker universe|)}
