This section provides a brief summary of what can be done once jobs
are submitted. The basic mechanisms for monitoring a job are
introduced, but the commands are discussed briefly.
You are encouraged to
look at the man pages of the commands referred to (located in
% Karen changed this by adding sec: to both lines
Chapter~\ref{sec:command-reference} beginning on
page~\pageref{sec:command-reference}) for more information. 

When jobs are submitted, Condor will attempt to find resources
to run the jobs. 
A list of all those with jobs submitted
may be obtained through \Condor{status}
\index{Condor commands!condor\_status}
with the 
\Arg{-submitters} option. 
An example of this would yield output similar to:
\footnotesize
\begin{verbatim}
%  condor_status -submitters

Name                 Machine      Running IdleJobs HeldJobs

ballard@cs.wisc.edu  bluebird.c         0       11        0
nice-user.condor@cs. cardinal.c         6      504        0
wright@cs.wisc.edu   finch.cs.w         1        1        0
jbasney@cs.wisc.edu  perdita.cs         0        0        5

                           RunningJobs           IdleJobs           HeldJobs

 ballard@cs.wisc.edu                 0                 11                  0
 jbasney@cs.wisc.edu                 0                  0                  5
nice-user.condor@cs.                 6                504                  0
  wright@cs.wisc.edu                 1                  1                  0

               Total                 7                516                  5
\end{verbatim}
\normalsize

%%%%%%%%%%%%%%%%%%%%%%%%%%%%%%%%%%%%%%%%%%%%%%%%%%%%%%%%%%%%%%%%%%%%%%
\subsection{Checking on the progress of jobs}
%%%%%%%%%%%%%%%%%%%%%%%%%%%%%%%%%%%%%%%%%%%%%%%%%%%%%%%%%%%%%%%%%%%%%%
At any time, you can check on the status of your jobs with the \Condor{q}
command.
\index{Condor commands!condor\_q}
This command displays the status of all queued jobs.
An example of the output from \Condor{q} is
\footnotesize
\begin{verbatim}
%  condor_q

-- Submitter: submit.chtc.wisc.edu : <128.104.55.9:32772> : submit.chtc.wisc.edu
 ID      OWNER            SUBMITTED     RUN_TIME ST PRI SIZE CMD               
711197.0   aragorn         1/15 19:18   0+04:29:33 H  0   0.0  script.sh         
894381.0   frodo           3/16 09:06  82+17:08:51 R  0   439.5 elk elk.in        
894386.0   frodo           3/16 09:06  82+20:21:28 R  0   219.7 elk elk.in        
894388.0   frodo           3/16 09:06  81+17:22:10 R  0   439.5 elk elk.in        
1086870.0   gollum          4/27 09:07   0+00:10:14 I  0   7.3  condor_dagman     
1086874.0   gollum          4/27 09:08   0+00:00:01 H  0   0.0  RunDC.bat         
1297254.0   legolas         5/31 18:05  14+17:40:01 R  0   7.3  condor_dagman     
1297255.0   legolas         5/31 18:05  14+17:39:55 R  0   7.3  condor_dagman     
1297256.0   legolas         5/31 18:05  14+17:39:55 R  0   7.3  condor_dagman     
1297259.0   legolas         5/31 18:05  14+17:39:55 R  0   7.3  condor_dagman     
1297261.0   legolas         5/31 18:05  14+17:39:55 R  0   7.3  condor_dagman     
1302278.0   legolas         6/4  12:22   1+00:05:37 I  0   390.6 mdrun_1.sh        
1304740.0   legolas         6/5  00:14   1+00:03:43 I  0   390.6 mdrun_1.sh        
1304967.0   legolas         6/5  05:08   0+00:00:00 I  0   0.0  mdrun_1.sh        

14 jobs; 4 idle, 8 running, 2 held

\end{verbatim} 
\normalsize
This output contains many columns of information about the
queued jobs.
\index{status!of queued jobs}
\index{job!state}
The \verb@ST@ column (for status) shows the status of
current jobs in the queue:
\begin{description} 
  \item{\verb@R@}:  The job is currently running.
  \item{\verb@I@}:  The job is idle.  It is not running right
now, because it is waiting for a machine to become available.
  \item{\verb@H@}:  The job is the hold state. In the hold state,
the job will not be scheduled to
run until it is released. See the \Condor{hold}
manual page located on page~\pageref{man-condor-hold}
and the \Condor{release}
manual page located on page~\pageref{man-condor-release}.
\end{description} 
The \verb@RUN_TIME@ time reported for a job is the time that has been
committed to the job.

Another useful method of tracking the progress of jobs is through the
user log.  If you have specified a \AdAttr{log} command in 
your submit file, the progress of the job may be followed by viewing the
log file.  Various events such as execution commencement, checkpoint, eviction 
and termination are logged in the file.
Also logged is the time at which the event occurred.

When a job begins to run, Condor starts up a \Condor{shadow} process
\index{condor\_shadow}
\index{remote system call!condor\_shadow}
on the submit machine.  The shadow process is the mechanism by which the
remotely executing jobs can access the environment from which it was
submitted, such as input and output files.  

It is normal for a machine which has submitted hundreds of jobs to have 
hundreds of \Condor{shadow} processes running on the machine.
Since the text segments of all these processes is the same,
the load on the submit machine is usually not significant.
If there is degraded performance, limit 
the number of jobs that can run simultaneously by reducing the 
\Macro{MAX\_JOBS\_RUNNING} configuration variable.

You can also find all the machines that are running your job through the
\Condor{status} command.
\index{Condor commands!condor\_status}
For example, to find all the machines that are
running jobs submitted by \Expr{breach@cs.wisc.edu}, type:
\footnotesize
\begin{verbatim}
%  condor_status -constraint 'RemoteUser == "breach@cs.wisc.edu"'

Name       Arch     OpSys        State      Activity   LoadAv Mem  ActvtyTime

alfred.cs. INTEL    LINUX        Claimed    Busy       0.980  64    0+07:10:02
biron.cs.w INTEL    LINUX        Claimed    Busy       1.000  128   0+01:10:00
cambridge. INTEL    LINUX        Claimed    Busy       0.988  64    0+00:15:00
falcons.cs INTEL    LINUX        Claimed    Busy       0.996  32    0+02:05:03
happy.cs.w INTEL    LINUX        Claimed    Busy       0.988  128   0+03:05:00
istat03.st INTEL    LINUX        Claimed    Busy       0.883  64    0+06:45:01
istat04.st INTEL    LINUX        Claimed    Busy       0.988  64    0+00:10:00
istat09.st INTEL    LINUX        Claimed    Busy       0.301  64    0+03:45:00
...
\end{verbatim}
\normalsize
To find all the machines that are running any job at all, type:
\footnotesize
\begin{verbatim}
%  condor_status -run

Name       Arch     OpSys        LoadAv RemoteUser           ClientMachine  

adriana.cs INTEL    LINUX        0.980  hepcon@cs.wisc.edu   chevre.cs.wisc.
alfred.cs. INTEL    LINUX        0.980  breach@cs.wisc.edu   neufchatel.cs.w
amul.cs.wi X86_64   LINUX        1.000  nice-user.condor@cs. chevre.cs.wisc.
anfrom.cs. X86_64   LINUX        1.023  ashoks@jules.ncsa.ui jules.ncsa.uiuc
anthrax.cs INTEL    LINUX        0.285  hepcon@cs.wisc.edu   chevre.cs.wisc.
astro.cs.w INTEL    LINUX        1.000  nice-user.condor@cs. chevre.cs.wisc.
aura.cs.wi X86_64   WINDOWS      0.996  nice-user.condor@cs. chevre.cs.wisc.
balder.cs. INTEL    WINDOWS      1.000  nice-user.condor@cs. chevre.cs.wisc.
bamba.cs.w INTEL    LINUX        1.574  dmarino@cs.wisc.edu  riola.cs.wisc.e
bardolph.c INTEL    LINUX        1.000  nice-user.condor@cs. chevre.cs.wisc.
...
\end{verbatim}
\normalsize

%%%%%%%%%%%%%%%%%%%%%%%%%%%%%%%%%%%%%%%%%%%%%%%%%%%%%%%%%%%%%%%%%%%%%%
\subsection{Removing a job from the queue}
%%%%%%%%%%%%%%%%%%%%%%%%%%%%%%%%%%%%%%%%%%%%%%%%%%%%%%%%%%%%%%%%%%%%%%
A job can be removed from the queue at any time by using the \Condor{rm}
\index{Condor commands!condor\_rm}
command.  If the job that is being removed is currently running, the job
is killed without a checkpoint, and its queue entry is removed.  
The following example shows the queue of jobs before and after
a job is removed.
\footnotesize
\begin{verbatim}
%  condor_q

-- Submitter: froth.cs.wisc.edu : <128.105.73.44:33847> : froth.cs.wisc.edu
 ID      OWNER            SUBMITTED    CPU_USAGE ST PRI SIZE CMD               
 125.0   jbasney         4/10 15:35   0+00:00:00 I  -10 1.2  hello.remote      
 132.0   raman           4/11 16:57   0+00:00:00 R  0   1.4  hello             

2 jobs; 1 idle, 1 running, 0 held

%  condor_rm 132.0
Job 132.0 removed.

%  condor_q

-- Submitter: froth.cs.wisc.edu : <128.105.73.44:33847> : froth.cs.wisc.edu
 ID      OWNER            SUBMITTED    CPU_USAGE ST PRI SIZE CMD               
 125.0   jbasney         4/10 15:35   0+00:00:00 I  -10 1.2  hello.remote      

1 jobs; 1 idle, 0 running, 0 held
\end{verbatim}
\normalsize

%%%%%%%%%%%%%%%%%%%%%%%%%%%%%%%%%%%%%%%%%%%%%%%%%%%%%%%%%%%%%%%%%%%%%%
\subsection{Placing a job on hold}
%%%%%%%%%%%%%%%%%%%%%%%%%%%%%%%%%%%%%%%%%%%%%%%%%%%%%%%%%%%%%%%%%%%%%%
\index{Condor commands!condor\_hold}
\index{Condor commands!condor\_release}
\index{job!state}
A job in the queue may be placed on hold by running the command
\Condor{hold}.
A job in the hold state remains in the hold state until later released
for execution by the command \Condor{release}.

Use of the \Condor{hold} command causes a hard kill signal to be sent
to a currently running job (one in the running state).
For a standard universe job, this means that no checkpoint is
generated before the job stops running and enters the hold state.
When released, this standard universe job continues its execution
using the most recent checkpoint available.

Jobs in universes other than the standard universe that are running
when placed on hold will start over from the beginning when 
released.

The manual page for \Condor{hold}
on page~\pageref{man-condor-hold}
and the manual page for \Condor{release}
on page~\pageref{man-condor-release}
contain usage details.

%%%%%%%%%%%%%%%%%%%%%%%%%%%%%%%%%%%%%%%%%%%%%%%%%%%%%%%%%%%%%%%%%%%%%%
\subsection{\label{sec:job-prio}Changing the priority of jobs}
%%%%%%%%%%%%%%%%%%%%%%%%%%%%%%%%%%%%%%%%%%%%%%%%%%%%%%%%%%%%%%%%%%%%%%

\index{job!priority}
\index{priority!of a job}
In addition to the priorities assigned to each user, Condor also provides
each user with the capability of assigning priorities to each submitted job.
These job priorities are local to each queue and can be any integer value, with
higher values meaning better priority.

The default priority of a job is 0, but can be changed using the \Condor{prio}
command.
\index{Condor commands!condor\_prio}
For example, to change the priority of a job to -15,
\footnotesize
\begin{verbatim}
%  condor_q raman

-- Submitter: froth.cs.wisc.edu : <128.105.73.44:33847> : froth.cs.wisc.edu
 ID      OWNER            SUBMITTED    CPU_USAGE ST PRI SIZE CMD               
 126.0   raman           4/11 15:06   0+00:00:00 I  0   0.3  hello             

1 jobs; 1 idle, 0 running, 0 held

%  condor_prio -p -15 126.0

%  condor_q raman

-- Submitter: froth.cs.wisc.edu : <128.105.73.44:33847> : froth.cs.wisc.edu
 ID      OWNER            SUBMITTED    CPU_USAGE ST PRI SIZE CMD               
 126.0   raman           4/11 15:06   0+00:00:00 I  -15 0.3  hello             

1 jobs; 1 idle, 0 running, 0 held
\end{verbatim}
\normalsize

It is important to note that these \emph{job} priorities are completely 
different from the \emph{user} priorities assigned by Condor.  Job priorities
do not impact user priorities.  They are only a mechanism for the user to
identify the relative importance of jobs among all the jobs submitted by the
user to that specific queue.

%%%%%%%%%%%%%%%%%%%%%%%%%%%%%%%%%%%%%%%%%%%%%%%%%%%%%%%%%%%%%%%%%%%%%%
\subsection{\label{sec:job-not-running}Why is the job not running?}
%%%%%%%%%%%%%%%%%%%%%%%%%%%%%%%%%%%%%%%%%%%%%%%%%%%%%%%%%%%%%%%%%%%%%%
\index{job!analysis}
\index{job!not running}
Users occasionally find that their jobs do not run.
There are many possible reasons why a specific job is not running.
The following prose attempts to identify some of the potential issues
behind why a job is not running.

At the most basic level, the user knows the status of a job by
using \Condor{q} to see that the job is not running.
By far, the most common reason (to the novice Condor job submitter)
why the job is not running is that Condor has not yet 
been through its periodic negotiation cycle,
in which queued jobs are assigned to machines within the pool 
and begin their execution.
This periodic event occurs by default once every 5 minutes,
implying that the user ought to wait a few minutes before
searching for reasons why the job is not running.

Further inquiries are dependent on whether the job has 
never run at all, or has run for at least a little bit.

For jobs that have never run,
\index{Condor commands!condor\_q}
many problems can be diagnosed by using the \Opt{-analyze}
option of the \Condor{q} command.
For example, a job (assigned the cluster.process value of
121.000) submitted to the local pool at UW-Madison
is not running.
Running \Condor{q}'s analyzer provided the following information:

\footnotesize
\begin{verbatim}
% condor_q -pool -analyze 121.000
-- Submitter: puffin.cs.wisc.edu : <128.105.185.14:34203> : puffin.cs.wisc.edu
---
121.000:  Run analysis summary.  Of 1592 machines,
   1382 are rejected by your job's requirements
     25 reject your job because of their own requirements
    185 match but are serving users with a better priority in the pool
      0 match but reject the job for unknown reasons
      0 match but will not currently preempt their existing job
      0 match but are currently offline
      0 are available to run your job

The Requirements expression for your job is:
( ( target.Arch == "X86_64" || target.Arch == "INTEL" ) &&
( target.Group == "TestPool" ) ) && ( target.OpSys == "LINUX" ) &&
( target.Disk >= DiskUsage ) && ( ( target.Memory * 1024 ) >= ImageSize ) &&
( TARGET.FileSystemDomain == MY.FileSystemDomain )

    Condition                         Machines Matched    Suggestion
    ---------                         ----------------    ----------
1   ( target.Group == "TestPool" )    274                  
2   ( TARGET.FileSystemDomain == "cs.wisc.edu" )1258                 
3   ( target.OpSys == "LINUX" )       1453                 
4   ( target.Arch == "X86_64" || target.Arch == "INTEL" )
                                      1573                 
5   ( target.Disk >= 100000 )         1589                 
6   ( ( 1024 * target.Memory ) >= 100000 )1592                 

The following attributes are missing from the job ClassAd:

CheckpointPlatform

\end{verbatim}
\normalsize

This example also shows that the job does not run because the job
does not have a high enough priority to cause any of 185 other running jobs
to be preempted.

While the analyzer can diagnose most common problems, there are some situations
that it cannot reliably detect due to the instantaneous and local nature of the
information it uses to detect the problem.  Thus, it may be that the analyzer
reports that resources are available to service the request, but the job still 
has not run.  In most of these situations, the delay is transient, and the
job will run following the next negotiation cycle.

A second class of problems represents jobs that do or did run,
for at least a short while, but are no longer running.
The first issue is identifying whether the job is in this category.
The \Condor{q} command is not enough; it only tells the
current state of the job.
The needed information will be in the \SubmitCmd{log} file 
or the \SubmitCmd{error} file, as defined in the submit description file
for the job.
If these files are not defined, then there is little hope of
determining if the job ran at all.
For a job that ran, even for the briefest amount of time,
the \SubmitCmd{log} file will contain an event of type 1,
which will contain the string
\verb@Job executing on host@.

A job may run for a short time, before failing due to a file permission
problem.
The log file used by the \Condor{shadow} daemon will contain more information
if this is the problem.
This log file is associated with the machine on which the job was submitted.
The location and name of this log file may be discovered on the
submitting machine, using the command
\footnotesize
\begin{verbatim}
%  condor_config_val SHADOW_LOG
\end{verbatim}
\normalsize

Memory and swap space problems may be identified by looking at the log
file used by the \Condor{schedd} daemon.
The location and name of this log file may be discovered on the
submitting machine, using the command
\footnotesize
\begin{verbatim}
%  condor_config_val SCHEDD_LOG
\end{verbatim}
\normalsize
A swap space problem will show in the log with the following message:
\footnotesize
\begin{verbatim}
2/3 17:46:53 Swap space estimate reached! No more jobs can be run!
12/3 17:46:53     Solution: get more swap space, or set RESERVED_SWAP = 0
12/3 17:46:53     0 jobs matched, 1 jobs idle
\end{verbatim}
\normalsize
As an explanation,
Condor computes the total swap space on the submit machine.
It then tries to limit the total number of jobs it
will spawn based on an estimate of the size of the \Condor{shadow}
daemon's memory footprint and a configurable amount of swap space
that should be reserved.
This is done to avoid the
situation within a very large pool
in which all the jobs are submitted from a single host.
The huge number of \Condor{shadow} processes would
overwhelm the submit machine,
and it would run out of swap space and thrash.

Things can go wrong if a machine has a lot of physical memory and
little or no swap space.
Condor does not consider the physical memory size,
so the situation occurs where Condor thinks
it has no swap space to work with,
and it will not run the submitted jobs.

To see how much swap space Condor thinks a given machine has, use
the output of a \Condor{status} command of the following form:

\footnotesize
\begin{verbatim}
% condor_status -schedd [hostname] -long | grep VirtualMemory
\end{verbatim}
\normalsize
If the value listed is 0, then this is what is confusing Condor.
There are two ways to fix the problem:

\begin{enumerate}
\item Configure the machine with some real swap space.

\item Disable this check within Condor.
Define the amount of reserved swap space for the submit machine to 0.
Set \Macro{RESERVED\_SWAP} to 0 in the configuration file:

\begin{verbatim}
RESERVED_SWAP = 0
\end{verbatim}

and then send a \Condor{restart} to the submit machine.
\end{enumerate}



%%%%%%%%%%%%%%%%%%%%%%%%%%%%%%%%%%%%%%%%%%%%%%%%%%%%%%%%%%%%%%%%%%%%%%
\subsection{\label{sec:job-log-events}In the Log File}
%%%%%%%%%%%%%%%%%%%%%%%%%%%%%%%%%%%%%%%%%%%%%%%%%%%%%%%%%%%%%%%%%%%%%%
\index{job!log events}
\index{log files!event descriptions}
In a job's log file are a listing of events in
chronological order that occurred during the life of the job.
The formatting of the events is always the same, 
so that they may be machine readable.
Four fields are always present,
and they will most often be followed by other fields that give further
information that is specific to the type of event.

The first field in an event is the numeric value assigned as the
event type in a 3-digit format.
The second field identifies the job which generated the event. 
Within parentheses are the ClassAd job attributes of
\AdAttr{ClusterId} value, 
\AdAttr{ProcId} value, 
and the node number  for parallel universe jobs or a set of zeros
(for jobs run under all other universes),
separated by periods.
The third field is the date and time of the event logging.  
The fourth field is a string that briefly describes the event.
Fields that follow the fourth field give further information for the specific
event type.

These are all of the events that can show up in a job log file:

\noindent\Bold{Event Number:} 000 \\
\Bold{Event Name:} Job submitted \\
\Bold{Event Description:} This event occurs when a user submits a job.
It is the first event you will see for a job, and it should only occur
once. 

\noindent\Bold{Event Number:} 001 \\
\Bold{Event Name:} Job executing \\
\Bold{Event Description:} This shows up when a job is running.
It might occur more than once.

\noindent\Bold{Event Number:} 002 \\
\Bold{Event Name:} Error in executable \\
\Bold{Event Description:} The job couldn't be run because the
executable was bad.

\noindent\Bold{Event Number:} 003 \\
\Bold{Event Name:} Job was checkpointed \\
\Bold{Event Description:} The job's complete state was written to a checkpoint
file.  
This might happen without the job being removed from a machine,
because the checkpointing can happen periodically. 

\noindent\Bold{Event Number:} 004 \\
\Bold{Event Name:} Job evicted from machine \\
\Bold{Event Description:} A job was removed from a machine before it finished,
usually for a policy reason: perhaps an interactive user has claimed
the computer, or perhaps another job is higher priority.

\noindent\Bold{Event Number:} 005 \\
\Bold{Event Name:} Job terminated \\
\Bold{Event Description:} The job has completed.

\noindent\Bold{Event Number:} 006 \\
\Bold{Event Name:} Image size of job updated \\
\Bold{Event Description:} This is informational. 
It is referring to the memory that the job is using while running. It
does not reflect the state of the job.

\noindent\Bold{Event Number:} 007 \\
\Bold{Event Name:} Shadow exception \\
\Bold{Event Description:} 
The \Condor{shadow}, a program on the submit computer that watches
over the job and performs some services for the job, failed for some
catastrophic reason. The job will leave the machine and go back into
the queue.

\noindent\Bold{Event Number:} 008 \\
\Bold{Event Name:} Generic log event \\
\Bold{Event Description:} Not used.

\noindent\Bold{Event Number:} 009 \\
\Bold{Event Name:} Job aborted \\
\Bold{Event Description:} The user canceled the job.

\noindent\Bold{Event Number:} 010 \\
\Bold{Event Name:} Job was suspended \\
\Bold{Event Description:} The job is still on the computer, but it is no longer
executing. 
This is usually for a policy reason, like an interactive user using
the computer. 

\noindent\Bold{Event Number:} 011 \\
\Bold{Event Name:} Job was unsuspended \\
\Bold{Event Description:} The job has resumed execution, after being
suspended earlier. 

\noindent\Bold{Event Number:} 012 \\
\Bold{Event Name:} Job was held \\
\Bold{Event Description:} The user has paused the job, perhaps with
the \Condor{hold} command.
It was stopped, and will go back into the queue again until it is
aborted or released. 

\noindent\Bold{Event Number:} 013 \\
\Bold{Event Name:} Job was released \\
\Bold{Event Description:} The user is requesting that a job on hold be re-run.

\noindent\Bold{Event Number:} 014 \\
\Bold{Event Name:} Parallel node executed \\
\Bold{Event Description:} A parallel (MPI) program is running on a node.

\noindent\Bold{Event Number:} 015 \\
\Bold{Event Name:} Parallel node terminated \\
\Bold{Event Description:} A parallel (MPI) program has completed on a node.

\noindent\Bold{Event Number:} 016 \\
\Bold{Event Name:} POST script terminated \\
\Bold{Event Description:} A node in a DAGMan work flow has a script
that should be run after a job. 
The script is run on the submit host. 
This event signals that the post script has completed.

\noindent\Bold{Event Number:} 017 \\
\Bold{Event Name:} Job submitted to Globus \\
\Bold{Event Description:} A grid job has been delegated to Globus
(version 2, 3, or 4).

\noindent\Bold{Event Number:} 018 \\
\Bold{Event Name:} Globus submit failed \\
\Bold{Event Description:} The attempt to delegate a job to Globus
failed. 

\noindent\Bold{Event Number:} 019 \\
\Bold{Event Name:} Globus resource up \\
\Bold{Event Description:} The Globus resource that a job wants to run
on was unavailable, but is now available.

\noindent\Bold{Event Number:} 020 \\
\Bold{Event Name:} Detected Down Globus Resource \\
\Bold{Event Description:} The Globus resource that a job wants to run
on has become unavailable. 

\noindent\Bold{Event Number:} 021 \\
\Bold{Event Name:} Remote error \\
\Bold{Event Description:} The \Condor{starter} (which monitors the job
on the execution machine) has failed.

\noindent\Bold{Event Number:} 022 \\
\Bold{Event Name:} Remote system call socket lost \\
\Bold{Event Description:} The \Condor{shadow} and \Condor{starter}
(which communicate while the job runs) have lost contact.

\noindent\Bold{Event Number:} 023 \\
\Bold{Event Name:} Remote system call socket reestablished \\
\Bold{Event Description:} The \Condor{shadow} and \Condor{starter}
(which communicate while the job runs) have been able to resume
contact before the job lease expired.

\noindent\Bold{Event Number:} 024 \\
\Bold{Event Name:} Remote system call reconnect failure \\
\Bold{Event Description:} The \Condor{shadow} and \Condor{starter}
(which communicate while the job runs) were unable to resume
contact before the job lease expired.

\noindent\Bold{Event Number:} 025 \\
\Bold{Event Name:} Grid Resource Back Up \\
\Bold{Event Description:} A grid resource that was previously
unavailable is now available.

\noindent\Bold{Event Number:} 026 \\
\Bold{Event Name:} Detected Down Grid Resource \\
\Bold{Event Description:} The grid resource that a job is to
run on is unavailable.

\noindent\Bold{Event Number:} 027 \\
\Bold{Event Name:} Job submitted to grid resource \\
\Bold{Event Description:} A job has been submitted,
and is under the auspices of the grid resource.

\noindent\Bold{Event Number:} 028 \\
\Bold{Event Name:} Job ad information event triggered. \\
\Bold{Event Description:} Extra job ad attributes. This event is
written as a supplement to other events when the configuration
parameter \Macro{EVENT\_LOG\_JOB\_AD\_INFORMATION\_ATTRS} is set.

\noindent\Bold{Event Number:} 029 \\
\Bold{Event Name:} The job's remote status is unknown \\
\Bold{Event Description:} No updates of the job's remote status
have been received for 15 minutes.

\noindent\Bold{Event Number:} 030 \\
\Bold{Event Name:} The job's remote status is known again \\
\Bold{Event Description:} An update has been received for a job whose
remote status was previous logged as unknown.

%%%%%%%%%%%%%%%%%%%%%%%%%%%%%%%%%%%%%%%%%%%%%%%%%%%%%%%%%%%%%%%%%%%%%e
\subsection{\label{sec:job-completion}Job Completion}
%%%%%%%%%%%%%%%%%%%%%%%%%%%%%%%%%%%%%%%%%%%%%%%%%%%%%%%%%%%%%%%%%%%%%%
\index{job!completion}

When your Condor job completes(either through normal means or abnormal
termination by signal), Condor will remove it from the job queue (i.e.,
it will no longer appear in the output of \Condor{q}) and insert it into
the job history file.  You can examine the job history file with the
\Condor{history} command. If you specified a log file in your submit
description file, then the job exit status will be recorded there as well.

By default, Condor will send you an email message
when your job completes.  You can modify this behavior with the
\Condor{submit} ``notification'' command.
The message will include the exit status of your job (i.e., the
argument your job passed to the exit system call when it completed) or
notification that your job was killed by a signal.  It will also
include the following statistics (as appropriate) about your job:

\begin{description}

\item[Submitted at:] when the job was submitted with \Condor{submit}

\item[Completed at:] when the job completed

\item[Real Time:] elapsed time between when the job was submitted and
when it completed (days hours:minutes:seconds)

\item[Run Time:] total time the job was running (i.e., real time minus
queuing time)

\item[Committed Time:] total run time that contributed to job
completion (i.e., run time minus the run time that was lost because
the job was evicted without performing a checkpoint)

\item[Remote User Time:] total amount of committed time the job spent
executing in user mode

\item[Remote System Time:] total amount of committed time the job spent
executing in system mode 

\item[Total Remote Time:] total committed CPU time for the job

\item[Local User Time:] total amount of time this job's
\Condor{shadow} (remote system call server) spent executing in user
mode

\item[Local System Time:] total amount of time this job's
\Condor{shadow} spent executing in system mode

\item[Total Local Time:] total CPU usage for this job's \Condor{shadow}

\item[Leveraging Factor:] the ratio of total remote time to total
system time (a factor below 1.0 indicates that the job ran
inefficiently, spending more CPU time performing remote system calls
than actually executing on the remote machine)

\item[Virtual Image Size:] memory size of the job, computed when the
job checkpoints

\item[Checkpoints written:] number of successful checkpoints performed
by the job

\item[Checkpoint restarts:] number of times the job successfully
restarted from a checkpoint

\item[Network:] total network usage by the job for checkpointing and
remote system calls

\item[Buffer Configuration:] configuration of remote system call I/O
buffers

\item[Total I/O:] total file I/O detected by the remote system call
library

\item[I/O by File:] I/O statistics per file produced by the remote
system call library

\item[Remote System Calls:] listing of all remote system calls
performed (both Condor-specific and Unix system calls) with a count of
the number of times each was performed

\end{description}
