This section provides a brief summary of what can be done once your jobs
begin execution.  The basic mechanisms for monitoring a job are introduced,
but the commands that are discussed include a lot more functionality than is
displayed in this section.  You are encouraged to look at the man pages of 
the commands referred to for more information.

Once your jobs have been submitted, Condor will attempt to find resources
to run your jobs.  The existence of your requests is communicated to the
Condor manager by registering you as a ``submitter.''  A list of all the
current submitters may be obtained through \Condor{status} with the 
\Arg{-submitters} option, which would yield output similar to the following:
\begin{verbatim}
%  condor_status -submitters

Name                 Machine     Running  IdleJobs  MaxJobsRunning

ashoks@jules.ncsa.ui jules.ncsa       74       54       200
breach@cs.wisc.edu   bianca.cs.       11        0       500
breach@cs.wisc.edu   neufchatel       23        0       500
jbasney@cs.wisc.edu  froth.cs.w        0        1       500
wright@raven.cs.wisc raven.cs.w        1       48       200

                           RunningJobs             IdleJobs

wright@raven.cs.wisc                 1                   48
ashoks@jules.ncsa.ui                74                   54
 jbasney@cs.wisc.edu                 0                    1
  breach@cs.wisc.edu                34                    0

               Total               109                  103
\end{verbatim}

\subsection{Checking on the progress of your jobs}
At any time, you can check on the status of your jobs with the \Condor{q}
tool, which shows the status of all queued jobs along with other information.
To identify jobs which are running, type
\begin{verbatim}
%  condor_q

-- Submittor: froth.cs.wisc.edu : <128.105.73.44:33847> : froth.cs.wisc.edu
 ID      OWNER            SUBMITTED    CPU_USAGE ST PRI SIZE CMD               
 125.0   jbasney         4/10 15:35   0+00:00:00 U  -10 1.2  hello.remote      
 127.0   raman           4/11 15:35   0+00:00:00 R  0   1.4  hello             
 128.0   raman           4/11 15:35   0+00:02:33 I  0   1.4  hello             

3 jobs; 1 unexpanded, 1 idle, 1 running, 0 malformed

\end{verbatim}
The \verb@ST@ column (for status) shows the status of current jobs in the queue.
The ``U'' stands for unexpanded, which means that the job has never run,
``R'' means the the job is running and ``I'' stands for idle, which means the
job has run before, but is not currently running.

\textbf{Note:} The CPU time reported for a job is the time that has been
committed to the job.  Thus, the CPU time is not updated for a job until
the job checkpoints, at which time the job has made guaranteed forward 
progress.

Another useful method of tracking the progress of jobs is through the
\Term{user log} mechanism.  If you have specified a \texttt{log} command in 
your submit file, the progress of the job may be followed by viewing the
log file.  Various events such as execution commencement, checkpoint, eviction 
and termination are logged in the file along with the time at which the event 
occurred.

\subsection{Removing the job from the queue}
A job can be removed from the queue at any time by using the \Condor{rm}
command.  If the job that is being removed is currently running, the job
is killed without a checkpoint, and its queue entry removed.  For example:
\begin{verbatim}
%  condor_q

-- Submittor: froth.cs.wisc.edu : <128.105.73.44:33847> : froth.cs.wisc.edu
 ID      OWNER            SUBMITTED    CPU_USAGE ST PRI SIZE CMD               
 125.0   jbasney         4/10 15:35   0+00:00:00 U  -10 1.2  hello.remote      
 132.0   raman           4/11 16:57   0+00:00:00 R  0   1.4  hello             

2 jobs; 1 unexpanded, 0 idle, 1 running, 0 malformed

%  condor_rm 132.0
Job 132.0 removed.

%  condor_q

-- Submittor: froth.cs.wisc.edu : <128.105.73.44:33847> : froth.cs.wisc.edu
 ID      OWNER            SUBMITTED    CPU_USAGE ST PRI SIZE CMD               
 125.0   jbasney         4/10 15:35   0+00:00:00 U  -10 1.2  hello.remote      

1 jobs; 1 unexpanded, 0 idle, 0 running, 0 malformed
\end{verbatim}

\subsection{While your job is running ...}
When your job begins to run, Condor starts up a \Condor{shadow} process
on the submit machine.  The shadow process is the mechanism by which the
remotely executing jobs can access the environment from which it was
submitted, such as input and output files.  

It is normal for a machine which has submitted hundreds of jobs to have 
hundreds of shadows running on the machine.  Since the text segments of 
all these processes is the same, the load on the submit machine is usually 
not significant.  If, however, you notice degraded performance, you can limit 
the number of jobs that can run simultaneously through the 
\Macro{MAX\_JOBS\_RUNNING} configuration parameter.  Please talk to your 
system administrator for the necessary configuration change.

You can also find all the machines that are running your job through the
\Condor{status} command.  For example, to find all the machines that are
running jobs submitted by ``breach@cs.wisc.edu,'' type:
\begin{verbatim}
%  condor_status -constraint 'RemoteUser == "breach@cs.wisc.edu"'

Name       Arch     OpSys        State      Activity   LoadAv Mem  ActvtyTime

alfred.cs. INTEL    SOLARIS251   Claimed    Busy       0.980  64    0+07:10:02
biron.cs.w INTEL    SOLARIS251   Claimed    Busy       1.000  128   0+01:10:00
cambridge. INTEL    SOLARIS251   Claimed    Busy       0.988  64    0+00:15:00
falcons.cs INTEL    SOLARIS251   Claimed    Busy       0.996  32    0+02:05:03
happy.cs.w INTEL    SOLARIS251   Claimed    Busy       0.988  128   0+03:05:00
istat03.st INTEL    SOLARIS251   Claimed    Busy       0.883  64    0+06:45:01
istat04.st INTEL    SOLARIS251   Claimed    Busy       0.988  64    0+00:10:00
istat09.st INTEL    SOLARIS251   Claimed    Busy       0.301  64    0+03:45:00
...
\end{verbatim}
To find all the machines that are running any job at all, type:
\begin{verbatim}
%  condor_status -run

Name       Arch     OpSys        LoadAv RemoteUser           ClientMachine  

adriana.cs INTEL    SOLARIS251   0.980  hepcon@cs.wisc.edu   chevre.cs.wisc.
alfred.cs. INTEL    SOLARIS251   0.980  breach@cs.wisc.edu   neufchatel.cs.w
amul.cs.wi SUN4u    SOLARIS251   1.000  nice-user.condor@cs. chevre.cs.wisc.
anfrom.cs. SUN4x    SOLARIS251   1.023  ashoks@jules.ncsa.ui jules.ncsa.uiuc
anthrax.cs INTEL    SOLARIS251   0.285  hepcon@cs.wisc.edu   chevre.cs.wisc.
astro.cs.w INTEL    SOLARIS251   1.000  nice-user.condor@cs. chevre.cs.wisc.
aura.cs.wi SUN4u    SOLARIS251   0.996  nice-user.condor@cs. chevre.cs.wisc.
balder.cs. INTEL    SOLARIS251   1.000  nice-user.condor@cs. chevre.cs.wisc.
bamba.cs.w INTEL    SOLARIS251   1.574  dmarino@cs.wisc.edu  riola.cs.wisc.e
bardolph.c INTEL    SOLARIS251   1.000  nice-user.condor@cs. chevre.cs.wisc.
...
\end{verbatim}

\subsection{Changing the priority of jobs}
In addition to the priorities assigned to each user, Condor also provides
each user with the capability of assigning priorities to each submitted job.
These job priorities are local to each queue and range from -20 to +20, with
higher values meaning better priority.

The default priority of a job is 0, but can be changed using the \Condor{prio}
command.  For example, to change the priority of a job to -15,
\begin{verbatim}
%  condor_q raman

-- Submittor: froth.cs.wisc.edu : <128.105.73.44:33847> : froth.cs.wisc.edu
 ID      OWNER            SUBMITTED    CPU_USAGE ST PRI SIZE CMD               
 126.0   raman           4/11 15:06   0+00:00:00 U  0   0.3  hello             

1 jobs; 1 unexpanded, 0 idle, 0 running, 0 malformed

%  condor_prio -p -15 126.0

%  condor_q raman

-- Submittor: froth.cs.wisc.edu : <128.105.73.44:33847> : froth.cs.wisc.edu
 ID      OWNER            SUBMITTED    CPU_USAGE ST PRI SIZE CMD               
 126.0   raman           4/11 15:06   0+00:00:00 U  -15 0.3  hello             

1 jobs; 1 unexpanded, 0 idle, 0 running, 0 malformed
\end{verbatim}

It is important to note that these \emph{job} priorities are completely 
different from the \emph{user} priorities assigned by Condor.  Job priorities
do not impact user priorities.  They are only a mechanism for the user to
identify the relative importance of jobs among all the jobs submitted by the
user to that specific queue.

\subsection{Why won't my job run?}
Users sometimes find that their jobs do not run.  There are several reasons why
a specific job does not run.  These reasons range from failed job or machine
constraints, bias due to preferences, insufficient priority, or the preemption
``throttle'' that is implemented by the \Condor{negotiator} to prevent
thrashing.  Many of these reasons can be diagnosed by using the \Arg{-analyze}
option of \Condor{q}.  For example, the following job submitted by user
``jbasney'' was found not to run for several days.
\begin{verbatim}
% condor_q

-- Submittor: froth.cs.wisc.edu : <128.105.73.44:33847> : froth.cs.wisc.edu
 ID      OWNER            SUBMITTED    CPU_USAGE ST PRI SIZE CMD               
 125.0   jbasney         4/10 15:35   0+00:00:00 U  -10 1.2  hello.remote      

1 jobs; 1 unexpanded, 0 idle, 0 running, 0 malformed
\end{verbatim}

Running \Condor{q}'s analyzer provided the following information:

\begin{verbatim}
%  condor_q 125.0 -analyze

-- Submittor: froth.cs.wisc.edu : <128.105.73.44:33847> : froth.cs.wisc.edu
---
125.000:  Run analysis summary.  Of 323 resource offers,
          323 do not satisfy the request's constraints
            0 resource offer constraints are not satisfied by this request
            0 are serving equal or higher priority customers
            0 are serving more preferred customers
            0 cannot preempt because preemption has been held
            0 are available to service your request

WARNING:  Be advised:
   No resources matched request's constraints
   Check the Requirements expression below:

Requirements = Arch == "INTEL" && OpSys == "SOLARIS251" && 0 && 
  Disk >= ExecutableSize && VirtualMemory >= ImageSize
\end{verbatim}

We see that user ``jbasney'' has inadvertently expressed a \Attr{Requirements}
expression that can never be satisfied due to the \verb@... && 0 && ...@
clause which always evaluates to false.

While the analyzer can diagnose most common problems, there are some situations
that it cannot reliably detect due to the instantaneous and local nature of the
information it uses to detect the problem.  Thus, it may be that the analyzer
reports that resources are available to service the request, but the job still 
does not run.  In most of these situations, the delay is transient, and the
job will run during the next negotiation cycle.

If the problem persists and the analyzer is unable to detect the situation, it
may be that the job begins to run but immediately terminates due to some 
problem.  Viewing the job's error and log files
(specified in the submit command file) and Condor's \Macro{SHADOW\_LOG} file
may assist in tracking down the problem.  If the cause is still unclear, please
contact your system administrator.
