This section provides a brief summary of what can be done once jobs
are submitted. The basic mechanisms for monitoring a job are
introduced, but the commands are discussed briefly.
You are encouraged to
look at the man pages of the commands referred to (located in
% Karen changed this by adding sec: to both lines
Chapter~\ref{sec:command-reference} beginning on
page~\pageref{sec:command-reference}) for more information. 

When jobs are submitted, Condor will attempt to find resources
to run the jobs. 
A list of all those with jobs submitted
may be obtained through \Condor{status}
\index{Condor commands!condor\_status}
with the 
\Arg{-submitters} option. 
An example of this would yield output similar to:
\footnotesize
\begin{verbatim}
%  condor_status -submitters

Name                 Machine      Running IdleJobs HeldJobs

ballard@cs.wisc.edu  bluebird.c         0       11        0
nice-user.condor@cs. cardinal.c         6      504        0
wright@cs.wisc.edu   finch.cs.w         1        1        0
jbasney@cs.wisc.edu  perdita.cs         0        0        5

                           RunningJobs           IdleJobs           HeldJobs

 ballard@cs.wisc.edu                 0                 11                  0
 jbasney@cs.wisc.edu                 0                  0                  5
nice-user.condor@cs.                 6                504                  0
  wright@cs.wisc.edu                 1                  1                  0

               Total                 7                516                  5
\end{verbatim}
\normalsize

%%%%%%%%%%%%%%%%%%%%%%%%%%%%%%%%%%%%%%%%%%%%%%%%%%%%%%%%%%%%%%%%%%%%%%
\subsection{Checking on the progress of jobs}
%%%%%%%%%%%%%%%%%%%%%%%%%%%%%%%%%%%%%%%%%%%%%%%%%%%%%%%%%%%%%%%%%%%%%%
At any time, you can check on the status of your jobs with the \Condor{q}
command.
\index{Condor commands!condor\_q}
This command displays the status of all queued jobs.
An example of the output from \Condor{q} is
\footnotesize
\begin{verbatim}
%  condor_q

-- Submitter: froth.cs.wisc.edu : <128.105.73.44:33847> : froth.cs.wisc.edu
 ID      OWNER            SUBMITTED    CPU_USAGE ST PRI SIZE CMD               
 125.0   jbasney         4/10 15:35   0+00:00:00 I  -10 1.2  hello.remote      
 127.0   raman           4/11 15:35   0+00:00:00 R  0   1.4  hello             
 128.0   raman           4/11 15:35   0+00:02:33 I  0   1.4  hello             

3 jobs; 2 idle, 1 running, 0 held

\end{verbatim} 
\normalsize
This output contains many columns of information about the
queued jobs.
\index{status!of queued jobs}
\index{job!state}
The \verb@ST@ column (for status) shows the status of
current jobs in the queue. An \verb@R@ in the status column
means the the job is currently running.
An \verb@I@ stands for idle. 
The job is not running right
now, because it is waiting for a machine to become available. 
% A suspended job may have an \verb@S@ in the status column.
The status
\verb@H@ is the hold state. In the hold state,
the job will not be scheduled to
run until it is released (see the \Condor{hold}
reference page located on page~\pageref{man-condor-hold}
and the \Condor{release}
reference page located on page~\pageref{man-condor-release}).
Older versions of Condor used a
\verb@U@ in the status column to stand for unexpanded.
In this state,
a job has never 
produced a checkpoint,
and when the job starts running, it will start running from the
beginning.
Newer versions of Condor do not use the \verb@U@ state.

The \verb@CPU_USAGE@ time reported for a job is the time that has been
committed to the job.  It is not updated for a job until
the job checkpoints. At that time, the job has made guaranteed forward 
progress.  Depending upon how the site administrator configured the pool,
several hours may pass between checkpoints, so do not worry if you do
not observe the \verb@CPU_USAGE@ entry changing by the hour.
Also note that this is actual CPU
time as reported by the operating system; it is not time as
measured by a wall clock.

Another useful method of tracking the progress of jobs is through the
user log.  If you have specified a \AdAttr{log} command in 
your submit file, the progress of the job may be followed by viewing the
log file.  Various events such as execution commencement, checkpoint, eviction 
and termination are logged in the file.
Also logged is the time at which the event occurred.

% Karen's note:  degraded performance where?
When your job begins to run, Condor starts up a \Condor{shadow} process
\index{condor\_shadow}
\index{remote system call!condor\_shadow}
on the submit machine.  The shadow process is the mechanism by which the
remotely executing jobs can access the environment from which it was
submitted, such as input and output files.  

It is normal for a machine which has submitted hundreds of jobs to have 
hundreds of shadows running on the machine.  Since the text segments of 
all these processes is the same, the load on the submit machine is usually 
not significant.  If, however, you notice degraded performance, you can limit 
the number of jobs that can run simultaneously through the 
\Macro{MAX\_JOBS\_RUNNING} configuration parameter.  Please talk to your 
system administrator for the necessary configuration change.

You can also find all the machines that are running your job through the
\Condor{status} command.
\index{Condor commands!condor\_status}
For example, to find all the machines that are
running jobs submitted by ``breach@cs.wisc.edu,'' type:
\footnotesize
\begin{verbatim}
%  condor_status -constraint 'RemoteUser == "breach@cs.wisc.edu"'

Name       Arch     OpSys        State      Activity   LoadAv Mem  ActvtyTime

alfred.cs. INTEL    SOLARIS251   Claimed    Busy       0.980  64    0+07:10:02
biron.cs.w INTEL    SOLARIS251   Claimed    Busy       1.000  128   0+01:10:00
cambridge. INTEL    SOLARIS251   Claimed    Busy       0.988  64    0+00:15:00
falcons.cs INTEL    SOLARIS251   Claimed    Busy       0.996  32    0+02:05:03
happy.cs.w INTEL    SOLARIS251   Claimed    Busy       0.988  128   0+03:05:00
istat03.st INTEL    SOLARIS251   Claimed    Busy       0.883  64    0+06:45:01
istat04.st INTEL    SOLARIS251   Claimed    Busy       0.988  64    0+00:10:00
istat09.st INTEL    SOLARIS251   Claimed    Busy       0.301  64    0+03:45:00
...
\end{verbatim}
\normalsize
To find all the machines that are running any job at all, type:
\footnotesize
\begin{verbatim}
%  condor_status -run

Name       Arch     OpSys        LoadAv RemoteUser           ClientMachine  

adriana.cs INTEL    SOLARIS251   0.980  hepcon@cs.wisc.edu   chevre.cs.wisc.
alfred.cs. INTEL    SOLARIS251   0.980  breach@cs.wisc.edu   neufchatel.cs.w
amul.cs.wi SUN4u    SOLARIS251   1.000  nice-user.condor@cs. chevre.cs.wisc.
anfrom.cs. SUN4x    SOLARIS251   1.023  ashoks@jules.ncsa.ui jules.ncsa.uiuc
anthrax.cs INTEL    SOLARIS251   0.285  hepcon@cs.wisc.edu   chevre.cs.wisc.
astro.cs.w INTEL    SOLARIS251   1.000  nice-user.condor@cs. chevre.cs.wisc.
aura.cs.wi SUN4u    SOLARIS251   0.996  nice-user.condor@cs. chevre.cs.wisc.
balder.cs. INTEL    SOLARIS251   1.000  nice-user.condor@cs. chevre.cs.wisc.
bamba.cs.w INTEL    SOLARIS251   1.574  dmarino@cs.wisc.edu  riola.cs.wisc.e
bardolph.c INTEL    SOLARIS251   1.000  nice-user.condor@cs. chevre.cs.wisc.
...
\end{verbatim}
\normalsize

%%%%%%%%%%%%%%%%%%%%%%%%%%%%%%%%%%%%%%%%%%%%%%%%%%%%%%%%%%%%%%%%%%%%%%
\subsection{Removing a job from the queue}
%%%%%%%%%%%%%%%%%%%%%%%%%%%%%%%%%%%%%%%%%%%%%%%%%%%%%%%%%%%%%%%%%%%%%%
A job can be removed from the queue at any time by using the \Condor{rm}
\index{Condor commands!condor\_rm}
command.  If the job that is being removed is currently running, the job
is killed without a checkpoint, and its queue entry is removed.  
The following example shows the queue of jobs before and after
a job is removed.
\footnotesize
\begin{verbatim}
%  condor_q

-- Submitter: froth.cs.wisc.edu : <128.105.73.44:33847> : froth.cs.wisc.edu
 ID      OWNER            SUBMITTED    CPU_USAGE ST PRI SIZE CMD               
 125.0   jbasney         4/10 15:35   0+00:00:00 I  -10 1.2  hello.remote      
 132.0   raman           4/11 16:57   0+00:00:00 R  0   1.4  hello             

2 jobs; 1 idle, 1 running, 0 held

%  condor_rm 132.0
Job 132.0 removed.

%  condor_q

-- Submitter: froth.cs.wisc.edu : <128.105.73.44:33847> : froth.cs.wisc.edu
 ID      OWNER            SUBMITTED    CPU_USAGE ST PRI SIZE CMD               
 125.0   jbasney         4/10 15:35   0+00:00:00 I  -10 1.2  hello.remote      

1 jobs; 1 idle, 0 running, 0 held
\end{verbatim}
\normalsize

%%%%%%%%%%%%%%%%%%%%%%%%%%%%%%%%%%%%%%%%%%%%%%%%%%%%%%%%%%%%%%%%%%%%%%
\subsection{Placing a job on hold}
%%%%%%%%%%%%%%%%%%%%%%%%%%%%%%%%%%%%%%%%%%%%%%%%%%%%%%%%%%%%%%%%%%%%%%
\index{Condor commands!condor\_hold}
\index{Condor commands!condor\_release}
\index{job!state}
A job in the queue may be placed on hold by running the command
\Condor{hold}.
A job in the hold state remains in the hold state until later released
for execution by the command \Condor{release}.

Use of the \Condor{hold} command causes a hard kill signal to be sent
to a currently running job (one in the running state).
For a standard universe job, this means that no checkpoint is
generated before the job stops running and enters the hold state.
When released, this standard universe job continues its execution
using the most recent checkpoint available.

Jobs in universes other than the standard universe that are running
when placed on hold will start over from the beginning when 
released.

The manual page for \Condor{hold}
on page~\pageref{man-condor-hold}
and the manual page for \Condor{release}
on page~\pageref{man-condor-release}
contain usage details.

%%%%%%%%%%%%%%%%%%%%%%%%%%%%%%%%%%%%%%%%%%%%%%%%%%%%%%%%%%%%%%%%%%%%%%
\subsection{\label{sec:job-prio}Changing the priority of jobs}
%%%%%%%%%%%%%%%%%%%%%%%%%%%%%%%%%%%%%%%%%%%%%%%%%%%%%%%%%%%%%%%%%%%%%%

\index{job!priority}
\index{priority!of a job}
In addition to the priorities assigned to each user, Condor also provides
each user with the capability of assigning priorities to each submitted job.
These job priorities are local to each queue and can be any integer value, with
higher values meaning better priority.

The default priority of a job is 0, but can be changed using the \Condor{prio}
command.
\index{Condor commands!condor\_prio}
For example, to change the priority of a job to -15,
\footnotesize
\begin{verbatim}
%  condor_q raman

-- Submitter: froth.cs.wisc.edu : <128.105.73.44:33847> : froth.cs.wisc.edu
 ID      OWNER            SUBMITTED    CPU_USAGE ST PRI SIZE CMD               
 126.0   raman           4/11 15:06   0+00:00:00 I  0   0.3  hello             

1 jobs; 1 idle, 0 running, 0 held

%  condor_prio -p -15 126.0

%  condor_q raman

-- Submitter: froth.cs.wisc.edu : <128.105.73.44:33847> : froth.cs.wisc.edu
 ID      OWNER            SUBMITTED    CPU_USAGE ST PRI SIZE CMD               
 126.0   raman           4/11 15:06   0+00:00:00 I  -15 0.3  hello             

1 jobs; 1 idle, 0 running, 0 held
\end{verbatim}
\normalsize

It is important to note that these \emph{job} priorities are completely 
different from the \emph{user} priorities assigned by Condor.  Job priorities
do not impact user priorities.  They are only a mechanism for the user to
identify the relative importance of jobs among all the jobs submitted by the
user to that specific queue.

\subsection{Why does the job not run?}
\index{job!analysis}
\index{job!not running}
Users sometimes find that their jobs do not run.  There are several reasons why
a specific job does not run.  These reasons include failed job or machine
constraints, bias due to preferences, insufficient priority, and the preemption
throttle that is implemented by the \Condor{negotiator} to prevent
thrashing.  Many of these reasons can be diagnosed by using the \Arg{-analyze}
option of \Condor{q}.
\index{Condor commands!condor\_q}
For example, a job (assigned the cluster.process value of
331228.2359) submitted to the local pool at UW-Madison
is not running.
Running \Condor{q}'s analyzer provided the following information:
\footnotesize
\begin{verbatim}
% condor_q -pool condor -name beak -analyze 331228.2359


-- Schedd: beak.cs.wisc.edu : <128.105.146.14:30918>
 ID      OWNER            SUBMITTED     RUN_TIME ST PRI SIZE CMD
 ---
331228.2359:  Run analysis summary.  Of 819 machines,
    159 are rejected by your job's requirements
    137 reject your job because of their own requirements
    488 match but are serving users with a better priority in the pool
     11 match but reject the job for unknown reasons
     24 match but will not currently preempt their existing job
      0 are available to run your job
\end{verbatim}
\normalsize

A second example shows a job that does not run because the job
does not have a high enough priority to cause other running jobs
to be preempted.
\footnotesize
\begin{verbatim}
% condor_q -pool condor -name beak -analyze 207525.0


-- Schedd: beak.cs.wisc.edu : <128.105.146.14:30918>
 ID      OWNER            SUBMITTED     RUN_TIME ST PRI SIZE CMD
 ---
 207525.000:  Run analysis summary.  Of 818 machines,
     317 are rejected by your job's requirements
     419 reject your job because of their own requirements
      79 match but are serving users with a better priority in the pool
       3 match but reject the job for unknown reasons
       0 match but will not currently preempt their existing job
       0 are available to run your job
\end{verbatim}
\normalsize

While the analyzer can diagnose most common problems, there are some situations
that it cannot reliably detect due to the instantaneous and local nature of the
information it uses to detect the problem.  Thus, it may be that the analyzer
reports that resources are available to service the request, but the job still 
does not run.  In most of these situations, the delay is transient, and the
job will run during the next negotiation cycle.

If the problem persists and the analyzer is unable to detect the situation, it
may be that the job begins to run but immediately terminates due to some 
problem.  Viewing the job's error and log files
(specified in the submit command file) and Condor's \Macro{SHADOW\_LOG} file
may assist in tracking down the problem.  If the cause is still unclear, please
contact your system administrator.

%%%%%%%%%%%%%%%%%%%%%%%%%%%%%%%%%%%%%%%%%%%%%%%%%%%%%%%%%%%%%%%%%%%%%%
\subsection{\label{sec:job-log-events}In the log file}
%%%%%%%%%%%%%%%%%%%%%%%%%%%%%%%%%%%%%%%%%%%%%%%%%%%%%%%%%%%%%%%%%%%%%%
\index{job!log events}
\index{log files!event descriptions}
In a job's log file are a log of events (a listing of events in
chronological order) that occurred during the life of the job.
The formatting of the events is always the same, 
so that they may be machine readable.
Four fields are always present,
and they will most often be followed by other fields that give further
information that is specific to the type of event.

The first field in an event is the numeric value assigned as the
event type in a 3-digit format.
The second field identifies the job which generated the event. 
Within parentheses are the ClassAd job attributes of
\AdAttr{ClusterId} value, 
\AdAttr{ProcId} value, 
and the MPI-specific rank for MPI universe jobs or a set of zeros
(for jobs run under universes other than MPI),
separated by periods.
The third field is the date and time of the event logging.  
The fourth field is a string that briefly describes the event.
Fields that follow the fourth field give further information for the specific
event type.

These are all of the events that can show up in a job log file:

\noindent\Bold{Event Number:} 000 \\
\Bold{Event Name:} Job submitted \\
\Bold{Event Description:} This event occurs when a user submits a job.
It is the first event you will see for a job, and it should only occur
once. 

\noindent\Bold{Event Number:} 001 \\
\Bold{Event Name:} Job executing \\
\Bold{Event Description:} This shows up when a job is running.
It might occur more than once.

\noindent\Bold{Event Number:} 002 \\
\Bold{Event Name:} Error in executable \\
\Bold{Event Description:} The job couldn't be run because the
executable was bad.

\noindent\Bold{Event Number:} 003 \\
\Bold{Event Name:} Job was checkpointed \\
\Bold{Event Description:} The job's complete state was written to a checkpoint
file.  
This might happen without the job being removed from a machine,
because the checkpointing can happen periodically. 

\noindent\Bold{Event Number:} 004 \\
\Bold{Event Name:} Job evicted from machine \\
\Bold{Event Description:} A job was removed from a machine before it finished,
usually for a policy reason: perhaps an interactive user has claimed
the computer, or perhaps another job is higher priority.

\noindent\Bold{Event Number:} 005 \\
\Bold{Event Name:} Job terminated \\
\Bold{Event Description:} The job has completed.

\noindent\Bold{Event Number:} 006 \\
\Bold{Event Name:} Image size of job updated \\
\Bold{Event Description:} This is informational. 
It is referring to the memory that the job is using while running. It
does not reflect the state of the job.

\noindent\Bold{Event Number:} 007 \\
\Bold{Event Name:} Shadow exception \\
\Bold{Event Description:} 
The \Condor{shadow}, a program on the submit computer that watches
over the job and performs some services for the job, failed for some
catastrophic reason. The job will leave the machine and go back into
the queue.

\noindent\Bold{Event Number:} 008 \\
\Bold{Event Name:} Generic log event \\
\Bold{Event Description:} Not used.

\noindent\Bold{Event Number:} 009 \\
\Bold{Event Name:} Job aborted \\
\Bold{Event Description:} The user cancelled the job.

\noindent\Bold{Event Number:} 010 \\
\Bold{Event Name:} Job was suspended \\
\Bold{Event Description:} The job is still on the computer, but it is no longer
executing. 
This is usually for a policy reason, like an interactive user using
the computer. 

\noindent\Bold{Event Number:} 011 \\
\Bold{Event Name:} Job was unsuspended \\
\Bold{Event Description:} The job has resumed execution, after being
suspended earlier. 

\noindent\Bold{Event Number:} 012 \\
\Bold{Event Name:} Job was held \\
\Bold{Event Description:} The user has paused the job, perhaps with
the \Condor{hold} command.
It was stopped, and will go back into the queue again until it is
aborted or released. 

\noindent\Bold{Event Number:} 013 \\
\Bold{Event Name:} Job was released \\
\Bold{Event Description:} The user is requesting that a job on hold be re-run.

\noindent\Bold{Event Number:} 014 \\
\Bold{Event Name:} Parallel node executed \\
\Bold{Event Description:} A parallel (MPI) program is running on a node.

\noindent\Bold{Event Number:} 015 \\
\Bold{Event Name:} Parallel node terminated \\
\Bold{Event Description:} A parallel (MPI) program has completed on a node.

\noindent\Bold{Event Number:} 016 \\
\Bold{Event Name:} POST script terminated \\
\Bold{Event Description:} A node in a DAGMan workflow has a script
that should be run after a job. 
The script is run on the submit host. 
This event signals that the post script has completed.

\noindent\Bold{Event Number:} 017 \\
\Bold{Event Name:} Job submitted to Globus \\
\Bold{Event Description:} A grid job has been delegated to Globus
(version 2, 3, or 4).

\noindent\Bold{Event Number:} 018 \\
\Bold{Event Name:} Globus submit failed \\
\Bold{Event Description:} The attempt to delegate a job to Globus
failed. 

\noindent\Bold{Event Number:} 019 \\
\Bold{Event Name:} Globus resource up \\
\Bold{Event Description:} The Globus resource that a job wants to run
on was unavailable, but is now available.

\noindent\Bold{Event Number:} 020 \\
\Bold{Event Name:} Detected Down Globus Resource \\
\Bold{Event Description:} The Globus resource that a job wants to run
on has become unavailable. 

\noindent\Bold{Event Number:} 021 \\
\Bold{Event Name:} Remote error \\
\Bold{Event Description:} The \Condor{starter} (which monitors the job
on the execution machine) has failed.

\noindent\Bold{Event Number:} 022 \\
\Bold{Event Name:} Remote system call socket lost \\
\Bold{Event Description:} The \Condor{shadow} and \Condor{starter}
(which communicate while the job runs) have lost contact.

\noindent\Bold{Event Number:} 023 \\
\Bold{Event Name:} Remote system call socket reestablished \\
\Bold{Event Description:} The \Condor{shadow} and \Condor{starter}
(which communicate while the job runs) have been able to resume
contact before the job lease expired.

\noindent\Bold{Event Number:} 024 \\
\Bold{Event Name:} Remote system call reconnect failure \\
\Bold{Event Description:} The \Condor{shadow} and \Condor{starter}
(which communicate while the job runs) were unable to resume
contact before the job lease expired.

\noindent\Bold{Event Number:} 025 \\
\Bold{Event Name:} Grid Resource Back Up \\
\Bold{Event Description:} A grid resource that was previously
unavailable is now available.

\noindent\Bold{Event Number:} 026 \\
\Bold{Event Name:} Detected Down Grid Resource \\
\Bold{Event Description:} The grid resource that a job is to
run on is unavailable.

\noindent\Bold{Event Number:} 027 \\
\Bold{Event Name:} Job submitted to grid resource \\
\Bold{Event Description:} A job has been submitted,
and is under the auspices of the grid resource.

%%%%%%%%%%%%%%%%%%%%%%%%%%%%%%%%%%%%%%%%%%%%%%%%%%%%%%%%%%%%%%%%%%%%%e
\subsection{\label{sec:job-completion}Job Completion}
%%%%%%%%%%%%%%%%%%%%%%%%%%%%%%%%%%%%%%%%%%%%%%%%%%%%%%%%%%%%%%%%%%%%%%
\index{job!completion}

When your Condor job completes(either through normal means or abnormal
termination by signal), Condor will remove it from the job queue (i.e.,
it will no longer appear in the output of \Condor{q}) and insert it into
the job history file.  You can examine the job history file with the
\Condor{history} command. If you specified a log file in your submit
description file, then the job exit status will be recorded there as well.

By default, Condor will send you an email message
when your job completes.  You can modify this behavior with the
\Condor{submit} ``notification'' command.
The message will include the exit status of your job (i.e., the
argument your job passed to the exit system call when it completed) or
notification that your job was killed by a signal.  It will also
include the following statistics (as appropriate) about your job:

\begin{description}

\item[Submitted at:] when the job was submitted with \Condor{submit}

\item[Completed at:] when the job completed

\item[Real Time:] elapsed time between when the job was submitted and
when it completed (days hours:minutes:seconds)

\item[Run Time:] total time the job was running (i.e., real time minus
queuing time)

\item[Committed Time:] total run time that contributed to job
completion (i.e., run time minus the run time that was lost because
the job was evicted without performing a checkpoint)

\item[Remote User Time:] total amount of committed time the job spent
executing in user mode

\item[Remote System Time:] total amount of committed time the job spent
executing in system mode 

\item[Total Remote Time:] total committed CPU time for the job

\item[Local User Time:] total amount of time this job's
\Condor{shadow} (remote system call server) spent executing in user
mode

\item[Local System Time:] total amount of time this job's
\Condor{shadow} spent executing in system mode

\item[Total Local Time:] total CPU usage for this job's \Condor{shadow}

\item[Leveraging Factor:] the ratio of total remote time to total
system time (a factor below 1.0 indicates that the job ran
inefficiently, spending more CPU time performing remote system calls
than actually executing on the remote machine)

\item[Virtual Image Size:] memory size of the job, computed when the
job checkpoints

\item[Checkpoints written:] number of successful checkpoints performed
by the job

\item[Checkpoint restarts:] number of times the job successfully
restarted from a checkpoint

\item[Network:] total network usage by the job for checkpointing and
remote system calls

\item[Buffer Configuration:] configuration of remote system call I/O
buffers

\item[Total I/O:] total file I/O detected by the remote system call
library

\item[I/O by File:] I/O statistics per file produced by the remote
system call library

\item[Remote System Calls:] listing of all remote system calls
performed (both Condor-specific and Unix system calls) with a count of
the number of times each was performed

\end{description}
