%%%%%%%%%%%%%%%%%%%%%%%%%%%%%%%%%%%%%%%%%%%%%%%%%%%%%%%%%%%%%%%%%%%%%%
\section{\label{sec:MPI}Running MPICH jobs in Condor}
%%%%%%%%%%%%%%%%%%%%%%%%%%%%%%%%%%%%%%%%%%%%%%%%%%%%%%%%%%%%%%%%%%%%%%
In addition to PVM, Condor also supports the execution of parallel jobs
that utilize MPI.
Our current implementation supports the following features:
\begin{itemize}
\item There are no alterations to the MPICH implementation.  You can directly
use the version from Argonne National Labs.

\item You do not have to re-compile or re-link your MPICH job.  Just
compile it using the regular \Prog{mpicc}.  Note that you have to be using
the ch\_p4 subsystem provided by Argonne.

\item The communication speed of the MPI nodes is not affected by 
running it under Condor.
\end{itemize}
However, there are some limitations to our current implementation.

\subsection{\label{sec:MPI-caveats}Caveats}
\begin{description}
\item[MPICH] Your MPI job must be compiled with MPICH, Argonne National
Labs' implementation of MPI.  Specifically, you must use the ``ch\_p4'' 
device for MPICH.  For information on MPICH, see Argonne's web page
at \Url{http://www-unix.mcs.anl.gov/mpi/mpich/}. Your version of MPICH must
not be compiled with the path to RSH hard-coded into the library 
(As a result of running configure as \verb ./configure -rsh=/path/to/your/rsh
possilbly.) Condor provides a special version of rsh that it uses to start 
jobs. 

\item[Dedicated Resources] You must make sure that your MPICH jobs
will be running on machines that will not vacate the job before the job
terminates naturally.  (This is a limitation of MPICH and the MPI 
specification.) Unlike PVM (Section~\ref{sec:PVM}), the current MPICH
implementation does not support dynamic resource management.  That is, 
processes in the virtual machine may NOT join or leave the computation at 
any time.  If you start an MPI job with 4 nodes, for example, none of those 4 
nodes can be preempted by other Condor jobs or the machine's owner.

\item[Scheduling] We do not yet have a sophisticated scheduling 
algorithm in place for MPI jobs.  If you set things up properly, 
there shouldn't be much of a problem.  However, if there are several
users trying to run MPI jobs on the same machines, it may be the case 
that no jobs will run at all and Condor's scheduling will deadlock.  
Writing a good scheduler for this environment is high on the priority 
list for Condor version 6.5.

\item[``New'' shadow and starter] We have been developing new versions
of the \Condor{shadow} and the \Condor{starter}.  You have to use these
new versions to run MPI jobs.  For information on obtaining these 
binaries, see below.

\item[Shared File System] The machines where you want your MPI job
to run must have a shared file system.  There is no remote I/O for
our MPI support like there is for our Standard Universe jobs.

\item[Condor Version 6.1.15+] You must be running this version of 
the Condor distribution (or greater) in order to use this contrib 
module.
\end{description}


%%%%%%%%%%%%%%%%%%%%%%%%%%%%%%%%%%%%%%%%%%%%%%%%%%%%%%%%%%%%%%%%%%%
\subsection{\label{sec:MPI-binaries}Getting the Binaries}

There is now an MPI ``contrib'' module available with Condor.  It can
be found in the contrib section of the downloads.  When you un-tar the 
tarfile, there will be three files:

\begin{itemize}
\item\Condor{starter.v61}
\item\Condor{shadow.v61}
\item\Prog{rsh}
\end{itemize}

The last item is named \File{rsh}, but it is not the \Prog{rsh} utility you're 
familiar with --- it's a wrapper that is required for our implementation to 
function correctly. 
These three binaries should go in Condor's \File{sbin} directory, where
many other files like them reside.

%%%%%%%%%%%%%%%%%%%%%%%%%%%%%%%%%%%%%%%%%%%%%%%%%%%%%%%%%%%%%%%%%%%
\subsection{\label{sec:MPI-config}Configuring Condor }

Now that you've got the necessary binaries, you'll have to configure Condor
to use MPI.  Insert the following lines in the main \condor{config} file:
\begin{verbatim}
ALTERNATE_STARTER_2	= $(SBIN)/condor_starter.v61
STARTER_2_IS_DC		= TRUE
MPI_CONDOR_RSH_PATH	= $(SBIN)
SHADOW_MPI		= $(SBIN)/condor_shadow.v61
\end{verbatim}
Reconfigure your pool by typing
\begin{verbatim}
condor_reconfig `condor_status -m`
\end{verbatim}
The -m argument tells \Condor{status} to return just the names of all
the running \Condor{master} daemons in your pool.  Note that you have to do 
this from a machine with administrator privileges.

%%%%%%%%%%%%%%%%%%%%%%%%%%%%%%%%%%%%%%%%%%%%%%%%%%%%%%%%%%%%%%%%%%%
\subsection{\label{sec:MPI-machines}Managing Dedicated Machines}

There are several ways that you can set up a pool to run MPI jobs without 
interruption.  We will cover two methods that will work, although more
sophisticated solutions are possible. 
Familiarity with Startd policy configuration
(Section~\ref{sec:Configuring-Policy}) is necessary to understand the
following examples.

For the first example, let's assume that you have a cluster of machines which
do not have regular users on them.  Let's also assume that these machines are 
solely dedicated to the use of Condor.  
The simplest way to set up your policy is as follows:
\begin{verbatim}
START       = TRUE
CONTINUE    = TRUE
SUSPEND     = FALSE
PREEMPT     = FALSE
KILL        = FALSE
\end{verbatim}

With the above configuration, the machines will accept any Condor job, and the
jobs will never be suspended, preempted, or killed.  You will never have to 
worry about an MPI job (or any job, for that matter) being evicted from the 
machines.

For a more complex example, let us assume you have machines with sophisticated 
policies already in place, and you'd like the machines to manage MPI jobs 
differently.  The following macros (which should be specified
near other Startd policy support macros) allow you to accomplish the task 
easily.
\begin{verbatim}
MPI	  = 8
IsMPI = (JobUniverse == $(MPI))
\end{verbatim}
Now change your configuration from
\begin{verbatim}
START	= /* your interesting policy here */
\end{verbatim}
to 
\begin{verbatim}
FORMER_START	= /* your interesting policy here */
\end{verbatim}
Similarly, the \Macro{CONTINUE}, \Macro{SUSPEND}, \Macro{PREEMPT}, and 
\Macro{KILL} expressions should be changed to macros named 
\MacroNI{FORMER\_CONTINUE}, etc.  The following configuration will ensure that
MPI jobs are never suspended or evicted while implementing your former policy
for all other jobs.
\begin{verbatim}
START		= ( $(FORMER_START) )
CONTINUE	= ( $(FORMER_CONTINUE) )
SUSPEND		= ( $(FORMER_SUSPEND) && ((IsMPI) == FALSE ) )
PREEMPT		= ( $(FORMER_PREEMPT) && ((IsMPI) == FALSE ) )
KILL		= ( $(FORMER_KILL) && ((IsMPI) == FALSE ) )
\end{verbatim}
Thus, Condor will never attempt to vacate an MPI job from a machine once 
it starts running on that machine.  
Some machine owners may not like this setup, so you may need to customize 
your configuration to suit your needs.  
The most important point to remember when creating your Startd policy is 
that MPI jobs are immediately killed if one or more nodes of the job leave
the computation.

%%%%%%%%%%%%%%%%%%%%%%%%%%%%%%%%%%%%%%%%%%%%%%%%%%%%%%%%%%%%%%%%%%%
\subsection{\label{sec:MPI-submit}Submitting to Condor}

Here is a minimal submit file to submit an MPI job to Condor.  For more 
information on writing submit files, see Section~\ref{sec:sample-submit-files}.

\begin{verbatim}
universe = MPI
executable = your_mpi_program
machine_count = 4
queue 
\end{verbatim}

This tells Condor to start the executable named \File{your\_mpi\_program}
on four machines.  These four machines will be of the same architechture
and operating system as the submitting machine.  Note the 
\verb+universe = MPI+ line tells Condor that an MPICH job is being submitted.  

Now let's try a more sophisticated submit file:
\begin{verbatim}
###################################################################
## submitfile                                                    ##
###################################################################
universe = MPI
executable = simplempi
log = logfile
input = infile.$(NODE)
output = outfile.$(NODE)
error = errfile.$(NODE)
machine_count = 4
queue
\end{verbatim}

Notice the \MacroU{NODE} macro, which is expanded when the job starts so that
it becomes equivalent to the MPI ``id'' of the MPICH job.  The first 
process started becomes ``0'', the second is ``1'', etc.  For example, 
let's say I prepared four input files, named \File{infile.0} through 
\File{infile.3}:
\begin{verbatim}
infile.0: 
Hello number zero.

infile.1: 
Hello number one.
\end{verbatim}
etc.  I then created a simple MPI job, named \File{simplempi.c}
\begin{verbatim}
/******************************************************************
 * simplempi.c
 ******************************************************************/
#include <stdio.h>
#include "mpi.h"

int main(argc,argv)
    int argc;
    char *argv[];
{
    int myid;
    char line[128];

    MPI_Init(&argc,&argv);
    MPI_Comm_rank(MPI_COMM_WORLD,&myid);

    fprintf ( stdout, "Printing to stdout...%d\n", myid );
    fprintf ( stderr, "Printing to stderr...%d\n", myid );
    fgets ( line, 128, stdin );
    fprintf ( stdout, "From stdin: %s", line );

    MPI_Finalize();
    return 0;
}
\end{verbatim}
And to complete the demonstration, here's the \File{Makefile}:
\begin{verbatim}
###################################################################
## This is a very basic Makefile                                 ##
###################################################################

# Change this part to your mpicc, obviously....
CC          = /usr/local/bin/mpicc
CLINKER     = $(CC)

CFLAGS    = -g
EXECS     = simplempi

all: $(EXECS)

simplempi: simplempi.o
        $(CLINKER) -o simplempi simplempi.o -lm

.c.o:
        $(CC) $(CFLAGS) -c $*.c
\end{verbatim}

Once \File{simplempi} is built, use \Condor{submit} to submit your job.
This job should finish pretty quickly once it finds machines to run on,
and the results will be what you expect:  8 files will be created:  
\File{errfile.[0-3]} and \File{outfile.[0-3]}.  For example, \File{outfile.0}
will contain
\begin{verbatim}
Printing to stdout...0
From stdin: Hello number zero.
\end{verbatim}
and \File{errfile.0} will contain
\begin{verbatim}
Printing to stderr...0
\end{verbatim}

Of course, individual tasks may open other files; this example was 
constructed to demonstrate the \MacroUNI{NODE} feature and the setup of
the expected \File{stdin}, \File{stdout}, and \File{stderr} files in the MPI
universe.  
