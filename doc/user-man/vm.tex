%%%%%%%%%%%%%%%%%%%%%%%%%%%%%%%%%%%%%%%
\section{\label{sec:vmuniverse}Virtual Machine Applications}
%%%%%%%%%%%%%%%%%%%%%%%%%%%%%%%%%%%%%%%
\index{virtual machine universe|(}
\index{universe!vm}
\index{vm universe}

The \SubmitCmd{vm} universe facilitates an HTCondor job
that matches and then lands a disk image on an execute machine
within an HTCondor pool.
This disk image is intended to be a virtual machine.
In this manner, the virtual machine is the job to be executed.

This section describes this type of HTCondor job.
See section~\ref{sec:Config-VMs}
for details of configuration variables.

%%%%%%%%%%%%%%%%%%%%%%%%%%%%%%%%%%%%%%%
\subsection{\label{sec:vm-submitfile}The Submit Description File}
%%%%%%%%%%%%%%%%%%%%%%%%%%%%%%%%%%%%%%%

Different than all other universe jobs,
the \SubmitCmd{vm} universe job specifies a disk image,
not an executable.
Therefore, the submit commands \SubmitCmd{input}, \SubmitCmd{output},
and \SubmitCmd{error} do not apply.
If specified, \Condor{submit} rejects the job with an error.
The \SubmitCmd{executable} command changes definition within a
\SubmitCmd{vm} universe job.
It no longer specifies an executable file, but instead
provides a string that identifies the job for tools such
as \Condor{q}.
Other commands specific to the type of virtual machine software
identify the disk image.

VMware, Xen, and KVM virtual machine software are supported.
As these differ from each other, the submit description file
specifies one of
\begin{verbatim}
  vm_type = vmware
\end{verbatim}
or
\begin{verbatim}
  vm_type = xen
\end{verbatim}
or
\begin{verbatim}
  vm_type = kvm
\end{verbatim}

The job is required to specify its memory needs 
for the disk image with \SubmitCmd{vm\_memory},
which is given in Mbytes.
HTCondor uses this number to assure a match with a machine
that can provide the needed memory space.

Virtual machine networking is enabled with the command
\begin{verbatim}
  vm_networking = true
\end{verbatim}
And, when networking is enabled, a definition of
\SubmitCmd{vm\_networking\_type} as \SubmitCmd{bridge}
matches the job only with a machine that is configured to use
bridge networking.
A definition of
\SubmitCmd{vm\_networking\_type} as \SubmitCmd{nat}
matches the job only with a machine that is configured to use
NAT networking.
When no definition of
\SubmitCmd{vm\_networking\_type} is given,
HTCondor may
match the job with a machine that enables networking,
and further, the choice of bridge or NAT networking
is determined by the machine's configuration.

Modified disk images are transferred back to the machine from which
the job was submitted as the \SubmitCmd{vm} universe job completes.
Job completion for a \SubmitCmd{vm} universe job occurs when 
the virtual machine is shut down, and HTCondor notices 
(as the result of a periodic check on the state of the virtual machine).
Should the job not want any files transferred back (modified or not),
for example because the job explicitly transferred its own files,
the submit command to prevent the transfer is
\begin{verbatim}
  vm_no_output_vm = true
\end{verbatim}

The required disk image must be identified for a virtual machine.
This \SubmitCmd{vm\_disk} command specifies a list of comma-separated files.
Each disk file is specified by colon-separated fields.
The first field is the path and file name of the disk file.
The second field specifies the device.
The third field specifies permissions, and the optional 
fourth specifies the format.
Here is an example that identifies a single file:
\footnotesize
\begin{verbatim}
  vm_disk = /var/lib/libvirt/images/swap.img:sda2:w:raw
\end{verbatim}
\normalsize

Setting values in the submit description file for some commands
have consequences for the virtual machine description file.
These commands are
\begin{itemize}
  \item \SubmitCmd{vm\_memory}
  \item \SubmitCmd{vm\_macaddr}
  \item \SubmitCmd{vm\_networking}
  \item \SubmitCmd{vm\_networking\_type}
  \item \SubmitCmd{vm\_disk}
\end{itemize}
For VMware virtual machines,
setting values for these commands causes HTCondor to modify the
\File{.vmx} file, overwriting existing values.
For KVM and Xen virtual machines,
HTCondor uses these values when it produces the description file.

For Xen and KVM jobs, if any files need to be transferred from the submit machine
to the machine where the \SubmitCmd{vm} universe job will execute, 
HTCondor must be explicitly told to do so with the 
standard file transfer attributes:
\footnotesize
\begin{verbatim}
  should_transfer_files = YES
  when_to_transfer_output = ON_EXIT
  transfer_input_files = /myxen/diskfile.img,/myxen/swap.img
\end{verbatim}
\normalsize
Any and all needed files on a system without a shared file
system (between the submit machine and the machine where the
job will execute) must be listed.

Further commands specify information that is specific to the
virtual machine type targeted.

%%%%%%%%%%%%%%%%%%%%%%%%%%%%%%%%%%%%%%%
\subsubsection{\label{sec:vm-VMwaresubmitfile}VMware-Specific Submit Commands}
%%%%%%%%%%%%%%%%%%%%%%%%%%%%%%%%%%%%%%%
\index{vm universe!submit commands specific to VMware}

Specific to VMware, the submit description file command
\SubmitCmd{vmware\_dir} gives the path and directory
(on the machine from which the job is submitted)
to where VMware-specific files and applications reside.
One example of a VMware-specific application is the VMDK files,
which form a virtual hard drive (disk image) for the virtual machine.
VMX files containing the primary configuration for the virtual
machine would also be in this directory.

HTCondor must be told whether or not the contents of the \SubmitCmd{vmware\_dir}
directory must be transferred to the machine where the job is
to be executed.
This required information is given with the submit command
\SubmitCmd{vmware\_should\_transfer\_files}.
With a value of \Expr{True},
HTCondor does transfer the contents of the directory.
With a value of \Expr{False},
HTCondor does not transfer the contents of the directory,
and instead presumes that access to this directory is
available through a shared file system.

By default, HTCondor uses a snapshot disk for new and modified files.
They may also be utilized for checkpoints.
The snapshot disk is initially quite small,
growing only as new files are created or files are modified.
When \SubmitCmd{vmware\_should\_transfer\_files} is \Expr{True},
a job may specify that a snapshot disk is \emph{not} to be
used with the command
\begin{verbatim}
  vmware_snapshot_disk = False
\end{verbatim}
In this case, HTCondor will utilize original disk files in producing
checkpoints. 
Note that \Condor{submit} issues an error message and does not
submit the job if both \SubmitCmd{vmware\_should\_transfer\_files}
and \SubmitCmd{vmware\_snapshot\_disk} are \Expr{False}.

Because \Prog{VMware Player} does not support snapshots, 
machines using \Prog{VMware Player} may only run \SubmitCmd{vm} jobs
that set \SubmitCmd{vmware\_snapshot\_disk} to \Expr{False}.
These jobs will also set
\SubmitCmd{vmware\_should\_transfer\_files} to \Expr{True}.
A job using \Prog{VMware Player} will go on hold if it attempts
to use a snapshot.
The pool administrator should have configured the pool
such that machines will not start jobs they can not run.

Note that if snapshot disks are requested and file transfer is not
being used, the \SubmitCmd{vmware\_dir} setting given in 
the submit description file
should not contain any symbolic link path components,
as described on the
\URL{https://htcondor-wiki.cs.wisc.edu/index.cgi/wiki?p=HowToAdminRecipes}
page under the answer to why VMware jobs with symbolic links fail.

Here is a sample submit description file for a VMware virtual machine:
\begin{verbatim}
universe                     = vm
executable                   = vmware_sample_job
log                          = simple.vm.log.txt
vm_type                      = vmware
vm_memory                    = 64
vmware_dir                   = C:\condor-test
vmware_should_transfer_files = True
queue
\end{verbatim}
This sample uses the \SubmitCmd{vmware\_dir} command to identify
the location of the disk image to be executed as an HTCondor job.
The contents of this directory are transferred to the machine assigned
to execute the HTCondor job.

%%%%%%%%%%%%%%%%%%%%%%%%%%%%%%%%%%%%%%%
\subsubsection{\label{sec:vm-Xensubmitfile}Xen-Specific Submit Commands}
%%%%%%%%%%%%%%%%%%%%%%%%%%%%%%%%%%%%%%%
\index{vm universe!submit commands specific to Xen}

% xen_kernel description
A Xen \SubmitCmd{vm} universe job requires specification of the
guest kernel. 
The \SubmitCmd{xen\_kernel} command accomplishes this, 
utilizing one of the following definitions.
\begin{enumerate}
\item \SubmitCmd{xen\_kernel = included} implies that the kernel
  is to be found in disk image given by the definition of the single file
  specified in \SubmitCmd{vm\_disk}. 

\item \SubmitCmd{xen\_kernel = path-to-kernel} gives a full path and
  file name of the required kernel.  If this kernel must be transferred
  to machine on which the \SubmitCmd{vm} universe job will execute,
  it must also be included in the \SubmitCmd{xen\_transfer\_files} command. 

  This form of the \SubmitCmd{xen\_kernel} command also requires further
  definition of the \SubmitCmd{xen\_root} command.
  \SubmitCmd{xen\_root} defines the device containing files needed by
  \Login{root}.

\end{enumerate}

%%%%%%%%%%%%%%%%%%%%%%%%%%%%%%%%%%%%%%%
\subsection{\label{sec:vm-checkpoints}Checkpoints}
%%%%%%%%%%%%%%%%%%%%%%%%%%%%%%%%%%%%%%%
\index{vm universe!checkpoints}

Creating a checkpoint is straightforward for a virtual machine,
as a checkpoint is a set of files that represent
a snapshot of both disk image and memory.
The checkpoint is created and all files are transferred back
to the \MacroUNI{SPOOL} directory on the machine from which
the job was submitted.
The submit command to create checkpoints is
\begin{verbatim}
  vm_checkpoint = true
\end{verbatim}
Without this command, no checkpoints are created (by default).
With the command, a checkpoint is created any time the \SubmitCmd{vm}
universe jobs is evicted from the machine upon which it is executing.
This occurs as a result of the machine configuration indicating
that it will no longer execute this job.

\SubmitCmd{vm} universe jobs can \emph{not} use a checkpoint server.

Periodic creation of checkpoints is not supported at this time.

Enabling both networking and checkpointing for a \SubmitCmd{vm}
universe job can cause networking problems when the job restarts,
particularly if the job migrates to a different machine.
\Condor{submit} will normally reject such jobs.
To enable both, then add the command
\begin{verbatim}
  when_to_transfer_output = ON_EXIT_OR_EVICT
\end{verbatim}

Take care with respect to the use of network connections within
the virtual machine and their interaction with checkpoints.
Open network connections at the time of the checkpoint will likely
be lost when the checkpoint is subsequently used to resume execution
of the virtual machine.
This occurs whether or not the execution resumes
on the same machine or a different one within the HTCondor pool.   

%%%%%%%%%%%%%%%%%%%%%%%%%%%%%%%%%%%%%%%
\subsection{\label{sec:vm-disk-image-details}Disk Images}
%%%%%%%%%%%%%%%%%%%%%%%%%%%%%%%%%%%%%%%

%%%%%%%%%%%%%%%%%%%%%%%%%%%%%%%%%%%%%%%
\subsubsection{\label{sec:vm-disk-image-details-vmware}
VMware on Windows and Linux}
%%%%%%%%%%%%%%%%%%%%%%%%%%%%%%%%%%%%%%%

Following the platform-specific
guest OS installation instructions found at
\URL{http://partnerweb.vmware.com/GOSIG/home.html},
creates a VMware disk image.

%%%%%%%%%%%%%%%%%%%%%%%%%%%%%%%%%%%%%%%
\subsubsection{\label{sec:vm-disk-image-details-xen}Xen and KVM}
%%%%%%%%%%%%%%%%%%%%%%%%%%%%%%%%%%%%%%%
While the following web page contains instructions specific to
Fedora on how to create a virtual guest image,
it should provide a good starting point for 
other platforms as well.

\URL{http://fedoraproject.org/wiki/Virtualization\_Quick\_Start}

%%%%%%%%%%%%%%%%%%%%%%%%%%%%%%%%%%%%%%%
\subsection{\label{sec:vm-job-completion-details}Job Completion in the vm Universe}
%%%%%%%%%%%%%%%%%%%%%%%%%%%%%%%%%%%%%%%

Job completion for a \SubmitCmd{vm} universe job occurs when 
the virtual machine is shut down, and HTCondor notices 
(as the result of a periodic check on the state of the virtual machine).
This is different from jobs executed under the environment of other 
universes.

Shut down of a virtual machine occurs from within the virtual
machine environment.
A script, executed with the proper authorization level,
is the likely source of the shut down commands.

Under a Windows 2000, Windows XP, or Vista virtual machine,
an administrator issues the command
\begin{verbatim}
  shutdown -s -t 01
\end{verbatim}

Under a Linux virtual machine,
the \Login{root} user executes
\begin{verbatim}
  /sbin/poweroff
\end{verbatim}
The command \verb@/sbin/halt@ will not completely
shut down some Linux distributions, and instead
causes the job to hang.

Since the successful completion of the \SubmitCmd{vm} universe job
requires the successful shut down of the virtual machine,
it is good advice to try the shut down procedure outside of
HTCondor, before a \SubmitCmd{vm} universe job is submitted.


\index{virtual machine universe|)}
