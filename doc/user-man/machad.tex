\begin{description}
%
\index{ClassAd!machine attributes}
\index{ClassAd machine attribute!Activity}
\item[\AdAttr{Activity}] : String which describes Condor job activity on the machine.
Can have one of the following values:
	\begin{description}
	\item[\AdStr{Idle}] : There is no job activity
	\item[\AdStr{Busy}] : A job is busy running
	\item[\AdStr{Suspended}] : A job is currently suspended
	\item[\AdStr{Vacating}] : A job is currently checkpointing
	\item[\AdStr{Killing}] : A job is currently being killed
	\item[\AdStr{Benchmarking}] : The startd is running benchmarks
	\end{description}
%
\index{ClassAd machine attribute!Arch}
\item[\AdAttr{Arch}] : String with the architecture of the machine.  Typically
one of the following: 
	\begin{description}
	\item[\AdStr{INTEL}] : Intel x86 CPU (Pentium, Xeon, etc).
	\item[\AdStr{IA64}] : Intel 64-bit CPU
	\item[\AdStr{ALPHA}] : Digital Alpha CPU
	\item[\AdStr{SGI}] : Silicon Graphics MIPS CPU
	\item[\AdStr{SUN4u}] : Sun UltraSparc CPU
	\item[\AdStr{SUN4x}] : A Sun Sparc CPU other than an UltraSparc, i.e.
sun4m or sun4c CPU found in older Sparc workstations such as the Sparc~10, 
Sparc~20, IPC, IPX, etc.
	\item[\AdStr{PPC}] : Power Macintosh
	\item[\AdStr{HPPA1}] :  Hewlett Packard PA-RISC 1.x CPU (i.e. PA-RISC    
                      7000 series CPU) based workstation
	\item[\AdStr{HPPA2}] :  Hewlett Packard PA-RISC 2.x CPU (i.e. PA-RISC    
                      8000 series CPU) based workstation
	\end{description}
%
\index{ClassAd machine attribute!ClockDay}
\item[\AdAttr{ClockDay}] : The day of the week, where 0 = Sunday, 1 = Monday, \Dots, 6 = Saturday. 
%
\index{ClassAd machine attribute!ClockMin}
\item[\AdAttr{ClockMin}] : The number of minutes passed since midnight.
%
\index{ClassAd machine attribute!CondorLoadAvg}
\item[\AdAttr{CondorLoadAvg}] : The portion of the load average generated by Condor (either
from remote jobs or running benchmarks).
%
\index{ClassAd machine attribute!ConsoleIdle}
\item[\AdAttr{ConsoleIdle}] : The number of seconds since activity on the system
console keyboard or console mouse has last been detected.
%
\index{ClassAd machine attribute!Cpus}
\item[\AdAttr{Cpus}] : Number of CPUs in this machine, i.e. 1 = single CPU machine, 2 = dual
CPUs, etc.
%
\index{ClassAd machine attribute!CurrentRank}
\item[\AdAttr{CurrentRank}] : A float which represents this machine owner's affinity
for running the Condor job which it is currently hosting.  If not
currently hosting a Condor job, \AdAttr{CurrentRank} is -1.0.
%
\index{ClassAd machine attribute!Disk}
\item[\AdAttr{Disk}] : The amount of disk space on this machine available for
the job in kbytes ( e.g. 23000 = 23 megabytes ).  Specifically, this
is the amount of disk space available in the directory specified in
the Condor configuration files by the \Macro{EXECUTE} macro, minus any
space reserved with the \Macro{RESERVED\_DISK} macro.
%
\index{ClassAd machine attribute!EnteredCurrentActivity}
\item[\AdAttr{EnteredCurrentActivity}] : Time at which the machine
entered the current Activity (see \AdAttr{Activity} entry above).  On
all platforms (including NT), this is measured in the number of
integer seconds since the Unix epoch (00:00:00 UTC, Jan 1, 1970).
%
\index{ClassAd machine attribute!FileSystemDomain}
\item[\AdAttr{FileSystemDomain}] : A ``domain'' name configured by the
Condor administrator which describes a cluster of machines which all
access the same, uniformly-mounted, networked file systems usually via
NFS or AFS.  This is useful for Vanilla universe jobs which require
remote file access.
%
\index{ClassAd machine attribute!KeyboardIdle}
\item[\AdAttr{KeyboardIdle}] : The number of seconds since activity on any
keyboard or mouse associated with this machine has last been detected.
Unlike \AdAttr{ConsoleIdle}, \AdAttr{KeyboardIdle} also takes activity 
on pseudo-terminals into
account (i.e. virtual ``keyboard'' activity from telnet and rlogin
sessions as well).  Note that \AdAttr{KeyboardIdle} will always be equal to or
less than \AdAttr{ConsoleIdle}.
%
\index{ClassAd machine attribute!KFlops}
\item[\AdAttr{KFlops}] : Relative floating point performance as determined via a
Linpack benchmark.
%
\index{ClassAd machine attribute!LastHeardFrom}
\item[\AdAttr{LastHeardFrom}] : Time when the Condor central manager last
received a status update from this machine.  
Expressed as 
the number of integer seconds since the Unix epoch (00:00:00 UTC, Jan 1, 1970).
Note: This attribute is only inserted by the central manager once it
receives the ClassAd.
It is not present in the \Condor{startd} copy of the ClassAd.
Therefore, you could not use this attribute in defining \Condor{startd}
expressions (and you would not want to).
%
\index{ClassAd machine attribute!LoadAvg}
\item[\AdAttr{LoadAvg}] : A floating point number with the machine's current load
average.
%
\index{ClassAd machine attribute!Machine}
\item[\AdAttr{Machine}] : A string with the machine's fully qualified hostname.
%
\index{ClassAd machine attribute!Memory}
\item[\AdAttr{Memory}] : The amount of RAM in megabytes.
%
\index{ClassAd machine attribute!Mips}
\item[\AdAttr{Mips}] : Relative integer performance as determined via a Dhrystone
benchmark.
%
\index{ClassAd machine attribute!MyType}
\item[\AdAttr{MyType}] : The ClassAd type; always set to the literal string \AdStr{Machine}.
%
\index{ClassAd machine attribute!Name}
\item[\AdAttr{Name}] : The name of this resource; typically the same value as
the \AdAttr{Machine} attribute, but could be customized by the site
administrator.
On SMP machines, the \Condor{startd} will divide the CPUs up into separate
virtual machines, each with with a unique name.
These names will be of the form ``vm\#@full.hostname'', for example,
``vm1@vulture.cs.wisc.edu'', which signifies virtual machine 1 from
vulture.cs.wisc.edu. 
%
\index{ClassAd machine attribute!OpSys}
\item[\AdAttr{OpSys}] : String describing the operating system running on this
machine.  For Condor \VersionNotice\ typically one of the following:
	\begin{description}
	\item[\AdStr{HPUX10}] : for HPUX 10.20
	\item[\AdStr{HPUX11}] : for HPUX B.11.00
	\item[\AdStr{IRIX6}] : for IRIX 6.2, 6.3, or 6.4
	\item[\AdStr{IRIX65}] : for IRIX 6.5
	\item[\AdStr{IRIX62}] : for IRIX 6.2
	\item[\AdStr{LINUX}] : for LINUX 2.0.x or LINUX 2.2.x kernel systems
	\item[\AdStr{OSF1}] : for Digital Unix 4.x
	\item[\AdStr{SOLARIS25}] : for Solaris 2.4 or 5.5
	\item[\AdStr{SOLARIS251}] : for Solaris 2.5.1 or 5.5.1
	\item[\AdStr{SOLARIS26}] : for Solaris 2.6 or 5.6
	\item[\AdStr{SOLARIS27}] : for Solaris 2.7 or 5.7
	\item[\AdStr{SOLARIS28}] : for Solaris 2.8 or 5.8
	\item[\AdStr{SOLARIS29}] : for Solaris 2.9 or 5.9
	\item[\AdStr{WINNT40}] : for Windows NT 4.0
	\item[\AdStr{OSX}] : for Darwin
	\item[\AdStr{OSX10\_2}] : for Darwin 6.4
	\end{description}
%
\index{ClassAd machine attribute!Requirements}
\item[\AdAttr{Requirements}] : A boolean, which when evaluated within the context
of the machine ClassAd and a job ClassAd, must evaluate to
TRUE before Condor will allow the job to use this machine.
%
\index{ClassAd machine attribute!StartdIpAddr}
\item[\AdAttr{StartdIpAddr}] : String with the IP and port address of the
\Condor{startd} daemon which is publishing this machine ClassAd.
%
\index{ClassAd machine attribute!State}
\item[\AdAttr{State}] : String which publishes the machine's Condor state.
Can be:
	\begin{description}
	\item[\AdStr{Owner}] : The machine owner is using the machine, and
it is unavailable to Condor.
	\item[\AdStr{Unclaimed}] : The machine is available to run Condor jobs,
but a good match is either not available or not 
yet found.
	\item[\AdStr{Matched}] : The Condor central manager has found a good
match for this resource, but a Condor scheduler has not yet claimed it.
	\item[\AdStr{Claimed}] : The machine is claimed by a remote
\Condor{schedd} and is probably running a job.
	\item[\AdStr{Preempting}] : A Condor job is being preempted (possibly
via checkpointing) in order to clear the machine for either a higher
priority job or because the machine owner wants the machine back.
	\end{description}   % of State
%
\index{ClassAd machine attribute!TargetType}
\item[\AdAttr{TargetType}] : Describes what type of ClassAd to match with.
Always set to the string literal \AdStr{Job}, because machine ClassAds
always want to be matched with jobs, and vice-versa.
%
\index{ClassAd machine attribute!UidDomain}
\item[\AdAttr{UidDomain}] : a domain name configured by the Condor 
administrator which describes a cluster of machines which all have 
the same \File{passwd} file entries, and therefore all have the same logins.
%
\index{ClassAd machine attribute!VirtualMachineID}
\item[\AdAttr{VirtualMachineID}] : For SMP machines, the integer
that identifies the VM.
The value will be \verb@X@ for the VM with 
\begin{verbatim}
name="vmX@full.hostname"
\end{verbatim}
For non-SMP machines with one virtual machine, the value will be 1.
%
\index{ClassAd machine attribute!VirtualMemory}
\item[\AdAttr{VirtualMemory}] : The amount of currently available virtual memory 
(swap space) expressed in kbytes.

\end{description}

In addition, there are a few attributes that are automatically
inserted into the machine ClassAd whenever a resource is in the
Claimed state:

\begin{description}

\index{ClassAd machine attribute (in Claimed State)!ClientMachine}
\item[\AdAttr{ClientMachine}] : The hostname of the machine that has
claimed this resource

\index{ClassAd machine attribute (in Claimed State)!CurrentRank}
\item[\AdAttr{CurrentRank}] : The value of the \Expr{RANK} expression
when evaluated against the ClassAd of the ``current'' job using this
machine.
If the resource has been claimed but no job is running, the
``current'' job ClassAd is the one that was used when claiming the
resource.
If a job is currently running, that job's ClassAd is the ``current''
one.  
If the resource is between jobs, the ClassAd of the last job that was
run is used for \AdAttr{CurrentRank}.

\index{ClassAd machine attribute (in Claimed State)!RemoteOwner}
\item[\AdAttr{RemoteOwner}] : The name of the user who originally
claimed this resource.

\index{ClassAd machine attribute (in Claimed State)!RemoteUser}
\item[\AdAttr{RemoteUser}] : The name of the user who is currently
using this resource.
In general, this will always be the same as the \AdAttr{RemoteOwner},
but in some cases, a resource can be claimed by one entity that hands
off the resource to another entity which uses it.
In that case, \AdAttr{RemoteUser} would hold the name of the entity
currently using the resource, while \AdAttr{RemoteOwner} would hold
the name of the entity that claimed the resource.

\end{description}

There are a few attributes that are only inserted into the
machine ClassAd if a job is currently executing.  
If the resource is claimed but no job are running, none of these
attributes will be defined.

\begin{description}

\index{ClassAd machine attribute (when running)!JobId}
\item[\AdAttr{JobId}] : The job's identifier (for example,
\begin{verbatim}152.3\end{verbatim}), like you would see in \Condor{q}
on the submitting machine.

\index{ClassAd machine attribute (when running)!JobStart}
\item[\AdAttr{JobStart}] : The timestamp in integer seconds of when the job began
executing, since the Unix epoch (00:00:00 UTC, Jan 1, 1970).  For idle
machines, the value is UNDEFINED.

\index{ClassAd machine attribute (when running)!LastPeriodicCheckpoint}
\item[\AdAttr{LastPeriodicCheckpoint}] : If the job has performed a
periodic checkpoint, this attribute will be defined and will hold the
timestamp of when the last periodic checkpoint was begun.
If the job has yet to perform a periodic checkpoint, or cannot
checkpoint at all, the \AdAttr{LastPeriodicCheckpoint} attribute will
not be defined.

\end{description}

Finally, the single attribute, 
\Attr{CurrentTime}, is defined by the ClassAd
environment.
\begin{description}
\index{ClassAd attribute!CurrentTime}
\item[\AdAttr{CurrentTime}] : Evaluates to the 
the number of integer seconds since the Unix epoch (00:00:00 UTC, Jan 1, 1970).
\end{description}
