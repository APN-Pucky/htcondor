\section{\label{sec:Glidein}Extending your Condor pool by Gliding to remote machines}

\Condor{glidein} is a program that can be used to add resources to a Condor 
pool on a temporary basis. During this period, these resources are visible 
to users of the pool, but only the user performing the glidein is allowed 
to use them. The machine in the Condor pool is referred to herein as the
local node, while the resource you are adding to your local Condor pool
will be referred to as the remote node.

\subsection{Globus requirements}
These requirements are general to using any Globus resource:
\begin{enumerate}

\item You must be a valid Globus user and be mapped to a valid login account by
the site's Globus administrator on every Globus node you will glide to.
See \Url{http://www.globus.org} for information.

\item You must have \Env{HOME} and either \Env{GLOBUS\_INSTALL\_PATH} or \Env{GLOBUS\_DEPLOY\_PATH}
set in your environment.

\item You must have valid Globus credentials (X509 certificate directory).

\item Testing the Globus requirements:
\begin{verbatim}
     --You can run globusrun -version to display the version number.
     --You can run globusrun -a -r $<$FQHN$>$/jobmanager-fork to authenticate 
       to the host, as well as testing the validity of your credentials 
       and environment variables.
\end{verbatim}
\end{enumerate}

\subsection{Requirements for running \Condor{glidein}}
\begin{enumerate}
\item You must meet all Globus requirements (see above), and use Globus v1.1 or higher.

\item You must be an authorized user of the local Condor pool.

\item Your Condor pool's configuration file(s) must give \Macro{HOSTALLOW\_WRITE}
  permission to every machine you want to glide to. Since wildcards
  are permitted in this value, you can glide in to every machine at
  mcs.anl.gov by adding "*.mcs.anl.gov" to the \Macro{HOSTALLOW\_WRITE} list.
  All machines in the pool must be sent a reconfigure command when
  changes are made to the \Macro{HOSTALLOW\_WRITE} list.

\item You must have \Env{X509\_USER\_PROXY} set in your environment, pointing to a
  valid user proxy.

\item You must have the common user programs (/bin), globus tools, and Condor
  tools included in your PATH.
\end{enumerate}

\subsection{Requirements for running \Condor{glidein} as a Condor job}
\begin{enumerate}
\item Files must be prestaged (once per machine, not each time) by running
\Condor{glidein} --setuponly $<$FQHN$>$ outside of Condor.

\item Use the --runonly option.

\item Run \Condor{glidein} as a vanilla universe job or as a scheduler universe 
job. Since scheduler universe jobs run on the submitting machine, some potential
PATH and \Env{X509\_USER\_PROXY} problems might be avoided by running as scheduler universe jobs (completely unrelated to the --scheduler option to \Condor{glidein}).

\item The proxy (located by \Env{X509\_USER\_PROXY} env var) must be readable by
Condor. At UW, that means that AFS acls apply, and must be readable by net:cs. 
This is a security hole and is being addressed.

\item The environment variables \Env{GLOBUS\_INSTALL\_PATH}  \Env{GLOBUS\_DEPLOY\_PATH},

\Env{PATH}, \Env{X509\_USER\_PROXY}, and \Env{HOME} must be exported using the submit description 
file options of Getenv = True, or by setting these values with the Environment 
option.

\item Remember that, unless you run \Condor{glidein} with the --runfor option,
the daemons will run on the Globus resource indefinitely. When you no longer
need the daemons, run \Condor{glidein} with the --kill option.
\end{enumerate}

Here is a sample submit description file:

\begin{verbatim}
   universe = vanilla
   Executable = /unsup/condor/bin/condor_glidein
   arguments = --runonly pitcairn.mcs.anl.gov

   environment = \
   PATH=/bin:/unsup/condor/bin:/p/condor/workspaces/globus/bin;\
   HOME=/u/m/y/myhome;\
   GLOBUS_INSTALL_PATH=/p/condor/workspaces/globus/common;\
   X509_USER_PROXY=/u/m/y/myhome/proxy/proxyfile

   Output = outglide.\$(Process)
   Error = errglide.\$(Process)

   +memoryrequirements = 20

   queue
\end{verbatim}

subsection{Usage example}
Example: a Condor user at UW has access to Globus resources at 
Argonne, and wants to add the resources to his local Condor pool.

\begin{enumerate}

\item From his local machine (e.g., monterey6@cs.wisc.edu) the user runs 
   the \Condor{glidein} script specifying the remote resource he wants 
   to acquire:

monterey6@cs.wisc.edu\% \Condor{glidein} goshen.mcs.anl.gov

\item The remote node is checked to see if the user has permission to
   use the remote node as a resource. The remote node is also checked
   to ensure the necessary Condor configuration files and executables
   are correctly located. If not, they are placed there by the glidein
   script.

\item The executables are run on the remote node and join the pool.
   After a couple of seconds you can see the remote resources in your 
   local Condor pool, e.g.:

   monterey6@cs.wisc.edu\% \Condor{status} | grep goshen
      vm1@goshen.mc SOLARIS26   SUN4u  Unclaimed  Idle       0.000   128 ...

\item \Condor{glidein} offers you several options to manage the remote resource
   Use "\Condor{glidein} --help" for details.

\item You are ready to submit job(s) to your Condor pool. If you wish to force
   a job to run on the remote node you started with \Condor{glidein}, you can
   specify the remote node as a machine requirement in your Condor submit
   file: 
\begin{verbatim}
      requirements = ( machine == "goshen.mcs.anl.gov" ) \
         && FileSystemDomain != "" \
         && Arch != "" && OpSys != ""
\end{verbatim}
   (This example requires that the job run only on goshen.mcs.anl.gov, and
   prevents Condor from inserting the filesystem domain, architecture, and 
   operating system attributes as requirements in the matchmaking process,
   since it is likely you may submit Condor jobs from a machine whose
   attributes don't match the remote node's attributes).

\end{enumerate}

\subsection{How it works}
\Condor{glidein} ensures that you have a valid proxy and that the files
necessary for performing glidein setup are available to the glidein
program. 

\Condor{glidein} then contacts the remote node and checks for the
presence of the necessary configuration files and Condor executables.
If the executables are not present for the machine architecture, 
operating system version, and Condor pool version required, a 
globus-gass-server running at UW is contacted to stage the needed
executables.

Once \Condor{glidein} determines that the files are correctly in place,
it runs a script on the remote node that runs the Condor executables.
An optional time limit can be specified, after which time the Condor 
executables gracefully shut down. If no time limit is specified, the 
Condor executables on the remote node can be shut down when no longer
needed by re-running \Condor{glidein} with the --kill option.

The Condor executables on the remote node contact the local pool and
attempt to join the pool (see Condor requirements above). The START
expression for the \Condor{startd} program requires that the username
of the person who ran the glidein script matches the username of the
jobs submitted through Condor, although this can be overridden by
editing the configuration file installed on the remote node by the
\Condor{glidein} program.

\subsection{Invocation options}
\begin{description}
\item[\Arg{--scheduler $<$name$>$}]
	Select the scheduler type to be used by Globus. Defaults to "fork".

\item[\Arg{--queue $<$name$>$}]
	Specify which queue=$<$queuename$>$ to submit to for the Globus scheduler.

\item[\Arg{--runfor $<$mins$>$}]
   How many minutes the condor daemons should run for before gracefully
   exiting.  
   Be aware that if you do this, the daemons will stay running indefinitely
   until you shut them down with the "kill" option.

\item[\Arg{--setuponly}]
   Performs only the setup of files on the remote host, but does not
   run the generated script for launching the Condor daemons.
   (Cannot be run simultaneously with --runonly)

\item[\Arg{--runonly}]
   Checks that the files are in place, but does not do the remote setup
   (with the exception of generating and placing the script to launch
   the daemons). If any of the other files are missing, exits with
   an error code.
   (Cannot be run simultaneously with --setuponly)

\item[\Arg{--kill}]
   Gracefully shut down the Condor daemons running on the remote host
   that were started by flying in. This should not interfere with other
   Condor daemons running on the host that were not started by your
   flyin session.
   See also --runfor option.

\item[\Arg{--help}]
   Prints terse description and usage, then exits.

\item[\Arg{--count $<$CPU count$>$}]
	Number of CPUs to request, default 1.

\item[\Arg{--dump		}]
	Prints the generated globusrun string to stdout.

\item[\Arg{--probe}]
	Calculate and display values for the local Condor pool.

\item[\Arg{--admin $<$address$>$}]
	Who to email with problems. Defaults to $<$your\_login$>$Condor \MacroU{UID\_DOMAIN}$>$

\item[\Arg{--pool $<$FQHN of collector$>$}]
	Specify fully-qualified hostname of the collector of your Condor pool.
	Defaults to the value for \Macro{CONDOR\_HOST} for the local Condor pool.

\item[\Arg{--condorversion $<$version$>$}]
	Specify the version of Condor running in the pool to which you are gliding.
	Defaults to the version of the local Condor pool.
\end{description}
