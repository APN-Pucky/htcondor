\section{\label{sec:Glidein}Extending your Condor pool by Gliding into Globus-controlled machines}
\index{universe!Globus}
\index{Globus}

%WHAT GLOBUS IS.
%Globus provides software that allows users access to 

\Condor{glidein} is a program that can be used to add resources from an
existing Globus resource to a Condor 
pool on a temporary basis. During this period, these resources are visible 
to users of the pool, but only the user performing the glidein is allowed 
to use them. The machine in the Condor pool is referred to herein as the
local node, while the resource you are adding to your local Condor pool
will be referred to as the remote node.

\subsection{Globus requirements}
These requirements are general to using any Globus resource:
\begin{enumerate}

\item You must be a valid Globus user and be mapped to a valid login account by
the site's Globus administrator on every Globus node you will glide to.
See \Url{http://www.globus.org} for information.

\item You must have \Env{HOME} and either \Env{GLOBUS\_INSTALL\_PATH} or 
\Env{GLOBUS\_DEPLOY\_PATH} set in your environment.

\item You must have valid Globus credentials (X509 certificate directory).

%\item Testing the Globus requirements:
%\begin{verbatim}
%--You can run globusrun -version to display the version number.
%--You can run globusrun -a -r <FQHN>/jobmanager-fork to authenticate 
%to the host, as well as testing the validity of your credentials 
%and environment variables.
%\end{verbatim}

\end{enumerate}

\subsection{Requirements for running \Condor{glidein}}
\begin{enumerate}
\item You must meet all Globus requirements (see above), and use Globus v1.1 or higher.

\item You must be an authorized user of the local Condor pool.

\item Your Condor pool's configuration file(s) must give \Macro{HOSTALLOW\_WRITE}
  permission to every machine you want to glide to. Since wildcards
  are permitted in this value, you can glide in to every machine at
  cs.wisc.edu by adding "*.cs.wisc.edu" to the \Macro{HOSTALLOW\_WRITE} list.
  All machines in the pool must be sent a reconfigure command when
  changes are made to the \Macro{HOSTALLOW\_WRITE} list.

\item Your Condor pool's configuration file(s) must set the \Macro{GLOBUSRUN}
and \Macro{SHADOW\_GLOBUS} values to be the path of the \Prog{globusrun} and
\Condor{shadow.globus}, respectively.


\item You must have the common user programs directory (/bin), globus tools, 
and Condor user program directory (.../bin) included in your PATH.

\item You must have \Env{X509\_USER\_PROXY} set in your environment, pointing to a valid user proxy.

\end{enumerate}

%\subsection{Requirements for running \Condor{glidein} as a Condor job}
%\begin{enumerate}
%\item The proxy (located by \Env{X509\_USER\_PROXY} env var) must be readable by
%Condor. At sites using an AFS filesystem, that means that AFS ACLs apply, and must be readable by the Condor programs.
%
%\item Run \Condor{glidein} as a vanilla universe job or as a scheduler universe 
%job. Since scheduler universe jobs run on the submitting machine, some potential
%PATH and \Env{X509\_USER\_PROXY} problems might be avoided by running as scheduler universe jobs (completely unrelated to the --scheduler option to \Condor{glidein}).
%
%\item The environment variables \Env{GLOBUS\_INSTALL\_PATH}  \Env{GLOBUS\_DEPLOY\_PATH}, \Env{PATH}, \Env{X509\_USER\_PROXY}, and \Env{HOME} must be exported using the submit description file options of Getenv = True, or by setting these values with the Environment option.
%
%\item Since \Condor{glidein} is a script which itself submits job(s) to Condor, 
%the \Condor{glidein} script exits normally once the submission completes on
%the remote machine. Because of this, even sucessful glideins will not show up
%when running \Condor{queue}, because \Condor{glidein} has completed, and the
%\Prog{globusrun} program is running on another machine in the pool. Do not
%worry, though-- the glided-in daemons will shut down gracefully after the
%time interval specified with the \Opt{runfor} option [default 20 minutes].
%
%Here is a sample submit description file:
%
%\begin{verbatim}
   %universe = vanilla
   %Executable = /unsup/condor/bin/condor_glidein
   %arguments = denali.mcs.anl.gov
%
   %environment = \
   %PATH=/bin:/unsup/condor/bin:/p/condor/workspaces/globus/bin;\
   %HOME=/u/m/y/myhome;\
   %GLOBUS_INSTALL_PATH=/p/condor/workspaces/globus/common;\
   %X509_USER_PROXY=/u/m/y/myhome/proxy/proxyfile
%
   %Output = outglide.\$(Process)
   %Error = errglide.\$(Process)
%
   %+memoryrequirements = 20
%
   %queue
%\end{verbatim}
%\end{enumerate}
%
\subsection{Usage example}
Example: a Condor user at UW has access to Globus resources at 
Argonne, and wants to add the resources to his local Condor pool.

\begin{enumerate}

\item From his local machine (e.g., froth@cs.wisc.edu) the user runs 
   the \Condor{glidein} script specifying the remote resource he wants 
   to acquire:

mikeu@froth\% \Condor{glidein} denali.mcs.anl.gov

\item The remote node is checked to see if the user has permission to
   use the remote node as a resource. The remote node is also checked
   to ensure the necessary Condor configuration files and executables
   are correctly located. If not, they are placed there by the glidein
   script.

\item The executables are run on the remote node and join the pool.
   After a couple of seconds you can see the remote resources in your 
   local Condor pool, e.g.:

\begin{verbatim}
mikeu@froth\% \Condor{status} | grep denal
7591386@denal IRIX65      SGI    Unclaimed  Idle       3.700  24064  0+00:06:35
\end{verbatim}

\item \Condor{glidein} offers you several options to manage the remote resource
   Use "\Condor{glidein} --help" for details.

\item You are ready to submit job(s) to your Condor pool. If you wish to force
   a job to run on the remote node you started with \Condor{glidein}, you can
   specify the remote node as a machine requirement in your Condor submit
   file: 
\begin{verbatim}
      requirements = ( machine == "denali.mcs.anl.gov" ) \
         && FileSystemDomain != "" \
         && Arch != "" && OpSys != ""
\end{verbatim}
   (This example requires that the job run only on denali.mcs.anl.gov, and
   prevents Condor from inserting the filesystem domain, architecture, and 
   operating system attributes as requirements in the matchmaking process,
   since it is likely you may submit Condor jobs from a machine whose
   attributes don't match the remote node's attributes).

\end{enumerate}

\subsection{How it works}
\Condor{glidein} ensures that you have a valid proxy and that the files
necessary for performing glidein setup are available to the glidein
program. 

\Condor{glidein} then contacts the remote node and checks for the
presence of the necessary configuration files and Condor executables.
If the executables are not present for the machine architecture, 
operating system version, and Condor pool version required, a 
server running at UW is contacted to stage the needed executables.

Once \Condor{glidein} determines that the files are correctly in place,
it runs the Condor executables by creating a submit description file for 
\Condor{submit}, which, in turn runs the \Condor{master} under the Globus 
universe.
The Condor daemons exit gracefully when no jobs run on the daemons for a 
configurable period of time. The default length of time is 20 minutes.

The Condor executables on the remote node contact the local pool and
attempt to join the pool (see Condor requirements above). The START
expression for the \Condor{startd} program requires that the username
of the person who ran the glidein script matches the username of the jobs 
submitted through Condor. This can be overridden with the 'anybody' option.

\subsection{Invocation options}
\begin{description}
\item[\Arg{--basedir $<$name$>$}]
	Specify the remote base directory for staging files, daemons, etc. 
	This defaults to \$HOME/Condor\_glidein on the remote node.

\item[\Arg{--archdir $<$name$>$}]
	Specify the remote directory for the executables, 
	For example, "6.1.9-sparc-sun-solaris-2.6" is the architecture description 
	for Condor version 6.1.9, running on a Sun Sparc machine with Solaris 2.6. 
	The default value for archdir, named according to version information, on 
	the remote node is:
	\$(BASEDIR)/<condor-version>-<Globus canonicalsystemname>

\item[\Arg{--localdir $<$name$>$}]
	Specify the remote directory in which to create log and execution 
	subdirectories needed by Condor. If you have limited disk quota in your 
	HOME or BASEDIR (see above) directories on the remote machine, setting 
	this value to a large temporary space, e.g., "/tmp" or "/scratch" 
	will allow the jobs to run with fewer disk quota problems.

\item[\Arg{--contactfile $<$file$>$}]
	Allows the use of a file of contact strings, rather than a single
	contact string allowed on the glidein command line. If this option
	is used, a glidein will be performed to each of the contacts listed,
	using the rest of the command line options.

\item[\Arg{--setuponly}]
   Performs only the setup of files on the remote host, but does not
   run the generated script for launching the Condor daemons.
   (Cannot be run simultaneously with --runonly)

\item[\Arg{--runonly}]
   Checks that the files are in place, but does not do the remote setup
   (with the exception of generating and placing the script to launch
   the daemons). If any of the other files are missing, exits with
   an error code.
   (Cannot be run simultaneously with --setuponly)

\item[\Arg{--scheduler $<$name$>$}]
	Select the scheduler type to be used by Globus. Defaults to "fork".
	NOTE: contact strings which already contain the scheduler type will NOT
	be overridden by this option.

\item[\Arg{--queue $<$name$>$}]
	Specify which queue=$<$queuename$>$ to submit to for the Globus scheduler.

\item[\Arg{--count $<$CPU count$>$}]
	Number of CPUs to request, default 1.

\item[\Arg{--runfor $<$mins$>$}]
	How long the glided-in Condor daemons should remain idle before the
	glidein shuts itself down. A value of 0 (zero) means no automatic shutdown,
	in which case the daemons will run until killed with a \Condor{rm}.
	If this option is not specified, the default value of 20 minutes is used.

\item[\Arg{--anybody}]
	Overrides the default Condor start expression to allow any user job which
	meets other requirements to run on your glidein node(s). The default START
	expression for glidein nodes requires that the job's Owner matches the
	username (on the LOCAL node) of the user running glidein.

\item[\Arg{--admin $<$address$>$}]
	Who to email with problems. Defaults to your login name at UID\_DOMAIN of the Condor pool.

\item[\Arg{--genconfig}]
	Creates a file, condor\_config.glidein, which is a copy of the configuration
	file needed by glidein nodes.

\item[\Arg{--help}]
   Prints terse description and usage, then exits.
\end{description}
