\section{\label{sec:Glidein}Extending your Condor pool by Gliding to remote machines}

\Condor{glidein} is a program that can be used to add resources to a Condor 
pool on a temporary basis. During this period, these resources are visible 
to users of the pool, but only the user performing the glidein is allowed 
to use them. The machine in the Condor pool is referred to herein as the
local node, while the resource you are adding to your local Condor pool
will be referred to as the remote node.

\subsection{Globus requirements}
These requirements are general to using any Globus resource:
\begin{enumerate}

\item You must have a Globus X.509 identity to use any Globus resource.
   See \Url{http://www-fp.globus.org/security/tutorial.html} for information.

\item The local node (machine where you will run \Condor{glidein} from)
   must have the Globus tools in your PATH environment variable.
   Example for a user running the "csh" shell at UW:
   setenv \Env{PATH} "\$PATH:/p/condor/workspaces/globus/tools"

\item You must have access on the local node to your Globus X.509 credentials. 
   Normally, these are located in the \$HOME/.globus/ subdirectory of your 
   account on the local node. If they are located elsewhere, but accessible
   from the local node, see \Url{http://www-fp.globus.org/security/environment.html} 
   for requirements. If not, you should copy the entire .globus/ subdirectory
   tree to the local node. If you use the \Prog{tar} utility to do this, use the
   'h' option to follow symbolic links.
   Example:
    (at the node where \texttt{\$HOME}/.globus/ exists) \Prog{tar} cfh globuscert.tar .globus
    (then, copy globuscert.tar to the local node you want to run
       \Condor{glidein} from)
    (at the local Condor node, in your \texttt{\$HOME} directory) \Prog{tar} xf globuscert.tar
   
\item Ensure the resource to which you want to glide in is running
   a Globus gatekeeper.

\item Ensure that you have the proper authorization to run as a Globus
   user on the remote resource. This generally includes a regular
   user account on the remote resource, as well as authorization to
   use Globus on that resource (2 separate steps!). 
   \Url{http://www.globus.org} has some information on Globus contacts for
   various Globus sites. The contact for MCS at Argonne is 
   bester@mcs.anl.gov

\item You can test to ensure all the Globus requirements above are met by 
   using the authentication-only feature of the Globus tool, \Prog{globusrun}.
   Example: haha@cs.wisc.edu\% \Prog{globusrun} -a -r goshen.mcs.anl.gov-fork
   This example contacts the "fork" gatekeeper for the remote node
   "goshen" at Argonne Labs. A message indicating failure means that
   one of these steps is not configured correctly.

\end{enumerate}

\subsection{Condor requirements}
\begin{enumerate}
\item You must be an authorized user of the local Condor pool.

\item Your pool's Condor configuration file(s) must have the remote Globus node's
   hostname in the HOSTALLOW\_WRITE list. Your Condor administrator should
   do this for you.
   Example- to allow ALL machines from MCS at Argonne Labs to glide into the
     Condor pool at UW, add to the HOSTALLOW\_WRITE line in the appropriate
     \condor{config} file:
        HOSTALLOW\_WRITE = *.cs.wisc.edu 
        ===> (change to)
        HOSTALLOW\_WRITE = *.cs.wisc.edu, *.mcs.anl.gov
\end{enumerate}

\subsection{Usage example}
Example: a Condor user at UW has access to Globus resources at 
Argonne, and wants to add the resources to his local Condor pool.

\begin{enumerate}

\item From his local machine (e.g., monterey6@cs.wisc.edu) the user runs 
   the \Condor{glidein} script specifying the remote resource he wants 
   to acquire:

monterey6@cs.wisc.edu\% \Condor{glidein} goshen.mcs.anl.gov

\item The user is prompted for his Globus identity's secret key passphrase
   (PEM passphrase). If the passphrase is not correct, the script exits,
   at which point the user may try again.

\item The remote node is checked to see if the user has permission to
   use the remote node as a resource.

\item If the necessary files are not on the remote node, the user
   is asked where the files are located for the correct versions of
   the Condor executables, which are then copied to the remote node.

\item The executables are run on the remote node and join the pool.
   After a couple of seconds you can see the remote resources in your 
   local Condor pool, e.g.:

   monterey6@cs.wisc.edu\% \Condor{status} | grep goshen
      vm1@goshen.mc SOLARIS26   SUN4u  Unclaimed  Idle       0.000   128 ...
      vm2@goshen.mc SOLARIS26   SUN4u  Unclaimed  Idle       0.355   128 ...

\item \Condor{glidein} offers you several option to manage the remote resource. 
   Use "\Condor{glidein} --help" for details.

\item You are ready to submit job(s) to your Condor pool. If you wish to force
   a job to run on the remote node you started with \Condor{glidein}, you can
   specify the remote node as a machine requirement in your Condor submit
   file: 
\begin{verbatim}
      requirements = ( machine == "goshen.mcs.anl.gov" ) \
         && FileSystemDomain != "" \
         && Arch != "" && OpSys != ""
\end{verbatim}
   (This example requires that the job run only on goshen.mcs.anl.gov, and
   prevents Condor from inserting the filesystem domain, architecture, and 
   operating system attributes as requirements in the matchmaking process,
   since it is likely you may submit Condor jobs from a machine whose
   attributes don't match the remote node's attributes).

\end{enumerate}

\subsection{How it works}
\Condor{glidein} creates a Globus proxy certificate (prompting the user
for their "PEM passphrase" if their Globus private key is passphrase
protected) to authenticate the user to the Globus gatekeeper running
on the Globus machine to which he is gliding in (remote node). The
(temporary) proxy certificate is destroyed upon completion of the 
\Condor{glidein} program.

\Condor{glidein} then contacts the remote node and checks for the
presence of the necessary configuration files and Condor executables.
If the executables are not present for the machine architecture, 
operating system version, and Condor pool version required, the
user is prompted for the path to the local Condor version of the
executables to copy to the remote node.
Copying these files is performed only once for a given combination of
architecture, OS and pool versions.

Once \Condor{glidein} determines that the files are correctly in place,
it runs a script on the remote node that runs the Condor executables.
An optional time limit can be specified, after which time the Condor 
executables gracefully shut down. If no time limit is specified, the 
Condor executables on the remote node can be shut down when no longer
needed by re-running \Condor{glidein} with the --kill option.

The Condor executables on the remote node contact the local pool and
attempt to join the pool (see Condor requirements above). The START
expression for the \Condor{startd} program requires that the username
of the person who ran the glidein script matches the username of the
jobs submitted through Condor, although this can be overridden by
editing the configuration file installed on the remote node by the
\Condor{glidein} program.

\subsection{Configuration options}
Default values you can configure:
\begin{verbatim}
$CONDOR_ADMIN
   The address that condor daemon problems are emailed to.
   (Not all problems generate email messages). Because of Perl's
   syntax, the address must either be in single quotes, or the @ sign
   must be escaped (e.g. $CONDOR_ADMIN = "user\@host.cs.wisc.edu").

$BASE_DIR
   This directory is searched for the necessary files. If the files
   aren't there, they are generated or copied in (in the case of
   executables).  If a fully qualfied path name (from /) is not
   specified, it will be relative to your $HOME directory on the
   remote Globus host.

$SCRIPT
   Prefix for the name of the script generated to remotely run the
   Condor daemons. The hostname of the remote Globus machine is appended
   to this prefix.
   This prefix can be overridden by using the --savescript option and
   specifying a script name prefix. This is useful if you want to
   avoid the overhead of running this program, and simply use the
   generated script for future fly-ins.
   See the section on INVOCATION OPTIONS below for details.

$START_EXPRESSION
   Condor configuration expression that is required to be satisfied
   before a job can be matched by Condor and run on the remote Globus
   node. It is recommended that you use caution if you change this,
   since modifying the default start expression could allow other users
   who submit jobs to your Condor pool to run Condor jobs on your
   remote Globus resource, possibly violating usage agreements with
   the owners of the Globus resources.

$RUN_FOR
   Default time (in minutes) that the condor daemons will run on the
   remote host.  If you set the default to 0 (zero), the daemons will run
   until shut down with the --kill option.
   This value can be overridden with the --runfor option
   See the section on INVOCATION OPTIONS below for details.

$INCLUDE_PVM
   Setting this variable to non-zero will cause the Condor binaries
   needed by Condor-PVM to be copied along with the necessary standard
   binaries, and add the appropriate entries to the configuration files.

OTHER USEFUL INFORMATION
   PVM user \Note If you are using Condor-PVM, PVM jobs only use one
   (of potentially many) of the CPUs on a multi-processor machine.
   What this means is that flying into a multiprocessor machine may
   look like you have several processors available to your jobs, but
   only one CPU per machine will be utilized by PVM. If you need to
   add more than one CPU to your pool, you will therefore need to fly
   in to multiple hosts.
\end{verbatim}

\subsection{Invocation options}
--runfor[ mins]
   How many minutes the condor daemons should run for before gracefully
   exiting. Use of this option overrides the \texttt{\$RUN\_FOR} default value.
   If you don't know how long you need the daemons to run for, you
   can use this option without a value (or specify 0).
   Be aware that if you do this, the daemons will stay running indefinitely
   until you shut them down with the "kill" option.

   An example of using this is:

      /* if you don't know how long your Condor jobs will take */
      \Condor{glidein} --runfor 0 globushost1.some.edu globushost2.other.gov
      \Condor{submit} <submit file using these resources>
      /* wait for your jobs to complete */
      \Condor{glidein} --kill globushost1.some.edu globushost2.other.gov

      /* if you know your jobs will all be done in 60 mins */
      \Condor{glidein} --runfor 60 globushost1.some.edu globushost2.other.gov
      \Condor{submit} <submit file using these resources>

--setuponly
   Performs only the setup of files on the remote host, but does not
   run the generated script for launching the Condor daemons.
   (Cannot be run simultaneously with --runonly)

--runonly
   Checks that the files are in place, but does not do the remote setup
   (with the exception of generating and placing the script to launch
   the daemons). If any of the other files are missing, exits with
   an error code.
   (Cannot be run simultaneously with --setuponly)

--savescript <prefix>
   Specify the prefix for the script generated for launching Condor
   on the remote host. The hostname of the remote node is appended
   to this prefix. The script is not deleted, but is retained for
   debugging or customization options.

--kill
   Gracefully shut down the Condor daemons running on the remote host
   that were started by flying in. This should not interfere with other
   Condor daemons running on the host that were not started by your
   flyin session.
   See also --runfor option.

--help
   Prints terse description and usage, then exits.

--showdefaults
   Prints the current settings of configuration values and exits.

--scheduler
	Select the scheduler type to be used by Globus. Defaults to "fork".

\subsection{Future plans}
--Use a separate configuration file for default values, rather than
having to set values within the perl script itself.

--Utilize an FTP server with the necessary Condor executables so the
user need not be bothered with trying to figure out the location of 
the executable files needed on the remote host.

--Explore the use of GASS tools as an alternative to copying the
Condor executables to the remote node.

--Add support to utilize ssh and rsh tools as an alternative to Globus
tools to perform a glidein.
