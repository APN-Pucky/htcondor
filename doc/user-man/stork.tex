%%%%%%%%%%%%%%%%%%%%%%%%%%%%%%%%%%%%%%%
\section{Stork Applications}\label{sec:Stork}
%%%%%%%%%%%%%%%%%%%%%%%%%%%%%%%%%%%%%%%
\index{Stork|(}

Today's scientific applications have huge data requirements,
which continue to increase drastically every year.
These data are generally accessed by many
users from all across the the globe.
This requires moving huge amounts of data
around wide area networks to complete the computation cycle,
which brings with
it the problem of efficient and reliable data placement.

Stork is a scheduler for data placement.
With Stork, \Term{data placement jobs}
have been elevated to the same level as Condor's computational jobs;
data placements are queued, managed, queried and
autonomously restarted upon error.
Stork understands the semantics and protocols of data placement.

The underlying data placement jobs are performed by Stork
\Term{modules}, typically installed in the Condor \File{libexec}
directory.  The module name is encoded from the data placement type
and functions.  
For example, the \File{stork.transfer.file-file} module transfers data
from the \File{file:/} (local file system) to the \File{file:/}
protocol.  The \File{stork.transfer.file-file} module is the only
module bundled with Condor/Stork.  Additionally, contributed modules
may be \htmladdnormallink{downloaded}
{http://www.cs.wisc.edu/condor/stork/download.html}
for these data transfer protocols:

\begin{table}[hbt]
\begin{tabular}{ l l }
%file:/		& (Stork Server) local file system \\
ftp://		& FTP File Transfer Protocol \\
http://		& HTTP Hypertext Transfer Protocol \\
gsiftp://	& Globus Grid FTP  \\ 
nest://		& Condor NeST  network storage appliance (see
\URL{http://www.cs.wisc.edu/condor/nest/}) \\
srb://		& SDSC  Storage Resource Broker (SRB) (see
\URL{http://www.sdsc.edu/srb/}) \\
srm://		& Storage Resource Manager (SRM) (see
\URL{http://sdm.lbl.gov/srm-wg/}) \\
csrm://		& Castor Storage Resource Manager (Castor SRM) (see
\URL{http://castor.web.cern.ch/castor/}) \\
unitree://    & NCSA UniTree (see
\URL{http://www.ncsa.uiuc.edu/Divisions/CC/HPDM/unitree/}) \\
%		diskrouter:// -> UW DiskRouter Tool
\end{tabular}
\end{table}

The Stork
module API is simple and extensible, enabling users to create 
and use their own modules.

Stork includes high level features for managing data transfers.
By configuration, the number of active jobs running
from a Stork server may be limited.
Stork includes built in fault tolerance,
with capabilities for retrying failed jobs,
together with the specification of alternate protocols.
Stork users also have access to a higher level job manager, 
Condor DAGMan (section \ref{sec:DAGMan}),
which can manage both Stork data placement jobs
and traditional Condor jobs at the same time.


%%%%%%%%%%%%%%%%%%%%%%%%%%%%%%%%%%%%%%%
\subsection{\label{sec:Stork-Job-Submission}Submitting Stork Jobs}
%%%%%%%%%%%%%%%%%%%%%%%%%%%%%%%%%%%%%%%
\index{Stork!submit description file}

As with Condor jobs, Stork jobs are specified with a
submit description file.
It is important to note the syntax of the submit description file
for a Stork job is different than that used by Condor jobs.
Specifically,
Stork submit description files are written in the
\htmladdnormallink{ClassAd}{http://www.cs.wisc.edu/condor/classad}
language.  
See the ClassAd Language Reference Manual for complete details.
Please note that while most of Condor uses ClassAds,
Stork utilizes the most recent version of this language,
which has evolved over time.
Stork defines keywords.
When present in the job submit file,
keywords define the function of the job.

Here is sample Stork job submit description file,
showing file syntax and keywords.
A job specifies a 1-to-1 mapping of a data source URL to
destination URL.

\footnotesize
\begin{verbatim}
// This is a comment line.
[
    dap_type = transfer;
    src_url = "file:/etc/termcap";
    dest_url = "file:/tmp/stork/file-termcap";
]
\end{verbatim}
\normalsize


This example shows the ClassAd pairs that form the
heart of a Stork job specification.
The minimum keywords required to specify a Stork job are:

\begin{description}
  \item[dap\_type] Currently, the data type is constrained to 
  \SubmitCmd{transfer}.

  \item[src\_url]  Specify the data protocol and URL of the source.

  \item[dest\_url]  Specify the data protocol and URL of the destination.
\end{description}

Additionally, the following keywords may be used in 
a Stork submit description file:

\begin{description}
    \item[x509proxy] Specifies the location of the
    X.509 proxy file for protocols
    that use GSI authentication, such as \SubmitCmd{gsiftp://}.
    The special value of \emph{"default"} (quotes are required) invokes
    GSI libraries to search for the user credential in the standard locations.

    \item[alt\_protocols] A comma separated list of
    alternative protocol pairs (for source and destination protocols),
    used in a round robin fashion
    when transfers fail.
    See section \ref{sec:Stork-Fault-Protection} for a further discussion
    and examples.


\end{description}

Stork places no restriction on the submit file name or extension, and will
accept any valid file name for a Stork submit description file.

Submit data placement jobs to Stork using the
\Stork{submit} tool.
For example, after creating the submit description file
\File{sample.stork} with an
editor, submit the data transfer job with the command:

\begin{verbatim}
stork_submit sample.stork
\end{verbatim}

Stork then returns the associated job id, which is used by other Stork
job control tools.

Only the first ClassAd (a record expression within brackets) within 
a Stork submit description file becomes a data placement job
upon submission.
Other ClassAds within the file are ignored.

%%%%%%%%%%%%%%%%%%%%%%%%%%%%%%%%%%%%%%%
\subsection{\label{sec:Stork-Job-Management}Managing Stork Jobs}
%%%%%%%%%%%%%%%%%%%%%%%%%%%%%%%%%%%%%%%
Stork provides a set of command-line user tools for job management, including
submitting, querying, and removing data placement jobs.

%%%%%%%%%%%%%%%%%%%%%%%%%%%%%%%%%%%%%%%
\subsubsection{\label{sec:stork-query}Querying Stork Jobs}
%%%%%%%%%%%%%%%%%%%%%%%%%%%%%%%%%%%%%%%

Use \Stork{status} to check the status of any active
or completed Stork job.
\Stork{status} takes a single argument: the job id.
For example, to check the status of the Stork job with job id 3:

\begin{verbatim}
stork_status 3
\end{verbatim}

Use \Stork{q} to query all active Stork jobs.
\Stork{q} does not report on completed Stork jobs.

For example, to check the status all active Stork jobs:

\begin{verbatim}
stork_q
\end{verbatim}

%%%%%%%%%%%%%%%%%%%%%%%%%%%%%%%%%%%%%%%
\subsubsection{\label{sec:stork-rm}Removing Stork Jobs}
%%%%%%%%%%%%%%%%%%%%%%%%%%%%%%%%%%%%%%%

Active jobs may be removed from the job queue with the 
\Stork{rm} tool.  
\Stork{rm} takes a single argument: the job id of the job to remove.
All jobs may
be removed, provided they have not completed.

For example, to remove the queued job with job id 4:

\begin{verbatim}
stork_rm 4
\end{verbatim}

%%%%%%%%%%%%%%%%%%%%%%%%%%%%%%%%%%%%%%%
\subsection{\label{sec:Stork-Fault-Protection}Fault Tolerance}
%%%%%%%%%%%%%%%%%%%%%%%%%%%%%%%%%%%%%%%

In an ideal world, all data transfers succeed on the first attempt.
However, data transfers do fail for various reasons.
Stork is designed with data transfer fault tolerance.
Based on configuration, Stork retries failed data transfer jobs
using specified protocols.

If a  transfer fails, Stork attempts the transfer again,
until the number of attempts reaches the limit,
as defined by the 
configuration variable
\Macro{STORK\_MAX\_RETRY} 
(section \ref{param:StorkMaxRetry}).  

For each attempt at transfer,
the transfer protocols to be used at both source and destination are defined.
These transfer protocols may vary,
when defined by an \SubmitCmd{alt\_protocols} entry in the
submit description file.
The location of the data at the source and destination
is unchanged by the \SubmitCmd{alt\_protocols} entry.
\SubmitCmd{alt\_protocols} defines an ordered list of alternative
translation protocols to be used.
Each entry in the list is a pair.
The first of the pair defines the protocol to be used at the source 
of the transfer.
The second of the pair defines the protocol to be used at the destination 
of the transfer.

The syntax is a comma-separated list of pairs.
A dash character separated the pairs.
The protocol name is given in all lower case letters,
without colons or slash characters.
Stork uses these strings to identify the protocol translation
and transfer module to be used. 

The initial translation protocol
(specified in the \SubmitCmd{src\_url} and \SubmitCmd{dest\_url} entries)
together with the list defined by  
an \SubmitCmd{alt\_protocols} entry form the ordered list
of protocols to be utilized in a round robin fashion.

For example, if \MacroNI{STORK\_MAX\_RETRY} has the value 4,
and the Stork job submit description file contains
\footnotesize
\begin{verbatim}
[
    dap_type = transfer;
    src_url = "gsiftp://serverA/dirA/fileA";
    dest_url = "http://serverB/dirB/fileB";
]
\end{verbatim}
\normalsize

then Stork will attempt up to 4 transfers,
with each using the same translation protocol.
\SubmitCmd{gsiftp://} is used at the source,
and \SubmitCmd{http://} is used at the destination.
The Stork job fails if it has not been completed after
4 attempts.

A second example shows the transfer protocols used for
each attempted transfer, 
when \SubmitCmd{alt\_protocols} is used.
For this example, assume that 
\MacroNI{STORK\_MAX\_RETRY} has the value 7.
\footnotesize
\begin{verbatim}
[
    dap_type = transfer;
    src_url = "gsiftp://no-such-server/dir/file";
    dest_url = "file:/dir/file";
    alt_protocols = "ftp-file, http-file";
]
\end{verbatim}
\normalsize

Stork attempts the following transfers, in the given order,
stopping when the transfer succeeds.
\begin{enumerate}
    \item from 
			\File{gsiftp://no-such-server/dir/file}
            to
			\File{file:/dir/file}
    \item from 
			\File{ftp://no-such-server/dir/file}
            to
			\File{file:/dir/file}
    \item from 
			\File{http://no-such-server/dir/file}
            to
			\File{file:/dir/file}
    \item from 
			\File{gsiftp://no-such-server/dir/file}
            to
			\File{file:/dir/file}
    \item from 
			\File{ftp://no-such-server/dir/file}
            to
			\File{file:/dir/file}
    \item from 
			\File{http://no-such-server/dir/file}
            to
			\File{file:/dir/file}
    \item from 
			\File{gsiftp://no-such-server/dir/file}
            to
			\File{file:/dir/file}
 \end{enumerate}

%%%%%%%%%%%%%%%%%%%%%%%%%%%%%%%%%%%%%%%
\subsection{\label{sec:Stork-Advanced}Running Stork Jobs Under DAGMan}
%%%%%%%%%%%%%%%%%%%%%%%%%%%%%%%%%%%%%%%
\index{Stork!jobs under DAGMan}

Condor DAGMan (section \ref{sec:DAGMan}) provides high level management
of both traditional CPU jobs and Stork data placement jobs. 
Using DAGMan, users can specify data placement
using the \Arg{DATA} keyword.
DAGMan can mix Stork data transfer jobs 
and Condor jobs.
This capability lends itself well to grid computing,
as data is often staged in (transferred)
before processing the data.
After processing, output is often staged out (transferred).

Here is a sample DAGMan input file
that stages in input files using Stork transfers,
processes the data as a Condor job,
and stages out the result using a Stork transfer.

\footnotesize
\begin{verbatim}
# Transfer input files using Stork
DATA INPUT1 transfer_input_data1.stork
DATA INPUT1 transfer_input_data2.stork

DATA INPUT2 transfer_data
#
# Process the data using Condor
JOB PROCESS process.condor
#
# Transfer output file using Stork
DATA RESULT transfer_result_data.stork
#
# Specify job dependencies
PARENT INPUT1 INPUT2 CHILD PROCESS
PARENT PROCESS CHILD RESULT
\end{verbatim}
\normalsize

%%%%%%%%%%%%%%%%%%%%%%%%%%%%%%%%%%%%%%%
\subsection{\label{sec:Stork-Lease-Manager}The Lease Manager}
%%%%%%%%%%%%%%%%%%%%%%%%%%%%%%%%%%%%%%%
\index{Stork!condor\_lease\_manager@\Condor{lease\_manager}}
\index{Condor daemon!condor\_lease\_manager@\Condor{lease\_manager}}
\index{daemon!condor\_lease\_manager@\Condor{lease\_manager}}
\index{condor\_lease\_manager daemon}

The Lease Manager provides a mechanism for managing leases to resources, 
as described by Condor's ClassAd mechanism.  
The resources and leases are persistent,
so that state may be restored after a shutdown or crash.

Resources are advertised to the Lease Manager by publishing a Condor ClassAd 
with a \Condor{collector}.
These leases describe the number of resources available (the number of leases),
and they can also specify a requirements expression 
using Condor's ClassAd mechanism.
The resource may also specify the maximum 
duration of the leases it will allow.

Similarly, leases are requested by through the Condor ClassAd mechanism;
a Condor daemon client API provides an interface through which to 
request a lease.  
This request ClassAd may specify the number of and the duration of 
the resource leases that are being requested.
It may also specify a requirements expression.

With both the resource and request able to specify requirements expressions, 
the Lease Manager performs 2-way match making, providing a great deal 
of flexibility.

\index{Stork|)}
