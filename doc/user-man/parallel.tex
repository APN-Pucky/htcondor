%%%%%%%%%%%%%%%%%%%%%%%%%%%%%%%%%%%%%%%%%%%%%%%%%%%%%%%%%%%%%%%%%%%%%%
\section{\label{sec:Parallel}Parallel Applications (Including MPI Applications)}
%%%%%%%%%%%%%%%%%%%%%%%%%%%%%%%%%%%%%%%%%%%%%%%%%%%%%%%%%%%%%%%%%%%%%%
\index{parallel universe|(}
\index{MPI application}

HTCondor's parallel universe supports jobs that span multiple machines,
where the multiple processes within a job must be running concurrently
on these multiple machines, perhaps communicating with each other.
The parallel universe provides machine scheduling,
but does not enforce a particular programming paradigm for the
underlying applications.
Thus, parallel universe jobs may run under various MPI implementations
as well as under other programming environments. 

The parallel universe supersedes the mpi universe.
The mpi universe eventually will be removed from HTCondor.

%%%%%%%%%%%%%%%%%%%%%%%%%%%%%%%%%%%%%%%%%%%%%%%%%%%%%%%%%%%%%%%%%%%
\subsection{\label{sec:parallel-model}How Parallel Jobs Run}
%%%%%%%%%%%%%%%%%%%%%%%%%%%%%%%%%%%%%%%%%%%%%%%%%%%%%%%%%%%%%%%%%%%

Parallel universe jobs are submitted from the machine running 
the dedicated scheduler.
The dedicated scheduler matches and claims a fixed number of machines (slots)
for the parallel universe job,
and when a sufficient number of machines are claimed,
the parallel job is started on each claimed slot.

Each invocation of \Condor{submit} assigns a single \Attr{ClusterId}
for what is considered the single parallel job submitted.
The \SubmitCmd{machine\_count} submit command identifies how 
many machines (slots) are to be allocated.
Each instance of the \SubmitCmd{queue} submit command acquires
and claims the number of slots specified by \SubmitCmdNI{machine\_count}.
Each of these slots shares a common job ClassAd and
will have the same \Attr{ProcId} job ClassAd attribute value.

Once the correct number of machines are claimed, 
the \SubmitCmd{executable} is started at more or less the same
time on all machines.
If desired, a monotonically increasing integer value that starts
at 0 may be provided to each of these machines.
The macro \MacroUNI{Node} is similar to the MPI \Term{rank}
construct. This macro may be used within the submit description
file in either the \SubmitCmd{arguments} or \SubmitCmd{environment}
command.
Thus, as the executable runs, it may discover its own \MacroUNI{Node}
value.

Node 0 has special meaning and consequences for the parallel job.
The completion of a parallel job is implied and taken to be
when the Node 0 executable exits.  All other nodes that are
part of the parallel job and that have not yet exited on their 
own are killed.
This default behavior may be altered by placing the line
\begin{verbatim}
+ParallelShutdownPolicy = "WAIT_FOR_ALL"
\end{verbatim}
in the submit description file.
It causes HTCondor to wait until every node in the parallel 
job has completed to consider the job finished. 

%%%%%%%%%%%%%%%%%%%%%%%%%%%%%%%%%%%%%%%%%%%%%%%%%%%%%%%%%%%%%%%%%%%
\subsection{\label{sec:parallel-setup}Parallel Jobs and the Dedicated Scheduler}
%%%%%%%%%%%%%%%%%%%%%%%%%%%%%%%%%%%%%%%%%%%%%%%%%%%%%%%%%%%%%%%%%%%

To run parallel universe jobs, HTCondor must be configured such that 
\index{scheduling!dedicated}
machines running parallel jobs are \Term{dedicated}.  
Note that dedicated has a very specific meaning in HTCondor:
while dedicated machines can run serial jobs, they prefer to run
parallel jobs, and dedicated machines never preempt a parallel job 
once it starts running.

A machine becomes a dedicated machine when an administrator configures
it to accept parallel jobs from one specific dedicated scheduler.  
Note the difference between parallel and serial jobs.
While any scheduler in a pool can send serial jobs to any machine,
only the designated dedicated scheduler may send parallel universe
jobs to a dedicated machine.
Dedicated machines must be specially configured.  
See section~\ref{sec:Config-Dedicated-Jobs} for a description
of the necessary configuration, as well as examples.
Usually, a single dedicated scheduler is configured for a pool
which can run parallel universe jobs, and this \Condor{schedd} daemon
becomes the single machine from which parallel universe
jobs are submitted.

The following command line will list the execute machines 
in the local pool which have been configured to use a dedicated
scheduler, also printing the name of that dedicated scheduler.
In order to run parallel jobs, this name will be defined to be
the string \AdStr{DedicatedScheduler@}, prepended to the name of the
scheduler host.

\footnotesize
\begin{verbatim}
  condor_status -const '!isUndefined(DedicatedScheduler)' \
	-format "%s\t" Machine -format "%s\n" DedicatedScheduler

execute1.example.com	DedicatedScheduler@submit.example.com
execute2.example.com	DedicatedScheduler@submit.example.com

\end{verbatim}
\normalsize

If this command emits no lines of output, then then pool is
not correctly configured to run parallel jobs.
Make sure that the name of the scheduler is correct. 
The string after the \Expr{@} sign should match the name of the 
\Condor{schedd} daemon, as returned by the command

\begin{verbatim}
  condor_status -schedd
\end{verbatim}

%%%%%%%%%%%%%%%%%%%%%%%%%%%%%%%%%%%%%%%%%%%%%%%%%%%%%%%%%%%%%%%%%%%
\subsection{\label{sec:parallel-submit}Submission Examples}
%%%%%%%%%%%%%%%%%%%%%%%%%%%%%%%%%%%%%%%%%%%%%%%%%%%%%%%%%%%%%%%%%%%

\begin{description}
\item[Simplest Example]
\end{description}

Here is a submit description file for a parallel universe
job example that is as simple as possible:

\begin{verbatim}
#############################################
##  submit description file for a parallel universe job
#############################################
universe = parallel
executable = /bin/sleep
arguments = 30
machine_count = 8
log = log
should_transfer_files = IF_NEEDED
when_to_transfer_output = ON_EXIT
queue 
\end{verbatim}

This job specifies the \SubmitCmdNI{universe} as \SubmitCmdNI{parallel}, letting
HTCondor know that dedicated resources are required.  The
\SubmitCmd{machine\_count} command identifies that eight machines are
required for this job. 

Because no \SubmitCmd{requirements} are specified,
the dedicated scheduler claims eight machines 
with the same architecture and operating system as the submit machine.
When all the machines are ready, it invokes the
\Prog{/bin/sleep} command, with a command line argument of 30
on each of the eight machines more or less simultaneously.
Job events are written to the log specified in the \SubmitCmd{log} command.

The file transfer mechanism is enabled for this parallel job,
such that if any of the eight claimed execute machines does not
share a file system with the submit machine, 
HTCondor will correctly transfer the executable. 
This \Prog{/bin/sleep} example implies that the submit machine
is running a Unix operating system,
and the default assumption for submission from a Unix machine
would be that there is a shared file system.

\begin{description}
\item[Example with Operating System Requirements]
\end{description}

Assume that the pool contains
Linux machines installed with either a RedHat or an Ubuntu operating system.
If the job should run only on RedHat platforms,
the requirements expression may specify this:

\footnotesize
\begin{verbatim}
#############################################
##  submit description file for a parallel program
##  targeting RedHat machines
#############################################
universe = parallel
executable = /bin/sleep
arguments = 30
machine_count = 8
log = log
should_transfer_files = IF_NEEDED
when_to_transfer_output = ON_EXIT
requirements = (OpSysName == "RedHat")
queue 
\end{verbatim}
\normalsize

The machine selection may be further narrowed, instead using the
\Attr{OpSysAndVer} attribute.

\footnotesize
\begin{verbatim}
#############################################
##  submit description file for a parallel program 
##  targeting RedHat 6 machines
#############################################
universe = parallel
executable = /bin/sleep
arguments = 30
machine_count = 8
log = log
should_transfer_files = IF_NEEDED
when_to_transfer_output = ON_EXIT
requirements = (OpSysAndVer == "RedHat6")
queue
\end{verbatim}
\normalsize

\begin{description}
\item[Using the \MacroUNI{Node} Macro]
\end{description}

\footnotesize
\begin{verbatim}
######################################
## submit description file for a parallel program
## showing the $(Node) macro
######################################
universe = parallel
executable = /bin/cat
log = logfile
input = infile.$(Node)
output = outfile.$(Node)
error = errfile.$(Node)
machine_count = 4
should_transfer_files = IF_NEEDED
when_to_transfer_output = ON_EXIT
queue
\end{verbatim}
\normalsize

The \MacroUNI{Node} macro is expanded to values of 0-3
as the job instances are about to be started.
This assigns unique names to the input and output files
to be transferred or accessed from the shared file system.
The \MacroUNI{Node} value is fixed for the entire length of the job.

\begin{description}
\item[Differing Requirements for the Machines]
\end{description}

Sometimes one machine's part in a parallel job will have specialized needs.
These can be handled with a \SubmitCmd{Requirements} submit command
that also specifies the number of needed  machines. 

\footnotesize
\begin{verbatim}
######################################
## Example submit description file
## with 4 total machines and differing requirements
######################################
universe = parallel
executable = special.exe
machine_count = 1
requirements = ( machine == "machine1@example.com")
queue

machine_count = 3
requirements = ( machine =!= "machine1@example.com")
queue
\end{verbatim}
\normalsize

The dedicated scheduler acquires and claims four machines.
All four share the same value of \Attr{ClusterId},
as this value is associated with this single parallel job.
The existence of a second \SubmitCmd{queue} command causes a total of two
\Attr{ProcId} values to be assigned for this parallel job.
The \Attr{ProcId} values are assigned based on ordering
within the submit description file. 
Value 0 will be assigned for the single executable
that must be executed on machine1@example.com, and
the value 1 will be assigned for the other three 
that must be executed elsewhere.

\begin{description}
\item[Requesting multiple cores per slot]
\end{description}

If the parallel program has a structure that benefits from
running on multiple cores within the same slot,
multi-core slots may be specified.

\footnotesize
\begin{verbatim}
######################################
## submit description file for a parallel program
## that needs 8-core slots
######################################
universe = parallel
executable = foo.sh
log = logfile
input = infile.$(Node)
output = outfile.$(Node)
error = errfile.$(Node)
machine_count = 2
request_cpus = 8
should_transfer_files = IF_NEEDED
when_to_transfer_output = ON_EXIT
queue
\end{verbatim}
\normalsize

This parallel job causes the scheduler to match and claim two
machines, where each of the machines (slots) has eight cores.
The parallel job is assigned a single \Attr{ClusterId}
and a single \Attr{ProcId}, meaning that there is a single job
ClassAd for this job.

The executable, \File{foo.sh}, is started at the same time
on a single core within each of the two machines (slots).
It is presumed that the executable will take care of invoking
processes that are to run on the other seven CPUs (cores)
associated with the slot. 

Potentially fewer machines are impacted with this specification,
as compared with the request that contains
\footnotesize
\begin{verbatim}
machine_count = 16
request_cpus = 1
\end{verbatim}
\normalsize
The interaction of the eight cores within the single slot may 
be advantageous with respect to communication delay or memory access.
But, 8-core slots must be available within the pool.

\label{sec:parallel-mpi-submit}
\begin{description}
\item[MPI Applications]
\end{description}

\index{parallel universe!running MPI applications}
\index{MPI application}

MPI applications use a single executable, 
invoked on one or more machines (slots), executing in parallel. 
The various implementations of MPI
such as Open MPI and MPICH require further framework.
HTCondor supports this necessary framework through 
a user-modified script.
This implementation-dependent script becomes the HTCondor executable.
The script sets up the framework,
and then it invokes the MPI application's executable.

The scripts are located in the
\File{\MacroUNI{RELEASE\_DIR}/etc/examples} directory.
The script for the Open MPI implementation is \File{openmpiscript}.
The scripts for MPICH implementations are \File{mp1script} and \File{mp2script}.
An MPICH3 script is not available at this time.
These scripts rely on running \Prog{ssh} 
for communication between the nodes of the MPI application.
The \Prog{ssh} daemon on Unix platforms restricts
connections to the approved shells listed in the \File{/etc/shells} file.

Here is a sample submit description file for an MPICH MPI application:

\footnotesize
\begin{verbatim}
######################################
## Example submit description file
## for MPICH 1 MPI
## works with MPICH 1.2.4, 1.2.5 and 1.2.6
######################################
universe = parallel
executable = mp1script
arguments = my_mpich_linked_executable arg1 arg2
machine_count = 4
should_transfer_files = yes
when_to_transfer_output = on_exit
transfer_input_files = my_mpich_linked_executable
queue
\end{verbatim}
\normalsize

The \SubmitCmd{executable} is the \File{mp1script} script that will have been
modified for this MPI application.
This script is invoked on each slot or core.
The script, in turn, is expected to invoke the MPI application's executable.
To know the MPI application's executable,
it is the first in the list of \SubmitCmd{arguments}.
And, since HTCondor must transfer this executable to the machine where
it will run,
it is listed with the \SubmitCmd{transfer\_input\_files} command,
and the file transfer mechanism is enabled with
the \SubmitCmd{should\_transfer\_files} command.

Here is the equivalent sample submit description file,
but for an Open MPI application:

\footnotesize
\begin{verbatim}
######################################
## Example submit description file
## for Open MPI
######################################
universe = parallel
executable = openmpiscript
arguments = my_openmpi_linked_executable arg1 arg2
machine_count = 4
should_transfer_files = yes
when_to_transfer_output = on_exit
transfer_input_files = my_openmpi_linked_executable
queue
\end{verbatim}
\normalsize

Most MPI implementations require two system-wide prerequisites.
The first prerequisite is the ability to run a command
on a remote machine without being prompted for a password.
\Prog{ssh} is commonly used.
The second prerequisite is an ASCII file containing the
list of machines that may utilize \Prog{ssh}.
These common prerequisites are implemented in a further script
called \File{sshd.sh}.
\File{sshd.sh} generates ssh keys to enable password-less remote execution
and starts an \Prog{sshd} daemon.
Use of the \Prog{sshd.sh} script requires the definition of
two HTCondor configuration variables.
Configuration variable \Macro{CONDOR\_SSHD} is an absolute path to
an implementation of \Prog{sshd}.
\Prog{sshd.sh} has been tested with \Prog{openssh} version 3.9,
but should work with more recent versions.
Configuration variable \Macro{CONDOR\_SSH\_KEYGEN} points
to the corresponding \Prog{ssh-keygen} executable.

%Both \Prog{mp1script} and \Prog{openmpiscript}
%require environmental support.
\Prog{mp1script} and \Prog{mp2script} require the \Env{PATH} to the MPICH installation to be set.
The variable \Env{MPDIR} may be modified in the scripts to indicate its proper value.
This directory contains the MPICH \Prog{mpirun} executable.

\Prog{openmpiscript} also requires the \Env{PATH} to the Open MPI installation.
Either the variable \Env{MPDIR} can be set manually in the script,
or the administrator can define \Env{MPDIR}
using the configuration variable \Macro{OPENMPI\_INSTALL\_PATH}.
When using Open MPI on a multi-machine HTCondor cluster,
the administrator may also want to consider tweaking the
\Macro{OPENMPI\_EXCLUDE\_NETWORK\_INTERFACES} configuration variable
as well as set \MacroNI{MOUNT\_UNDER\_SCRATCH} = \Attr{/tmp}.
%Open MPI also uses \Env{MPdir}.
%For OpenMPI, there is a similar path setting, but it is called \Env{LAMDIR}
%in the \Prog{openmpiscript} script.  
%In addition, this path must be part of the
%path set in the user's \File{.cshrc} script.
%As of this writing, the LAM implementation does not work
%if the user's login shell is the Bourne or compatible shell.

\index{parallel universe|)}

%%%%%%%%%%%%%%%%%%%%%%%%%%%%%%%%%%%%%%%%%%%%%%%%%%%%%%%%%%%%%%%%%%%
\subsection{\label{sec:parallel-mpi-submit-single}MPI Applications Within HTCondor's Vanilla Universe}
%%%%%%%%%%%%%%%%%%%%%%%%%%%%%%%%%%%%%%%%%%%%%%%%%%%%%%%%%%%%%%%%%%%

The vanilla universe may be preferred over the parallel universe
for certain parallel applications such as MPI ones.
These applications are ones in which the allocated cores need
to be within a single slot.
The \SubmitCmd{request\_cpus} command causes a claimed slot to
have the required number of CPUs (cores).

There are two ways to ensure that the MPI job can run 
on any machine that it lands on:
\begin{enumerate}
\item Statically build an MPI library and statically compile the MPI code.
\item Use CDE to create a directory tree that contains all of the libraries 
needed to execute the MPI code.
\end{enumerate}

For Linux machines, our experience recommends using CDE,
as building static MPI libraries can be difficult.
CDE can be found at \URL{http://www.pgbovine.net/cde.html}.

Here is a submit description file example assuming that
MPI is installed on all machines on which the MPI job
may run, 
or that the code was built using static libraries and a static version of 
\File{mpirun} is available.

\footnotesize
\begin{verbatim}
############################################################
##   submit description file for 
##   static build of MPI under the vanilla universe
############################################################
universe = vanilla
executable = /path/to/mpirun
request_cpus = 2
arguments = -np 2 my_mpi_linked_executable arg1 arg2 arg3
should_transfer_files = yes
when_to_transfer_output = on_exit
transfer_input_files = my_mpi_linked_executable

queue
\end{verbatim}
\normalsize

If CDE is to be used,
then CDE needs to be run first to create the directory tree. 
On the host machine which has the original program, the command

\footnotesize
\begin{verbatim}
prompt-> cde mpirun -n 2 my_mpi_linked_executable
\end{verbatim}
\normalsize

creates a directory tree that will contain all libraries needed for the 
program.
By creating a tarball of this directory, the user can package up
the executable itself, 
any files needed for the executable, 
and all necessary libraries. 
The following example assumes that the user has created a 
tarball called \File{cde\_my\_mpi\_linked\_executable.tar} which contains the 
directory tree created by CDE.

\footnotesize
\begin{verbatim}
############################################################
##   submit description file for 
##   MPI under the vanilla universe; CDE used
############################################################
universe = vanilla
executable = cde_script.sh
request_cpus = 2
should_transfer_files = yes
when_to_transfer_output = on_exit
transfer_input_files = cde_my_mpi_linked_executable.tar
transfer_output_files = cde-package/cde-root/path/to/original/directory
queue
\end{verbatim}
\normalsize

The executable is now a specialized shell script tailored to this job.
In this example, \Prog{cde\_script.sh} contains:

\footnotesize
\begin{verbatim}
#!/bin/sh
# Untar the CDE package
tar xpf cde_my_mpi_linked_executable.tar
# cd to the subdirectory where I need to run
cd cde-package/cde-root/path/to/original/directory
# Run my command
./mpirun.cde -n 2 ./my_mpi_linked_executable
# Since HTCondor will transfer the contents of this directory
# back upon job completion.
# We do not want the .cde command and the executable transferred back.
# To prevent the transfer, remove both files.
rm -f mpirun.cde
rm -f my_mpi_linked_executable
\end{verbatim}
\normalsize

Any additional input files that will be needed for the executable
that are not already in the tarball
should be included in the list in \SubmitCmd{transfer\_input\_files} command. 
The corresponding script should then also be updated to move those files into
the directory where the executable will be run.


