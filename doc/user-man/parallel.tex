%%%%%%%%%%%%%%%%%%%%%%%%%%%%%%%%%%%%%%%%%%%%%%%%%%%%%%%%%%%%%%%%%%%%%%
\section{\label{sec:Parallel}Parallel Applications (Including MPI Applications)}
%%%%%%%%%%%%%%%%%%%%%%%%%%%%%%%%%%%%%%%%%%%%%%%%%%%%%%%%%%%%%%%%%%%%%%
\index{parallel universe|(}
\index{MPI application}

HTCondor's parallel universe supports jobs that span multiple machines.
All other job universe in HTCondor are serial, that is, each job runs
on exactly one machine, only in the parallel universe are jobs consisting
of multiple processes allowed to be concurrently scheduled across multiple
machines.  The parallel universe provides machine scheduling and allocation,
but does not enforce any particular programming paradigm.  Thus, with
the parallel universe, users can launch jobs using various MPI implementations, 
or other programming environments. The parallel universe supersedes the mpi universe.
The mpi universe eventually will be removed from HTCondor.


%%%%%%%%%%%%%%%%%%%%%%%%%%%%%%%%%%%%%%%%%%%%%%%%%%%%%%%%%%%%%%%%%%%
\subsection{\label{sec:parallel-setup}Prerequisites to Running Parallel Jobs}
%%%%%%%%%%%%%%%%%%%%%%%%%%%%%%%%%%%%%%%%%%%%%%%%%%%%%%%%%%%%%%%%%%%

To run parallel universe jobs, HTCondor must be configured such that machines running
parallel jobs are dedicated.  
\index{scheduling!dedicated}
Note that \Term{dedicated} has a very specific meaning in HTCondor:
while dedicated machine can run serial jobs, they prefer to run
parallel jobs, and never preempt a parallel job once it starts running.
A machine becomes a dedicated machine when an administrator configures
it to accept parallel jobs from one specific dedicated scheduler.  Note
the difference from serial jobs.  Generally speaking, any scheduler in
a pool can send serial jobs to a any machine, but for parallel jobs,
only the designated dedicated scheduler can send parallel universe
jobs to a given machine.

Since HTCondor does not ordinarily run this way, (HTCondor usually uses
opportunistic scheduling), dedicated machines must be specially
configured.  Section~\ref{sec:Config-Dedicated-Jobs} of the
Administrator's Manual describes the necessary configuration and
provides detailed examples.

Usually, a single dedicated scheduler is configured for a pool
which can run parallel universe jobs, and this \Condor{schedd}
becomes the single machine from which parallel universe
jobs are submitted.

The following command line will list the execute machines 
in the local pool which have been configured to use a dedicated
scheduler, and print out the name of that dedicated scheduler.
In order to run parallel jobs, this name must be defined to be
the string "DedicatedScheduler@" prepended to the name of the
scheduler.

\begin{verbatim}
$ condor_status -const '!isundefined(DedicatedScheduler)' \
	-format "%s\t" Machine -format "%s\n" DedicatedScheduler

execute1.example.com	DedicatedScheduler@submit.example.com
execute2.example.com	DedicatedScheduler@submit.example.com

\end{verbatim}

If this command emits no lines of output, then then pool is
not correctly configured to run parallel jobs.  Make sure that
the name of the scheduler is correct, the string after the
at sign should match the name of the schedd, as returned by the
command

\begin{verbatim}
$ condor_status -sched
\end{verbatim}

%%%%%%%%%%%%%%%%%%%%%%%%%%%%%%%%%%%%%%%%%%%%%%%%%%%%%%%%%%%%%%%%%%%
\subsection{\label{sec:parallel-submit}Parallel Job Submission}
%%%%%%%%%%%%%%%%%%%%%%%%%%%%%%%%%%%%%%%%%%%%%%%%%%%%%%%%%%%%%%%%%%%

Given correct configuration, parallel universe jobs may be submitted
from the machine running the dedicated scheduler.
The dedicated scheduler claims machines for the parallel universe job,
and starts the job when the correct number of machines which match
the job's requirements are claimed.
Note that the job likely consists of more than one process,
each to be executed on a separate machine.  If the pool is
not correctly configured, the job will stay in the Idle state
indefinitely.

An overly simplified submit description file for a parallel universe
job appears as

\begin{verbatim}
#############################################
##   submit description file for a parallel program
#############################################
universe = parallel
executable = /bin/sleep
arguments = 30
machine_count = 8
log = log
output = output
error  = error
notification = never
should_transfer_files = always
when_to_transfer_output = on_exit
queue 
\end{verbatim}

This job specifies the \SubmitCmd{universe} as \SubmitCmd{parallel}, letting
HTCondor know that dedicated resources are required.  The
\SubmitCmd{machine\_count} command identifies the number of machines
required by the job. 

When submitted, the dedicated scheduler allocates eight
machines with the same architecture and operating system as the submit
machine.  It waits until all eight machines are available before
starting the job.  When all the machines are ready, it invokes the
\Prog{/bin/sleep} command, with a command line argument of 30
on all eight machines more or less simultaneously.  As with serial
jobs, important event in the lifecycle of the job are written
to the user log noted in the log command.

Parallel universe jobs have the familiar Requirements expression
that serial jobs do, and setting the requirements expression will
restrict the machines that match your job.

For example, if your pool consists of Linux machines installed with the RedHat and Ubuntu operating systems, and you'd like to run on only the RedHat machines, you can use
the following requirements expression:

\begin{verbatim}
#############################################
##   submit description file for a parallel program targeting RedHat machines
#############################################
universe = parallel
executable = /bin/sleep
arguments = 30
machine_count = 8
log = log
output = output
error  = error
notification = never
should_transfer_files = always
when_to_transfer_output = on_exit
requirements = (OpSysName == "RedHat")
queue 

\end{verbatim}

In addition, you may narrow down your machine selection to the version you'd like to run on using the OpSysAndVer attribute.

\begin{verbatim}
#############################################
##   submit description file for a parallel program targeting RedHat 6 machines
#############################################
universe = parallel
executable = /bin/sleep
arguments = 30
machine_count = 8
log = log
output = output
error  = error
notification = never
should_transfer_files = always
when_to_transfer_output = on_exit
requirements = (OpSysAndVer == "RedHat6")
queue
\end{verbatim}

A more realistic example of a parallel job utilizes other features.

\begin{verbatim}
######################################
## Parallel example submit description file
######################################
universe = parallel
executable = /bin/cat
log = logfile
input = infile.$(NODE)
output = outfile.$(NODE)
error = errfile.$(NODE)
machine_count = 4
notification = never
should_transfer_files = always
when_to_transfer_output = on_exit
queue
\end{verbatim}

The specification of the \SubmitCmd{input}, \SubmitCmd{output},
and \SubmitCmd{error} files utilize the predefined macro 
\MacroUNI{NODE}.
\index{macro!predefined}
See the \Condor{submit}
manual page on page~\pageref{man-condor-submit} for further
description of predefined macros.
The \MacroU{NODE} macro is given a
unique integer value, starting at zero as processes are assigned to machines.
This is similar to the MPI notion of "rank".
The \MacroUNI{NODE} value is fixed for the entire length of the job.
It can therefore be used to identify individual aspects of the computation.
In this example, it is used to utilize and assign unique names to
input and output files.

%%%%%%%%%%%%%%%%%%%%%%%%%%%%%%%%%%%%%%%%%%%%%%%%%%%%%%%%%%%%%%%%%%%
\subsection{\label{sec:parallel-multi-proc}Parallel Jobs with Separate Requirements}
%%%%%%%%%%%%%%%%%%%%%%%%%%%%%%%%%%%%%%%%%%%%%%%%%%%%%%%%%%%%%%%%%%%

Usually, all the machines running a parallel job will have the
same requirements.  Sometimes, though, one machine will need
to be different than the rest.  
The different subsets of machines executing a parallel universe job
may specify different machine requirements.  A common example requires that the
head node execute on a specific machine.  It may be also useful for debugging purposes.

Consider the following example.

\begin{verbatim}
######################################
## Example submit description file
## with multiple procs
######################################
universe = parallel
executable = example
machine_count = 1
requirements = ( machine == "machine1")
queue

requirements = ( machine =!= "machine1")
machine_count = 3
queue
\end{verbatim}

The dedicated scheduler allocates four machines.
All four executing jobs have the same value for \MacroUNI{Cluster}
macro.
The \MacroUNI{Process} macro takes on two values;
the value 0 will be assigned for the single executable
that must be executed on machine1, and
the value 1 will be assigned for the other three 
that must be executed anywhere but on machine1.

Carefully consider the ordering and nature of multiple
sets of requirements in the same submit description file.
The scheduler matches jobs to machines based on the ordering
within the submit description file.
Mutually exclusive requirements eliminate the dependence on
ordering within the submit description file.
Without mutually exclusive requirements,
the scheduler may be unable to schedule the job.
The ordering within the submit description file may preclude
the scheduler considering the specific allocation that
could satisfy the requirements.


%%%%%%%%%%%%%%%%%%%%%%%%%%%%%%%%%%%%%%%%%%%%%%%%%%%%%%%%%%%%%%%%%%%%%%
\subsubsection{\label{sec:Configure-Clusters-Processes-Subprocesses}
Parallel Universe: Clusters, Processes, and Subprocesses} 
%%%%%%%%%%%%%%%%%%%%%%%%%%%%%%%%%%%%%%%%%%%%%%%%%%%%%%%%%%%%%%%%%%%%%%
\index{parallel-clusters-proceses-subprocesses}

When 
a user submits a job to HTCondor with \verb=condor_submit=, HTCondor in parallel 
universe processes the submit file and assumes that all jobs submitted in this 
file are part of a single parallel (e.g.~MPI) job. HTCondor processes the submit 
file as a single job and
creates a new cluster with a new \Attr{ClusterID} for the job. The user though can 
specify different requirements for different parts of the job and queue each 
part separately with each part being part of the same cluster but receiving a 
different \Attr{ProcID}. Within each processes, the user can specify particular
requirements.
% (by using 
%partitionable slots) can also request more than one CPU (\verb+num_cpus = 2+) or
%(by using static slots) \verb+machine_count = 2+. 
Each request for a slot is known as a subprocess; each subprocess has the same 
\Attr{ClusterID} and \Attr{ProcID} as the over-arching process but has its 
own unique 
\Attr{SubprocID}. Note that \Attr{SubprocID} is not advertised in ClassAds.

The way to create multiple processes is to place a number after the \verb=queue= 
command, such as \verb=queue 8= to create eight different processes with the 
same cluster. After a particular \verb=queue= command, the user can also change 
parameters previously specified in the submit file or add parameters not 
previously specified. For example, suppose that the head node of a parallel job 
needs additional RAM compared to the rest of the nodes. The user can write

\begin{verbatim}
request_memory = 5G
queue
request_meory = 1G
queue 31
\end{verbatim}

If a user wishes to submit multiple independent parallel universe jobs, the 
user cannot simply append a number to the \verb=queue= command. HTCondor will 
assume that each job is not independent -- as by definition, a parallel job is a 
job is at least setup to handle multiple (sub)processes. Instead, the user will
need to run \Condor{submit} multiple times.

%%%%%%%%%%%%%%%%%%%%%%%%%%%%%%%%%%%%%%%%%%%%%%%%%%%%%%%%%%%%%%%%%%%
\subsection{\label{sec:parallel-mpi-submit}MPI Applications Within HTCondor's Parallel Universe}
%%%%%%%%%%%%%%%%%%%%%%%%%%%%%%%%%%%%%%%%%%%%%%%%%%%%%%%%%%%%%%%%%%%
\index{parallel universe!running MPI applications}
\index{MPI application}

%%%%%%%%%%%%%%%%%%%%%%%%%%%%%%%%%%%%%%%%%%%%%%%%%%%%%%%%%%%%%%%%%%%
\subsubsection{\label{sec:parallel-mpi-submit-multiple}
Running MPI Jobs On Multiple Machines}
%%%%%%%%%%%%%%%%%%%%%%%%%%%%%%%%%%%%%%%%%%%%%%%%%%%%%%%%%%%%%%%%%%%

MPI applications consist of a single executable that is invoked in order to
execute in parallel on one or more machines. 
HTCondor's parallel universe provides the environment within
which this executable is executed in parallel.
However, the various implementations of MPI
(for example, LAM or MPICH) require further framework items within
a system-wide environment.
HTCondor supports this necessary framework through 
user visible and modifiable scripts.
An MPI implementation-dependent script becomes the HTCondor job.
The script sets up the extra, necessary framework,
and then invokes the MPI application's executable.

HTCondor provides these scripts in the
\File{\MacroUNI{RELEASE\_DIR}/etc/examples}
directory.
The script for the OpenMPI implementation is \File{openmpiscript}.
The script for the MPICH implementation is \File{mp1script}.
Therefore, an HTCondor submit description file for these
implementations would appear similar to:

\begin{verbatim}
######################################
## Example submit description file
## for MPICH 1 MPI
## works with MPICH 1.2.4, 1.2.5 and 1.2.6
######################################
universe = parallel
executable = mp1script
arguments = my_mpich_linked_executable arg1 arg2
machine_count = 4
should_transfer_files = yes
when_to_transfer_output = on_exit
transfer_input_files = my_mpich_linked_executable
queue
\end{verbatim}

or

\begin{verbatim}
######################################
## Example submit description file
## for OpenMPI
######################################
universe = parallel
executable = openmpiscript
arguments = my_openmpi_linked_executable arg1 arg2
machine_count = 4
should_transfer_files = yes
when_to_transfer_output = on_exit
transfer_input_files = my_openmpi_linked_executable
queue
\end{verbatim}

The \SubmitCmd{executable} is the MPI implementation-dependent script.
The first argument to the script is the MPI application's 
executable.
Further arguments to the script are the MPI application's arguments.
HTCondor must transfer this executable;
do this with the \SubmitCmd{transfer\_input\_files} command.

For other implementations of MPI,
copy and modify one of the given scripts.
Most MPI implementations require two system-wide prerequisites.
The first prerequisite is the ability to run a command
on a remote machine without being prompted for a password.
\Prog{ssh} is commonly used, but other
commands may be used.
The second prerequisite is an ASCII file containing the
list of machines that may utilize \Prog{ssh}.
These common prerequisites are implemented in a further script
called \File{sshd.sh}.
\File{sshd.sh} generates ssh keys 
(to enable password-less remote execution),
and starts an \Prog{sshd} daemon.
The machine name and MPI rank are given to the submit machine.

%So, to run MPI application in the parallel universe, we run a script
%on each node we submit to.  This script generates ssh keys, to enable
%password-less remote execution, start an sshd daemon, and send the
%names and rank (node number) back to the submit directory.  Thus, for
%each HTCondor job submitted, the scripts set up an ad-hoc MPI
%environment, which is torn down at the end of the job run.  This ssh
%script is a common requirement for running MPI jobs, so we have
%factored it out into a common script, which is called from each of the
%MPI-specific scripts.  After the ssh script has been started, the
%MPI-specific script runs, starts the rest of the MPI job by looking at
%its arguments, and waits for the MPI job to finish.  HTCondor provides
%the ssh script, and example MPI scripts for both LAM and MPICH.  The
%former is named ``lamscript'', and the latter ``mp1script''.  The
%first argument to each script is the name of the real MPI executable,
%and any subsequent arguments are arguments to that executable.  Other
%implementations should be easy to add, by modifying the given
%examples.  Note that because the actual MPI executable (i.e. the
%output of mpicc) is not the named executable in the submit script, it
%must be accessible either via a network file system, or by condor file
%transfer.

The \Prog{sshd.sh} script requires the definition of
two HTCondor configuration variables.
Configuration variable \Macro{CONDOR\_SSHD} is an absolute path to
an implementation of \Prog{sshd}.
\Prog{sshd.sh} has been tested with \Prog{openssh} version 3.9,
but should work with more recent versions.
Configuration variable \Macro{CONDOR\_SSH\_KEYGEN} points
to the corresponding \Prog{ssh-keygen} executable.

Scripts \Prog{openmpiscript} and \Prog{mp1script}
each have their own idiosyncrasies.
In \Prog{mp1script}, the \Env{PATH} to the MPICH installation must be set.
The shell variable MPDIR indicates its proper value.
This directory contains the MPICH \Prog{mpirun} executable; OpenMPI also
uses MPDIR.
%For OpenMPI, there is a similar path setting, but it is called \Env{LAMDIR}
%in the \Prog{openmpiscript} script.  
%In addition, this path must be part of the
%path set in the user's \File{.cshrc} script.
%As of this writing, the LAM implementation does not work
%if the user's login shell is the Bourne or compatible shell.

These MPI jobs operate as all parallel universe jobs do.
The default policy is that when the first node exits,
the whole job is considered done, 
and HTCondor kills all other running nodes in that parallel job.
Alternatively, a parallel universe job that sets the attribute
\begin{verbatim}
+ParallelShutdownPolicy = "WAIT_FOR_ALL"
\end{verbatim}
in its submit description file changes the policy,
such that HTCondor will wait until every node in the parallel 
job has completed to consider the job finished. 

\index{parallel universe|)}

%%%%%%%%%%%%%%%%%%%%%%%%%%%%%%%%%%%%%%%%%%%%%%%%%%%%%%%%%%%%%%%%%%%
\subsubsection{\label{sec:parallel-mpi-submit-single}MPI Applications Within HTCondor's Parallel Universe}
%Within HTCondor's Parallel Universe}
%Running MPI Jobs On A Single Machine}
%%%%%%%%%%%%%%%%%%%%%%%%%%%%%%%%%%%%%%%%%%%%%%%%%%%%%%%%%%%%%%%%%%%

With partitionable slots, a user may wish to run a parallel job written with
MPI but run only on one machine. Parallel universe is designed to help 
coordinate a job on multiple machines and thus is not the correct route for
this kind of job. Instead, vanilla universe should be used with partitionable
slots and the \verb|request_cpus| option should be used in the HTCondor submit
file.

\begin{verbatim}
request_cpus = 8
\end{verbatim}

A frequent question that gets asked is: ``How do I make sure my MPI job can run 
on any machine that it lands on?'' There are two answers:
\begin{enumerate}
\item Statically build an MPI library and statically compile your MPI code.
\item Use CDE to create a directory tree that contains all of the libraries 
needed to execute your code.
\end{enumerate}

For Linux machines, we have found that building static MPI libraries can be 
difficult and that 
using CDE and some shell scripts provides the requested results. CDE can be
found at \URL{http://www.pgbovine.net/cde.html}.

If you have MPI installed on all machines on which your program could run, this
submit description file works well.
\begin{verbatim}
################################################################################
##   Submit Description File for a Parallel Program in Vanilla Universe
################################################################################
universe = vanilla
executable = /path/to/mpirun
request_cpus = 2
arguments = -np 2 my_openmpi_linked_executable arg1 arg2 arge
should_transfer_files = yes
when_to_transfer_output = on_exit
transfer_input_files = my_openmpi_linked_executable

queue
\end{verbatim}

If you do not have MPI installed on all potential machines and want to use CDE,
this submit description file works well.
\begin{verbatim}
#############################################
##   submit description file for a parallel program
#############################################
universe = vanilla
executable = cde_script.sh
request_cpus = 2
should_transfer_files = yes
when_to_transfer_output = on_exit
transfer_input_files = cde_my_mpi_linked_executable.tar
transfer_output_files = cde-package/cde-root/path/to/original/directory
queue
\end{verbatim}

We assume that you the user have created a tarball called 
\File{cde\_my\_mpi\_linked\_executable.tar} which contains the directory tree
created by CDE.
The contents of \Prog{cde\_script.sh} are:
\begin{verbatim}
#!/bin/sh
# Untar the CDE package
tar xpf cde_my_mpi_linked_executable.tar
# cd to the subdirectory where I need to run
cd cde-package/cde-root/path/to/original/directory
# Run my command
./mpirun.cde -n 2 ./my_mpi_linked_executable
# Normally, HTCondor will transfer the contents of this directory.
# However, we don't want the .cde command and the executable transfered back
# To prevent this from happening, we manually remove both files.
rm -f mpirun.cde
rm -f my_mpi_linked_executable
\end{verbatim}
