%%%%%%%%%%%%%%%%%%%%%%%%%%%%%%%%%%%%%%%%%%%%%%%%%%%%%%%%%%%%%%%%%%%%%%
\section{\label{sec:Parallel}Parallel Applications (Including MPI Applications)}
%%%%%%%%%%%%%%%%%%%%%%%%%%%%%%%%%%%%%%%%%%%%%%%%%%%%%%%%%%%%%%%%%%%%%%
\index{Parallel universe|(}

Condor's Parallel universe supports a wide variety of parallel
programming environments, and it encompasses the execution 
of MPI jobs.
It supports jobs which need to be co-scheduled.
A co-scheduled job has
more than one process that must be running at the same time on different
machines to work correctly.
The parallel universe supersedes the mpi universe.
The mpi universe eventually will be removed from Condor.


%%%%%%%%%%%%%%%%%%%%%%%%%%%%%%%%%%%%%%%%%%%%%%%%%%%%%%%%%%%%%%%%%%%
\subsection{\label{sec:parallel-setup}Prerequisites to Running Parallel Jobs}
%%%%%%%%%%%%%%%%%%%%%%%%%%%%%%%%%%%%%%%%%%%%%%%%%%%%%%%%%%%%%%%%%%%

Condor must be configured such that resources (machines) running
parallel jobs are dedicated.  
\index{scheduling!dedicated}
Note that \Term{dedicated} has a very specific meaning in Condor:
dedicated
machines never vacate their executing Condor jobs,
should the machine's interactive owner return.
This is implemented by running a single dedicated scheduler
process on a machine in the pool,
which becomes the single machine from which parallel universe
jobs are submitted.
Once the dedicated scheduler claims a
dedicated machine for use, 
the dedicated scheduler will try to use that machine to satisfy
the requirements of the queue of parallel universe or MPI universe jobs.
If the dedicated scheduler cannot use a machine for a
configurable amount of time, it will release its claim on the machine,
making it available again for the opportunistic scheduler.

Since Condor does not ordinarily run this way, (Condor usually uses
opportunistic scheduling), dedicated machines must be specially
configured.  Section~\ref{sec:Config-Dedicated-Jobs} of the
Administrator's Manual describes the necessary configuration and
provides detailed examples.

To simplify the scheduling of dedicated resources, a single machine
becomes the scheduler of dedicated resources.  This leads to a further
restriction that jobs submitted to execute under the parallel universe
must be submitted from the machine acting as the dedicated scheduler.

%%%%%%%%%%%%%%%%%%%%%%%%%%%%%%%%%%%%%%%%%%%%%%%%%%%%%%%%%%%%%%%%%%%
\subsection{\label{sec:parallel-submit}Parallel Job Submission}
%%%%%%%%%%%%%%%%%%%%%%%%%%%%%%%%%%%%%%%%%%%%%%%%%%%%%%%%%%%%%%%%%%%

Given correct configuration, parallel universe jobs may be submitted
from the machine running the dedicated scheduler.
The dedicated scheduler claims machines for the parallel universe job,
and invokes the job when the correct number of machines of the
correct platform (architecture and operating system) are claimed.
Note that the job likely consists of more than one process,
each to be executed on a separate machine.
The first process (machine) invoked is treated
different than the others.
When this first process exits, Condor shuts down all the others,
even if they have not yet completed their execution.

An overly simplified submit description file for a parallel universe
job appears as

\begin{verbatim}
#############################################
##   submit description file for a parallel program
#############################################
universe = parallel
executable = /bin/sleep
arguments = 30
machine_count = 8
queue 
\end{verbatim}

This job specifies the \SubmitCmd{universe} as \SubmitCmd{parallel}, letting
Condor know that dedicated resources are required.  The
\SubmitCmd{machine\_count} command identifies the number of machines
required by the job. 

When submitted, the dedicated scheduler allocates eight
machines with the same architecture and operating system as the submit
machine.  It waits until all eight machines are available before
starting the job.  When all the machines are ready, it invokes the
\Prog{/bin/sleep} command, with a command line argument of 30
on all eight machines more or less simultaneously.

A more realistic example of a parallel job utilizes other features.
\begin{verbatim}
######################################
## Parallel example submit description file
######################################
universe = parallel
executable = /bin/cat
log = logfile
input = infile.$(NODE)
output = outfile.$(NODE)
error = errfile.$(NODE)
machine_count = 4
queue
\end{verbatim}

The specification of the \SubmitCmd{input}, \SubmitCmd{output},
and \SubmitCmd{error} files utilize the predefined macro 
\MacroUNI{NODE}.
\index{macro!predefined}
See the \Condor{submit}
manual page on page~\pageref{man-condor-submit} for further
description of predefined macros.
The \MacroU{NODE} macro is given a
unique value as processes are assigned to machines.
The \MacroUNI{NODE} value is fixed for the entire length of the job.
It can therefore be used to identify individual aspects of the computation.
In this example, it is used to utilize and assign unique names to
input and output files.

This example presumes a shared file system across all the machines
claimed for the parallel universe job. 
Where no shared file system is either available or guaranteed,
use Condor's file transfer mechanism,
as described in section~\ref{sec:file-transfer}
on page~\pageref{sec:file-transfer}.
This example uses the file transfer mechanism.

\begin{verbatim}
######################################
## Parallel example submit description file
## without using a shared file system
######################################
universe = parallel
executable = /bin/cat
log = logfile
input = infile.$(NODE)
output = outfile.$(NODE)
error = errfile.$(NODE)
machine_count = 4
should_transfer_files = yes
when_to_transfer_output = on_exit
queue
\end{verbatim}

The job requires exactly four machines,
and queues four processes.
Each of these processes requires a correctly named input file,
and produces an output file.

%%%%%%%%%%%%%%%%%%%%%%%%%%%%%%%%%%%%%%%%%%%%%%%%%%%%%%%%%%%%%%%%%%%
\subsection{\label{sec:parallel-mpi-submit}Submitting MPI Jobs With the Parallel Universe}
%%%%%%%%%%%%%%%%%%%%%%%%%%%%%%%%%%%%%%%%%%%%%%%%%%%%%%%%%%%%%%%%%%%

To run MPI jobs within the parallel universe environment,
more framework is needed.
Condor provides this
framework in the form of user visible and modifiable scripts, to allow
flexibility for the different kinds of parallel systems it can support.
The Condor parallel universe works somewhat like a SIMD
(Single Instruction, Multiple Data) machine.
A single
executable is run on all the machines in parallel, while this single
executable may have different inputs and outputs.  
If different
executables must be executed on the different machines,
the submit description file specifies a script as
the executable.
This script needs to identify the machine it runs on,
utilizing the predifined macro, \MacroUNI{NODE},
in order to invoke the appropriate executable.

Most MPI implementations require two system-wide prerequisites.
The first prerequisite is the ability to run a command
on a remote machine without being prompted for a password.
Usually, \Prog{ssh} is used for this, but the
specific command may be configurable within the MPI implementation.
The second prerequisite is an ASCII file with the
list of machines that may utilize \Prog{ssh}.

So, to run MPI application in the parallel universe, we run a script
on each node we submit to.  This script generates ssh keys, to enable
password-less remote execution, start an sshd daemon, and send the
names and rank (node number) back to the submit directory.  Thus, for
each Condor job submitted, the scripts set up an ad-hoc MPI
environment, which is torn down at the end of the job run.  This ssh
script is a common requirement for running MPI jobs, so we have
factored it out into a common script, which is called from each of the
MPI-specific scripts.  After the ssh script has been started, the
MPI-specific script runs, starts the rest of the MPI job by looking at
its arguments, and waits for the MPI job to finish.  Condor provides
the ssh script, and example MPI scripts for both LAM and MPICH.  The
former is named ``lamscript'', and the latter ``mp1script''.  The
first argument to each script is the name of the real MPI executable,
and any subsequent arguments are arguments to that executable.  Other
implementations should be easy to add, by modifying the given
examples.  Note that because the actual MPI executable (i.e. the
output of mpicc) is not the named executable in the submit script, it
must be accessible either via a network file system, or by condor file
transfer.

The sshd.sh script requires several configuration file settings.
\Macro{CONDOR\_SSHD} should be an absolute path to an implementation of
sshd.  sshd.sh has been tested with openssh version 3.9, but should
work with more recent versions.  \Macro{CONDOR\_SSH\_KEYGEN} should
point to the corresponding ssh-keygen executable.

The LAM and MPICH scripts each have their own idiosyncrasies.  In the
mp1script, the PATH to the MPICH installation must be set.  Look
for the shell variable MPDIR, and set it to the proper value.  This
directory should contain the MPICH mpirun command.

For LAM, there is a similar path setting, but called LAMDIR in the
lamscript shell script.  In addition, this path must be part of the
path set in the user's .cshrc script.  (As of this writing, lam doesn't
work if the user's login shell is the Bourne or compatible shell).

\begin{verbatim}
######################################
## Example submit description file
## for MPICH 1 MPI
## works with MPICH 1.2.4, 1.2.5 and 1.2.6
######################################
universe = parallel
executable = mp1script
arguments = my_mpich_linked_executable arg1 arg2
machine_count = 4
should_transfer_files = yes
when_to_transfer_output = on_exit
transfer_input_files = my_mpich_linked_executable
queue
\end{verbatim}

\begin{verbatim}
######################################
## Example submit description file
## for LAM MPI
######################################
universe = parallel
executable = lamscript
arguments = my_lam_linked_executable arg1 arg2
machine_count = 4
should_transfer_files = yes
when_to_transfer_output = on_exit
transfer_input_files = my_lam_linked_executable
queue
\end{verbatim}

\subsection{\label{sec:parallel-multi-proc}Submitting parallel jobs with multiple requirements}
Different nodes for a parallel job can have different machine
requirements.  For example, often the first node, sometimes called the
head node, needs to run on a specific machine.  This can be also
useful for debugging.  Condor accommodates this by supporting multiple
\Attr{queue} statements in the submit file, much like with the other
universes.  For example:

\begin{verbatim}
######################################
## Example submit description file
## with multiple procs
######################################
universe = parallel
executable = example
machine_count = 1
requirements = ( machine == "machine1")
queue

requirements = ( machine =!= "machine1")
machine_count = 3
queue
\end{verbatim}

The dedicated scheduler will allocate four machines (nodes) total across
two procs for this job.  The first proc has one node,
 and will run on the machine named machine1.  The 
other three nodes, in the second proc, will run on other machines.  
Like in the other condor universes, the second requirements command 
overwrites the first, but the other commands are inherited from the 
first proc.

When submitting jobs with multiple requirements, it is
best to write the requirements to be mutually exclusive,
or to have the most selective requirement first in the submit file.
This is because the scheduler tries to match jobs to machine in
submit file order.  If the requirements are not mutually exclusive,
it can happen that the scheduler may unable to schedule the job, even
if all needed resources are available.

%%%%%%%%%%%%%%%%%%%%%%%%%%%%%%%%%%%%%%%%%%%%%%%%%%%%%%%%%%%%%%%%%%%%%%
%%%%%%%%%%%%%%%%%%%%%%%%%%%%%%%%%%%%%%%%%%%%%%%%%%%%%%%%%%%%%%%%%%%%%%
\section{\label{sec:MPI}MPI Applications}
%%%%%%%%%%%%%%%%%%%%%%%%%%%%%%%%%%%%%%%%%%%%%%%%%%%%%%%%%%%%%%%%%%%%%%
\index{MPI|(}
MPI stands for Message Passing Interface.
It provides an environment under which parallel programs
may synchronize, 
by providing communication support.
Running the MPI-based parallel programs within Condor 
eases the programmer's effort.
Condor dedicates machines for running the programs,
and it does so using the same interface used when submitting
non-MPI jobs.

Condor currently supports MPICH versions 122, 123, and 124
using the ch\_p4 device. 
Condor does not support MPICH version 125.
These supported implementations are
offered by Argonne National Labs
without charge by download.
See the web page at
\URL{http://www-unix.mcs.anl.gov/mpi/mpich/}
for details and availability.
Programs to be submitted for execution under Condor will have
been compiled using \Prog{mpicc}.
No further compilation or linking is necessary to run jobs
under Condor.

%%%%%%%%%%%%%%%%%%%%%%%%%%%%%%%%%%%%%%%%%%%%%%%%%%%%%%%%%%%%%%%%%%%
\subsection{\label{sec:MPI-setup}MPI Details of Set Up}

Administratively, Condor must be configured such that resources
(machines) running MPI jobs are dedicated.
\index{scheduling!dedicated}
Dedicated machines are ones that, once they begin execution of
a program, will continue executing the program until
the program ends.
The program will not be preempted (to run another program) or
suspended.
Since Condor is not ordinarily used in this manner (Condor uses
opportunistic scheduling),
machines that are to be used as dedicated resources
must be configured as such.
Section~\ref{sec:Config-Dedicated-Jobs} of
Administrator's Manual describes the necessary
configuration and provides detailed examples.

To simplify the dedicated scheduling of resources,
a single machine becomes the scheduler of dedicated resources.
This leads to a further restriction that jobs submitted
to execute under the MPI universe (with dedicated machines)
must be
submitted from the machine running as the dedicated scheduler.

%%%%%%%%%%%%%%%%%%%%%%%%%%%%%%%%%%%%%%%%%%%%%%%%%%%%%%%%%%%%%%%%%%%
\subsection{\label{sec:MPI-submit}MPI Job Submission}

Once the programs are written and compiled, and Condor resources
are correctly configured, jobs may be submitted.
Each Condor job requires a submit description file.
The simplest submit description file for an MPI job:

\begin{verbatim}
#############################################
##   submit description file for mpi_program
#############################################
universe = MPI
executable = mpi_program
machine_count = 4
queue 
\end{verbatim}

This job specifies the \Attr{universe} as \Attr{mpi},
letting Condor know that dedicated resources will be required.
The \Attr{machine\_count} command identifies the number
of machines required by the job.
The four machines that run the program will default to
be of the same architecture
and operating system as the machine on which the job is submitted,
since a platform is not specified as a requirement.

The simplest example does not specify an input or output,
meaning that the computation completed is useless,
since both input comes from and the output goes to \File{/dev/null}.
A more complex example of a submit description file
utilizes other features.
\begin{verbatim}
######################################
## MPI example submit description file
######################################
universe = MPI
executable = simplempi
log = logfile
input = infile.$(NODE)
output = outfile.$(NODE)
error = errfile.$(NODE)
machine_count = 4
queue
\end{verbatim}

The specification of the input, output, and error files
utilize a predefined macro that is only relevant to
mpi universe jobs.
\index{macro!predefined}
See the \Condor{submit} manual page on
page~\pageref{man-condor-submit} 
for further description of predefined macros.
The \MacroU{NODE} macro is given a unique value as
programs are assigned to machines.
This value is what the MPICH version ch\_p4 implementation
terms the rank of a program.
Note that this term is unrelated and independent of the
Condor term rank.
The \MacroUNI{NODE} value is fixed for the entire length
of the job.
It can therefore be used to identify individual aspects
of the computation.
In this example, it is used to give unique names to input
and output files.

If your site does NOT have a shared filesystem across all the nodes
where your MPI computation will execute, you can use Condor's file
transfer mechanism.
You can find out more details about these settings by reading the
\Condor{submit} man page or section~\ref{sec:file-transfer} on
page~\pageref{sec:file-transfer}. 
Assuming your job only reads input from STDIN, here is an example
submit file for a site without a shared filesystem:

\begin{verbatim}
######################################
## MPI example submit description file
## without using a shared filesystem
######################################
universe = MPI
executable = simplempi
log = logfile
input = infile.$(NODE)
output = outfile.$(NODE)
error = errfile.$(NODE)
machine_count = 4
should_transfer_files = yes
when_to_transfer_output = on_exit
queue
\end{verbatim}

Consider the following C program that uses this example submit
description file.

\begin{verbatim}
/**************
 * simplempi.c
 **************/
#include <stdio.h>
#include "mpi.h"

int main(argc,argv)
    int argc;
    char *argv[];
{
    int myid;
    char line[128];

    MPI_Init(&argc,&argv);
    MPI_Comm_rank(MPI_COMM_WORLD,&myid);

    fprintf ( stdout, "Printing to stdout...%d\n", myid );
    fprintf ( stderr, "Printing to stderr...%d\n", myid );
    fgets ( line, 128, stdin );
    fprintf ( stdout, "From stdin: %s", line );

    MPI_Finalize();
    return 0;
}
\end{verbatim}

Here is a makefile that works with the example.
It would build the MPI executable, using the MPICH
version ch\_p4 implementation.
\begin{verbatim}
###################################################################
## This is a very basic Makefile                                 ##
###################################################################

# the location of the MPICH compiler
CC          = /usr/local/bin/mpicc
CLINKER     = $(CC)

CFLAGS    = -g
EXECS     = simplempi

all: $(EXECS)

simplempi: simplempi.o
        $(CLINKER) -o simplempi simplempi.o -lm

.c.o:
        $(CC) $(CFLAGS) -c $*.c
\end{verbatim}

The submission to Condor requires exactly four machines,
and queues four programs.
Each of these programs requires an input file (correctly
named) and produces an output file.

If input file for \MacroUNI{NODE} = 0 (called \File{infile.0}) contains
\begin{verbatim}
Hello number zero.
\end{verbatim}
and
the input file for \MacroUNI{NODE} = 1 (called \File{infile.1}) contains
\begin{verbatim}
Hello number one.
\end{verbatim}
then after the job is submitted to Condor,
there will be 
eight files created:  
\File{errfile.[0-3]} and \File{outfile.[0-3]}.
\File{outfile.0} will contain
\begin{verbatim}
Printing to stdout...0
From stdin: Hello number zero.
\end{verbatim}
and \File{errfile.0} will contain
\begin{verbatim}
Printing to stderr...0
\end{verbatim}

\index{MPI|)}

%%%%%%%%%%%%%%%%%%%%%%%%%%%%%%%%%%%%%%%%%%%%%%%%%%%%%%%%%%%%%%%%%%%%%%

\index{Parallel universe|)}
