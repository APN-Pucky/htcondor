%%%%%%%%%%%%%%%%%%%%%%%%%%%%%%%%%%%%%%%%%%%%%%%%%%%%%%%%%%%%%%%%%%%%%%
\section{\label{sec:Parallel}Parallel Applications (Including MPI Applications)}
%%%%%%%%%%%%%%%%%%%%%%%%%%%%%%%%%%%%%%%%%%%%%%%%%%%%%%%%%%%%%%%%%%%%%%
\index{parallel universe|(}
\index{MPI application}

Condor's Parallel universe supports a wide variety of parallel
programming environments, and it encompasses the execution 
of MPI jobs.
It supports jobs which need to be co-scheduled.
A co-scheduled job has
more than one process that must be running at the same time on different
machines to work correctly.
The parallel universe supersedes the mpi universe.
The mpi universe eventually will be removed from Condor.


%%%%%%%%%%%%%%%%%%%%%%%%%%%%%%%%%%%%%%%%%%%%%%%%%%%%%%%%%%%%%%%%%%%
\subsection{\label{sec:parallel-setup}Prerequisites to Running Parallel Jobs}
%%%%%%%%%%%%%%%%%%%%%%%%%%%%%%%%%%%%%%%%%%%%%%%%%%%%%%%%%%%%%%%%%%%

Condor must be configured such that resources (machines) running
parallel jobs are dedicated.  
\index{scheduling!dedicated}
Note that \Term{dedicated} has a very specific meaning in Condor:
dedicated
machines never vacate their executing Condor jobs,
should the machine's interactive owner return.
This is implemented by running a single dedicated scheduler
process on a machine in the pool,
which becomes the single machine from which parallel universe
jobs are submitted.
Once the dedicated scheduler claims a
dedicated machine for use, 
the dedicated scheduler will try to use that machine to satisfy
the requirements of the queue of parallel universe or MPI universe jobs.
If the dedicated scheduler cannot use a machine for a
configurable amount of time, it will release its claim on the machine,
making it available again for the opportunistic scheduler.

Since Condor does not ordinarily run this way, (Condor usually uses
opportunistic scheduling), dedicated machines must be specially
configured.  Section~\ref{sec:Config-Dedicated-Jobs} of the
Administrator's Manual describes the necessary configuration and
provides detailed examples.

To simplify the scheduling of dedicated resources, a single machine
becomes the scheduler of dedicated resources.  This leads to a further
restriction that jobs submitted to execute under the parallel universe
must be submitted from the machine acting as the dedicated scheduler.

%%%%%%%%%%%%%%%%%%%%%%%%%%%%%%%%%%%%%%%%%%%%%%%%%%%%%%%%%%%%%%%%%%%
\subsection{\label{sec:parallel-submit}Parallel Job Submission}
%%%%%%%%%%%%%%%%%%%%%%%%%%%%%%%%%%%%%%%%%%%%%%%%%%%%%%%%%%%%%%%%%%%

Given correct configuration, parallel universe jobs may be submitted
from the machine running the dedicated scheduler.
The dedicated scheduler claims machines for the parallel universe job,
and invokes the job when the correct number of machines of the
correct platform (architecture and operating system) are claimed.
Note that the job likely consists of more than one process,
each to be executed on a separate machine.
The first process (machine) invoked is treated
different than the others.
When this first process exits, Condor shuts down all the others,
even if they have not yet completed their execution.

An overly simplified submit description file for a parallel universe
job appears as

\begin{verbatim}
#############################################
##   submit description file for a parallel program
#############################################
universe = parallel
executable = /bin/sleep
arguments = 30
machine_count = 8
queue 
\end{verbatim}

This job specifies the \SubmitCmd{universe} as \SubmitCmd{parallel}, letting
Condor know that dedicated resources are required.  The
\SubmitCmd{machine\_count} command identifies the number of machines
required by the job. 

When submitted, the dedicated scheduler allocates eight
machines with the same architecture and operating system as the submit
machine.  It waits until all eight machines are available before
starting the job.  When all the machines are ready, it invokes the
\Prog{/bin/sleep} command, with a command line argument of 30
on all eight machines more or less simultaneously.

A more realistic example of a parallel job utilizes other features.
\begin{verbatim}
######################################
## Parallel example submit description file
######################################
universe = parallel
executable = /bin/cat
log = logfile
input = infile.$(NODE)
output = outfile.$(NODE)
error = errfile.$(NODE)
machine_count = 4
queue
\end{verbatim}

The specification of the \SubmitCmd{input}, \SubmitCmd{output},
and \SubmitCmd{error} files utilize the predefined macro 
\MacroUNI{NODE}.
\index{macro!predefined}
See the \Condor{submit}
manual page on page~\pageref{man-condor-submit} for further
description of predefined macros.
The \MacroU{NODE} macro is given a
unique value as processes are assigned to machines.
The \MacroUNI{NODE} value is fixed for the entire length of the job.
It can therefore be used to identify individual aspects of the computation.
In this example, it is used to utilize and assign unique names to
input and output files.

This example presumes a shared file system across all the machines
claimed for the parallel universe job. 
Where no shared file system is either available or guaranteed,
use Condor's file transfer mechanism,
as described in section~\ref{sec:file-transfer}
on page~\pageref{sec:file-transfer}.
This example uses the file transfer mechanism.

\begin{verbatim}
######################################
## Parallel example submit description file
## without using a shared file system
######################################
universe = parallel
executable = /bin/cat
log = logfile
input = infile.$(NODE)
output = outfile.$(NODE)
error = errfile.$(NODE)
machine_count = 4
should_transfer_files = yes
when_to_transfer_output = on_exit
queue
\end{verbatim}

The job requires exactly four machines,
and queues four processes.
Each of these processes requires a correctly named input file,
and produces an output file.

%%%%%%%%%%%%%%%%%%%%%%%%%%%%%%%%%%%%%%%%%%%%%%%%%%%%%%%%%%%%%%%%%%%
\subsection{\label{sec:parallel-multi-proc}Parallel Jobs with Separate Requirements}
%%%%%%%%%%%%%%%%%%%%%%%%%%%%%%%%%%%%%%%%%%%%%%%%%%%%%%%%%%%%%%%%%%%

The different machines executing for a parallel universe job
may specify different machine requirements.
A common example requires that the
head node execute on a specific machine.
It may be also useful for debugging purposes.

Consider the following example.

\begin{verbatim}
######################################
## Example submit description file
## with multiple procs
######################################
universe = parallel
executable = example
machine_count = 1
requirements = ( machine == "machine1")
queue

requirements = ( machine =!= "machine1")
machine_count = 3
queue
\end{verbatim}

The dedicated scheduler allocates four machines.
All four executing jobs have the same value for \MacroUNI{Cluster}
macro.
The \MacroUNI{Process} macro takes on two values;
the value 0 will be assigned for the single executable
that must be executed on machine1, and
the value 1 will be assigned for the other three 
that must be executed anywhere but on machine1.

Carefully consider the ordering and nature of multiple
sets of requirements in the same submit description file.
The scheduler matches jobs to machines based on the ordering
within the submit description file.
Mutually exclusive requirements eliminate the dependence on
ordering within the submit description file.
Without mutually exclusive requirements,
the scheduler may be unable to schedule the job.
The ordering within the submit description file may preclude
the scheduler considering the specific allocation that
could satisfy the requirements.

%%%%%%%%%%%%%%%%%%%%%%%%%%%%%%%%%%%%%%%%%%%%%%%%%%%%%%%%%%%%%%%%%%%
\subsection{\label{sec:parallel-mpi-submit}MPI Applications Within Condor's Parallel Universe}
%%%%%%%%%%%%%%%%%%%%%%%%%%%%%%%%%%%%%%%%%%%%%%%%%%%%%%%%%%%%%%%%%%%
\index{parallel universe!running MPI applications}
\index{MPI application}

MPI applications utilize a single executable that is invoked in order to
execute in parallel on one or more machines. 
Condor's parallel universe provides the environment within
which this executable is executed in parallel.
However, the various implementations of MPI
(for example, LAM or MPICH) require further framework items within
a system-wide environment.
Condor supports this necessary framework through 
user visible and modifiable scripts.
An MPI implementation-dependent script becomes the Condor job.
The script sets up the extra, necessary framework,
and then invokes the MPI application's executable.

Condor provides these scripts in the
\File{\MacroUNI{RELEASE\_DIR}/etc/examples}
directory.
The script for the LAM implementation is \File{lamscript}.
The script for the MPICH implementation is \File{mp1script}.
Therefore, a Condor submit description file for these
implementations would appear similar to:

\begin{verbatim}
######################################
## Example submit description file
## for MPICH 1 MPI
## works with MPICH 1.2.4, 1.2.5 and 1.2.6
######################################
universe = parallel
executable = mp1script
arguments = my_mpich_linked_executable arg1 arg2
machine_count = 4
should_transfer_files = yes
when_to_transfer_output = on_exit
transfer_input_files = my_mpich_linked_executable
queue
\end{verbatim}

or

\begin{verbatim}
######################################
## Example submit description file
## for LAM MPI
######################################
universe = parallel
executable = lamscript
arguments = my_lam_linked_executable arg1 arg2
machine_count = 4
should_transfer_files = yes
when_to_transfer_output = on_exit
transfer_input_files = my_lam_linked_executable
queue
\end{verbatim}

The \SubmitCmd{executable} is the MPI implementation-dependent script.
The first argument to the script is the MPI application's 
executable.
Further arguments to the script are the MPI application's arguments.
Condor must transfer this executable;
do this with the \SubmitCmd{transfer\_input\_files} command.

For other implementations of MPI,
copy and modify one of the given scripts.
Most MPI implementations require two system-wide prerequisites.
The first prerequisite is the ability to run a command
on a remote machine without being prompted for a password.
\Prog{ssh} is commonly used, but other
commands may be used.
The second prerequisite is an ASCII file containing the
list of machines that may utilize \Prog{ssh}.
These common prerequisites are implemented in a further script
called \File{sshd.sh}.
\File{sshd.sh} generates ssh keys 
(to enable password-less remote execution),
and starts an \Prog{sshd} daemon.
The machine name and MPI rank are given to the submit machine.

%So, to run MPI application in the parallel universe, we run a script
%on each node we submit to.  This script generates ssh keys, to enable
%password-less remote execution, start an sshd daemon, and send the
%names and rank (node number) back to the submit directory.  Thus, for
%each Condor job submitted, the scripts set up an ad-hoc MPI
%environment, which is torn down at the end of the job run.  This ssh
%script is a common requirement for running MPI jobs, so we have
%factored it out into a common script, which is called from each of the
%MPI-specific scripts.  After the ssh script has been started, the
%MPI-specific script runs, starts the rest of the MPI job by looking at
%its arguments, and waits for the MPI job to finish.  Condor provides
%the ssh script, and example MPI scripts for both LAM and MPICH.  The
%former is named ``lamscript'', and the latter ``mp1script''.  The
%first argument to each script is the name of the real MPI executable,
%and any subsequent arguments are arguments to that executable.  Other
%implementations should be easy to add, by modifying the given
%examples.  Note that because the actual MPI executable (i.e. the
%output of mpicc) is not the named executable in the submit script, it
%must be accessible either via a network file system, or by condor file
%transfer.

The \Prog{sshd.sh} script requires the definition of
two Condor configuration variables.
Configuration variable \Macro{CONDOR\_SSHD} is an absolute path to
an implementation of \Prog{sshd}.
\Prog{sshd.sh} has been tested with \Prog{openssh} version 3.9,
but should work with more recent versions.
Configuration variable \Macro{CONDOR\_SSH\_KEYGEN} points
to the corresponding \Prog{ssh-keygen} executable.

Scripts \Prog{lamscript} and \Prog{mp1script}
each have their own idiosyncrasies.
In \Prog{mp1script}, the \Env{PATH} to the MPICH installation must be set.
The shell variable MPDIR indicates its proper value.
This directory contains the MPICH \Prog{mpirun} executable.
For LAM, there is a similar path setting, but it is called \Env{LAMDIR}
in the \Prog{lamscript} script.  In addition, this path must be part of the
path set in the user's \File{.cshrc} script.
As of this writing, the LAM implementation does not work
if the user's login shell is the Bourne or compatible shell.

These MPI jobs operate as all parallel universe jobs do.
The default policy is that when the first node exits,
the whole job is considered done, 
and Condor kills all other running nodes in that parallel job.
Alternatively, a parallel universe job that sets the attribute
\begin{verbatim}
+ParallelShutdownPolicy = "WAIT_FOR_ALL"
\end{verbatim}
in its submit description file changes the policy,
such that Condor will wait until every node in the parallel 
job has completed to consider the job finished. 

\index{parallel universe|)}
