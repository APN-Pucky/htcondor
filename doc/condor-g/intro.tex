%%%%%%%%%%%%%%%%%%%%%%%%%%%%%%%%%%%%%%%%%%%%%%%%%%%%%%%%%%%%%%%%%%%%%%%%%%%
\section{\label{sec:Condor-G-Intro}Condor-G Introduction}
%%%%%%%%%%%%%%%%%%%%%%%%%%%%%%%%%%%%%%%%%%%%%%%%%%%%%%%%%%%%%%%%%%%%%%%%%%%

Condor-G is a ``window to the grid.''
The function of Condor-G becomes clear
with a brief overview of the software that forms a Condor pool.
For this discussion, the software of a Condor pool is divided
into two parts.
The first part does job management.
It keeps track of a user's jobs.
You can ask the job management part of Condor to show
you the job queue, to submit new jobs to the system,
to put jobs on hold,
and to request information about jobs that have completed.
The other part of the Condor software
does resource management.
It keeps track of which machines are available to run jobs,
how the available machines should be utilized given all the users
who want to run jobs on them,
and when a machine is no longer available.
works with grid resources, allowing users to
effectively submit jobs, manage jobs, and have jobs execute
on widely distributed machines.

A machine with the job management part installed
is called a submit machine.
A machine with the resource management part installed 
is called an execute machine.
Each machine may have one part or both.
Condor-G is the job management part of Condor.
\index{Condor-G!functionality}
Condor-G lets you submit jobs into a queue,
have a log detailing the life cycle of your jobs,
manage your input and output files,
along with everything else you expect from a job queuing system.

Condor uses 
Globus to provide underlying software needed to utilize
grid resources, such as authentication, remote program
execution and data transfer.
Condor's capabilities when executing jobs on Globus resources have
significantly increased.
The same Condor tools that access local resources 
are now able to use the Globus protocols to access resources at multiple
sites. 

Condor-G is a program that manages both a queue of jobs
and the resources from one or more sites where those jobs can execute. 
It communicates with these resources and transfers files
to and from these resources using Globus mechanisms.
(In particular, Condor-G uses the GRAM protocol for job submission,
and it runs a local GASS server for file transfers).

It may appear that Condor-G is a simple replacement
for the Globus toolkit's \Prog{globusrun} command.
However, Condor-G does much more.
It allows you to submit many jobs at once
and then to monitor those jobs with a convenient interface,
receive notification when jobs complete or fail,
and maintain your Globus credentials
which may expire while a job is running.
On top of this, Condor-G is a fault-tolerant system;
if your machine crashes,
you can still perform all of these functions when your machine returns to life.


%%%%%%%%%%%%%%%%%%%%%%%%%%%%%%%%%%%%%%%%%%%%%%%%%%%%%%%%%%%%%%%%%%%%%%%%%%%
\section{\label{sec:Globus-intro}Working with Globus}
%%%%%%%%%%%%%%%%%%%%%%%%%%%%%%%%%%%%%%%%%%%%%%%%%%%%%%%%%%%%%%%%%%%%%%%%%%%

The Globus software provides a well-defined set of protocols
that allow authentication, data transfer, and remote job submission.

Authentication is a mechanism by which an identity is verified.
Given proper authentication, authorization to use a resource
is required.
Authorization is a policy that determines who is allowed to do what. 


%%%%%%%%%%%%%%%%%%%%%%%%%%%%%%%%%%%%%%%%%%%%%%%%%%%%%%%%%%%%%%%%%%%%%%%%%%%
\subsection{\label{sec:Globus-Protocols}Globus Protocols}
%%%%%%%%%%%%%%%%%%%%%%%%%%%%%%%%%%%%%%%%%%%%%%%%%%%%%%%%%%%%%%%%%%%%%%%%%%%
Condor uses the following Globus protocols.
These protocols allow Condor to utilize grid machines for
the execution of jobs.
\begin{description}
\item[GSI]
The Globus Toolkit's Grid Security Infrastructure (GSI) provides essential
\index{Condor-G!GSI}
building blocks for other Grid protocols and Condor-G.
This authentication and authorization system
makes it possible to authenticate a user just once,
using public key infrastructure (PKI) mechanisms to verify
a user-supplied grid credential.
GSI then handles the mapping of the grid credential to the
diverse local credentials and authentication/authorization mechanisms that
apply at each site. 
\item[GRAM]
The Grid Resource Allocation and Management (GRAM) protocol supports remote
\index{Condor-G!GRAM}
submission of a computational request (for example, to run program P)
to a remote computational resource,
and it supports subsequent monitoring and control of the resulting
computation. 
\item[GASS]
The Globus Toolkit's Global Access to Secondary Storage (GASS) service provides
\index{Condor-G!GASS}
mechanisms for transferring data between a remote HTTP, FTP, or GASS server. 
Condor-G uses GASS to transfer the executable, stdin, stdout, and stderr
between the submission local and the remote resource.
\end{description}

