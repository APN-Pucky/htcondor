%%%%%%%%%%%%%%%%%%%%%%%%%%%%%%%%%%%%%%%%%%%%%%%%%%%%%%%%%%%%%%%%%%%%%%
\section{\label{sec:platform-linux}Linux}
\index{platform-specific information!Linux}
%%%%%%%%%%%%%%%%%%%%%%%%%%%%%%%%%%%%%%%%%%%%%%%%%%%%%%%%%%%%%%%%%%%%%%

This section provides information specific to the Linux port of
HTCondor.
Linux is a difficult platform to support.
It changes very frequently, and HTCondor has some extremely
system-dependent code (for example, the checkpointing library).

HTCondor is sensitive to changes in the following elements of the
system: 
\begin{itemize}
\item The kernel version
\item The version of the GNU C library (glibc)
\item the version of GNU C Compiler (GCC) used to build and link
  HTCondor jobs (this only matters for HTCondor's Standard universe which
  provides checkpointing and remote system calls)
\end{itemize}

The HTCondor Team tries to provide support for various releases of the
distribution of Linux.
Red Hat is probably the most popular Linux distribution, and it
provides a common set of versions for the above system components
at which HTCondor can aim support.
HTCondor will often work with Linux distributions other than Red Hat (for
example, Debian or SuSE) that have the same versions of the above
components.
However, we do not usually test HTCondor on other Linux distributions
and we do not provide any guarantees about this.

New releases of Red Hat usually change the versions of some or all of
the above system-level components.
A version of HTCondor that works with one release of Red Hat might not
work with newer releases.
The following sections describe the details of HTCondor's support for
the currently available versions of Red Hat Linux on x86 architecture
machines.

%%%%%%%%%%%%%%%%%%%%%%%%%%%%%%%%%%%%%%%%%%%%%%%%%%%%%%%%%%%%%%%%%%%%%%
\subsection{\label{sec:platform-linux-activity}Linux Kernel-specific Information}
%%%%%%%%%%%%%%%%%%%%%%%%%%%%%%%%%%%%%%%%%%%%%%%%%%%%%%%%%%%%%%%%%%%%%%
\index{platform-specific information!Linux keyboard and mouse activity}
\index{Linux!keyboard and mouse activity}

Distributions that rely on the Linux 2.4.x and all Linux 2.6.x kernels
through version 2.6.10
do not modify the \Code{atime} of the input device file.
This leads to difficulty when HTCondor is run using one of these
kernels. 
The problem manifests itself in that HTCondor cannot properly
detect keyboard or mouse activity.
Therefore, using the activity in policy setting cannot
signal that HTCondor should stop running a job on a machine.

HTCondor version 6.6.8 implements a workaround for PS/2 devices.
A better fix is the Linux 2.6.10 kernel
patch linked to from the directions posted at
\URL{http://www.cs.wisc.edu/condor/kernel.patch.html}.
This patch works better for PS/2 devices, and
may also work for USB devices.
A future version of HTCondor will implement better recognition
of USB devices,
such that the kernel patch will also definitively work for USB devices.

HTCondor additionally has problems running on some older Xen kernels,
which interact badly with assumptions made by the \Condor{procd}
daemon. See the FAQ entry in section~\ref{sec:xen-jiffies-bug} for
details.

%%%%%%%%%%%%%%%%%%%%%%%%%%%%%%%%%%%%%%%%%%%%%%%%%%%%%%%%%%%%%%%%%%%%%%
\subsection{\label{sec:platform-linux-addrspace-random}Address Space Randomization}
%%%%%%%%%%%%%%%%%%%%%%%%%%%%%%%%%%%%%%%%%%%%%%%%%%%%%%%%%%%%%%%%%%%%%%
\index{platform-specific information!address space randomization}

Modern versions of Red Hat and Fedora do address space randomization,
which randomizes the memory layout of a process
to reduce the possibility of security exploits. 
This makes it impossible
for standard universe jobs to resume execution using a checkpoint.
When starting or resuming a standard universe job,
HTCondor disables the randomization.

To run a binary compiled with  \Condor{compile} in standalone mode,
either initially or in resumption mode,
manually disable the address space randomization by modifying the
command line.
For a 32-bit architecture, assuming an
HTCondor-linked binary called \Prog{myapp},
invoke the standalone executable with:
\begin{verbatim}
  setarch i386 -L -R ./myapp
\end{verbatim}
For a 64-bit architecture, the resumption command will be: 
\begin{verbatim}
  setarch x86_64 -L -R ./myapp
\end{verbatim}
Some applications will also need the \Opt{-B} option.

The command to resume execution using the checkpoint must also
disable address space randomization, 
as the 32-bit architecture example:
\begin{verbatim}
  setarch i386 -L -R myapp -_condor_restart myapp.ckpt
\end{verbatim}


% TODO Need section talking about standard universe, CheckpointPlatform
%   and incompatible process memory layouts.
