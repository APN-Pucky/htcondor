%%%%%%%%%%%%%%%%%%%%%%%%%%%%%%%%%%%%%%%%%%%%%%%%%%%%%%%%%%%%%%%%%%%%%%
\section{\label{sec:platform-linux}Linux}
\index{Platforms !Linux}
%%%%%%%%%%%%%%%%%%%%%%%%%%%%%%%%%%%%%%%%%%%%%%%%%%%%%%%%%%%%%%%%%%%%%%

This section provides information specific to the Linux port of
Condor.
Linux is a difficult platform to support.
It changes very frequently, and Condor has some extremely
system-dependent code (for example, the checkpointing library).

Condor is sensitive to changes in the following elements of the
system: 
\begin{itemize}
\item The kernel version
\item The version of the GNU C library (glibc)
\item the version of GNU C Compiler (GCC) used to build and link
  Condor jobs (this only matters for Condor's Standard universe which
  provides checkpointing and remote system calls)
\end{itemize}

The Condor Team tries to provide support for various releases of the
RedHat distribution of Linux.
RedHat is probably the most popular Linux distribution, and it
provides a common set of versions for the above system components
which Condor can aim to support.
Condor will often work with Linux distributions other than RedHat (for
example, Debian or SuSE) that have the same versions of the above
components.
However, we do not usually test Condor on other Linux distributions
and we do not provide any guarantees about this.

New releases of RedHat usually change the versions of some or all of
the above system-level components.
A version of Condor that works with one release of RedHat might not
work with newer releases.
The following sections describe the details of Condor's support for
the currently available versions of RedHat Linux on x86 architecture
machines.


%%%%%%%%%%%%%%%%%%%%%%%%%%%%%%%%%%%%%%%%%%%%%%%%%%%%%%%%%%%%%%%%%%%%%%
\subsection{\label{sec:platform-linux-rh6}RedHat Version 6.x}
%%%%%%%%%%%%%%%%%%%%%%%%%%%%%%%%%%%%%%%%%%%%%%%%%%%%%%%%%%%%%%%%%%%%%%
\index{Platforms !RedHat 6.x}

RedHat version 6.x is an older release of RedHat, but it is still used
by a number of sites.
Since Condor has worked on this platform since it was first released,
we still provide binaries that support it.
RedHat 6.2 uses the 2.2.x Linux kernel series, glibc version 2.1.x,
and GCC version egcs-1.1.2.
To run Condor on this platform and be able to use \Condor{compile},
you should download the Condor \VersionNotice  binaries listed for use
with ``Linux 2.0.x and 2.2.x (glibc 2.1)''.


%%%%%%%%%%%%%%%%%%%%%%%%%%%%%%%%%%%%%%%%%%%%%%%%%%%%%%%%%%%%%%%%%%%%%%
\subsection{\label{sec:platform-linux-rh7}RedHat Version 7.x}
%%%%%%%%%%%%%%%%%%%%%%%%%%%%%%%%%%%%%%%%%%%%%%%%%%%%%%%%%%%%%%%%%%%%%%
\index{Platforms !RedHat 7.x}

RedHat version 7.x is fully supported in Condor \VersionNotice.
\Condor{compile} works to link user jobs for the Standard universe
with the versions of GCC and glibc that come with RedHat 7.x.
Both the statically linked and dynamically linked versions of the
Condor binaries listed for use with ``Linux 2.4.x (glibc 2.2) - RedHat
7.1, 7.2, 7.3'' will work with no additional effort.


%%%%%%%%%%%%%%%%%%%%%%%%%%%%%%%%%%%%%%%%%%%%%%%%%%%%%%%%%%%%%%%%%%%%%%
\subsection{\label{sec:platform-linux-rh8}RedHat Version 8.x}
%%%%%%%%%%%%%%%%%%%%%%%%%%%%%%%%%%%%%%%%%%%%%%%%%%%%%%%%%%%%%%%%%%%%%%
\index{Platforms !RedHat 8.x}

RedHat version 8.x is not officially supported in Condor
\VersionNotice.
We have had reports that the Condor binaries built for RedHat 7.x
(either the statically or dynamically linked versions) work with
RedHat 8.x. 

Unfortunately, \Condor{compile} does not yet work with GCC (gcc, g++
or g77) version 3.0 or higher.
RedHat 8.0 uses GCC version 3.2.
So, if you wish to run Standard universe jobs, you can not link them
with Condor's checkpointing libraries on a RedHat 8.x machine.
However, if you have a RedHat 7.x machine, you can run
\Condor{compile} there, and the resulting binary can be submitted to
run on RedHat 8.x machines.
We are planning to provide full support for RedHat 8.x as soon as
possible, but in the meantime, this is the best method for running
Standard universe jobs.
If you do not use \Condor{compile} and the Standard universe, the
RedHat 7.x binaries will probably work as-is for your site. 


%%%%%%%%%%%%%%%%%%%%%%%%%%%%%%%%%%%%%%%%%%%%%%%%%%%%%%%%%%%%%%%%%%%%%%
\subsection{\label{sec:platform-linux-rh9}RedHat Version 9.x}
%%%%%%%%%%%%%%%%%%%%%%%%%%%%%%%%%%%%%%%%%%%%%%%%%%%%%%%%%%%%%%%%%%%%%%
\index{Platforms !RedHat 9.x}

Condor \VersionNotice  does not yet work with RedHat version 9.x at
all.
The statically linked Condor binaries for RedHat 7.x crash right away
when they are executed on a RedHat 9.x machine due to incompatible
versions of various system libraries, including glibc.
RedHat 9.x uses glibc 2.3.x, and Condor has not yet been ported to
this version of the C library.
However, we have heard reports that the dynamically linked version of
the Condor binaries will run on RedHat 9.x machines.
\Condor{compile} does not yet work with RedHat 9.x, and we are not
sure if a job linked with the Condor libraries on a RedHat 7.x machine
will run on a RedHat 9.x machine.  
If you are planning on using Condor, we do NOT recommend upgrading
your machines to RedHat 9.x at this time. 

