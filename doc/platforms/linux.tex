%%%%%%%%%%%%%%%%%%%%%%%%%%%%%%%%%%%%%%%%%%%%%%%%%%%%%%%%%%%%%%%%%%%%%%
\section{\label{sec:platform-linux}Linux}
\index{platform-specific information!Linux}
%%%%%%%%%%%%%%%%%%%%%%%%%%%%%%%%%%%%%%%%%%%%%%%%%%%%%%%%%%%%%%%%%%%%%%

This section provides information specific to the Linux port of
Condor.
Linux is a difficult platform to support.
It changes very frequently, and Condor has some extremely
system-dependent code (for example, the checkpointing library).

Condor is sensitive to changes in the following elements of the
system: 
\begin{itemize}
\item The kernel version
\item The version of the GNU C library (glibc)
\item the version of GNU C Compiler (GCC) used to build and link
  Condor jobs (this only matters for Condor's Standard universe which
  provides checkpointing and remote system calls)
\end{itemize}

The Condor Team tries to provide support for various releases of the
distribution of Linux.
Red Hat is probably the most popular Linux distribution, and it
provides a common set of versions for the above system components
at which Condor can aim support.
Condor will often work with Linux distributions other than Red Hat (for
example, Debian or SuSE) that have the same versions of the above
components.
However, we do not usually test Condor on other Linux distributions
and we do not provide any guarantees about this.

New releases of Red Hat usually change the versions of some or all of
the above system-level components.
A version of Condor that works with one release of Red Hat might not
work with newer releases.
The following sections describe the details of Condor's support for
the currently available versions of Red Hat Linux on x86 architecture
machines.

%%%%%%%%%%%%%%%%%%%%%%%%%%%%%%%%%%%%%%%%%%%%%%%%%%%%%%%%%%%%%%%%%%%%%%
\subsection{\label{sec:platform-linux-activity}Linux Kernel-specific Information}
%%%%%%%%%%%%%%%%%%%%%%%%%%%%%%%%%%%%%%%%%%%%%%%%%%%%%%%%%%%%%%%%%%%%%%
\index{platform-specific information!Linux keyboard and mouse activity}
\index{Linux!keyboard and mouse activity}

Distributions that rely on the Linux 2.4.x and all Linux 2.6.x kernels
through version 2.6.10
do not modify the \Code{atime} of the input device file.
This leads to difficulty when Condor is run using one of these
kernels. 
The problem manifests itself in that Condor cannot properly
detect keyboard or mouse activity.
Therefore, using the activity in policy setting cannot
signal that Condor should stop running a job on a machine.

Condor version 6.6.8 implements a workaround for PS/2 devices.
A better fix known to fix the Linux 2.6.10 kernel is
posted at \URL{http://www.cs.wisc.edu/condor/input\_patch-2.6.10}.
This patch works better for PS/2 devices, and
may also work for USB devices.
A future version of Condor will implement better recognition
of USB devices,
such that the kernel patch will also definitively work for USB devices.

%%%%%%%%%%%%%%%%%%%%%%%%%%%%%%%%%%%%%%%%%%%%%%%%%%%%%%%%%%%%%%%%%%%%%%
\subsection{\label{sec:platform-linux-rh6}Red Hat Version 6.x}
%%%%%%%%%%%%%%%%%%%%%%%%%%%%%%%%%%%%%%%%%%%%%%%%%%%%%%%%%%%%%%%%%%%%%%
\index{platform-specific information!Red Hat 6.x}

Red Hat version 6.x is an older release of Red Hat, but it is still used
by a number of sites.
Since Condor has worked on this platform since it was first released,
we still provide binaries that support it.
Red Hat 6.2 uses the 2.2.x Linux kernel series, glibc version 2.1.x,
and GCC version egcs-1.1.2.
To run Condor on this platform and be able to use \Condor{compile},
you should download the Condor \VersionNotice\  binaries listed for use
with ``Linux 2.0.x and 2.2.x (glibc 2.1)''.


%%%%%%%%%%%%%%%%%%%%%%%%%%%%%%%%%%%%%%%%%%%%%%%%%%%%%%%%%%%%%%%%%%%%%%
\subsection{\label{sec:platform-linux-rh7}Red Hat Version 7.x}
%%%%%%%%%%%%%%%%%%%%%%%%%%%%%%%%%%%%%%%%%%%%%%%%%%%%%%%%%%%%%%%%%%%%%%
\index{platform-specific information!Red Hat 7.x}

Red Hat version 7.x is fully supported in Condor \VersionNotice.
\Condor{compile} works to link user jobs for the Standard universe
with the versions of GCC and glibc that comes with Red Hat 7.x.
Both the statically linked and dynamically linked versions of the
Condor binaries listed for use with ``Linux 2.4.x (glibc 2.2) - Red Hat
7.1, 7.2, 7.3'' will work with no additional effort.


%%%%%%%%%%%%%%%%%%%%%%%%%%%%%%%%%%%%%%%%%%%%%%%%%%%%%%%%%%%%%%%%%%%%%%
\subsection{\label{sec:platform-linux-rh8}Red Hat Version 8.x}
%%%%%%%%%%%%%%%%%%%%%%%%%%%%%%%%%%%%%%%%%%%%%%%%%%%%%%%%%%%%%%%%%%%%%%
\index{platform-specific information!Red Hat 8.x}

Red Hat version 8.x is fully supported in Condor \VersionNotice.
\Condor{compile} works to link user jobs for the Standard universe
with the versions of gcc and glibc that come with Red Hat 8.x.

%%%%%%%%%%%%%%%%%%%%%%%%%%%%%%%%%%%%%%%%%%%%%%%%%%%%%%%%%%%%%%%%%%%%%%
\subsection{\label{sec:platform-linux-rh9}Red Hat Version 9.x}
%%%%%%%%%%%%%%%%%%%%%%%%%%%%%%%%%%%%%%%%%%%%%%%%%%%%%%%%%%%%%%%%%%%%%%
\index{platform-specific information!Red Hat 9.x}

Red Hat version 9.x is fully supported in Condor \VersionNotice.
\Condor{compile} works to link user jobs for the Standard universe
with the versions of gcc and glibc that come with Red Hat 9.x.
