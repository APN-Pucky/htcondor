%%%%%%%%%%%%%%%%%%%%%%%%%%%%%%%%%%%%%%%%%%%%%%%%%%%%%%%%%%%%%%%%%%%%%%
\section{\label{sec:platform-windows}Microsoft Windows}
\index{platform-specific information!Windows}
\index{Windows NT!introduction|(}
%%%%%%%%%%%%%%%%%%%%%%%%%%%%%%%%%%%%%%%%%%%%%%%%%%%%%%%%%%%%%%%%%%%%%%

Windows NT is a strategic platform for Condor,
and therefore we have been working toward a complete
port to Windows NT.
Our goal is to make Condor every bit as capable on Windows NT as it is on
Unix -- or even more capable.  

Porting Condor from Unix to Windows NT is a formidable task,
because many
components of Condor must interact closely with the underlying operating
system.
Instead of waiting until all components of Condor are running
and stabilized on Windows NT,
we have decided to make a clipped version of Condor for Windows NT.
A clipped version is one in which there is no checkpointing
and there are no remote system calls.

This section contains additional information specific to running
Condor on Windows NT.  Eventually this information will be integrated
into the Condor Manual as a whole, and this section will disappear.
In order to effectively use Condor NT, first read the overview
chapter (section~\ref{sec:overview})
and the user's manual (section~\ref{sec:usermanual}).
If you will
also be administrating or customizing the policy and set up of Condor NT,
also read the administrator's manual 
chapter (section~\ref{sec:Admin-Intro}).
After reading these chapters,
review the information in this chapter for
important information and differences when using and administrating
Condor on Windows NT.
For information on installing Condor for Windows, see
section~\ref{sec:Windows-Install}.
\index{Windows NT!introduction|)}


\index{Windows NT!release notes|(}

\subsection{What is missing from Condor NT \VersionNotice?}

In general, this release on NT works the same as the 
release of Condor for Unix.  
However, the following items are not supported in this version:

\begin{itemize}

\item The Standard, Globus, and PVM job universes are not yet present.
This is no transparent process checkpoint/migration, and there are no
remote system calls.

\item Accessing files via a network share that requires a kerberos ticket
(such as AFS) is not yet supported.

\item The ability to run the job with the same credentials as the
submitting user is not yet supported.  Instead, Condor will use a set
of accounts (one per VM) with minimal access rights to run the job.

\end{itemize}

\subsection{What is included in Condor NT \VersionNotice?}

Except for those items listed above, most everything works
the same way in Condor NT as it does in the Unix release.
This release is based on the Condor \VersionNotice\ source tree, and thus the
feature set is the same as Condor \VersionNotice\ for Unix.  
For instance, all of the following work in Condor NT:
\begin{itemize}

\item The ability to submit, run, and manage queues of jobs running on a
cluster of NT machines.

\item All tools such as \Condor{q}, \Condor{status}, \Condor{userprio},
are included. Only \Condor{compile} is
\emph{not} included.

\item The ability to customize job policy using ClassAds.
The machine ClassAds contain all the information included in the Unix version,
including current load average, RAM and virtual memory sizes, integer and
floating-point performance, keyboard/mouse idle time, etc.  Likewise, job
ClassAds contain a full complement of information, including system
dependent entries such as dynamic updates of the job's image size and CPU
usage.

\item Everything necessary to run a Condor central manager on Windows NT.

\item Security mechanisms.

\item Support for SMP machines.

\item Condor NT can run jobs at a lower operating system priority level.
Jobs can be suspended, soft-killed by using a WM\_CLOSE message,
or hard-killed automatically based upon policy expressions.
For example, Condor NT can automatically suspend a job
whenever keyboard/mouse or non-Condor created CPU activity is detected, and
continue the job after the the machine has been idle for a specified amount
of time.

\item Condor NT correctly manages jobs which create multiple processes.  For
instance, if a Condor job spawns multiple processes and Condor
needs to kill the job,
all processes created by the job will be terminated.

\item In addition to interactive tools, users and administrators can receive
information from Condor by e-mail (standard SMTP) and/or by log files.

\item Condor NT includes a friendly GUI installation and set up program,
which can perform a full install or deinstall of Condor.
Information specified by the user in the set up program is stored in the
system registry.  
The set up program can update a current installation with a
new release using a minimal amount of effort.

\end{itemize}


\subsection{Details on how Condor NT starts/stops a job}

This section provides some details on how Condor NT starts and stops jobs.
This discussion is geared for the Condor administrator or advanced user who is
already familiar with the material in the Administrators' Manual
and wishes to know detailed information on what Condor NT does when
starting and stopping jobs.

When Condor NT is about to start a job, the \Condor{startd} on the execute
machine spawns a \Condor{starter} process.  The \Condor{starter} then
creates:
\begin{enumerate}

\item a run account on the machine with a login name of
``condor-reuse-vmX'', where X is the Virtual Machine number of the
\Condor{starter}.  This account is added to group Users.

\item a new temporary working directory for the job on the execute machine.
This directory is
named ``dir\_XXX'', where XXX is the process ID of the \Condor{starter}.
The directory is created in the \MacroUNI{EXECUTE} directory as
specified in Condor's configuration file.  Condor then grants write
permission to this directory for the user account newly created for the
job.

\item a new, non-visible Window Station and Desktop for the job.
Permissions are set so that only the
user account newly created has access rights to this Desktop.  Any windows
created by this job are not seen by anyone; the job is run in the
background.

\end{enumerate}

Next, the \Condor{starter} (called the starter) contacts the \Condor{shadow}
(called the shadow) process, which is
running on the submitting machine, and pulls over the job's executable and
input files.
These files are placed into the temporary working directory for the job.
After all files have been received,
the starter spawns the user's executable as user ``condor-reuse-vmX'', where X is the number of the virtual machine (or 1 on a uniprocessor machine).
Its current working directory set to the temporary working directory
(that is, \MacroUNI{EXECUTE}/dir\_XXX, where XXX is the process id of the \Condor{starter} daemon).

While the job is running, the starter closely monitors the CPU
usage and image size of all processes started by the job.
Every 20 minutes the starter sends this information,
along with the total size of all files contained in the job's
temporary working directory, to the shadow.
The shadow then
inserts this information into the job's ClassAd so that policy and
scheduling expressions can make use of this dynamic information.

If the job exits of its own accord (that is, the job completes),
the starter
first terminates any processes started by the job which could still be
around if the job did not clean up after itself.
The starter examines the job's temporary working directory for any
files which have been created or modified and sends these files back
to the shadow running on the submit machine.
The shadow
places these files into the \Opt{initialdir} specified in the
submit description file; if no \Opt{initialdir} was specified, the files go
into the directory where the user invoked \Condor{submit}.
Once all the output files are safely transferred back,
the job is removed from the queue.
If, however, the \Condor{startd} forcibly kills the job before all output files
could be transferred, the job is not removed from the queue but instead
switches back to the Idle state.  

If the \Condor{startd} decides to vacate a job prematurely,
the starter sends a WM\_CLOSE message to the job.
If the job spawned multiple child processes, the WM\_CLOSE message is only
sent to the parent process (that is, the one started by the starter).
The
WM\_CLOSE message is the preferred way to terminate a process on Windows NT,
since this method allows the job to cleanup and free any resources it may
have allocated.
When the job exits, the starter cleans up any processes left behind.
At this point, if \Opt{transfer\_files} is set to
\Arg{ONEXIT} (the default) in the job's submit description file,
the job switches from states, from Running to Idle,
and no files are transferred back.
If \Opt{transfer\_files} is set to \Arg{ALWAYS}, then any files
in the job's temporary working directory which were changed or modified are
first sent back to the submitting machine.
But this time, the shadow places these
so-called intermediate files into a subdirectory created in the
\MacroUNI{SPOOL} directory on the submitting machine
(\MacroUNI{SPOOL} is specified in Condor's configuration file).
The job is then switched back to the Idle state until Condor finds
a different machine on which to run.
When the job is started again,
Condor places into the job's temporary working directory the executable
and input files as before,
\emph{plus} any files stored in the submit machine's \MacroUNI{SPOOL} directory for that job.  

\Note A Windows console process can intercept a WM\_CLOSE message
via the Win32 SetConsoleCtrlHandler() function if it needs to do special
cleanup work at vacate time; a WM\_CLOSE message
generates a CTRL\_CLOSE\_EVENT.  See SetConsoleCtrlHandler() in the Win32
documentation for more info.

\Note The default handler in Windows NT for a WM\_CLOSE message is for the
process to exit.  Of course, the job could be coded to ignore it and not
exit, but eventually the \Condor{startd} will get impatient and hard-kill
the job (if that is the policy desired by the administrator).

Finally, after the job has left and any files transferred back,
the starter
deletes the temporary working directory, the temporary
account, the WindowStation and the Desktop before exiting itself.
If the starter should terminate abnormally, the \Condor{startd}
attempts the clean up.
If for some reason the \Condor{startd} should disappear as well
(that is, if the entire machine was power-cycled hard),
the \Condor{startd} will clean up when Condor is restarted.

\subsection{Security considerations in Condor NT}

% WRT the backslash character, extra spaces are added before it
% as viewed from the html generated.
%   Karen has tried
%         \File{C:$\backslash$WINNT}
%         \File{C:\Bs WINNT}
% and neither works.

On the execute machine, the user job is run using the access token of an
account dynamically created by Condor which has bare-bones access rights and
privileges.  For instance, if your machines are configured so that only
Administrators have write access to
%\File{C:\Bs WINNT},
\verb@C:\WINNT@,
then certainly no
Condor job run on that machine would be able to write anything there.
The only files the job should be able to access on the execute machine
are files accessible by the Users and Everyone groups, and files in the
job's temporary working directory.

On the submit machine, Condor impersonates the submitting user, therefore
the File Transfer mechanism has the same access rights as the submitting
user.  For example, say only Administrators can write to
%\File{C:\Bs WINNT}
\verb@C:\WINNT@
on the submit machine,
and a user gives the following to \Condor{submit} :
\begin{verbatim}
         executable = mytrojan.exe
         initialdir = c:\winnt
         output = explorer.exe
         queue
\end{verbatim}
Unless that user is in group Administrators, Condor will not permit
\File{explorer.exe} to be overwritten.  

If for some reason the submitting user's account disappears between the time
\Condor{submit} was run and when the job runs, Condor is not able to check
and see if the now-defunct submitting user has read/write access to a given
file.  In this case, Condor will ensure that group ``Everyone'' has read or
write access to any file the job subsequently tries to read or write.  This
is in consideration for some network setups, where the user account only
exists for as long as the user is logged in.

Condor also provides protection to the job queue.  It would be bad if the
integrity of the job queue is compromised, because a malicious user could
remove other user's jobs or even change what executable a user's job will
run.  To guard against this, in Condor's default configuration all connections to the \Condor{schedd} (the
process which manages the job queue on a given machine) are authenticated
using Windows NT's SSPI security layer.  The user is then authenticated
using the same challenge-response protocol that NT uses to authenticate
users to Windows NT file servers.  Once authenticated, the only users
allowed to edit job entry in the queue are:
\begin{enumerate}
\item the user who originally submitted that job (i.e. Condor allows users
to remove or edit their own jobs)
\item users listed in the \File{condor\_config} file parameter
\MacroNI{QUEUE\_SUPER\_USERS}.  In the default configuration, only the
``SYSTEM'' (LocalSystem) account is listed here.  
\end{enumerate}
\Warn Do not remove ``SYSTEM'' from \MacroNI{QUEUE\_SUPER\_USERS}, or
Condor itself will not be able to access the job queue when needed.  If the
LocalSystem account on your machine is compromised, you have all sorts of
problems!

To protect the actual job queue files themselves, the Condor NT installation
program will automatically set permissions on the entire Condor release
directory so that only Administrators have write access.

Finally, Condor NT has all the IP/Host-based security mechanisms present
in the full-blown version of Condor.  See section~\ref{sec:Host-Security}
starting on page~\pageref{sec:Host-Security} for complete information
on how to allow/deny access to Condor based upon machine host name or
IP address.

\subsection{Interoperability between Condor for Unix and Condor NT}

Unix machines and Windows NT machines running Condor can happily
co-exist in the same Condor pool without any problems.
Jobs submitted on Windows NT can run on Windows NT or Unix,
and jobs submitted on Unix can run on Unix or Windows NT.
Without any specification
(using the \AdAttr{requirements} expression in the submit description file),
the default behavior will be to 
require the execute machine to be of the same architecture and operating
system as the submit machine.

There is absolutely no need to run more than one Condor central manager,
even if you have both Unix and NT machines.  The Condor central manager
itself can run on either Unix or NT; there is no advantage to choosing
one over the other.  Here at University of Wisconsin-Madison, for
instance, we have hundreds of Unix (Solaris, Linux, Irix, etc) and
Windows NT machines in our Computer Science Department Condor pool.
Our central manager is running on Windows NT.  All is happy.

\subsection{Some differences between Condor for Unix -vs- Condor NT}

\begin{itemize}

\item On Unix, we recommend the creation of a ``\textit{condor}'' account
when installing Condor.  On NT, this is not necessary, as Condor NT is
designed to run as a system service as user LocalSystem.

\item On Unix, Condor finds the \File{condor\_config} main configuration
file by looking in ~condor, in /etc, or via an environment variable.
On NT, the location of \File{condor\_config} file is determined
via the registry key \File{HKEY\_LOCAL\_MACHINE/Software/Condor}.
You can override this value by setting an environment variable named
\Env{CONDOR\_CONFIG}.

\item On Unix, in the VANILLA universe at job vacate time Condor sends the
job a softkill signal defined in the submit-description file (defaults to
SIGTERM).  On NT, Condor sends a WM\_CLOSE message to the job at vacate
time.

\item On Unix, if one of the Condor daemons has a fault, a core file
will be created in the \MacroUNI{Log} directory.  On Condor NT, a
``core'' file will also be created, but instead of a memory dump of the
process it will be a very short ASCII text file which describes what
fault occurred and where it happened.  This information can be used by
the Condor developers to fix the problem.

\end{itemize}
\index{Windows NT!release notes|)}

%%%%%%%%%%%%%%%%%%%%%%%%%%%%%%%%%%%%%%%%%%%%%%%%%%%%%%%%%%%%%%%%%%%%%%
\subsection{\label{sec:Windows-Install}Installation on Windows}
%%%%%%%%%%%%%%%%%%%%%%%%%%%%%%%%%%%%%%%%%%%%%%%%%%%%%%%%%%%%%%%%%%%%%%

\index{installation!Windows NT|(}
\index{Windows NT!installation|(}
This section contains the instructions for installing the Microsoft
Windows NT version of Condor (Condor NT) at your site.  
The install program will set you up with a slightly customized configuration
file that you can further customize after the installation has completed.

Please read the copyright and disclaimer information in 
section~\ref{sec:copyright-and-disclaimer} on
page~\pageref{sec:copyright-and-disclaimer} of the manual, or in the
file 
\File{LICENSE.TXT}, before proceeding.  Installation and
use of Condor is acknowledgement that you have read and agreed to these
terms.

Be sure that the Condor tools that get run are of the same version
as the daemons installed.
If they were not (such as 6.5.3 daemons, when running 6.4 \Condor{submit}),
then things will not work.
There may be errors generated by the \Condor{schedd} daemon (in the log).
It is likely that a job would be correctly placed in the queue,
but the job will never run.

The Condor NT executable for distribution is packaged in
a single file such as:
\begin{verbatim}
  condor-6.6.0-winnt40-x86.exe
\end{verbatim}

\index{Windows NT!installation!initial file size}
This file is approximately 20 Mbytes in size, and may be
removed once Condor is fully installed.

Before installing Condor, please consider joining the condor-world mailing
list.  Traffic on this list is kept to an absolute minimum.  It is only
used to announce new releases of Condor.  To subscribe, send an email
to \Email{majordomo@cs.wisc.edu} with the body:
\begin{verbatim}
   subscribe condor-world 
\end{verbatim}

\subsubsection{Installation Requirements}

\begin{itemize}

\item Condor NT requires Microsoft Windows NT 4.0 with Service Pack 3
or above.  Service Pack 5 is recommended. \Note Condor NT has been fully
tested with Windows 2000 and Windows XP.

\item 50 megabytes of free disk space is recommended.  Significantly more 
disk space could be desired to be able to run jobs with large data files.

\item Condor NT will operate on either an NTFS or FAT filesystem.  However, for security purposes, NTFS is preferred.

\end{itemize}

%%%%%%%%%%%%%%%%%%%%%%%%%%%%%%%%%%%%%%%%%%%%%%%%%%%%%%%%%%%%%%%%%%%%%%
\subsubsection{\label{sec:NT-Preparing-to-Install}Preparing to Install
Condor under Windows NT} 
%%%%%%%%%%%%%%%%%%%%%%%%%%%%%%%%%%%%%%%%%%%%%%%%%%%%%%%%%%%%%%%%%%%%%%

\index{Windows NT!installation!preparation}
Before you install the Windows NT version of Condor at your site,
there are two major
decisions to make about the basic layout of your pool.

\begin{enumerate}
\item What machine will be the central manager?
\item Do I have enough disk space for Condor?
\end{enumerate}

If you feel that you already know the answers to these questions,
skip to the Windows NT Installation Procedure section below,
section~\ref{sec:nt-install-procedure} on
page~\pageref{sec:nt-install-procedure}.
If you are unsure, read on.

\begin{itemize} 

%%%%%%%%%%%%%%%%%%%%%%%%%%%%%%%%%%%%%%%%%%%%%%%%%%%%%%%%%%%%%%%%%%%%%%
\item{What machine will be the central manager?}
%%%%%%%%%%%%%%%%%%%%%%%%%%%%%%%%%%%%%%%%%%%%%%%%%%%%%%%%%%%%%%%%%%%%%%

One machine in your pool must be the central manager.
This is the
centralized information repository for the Condor pool and is also the
machine that matches available machines with waiting
jobs.  If the central manager machine crashes, any currently active
matches in the system will keep running, but no new matches will be
made.  Moreover, most Condor tools will stop working.  Because of the
importance of this machine for the proper functioning of Condor, we
recommend you install it on a machine that is likely to stay up all the
time, or at the very least, one that will be rebooted quickly if it
does crash.  Also, because all the services will send updates (by
default every 5 minutes) to this machine, it is advisable to consider
network traffic and your network layout when choosing the central
manager.

For Personal Condor, your machine will act as your central manager.

Install Condor on the central manager before installing
on the other machines within the pool.

%%%%%%%%%%%%%%%%%%%%%%%%%%%%%%%%%%%%%%%%%%%%%%%%%%%%%%%%%%%%%%%%%%%%%%
\item{Do I have enough disk space for Condor?}
%%%%%%%%%%%%%%%%%%%%%%%%%%%%%%%%%%%%%%%%%%%%%%%%%%%%%%%%%%%%%%%%%%%%%%

\index{Windows NT!installation!required disk space}
The Condor release directory takes up a fair amount of space.
The size requirement for the release
directory is approximately 50 Mbytes.

Condor itself, however, needs space to store all of your jobs, and their
input files.  If you will be submitting large amounts of jobs,
you should consider installing Condor on a volume with a large amount
of free space.

\end{itemize}


%%%%%%%%%%%%%%%%%%%%%%%%%%%%%%%%%%%%%%%%%%%%%%%%%%%%%%%%%%%%%%%%%%%%%%
\subsubsection{\label{sec:nt-install-procedure}
Installation Procedure using the included Setup Program}
%%%%%%%%%%%%%%%%%%%%%%%%%%%%%%%%%%%%%%%%%%%%%%%%%%%%%%%%%%%%%%%%%%%%%%

% condor MUST be run as local system
% 
%  root == administrator
%  to install, must be running with administrator privileges
%  the kernel runs as == local system

Installation of Condor must be done by a user with administrator privileges.
After installation, the Condor services will be run under the local system account.
When Condor is running a user job, however, it will run that User job with normal user permissions.
Condor will dynamically enable a special account for running Condor
jobs and disable the account when the job is finished or removed from
the machine.

Download Condor, and start the installation process by running the file (or by double clicking on the file).
The Condor installation is completed by answering questions and choosing options within the following steps.


\begin{description}
\item[If Condor is already installed.]

     For upgrade purposes, you may be running the installation of Condor
     after it has been previously installed.
     In this case, a dialog box will appear before the
     installation of Condor proceeds.
     The question asks if you wish to preserve your current
     Condor configuration files.
     Answer yes or no, as appropriate.
	 
	 If you answer yes, your configuration files will not be changed, and you will proceed to the point where the new binaries will be installed.

     If you answer no, then there will be a second question
     that asks if you want to use answers
     given during the previous installation
     as default answers.

\item[STEP 1: License Agreement.]

     The first step in installing Condor
     is a welcome screen and license agreement.
     You are reminded that it is best to run the installation
     when no other Windows programs are running.
	 If you need to close other Windows NT programs, it is safe to cancel the
	 installation and close them.
     You are asked to agree to the license.
     Answer yes or no.  If you should disagree with the License, the
	 installation will not continue.

     After agreeing to the license terms, the next Window is where 
     fill in your name and company information,
     or use the defaults as given.

\item[STEP 2: Condor Pool Configuration.]

     The Condor NT installation will require different
     information depending on whether the installer will
	 be creating a new pool, or joining an existing one.

     If you are creating a new pool, the installation program
	 requires that this machine is the central manager.  
     For the creation of a new Condor pool, you will be asked
	 some basic information about your new pool:
     \begin{description}
     \item[Name of the pool]
     \item[hostname] of this machine.
%  Derek hath declared the Statistics not worthy of prime time.
%     \item[Do you want to keep statistics?]
%       Answer yes or no, as appropriate.
%       If yes, then the maximum amount of data accumulated will
%       be 10 Mbytes.
%       A configurable quantity, \Macro{POOL\_HISTORY\_MAX\_STORAGE}
%       sets the maximum amount of data, and it
%       defaults to 10 Mbytes.
%       If no, then the Condor View client will not have data to display.
     \item[Size of pool]
       Condor needs to know if this a Personal Condor installation,
       or if there will be more than one machine in the pool.
\index{Windows NT!installation!Personal Condor}
\index{Personal Condor}
       A Personal Condor pool
       implies that there is only one machine in the pool.
       For Personal Condor, several of the following
       steps are omitted as noted.
     \end{description}

     If you are joining an existing pool, all the installation program
	 requires is the hostname of the central manager for your pool.

\item[STEP 3: This Machine's Roles.] 

     This step is omitted for the installation of Personal Condor.

     Each machine within a Condor pool may either
     submit jobs or execute submitted jobs, or both
     submit and execute jobs.
     This step allows the installation on this machine
     to choose if the machine will only submit jobs,
     only execute submitted jobs, or both.
     The common case is both, so the default is both.

\item[STEP 4: Where will Condor be installed?]

\index{Windows NT!installation!location of files}
The next step is where the destination of the Condor files will be
decided.
It is recommended that Condor be installed in the location shown as the default in the dialog box:
\File{C:\Bs Condor}.

Installation on the local disk is chosen for several reasons.

The Condor services run as local system, and within Microsoft Windows NT, local system has no network privileges.
Therefore, for Condor to operate, Condor should be installed on a local hard drive as opposed to a network drive (file server).

The second reason for installation on the local disk is that
the Windows NT usage of drive letters has implications for where
Condor is placed.
The drive letter used must be not change, even when different users are
logged in.
Local drive letters do not change under normal operation of Windows NT.

While it is strongly discouraged, it may be possible to place Condor on a hard drive that is not local,  if a dependency is added to the service control manager
such that Condor starts after the required file services
are available.

%  !! goes in C:/condor   (default)
%  !! advice is really should go on local hard drive,
%  as opposed to a network drive (also called file server)
%  Because,
%    1. Condor runs as local system, and accesses to a network
%      drive can't be authenticated  -- local system has
%      no network privileges.
%    2.  it is likely that you don't have this set up:
%    (and you need it to make it work)
%    you can add a dependency in the service control manager
%    that condor should start after the file services are
%    available
%    3. drive letters are "system-wide"
%    Must have dedicated letter (for all users), that remains
%    intact for all time, or condor won't know where
%    things are and can't get access (without its "letter")


\item[STEP 5: Where is the Java Virtual Machine?]
	While not required, it is possible for Condor to run jobs in the
	Java universe. In order for Condor to have support for java,
	you must supply a path to \verb@java.exe@ on your system. The
	installer will tell you if the path is invalid before proceeding
	to the next step. To disable the Java universe, simply leave
	this field blank.

\item[STEP 6: Where should Condor send e-mail if things go wrong?]

     Various parts of Condor will send e-mail to a Condor administrator
     if something goes wrong and requires human attention.
     You specify the e-mail address and the SMTP relay host
     of this administrator.  Please pay close attention to this email
	 since it will indicate problems in your Condor pool.

\item[STEP 7: The domain.]

% not really used right now.  "Things that suck about NT."
% UNIX has 2 domains:  file system domain and user-ID domain
% NT has only 1:  a combination, and so going back to letter
% drives, things get screwed up.
     This step is omitted for the installation of Personal Condor.

     Enter the machine's accounting (or UID) domain.
	 On this version of Condor for Windows NT, this setting only used for User
	 priorities (see section~\ref{sec:UserPrio} on
	 page~\pageref{sec:UserPrio}) and to form a default email address for
	 the user.

\item[STEP 8: Access permissions.]
     This step is omitted for the installation of Personal Condor.

     Machines within the Condor pool will need
     various types of access permission. 
     The three categories of permission are read, write,
     and administrator. Enter the machines to be given
     access permissions.

     \begin{description}
     \item[Read]
     Read access allows a machine to obtain information about
     Condor such as the status of machines in the pool and the
     job queues.
     All machines in the pool should be given read access. 
     In addition, giving read access to *.cs.wisc.edu 
     will allow the Condor team to obtain information about
     your Condor pool in the event that debugging is needed.
     \item[Write]
     All machines in the pool should be given write access. 
     It allows the machines you specify to send information to your
	 local Condor daemons, for example, to start a Condor Job.
     Note that for a machine to join the Condor pool, it must have both read and write access to all of the machines in the pool.
     \item[Administrator]
     A machine with administrator access will be allowed more
     extended permission to to things such as
     change other user's priorities, modify the job queue,
     turn Condor services on and off,
     and restart Condor.
     The central manager should be given administrator access
     and is the default listed.
	 This setting is granted to the entire machine, so care should be taken not to make this too open.
     \end{description}

	 For more details on these access permissions, and others that can be
	 manually changed in your \File{condor\_config} file, please
	 see the section titled Setting Up IP/Host-Based Security in Condor
	 in section
	 section~\ref{sec:Host-Security}
	 on page~\pageref{sec:Host-Security}.

\item[STEP 9: Job Start Policy.]
     Condor will execute submitted jobs on machines based on
     a preference given at installation.
     Three options are given, and the first is most commonly used
     by Condor pools.
     This specification may be changed or refined in
     the machine ClassAd requirements attribute.

     The three choices:
     \begin{description}
     \item[After 15 minutes of no console activity and low CPU activity.]
     \item[Always run Condor jobs.]
     \item[After 15 minutes of no console activity.]
     \end{description}

\index{Console activity}
     Console activity is the use of the mouse or keyboard.  For instance,
	 if you are reading this document online, and are using either the
	 mouse or the keyboard to change your position, you are generating
	 Console activity.

\index{CPU activity}
     Low CPU activity is defined as a load of less than 30\Percent
	 (and is configurable in your \File{condor\_config} file).  If you have
	 a multiple processor machine, this is the average percentage of
	 CPU activity for both processors.

	For testing purposes, it is often helpful to use use the Always run Condor
	jobs option.  For production mode, however, most people chose the
	After 15 minutes of no console activity and low CPU activity.

\item[STEP 10: Job Vacate Policy.]
     This step is omitted if Condor jobs are always run as
     the option chosen in STEP 9.

     If Condor is executing a job and the user returns,
	 Condor will immediately suspend the job, and after five minutes
	 Condor will decide what to do with the partially completed job.
     There are currently two options for the job.

     \begin{description}
     \item[The job is killed 5 minutes after your return.]
     The job is suspended immediately once there is console activity.
     If the console activity continues, then the job is
     vacated (killed) after 5 minutes. 
     Since this version does not include check-pointing, the job will
     be restarted from the beginning at a later time.
     The job will be placed back into the queue.
     \item[ Suspend job, leaving it in memory.]
     The job is suspended immediately.  At a later time, when the
	 console activity has stopped for ten minutes, the execution of
	 Condor job will be resumed (the job will be unsuspended).
	 The drawback to this option is that since the job will remain
	 in memory, it will occupy swap space.  In many instances, however,
	 the amount of swap space that the job will occupy is small.
     \end{description}

%    Advice on which to choose goes here.
     So which one do you choose?  Killing a job is less intrusive
	 on the workstation owner than leaving it in memory for a later time.
     A suspended job left in memory will require swap space,
     which could possibly be a scarce resource.
     Leaving a job in memory, however, has the benefit that accumulated
     run time is not lost for a partially completed job.

\item[STEP 11: Review entered information.]
     Check that the entered information is correctly entered.
     You have the option to return to previous dialog boxes to fix entries.
\end{description}


%%%%%%%%%%%%%%%%%%%%%%%%%%%%%%%%%%%%%%%%%%%%%%%%%%%%%%%%%%%%%%%%%%%%%%
\subsubsection{\label{sec:NT-Manual-Install}Manual Installation Condor on Windows NT}
%%%%%%%%%%%%%%%%%%%%%%%%%%%%%%%%%%%%%%%%%%%%%%%%%%%%%%%%%%%%%%%%%%%%%%

\index{Windows NT!manual install|(}
If you are to install Condor on many different machines, you may wish
to use some other mechanism to install Condor NT on additional machines
rather than running the Setup program described above on each machine.

\Warn This is for advanced users only!  All others should use the Setup program described above. 

Here is a brief overview of how to install Condor NT manually without using the provided GUI-based setup program:

\begin{description}
\item [The Service]
The service that Condor NT will install is called "Condor".  The Startup
Type is Automatic.  The service should log on as System Account, but
\Bold{do not enable} "Allow Service to Interact with Desktop".  The
program that is run is \Condor{master.exe}.

For your convenience, we have included a file called \File{install.exe} in 
the bin directory that will install a service.  It is typically called in the
following way:
\begin{verbatim}
install Condor Condor c:\condor\bin\condor_master.exe
\end{verbatim}

If you wish to remove the service, we have provided a file called
\File{remove.exe}.  To use it, call it in the following way:
\begin{verbatim}
remove Condor
\end{verbatim}

\item [The Registry]
Condor NT uses a few registry entries in its operation.  The key that Condor
uses is HKEY\_LOCAL\_MACHINE/Software/Condor.  The values that Condor puts
in this registry key serve two purposes.
\begin{enumerate}
\item The values of CONDOR\_CONFIG and RELEASE\_DIR are used for Condor
to start its service.

CONDOR\_CONFIG should point to the \File{condor\_config} file.  In this version
of Condor NT, it \Bold{must} reside on the local disk.

RELEASE\_DIR should point to the directory where Condor is installed.  This
is typically \File{C:\Bs Condor}, and again, this \Bold{must} reside on the
local disk.

\item The other purpose is storing the entries from the last installation
so that they can be used for the next one.
\end{enumerate}

\item [The Filesystem]
The files that are needed for Condor to operate are identical to the Unix
version of Condor, except that executable files end in \File{.exe}.  For
example the on Unix one of the files is \File{condor\_master} and on Condor
NT the corresponding file is \File{condor\_master.exe}.

These files currently must reside on the local disk for a variety of reasons.
Advanced Windows NT users might be able to put the files on remote resources.
The main concern is twofold.  First, the files must be there when the service
is started.  Second, the files must always be in the same spot (including
drive letter), no matter who is logged into the machine.  

\end{description}

\index{manual installation!Windows NT|)}

%%%%%%%%%%%%%%%%%%%%%%%%%%%%%%%%%%%%%%%%%%%%%%%%%%%%%%%%%%%%%%%%%%%%%%
\subsubsection{\label{nt-installed-now-what}
Condor is installed... now what?}
%%%%%%%%%%%%%%%%%%%%%%%%%%%%%%%%%%%%%%%%%%%%%%%%%%%%%%%%%%%%%%%%%%%%%%
\index{Windows NT!starting the Condor service|(}

After the installation of Condor is completed, the Condor service
must be started.  If you used the GUI-based setup program to install
Condor, the Condor service should already be started.  If you installed
manually, Condor must
be started by hand, or you can simply reboot. \Note The Condor service
will start automatically whenever you reboot your machine.

To start condor by hand:
\begin{enumerate}
\item From the Start menu, choose Settings.
\item From the Settings menu, choose Control Panel.
\item From the Control Panel, choose Services.
\item From Services, choose Condor, and Start.
\end{enumerate}

Or, alternatively you can enter the following command from a command prompt:
\begin{verbatim}
         net start condor
\end{verbatim}
\index{Windows NT!starting the Condor service|)}

\index{Windows NT!Condor daemon names}
Run the Task Manager (Control-Shift-Escape) to check that Condor
services are running.  The following tasks should
be running:  
\begin{itemize}
\item \Condor{master.exe}
\item \Condor{negotiator.exe}, if this machine is a central manager.
\item \Condor{collector.exe}, if this machine is a central manager.
\item \Condor{startd.exe}, if you indicated that this Condor node should start jobs
\item \Condor{schedd.exe}, if you indicated that this Condor node should submit jobs
to the Condor pool.
\end{itemize}

Also, you should now be able to open up a new cmd (DOS prompt) window, and
the Condor bin directory should be in your path, so you can issue the normal
Condor commands, such as \condor{q} and \condor{status}.

\index{installation!Windows NT|)}
\index{Windows NT!installation|)}

\subsubsection{\label{nt-running-now-what}
Condor is running... now what?}
%%%%%%%%%%%%%%%%%%%%%%%%%%%%%%%%%%%%%%%%%%%%%%%%%%%%%%%%%%%%%%%%%%%%%%

Once Condor services are running, try building
and submitting some test jobs.  See the \File{README.TXT} file in the
examples directory
for details.

