%%%%%%%%%%%%%%%%%%%%%%%%%%%%%%%%%%%%%%%%%%%%%%%%%%%%%%%%%%%%%%%%%%%%%%
\section{\label{sec:platform-windows}Microsoft Windows}
%%%%%%%%%%%%%%%%%%%%%%%%%%%%%%%%%%%%%%%%%%%%%%%%%%%%%%%%%%%%%%%%%%%%%%
\index{platform-specific information!Windows|(}

Windows is a strategic platform for Condor,
and therefore we have been working toward a complete
port to Windows.
Our goal is to make Condor every bit as capable on Windows as it is on
Unix -- or even more capable.

Porting Condor from Unix to Windows is a formidable task,
because many
components of Condor must interact closely with the underlying operating
system.
Provided is a clipped version of Condor for Windows.
A clipped version is one in which there is no checkpointing
and there are no remote system calls.

This section contains additional information specific to running
Condor on Windows.
In order to effectively use Condor, first read the overview
chapter (section~\ref{sec:overview})
and the user's manual (section~\ref{sec:usermanual}).
If administrating or customizing the policy and set up of Condor,
also read the administrator's manual 
chapter (section~\ref{sec:Admin-Intro}).
After reading these chapters,
review the information in this chapter for
important information and differences when using and administrating
Condor on Windows.
For information on installing Condor for Windows, see
section~\ref{sec:Windows-Install}.



%%%%%%%%%%%%%%%%%%%%%%%%%%%%%%%%%%%%%%%%%%%%%%%%%%%%%%%%%%%%%%%%%%%%%%
\subsection{Limitations under Windows}
%%%%%%%%%%%%%%%%%%%%%%%%%%%%%%%%%%%%%%%%%%%%%%%%%%%%%%%%%%%%%%%%%%%%%%
\index{Windows!release notes}

In general, this release for Windows works the same as the 
release of Condor for Unix.
However, the following items are not supported in this version:

\begin{itemize}

\item The standard job universe is not present.  This means
transparent process checkpoint/migration and remote system calls are
not supported.

\item For \SubmitCmd{grid} universe jobs, the only supported grid type is
\SubmitCmd{condor}.

\item Accessing files via a network share that requires a Kerberos ticket
(such as AFS) is not yet supported.

\end{itemize}

%%%%%%%%%%%%%%%%%%%%%%%%%%%%%%%%%%%%%%%%%%%%%%%%%%%%%%%%%%%%%%%%%%%%%%
\subsection{Supported Features under Windows}
%%%%%%%%%%%%%%%%%%%%%%%%%%%%%%%%%%%%%%%%%%%%%%%%%%%%%%%%%%%%%%%%%%%%%%

Except for those items listed above, most everything works
the same way in Condor as it does in the Unix release.
This release is based on the Condor \VersionNotice\ source tree, and thus the
feature set is the same as Condor \VersionNotice\ for Unix.  
For instance, all of the following work in Condor:
\begin{itemize}

\item The ability to submit, run, and manage queues of jobs running on a
cluster of Windows machines.

\item All tools such as \Condor{q}, \Condor{status}, \Condor{userprio},
are included. Only \Condor{compile} is
\emph{not} included.

\item The ability to customize job policy using ClassAds.
The machine ClassAds contain all the information included in the Unix version,
including current load average, RAM and virtual memory sizes, integer and
floating-point performance, keyboard/mouse idle time, etc.  Likewise, job
ClassAds contain a full complement of information, including system
dependent entries such as dynamic updates of the job's image size and CPU
usage.

\item Everything necessary to run a Condor central manager on Windows.

\item Security mechanisms.

\item Support for SMP machines.

\item Condor for Windows can run jobs at a lower operating system
priority level.
Jobs can be suspended, soft-killed by using a WM\_CLOSE message,
or hard-killed automatically based upon policy expressions.
For example, Condor can automatically suspend a job
whenever keyboard/mouse or non-Condor created CPU activity is detected, and
continue the job after the machine has been idle for a specified amount
of time.

\item Condor correctly manages jobs which create multiple processes.  For
instance, if a Condor job spawns multiple processes and Condor
needs to kill the job,
all processes created by the job will be terminated.

\item In addition to interactive tools, users and administrators can receive
information from Condor by e-mail (standard SMTP) and/or by log files.

\item Condor includes a friendly GUI installation and set up program,
which can perform a full install or deinstall of Condor.
Information specified by the user in the set up program is stored in the
system registry.  
The set up program can update a current installation with a
new release using a minimal amount of effort.

\item Condor can give a job access to the running user's Registry hive.

\end{itemize}

%%%%%%%%%%%%%%%%%%%%%%%%%%%%%%%%%%%%%%%%%%%%%%%%%%%%%%%%%%%%%%%%%%%%%%
\subsection{\label{sec:windows-sps}Secure Password Storage}
%%%%%%%%%%%%%%%%%%%%%%%%%%%%%%%%%%%%%%%%%%%%%%%%%%%%%%%%%%%%%%%%%%%%%%

In order for Condor to operate properly, it must at times be able to
act on behalf of users who submit jobs.  
This is required on submit machines,
so that Condor can access a job's input files, 
create and access the job's output files, 
and write to the job's log file from within the appropriate security context.
On Unix systems, arbitrarily changing what user Condor performs its
actions as is easily done when Condor is started with root privileges.
On Windows, however, performing an action as a particular user
or on behalf of a particular user requires knowledge of that user's password,
even when running at the maximum privilege level.
Condor provides secure password storage through the use of the 
\Condor{store\_cred} tool.
Passwords managed by Condor are encrypted and stored 
in a secure location within the Windows registry.
When Condor needs to perform an action as or on behalf of a particular user,
it uses the securely stored password to do so.
This implies that a password is stored for every user that will
submit jobs from the Windows submit machine.

\index{Condor daemon!condor\_credd@\Condor{credd}}
\index{daemon!condor\_credd@\Condor{credd}}
\index{condor\_credd daemon}
\index{submit commands!run\_as\_owner}
A further feature permits Condor to execute the job itself under the
security context of its submitting user, specifying 
the \SubmitCmd{run\_as\_owner} command in the job's
submit description file. 
With this feature, 
it is necessary to configure and run a centralized \Condor{credd} daemon
to manage the secure password storage.
This makes each user's password available, via an encrypted
connection to the \Condor{credd}, to any execute machine that may need it.

By default, the secure password store for a submit machine when no
\Condor{credd} is running is managed by the \Condor{schedd}.
This approach works in environments where
the user's password is only needed on the submit machine.

%%%%%%%%%%%%%%%%%%%%%%%%%%%%%%%%%%%%%%%%%%%%%%%%%%%%%%%%%%%%%%%%%%%%%%
\subsection{\label{sec:windows-run-as-owner}Executing Jobs as the Submitting User}
%%%%%%%%%%%%%%%%%%%%%%%%%%%%%%%%%%%%%%%%%%%%%%%%%%%%%%%%%%%%%%%%%%%%%%
\index{submit commands!run\_as\_owner}

By default, Condor executes jobs on Windows using dedicated run
accounts that have minimal access rights and privileges,
and which are recreated for each new job.
As an alternative, Condor can be configured to allow users to run jobs using
their Windows login accounts. 
This may be useful if jobs need access to files on a network share,
or to other resources that are not available to the low-privilege run account.

This feature requires use of a \Condor{credd} daemon for secure
password storage and retrieval. 
With the \Condor{credd} daemon running,
the user's password must be stored, using the \Condor{store\_cred} tool.
Then,
a user that wants a job to run using their own account
places into the job's submit description file
\begin{verbatim}
  run_as_owner = True
\end{verbatim}

%%%%%%%%%%%%%%%%%%%%%%%%%%%%%%%%%%%%%%%%%%%%%%%%%%%%%%%%%%%%%%%%%%%%%%
\subsection{\label{sec:credd}The condor\_credd Daemon}
%%%%%%%%%%%%%%%%%%%%%%%%%%%%%%%%%%%%%%%%%%%%%%%%%%%%%%%%%%%%%%%%%%%%%%
\index{Condor daemon!condor\_credd@\Condor{credd}}
\index{daemon!condor\_credd@\Condor{credd}}
\index{condor\_credd daemon}
The \Condor{credd} daemon manages secure password storage.
A single running instance of the \Condor{credd} within a Condor pool
is necessary in order to provide the feature described in 
section \ref{sec:windows-run-as-owner},
where a job runs as the submitting user, 
instead of as a temporary user that has strictly limited access capabilities.

It is first necessary to select
the single machine on which to run the \Condor{credd}. 
Often, the machine acting as the pool's central manager is a good choice.
An important restriction, however, is that the \Condor{credd} host must be a
machine running Windows.

All configuration settings necessary to enable the \Condor{credd} are
contained in the example file \verb|etc\condor_config.local.credd|
from the Condor distribution. Copy these settings into a local
configuration file for the machine that will run the \Condor{credd}.
Run \Code{condor\_restart} for these new settings to take effect, then
verify (via Task Manager) that a \Condor{credd} process is running.

A second set of configuration variables specify security for the
communication among Condor daemons.
These variables must be set for all machines in the pool.
The following example settings are in the comments contained in the
\verb|etc\condor_config.local.credd| example file.
These sample settings rely on the \Expr{PASSWORD} method for
authentication among daemons, 
including communication with the \Condor{credd} daemon.
The \Macro{LOCAL\_CREDD} variable must be customized to point to the machine
hosting the \Condor{credd} and the \Macro{ALLOW\_CONFIG} variable
will be customized, if needed, to refer to an administrative account 
that exists on all Condor nodes. 
\begin{verbatim}
CREDD_HOST = credd.cs.wisc.edu
CREDD_CACHE_LOCALLY = True

STARTER_ALLOW_RUNAS_OWNER = True

ALLOW_CONFIG = Administrator@*
SEC_CLIENT_AUTHENTICATION_METHODS = NTSSPI, PASSWORD
SEC_CONFIG_NEGOTIATION = REQUIRED
SEC_CONFIG_AUTHENTICATION = REQUIRED
SEC_CONFIG_ENCRYPTION = REQUIRED
SEC_CONFIG_INTEGRITY = REQUIRED
\end{verbatim}

In order for \Expr{PASSWORD} authenticated communication to work,
a \Term{pool password} must be chosen and distributed.
The chosen pool password must be stored identically for each machine.
The pool password first should be
stored on the \Condor{credd} host, then on the other machines in the pool.

To store the pool password on a Windows machine, run
\begin{verbatim}
  condor_store_cred add -c
\end{verbatim}
when logged in with the administrative account on that machine,
and enter the password when prompted. 
If the administrative account is shared across all machines,
that is if it is a domain account or has the same password on all machines,
logging in separately to each machine in the pool can be avoided.
Instead, the pool password can be securely pushed out for each Windows machine
using a command of the form
\begin{verbatim}
  condor_store_cred add -c -n exec01.cs.wisc.edu
\end{verbatim}

Once the pool password is distributed, 
but before submitting jobs,
all machines must reevaluate their configuration,
so execute
\begin{verbatim}
  condor_reconfig -all
\end{verbatim}
from the central manager.
This will cause each execute machine to test its ability
to authenticate with the \Condor{credd}.
To see whether this test worked for each machine in the pool, run the command
\footnotesize
\begin{verbatim}
  condor_status -f "%s\t" Name -f "%s\n" ifThenElse(isUndefined(LocalCredd),\"UNDEF\",LocalCredd)
\end{verbatim}
\normalsize
Any rows in the output with the \Expr{UNDEF} string indicate machines where
secure communication is not working properly. Verify that the pool password
is stored correctly on these machines.

%%%%%%%%%%%%%%%%%%%%%%%%%%%%%%%%%%%%%%%%%%%%%%%%%%%%%%%%%%%%%%%%%%%%%%
\subsection{\label{sec:windows-load-profile}Executing Jobs with the User's Profile Loaded}
%%%%%%%%%%%%%%%%%%%%%%%%%%%%%%%%%%%%%%%%%%%%%%%%%%%%%%%%%%%%%%%%%%%%%%
\index{Windows!loading account profile}
Condor can be configured when using dedicated run accounts, 
to load the account's profile.  A user's profile includes a set of personal 
directories and a registry hive loaded under \texttt{HKEY\_CURRENT\_USER}.

This may be useful if the job requires direct access to the user's registry 
entries.
It also may be useful when the job requires an application, 
and the application requires registry access. 
This feature is always enabled on the \Condor{startd}, 
but it is limited to the dedicated run account.
For security reasons, the profiles are
removed after the job has completed and exited.  
This ensures 
that malicious jobs cannot discover what any previous job has done, nor 
sabotage the registry for future jobs. It also ensures the next job has 
a fresh registry hive.

A user that then wants a job to run with a profile uses the
\SubmitCmd{load\_profile} command in the job's submit description file:
\begin{verbatim}
load_profile = True
\end{verbatim}

This feature is currently not compatible with \SubmitCmd{run\_as\_owner}, 
and will be ignored if both are specified.

%%%%%%%%%%%%%%%%%%%%%%%%%%%%%%%%%%%%%%%%%%%%%%%%%%%%%%%%%%%%%%%%%%%%%%
\subsection{\label{sec:windows-scripts-as-executables}Using Windows Scripts as Job Executables}
%%%%%%%%%%%%%%%%%%%%%%%%%%%%%%%%%%%%%%%%%%%%%%%%%%%%%%%%%%%%%%%%%%%%%%

Condor has added support for scripting jobs on Windows.
Previously, Condor jobs on Windows were limited to executables or batch files.
With this new support,
Condor determines how to interpret the script using 
the file name's extension.
Without a file name extension, 
the file will be treated as it has been in the past:
as a Windows executable.

This feature may not require any modifications to Condor's configuration.  
An example that does not require administrative intervention
are Perl scripts using \Prog{ActivePerl}.

\Prog{Windows Scripting Host} scripts do require
configuration to work correctly.
The configuration variables set values to be used in registry look up,
which results in a command that invokes the correct interpreter,
with the correct command line arguments for the specific scripting
language.
In Microsoft nomenclature, 
\Term{verbs} are actions that can be taken upon a given a file.
The familiar examples of
\textbf{Open}, \textbf{Print}, and \textbf{Edit},
can be found on the context menu when a user right clicks on a file.
The command lines to be used for each of these verbs are stored in
the registry under the \Code{HKEY\_CLASSES\_ROOT} hive.
In general, a registry look up uses the form:

\footnotesize
\begin{verbatim}
HKEY_CLASSES_ROOT\<FileType>\Shell\<OpenVerb>\Command
\end{verbatim}
\normalsize

Within this specification, 
\verb@<FileType>@ is the name of a file type
(and therefore a scripting language),
and is obtained from the file name extension.
\verb@<OpenVerb>@ identifies the verb,
and is obtained from the Condor configuration.

The Condor configuration sets the selection of a verb,
to aid in the registry look up.
The file name extension sets the name of the Condor configuration variable.
This variable name is of the form:
\begin{verbatim}
OPEN_VERB_FOR_<EXT>_FILES
\end{verbatim}
\verb@<EXT>@ represents the file name extension.
The following configuration example uses the \verb@Open2@ verb for
a \Prog{Windows Scripting Host} registry look up for several scripting
languages:

\begin{verbatim}
OPEN_VERB_FOR_JS_FILES  = Open2
OPEN_VERB_FOR_VBS_FILES = Open2
OPEN_VERB_FOR_VBE_FILES = Open2
OPEN_VERB_FOR_JSE_FILES = Open2
OPEN_VERB_FOR_WSF_FILES = Open2
OPEN_VERB_FOR_WSH_FILES = Open2
\end{verbatim}

In this example, Condor specifies the \verb@Open2@ verb,
instead of the default \verb@Open@ verb,
for a script with the file name extension of \verb@wsh@.
The \Prog{Windows Scripting Host}'s \verb@Open2@ verb allows standard input,
standard output, and standard error to be redirected
as needed for Condor jobs.

A common difficulty is encountered when
a script interpreter requires access to the user's registry.
Note that the user's registry is different than the root registry.
If not given access to the user's registry,
some scripts, such as \Prog{Windows Scripting Host} scripts,
will fail.
The failure error message appears as: 

\begin{verbatim}
CScript Error: Loading your settings failed. (Access is denied.)
\end{verbatim}

The fix for this error is to give explicit access to the submitting
user's registry hive.  This can be accomplished with the addition of
the \SubmitCmd{load\_profile} command in the job's submit description
file:

\begin{verbatim}
load_profile = True
\end{verbatim}

With this command,
there should be no registry access errors.
This command should also work for other interpreters.
Note that not all interpreters will require access.
For example,
\Prog{ActivePerl} does not by default require access to the user's
registry hive.

%%%%%%%%%%%%%%%%%%%%%%%%%%%%%%%%%%%%%%%%%%%%%%%%%%%%%%%%%%%%%%%%%%%%%%
\subsection{Details on how Condor for Windows starts/stops a job}
%%%%%%%%%%%%%%%%%%%%%%%%%%%%%%%%%%%%%%%%%%%%%%%%%%%%%%%%%%%%%%%%%%%%%%

This section provides some details on how Condor starts and stops jobs.
This discussion is geared for the Condor administrator or advanced user who is
already familiar with the material in the Administrator's Manual
and wishes to know detailed information on what Condor does when
starting and stopping jobs.

When Condor is about to start a job, the \Condor{startd} on the execute
machine spawns a \Condor{starter} process.  The \Condor{starter} then
creates:
\begin{enumerate}

\item a run account on the machine with a login name of
``condor-reuse-slotX'', where X is the slot number of the
\Condor{starter}.  This account is added to group Users.  This step is
skipped if the job is to be run using the submitting user's account
(see section \ref{sec:windows-run-as-owner}).

\item a new temporary working directory for the job on the execute machine.
This directory is
named ``dir\_XXX'', where XXX is the process ID of the \Condor{starter}.
The directory is created in the \MacroUNI{EXECUTE} directory as
specified in Condor's configuration file.  Condor then grants write
permission to this directory for the user account newly created for the
job.

\item a new, non-visible Window Station and Desktop for the job.
Permissions are set so that only the account that will run the job has
access rights to this Desktop.  Any windows created by this job are
not seen by anyone; the job is run in the background.  (Note: Setting
\Macro{USE\_VISIBLE\_DESKTOP} to True will allow the job to access the
default desktop instead of a newly created one.)

\end{enumerate}

Next, the \Condor{starter} (called the starter) contacts the
\Condor{shadow} (called the shadow) process, which is running on the
submitting machine, and pulls over the job's executable and input
files.  These files are placed into the temporary working directory
for the job.  After all files have been received, the starter spawns
the user's executable.  Its current working directory set to the
temporary working directory (that is, \MacroUNI{EXECUTE}/dir\_XXX,
where XXX is the process id of the \Condor{starter} daemon).

While the job is running, the starter closely monitors the CPU
usage and image size of all processes started by the job.
Every 20 minutes the starter sends this information,
along with the total size of all files contained in the job's
temporary working directory, to the shadow.
The shadow then
inserts this information into the job's ClassAd so that policy and
scheduling expressions can make use of this dynamic information.

If the job exits of its own accord (that is, the job completes),
the starter
first terminates any processes started by the job which could still be
around if the job did not clean up after itself.
The starter examines the job's temporary working directory for any
files which have been created or modified and sends these files back
to the shadow running on the submit machine.
The shadow
places these files into the \Opt{initialdir} specified in the
submit description file; if no \Opt{initialdir} was specified, the files go
into the directory where the user invoked \Condor{submit}.
Once all the output files are safely transferred back,
the job is removed from the queue.
If, however, the \Condor{startd} forcibly kills the job before all output files
could be transferred, the job is not removed from the queue but instead
switches back to the Idle state.  

If the \Condor{startd} decides to vacate a job prematurely,
the starter sends a WM\_CLOSE message to the job.
If the job spawned multiple child processes, the WM\_CLOSE message is only
sent to the parent process (that is, the one started by the starter).
The
WM\_CLOSE message is the preferred way to terminate a process on Windows,
since this method allows the job to cleanup and free any resources it may
have allocated.
When the job exits, the starter cleans up any processes left behind.
At this point, if \Opt{transfer\_files} is set to
\Arg{ONEXIT} (the default) in the job's submit description file,
the job switches from states, from Running to Idle,
and no files are transferred back.
If \Opt{transfer\_files} is set to \Arg{ALWAYS}, then any files
in the job's temporary working directory which were changed or modified are
first sent back to the submitting machine.
But this time, the shadow places these
so-called intermediate files into a subdirectory created in the
\MacroUNI{SPOOL} directory on the submitting machine
(\MacroUNI{SPOOL} is specified in Condor's configuration file).
The job is then switched back to the Idle state until Condor finds
a different machine on which to run.
When the job is started again,
Condor places into the job's temporary working directory the executable
and input files as before,
\emph{plus} any files stored in the submit machine's \MacroUNI{SPOOL} directory for that job.  

\Note A Windows console process can intercept a WM\_CLOSE message
via the Win32 SetConsoleCtrlHandler() function if it needs to do special
cleanup work at vacate time; a WM\_CLOSE message
generates a CTRL\_CLOSE\_EVENT.  See SetConsoleCtrlHandler() in the Win32
documentation for more info.

\Note The default handler in Windows for a WM\_CLOSE message is for the
process to exit.  Of course, the job could be coded to ignore it and not
exit, but eventually the \Condor{startd} will become impatient and hard-kill
the job (if that is the policy desired by the administrator).

Finally, after the job has left and any files transferred back, the
starter deletes the temporary working directory, the temporary account
(if one was created), the WindowStation, and the Desktop before
exiting.  If the starter should terminate abnormally, the
\Condor{startd} attempts the clean up.  If for some reason the
\Condor{startd} should disappear as well (that is, if the entire
machine was power-cycled hard), the \Condor{startd} will clean up when
Condor is restarted.

%%%%%%%%%%%%%%%%%%%%%%%%%%%%%%%%%%%%%%%%%%%%%%%%%%%%%%%%%%%%%%%%%%%%%%
\subsection{Security Considerations in Condor for Windows}
%%%%%%%%%%%%%%%%%%%%%%%%%%%%%%%%%%%%%%%%%%%%%%%%%%%%%%%%%%%%%%%%%%%%%%

% WRT the backslash character, extra spaces are added before it
% as viewed from the html generated.
%   Karen has tried
%         \File{C:$\backslash$WINNT}
%         \File{C:\Bs WINNT}
% and neither works.

On the execute machine (by default), the user job is run using the
access token of an account dynamically created by Condor which has
bare-bones access rights and privileges.  For instance, if your
machines are configured so that only Administrators have write access
to
%\File{C:\Bs WINNT},
\verb@C:\WINNT@, then certainly no Condor job run on that machine
would be able to write anything there.  The only files the job should
be able to access on the execute machine are files accessible by the
Users and Everyone groups, and files in the job's temporary working
directory.  Of course, if the job is configured to run using the
account of the submitting user (as described in section
\ref{sec:windows-run-as-owner}), it will be able to do anything that
the user is able to do on the execute machine it runs on.

On the submit machine, Condor impersonates the submitting user, therefore
the File Transfer mechanism has the same access rights as the submitting
user.  For example, say only Administrators can write to
%\File{C:\Bs WINNT}
\verb@C:\WINNT@
on the submit machine,
and a user gives the following to \Condor{submit} :
\begin{verbatim}
         executable = mytrojan.exe
         initialdir = c:\winnt
         output = explorer.exe
         queue
\end{verbatim}
Unless that user is in group Administrators, Condor will not permit
\File{explorer.exe} to be overwritten.  

If for some reason the submitting user's account disappears between the time
\Condor{submit} was run and when the job runs, Condor is not able to check
and see if the now-defunct submitting user has read/write access to a given
file.  In this case, Condor will ensure that group ``Everyone'' has read or
write access to any file the job subsequently tries to read or write.  This
is in consideration for some network setups, where the user account only
exists for as long as the user is logged in.

Condor also provides protection to the job queue.  It would be bad if the
integrity of the job queue is compromised, because a malicious user could
remove other user's jobs or even change what executable a user's job will
run.  To guard against this, in Condor's default configuration all connections to the \Condor{schedd} (the
process which manages the job queue on a given machine) are authenticated
using Windows' eSSPI security layer.  The user is then authenticated
using the same challenge-response protocol that Windows uses to authenticate
users to Windows file servers.  Once authenticated, the only users
allowed to edit job entry in the queue are:
\begin{enumerate}
\item the user who originally submitted that job (i.e. Condor allows users
to remove or edit their own jobs)
\item users listed in the \File{condor\_config} file parameter
\MacroNI{QUEUE\_SUPER\_USERS}.  In the default configuration, only the
``SYSTEM'' (LocalSystem) account is listed here.  
\end{enumerate}
\Warn Do not remove ``SYSTEM'' from \MacroNI{QUEUE\_SUPER\_USERS}, or
Condor itself will not be able to access the job queue when needed.  If the
LocalSystem account on your machine is compromised, you have all sorts of
problems!

To protect the actual job queue files themselves, the Condor installation
program will automatically set permissions on the entire Condor release
directory so that only Administrators have write access.

Finally, Condor has all the IP/Host-based security mechanisms present
in the full-blown version of Condor.  See section~\ref{sec:Host-Security}
starting on page~\pageref{sec:Host-Security} for complete information
on how to allow/deny access to Condor based upon machine host name or
IP address.


%%%%%%%%%%%%%%%%%%%%%%%%%%%%%%%%%%%%%%%%%%%%%%%%%%%%%%%%%%%%%%%%%%%%%%
\subsection{\label{sec:network-files-solutions}Network files and Condor}
%%%%%%%%%%%%%%%%%%%%%%%%%%%%%%%%%%%%%%%%%%%%%%%%%%%%%%%%%%%%%%%%%%%%%%

Condor can work well with a network file server.  The recommended
approach to having jobs access files on network shares is to configure
jobs to run using the security context of the submitting user (see
section \ref{sec:windows-run-as-owner}).  If this is done, the job
will be able to access resources on the network in the same way as the
user can when logged in interactively.

In some environments, running jobs as their submitting users is not a
feasible option.  This section outlines some possible
alternatives. The heart of the difficulty in this case is that on the
execute machine, Condor creates a temporary user that will run the
job.  The file server has never heard of this user before.

Choose one of these methods to make it work:

\begin{itemize}
\item METHOD A: access the file server as a different user via a net use command
with a login and password
\item METHOD B: access the file server as guest
\item METHOD C: access the file server with a "NULL" descriptor
\item METHOD D: create and have Condor use a special account 
\item METHOD E: use the contrib module from the folks at Bristol University
\end{itemize}

All of these methods have advantages and disadvantages.

Here are the methods in more detail:

METHOD A - access the file server as a different user via a net use command 
with a login and password

Example: you want to copy a file off of a server before running it....

\footnotesize
\begin{verbatim}
   @echo off
   net use \\myserver\someshare MYPASSWORD /USER:MYLOGIN
   copy \\myserver\someshare\my-program.exe
   my-program.exe
\end{verbatim}
\normalsize

The idea here is to simply authenticate to the file server with a different 
login than the temporary Condor login.  This is easy with the "net use" 
command as shown above.  Of course, the obvious disadvantage is this user's 
password is stored and transferred as clear text.

METHOD B - access the file server as guest

Example: you want to copy a file off of a server before running it as GUEST

\begin{verbatim}
   @echo off
   net use \\myserver\someshare
   copy \\myserver\someshare\my-program.exe
   my-program.exe
\end{verbatim}

In this example, you'd contact the server MYSERVER as the Condor temporary 
user.  However, if you have the GUEST account enabled on MYSERVER, you will 
be authenticated to the server as user "GUEST".  If your file permissions 
(ACLs) are setup so that either user GUEST (or group EVERYONE) has access 
the share "someshare" and the directories/files that live there, you can 
use this method.  The downside of this method is you need to enable the 
GUEST account on your file server.   \Warn This should be done *with 
extreme caution* and only if your file server is well protected behind a 
firewall that blocks SMB traffic.

METHOD C - access the file server with a "NULL" descriptor

One more option is to use NULL Security Descriptors.  In this way, you
can specify which shares are accessible by NULL Descriptor by adding
them to your registry.  You can then use the batch file wrapper like:

\begin{verbatim}
net use z: \\myserver\someshare /USER:""
z:\my-program.exe
\end{verbatim}

so long as 'someshare' is in the list of allowed NULL session shares.  To
edit this list, run regedit.exe and navigate to the key:

\begin{verbatim}
HKEY_LOCAL_MACHINE\
   SYSTEM\
     CurrentControlSet\
       Services\
         LanmanServer\
           Parameters\
             NullSessionShares
\end{verbatim}

and edit it.  unfortunately it is a binary value, so you'll then need to
type in the hex ASCII codes to spell out your share.  each share is
separated by a null (0x00) and the last in the list is terminated with
two nulls.

although a little more difficult to set up, this method of sharing is a
relatively safe way to have one quasi-public share without opening the
whole guest account.  you can control specifically which shares can be 
accessed or not via the registry value mentioned above.


METHOD D -  create and have Condor use a special account

Create a permanent account (called condor-guest in this description)
under which Condor will run jobs.
On all Windows machines, and on the file server, create the
condor-guest account.

On the network file server, give the condor-guest user permissions
to access files needed to run Condor jobs.

Securely store the password of the condor-guest user in the
Windows registry using \Condor{store\_cred} on all Windows
machines.

Tell Condor to use the condor-guest user as the owner of jobs,
when required.
Details for this are in 
section~\ref{sec:RunAsNobody}.

METHOD E -  access with the contrib module from Bristol

Another option: some hardcore Condor users at Bristol University developed 
their own module for starting jobs under Condor NT to access file 
servers.  It involves storing submitting user's passwords on a centralized 
server.  Below I have included the README from this contrib module, which 
will soon appear on our website within a week or two.  If you want it 
before that, let me know, and I could e-mail it to you.

Here is the README from the Bristol Condor contrib module:

\begin{verbatim}
README
Compilation Instructions
Build the projects in the following order

CondorCredSvc
CondorAuthSvc
Crun
Carun
AfsEncrypt
RegisterService
DeleteService
Only the first 3 need to be built in order. This just makes sure that the 
RPC stubs are correctly rebuilt if required. The last 2 are only helper 
applications to install/remove the services. All projects are Visual Studio 
6 projects. The nmakefiles have been exported for each. Only the project 
for Carun should need to be modified to change the location of the AFS 
libraries if needed.

Details
CondorCredSvc
CondorCredSvc is a simple RPC service that serves the domain account 
credentials. It reads the account name and password from the registry of 
the machine it's running on. At the moment these details are stored in 
clear text under the key

HKEY_LOCAL_MACHINE\Software\Condor\CredService

The account name and password are held in REG_SZ values "Account" and 
"Password" respectively. In addition there is an optional REG_SZ value 
"Port" which holds the clear text port number (e.g. "1234"). If this value 
is not present the service defaults to using port 3654.

At the moment there is no attempt to encrypt the username/password when it 
is sent over the wire - but this should be reasonably straightforward to 
change. This service can sit on any machine so keeping the registry entries 
secure ought to be fine. Certainly the ACL on the key could be set to only 
allow administrators and SYSTEM access.

CondorAuthSvc and Crun
These two programs do the hard work of getting the job authenticated and 
running in the right place. CondorAuthSvc actually handles the process 
creation while Crun deals with getting the winstation/desktop/working 
directory and grabbing the console output from the job so that Condor's 
output handling mechanisms still work as advertised. Probably the easiest 
way to see how the two interact is to run through the job creation process:

The first thing to realize is that condor itself only runs Crun.exe. Crun 
treats its command line parameters as the program to really run. e.g. "Crun 
\\mymachine\myshare\myjob.exe" actually causes 
\\mymachine\myshare\myjob.exe to be executed in the context of the domain 
account served by CondorCredSvc. This is how it works:

When Crun starts up it gets its window station and desktop - these are the 
ones created by condor. It also gets its current directory - again already 
created by condor. It then makes sure that SYSTEM has permission to modify 
the DACL on the window station, desktop and directory. Next it creates a 
shared memory section and copies its environment variable block into it. 
Then, so that it can get hold of STDOUT and STDERR from the job it makes 
two named pipes on the machine it's running on and attaches a thread to 
each which just prints out anything that comes in on the pipe to the 
appropriate stream. These pipes currently have a NULL DACL, but only one 
instance of each is allowed so there shouldn't be any issues involving 
malicious people putting garbage into them. The shared memory section and 
both named pipes are tagged with the ID of Crun's process in case we're on 
a multi-processor machine that might be running more than one job. Crun 
then makes an RPC call to CondorAuthSvc to actually start the job, passing 
the names of the window station, desktop, executable to run, current 
directory, pipes and shared memory section (it only attempts to call 
CondorAuthSvc on the same machine as it is running on). If the jobs starts 
successfully it gets the process ID back from the RPC call and then just 
waits for the new process to finish before closing the pipes and exiting. 
Technically, it does this by synchronizing on a handle to the process and 
waiting for it to exit. CondorAuthSvc sets the ACL on the process to allow 
EVERYONE  to synchronize on it.

[ Technical note: Crun adds "C:\WINNT\SYSTEM32\CMD.EXE /C" to the start of 
the command line. This is because the process is created with the network 
context of the caller i.e. LOCALSYSTEM. Pre-pending cmd.exe gets round any 
unexpected "Access Denied" errors. ]

If Crun gets a WM_CLOSE (CTRL_CLOSE_EVENT) while the job is running it 
attempts to stop the job, again with an RPC call to CondorAuthSvc passing 
the job's process ID.

CondorAuthSvc runs as a service under the LOCALSYSTEM account and does the 
work of starting the job. By default it listens on port 3655, but this can 
be changed by setting the optional REG_SZ value "Port" under the registry key

HKEY_LOCAL_MACHINE\Software\Condor\AuthService

(Crun also checks this registry key when attempting to contact 
CondorAuthSvc.) When it gets the RPC to start a job CondorAuthSvc first 
connects to the pipes for STDOUT and STDERR to prevent anyone else sending 
data to them. It also opens the shared memory section with the environment 
stored by Crun.  It then makes an RPC call to CondorCredSvc (to get the 
name and password of the domain account) which is most likely running on 
another system. The location information is stored in the registry under 
the key

HKEY_LOCAL_MACHINE\Software\Condor\CredService

The name of the machine running CondorCredSvc must be held in the REG_SZ 
value "Host". This should be the fully qualified domain name of the 
machine. You can also specify the optional "Port" REG_SZ value in case you 
are running CondorCredSvc on a different port.

Once the domain account credentials have been received the account is 
logged on through a call to LogonUser. The DACLs on the window station, 
desktop and current directory are then modified to allow the domain account 
access to them and the job is started in that window station and desktop 
with a call to CreateProcessAsUser. The starting directory is set to the 
same as sent by Crun, STDOUT and STDERR handles are set to the named pipes 
and the environment sent by Crun is used. CondorAuthSvc also starts a 
thread which waits on the new process handle until it terminates to close 
the named pipes. If the process starts correctly the process ID is returned 
to Crun.

If Crun requests that the job be stopped (again via RPC), CondorAuthSvc 
loops over all windows on the window station and desktop specified until it 
finds the one associated with the required process ID. It then sends that 
window a WM_CLOSE message, so any termination handling built in to the job 
should work correctly.

[Security Note: CondorAuthSvc currently makes no attempt to verify the 
origin of the call starting the job. This is, in principal, a bad thing 
since if the format of the RPC call is known it could let anyone start a 
job on the machine in the context of the domain user. If sensible security 
practices have been followed and the ACLs on sensitive system directories 
(such as C:\WINNT) do not allow write access to anyone other than trusted 
users the problem should not be too serious.]

Carun and AFSEncrypt
Carun and AFSEncrypt are a couple of utilities to allow jobs to access AFS 
without any special recompilation. AFSEncrypt encrypts an AFS 
username/password into a file (called .afs.xxx) using a simple XOR 
algorithm. It's not a particularly secure way to do it, but it's simple and 
self-inverse. Carun reads this file and gets an AFS token before running 
whatever job is on its command line as a child process. It waits on the 
process handle and a 24 hour timer. If the timer expires first it briefly 
suspends the primary thread of the child process and attempts to get a new 
AFS token before restarting the job, the idea being that the job should 
have uninterrupted access to AFS if it runs for more than 25 hours (the 
default token lifetime). As a security measure, the AFS credentials are 
cached by Carun in memory and the .afs.xxx file deleted as soon as the 
username/password have been read for the first time.

Carun needs the machine to be running either the IBM AFS client or the 
OpenAFS client to work. It also needs the client libraries if you want to 
rebuild it.

For example, if you wanted to get a list of your AFS tokens under Condor 
you would run the following:

Crun \\mymachine\myshare\Carun tokens.exe

Running a job
To run a job using this mechanism specify the following in your job 
submission (assuming Crun is in C:\CondorAuth):

Executable= c:\CondorAuth\Crun.exe
Arguments = \\mymachine\myshare\carun.exe 
\\anothermachine\anothershare\myjob.exe
Transfer_Input_Files = .afs.xxx

along with your usual settings.

Installation
A basic installation script for use with the Inno Setup installation 
package compiler can be found in the Install folder.
\end{verbatim}




%%%%%%%%%%%%%%%%%%%%%%%%%%%%%%%%%%%%%%%%%%%%%%%%%%%%%%%%%%%%%%%%%%%%%%
\subsection{Interoperability between Condor for Unix and Condor for Windows}
%%%%%%%%%%%%%%%%%%%%%%%%%%%%%%%%%%%%%%%%%%%%%%%%%%%%%%%%%%%%%%%%%%%%%%

Unix machines and Windows machines running Condor can happily
co-exist in the same Condor pool without any problems.
Jobs submitted on Windows can run on Windows or Unix,
and jobs submitted on Unix can run on Unix or Windows.
Without any specification
(using the \AdAttr{requirements} expression in the submit description file),
the default behavior will be to 
require the execute machine to be of the same architecture and operating
system as the submit machine.

There is absolutely no need to run more than one Condor central manager,
even if you have both Unix and Windows machines.  The Condor central manager
itself can run on either Unix or Windows; there is no advantage to choosing
one over the other.  Here at University of Wisconsin-Madison, for
instance, we have hundreds of Unix and
Windows machines in our Computer Science Department Condor pool.

%%%%%%%%%%%%%%%%%%%%%%%%%%%%%%%%%%%%%%%%%%%%%%%%%%%%%%%%%%%%%%%%%%%%%%
\subsection{Some differences between Condor for Unix -vs- Condor for Windows}
%%%%%%%%%%%%%%%%%%%%%%%%%%%%%%%%%%%%%%%%%%%%%%%%%%%%%%%%%%%%%%%%%%%%%%

\begin{itemize}

\item On Unix, we recommend the creation of a ``\textit{condor}'' account
when installing Condor.  On Windows, this is not necessary, as Condor is
designed to run as a system service as user LocalSystem.

\item On Unix, Condor finds the \File{condor\_config} main configuration
file by looking in \Tilde condor, in /etc, or via an environment variable.
On NT, the location of \File{condor\_config} file is determined
via the registry key \File{HKEY\_LOCAL\_MACHINE/Software/Condor}.
You can override this value by setting an environment variable named
\Env{CONDOR\_CONFIG}.

\item On Unix, in the VANILLA universe at job vacate time Condor sends the
job a softkill signal defined in the submit-description file (defaults to
SIGTERM).  On NT, Condor sends a WM\_CLOSE message to the job at vacate
time.

\item On Unix, if one of the Condor daemons has a fault, a core file
will be created in the \MacroUNI{Log} directory.  On Condor NT, a
``core'' file will also be created, but instead of a memory dump of the
process it will be a very short ASCII text file which describes what
fault occurred and where it happened.  This information can be used by
the Condor developers to fix the problem.

\end{itemize}

\index{platform-specific information!Windows|)}
