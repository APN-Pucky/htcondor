%%%%%%%%%%%%%%%%%%%%%%%%%%%%%%%%%%%%%%%%%%%%%%%%%%%%%%%%%%%%%%%%%%%%%%
\section{\label{sec:platform-macos}Macintosh OS X}
\index{platform-specific information!Macintosh OS X}
%%%%%%%%%%%%%%%%%%%%%%%%%%%%%%%%%%%%%%%%%%%%%%%%%%%%%%%%%%%%%%%%%%%%%%

This section provides information specific to the Macintosh OS X port of
Condor.
The Macintosh port of Condor is more accurately a port of Condor to
Darwin, the BSD core of OS X. Condor uses the Carbon library only to
detect keyboard activity, and it does not use Cocoa at all.
Condor on the Macintosh is a relatively new port, and it 
is not yet well-integrated
into the Macintosh environment. 

Condor on the Macintosh has a few shortcomings:
\begin{itemize}
\item Users connected to the Macintosh via \Prog{ssh} are not
noticed for console activity.
\item The memory size of threaded programs is reported incorrectly.
\item No Macintosh-based installer is provided.
\item The example start up scripts do not follow Macintosh conventions.
\item Kerberos is not supported.
\end{itemize}


%%%%%%%%%%%%%%%%%%%%%%%%%%%%%%%%%%%%%%%%%%%%%%%%%%%%%%%%%%%%%%%%%%%%%%
\subsection{\label{sec:platform-macos-panther}Macintosh OS X Panther (10.3)}
%%%%%%%%%%%%%%%%%%%%%%%%%%%%%%%%%%%%%%%%%%%%%%%%%%%%%%%%%%%%%%%%%%%%%%
\index{platform-specific information!Mac OS 10.3}

Condor \VersionNotice 
is built on a machine running Panther (10.3).
Therefore, these binaries work on any PPC machine running MacOS 10.3.


%%%%%%%%%%%%%%%%%%%%%%%%%%%%%%%%%%%%%%%%%%%%%%%%%%%%%%%%%%%%%%%%%%%%%%
\subsection{\label{sec:platform-macos-tiger}Macintosh OS X Tiger (10.4)}
%%%%%%%%%%%%%%%%%%%%%%%%%%%%%%%%%%%%%%%%%%%%%%%%%%%%%%%%%%%%%%%%%%%%%%
\index{platform-specific information!Mac OS 10.4}

Even though Condor \VersionNotice 
is built on a machine running Panther (10.3), the resulting binaries
are known to work on PPC machines running MacOS 10.4.


%%%%%%%%%%%%%%%%%%%%%%%%%%%%%%%%%%%%%%%%%%%%%%%%%%%%%%%%%%%%%%%%%%%%%%
\subsection{\label{sec:platform-macos-tiger-x86}Macintosh OS X Tiger
 (10.4) for Intel CPUs}
%%%%%%%%%%%%%%%%%%%%%%%%%%%%%%%%%%%%%%%%%%%%%%%%%%%%%%%%%%%%%%%%%%%%%%
\index{platform-specific information!Mac OS 10.4 for Intel}

Condor does not yet provide Universal binaries or natively-built
binaries for Intel MacOS 10.4.
Given MacOS's support for running PPC binaries in emulation (their
``Rosetta'' system), the existing Condor binaries built for PPC MacOS
10.3 might work.
Very few parts of the Condor system are performance bottle-necks, so
running in emulation will hopefully cause no problems.
The end-user applications (the jobs submitted to Condor), should
probably be Universal binaries or x86-specific, to provide the most
throughput on these machines. 

