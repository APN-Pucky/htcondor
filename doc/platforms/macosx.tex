%%%%%%%%%%%%%%%%%%%%%%%%%%%%%%%%%%%%%%%%%%%%%%%%%%%%%%%%%%%%%%%%%%%%%%
\section{\label{sec:platform-macos}Macintosh OS X}
\index{platform-specific information!Macintosh OS X}
%%%%%%%%%%%%%%%%%%%%%%%%%%%%%%%%%%%%%%%%%%%%%%%%%%%%%%%%%%%%%%%%%%%%%%

This section provides information specific to the Macintosh OS X port of
Condor.
The Macintosh port of Condor is more accurately a port of Condor to
Darwin, the BSD core of OS X. Condor uses the Carbon library only to
detect keyboard activity, and it does not use Cocoa at all.
Condor on the Macintosh is a relatively new port, and it 
is not yet well-integrated
into the Macintosh environment. 

Condor on the Macintosh has a few shortcomings:
\begin{itemize}
\item Users connected to the Macintosh via \Prog{ssh} are not
noticed for console activity.
\item The memory size of threaded programs is reported incorrectly.
\item No Macintosh-based installer is provided.
\item The example start up scripts do not follow Macintosh conventions.
\item Kerberos is not supported.
\end{itemize}


Condor does not yet provide Universal binaries for MacOSX.
There are separate down loadable packages for both PowerPC (ppc) and
Intel (x86) architectures, so please ensure you are using the right
Condor binaries for the platform you are trying to run on.
