
This chapter provides a basic, high-level overview of Condor, including
Condor�s major features and limitations. Because Condor is a system to
implement a High-Throughput Computing environment, this section begins
defining what is meant by High-Throughput Computing.


\section{What is High-Throughput Computing (HTC) ?}

For many research and engineering projects, the quality of the research
or the product is heavily dependent upon the quantity of computing
cycles available. It is not uncommon to find problems that require weeks
or months of computation to solve. Scientists and engineers engaged in
this sort of work need a computing environment that delivers large
amounts of computational power over a long period of time. Such an
environment is called a High- Throughput Computing (HTC) environment. In
contrast, High Performance Computing (HPC) environments deliver a
tremendous amount of compute power over a short period of time. HPC
environments are often measured in terms of FLoating point Operations
per Second (FLOPS). But a growing community is not concerned about
FLOPS, as their problems are on a much larger scale. These people are
concerned with floating point operations per month or per year. They are
more interested in how many jobs they can complete over a long period of
time instead of how fast an individual job can complete.

\section{HTC and Distributed Ownership}

The key to HTC is the efficiently harness the use of all available
resources. Years ago, the engineering and scientific community relied on
large centralized mainframe and/or big-iron supercomputers to do
computational work. A large number of individuals and groups would have
to pool their financial resources to afford such a machine. Users would
then have to wait for their turn on the mainframe, and would only have a
certain amount of time allotted to them. While this environment was
inconvenient for the users, it was very efficient, since the mainframe
was busy nearly all the time.

As computers became smaller, faster, and cheaper, users started to move
away from centralized mainframes and started purchasing personal desktop
workstations and PCs. An individual or small group could afford a
computing resource that was available whenever they wanted it. It was
usually far slower than the large centralized machine, but since they
had exclusive access, it was worth it. Now, instead of one giant
computer for a large institution, there might be hundreds or thousands
of personal computers. This is an environment of distributed ownership,
where individuals throughout an organization own their own resources.
The total computational power of the institution as a whole might rise
dramatically as the result of such a change, but because of distributed
ownership individuals could not capitalize on the institutional growth of
computing power. And while distributed ownership is more convenient for
the users, it is also much less efficient. Many personal desktop
machines sit idle for very long periods of time while their owners are
busy doing other things (such as at lunch, in meetings, or at home
sleeping). 

\section{What is Condor ?}
\label{sec:what-is-condor}

Condor is a software system that creates a High Throughput Computing
(HTC) environment by effectively harnessing the power of a cluster of
UNIX workstations on a network. Although Condor can manage a dedicated
cluster of workstations, a key appeal of Condor is its ability to
effectively harness non-dedicated, preexisting resources in a
distributed ownership setting such as machines sitting on people�s desks
in offices and labs. 

\subsection{A Hunter of Available Workstations}

Instead of running a CPU-intensive job in the background on their own
workstation, users submit their job to Condor. Condor will then find an
available machine on the network and begin running the job on
that machine. When Condor detects that a machine running a Condor job
would no longer be available (perhaps because the owner of the machine
came back from lunch and started typing on the keyboard), Condor
checkpoints the job and then migrates it over the network to a
different machine which would otherwise be idle. Condor then restarts
the job on the new machine from precisely where it left off. If no
machine on the network is currently available, then the job is stored in
a queue on disk until a machine becomes available.

So, for example, say you submit a compute job that typically takes 5
hours to run. Condor may start running it on Machine A, but after 3
hours Condor notices activity on the keyboard. So Condor checkpoints
your job and migrates it to Machine B. After two hours on Machine B,
your job completes (and Condor notifies you via email). 

Perhaps you have to run this 5 hour compute job 250 different times
(perhaps on 250 different data sets). In this case, Condor can be a real
time saver. With one command you can submit all 250 runs into Condor.
Depending upon the number of machines in your organization�s Condor
pool, there could be dozens or even hundreds of otherwise idle machines
(especially at night, for example) at any given moment running your
job. 

Condor makes it easy to maximize the number of machines which can run
your job. Because Condor does not require participating machines to
share file systems (via NFS or AFS for example), machines across the
entire enterprise can run your job, including machines in different
administrative domains. Condor does not even require you to have an
account (login) on machines where it runs your job. Condor can do this
because of its \Term{Remote System Call} technology, which traps
operating system calls for such operations as reading/writing from disk
files and sends them back over the network to be performed on the machine
where the job was submitted. 

\subsection{Effective Resource Management}

In addition to migrating jobs to available machines, Condor provides
sophisticated and distributed resource management. Match-making resource
owners with resource consumers is the cornerstone of a successful HTC
environment. Unlike many other compute cluster resource management
systems which attach properties to the job queues themselves (resulting
in user confusion over which queue to use as well as administrative
hassle in constantly adding and editing queue properties to satisfy user
demands), Condor implements a clean design called \Term{ClassAds}. 

ClassAds work in a fashion similar to the newspaper classified
advertising want-ads. All machines in the Condor pool advertise their
resource properties, such as available RAM memory, CPU type and speed,
virtual memory size, physical location, current load average, and many
other static and dynamic properties, into a \Term{Resource Offer} ad. Likewise,
when submitting a job, users can specify a Resource Request ad which
defines both the required and desired set of resources to run the job.
Similarly, a Resource Offer ad can define requirements and preferences.
Condor then acts as a broker by matching and ranking Resource
Offer ads with Resource Request ads, making certain that all
requirements in both ads are satisfied. During this match-making
process, Condor also takes several layers of priority values into
consideration: the priority the user assigned to the Resource Request
ad, the priority of the user which submitted the ad, and desire of
machines in the pool to accept certain types of ads over others. 

\section{Distinguishing Features}

\begin{description}
	\item[Checkpoint and Migration.] Users of Condor may be assured that
their jobs will eventually complete even in an opportunistic computing
environment. If a user submits a job to Condor which runs on somebody
else's workstation, but the job is not finished when the workstation
owner returns, the job can be checkpointed and restarted as soon as
possible on another machine and will keep on executing in this manner
until the job is completed. Condor�s Periodic Checkpoint feature can
periodically checkpoint the job even in lieu of migration in order to
safeguard the accumulated computation time on job from being lost in the
event of a system failure (such as the machine being shutdown, a crash,
etc). 
	\item[Remote System Calls.] In Condor�s Standard Universe execution
mode, the local execution environment is preserved for remotely
executing processes via Remote System Calls. Users do not have to worry
about making data files available to remote workstations or even
obtaining a login account on remote workstations before Condor executes
their programs there. The program behaves under Condor as if it was
running as the user whom submitted the job on the workstation where it
was originally submitted, no matter on which machine it really ends up
executing on.
	\item[No Changes Necessary to User�s Source Code.] No special programming
is required to use Condor. Condor is able to run normal UNIX programs.
The checkpoint and migration of programs by Condor is transparent and
automatic, as is the use of Remote System Calls.  These facilities are
provided by Condor
and, if these facilities are desired, only requires the user to re-link
their program, not recompile it or change any code.
	\item[Sensitive to the desires of workstation owners.] ``Owners'' of
workstations have by default complete priority over their own machines.
Workstation owners are generally happy to let somebody else compute on
their machines while they are out, but they want their machines back
promptly upon returning, and they don't want to have to take special
action to regain control. Condor handles this automatically. 
	\item[ClassAds.]The ClassAd mechanism in Condor provides an extremely
flexible and semantic-free, expressive framework for match-making
Resource Requests and Resource Offers. One result is that users can
easily request practically any resource, both in terms of what their job
requires and/or what they desire for their job if available. For
instance, a User can require their job run a machine with 64 megs of
RAM, but state a preference for 128 megs if available. Likewise, machines
could state for example in a Resource Offer ad that they prefer to run jobs
from a certain set of users, and require that there be no interactive user
activity detectable between 9 am and 5 pm before starting a job.  Job
requirements/preferences and resource availability constraints can be
described in terms of powerful, arbitrary expressions, resulting in
Condor being flexible enough to adapt to nearly any desired policy. 
\end{description}

\section{Current Limitations}

\begin{description}

	\item[Limitations on Jobs which can Checkpointed] Although Condor can schedule and
run any type of process, Condor does have some limitations on jobs that it can
transparently checkpoint and migrate:
\begin{enumerate}
	\item On some platforms, specifically HPUX and Digital Unix
(OSF/1), shared libraries are not supported; therefore on these
platforms applications must be statically linked (Note: shared library
checkpoint support is available on IRIX, Solaris, and LINUX). 
	\item Only single process jobs are supported, i.e. the fork(2), exec(2),
system(3) and similar calls are not implemented.
	\item Signals and signal handlers are supported, but Condor reserves the 
SIGUSR2 and SIGTSTP signals and does not permit their use by user code.
	\item Many interprocess communication (IPC) calls are not supported, i.e. the 
socket(2), send(2), recv(2), and similar calls are not implemented.
	\item All file operations must be idempotent --- read-only and write-only file 
accesses work correctly, but programs which both read and write to the 
same file may not.
	\item Each Condor job that has been checkpointed has an associated 
\Term{checkpoint file} which is approximately the size of the address space of the 
process. Disk space must be available to store the checkpoint file on the 
submitting machines (or on the optional Checkpoint Server module).
\end{enumerate}

	Note: these limitations \underline{only} apply to jobs which Condor
has been asked to transparently checkpoint.  If job checkpointing is not
desired, the limitations above do not apply.

	\item[Security Implications.] Condor does a significant amount of work to prevent 
security hazards, but loopholes are known to exist.  Condor can be instructed 
to run user programs only as user ``nobody'', a user login which traditionally has very 
restricted access.  But even with access solely as user nobody, a sufficiently 
malicious individual could do such things as fill up /tmp (which is world writable) and/or gain 
read access to world readable files (which are the only files user nobody can 
access).  Furthermore, where security of the machines in the pool is a high concern, 
only machines where the ``root'' user on that machine can be trusted should be admitted
into the pool.  (Note: Condor provides the administrator with IP-based security mechanisms 
to enforce this).

	\item[Jobs need to be relinked to get Checkpointing and Remote System Calls] Although 
typically few to none source code changes are required,
Condor requires
that the jobs be relinked with the Condor libraries to offer checkpointing and
remote system calls. This often
precludes commercial software binaries from taking advantage of these services
 because commercial packages rarely make their object code
available. However, one can certainly still submit and run commercial packages in Condor 
and still take advantage of Condor's other services.

\end{description}

\section{Availability}
Condor is currently available as a free download from the Internet via the World Wide Web at  
URL \Url{http://www.cs.wisc.edu/condor}.
Binary distributions of Condor version 6.x are available for the following UNIX platforms 
detailed below in Table~\ref{supported-platforms}.  We define a platform to be an 
Architecture/Operating System combination.  Condor support for Windows NT is 
expected in an upcoming release.  

The Condor source code is no longer available for public download from the Internet.  If you 
desire the Condor source code, please contact the Condor Team in order to discuss it further 
(see Section~\ref{contact-info}, on page~\pageref{contact-info}).

\begin{center}
\begin{table}
\begin{tabular}{|ll|} \hline
\emph{Architecture} & \emph{Operating System} \\ \hline \hline
Hewlett Packard PA-RISC (both PA7000 and PA8000 series) & HPUX 10.20 \\ \hline
Sun SPARC Sun4{m,c}, Sun UltraSPARC & Solaris 2.5.x, 2.6 \\ \hline
Silicon Graphics MIPS (R4400, R4600, R8000, R10000) & IRIX 6.2, 6.3, 6.4 \\ \hline
Digital ALPHA & OSF/1 (Digital Unix) 4.x \\ \hline
Intel x86, Pentium series & Linux 2.x \\
 & Solaris 2.5.x, 2.6 \\ \hline
\end{tabular}
\caption{\label{supported-platforms}Condor \VersionNotice\ supported platforms}
\end{table}
\end{center}

\section{Future Work}
\Todo

\section{Contact Information}
\label{contact-info}

The latest software releases, FAQs, and publications/papers regarding Condor and other High 
Performance Computing research can be found at the official web site for Condor at  
\Url{http://www.cs.wisc.edu/condor}.

In addition, there is an email listgroup at \Url{mailto:condor-world@cs.wisc.edu}.  The Condor Team 
uses this email listgroup to announce new releases of Condor and other major Condor-related 
news items.  Membership into condor-world is automated by MajorDomo software.  To 
subscribe or unsubscribe from the the list, follow the instructions at  
\Url{http://www.cs.wisc.edu/condor/condor-world/condor-world.html}.  Because many of us receive 
too much email as it is, you�ll be happy to know that the condor-world email listgroup is 
moderated and only major announcements of wide interest are distributed.

Finally, you can reach the Condor Team directly.  The Condor Team is comprised of the 
developers and administrators of Condor at the University of Wisconsin-Madison. Condor 
questions, comments, pleas for help, requests for commercial contract consultation or support, 
are all welcome; just send Internet email to \Url{mailto:condor-admin@cs.wisc.edu}.  Please include your 
name, organization, and telephone number in your message.  If you are having trouble with 
Condor, please help us troubleshoot by including as much pertinent information as you can, 
including snippets of Condor log files.  An archive of potentially useful previous email responses 
to/from condor-admin@cs.wisc.edu is also available on the web at  
\Url{http://www.cs.wisc.edu/condor/condor-admin}.



