%%%%%%%%%%%%%%%%%%%%%%%%%%%%%%%%%%%%%%%%%%%%%%%%%%
\section{\label{sec:overview}High-Throughput Computing (HTC) and its Requirements}
%%%%%%%%%%%%%%%%%%%%%%%%%%%%%%%%%%%%%%%%%%%%%%%%%%

\index{HTCondor!overview|(}
\index{overview|(}
For many research and engineering projects, the quality of the research
or the product is heavily dependent upon the quantity of computing
cycles available.
It is not uncommon to find problems that require weeks
or months of computation to solve.
Scientists and engineers engaged in
this sort of work need a computing environment that delivers large
amounts of computational power over a long period of time.
Such an environment is called a High-Throughput Computing (HTC) environment.
\index{High-Throughput Computing (HTC)}
\index{HTC (High-Throughput Computing)}
In contrast, High Performance Computing (HPC)
\index{High-Performance Computing (HPC)}
\index{HPC (High-Performance Computing)}
environments deliver a
tremendous amount of compute power over a short period of time.
HPC environments are often measured in terms of FLoating point Operations
Per Second (FLOPS). 
A growing community is not concerned about operations per second,
but operations per month or per year.
Their problems are of a much larger scale.
They are
more interested in how many jobs they can complete over a long period of
time instead of how fast an individual job can complete.

The key to HTC is to efficiently harness the use of all available
resources. Years ago, the engineering and scientific community relied on
a large, centralized mainframe or a supercomputer to do
computational work. 
A large number of individuals and groups needed
to pool their financial resources to afford such a machine.
Users had to wait for their turn on the mainframe, 
and they had a limited amount of time allocated.
While this environment was inconvenient for users,
the utilization of the mainframe was high;
it was busy nearly all the time.

As computers became smaller, faster, and cheaper, 
users moved away from centralized mainframes and purchased personal desktop
workstations and PCs.
An individual or small group could afford a
computing resource that was available whenever they wanted it.
The personal computer is slower than the large centralized machine,
but it provides exclusive access.
Now, instead of one giant computer for a large institution,
there may be hundreds or thousands of personal computers.
This is an environment of distributed ownership,
\index{distributed ownership!of machines}
where individuals throughout an organization own their own resources.
The total computational power of the institution as a whole may rise
dramatically as the result of such a change,
but because of distributed ownership,
individuals have not been able to capitalize on the institutional growth of
computing power.
And, while distributed ownership is more convenient for the users,
the utilization of the computing power is lower.
Many personal desktop
machines sit idle for very long periods of time while their owners are
busy doing other things (such as being away at lunch, in meetings,
or at home sleeping). 

%%%%%%%%%%%%%%%%%%%%%%%%%%%%%%%%%%%%%%%%%%%%%%%%%%
\section{\label{sec:what-is-condor}HTCondor's Power}
%%%%%%%%%%%%%%%%%%%%%%%%%%%%%%%%%%%%%%%%%%%%%%%%%%

HTCondor is a software system that creates a High-Throughput Computing
(HTC) environment.
It effectively utilizes the computing power of workstations that
communicate over a network.
HTCondor can manage a dedicated cluster of workstations.
Its power comes from the
ability to effectively harness non-dedicated,
preexisting resources under distributed ownership. 
\index{distributed ownership!of machines}

A user submits the job to HTCondor.
HTCondor finds an available machine on the network and begins
running the job on that machine.
HTCondor has the capability to detect that a machine running a HTCondor job
is no longer available (perhaps because the owner of the machine
came back from lunch and started typing on the keyboard).
It can checkpoint 
\index{checkpoint}
the job and move (migrate)
\index{migration}
the jobs to a different machine which would otherwise be idle.
HTCondor continues the job on the new machine from
precisely where it left off.

In those cases where HTCondor can checkpoint and migrate a job,
HTCondor makes it easy to maximize the number of machines which can run
a job.
In this case, there is no requirement for machines to
share file systems (for example, with NFS or AFS),
so that machines across an entire enterprise can run a job,
including machines in different administrative domains.

HTCondor can be a real time saver when a job
must be run many (hundreds of) different times,
\index{job!multiple data sets}
perhaps with hundreds of different data sets.
With one command, all of the hundreds of jobs are submitted to HTCondor.
Depending upon the number of machines in the HTCondor pool,
dozens or even hundreds of otherwise idle machines
can be running the job at any given moment.

HTCondor does not require an account (login) on machines where it runs a job.
HTCondor can do this because of its \Term{remote system call}
\index{remote system call}
technology,
which traps
library calls for such operations as reading or writing from disk
files.
The calls are transmitted over the network to be performed on the machine
where the job was submitted.

\index{HTCondor!resource management}
\index{resource!management}
HTCondor provides powerful resource management by
match-making resource
\index{matchmaking}
owners with resource consumers.
This is the cornerstone of a successful HTC environment.
Other compute cluster resource management
systems attach properties to the job queues themselves,
resulting in user confusion over which queue to use as well as administrative
hassle in constantly adding and editing queue properties to satisfy user
demands.
HTCondor implements 
\Term{ClassAds},
\index{ClassAd}
a clean design that simplifies the user's submission of jobs.

ClassAds work in a fashion similar to the newspaper classified
advertising want-ads. All machines in the HTCondor pool advertise their
resource properties, both static and dynamic,
such as available RAM memory, CPU type, CPU speed,
virtual memory size, physical location, and current load average,
in a \Term{resource offer} ad.
\index{resource!offer}
A user specifies a \Term{resource request} ad
\index{resource!request}
when submitting a job.
The request defines both the required and a desired set of properties
of the resource to run the job.
HTCondor acts as a broker by matching and ranking resource
offer ads with resource request ads, making certain that all
requirements in both ads are satisfied.
During this match-making process,
HTCondor also considers several layers of priority values:
the priority the user assigned to the resource request ad,
the priority of the user which submitted the ad,
and desire of
machines in the pool to accept certain types of ads over others. 

%%%%%%%%%%%%%%%%%%%%%%%%%%%%%%%%%%%%%%%%%%%%%%%%%%
\section{Exceptional Features}
%%%%%%%%%%%%%%%%%%%%%%%%%%%%%%%%%%%%%%%%%%%%%%%%%%

\begin{description}
	\item[Checkpoint and Migration.] Where programs can be
linked with HTCondor libraries, users of HTCondor may be assured that
their jobs will eventually complete,
even in the ever changing environment that HTCondor
utilizes.
As a machine running a job submitted to HTCondor
becomes unavailable,
the job can be check pointed.
\index{checkpoint}
The job may continue after migrating 
\index{migration}
to another machine.
HTCondor's checkpoint feature 
\index{checkpoint!periodic}
periodically checkpoints a job even in lieu of migration in order to
safeguard the accumulated computation time on a job from being lost in the
event of a system failure, such as the machine being shutdown or a crash.
	\item[Remote System Calls.] 
\index{remote system call}
Despite running jobs on remote machines,
the HTCondor standard universe execution
mode preserves the local execution environment
via remote system calls. Users do not have to worry
about making data files available to remote workstations or even
obtaining a login account on remote workstations before HTCondor executes
their programs there. The program behaves under HTCondor as if it were
running as the user that submitted the job on the workstation where it
was originally submitted, no matter on which machine it really ends up
executing on.
	\item[No Changes Necessary to User's Source Code.] No special
programming is required to use HTCondor.
HTCondor is able to run non-interactive programs.
The checkpoint and migration of
programs by HTCondor is transparent and automatic, as is the use of
remote system calls.
If these facilities are desired, the user only
re-links the program.  The code is neither recompiled nor changed.
	\item[Pools of Machines can be Hooked Together.] Flocking is
a feature of HTCondor that allows jobs submitted within a first pool of
HTCondor machines to execute on a second pool.
The mechanism is flexible, following requests from the job
submission,
while allowing the second pool, or a subset of machines within
the second pool to set policies over the conditions under
which jobs are executed.
	\item[Jobs can be Ordered.] The ordering of job execution
required by dependencies among jobs in a set is easily handled.
The set of jobs is specified using a directed acyclic graph,
where each job is a node in the graph.
Jobs are submitted to HTCondor following the dependencies given
by the graph.
	\item[HTCondor Enables Grid Computing.] As grid computing
becomes a reality, HTCondor is already there.
The technique of glidein allows jobs submitted to HTCondor
to be executed on grid machines in various locations worldwide.
As the details of grid computing evolve, so does HTCondor's
ability, starting with Globus-controlled resources.
	\item[Sensitive to the Desires of Machine Owners.] The
owner of a machine has complete priority over the use
of the machine.
An owner is generally happy to let others compute on
the machine while it is idle, but wants it back
promptly upon returning. The owner does not want to take special
action to regain control. HTCondor handles this automatically. 
	\item[ClassAds.]The ClassAd mechanism 
\index{ClassAd}
in HTCondor provides an extremely
flexible, expressive framework for matchmaking
resource requests with resource offers.
Users can easily request both job requirements and job desires.
For example, a user can require that a job run on a machine
with 64 Mbytes of RAM,
but state a preference for 128 Mbytes, if available.
A workstation owner
can state a preference that the workstation runs jobs
from a specified set of users. 
The owner can also require that there be no interactive workstation
activity detectable at certain hours before HTCondor could
start a job.
Job requirements/preferences and resource availability constraints can be
described in terms of powerful expressions, resulting in
HTCondor's adaptation to nearly any desired policy. 
\end{description}
\index{HTCondor!overview|)}
\index{overview|)}

%%%%%%%%%%%%%%%%%%%%%%%%%%%%%%%%%%%%%%%%%%%%%%%%%%
\section{\label{sec:current-limitations}Current Limitations}
%%%%%%%%%%%%%%%%%%%%%%%%%%%%%%%%%%%%%%%%%%%%%%%%%%

\begin{description}

\index{HTCondor!limitations, under UNIX}
	\item[Limitations on Jobs which can Checkpointed] Although HTCondor can schedule and
run any type of process, HTCondor does have some limitations on jobs that it can
transparently checkpoint and migrate:


\begin{enumerate}

\index{Unix!fork}
\index{Unix!exec}
\index{Unix!system}
\item Multi-process jobs are not allowed.  This includes system calls such as
\Syscall{fork}, \Syscall{exec}, and \Syscall{system}.

\index{Unix!pipe}
\index{Unix!semaphore}
\index{Unix!shared memory}
\item Interprocess communication is not allowed.  This includes pipes, semaphores, and shared memory.

\index{Unix!socket}
\index{network}
\item Network communication must be brief.  A job \emph{may} make network
connections using system calls such as \Syscall{socket}, but a network
connection left open for long periods will delay checkpointing and migration.

\index{signal}
\index{signal!SIGUSR2}
\index{signal!SIGTSTP}
\item Sending or receiving the SIGUSR2 or SIGTSTP signals is not allowed.
Condor reserves these signals for its own use.  Sending or receiving all
other signals \emph{is} allowed.

\index{Unix!alarm}
\index{Unix!timer}
\index{Unix!sleep}
\item Alarms, timers, and sleeping are not allowed.  This includes system
calls such as \Syscall{alarm}, \Syscall{getitimer}, and \Syscall{sleep}.

\index{thread!kernel-level}
\index{thread!user-level}
\item Multiple kernel-level threads are not allowed.  However,
multiple user-level threads \emph{are} allowed.

\index{file!memory-mapped}
\index{Unix!mmap}
\item Memory mapped files are not allowed.  This includes system calls such
as \Syscall{mmap} and \Syscall{munmap}.

\index{file!locking}
\index{Unix!flock}
\index{Unix!lockf}
\item File locks are allowed, but not retained between checkpoints.

\index{file!read only}
\index{file!write only}
\item All files must be opened read-only or write-only.  A file opened
for both reading and writing will cause trouble if a job must be rolled back
to an old checkpoint image.  For compatibility reasons, a file opened
for both reading and writing will result in a warning but not an error.

\item A fair amount of disk space must be available on the submitting machine
for storing a job's checkpoint images.  A checkpoint image is approximately
equal to the virtual memory consumed by a job while it runs.  If disk space
is short, a special \Term{checkpoint server} can be designated for storing
all the checkpoint images for a pool.

\index{linking!dynamic}
\index{linking!static}
\item On Linux, the job must be statically linked. 
\Condor{compile} does this by default.

\index{Unix!large files} 
\item Reading to or writing from files larger than 2 GB is only supported
when the submit side \Condor{shadow} and the standard universe user job
application itself are both 64-bit executables.

\end{enumerate}





	Note: these limitations \emph{only} apply to jobs which HTCondor
has been asked to transparently checkpoint.  If job checkpointing is not
desired, the limitations above do not apply.

	\item[Security Implications.] HTCondor does a significant amount of
	work to prevent security hazards, but loopholes are known to exist.
	HTCondor can be instructed to run user programs only as the UNIX
	user nobody, a user login which traditionally has very 
	restricted access.
	But even with access solely as user nobody,
	a sufficiently malicious individual could do such things as fill up
	\File{/tmp} (which is world writable) and/or gain read access to
	world readable files.
	Furthermore, where the security of machines in the pool is a
	high concern, 
	only machines where the UNIX user root on that machine can be
	trusted should be admitted into the pool.
	HTCondor provides the administrator with extensive security mechanisms 
	to enforce desired policies.

	\item[Jobs Need to be Re-linked to get Checkpointing and Remote System Calls] Although 
typically no source code changes are required,
HTCondor requires
that the jobs be re-linked with the HTCondor libraries to take
advantage of checkpointing and remote system calls. This often
precludes commercial software binaries from taking advantage of these services
because commercial packages rarely make their object code
available. 
HTCondor's other services are still available for these commercial packages.

\end{description}

%%%%%%%%%%%%%%%%%%%%%%%%%%%%%%%%%%%%%%%%%%%%%%%%%%
\section{\label{sec:Availability}Availability}
%%%%%%%%%%%%%%%%%%%%%%%%%%%%%%%%%%%%%%%%%%%%%%%%%%
\index{HTCondor!platforms available}
\index{available platforms}
\index{supported platforms}
\index{platforms supported}
HTCondor is currently available as a free download from the Internet via the World Wide Web at  
URL \URL{http://www.cs.wisc.edu/condor/downloads-v2}.
Binary distributions of this HTCondor \VersionNotice\ release
are available for the platforms 
detailed in Table~\ref{table:supported-platforms}.  A platform is an 
architecture/operating system combination.  
HTCondor binaries are available for most major versions of Unix, as well as
Windows.  

\index{clipped platform!definition of}
\index{clipped platform!availability}
In the table, \Term{clipped} means that HTCondor does not support
checkpointing or remote system calls on the given platform. 
This means that \Term{standard} universe jobs are not supported.
Some clipped platforms will have further limitations with respect
to supported universes.
See section~\ref{sec:Choosing-Universe} on
page~\pageref{sec:Choosing-Universe} for more details on job universes
within HTCondor and their abilities and limitations.

The HTCondor source code is available for 
public download alongside the binary distributions.

% Karen's table
\begin{center}
\begin{table}[hbt]
\begin{tabular}{|p{6cm}p{7cm}|} \hline
\emph{Architecture} & \emph{Operating System} \\ \hline \hline
Intel x86 & - RedHat Enterprise Linux 5 \\
 & - Debian Linux 5.0 (lenny) \\
 & - Windows 2000 Professional and Server (Win NT 5.0) (clipped) \\
 & - Windows 2003 Server (Win NT 5.2) (clipped) \\
 & - Windows 2008 Server (Win NT 6.0) (clipped) \\
 & - Windows XP Professional (Win NT 5.1) (clipped) \\
 & - Windows Vista (Win NT 6.0) (clipped) \\
 & - Windows 7 (clipped) \\
Opteron x86\_64 & - Red Hat Enterprise Linux 5 \\ 
 & - Red Hat Enterprise Linux 6 \\
 & - Debian Linux 5.0 (lenny) \\
 & - Debian Linux 6.0 (squeeze) \\ \hline 
 & - Macintosh OS X 10.7 (clipped) \\ \hline
\end{tabular}
\caption{\label{table:supported-platforms}Supported platforms in HTCondor \VersionNotice}
\end{table}
\end{center}


\Note Other Linux distributions likely work, but are not tested
or supported.

For more platform-specific information about HTCondor's support for
various operating systems, see Chapter~\ref{platforms} on
page~\pageref{platforms}. 



Jobs submitted to the standard universe utilize \Condor{compile}
to relink programs with libraries provided by HTCondor.
Table~\ref{supported-compile} lists supported compilers by
platform for this \VersionNotice\ release.
Other compilers may work, but are not supported.

\index{HTCondor commands!condor\_compile!list of supported compilers}
\index{condor\_compile command!list of supported compilers}
\index{compilers!supported with condor\_compile}

% condor_compile works on. . .
% This table must be formatted oddly, to make the pdf version look OK.
\begin{center}
\begin{table}[hbt]
\begin{tabular}{|ll|l|} \hline
\textbf{Platform} & \textbf{Compiler} & \textbf{Notes}\\ \hline \hline
Red Hat Enterprise Linux 5 on x86 and x86\_64 & gcc, g++, and g77 & as shipped  \\ 
\hline
Red Hat Enterprise Linux 6 on x86\_64 & gcc, g++, and g77 & as shipped  \\ 
\hline
Debian Linux 5.0 (lenny) on x86 and x86\_64 & gcc, g++, gfortran & as shipped \\ 
\hline
Debian Linux 6.0 (squeeze) on x86\_64 & gcc, g++, gfortran & as shipped \\ 
\hline
\end{tabular}
\caption{\label{supported-compile}Supported compilers in HTCondor \VersionNotice}
\end{table}
\end{center}

%%%%%%%%%%%%%%%%%%%%%%%%%%%%%%%%%%%%%%%%%%%%%%%%%%
%%%%%%%%%%%%%%%%%%%%%%%%%%%%%%%%%%%%%%%%%%%%%%%%%%
\section{\label{sec:contributions}Contributions and Acknowledgments}
%%%%%%%%%%%%%%%%%%%%%%%%%%%%%%%%%%%%%%%%%%%%%%%%%%

\index{HTCondor!contributions}
The quality of the HTCondor project is enhanced by the contributions
of external organizations.
We gratefully acknowledge the following contributions. 

\begin{itemize}

\item{The Globus Alliance} (\URL{http://www.globus.org}), 
for code and assistance in developing HTCondor-G
and the Grid Security Infrastructure (GSI)
for authentication and authorization. 

\item{The GOZAL Project}
from the Computer Science Department
of the Technion Israel Institute of Technology
(\URL{http://www.technion.ac.il/}),
for their enhancements for HTCondor's High Availability.
The \Condor{had} daemon allows one of multiple machines to function
as the central manager for a HTCondor pool.
Therefore, if an acting central manager fails,
another can take its place.

%\item{INFN} (\URL{http://www.infn.it/})
%and EGEE (\URL{http://public.eu-egee.org/}), 
%for for the PBS and LSF GAHP.

\item{Micron Corporation} (\URL{http://www.micron.com/})
for the MSI-based installer for HTCondor on Windows.

\item{Paradyn Project} (\URL{http://www.paradyn.org/})
%and the Universitat Autònoma de Barcelona
and the Universitat Aut\`{o}noma de Barcelona
(\URL{http://www.caos.uab.es/}) for work on the Tool Daemon Protocol (TDP).

\end{itemize}

Our Web Services API acknowledges the use of gSOAP with their
requested wording:

\begin{itemize}
\item
Part of the software embedded in this product is gSOAP software.
Portions created by gSOAP are Copyright (C) 2001-2004 Robert A. van Engelen, Genivia inc. All Rights Reserved.

THE SOFTWARE IN THIS PRODUCT WAS IN PART PROVIDED BY GENIVIA INC AND ANY EXPRESS OR IMPLIED WARRANTIES, INCLUDING, BUT NOT LIMITED TO, THE IMPLIED WARRANTIES OF MERCHANTABILITY AND FITNESS FOR A PARTICULAR PURPOSE ARE DISCLAIMED. IN NO EVENT SHALL THE AUTHOR BE LIABLE FOR ANY DIRECT, INDIRECT, INCIDENTAL, SPECIAL, EXEMPLARY, OR CONSEQUENTIAL DAMAGES (INCLUDING, BUT NOT LIMITED TO, PROCUREMENT OF SUBSTITUTE GOODS OR SERVICES; LOSS OF USE, DATA, OR PROFITS; OR BUSINESS INTERRUPTION) HOWEVER CAUSED AND ON ANY THEORY OF LIABILITY, WHETHER IN CONTRACT, STRICT LIABILITY, OR TORT (INCLUDING NEGLIGENCE OR OTHERWISE) ARISING IN ANY WAY OUT OF THE USE OF THIS SOFTWARE, EVEN IF ADVISED OF THE POSSIBILITY OF SUCH DAMAGE.

\item
Some distributions of HTCondor include the Google Coredumper library
(\URL{http://goog-coredumper.sourceforge.net/}).  The Google Coredumper
library is released under these terms:

Copyright (c) 2005, Google Inc. \\
All rights reserved.

Redistribution and use in source and binary forms, with or without
modification, are permitted provided that the following conditions are
met:

	\begin{itemize}
    \item Redistributions of source code must retain the above copyright
notice, this list of conditions and the following disclaimer.

    \item Redistributions in binary form must reproduce the above
copyright notice, this list of conditions and the following disclaimer
in the documentation and/or other materials provided with the
distribution.

    \item Neither the name of Google Inc. nor the names of its
contributors may be used to endorse or promote products derived from
this software without specific prior written permission.
	\end{itemize}

THIS SOFTWARE IS PROVIDED BY THE COPYRIGHT HOLDERS AND CONTRIBUTORS
"AS IS" AND ANY EXPRESS OR IMPLIED WARRANTIES, INCLUDING, BUT NOT
LIMITED TO, THE IMPLIED WARRANTIES OF MERCHANTABILITY AND FITNESS FOR
A PARTICULAR PURPOSE ARE DISCLAIMED. IN NO EVENT SHALL THE COPYRIGHT
OWNER OR CONTRIBUTORS BE LIABLE FOR ANY DIRECT, INDIRECT, INCIDENTAL,
SPECIAL, EXEMPLARY, OR CONSEQUENTIAL DAMAGES (INCLUDING, BUT NOT
LIMITED TO, PROCUREMENT OF SUBSTITUTE GOODS OR SERVICES; LOSS OF USE,
DATA, OR PROFITS; OR BUSINESS INTERRUPTION) HOWEVER CAUSED AND ON ANY
THEORY OF LIABILITY, WHETHER IN CONTRACT, STRICT LIABILITY, OR TORT
(INCLUDING NEGLIGENCE OR OTHERWISE) ARISING IN ANY WAY OUT OF THE USE
OF THIS SOFTWARE, EVEN IF ADVISED OF THE POSSIBILITY OF SUCH DAMAGE.

\end{itemize}

%%%%%%%%%%%%%%%%%%%%%%%%%%%%%%%%%%%%%%%%%%
\index{HTCondor!acknowledgments}
The HTCondor project wishes to acknowledge the following:

\begin{itemize}

\item This material is based upon work supported by the 
National Science Foundation under Grant Numbers
MCS-8105904, OCI-0437810, and OCI-0850745. 
Any opinions, findings, and conclusions or recommendations expressed 
in this material are those of the author(s) and do not necessarily 
reflect the views of the National Science Foundation.

\end{itemize}
 
%%%%%%%%%%%%%%%%%%%%%%%%%%%%%%%%%%%%%%%%%%%%%%%%%%


%%%%%%%%%%%%%%%%%%%%%%%%%%%%%%%%%%%%%%%%%%%%%%%%%%
\section{\label{contact-info}Contact Information}
%%%%%%%%%%%%%%%%%%%%%%%%%%%%%%%%%%%%%%%%%%%%%%%%%%

\index{HTCondor!contact information}
The latest software releases, publications/papers regarding HTCondor and other 
High-Throughput Computing
research can be found at the official web site for HTCondor at  
\URL{http://www.cs.wisc.edu/condor}.

\index{HTCondor!mailing lists}
\index{mailing lists}
In addition, there is an e-mail list at condor-world@cs.wisc.edu.
The HTCondor Team uses this e-mail list to announce new releases of
HTCondor and other major HTCondor-related news items.
To subscribe or unsubscribe from the the list, follow the instructions at  
\URL{http://www.cs.wisc.edu/condor/mail-lists/}.
Because many of us receive 
too much e-mail as it is, you will be happy to know that the
HTCondor World e-mail list group is 
moderated, and only major announcements of wide interest are distributed.

Our users support each other by belonging to an unmoderated mailing
list targeted at solving problems with HTCondor.
HTCondor team members attempt to monitor traffic to HTCondor Users,
responding as they can. 
Follow the instructions at
\URL{http://www.cs.wisc.edu/condor/mail-lists/}.

Finally, you can reach the HTCondor Team directly.
The HTCondor Team is comprised of the 
developers and administrators of HTCondor at the University of Wisconsin-Madison.
HTCondor questions, comments, pleas for help,
and requests for commercial contract consultation or support 
are all welcome;
send Internet e-mail to
\Email{condor-admin@cs.wisc.edu}.
Please include your name, organization, and telephone number in your message.
If you are having trouble with HTCondor,
please help us troubleshoot by including as much pertinent information
as you can, including snippets of HTCondor log files. 

\section{\label{privacy}Privacy Notice}

%
% Note to developers:
% If you change this text, also change
%   http://www.cs.wisc.edu/condor/privacy.html
%

The HTCondor software periodically sends short messages
to the HTCondor Project developers at the University of Wisconsin,
reporting totals of machines and jobs in each running HTCondor system.
An example of such a message is given below.

The HTCondor Project uses these collected reports to publish
summary figures and tables, such as the total of HTCondor systems
worldwide, or the geographic distribution of HTCondor systems.
This information helps the HTCondor Project to understand
the scale and composition of HTCondor in the real world
and improve the software accordingly.

The HTCondor Project will not use these reports to publicly
identify any HTCondor system or user without permission.
The HTCondor software does not collect or report any personal
information about individual users.

We hope that you will contribute to the development of HTCondor
through this reporting feature.
However, you are free to disable it at any time by
changing the configuration variables \Macro{CONDOR\_DEVELOPERS}
and \Macro{CONDOR\_DEVELOPERS\_COLLECTOR},
both described in section \ref{param:CondorDevelopers} of this manual.

Example of data reported:

\begin{verbatim}
This is an automated email from the HTCondor system
on machine "your.condor.pool.com".  Do not reply.

This Collector has the following IDs:
    HTCondor: 6.6.0 Nov 12 2003
    HTCondor: INTEL-LINUX-GLIBC22

                     Machines Owner Claimed Unclaimed Matched Preempting

         INTEL/LINUX      810    52     716        37       0          5
       INTEL/WINDOWS      120     5     115         0       0          0
     SUN4u/SOLARIS28      114    12      92         9       0          1
     SUN4x/SOLARIS28        5     1       0         4       0          0
               Total     1049    70     923        50       0          6

         RunningJobs                IdleJobs
                 920                    3868
\end{verbatim}



