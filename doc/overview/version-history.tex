%%%%%%%%%%%%%%%%%%%%%%%%%%%%%%%%%%%%%%%%%%%%%%%%%%%%%%%%%%%%%%%%%%%%%%
\section{\label{sec:Version-History}Version History}
%%%%%%%%%%%%%%%%%%%%%%%%%%%%%%%%%%%%%%%%%%%%%%%%%%%%%%%%%%%%%%%%%%%%%%

This section provides descriptions of what features have been added or
bugs fixed for each version of Condor.  
Each release series is covered in its own section.

%%%%%%%%%%%%%%%%%%%%%%%%%%%%%%%%%%%%%%%%%%%%%%%%%%%%%%%%%%%%%%%%%%%%%%
\subsection{\label{sec:History-6-1}Development Release Series 6.1}
%%%%%%%%%%%%%%%%%%%%%%%%%%%%%%%%%%%%%%%%%%%%%%%%%%%%%%%%%%%%%%%%%%%%%%

This is the first development release.
You should only install it if you really know what you are doing.
Development releases will come out quickly, with lots of new features
being added, many bugs fixed, etc.  
6.1.X should not be used for a production pool.

\Note Different releases within a development series cannot be
installed side-by-side within the same pool. 
For example, the protocols used by version 6.1.6 are not compatible with the
protocols used in version 6.1.5.  
When you upgrade to a new development release, make certain you upgrade all
machines in your pool to the same version.

%%%%%%%%%%%%%%%%%%%%%%%%%%%%%%%%%%%%%%%%%%%%%%%%%%%%%%%%%%%%%%%%%%%%%%
\subsubsection{\label{sec:New-6-1-7}Version 6.1.7}
%%%%%%%%%%%%%%%%%%%%%%%%%%%%%%%%%%%%%%%%%%%%%%%%%%%%%%%%%%%%%%%%%%%%%%

\Note Version 6.1.7 only adds support for platforms not supported in
6.1.6.  
There are no bug fixes, so there are no binaries released for any
other platforms. 
You do not need 6.1.7 unless you are using one of the two platforms we
released binaries for.

\begin{itemize}

\item Added ``clipped'' support for Alpha Linux machines running the
2.0.X kernel and glibc 2.0.X.
We do not yet support checkpointing and remote system calls on this
platform, but we can start ``vanilla'' jobs.
See section~\ref{sec:Choosing-Universe} on
page~\pageref{sec:Choosing-Universe} for details on vanilla
vs. standard jobs.

\item Re-added support for Intel Linux machines running the 2.0.X
Linux kernel, glibc 2.0.X, using the GNU C compiler (gcc/g++ 2.7.X) or
the EGCS compilers (versions 1.0.X, 1.1.1 and 1.1.2).
This includes RedHat 5.X, and Debian 2.0.
RedHat 6.0 and Debian 2.1 are not yet supported, since they use glibc
2.1.X and the 2.2.X Linux kernel.
Future versions of Condor will support all combinations of kernels,
compilers and versions of libc.

\end{itemize}


%%%%%%%%%%%%%%%%%%%%%%%%%%%%%%%%%%%%%%%%%%%%%%%%%%%%%%%%%%%%%%%%%%%%%%
\subsubsection{\label{sec:New-6-1-6}Version 6.1.6}
%%%%%%%%%%%%%%%%%%%%%%%%%%%%%%%%%%%%%%%%%%%%%%%%%%%%%%%%%%%%%%%%%%%%%%

\begin{itemize}
\item Changed the way that \Condor{master} spawns daemons and
\Condor{preen} which allows you to specify command line arguments for
any of them, though a \Macro{SUBSYS\_ARGS} setting.
Previously, when you specified \Macro{PREEN}, you added the command
line arguments directly to that setting, but that caused some
problems, and only worked for \Condor{preen}.
\Bold{Once you upgrade to version 6.1.6, if you continue to use your
old \File{condor\_config} files, you must change the \Macro{PREEN}
setting to remove any arguments you have defined and place those
arguments into a separate config setting, \Macro{PREEN\_ARGS}.}
See section~\ref{sec:Master-Config-File-Entries}, ``\condor{master}
Config File Entries'', on
page~\pageref{sec:Master-Config-File-Entries} for more details.

\item Fixed a very serious bug in the Condor library linked in with
\Condor{compile} to create standard jobs that was causing
checkpointing to fail in many cases.  
Any jobs that were linked with the 6.1.5 Condor libraries should
probably be removed, re-linked, and re-submitted. 

\item Fixed a bug in \Condor{userprio} that was introduced in version
6.1.5 that was preventing it from finding the address of the
\Condor{negotiator} for your pool.

\item Fixed a bug in \Condor{stats} that was introduced in version
6.1.5 that was preventing it from finding the address of the
\Condor{collector} for your pool.

\item Fixed a bug in the way the \Opt{-pool} option was handled by
many Condor tools that was introduced in version 6.1.5. 


\item \Condor{q} now displays job \emph{allocation time} by default, instead
of displaying CPU time.  
Job allocation time, or RUN\_TIME, is the amount of wall-clock time the job
has spent running.  
Unlike CPU time information which is only updated when a job is
checkpointed, the allocation time displayed by \Condor{q} is continuously
updated, even for vanilla universe jobs.  
By default, the allocation time displayed will be the total time across all
runs of the job.  
The new \Opt{-currentrun} option to \Condor{q} can be used to display the
allocation time for solely the current run of the job.
Additionally, the \Opt{-cputime} option can be used to view job CPU times as
in earlier versions of Condor.

\item \Condor{q} will display an error message if there is a timeout
fetching the job queue listing from a \condor{schedd}.  Previously,
\Condor{q} would simply list the queue as empty upon a communication error.

\item The \condor{schedd} daemon has been updated to verify all queue access
requests via Condor's IP/Host-Based Security mechanism (see
section~\ref{sec:Host-Security}).

\item Fixed a bug on platforms which require the \Condor{kbdd} (currently
Digital Unix and IRIX).  
This bug could have allowed Condor to start a job within the first five
minutes after the Condor daemons had been started, even if there is a user
typing on the keyboard.

\item \Condor{release} now gives an error message if the user tries to
release a job which either does not exist or is not in the hold state.

\item Added a new config file parameter, \Macro{USER\_JOB\_WRAPPER}, which
allows administrators to specify a file to act as a ``wrapper'' script
around all jobs started by Condor. 
See inside section~\ref{param:UserJobWrapper}, on 
page~\pageref{sec:Starter-Config-File-Entries}, for more details.

\item \Condor{dagman} now permits the backslash character (``\Bs'') to be used
as a line-continuation character for DAG Input Files, just like the
\condor{config} files.

\item The Condor version string is now included in all Condor
libraries.
You can now run \Prog{ident} on any program linked with
\Condor{compile} to view which version of the Condor libraries you
linked with.
In addition, the format of the version string changed in 6.1.6.
Now, the identifier used is ``CondorVersion'' instead of ``Version''
to prevent any potential ambiguity.
Also, the format of the date changed slightly.

\item The SMP startd can now handle dynamic reconfiguration of the
number of each type of virtual machine being reported.
This allows you, during the normal running of the startd, to increase
or decrease the number of CPUs that Condor is using.
If you reconfigure the startd to use less CPUs than it currently has
under its control, it will first remove CPUs that have no Condor jobs
running on them.
If more CPUs need to be evicted, the startd will checkpoint jobs and
evict them in reverse rank order (using the startd's \Macro{Rank}
expression).
So, the lower the value of the rank, the more likely a job will be
kicked off.

\item The SMP startd contrib module's \Condor{starter} no longer makes
a call that was causing warning messages about ``ERROR: Unknown System
Call (-58) - system call not supported by Condor'' when used with the
6.0.X \Condor{shadow}.
This was a harmless call, but removing the call prevents the error
message.

\item The SMP contrib module now includes the \Condor{checkpoint} and
\Condor{vacate} programs, which allow you to vacate or checkpoint jobs
on individual CPUs on the SMP, instead of checkpointing or vacating
everything.  
You can now use ``\condor{vacate} vm1@hostname'' to just vacate the
first virtual machine, or ``\condor{vacate} hostname'' to vacate all
virtual machines. 

\item Added support for SMP Digital Unix (Alpha) machines.

\item Fixed a bug that was causing an overflow in the computation of
free disk and swap space on Digital Unix (Alpha) machines.

\item The \Condor{startd} and \Condor{schedd} now can ``invalidate''
their classads from the collector.
So, when a daemon is shut down, or a machine is reconfigured to 
advertise fewer virtual machines, those changes will be instantly
visible with \Condor{status}, instead of having to wait 15 minutes for
the stale classads to time-out.

\item The \Condor{schedd} no longer forks a child process (a ``schedd
agent'') to claim available \Condor{startd}s.  
You should no longer see multiple \condor{schedd} processes running on
your machine after a negotiation cycle.
This is now accomplished in a non-blocking manner within the
\Condor{schedd} itself.

\item The startd now adds an \Attr{VirtualMachineID} attribute to
each virtual machine classad it advertises.
This is just an integer, starting at 1, and increasing for every
different virtual machine the startd is representing.
On regular hosts, this is the only ID you will ever see.
On SMP hosts, you will see the ID climb up to the number of different
virtual machines reported.
This ID can be used to help write more complex policy expressions on
SMP hosts, and to easily identify which hosts in your pool are in fact
SMP machines.

\item Modified the output for \Condor{q} -run for scheduler and PVM
universe jobs.  The host where the scheduler universe job is running
is now displayed correctly.  For PVM jobs, a count of the current
number of hosts where the job is running is displayed.

\item Fixed the \Condor{startd} so that it no longer prints lots of
ProcAPI errors to the log file when it is being run as non-root.

\item \Macro{FS\_PATHNAME} and \Macro{VOS\_PATHNAME} are no longer
used.  AFS support now works similar to NFS support, via the
\Macro{FILESYSTEM\_DOMAIN} macro.

\item Fixed a minor bug in the \File{Condor.pm} perl module that was
causing it to be case-sensitive when parsing the Condor submit file.
Now, the perl module is properly case-insensitive, as indicated in the
documentation.

\end{itemize}

%%%%%%%%%%%%%%%%%%%%%%%%%%%%%%%%%%%%%%%%%%%%%%%%%%%%%%%%%%%%%%%%%%%%%%
\subsubsection{\label{sec:New-6-1-5}Version 6.1.5}
%%%%%%%%%%%%%%%%%%%%%%%%%%%%%%%%%%%%%%%%%%%%%%%%%%%%%%%%%%%%%%%%%%%%%%

\begin{itemize}

\item Fixed a nasty bug in \Condor{preen} that would cause it to
remove files it shouldn't remove if the \Condor{schedd} and/or
\Condor{startd} were down at the time \Condor{preen} ran.
This was causing jobs to mysteriously disappear from the job queue.

\item Added preliminary support to Condor for running on machines with
multiple network interfaces.
On such machines, users can specify the IP address Condor should use
in the \Macro{NETWORK\_INTERFACE} config file parameter on each host. 
In addition, if the pool's central manager is on such a machine, users
should set the \Macro{CM\_IP\_ADDR} parameter to the ip address you wish
to use on that machine.
See section~\ref{sec:Multiple-Interfaces} on
page~\pageref{sec:Multiple-Interfaces} for more details.

\item The support for multiple network interfaces introduced bugs in
\Condor{userprio}, \Condor{stats}, CondorPVM, and the \Opt{-pool}
option to many Condor tools.
All of these will be fixed in version 6.1.6.

\item Fixed a bug in the remote system call library that was
preventing certain Fortran operations from working correctly on
Linux.  

\item The Linux binaries for GLIBC we now distribute are compiled on a
RedHat 5.2 machine.
If you're using this version of RedHat, you might have better luck
with the dynamically linked version of Condor than previous releases
of Condor.
Sites using other GLIBC Linux distributions should continue to use the
statically linked version of Condor.

\item Fixed a bug in the \Condor{shadow} that could cause it to die
with signal 11 (segmentation violation) under certain rare
circumstances. 

\item Fixed a bug in the \Condor{schedd} that could cause it to die
with signal 11 (segmentation violation) under certain rare
circumstances. 

\item Fixed a bug in the \Condor{negotiator} that could cause it to
die with signal 8 (floating point exception) on Digital Unix
machines. 

\item The following shadow parameters have been added to control
checkpointing: \Macro{COMPRESS\_PERIODIC\_CKPT},
\Macro{COMPRESS\_VACATE\_CKPT}, \Macro{PERIODIC\_MEMORY\_SYNC},
\Macro{SLOW\_CKPT\_SPEED}.  See
section~\ref{sec:Shadow-Config-File-Entries} on
page~\pageref{sec:Shadow-Config-File-Entries} for more details.
In addition, the shadow now honors the \Attr{CkptWanted} flag in a job
classad, and if it is set to ``False'', the job will never
checkpoint.

\item Fixed a bug in the \Condor{startd} that could cause it to
report negative values for the CondorLoadAvg on rare occasions. 

\item Fixed a bug in the \Condor{startd} that could cause it to die
with a fatal exception in situations where the act of getting claimed
by a remote schedd failed for some reason.  
This resulted in the \Condor{startd} exiting on rare occasions with a
message in its log file to the effect of \texttt{ERROR ``Match timed
out but not in matched state''}.

\item Fixed a bug in the \Condor{schedd} that under rare circumstances
could cause a job to be left in the ``Running'' state even after the
\Condor{shadow} for that job had exited.

\item Fixed a bug in the \Condor{schedd} and various tools that
prevented remote read-only access to the job queue from working.
So, for example, \texttt{condor\_q -name foo}, if run on any machine
other than foo, wouldn't display any jobs from foo's queue. 
This fix re-enables the following options to \Condor{q} to work:
\Opt{submitter}, \Opt{name}, \Opt{global}, etc.

\item Changed the \Condor{schedd} so that when starting jobs, it
always sorts on the cluster number, in addition to the date the jobs
were enqueued and the process number within clusters, so that if many
clusters were submitted at the same time, the jobs are started in
order.

\item Fixed a bug in \Condor{compile} that was modifying the
\Env{PATH} environment variable by adding things to the front of it.
This would potentially cause jobs to be compiled and linked with a
different version of a compiler than they thought they were getting.  

\item Minor change in the way the \Condor{startd} handles the
\Dflag{LOAD} and \Dflag{KEYBOARD} debug flags.  
Now, each one, when set, will only display every
\Macro{UPDATE\_INTERVAL}, regardless of the startd state.
If you wish to see the values for keyboard activity or load average
every \Macro{POLLING\_INTERVAL}, you must enable \Dflag{FULLDEBUG}. 

\end{itemize}

%%%%%%%%%%%%%%%%%%%%%%%%%%%%%%%%%%%%%%%%%%%%%%%%%%%%%%%%%%%%%%%%%%%%%%
\subsubsection{\label{sec:New-6-1-4}Version 6.1.4}
%%%%%%%%%%%%%%%%%%%%%%%%%%%%%%%%%%%%%%%%%%%%%%%%%%%%%%%%%%%%%%%%%%%%%%

\begin{itemize}

\item Fixed a bug in the socket communication library used by Condor
that was causing daemons and tools to die on some platforms (notably,
Digital Unix) with signal 8, SIGFPE (floating point exception).

\item Fixed a bug in the usage message of many Condor tools that
mentioned a \Opt{-all} option that isn't yet supported. 
This option will be supported in future versions of Condor.

\item Fixed a bug in the filesystem authentication code used to
authenticate operations on the job queue that left empty temporary
files in /tmp.  
These files are now properly removed after they are used.

\item Fixed a minor bug in the totals \Condor{status} displays when
you use the \Opt{ckptsrvr} option.

\item Fixed a minor syntax error in the \Condor{install} script that
would cause warnings.

\item the \File{Condor.pm} Perl module is now included in the
\File{lib} directory of the main release directory.

\end{itemize}

%%%%%%%%%%%%%%%%%%%%%%%%%%%%%%%%%%%%%%%%%%%%%%%%%%%%%%%%%%%%%%%%%%%%%%
\subsubsection{\label{sec:New-6-1-3}Version 6.1.3}
%%%%%%%%%%%%%%%%%%%%%%%%%%%%%%%%%%%%%%%%%%%%%%%%%%%%%%%%%%%%%%%%%%%%%%

\Note There are a lot of new, unstable features in 6.1.3.  
PLEASE do not install all of 6.1.3 on a production pool.
Almost all of the bug fixes in 6.1.3 are in the \Condor{startd} or
\Condor{starter}, so, unless you really know what you're doing, we
recommend you just upgrade SMP-Startd contrib module, not the entire
6.1.3 release. 

\begin{itemize}

\item Owners can now specify how the SMP-Startd partitions the system
resources into the different types and numbers of virtual machines,
specifying the number of CPUs, megs of RAM, megs of swap space, etc.,
in each.
Previously, each virtual machine reported to Condor from an SMP
machine always had one CPU, and all shared system resources were
evenly divided among the virtual machines.

\item Fixed a bug in the reporting of virtual memory and disk space on
SMP machines where each virtual machine represented was advertising
the total in the system for itself, instead of its own share.
Now, both the totals, and the virtual machine-specific values are
advertised.  

\item Fixed a bug in the \Condor{starter} when it was trying to
suspend jobs.
While we always killed all of the processes when we were trying to
vacate, if a vanilla job forked, the starter would sometimes not
suspend some of the children processes.
In addition, we could sometimes miss a standard universe job for
suspending as well.
This is all fixed.

\item Fixed a bug in the SMP-Startd's load average computation that
could cause processes spawned by Condor to not be associated w/ the
Condor load average.
This would cause the startd to over-estimate the owner's load average,
and under-estimate the Condor load, which would cause a cycle of
suspending and resuming a Condor job, instead of just letting it run.

\item Fixed a bug in the SMP-Startd's load average computation that
could cause certain rare exceptions to be treated as fatal, when in
fact, the Startd could recover from them.

\item Fixed a bug in the computation of the total physical memory in
IRIX machines that was resulting in an overflow on machines with
lots of ram (over 1 gigabyte).

\item Fixed some bugs that could cause \Condor{starter} processes to
be left as zombies underneath the \Condor{startd} under very rare
conditions.  

\item For sites using AFS, if there are problems in the
\Condor{startd} computing the AFS cell of the machine it's running on,
the startd will exit with an error message at start-up time.

\item Fixed a minor bug in \Condor{install} that would lead to a
syntax error in your config file given a certain set of installation
options.  

\item Added the \Opt{-maxjobs} option to the \Condor{submit\_dag}
script that can be used to specify the maximum number of jobs Condor
will run from a DAG at any given time.
Also, \Condor{submit\_dag} automatically creates a ``rescue DAG''.
See section~\ref{sec:DAGMan} on page~\pageref{sec:DAGMan} for details
on DAGMan.

\item Fixed bug in ClassAd printing when you tried to display an
integer or float attribute that didn't exist in the given ClassAd. 
This could show up in \Condor{status}, \Condor{q}, \Condor{history},
etc. 

\item Various commands sent to the Condor daemons now have separate
debug levels associated with them.
For example, commands such as ``keep-alives'', and the command sent
from the \Condor{kbdd} to the \Condor{startd} are only seen in the
various log files if \Dflag{FULLDEBUG} is turned on, instead of
\Dflag{COMMAND}, which the default and now enabled for all daemons on
all platforms by default.
Administrators retaining their old configuration when upgrading to
this version are encouraged to enable \Dflag{COMMAND} in the
\Macro{SCHEDD\_DEBUG} setting.  
In addition, for IRIX and Digital Unix machines, it should be enabled
in the \Macro{STARTD\_DEBUG} setting as well.
See section~\ref{sec:Daemon-Logging-Config-File-Entries} on
page~\pageref{sec:Daemon-Logging-Config-File-Entries} for details on
debug levels in Condor.

\item New debug levels added to Condor: 
\begin{itemize}
\item \Dflag{NETWORK}, used by various daemons in Condor to report
various network statistics about the Condor daemons. 
\item \Dflag{PROCFAMILY}, used to report information about various
families of processes that are monitored by Condor.
For example, this is used in the \Condor{startd} when monitoring the
family of processes spawned by a given user job for the purposes of
computing the Condor load average.
\item \Dflag{KEYBOARD}, used by the \Condor{startd} to print out
statistics about remote tty and console idle times in the
\Condor{startd}.
This information used to be logged at \Dflag{FULLDEBUG}, along with
everything else, so now, you can see just the idle times, and/or have
the information stored to a separate file.
\end{itemize}

\item Added a \Opt{-run} option to \Condor{q}, which displays
information for running jobs, including the remote host where each job
is running.

\item Macros can now be incrementally defined.  See
section~\ref{sec:Config-File-Macros} on
page~\pageref{sec:Config-File-Macros} for more details.

\item \Condor{config\_val} can now be used to set configuration
variables.  See the man page on page~\pageref{man-condor-config-val}
for more details.

\item The job log file now contains a record of network activity.  The
evict, terminate, and shadow exception events indicate the number of
bytes sent and received by the job for the specific run.  
The terminate event additionally indicates totals for the life of the
job.

\item \Macro{STARTER\_CHOOSES\_CKPT\_SERVER} now defaults to true.
See section~\ref{param:StarterChoosesCkptServer} on
page~\pageref{param:StarterChoosesCkptServer} for more details.

\item The infrastructure for authentication within Condor has been
overhauled, allowing for much greater flexibility in supporting new
forms of authentication in the future.
This means that the 6.1.3 schedd and queue management tools (like
\Condor{q}, \Condor{submit}, \Condor{rm} and so on) are incompatible
with previous versions of Condor.

\item Many of the Condor administration tools have been improved to
allow you to specify the ``subsystem'' you want them to effect.  
For example, you can now use ``\condor{reconfig} -startd'' to just
have the startd reconfigure itself.
Similarly, \condor{off}, \condor{on} and \condor{restart} can now all 
work on a single daemon, instead of machine-wide.
See the man pages (section~\ref{command-reference} on
page~\pageref{command-reference}) or run any command with \Opt{-help}
for details. 
\Note The usage message in 6.1.3 incorrectly reports \Opt{-all} as a
valid option.

\item Fixed a bug in the Condor tools that could cause a segmentation
violation in certain rare error conditions.

\end{itemize}

%%%%%%%%%%%%%%%%%%%%%%%%%%%%%%%%%%%%%%%%%%%%%%%%%%%%%%%%%%%%%%%%%%%%%%
\subsubsection{\label{sec:New-6-1-2}Version 6.1.2}
%%%%%%%%%%%%%%%%%%%%%%%%%%%%%%%%%%%%%%%%%%%%%%%%%%%%%%%%%%%%%%%%%%%%%%

\begin{itemize}

\item Fixed some bugs in the \Condor{install} script.
Also, enhanced \Condor{install} to customize the path to perl in
various perl scripts used by Condor.

\item Fixed a problem with our build environment that left some files
out of the \File{release.tar} files in the binary releases on some
platforms. 

\item \Condor{dagman}, ``DAGMan'' (see section~\ref{sec:DAGMan} on 
page~\pageref{sec:DAGMan} for details) is now included in the
development release by default.

\item Fixed a bug in the computation of the total physical memory in
HPUX machines that was resulting in an overflow on machines with
lots of ram (over 1 gigabyte).
Also, if you define ``MEMORY'' in your config file, that value will
override whatever value Condor computes for your machine.

\item Fixed a bug in \Condor{starter.pvm}, the PVM version of the
Condor starter (available as an optional ``Contrib module''), when you
disabled \Macro{STARTER\_LOCAL\_LOGGING}.
Now, having this set to ``False'' will properly place debug messages
from \Condor{starter.pvm} into the \File{ShadowLog} file of the
machine that submitted the job (as opposed to the \File{StarterLog}
file on the machine executing the job).  

\end{itemize}


%%%%%%%%%%%%%%%%%%%%%%%%%%%%%%%%%%%%%%%%%%%%%%%%%%%%%%%%%%%%%%%%%%%%%%
\subsubsection{\label{sec:New-6-1-1}Version 6.1.1}
%%%%%%%%%%%%%%%%%%%%%%%%%%%%%%%%%%%%%%%%%%%%%%%%%%%%%%%%%%%%%%%%%%%%%%

\begin{itemize}

\item Fixed a bug in the \Condor{startd} where we compute the load
average caused by Condor that was causing us to get the wrong values.
This could cause a cycle of continuous job suspends and job resumes.

\item Beginning with this version, any jobs linked with the Condor
checkpoint libraries will use the zlib compression code (used by gzip
and others) to compress periodic checkpoints before they are written
to the network.  
These compressed checkpoints are uncompressed at startup time.  
This saves network bandwidth, disk space, as well as time (if the
network is the bottleneck to checkpointing, which it usually is). 
In future versions of Condor, all checkpoints will probably be
compressed, but at this time, it is only used for periodic
checkpoints.  
Note, you have to relink your jobs with the \Condor{compile} command
to have this feature enabled.
Old jobs (not relinked) will continue to run just fine, they just
won't be compressed.

\item \Condor{status} now has better support for displaying checkpoint
server ClassAds. 

\item More contrib modules from the development series are now
available, such as the checkpoint server, PVM support, and the
CondorView server.  

\item Fixed some minor bugs in the UserLog code that were causing
problems for DAGMan in exceptional error cases.

\item Fixed an obscure bug in the logging code when \Dflag{PRIV} was
enabled that could result in incorrect file permissions on log files. 

\end{itemize}

%%%%%%%%%%%%%%%%%%%%%%%%%%%%%%%%%%%%%%%%%%%%%%%%%%%%%%%%%%%%%%%%%%%%%%
\subsubsection{\label{sec:New-6-1-0}Version 6.1.0}
%%%%%%%%%%%%%%%%%%%%%%%%%%%%%%%%%%%%%%%%%%%%%%%%%%%%%%%%%%%%%%%%%%%%%%

\begin{itemize}

\item Support has been added to the \condor{startd} to run multiple
jobs on SMP machines.
See section~\ref{sec:Configuring-SMP} on
page~\pageref{sec:Configuring-SMP} for details about setting up and
configuring SMP support.

\item The expressions that control the \condor{startd} policy for
vacating, jobs has been simplified.
See section~\ref{sec:Configuring-Policy} on
page~\pageref{sec:Configuring-Policy} for complete details on the new
policy expressions, and section~\ref{sec:V60-Policy-diffs} on
page~\pageref{sec:V60-Policy-diffs} for an explanation of what's
different from the version 6.0 expressions.

\item We now perform better tracking of processes spawned by Condor.
If children die and are inherited by init, we still know they belong
to Condor.
This allows us to better ensure we don't leave processes lying around
when we need to get off a machine, and enables us to have a much more
accurate computation of the load average generated by Condor (the
\Attr{CondorLoadAvg} as reported by the \Condor{startd}). 

\item The \condor{collector} now can store historical information
about your pool state.
This information can be queried with the \Condor{stats} program (see
the man page on page~\pageref{man-condor-stats}), which is used by the
\Condor{view} Java GUI, which is available as a separate contrib
module.

\item Condor jobs can now be put in a ``hold'' state with the
\Condor{hold} command.
Such jobs remain in the job queue (and can be viewed with \Condor{q}),
but there will not be any negotiation to find machines for them.
If a job is having a temporary problem (like the permissions are 
wrong on files it needs to access), the job can be put on hold until
the problem can be solved.
Jobs put on hold can be released with the \Condor{release} command.

\item \condor{userprio} now has the notion of \Term{user factors} as a
way to create different groups of users in different priority levels.
See section~\ref{sec:UserPrio} on page~\pageref{sec:UserPrio} for
details.
This includes the ability to specify a local priority domain, and all
users from other domains get a much worse priority.

\item Usage statistics by user is now available from
\condor{userprio}.
See the man page on page~\pageref{man-condor-userprio} for details. 

\item The \condor{schedd} has been enhanced to enable ``flocking'',
where it seeks matches with machines in multiple pools if its requests
cannot be serviced in the local pool.
See section~\ref{sec:Flocking} on page~\pageref{sec:Flocking} for more
details.

\item The \condor{schedd} has been enhanced to enable \condor{q} and
other interactive tools better response time.

\item The \condor{schedd} has also been enhanced to allow it to check
the permissions of the files you specify for input, output, error and
so on.  
If the schedd doesn't have the required access rights to the files,
the jobs will not be submitted, and \Condor{submit} will print an
error message.

\item When you perform a \Condor{rm} command, and the job you removed
was using a ``user log'', the remove event is now recorded into the
log. 

\item Two new attributes have been added to the job classad when it 
begins executing: \Attr{RemoteHost} and \Attr{LastRemoteHost}.
These attributes list the IP address and port of the startd that is
either currently running the job, or the last startd to run the job
(if it's run on more than one machine). 
This information helps users track their job's execution more closely,
and allows administrators to troubleshoot problems more effectively. 

\item The performance of checkpointing was increased by using larger
buffers for the network I/O used to get the checkpoint file on and off
the remote executing host (this helps for all pools, with or without
checkpoint servers). 

\end{itemize}

%%%%%%%%%%%%%%%%%%%%%%%%%%%%%%%%%%%%%%%%%%%%%%%%%%%%%%%%%%%%%%%%%%%%%%
\subsection{\label{sec:History-6-0}Stable Release Series 6.0}
%%%%%%%%%%%%%%%%%%%%%%%%%%%%%%%%%%%%%%%%%%%%%%%%%%%%%%%%%%%%%%%%%%%%%%

6.0 is the first version of Condor with \Term{ClassAds}.
It contains many other fundamental enhancements over version 5.
It is also the first official stable release series, with a
development series (6.1) simultaneously available.

%%%%%%%%%%%%%%%%%%%%%%%%%%%%%%%%%%%%%%%%%%%%%%%%%%%%%%%%%%%%%%%%%%%%%%
\subsubsection{\label{sec:New-6-0-3}Version 6.0.3}
%%%%%%%%%%%%%%%%%%%%%%%%%%%%%%%%%%%%%%%%%%%%%%%%%%%%%%%%%%%%%%%%%%%%%%

\begin{itemize}

\item Fixed a bug that was causing the hostname of the submit machine
that claimed a given execute machine to be incorrectly reported by the
\Condor{startd} at sites using NIS.

\item Fixed a bug in the \Condor{startd}'s benchmarking code that
could cause a floating point exception (SIGFPE, signal 8) on very,
very fast machines, such as newer Alphas.

\item Fixed an obscure bug in \Condor{submit} that could happen when
you set a requirements expression that references the ``Memory''
attribute.
The bug only showed up with certain formations of the requirement
expression.

\end{itemize}


%%%%%%%%%%%%%%%%%%%%%%%%%%%%%%%%%%%%%%%%%%%%%%%%%%%%%%%%%%%%%%%%%%%%%%
\subsubsection{\label{sec:New-6-0-2}Version 6.0.2}
%%%%%%%%%%%%%%%%%%%%%%%%%%%%%%%%%%%%%%%%%%%%%%%%%%%%%%%%%%%%%%%%%%%%%%

\begin{itemize}

\item Fixed a bug in the \Syscall{fcntl} call for Solaris 2.6 that was
causing problems with file I/O inside Fortran jobs.

\item Fixed a bug in the way the \Macro{DEFAULT\_DOMAIN\_NAME}
parameter was handled so that this feature now works properly.  

\item Fixed a bug in how the \Macro{SOFT\_UID\_DOMAIN} config file
parameter was used in the \Condor{starter}.
This feature is also documented in the manual now (see
section~\ref{param:SoftUidDomain} on
page~\pageref{param:SoftUidDomain}).

\item You can now set the RunBenchmarks expression to ``False'' and
the \Condor{startd} will never run benchmarks, not even at startup
time. 

\item Fixed a bug in \Syscall{getwd} and \Syscall{getcwd} for sites
that use the NFS automounter.
his bug was only present if user programs tried to call
\Syscall{chdir} themselves.
Now, this is supported. 

\item Fixed a bug in the way we were computing the available virtual
memory on HPUX 10.20 machines.

\item Fixed a bug in \Condor{q} -analyze so it will correctly identify
more situations where a job won't run.

\item Fixed a bug in \Condor{status} -format so that if the requested 
attribute isn't available for a given machine, the format string
(including spaces, tabs, newlines, etc) is still printed, just the
value for the requested attribute will be an empty string. 

\item Fixed a bug in the \Condor{schedd} that was causing
\Condor{history} to not print out the first ClassAd attribute of all
jobs that have completed

\item Fixed a bug in \Condor{q} that would cause a segmentation fault
if the argument list was too long.

\end{itemize}

%%%%%%%%%%%%%%%%%%%%%%%%%%%%%%%%%%%%%%%%%%%%%%%%%%%%%%%%%%%%%%%%%%%%%%
\subsubsection{\label{sec:New-6-0-1}Version 6.0.1}
%%%%%%%%%%%%%%%%%%%%%%%%%%%%%%%%%%%%%%%%%%%%%%%%%%%%%%%%%%%%%%%%%%%%%%

\begin{itemize}

\item Fixed bugs in the \Syscall{getuid}), \Syscall{getgid},
\Syscall{geteuid}, and \Syscall{getegid} system calls. 

\item Multiple config files are now supported as a list specified via
the \Macro{LOCAL\_CONFIG\_FILE} variable. 

\item \Macro{ARCH} and \Macro{OPSYS} are now automatically determined
on all machines (including HPUX 10 and Solaris). 

\item Machines running IRIX now correctly suspend vanilla jobs.

\item \Condor{submit} doesn't allow root to submit jobs.

\item The \Condor{startd} now notices if you have changed
\Macro{COLLECTOR\_HOST} on reconfig.

\item Physical memory is now correctly reported on Digital Unix when
daemons are not running as root. 

\item New \MacroU{SUBSYSTEM} macro in configuration files that changes
based on which daemon is reading the file (i.e. STARTD, SCHEDD, etc.) 
See section~\ref{sec:Condor-Subsystem-Names}, ``Condor Subsystem
Names'' on page~\pageref{sec:Condor-Subsystem-Names} for a complete
list of the subsystem names used in Condor.

\item Port to HP-UX 10.20.  

\item \Syscall{getrusage} is now a supported system call.  
This system call will allow you to get resource usage about the entire
history of your condor job.

\item Condor is now fully supported on Solaris 2.6 machines (both
Sparc and Intel). 

\item Condor now works on Linux machines with the GNU C library.  
This includes machines running RedHat 5.x and Debian 2.0. 
In addition, there seems to be a bug in RedHat that was causing the
output from \Condor{config\_val} to not appear when used in scripts
(like \Condor{compile}).
We put in explicit calls to flush the I/O buffers before
\Condor{config\_val} exits, which seems to solve the problem.

\item Hooks have been added to the checkpointing library to help
support the checkpointing of PVM jobs.

\item Condor jobs can now send signals to themselves when running in
the standard universe.
You do this just as you normally would:
\begin{verbatim}
        kill( getpid(), signal_number )
\end{verbatim}
Trying to send a signal to any other process will result in
\Syscall{kill} returning -1.

\item Support for NIS has been improved on Digital Unix and IRIX.

\item Fixed a bug that would cause the negotiator on IRIX machines to
never match jobs with available machines.  

\end{itemize}

%%%%%%%%%%%%%%%%%%%%%%%%%%%%%%%%%%%%%%%%%%%%%%%%%%%%%%%%%%%%%%%%%%%%%%
\subsubsection{\label{sec:New-6-0-pl4}Version 6.0 pl4}
%%%%%%%%%%%%%%%%%%%%%%%%%%%%%%%%%%%%%%%%%%%%%%%%%%%%%%%%%%%%%%%%%%%%%%

\Note Back in the bad old days, we used this evil ``patch level''
version number scheme, with versions like ``6.0pl4''.
This has all gone away in the current versions of Condor. 

\begin{itemize}

\item Fixed a bug that could cause a segmentation violation in the 
\Condor{schedd} under rare conditions when a \Condor{shadow} exited.

\item Fixed a bug that was preventing any core files that user jobs
submitted to Condor might create from being transferred back to the
submit machine for inspection by the user who submitted them.

\item Fixed a bug that would cause some Condor daemons to go into an
infinite loop if the "ps" command output duplicate entries.
This only happens on certain platforms, and even then, only under rare
conditions.
However, the bug has been fixed and Condor now handles this case
properly.

\item Fixed a bug in the \Condor{shadow} that would cause a
segmentation violation if there was a problem writing to the user log
file specified by "log = filename" in the submit file used with
\Condor{submit}.

\item Added new command line arguments for the Condor daemons to support
saving the PID (process id) of the given daemon to a file, sending a
signal to the PID specified in a given file, and overriding what
directory is used for logging for a given daemon.
These are primarily for use with the \Condor{kbdd} when it needs to be
started by XDM for the user logged onto the console, instead of
running as root.
See section~\ref{sec:kbdd} on ``Installing the \Condor{kbdd}'' on
page~\pageref{sec:kbdd} for details.

\item Added support for the \Macro{CREATE\_CORE\_FILES} config file
parameter.  
If this setting is defined, Condor will override whatever limits you
have set and in the case of a fatal error, will either create core
files or not depending on the value you specify ("true" or "false").

\item Most Condor tools (\Condor{on}, \Condor{off},
\Condor{master\_off}, \Condor{restart}, \Condor{vacate},
\Condor{checkpoint}, \Condor{reconfig}, \Condor{reconfig\_schedd},
\Condor{reschedule}) can now take the IP address and port you want to
send the command to directly on the command line, instead of only
accepting hostnames. 
This IP/port must be passed in a special format used in Condor (which
you will see in the daemon's log files, etc).
It is of the form: \Sinful{ip.address:port}.  
For example: \Sinful{123.456.789.123:4567}.

\end{itemize}

%%%%%%%%%%%%%%%%%%%%%%%%%%%%%%%%%%%%%%%%%%%%%%%%%%%%%%%%%%%%%%%%%%%%%%
\subsubsection{\label{sec:New-6-0-pl3}Version 6.0 pl3}
%%%%%%%%%%%%%%%%%%%%%%%%%%%%%%%%%%%%%%%%%%%%%%%%%%%%%%%%%%%%%%%%%%%%%%

\begin{itemize}

\item Fixed a bug that would cause a segmentation violation if a
machine was not configured with a full hostname as either the official
hostname or as any of the hostname aliases.

\item If your host information does not include a fully qualified
hostname anywhere, you can specify a domain in the
\Macro{DEFAULT\_DOMAIN\_NAME} parameter in your global config file
which will be appended to your hostname whenever Condor needs to use a
fully qualified name.

\item All Condor daemons and most tools now support a "-version"
option that displays the version information and exits.

\item The \Condor{install} script now prompts for a short description
of your pool, which it stores in your central manager's local config
file as \Macro{COLLECTOR\_NAME}.
This description is used to display the name of your pool when sending
information to the Condor developers.

\item When the \Condor{shadow} process starts up, if it is configured
to use a checkpoint server and it cannot connect to the server, the
shadow will check the \Macro{MAX\_DISCARDED\_RUN\_TIME} parameter.  
If the job in question has accumulated more CPU minutes than this
parameter, the \Condor{shadow} will keep trying to connect to the
checkpoint server until it is successful.
Otherwise, the \Condor{shadow} will just start the job over from
scratch immediately.

\item If Condor is configured to use a checkpoint server, it will only
use the checkpoint server.
Previously, if there was a problem connecting to the checkpoint
server, Condor would fall back to using the submit machine to store
checkpoints.
However, this caused problems with local disks filling up on machines
without much disk space.

\item Fixed a rare race condition that could cause a segmentation
violation if a Condor daemon or tool opened a socket to a daemon and
then closed it right away.

\item All TCP sockets in Condor now have the "keep alive" socket option
enabled.
This allows Condor daemons to notice if their peer goes away in a hard
crash.

\item Fixed a bug that could cause the \Condor{schedd} to kill jobs
without a checkpoint during its graceful shutdown method under certain
conditions.

\item The \Condor{schedd} now supports the
\Macro{MAX\_SHADOW\_EXCEPTIONS} parameter.
If the \Condor{shadow} processes for a given match die due to a fatal
error (an exception) more than this number of times, the
\Condor{schedd} will now relinquish that match and stop trying to
spawn \Condor{shadow} processes for it.

\item The "-master" option to \Condor{status} now displays the \Attr{Name}
attribute of all \Condor{master} daemons in your pool, as opposed
to the \Attr{Machine} attribute.
This helps for pools that have submit-only machines joining them, for
example.

\end{itemize}

%%%%%%%%%%%%%%%%%%%%%%%%%%%%%%%%%%%%%%%%%%%%%%%%%%%%%%%%%%%%%%%%%%%%%%
\subsubsection{\label{sec:New-6-0-pl2}Version 6.0 pl2}
%%%%%%%%%%%%%%%%%%%%%%%%%%%%%%%%%%%%%%%%%%%%%%%%%%%%%%%%%%%%%%%%%%%%%%

\begin{itemize}

\item In patch level 1, code was added to more accurately find the
full hostname of the local machine.
Part of this code relied on the resolver, which on many platforms is a
dynamic library.
On Solaris, this library has needed many security patches and the
installation of Solaris on our development machines produced binaries
that are incompatible with sites that haven't applied all the security
patches.
So, the code in Condor that relies on this library was simply removed
for Solaris.

\item Version information is now built into Condor.
You can see the \Attr{CondorVersion} attribute in every daemon's
ClassAd. 
You can also run the UNIX command "ident" on any Condor binary to see
the version. 

\item Fixed a bug in the "remote submit" mode of \Condor{submit}.
The remote submit wasn't connecting to the specified schedd, but was
instead trying to connect to the local schedd.

\item Fixed a bug in the \Condor{schedd} that could cause it to exit
with an error due to its log file being locked improperly under
certain rare circumstances.

\end{itemize}

%%%%%%%%%%%%%%%%%%%%%%%%%%%%%%%%%%%%%%%%%%%%%%%%%%%%%%%%%%%%%%%%%%%%%%
\subsubsection{\label{sec:New-6-0-pl1}Version 6.0 pl1}
%%%%%%%%%%%%%%%%%%%%%%%%%%%%%%%%%%%%%%%%%%%%%%%%%%%%%%%%%%%%%%%%%%%%%%

\begin{itemize}

\item \Condor{kbdd} bug patched: On Silicon Graphics and DEC Alpha
ports, if your X11 server is using Xauthority user authentication, and
the \Condor{kbdd} was unable to read the user's \File{.Xauthority}
file for some reason, the \Condor{kbdd} would fall into an infinite 
loop.

\item When using a Condor Checkpoint Server, the protocol between the
Checkpoint Server and the \Condor{schedd} has been made more robust
for a faulty network connection. Specifically, this improves
reliability when submitting jobs across the Internet and using a
remote Checkpoint Server.

\item Fixed a bug concerning \Macro{MAX\_JOBS\_RUNNING}: The parameter
\Macro{MAX\_JOBS\_RUNNING} in the config file controls the maximum
number of simultaneous \Condor{shadow} processes allowed on your
submission machine.
The bug was the number of shadow processes could, under certain
conditions, exceed the number specified by
\Macro{MAX\_JOBS\_RUNNING}. 

\item Added new parameter \Macro{JOB\_RENICE\_INCREMENT} that can be
specified in the config file.
This parameter specifies the UNIX nice level that the \Condor{starter}
will start the user job.
It works just like the \Cmd{renice}{1} command in UNIX. 
Can be any integer between 1 and 19; a value of 19 is the lowest
possible priority.

\item Improved response time for \Condor{userprio}.

\item Fixed a bug that caused periodic checkpoints to happen more
often than specified.

\item Fixed some bugs in the installation procedure for certain
environments that weren't handled properly, and made the documentation
for the installation procedure more clear.

\item Fixed a bug on IRIX that could allow vanilla jobs to be started
as root under certain conditions.
This was caused by the non-standard uid that user "nobody" has on
IRIX.
Thanks to Chris Lindsey at NCSA for help discovering this bug.

\item On machines where the \File{/etc/hosts} file is misconfigured to
list just the hostname first, then the full hostname as an alias,
Condor now correctly finds the full hostname anyway.

\item The local config file and local root config file are now only
found by the files listed in the \Macro{LOCAL\_CONFIG\_FILE} and
\Macro{LOCAL\_ROOT\_CONFIG\_FILE} parameters in the global config
files.
Previously, \File{/etc/condor} and user condor's home directory
(\~condor) were searched as well.
This could cause problems with submit-only installations of Condor at
a site that already had Condor installed.

\end{itemize}

%%%%%%%%%%%%%%%%%%%%%%%%%%%%%%%%%%%%%%%%%%%%%%%%%%%%%%%%%%%%%%%%%%%%%%
\subsubsection{\label{sec:New-6-0-pl0}Version 6.0 pl0}
%%%%%%%%%%%%%%%%%%%%%%%%%%%%%%%%%%%%%%%%%%%%%%%%%%%%%%%%%%%%%%%%%%%%%%

\begin{itemize}

\item Initial Version 6.0 release.

\end{itemize}

