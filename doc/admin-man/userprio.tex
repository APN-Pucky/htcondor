%%%%%%%%%%%%%%%%%%%%%%%%%%%%%%%%%%%%%%%%%%%%%%%%%%%%%%%%%%%%%%%%%%%%%%
\section{\label{sec:UserPrio}
User Priorities}
%%%%%%%%%%%%%%%%%%%%%%%%%%%%%%%%%%%%%%%%%%%%%%%%%%%%%%%%%%%%%%%%%%%%%%

% Karen's understanding of this stuff, in preparation for a re-write
% of the section:

% Users request machines (by submitting jobs).
% Each user has a calculated priority.
%   A larger priority is worse.
%   This priority essentially tells how many machines the user is
%   currently using.  The priority can be made worse (larger number)
%   by the settings of various configuration variables.
% During each negotiation cycle, all machines are allocated (presuming
%   that there are more requests than machines).
% Each user is allocated machines in a ratio of 1/users's priority.
% Within the negotiation cycle, each user is given an initial
%   allocation of machines.  From there, remaining unallocated machines
%   are divided up among users that want more.  In a round robin
%   manner, each user is allocated a fraction of the remaining
%   unallocated machines.  this fraction is 1/user's priority.

\index{priority!in machine allocation}
\index{user priority}
Condor uses priorities to determine machine allocation for jobs.
This section details the priorities.

For accounting purposes, each user is identified by username@uid\_domain.
Each user is assigned a priority value even if submitting jobs from
different machines in the same domain, or even if submitting from multiple
machines in the different domains.

The numerical priority value assigned to a user is inversely related to the 
\emph{goodness} of the priority.
A user with a numerical priority of 5 gets 
more resources than a user with a numerical priority of 50.
There are two 
priority values assigned to Condor users:
\begin{itemize}
	\item Real User Priority (RUP), which measures resource usage of the 
		user.
	\item Effective User Priority (EUP), which determines the number of
		resources the user can get.
\end{itemize}
This section describes these two priorities and how they affect resource
allocations in Condor.
Documentation on configuring and controlling 
priorities may be found in section~\ref{sec:Negotiator-Config-File-Entries}.

\subsection{Real User Priority (RUP)}
\index{real user priority (RUP)}
\index{user priority!real (RUP)}
A user's RUP measures the resource usage of the user 
through time.
Every user begins with a RUP of one half (0.5), and
at steady state, the RUP of a user equilibrates to the number of resources 
used by that user.  Therefore, if a specific user continuously uses exactly 
ten resources for a long period of time, the RUP of that user stabilizes at 
ten.

However, if the user decreases the number of resources used, the RUP
gets better.  The rate at which the priority value decays 
can be set by the macro \Macro{PRIORITY\_HALFLIFE}, a time period 
defined in seconds.   Intuitively, if the \Macro{PRIORITY\_HALFLIFE} in a pool 
is set to 86400 (one day), and if a user whose RUP was 10 removes all his 
jobs, the user's RUP would be 5 one day later, 2.5 two days later,
and so on.

\subsection{Effective User Priority (EUP)}
\index{effective user priority (EUP)}
\index{user priority!effective (EUP)}
The effective user priority (EUP) of a user is used to determine
how many resources that user may receive.
The EUP is linearly related to the RUP
by a \emph{priority factor} which may be defined on a per-user basis.
Unless otherwise configured, the priority factor for all users is 1.0,
and so the EUP is the same as the the RUP.
However, if desired, the priority factors of
specific users (such as remote submitters) can be increased so that 
others are served preferentially.

The number of resources that a user may receive is inversely related
to the ratio between the EUPs of submitting users.
Therefore user $A$ with EUP=5 will receive
twice as many resources as user $B$ with EUP=10 and four times as many 
resources as user $C$ with EUP=20.
However, if $A$ does not use the full number
of allocated resources,
the available resources are repartitioned and distributed among
remaining users according to the inverse ratio rule.

% editted to here

Condor supplies mechanisms to directly support two policies in which EUP may
be useful:
\begin{description}
	\item[Nice users]  A job may be submitted with the parameter 
	\AdAttr{nice\_user} set to TRUE in the submit command file.
	A nice user job gets its RUP boosted by the 
	\Macro{NICE\_USER\_PRIO\_FACTOR} priority factor specified in the 
	configuration file, leading to a (usually very large) EUP.
	This corresponds to a low priority for resources.
	These jobs are therefore equivalent to Unix background jobs,
	which use resources not used by other Condor users.

	\item[Remote Users] The flocking feature of Condor (see
	section~\ref{sec:Flocking}) allows the \Condor{schedd} to
	submit to more than one pool.
	In addition, the submit-only feature allows a user to run a
	\Condor{schedd} that is submitting jobs into another pool.
	In such situations, submitters from other domains
	can submit to the local pool.
	It is often desirable to have Condor treat local users
	preferentially over these remote users.
	If configured, Condor will boost the RUPs of remote users by
	\Macro{REMOTE\_PRIO\_FACTOR}
	specified in the configuration file,
	thereby lowering their priority for resources.
\end{description}

The priority boost factors for individual users can be set with the 
\Opt{setfactor} option of \Condor{userprio}.
Details may be found in the \Condor{userprio} manual page 
on page~\pageref{man-condor-userprio}.

\subsection{Priorities and Preemption}
\index{preemption!priority}
Priorities are used to ensure that users get their fair share of resources.  
The priority values are used at allocation time.
In addition, Condor preempts machine claims and reallocates them when
conditions change.

To ensure that preemptions do not lead to \Term{thrashing},
a \Macro{PREEMPTION\_REQUIREMENTS} expression is defined to specify the
conditions that must be met for a preemption to occur.
It is usually defined to deny preemption if a current running job
has been running for a relatively short period of time.
This effectively limits the number of preemptions per resource per time
interval.

Note that \MacroNI{PREEMPTION\_REQUIREMENTS} only applies to preemptions
due to user priority.  It does not have any effect if the machine rank
expression prefers a different job, or if the startd policy expression
causes the job to vacate due to other activity on the machine.

\subsection{Priority Calculation}
This section may be skipped if the reader so feels, but for the curious,
here is Condor's priority calculation algorithm.

The RUP of a user $u$ at time $t$, $\pi_r(u,t)$, is calculated 
every time interval $\delta t$ using the formula 
$$\pi_r(u,t) = \beta\times\pi(u,t-\delta t) + (1-\beta)\times\rho(u,t)$$
where $\rho(u,t)$ is the number of resources used by user $u$ at time $t$,
and $\beta=0.5^{{\delta t}/h}$. $h$ is the half life period set by 
\Macro{PRIORITY\_HALFLIFE}.

The EUP of user $u$ at time $t$, $\pi_e(u,t)$
is calculated by
$$\pi_e(u,t) = \pi_r(u,t)\times f(u,t)$$
where $f(u,t)$ is the priority boost factor for user $u$ at time $t$.

As mentioned previously, the RUP calculation is designed so that at steady
state, each user's RUP stabilizes at the number of resources used by that user. 
The definition of $\beta$ ensures that the calculation of $\pi_r(u,t)$ can be 
calculated over non-uniform time intervals $\delta t$ without affecting the 
calculation.  The time interval $\delta t$ varies due to events internal to 
the system, but Condor guarantees that unless the central manager machine is 
down, no matches will be unaccounted for due to this variance.

% Derek's explanation:
%  > Preferably the user priority is determined by the number of
%  > processors jobs of the user currently occupy, i.e., the "history"
%  > should not play a role.
%  
%  this is the responsibility of the condor "accountant", which lives
%  inside the condor_negotiator daemon.  the knob you want to turn is
%  called "PRIORITY_HALFLIFE".  think of your user priority as a
%  radioactive substance. :) consider a priority that exactly matches
%  your current resource usage the "stable state", and a priority
%  "contaminated" with past usage "radioactive."  if it's got a long
%  halflife, it takes a long time for your priority to decay back to
%  "normal".  if the halflife is very short, it'll decay very quickly,
%  and will remain very close to your current usage.  so, just set
%  PRIORITY_HALFLIFE to a small floating point value (like 0.0001), and
%  your user priority should always match your current usage.  if you're
%  not using any resources, your priority will go back to the baseline
%  value instantly.
