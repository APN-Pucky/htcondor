%%%%%%%%%%%%%%%%%%%%%%%%%%%%%%%%%%%%%%%%%%%%%%%%%%%%%%%%%%%%%%%%%%%%%%
\section{\label{sec:UserPrio}User Priorities and Negotiation}
%%%%%%%%%%%%%%%%%%%%%%%%%%%%%%%%%%%%%%%%%%%%%%%%%%%%%%%%%%%%%%%%%%%%%%

% Karen's understanding of this stuff, in preparation for a re-write
% of the section:

% Users request machines (by submitting jobs).
% Each user has a calculated priority.
%   A larger priority is worse.
%   This priority essentially tells how many machines the user is
%   currently using.  The priority can be made worse (larger number)
%   by the settings of various configuration variables.
% During each negotiation cycle, all machines are allocated (presuming
%   that there are more requests than machines).
% Each user is allocated machines in a ratio of 1/user's priority.
% Within the negotiation cycle, each user is given an initial
%   allocation of machines.  From there, remaining unallocated machines
%   are divided up among users that want more.  In a round robin
%   manner, each user is allocated a fraction of the remaining
%   unallocated machines.  this fraction is 1/user's priority.

\index{priority!in machine allocation}
\index{user priority}
HTCondor uses priorities to determine machine allocation for jobs.
This section details the priorities and the allocation of
machines (negotiation).

For accounting purposes, each user is identified by username@uid\_domain.
Each user is assigned a priority value even if submitting jobs from
different machines in the same domain, or even if submitting from multiple
machines in the different domains.

The numerical priority value assigned to a user is inversely related to the 
\emph{goodness} of the priority.
A user with a numerical priority of 5 gets 
more resources than a user with a numerical priority of 50.
There are two 
priority values assigned to HTCondor users:
\begin{itemize}
	\item Real User Priority (RUP), which measures resource usage of the 
		user.
	\item Effective User Priority (EUP), which determines the number of
		resources the user can get.
\end{itemize}
This section describes these two priorities and how they affect resource
allocations in HTCondor.
Documentation on configuring and controlling 
priorities may be found in section~\ref{sec:Negotiator-Config-File-Entries}.

%%%%%%%%%%%%%%%%%%%%%%%%%%%%%%%%%%%%%%%%%%%%%%%%%%%%%%%%%%%%%%%%%%%%%%
\subsection{\label{sec:RUP}Real User Priority (RUP)}
%%%%%%%%%%%%%%%%%%%%%%%%%%%%%%%%%%%%%%%%%%%%%%%%%%%%%%%%%%%%%%%%%%%%%%
\index{real user priority (RUP)}
\index{user priority!real (RUP)}
A user's RUP measures the resource usage of the user 
through time.
Every user begins with a RUP of one half (0.5), and
at steady state, the RUP of a user equilibrates to the number of resources 
used by that user.  Therefore, if a specific user continuously uses exactly 
ten resources for a long period of time, the RUP of that user stabilizes at 
ten.

However, if the user decreases the number of resources used, the RUP
gets better.  The rate at which the priority value decays 
can be set by the macro \Macro{PRIORITY\_HALFLIFE}, a time period 
defined in seconds.   Intuitively, if the \Macro{PRIORITY\_HALFLIFE} in a pool 
is set to 86400 (one day), and if a user whose RUP was 10 has no
running jobs, 
that user's RUP would be 5 one day later, 2.5 two days later,
and so on.

%%%%%%%%%%%%%%%%%%%%%%%%%%%%%%%%%%%%%%%%%%%%%%%%%%%%%%%%%%%%%%%%%%%%%%
\subsection{\label{sec:EUP}Effective User Priority (EUP)}
%%%%%%%%%%%%%%%%%%%%%%%%%%%%%%%%%%%%%%%%%%%%%%%%%%%%%%%%%%%%%%%%%%%%%%
\index{effective user priority (EUP)}
\index{user priority!effective (EUP)}
The effective user priority (EUP) of a user is used to determine
how many resources that user may receive.
The EUP is linearly related to the RUP
by a \emph{priority factor} which may be defined on a per-user basis.
Unless otherwise configured, 
an initial priority factor for all users as they first submit jobs
is set by the configuration variable \Macro{DEFAULT\_PRIO\_FACTOR},
and defaults to the value 1.0.
This default makes the EUP the same as the RUP.
If desired, the priority factors of
specific users can be increased using \Condor{userprio},
so that some are served preferentially.

The number of resources that a user may receive is inversely related
to the ratio between the EUPs of submitting users.
Therefore user $A$ with EUP=5 will receive
twice as many resources as user $B$ with EUP=10 and four times as many 
resources as user $C$ with EUP=20.
However, if $A$ does not use the full number
of resources that $A$ may be given,
the available resources are repartitioned and distributed among
remaining users according to the inverse ratio rule.

HTCondor supplies mechanisms to directly support two policies in which EUP may
be useful:
\begin{description}
	\item[Nice users]  A job may be submitted with the submit command 
	\SubmitCmd{nice\_user} set to \Expr{True}.
	This nice user job will have its RUP boosted by the 
	\Macro{NICE\_USER\_PRIO\_FACTOR} priority factor specified in the 
	configuration, leading to a very large EUP.
	This corresponds to a low priority for resources,
	therefore using resources not used by other HTCondor users.

	\item[Remote Users] HTCondor's flocking feature (see
	section~\ref{sec:Flocking}) allows jobs to run in a pool
        other than the local one.
	In addition, the submit-only feature allows a user 
	to submit jobs to another pool.
	In such situations, submitters from other domains
	can submit to the local pool.
	It may be desirable to have HTCondor treat local users
	preferentially over these remote users.
	If configured, HTCondor will boost the RUPs of remote users by
	\Macro{REMOTE\_PRIO\_FACTOR}
	specified in the configuration,
	thereby lowering their priority for resources.
\end{description}

The priority boost factors for individual users can be set with the 
\Opt{setfactor} option of \Condor{userprio}.
Details may be found in the \Condor{userprio} manual page 
on page~\pageref{man-condor-userprio}.


%%%%%%%%%%%%%%%%%%%%%%%%%%%%%%%%%%%%%%%%%%%%%%%%%%%%%%%%%%%%%%%%%%%%%%
\subsection{Priorities in Negotiation and Preemption}
%%%%%%%%%%%%%%%%%%%%%%%%%%%%%%%%%%%%%%%%%%%%%%%%%%%%%%%%%%%%%%%%%%%%%%
\index{negotiation!priority}
\index{matchmaking!priority}
\index{preemption!priority}
Priorities are used to ensure that users get their fair share of resources.  
The priority values are used at allocation time, meaning during
negotiation and matchmaking.
Therefore, there are ClassAd attributes that take on defined values
only during negotiation, making them ephemeral.
In addition to allocation, HTCondor may preempt a machine claim 
and reallocate it when conditions change.

Too many preemptions lead to thrashing,
a condition in which negotiation for a machine identifies a
new job with a better priority most every cycle.
Each job is, in turn, preempted, and no job finishes.
To avoid this situation, the \Macro{PREEMPTION\_REQUIREMENTS} 
configuration variable is defined for and used only by 
the \Condor{negotiator} daemon
to specify the conditions that must be met for a preemption to
occur.  It is usually defined to deny preemption if a current running
job has been running for a relatively short period of time.  This
effectively limits the number of preemptions per resource per time
interval.
Note that \MacroNI{PREEMPTION\_REQUIREMENTS} only applies to preemptions
due to user priority.  It does not have any effect if the machine's 
\MacroNI{RANK}
expression prefers a different job, or if the machine's policy
causes the job to vacate due to other activity on the machine.
See section \ref{sec:Disabling Preemption} for a general discussion of
limiting preemption.

The following ephemeral attributes may be used within policy definitions.
Care should be taken when using these attributes, 
due to their ephemeral nature;
they are not always defined, so the usage of an expression to 
check if defined such as
\begin{verbatim}
  (RemoteUserPrio =?= UNDEFINED)
\end{verbatim}
is likely necessary.

Within these attributes, those with names that contain the
string \Attr{Submitter} refer to characteristics about the candidate job's user;
those with names that contain the string \Attr{Remote}
refer to characteristics about the user currently using the resource.
Further,  those with names that end with the
string \Attr{ResourcesInUse} have values that may change within
the time period associated with a single negotiation cycle.
Therefore, the configuration variables \Macro{PREEMPTION\_REQUIREMENTS\_STABLE}
and and \Macro{PREEMPTION\_RANK\_STABLE} exist to inform the 
\Condor{negotiator} daemon that values may change.
See section~\ref{param:PreemptionRequirementsStable} on
page~\pageref{param:PreemptionRequirementsStable} for
definitions of these configuration variables.

\begin{description}

\item[\index{ClassAd attribute, ephemeral!SubmitterUserPrio}\AdAttr{SubmitterUserPrio}:] 
  A floating point value representing the user priority of the 
  candidate job.
\item[\index{ClassAd attribute, ephemeral!SubmitterUserResourcesInUse}\AdAttr{SubmitterUserResourcesInUse}:] 
  The integer number of slots currently utilized by the user
  submitting the candidate job.
\item[\index{ClassAd attribute, ephemeral!RemoteUserPrio}\AdAttr{RemoteUserPrio}:]
  A floating point value representing the user priority of the 
  job currently running on the machine.
  This version of the attribute, with no slot represented in the attribute name,
  refers to the current slot being evaluated.
\item[\index{ClassAd attribute, ephemeral!Slot<N>\_RemoteUserPrio}\AdAttr{Slot<N>\_RemoteUserPrio}:]
  A floating point value representing the user priority of the 
  job currently running on the particular slot represented
  by \verb@<N>@ on the machine.
\item[\index{ClassAd attribute, ephemeral!RemoteUserResourcesInUse}\AdAttr{RemoteUserResourcesInUse}:]
  The integer number of slots currently utilized by the user
  of the job currently running on the machine.
\item[\index{ClassAd attribute, ephemeral!SubmitterGroupResourcesInUse}\AdAttr{SubmitterGroupResourcesInUse}:]
  If the owner of the candidate job is a member of a valid accounting group,
  with a defined group quota,
  then this attribute is the integer number of slots currently 
  utilized by the group.
\item[\index{ClassAd attribute, ephemeral!SubmitterGroup}\AdAttr{SubmitterGroup}:]
  The accounting group name of the requesting submitter.
\item[\index{ClassAd attribute, ephemeral!SubmitterGroupQuota}\AdAttr{SubmitterGroupQuota}:]
  If the owner of the candidate job is a member of a valid accounting group,
  with a defined group quota,
  then this attribute is the integer number of slots defined as the
  group's quota.
\item[\index{ClassAd attribute, ephemeral!RemoteGroupResourcesInUse}\AdAttr{RemoteGroupResourcesInUse}:]
  If the owner of the currently running job is a member of a valid 
  accounting group, with a defined group quota,
  then this attribute is the integer number of slots currently 
  utilized by the group.
\item[\index{ClassAd attribute, ephemeral!RemoteGroup}\AdAttr{RemoteGroup}:]
  The accounting group name of the owner of the currently running job.
\item[\index{ClassAd attribute, ephemeral!RemoteGroupQuota}\AdAttr{RemoteGroupQuota}:]
  If the owner of the currently running job is a member of a valid 
  accounting group, with a defined group quota,
  then this attribute is the integer number of slots defined as the
  group's quota.
\item[\index{ClassAd attribute, ephemeral!SubmitterNegotiatingGroup}\AdAttr{SubmitterNegotiatingGroup}:]
  The accounting group name that the candidate job is negotiating under.
\item[\index{ClassAd attribute, ephemeral!RemoteNegotiatingGroup}\AdAttr{RemoteNegotiatingGroup}:]
  The accounting group name that the currently running job negotiated under.
\item[\index{ClassAd attribute, ephemeral!SubmitterAutoregroup}\AdAttr{SubmitterAutoregroup}:]
  Boolean attribute is \Expr{True} if candidate job is negotiated via autoregoup.
\item[\index{ClassAd attribute, ephemeral!RemoteAutoregroup}\AdAttr{RemoteAutoregroup}:]
  Boolean attribute is \Expr{True} if currently running job negotiated via autoregoup.
\end{description}


%%%%%%%%%%%%%%%%%%%%%%%%%%%%%%%%%%%%%%%%%%%%%%%%%%%%%%%%%%%%%%%%%%%%%%
\subsection{Priority Calculation}
%%%%%%%%%%%%%%%%%%%%%%%%%%%%%%%%%%%%%%%%%%%%%%%%%%%%%%%%%%%%%%%%%%%%%%
This section may be skipped if the reader so feels, but for the curious,
here is HTCondor's priority calculation algorithm.

The RUP of a user $u$ at time $t$, $\pi_r(u,t)$, is calculated 
every time interval $\delta t$ using the formula 
$$\pi_r(u,t) = \beta\times\pi(u,t-\delta t) + (1-\beta)\times\rho(u,t)$$
where $\rho(u,t)$ is the number of resources used by user $u$ at time $t$,
and $\beta=0.5^{{\delta t}/h}$. $h$ is the half life period set by 
\Macro{PRIORITY\_HALFLIFE}.

The EUP of user $u$ at time $t$, $\pi_e(u,t)$
is calculated by
$$\pi_e(u,t) = \pi_r(u,t)\times f(u,t)$$
where $f(u,t)$ is the priority boost factor for user $u$ at time $t$.

As mentioned previously, the RUP calculation is designed so that at steady
state, each user's RUP stabilizes at the number of resources used by that user. 
The definition of $\beta$ ensures that the calculation of $\pi_r(u,t)$ can be 
calculated over non-uniform time intervals $\delta t$ without affecting the 
calculation.  The time interval $\delta t$ varies due to events internal to 
the system, but HTCondor guarantees that unless the central manager machine is 
down, no matches will be unaccounted for due to this variance.

% Derek's explanation:
%  > Preferably the user priority is determined by the number of
%  > processors jobs of the user currently occupy, i.e., the "history"
%  > should not play a role.
%  
%  this is the responsibility of the condor "accountant", which lives
%  inside the condor_negotiator daemon.  the knob you want to turn is
%  called "PRIORITY_HALFLIFE".  think of your user priority as a
%  radioactive substance. :) consider a priority that exactly matches
%  your current resource usage the "stable state", and a priority
%  "contaminated" with past usage "radioactive."  if it's got a long
%  halflife, it takes a long time for your priority to decay back to
%  "normal".  if the halflife is very short, it'll decay very quickly,
%  and will remain very close to your current usage.  so, just set
%  PRIORITY_HALFLIFE to a small floating point value (like 0.0001), and
%  your user priority should always match your current usage.  if you're
%  not using any resources, your priority will go back to the baseline
%  value instantly.

%%%%%%%%%%%%%%%%%%%%%%%%%%%%%%%%%%%%%%%%%%%%%%%%%%%%%%%%%%%%%%%%%%%%%%
\subsection{\label{sec:negotiation}Negotiation}
%%%%%%%%%%%%%%%%%%%%%%%%%%%%%%%%%%%%%%%%%%%%%%%%%%%%%%%%%%%%%%%%%%%%%%
\index{negotiation}
\index{matchmaking!negotiation algorithm}

Negotiation is the method HTCondor undergoes periodically to
match queued jobs with resources capable of running jobs.
The \Condor{negotiator} daemon is responsible for
negotiation.

During a negotiation cycle, the \Condor{negotiator} daemon
accomplishes the following ordered list of items.
\begin{enumerate}
  \item Build a list of all possible resources,
  regardless of the state of those resources.
  \item Obtain a list of all job submitters (for the entire pool).
  \item Sort the list of all job submitters based on EUP
  (see section ~\ref{sec:EUP} for an explanation of EUP).
  The submitter with the best priority is first within the sorted list.
  \item Iterate until there are either no more resources to match,
  or no more jobs to match.
    \begin{description}
      \item For each submitter (in EUP order):
        \begin{description}
          \item For each submitter, get each job.
	  Since jobs may be submitted from more than one machine
	  (hence to more than one \Condor{schedd} daemon),
	  here is a further definition of the ordering of these jobs.
	  With jobs from a single \Condor{schedd} daemon, 
	  jobs are typically returned in job priority order.
	  When more than one \Condor{schedd} daemon is involved,
	  they are contacted in an undefined order. 
	  All jobs from a single \Condor{schedd} daemon are considered
	  before moving on to the next.
	  For each job:
              \begin{itemize}
	        \item For each machine in the pool that can execute jobs:
                  \begin{enumerate}
		  \item If \Expr{machine.requirements} evaluates to
		  \Expr{False} or
		  \Expr{job.requirements} evaluates to \Expr{False},
		  skip this machine
		  \item If the machine is in the Claimed state, but
		  not running a job, skip this machine.
		  \item If this machine is not running a job, add it to
		  the potential match list by reason of
		  No Preemption.
		  \item If the machine is running a job
                    \begin{itemize}
		    \item If the \Expr{machine.RANK} on this job is
		    better than the running job, add this machine to the
		    potential match list by reason of Rank.
		    \item  If the EUP of this job is better than the EUP
		    of the currently running job, and
		    \MacroNI{PREEMPTION\_REQUIREMENTS} is \Expr{True},
		    and the \Expr{machine.RANK} on this job is not worse
		    than the currently running job, add this machine to the
		    potential match list by reason of Priority.
                    \end{itemize}
                  \end{enumerate}
	        \item Of machines in the potential match list, sort by
		\MacroNI{NEGOTIATOR\_PRE\_JOB\_RANK}, \Expr{job.RANK},
		\MacroNI{NEGOTIATOR\_POST\_JOB\_RANK}, Reason for claim
		(No Preemption, then Rank, then Priority),
		\MacroNI{PREEMPTION\_RANK}
	        \item The job is assigned to the top machine on the
		potential match list.  The machine is removed
		from the list of resources to match
		(on this negotiation cycle).
              \end{itemize}
          \end{description}
    \end{description}
\end{enumerate}

The \Condor{negotiator} asks the \Condor{schedd} for the 
"next job" from a given submitter/user.
Typically, the \Condor{schedd} returns jobs in the order of
job priority.
If priorities are the same,
job submission time is used; older jobs go first.
If a cluster has multiple procs in it and one of the jobs cannot be matched,
the \Condor{schedd} will not return any more jobs in that cluster
on that negotiation pass.
This is an optimization based on the theory that the cluster jobs are similar.
The configuration variable \Macro{NEGOTIATE\_ALL\_JOBS\_IN\_CLUSTER}
disables the cluster-skipping optimization.
Use of the configuration variable \Macro{SIGNIFICANT\_ATTRIBUTES}
will change the
definition of what the \Condor{schedd} considers a cluster from the
default definition of all jobs that share the same \Attr{ClusterId}.

%%%%%%%%%%%%%%%%%%%%%%%%%%%%%%%%%%%%%%%%%%%%%%%%%%%%%%%%%%%%%%%%%%%%%%
\subsection{\label{sec:PieSlice}The Layperson's Description of the Pie Spin and Pie Slice}
%%%%%%%%%%%%%%%%%%%%%%%%%%%%%%%%%%%%%%%%%%%%%%%%%%%%%%%%%%%%%%%%%%%%%%
\index{pie slice}
\index{pie spin}
\index{scheduling!pie slice}
\index{scheduling!pie spin}

HTCondor schedules in a variety of ways.
First, it takes all users who have submitted jobs and calculates their priority.
Then, it totals the number of resources available at the moment,
and using the ratios of the user priorities,
it calculates the number of machines each user could get.
This is their \Term{pie slice}.

The HTCondor matchmaker goes in user priority order, 
contacts each user, and asks for job information. 
The \Condor{schedd} daemon (on behalf of a user)
tells the matchmaker about a job,
and the matchmaker looks at available resources
to create a list of resources that match the requirements expression.
With the list of resources that match,
it sorts them according to the rank expressions within ClassAds.
If a machine prefers a job, the job is assigned to that machine,
potentially preempting a job that might already be running on that machine.
Otherwise, give the machine to the job that the job ranks highest.
If the machine ranked highest is already running a job,
we may preempt running job
for the new job. 
A default policy for preemption states that the user must
have a 20\Percent\  better priority in order for preemption to succeed.
If the job has no preferences as to what sort of machine it gets,
matchmaking gives it the first idle resource to meet its requirements.

This matchmaking cycle continues until the user has received all of the
machines in their pie slice.
The matchmaker then contacts the next highest
priority user and offers that user their pie slice worth of machines.
After contacting all users,
the cycle is repeated with any still available resources
and recomputed pie slices.
The matchmaker continues \Term{spinning the pie} 
until it runs out of machines or all the \Condor{schedd} daemons
say they have no more jobs. 

%%%%%%%%%%%%%%%%%%%%%%%%%%%%%%%%%%%%%%%%%%%%%%%%%%%%%%%%%%%%%%%%%%%%%%
\subsection{\label{sec:group-accounting}Group Accounting}
%%%%%%%%%%%%%%%%%%%%%%%%%%%%%%%%%%%%%%%%%%%%%%%%%%%%%%%%%%%%%%%%%%%%%%
\index{groups!accounting}
\index{accounting!by group}
\index{priority!by group}

By default, HTCondor does all accounting on a per-user basis, and this
accounting is primarily used to compute priorities for HTCondor's
fair-share scheduling algorithms. 
However, accounting can also be done on a per-group basis.
Multiple users can all submit jobs into the same accounting group,
and all jobs with the same accounting group
will be treated with the same priority.
Jobs that do \emph{not} specify an accounting group have 
all accounting and priority based on the user, 
which may be identified by the job ClassAd attribute \Attr{Owner}.
Jobs that do specify an accounting group have 
all accounting and priority based on the specified accounting group.
Therefore, accounting based on groups only works when
the jobs correctly identify their group membership.

\index{ClassAd job attribute!AcctGroup}
\index{ClassAd job attribute!AcctGroupUser}
The preferred method for having a job associate itself with
an accounting group adds a command to the submit description file
that specifies the group name:
\begin{verbatim}
  accounting_group = group_physics
\end{verbatim}
This command causes the job ClassAd attribute \Attr{AcctGroup} to
be set with this group name.

If the user name of the job submitter should be other than the \Attr{Owner}
job ClassAd attribute, an additional command specifies the user name:
\begin{verbatim}
  accounting_group_user = albert
\end{verbatim}
This command causes the job ClassAd attribute \Attr{AcctGroupUser} to
be set with this user name.

\index{ClassAd job attribute!AccountingGroup}
The previous method for defining accounting groups
is no longer recommended. 
It inserted the job ClassAd attribute \AdAttr{AccountingGroup}
by setting it in the submit description file using the syntax 
in this example:
\begin{verbatim}
+AccountingGroup = "group_physics.albert"
\end{verbatim}

In this previous method for defining accounting groups,
the \AdAttr{AccountingGroup} attribute is a string,
and it therefore must be enclosed in double quote marks.

Much of the reason that the previous method for defining accounting groups
is no longer recommended  is that the name of an accounting is that it
used the period (.) character to separate the group name from the
user name.  Therefore, the syntax did not work if a user name contained
a period.

The name should \emph{not} be qualified with a domain.
Certain parts of the HTCondor system 
do append the value \MacroUNI{UID\_DOMAIN}
(as specified in the configuration file on the submit machine)
to this string for internal use.
For example, if the value of \MacroNI{UID\_DOMAIN} is
\Expr{example.com}, and the accounting group name
is as specified,
\Condor{userprio} will show statistics
for this accounting group using the appended domain, for example
\footnotesize
\begin{verbatim}
                                    Effective
User Name                           Priority
------------------------------      ---------
group_physics@example.com                0.50
user@example.com                        23.11
heavyuser@example.com                  111.13
...
\end{verbatim}
\normalsize

Additionally, the \Condor{userprio} command allows administrators to
remove an entity from the accounting system in HTCondor.
The \Opt{-delete} option to \Condor{userprio}
accomplishes this
if all the jobs from a given accounting group are completed,
and the administrator wishes to remove that group from the system.
The \Opt{-delete} option
identifies the accounting group with the fully-qualified name
of the accounting group.
For example
\footnotesize
\begin{verbatim}
condor_userprio -delete group_physics@example.com
\end{verbatim}
\normalsize

HTCondor removes entities itself as they are no longer
relevant.
Intervention by an administrator to delete entities can
be beneficial when the use of thousands
of short term accounting groups leads to scalability
issues.

%%%%%%%%%%%%%%%%%%%%%%%%%%%%%%%%%%%%%%%%%%%%%%%%%%%%%%%%%%%%%%%%%%%%%%
\subsection{\label{sec:group-quotas}Accounting Groups with Hierarchical Group Quotas}
%%%%%%%%%%%%%%%%%%%%%%%%%%%%%%%%%%%%%%%%%%%%%%%%%%%%%%%%%%%%%%%%%%%%%%

\index{hierarchical group quotas}
\index{negotiation!by group}
\index{groups!quotas}
\index{quotas!hierarchical quotas for a group}
An upper limit on the number of slots/machines allocated for a group
of users can be specified with group quotas.
The use of these group quotas modifies the negotiation for 
available resources (machines) within an HTCondor pool.
When accounting groups together with group quotas are specified,
the priorities and usage information are calculated per user.
Numbers of slots/machines can be negotiated for preferentially
by group.
This may be useful when
different groups (of varying size) own computers,
and the groups choose to combine their computers to
form an HTCondor pool.
Consider an imaginary HTCondor pool example with thirty computers;
twenty computers are owned by the physics group and ten
computers are owned by the chemistry group.
One notion of fair allocation could be implemented 
by configuring the twenty machines owned by the physics group
to prefer (using the \MacroNI{RANK} configuration macro)
jobs submitted by the users identified as associated
with the physics group.
Likewise, the ten machines owned by the chemistry group are
configured to prefer jobs from users associated with the
the chemistry group.
This routes jobs to execute on specific machines,
perhaps causing more preemption than necessary.
The (fair allocation) policy desired is likely somewhat different,
if these thirty machines have been pooled.
The desired policy does not tie users to specific sets of machines,
but caps the maximum number of slots/machines (a quota).
Given thirty similar machines,
the desired policy allows users within the physics group to have
preference on up to twenty of the machines within the pool,
and the machines can be any of the machines that are available.

The implementation of quotas is hierarchical,
such that quotas may be described for groups, subgroups,
sub subgroups, etc.  
The hierarchy is described by adherence to a naming scheme
set up in advance.
Configuration defines the quotas in terms of limiting the number of
cores allocated for a group or subgroup.  
And, the jobs state which group or subgroup they are part of.

A quota for a set of users requires an identification of
the set; members are called group users.
Group users' jobs identify themselves as desiring negotiation
under the group by setting the \SubmitCmd{accounting\_group}
command in the submit description file,
as specified in section~\ref{sec:group-accounting}.

The hierarchy is identified by using the period ('.') character
to separate a group name from a subgroup name from a sub subgroup
name, etc.
Group names are case-insensitive for negotiation.
The topmost level group name is not required to begin with the
string \AdStr{group\_},
as in the examples
\AdStr{group\_physics.experiment1} and \AdStr{group\_chemistry.organic},
but it is a useful convention,
because group names must not conflict with subgroup names.

Configuration controls the order of negotiation for
groups, subgroups within the hierarchy defined, and individual users,
as well as sets quotas.

\index{<none> group}
\index{group accounting!<none> group}
When at least one group quota is defined,
HTCondor must still do negotiation for jobs that are not party
to the groups.
Any job that is \emph{not} part of any group is considered part of the
invented \AdStr{<none>} group that is the top or root of the
hierarchical tree of groups.
This string will appear in the output of \Condor{status}.

Quotas are categorized as either static or dynamic.
A static quota specifies an integral numbers of machines (slots),
independent of the size of the pool.
A dynamic quota specifies a percentage of machines (slots) calculated
based on the current number of machines in the pool.
It is intended that only one of static or a dynamic quotas are defined 
for a specified group and for the pool.
If both are defined for a group, then the static quota is implemented, 
and the dynamic quota is ignored.
The behavior is not defined and not documented
for pools with both static and dynamic quotas defined.

\begin{description}
\item[Static Quotas]
In the hierarchical implementation,
there are two cases defined here,
to specify for the allocation of machines where
there is both a group and a subgroup.
In the first case, the sum for the numbers of machines
within all of a group's subgroups totals to fewer than the specification for
the group's static quota.
For example:
\begin{verbatim}
  GROUP_QUOTA_group_physics = 100
  GROUP_QUOTA_group_physics.experiment1 = 20
  GROUP_QUOTA_group_physics.experiment2 = 70
\end{verbatim}
In this case, the unused quota of 10 machines is assigned to
the \Expr{group\_physics} submitters.  

In the second case, the specification for the numbers of machines
of a set of subgroups totals to more than the specification for
the group's quota.
For example:
\begin{verbatim}
  GROUP_QUOTA_group_chemistry = 100
  GROUP_QUOTA_group_chemistry.lab1 = 40
  GROUP_QUOTA_group_chemistry.lab2 = 80
\end{verbatim}
In this case, 
a warning is written to the log for the \Condor{negotiator} daemon,
and each of the subgroups will have their static quota scaled.
In this example, the ratio 100/120 scales each subgroup.
\Expr{lab1} will have a revised (floating point) quota of 33.333 machines,
and \Expr{lab2} will have a revised (floating point) quota of 66.667 machines.
As numbers of machines are always integer values,
the floating point values are truncated for quota allocation.
Fractional remainders resulting from the truncation are summed
and assigned to the next higher level within the group hierarchy.

\item[Dynamic Quotas]
A dynamic quota specifies a percentage of machines (slots) calculated
based on the quota of the next higher level group within the hierarchy.
For groups at the top level, a dynamic quota specifies a percentage of
machines (slots) that currently exist in the pool.
The quota is specified for a group (subgroup, etc.) by a floating point
value in range 0.0 to 1.0 (inclusive).

Like static quota specification, there are two cases defined:
when the dynamic quotas of all sub groups of a specific group
sum to a fraction less than 1.0,
and when the dynamic quotas of all sub groups of a specific group
sum to greater than 1.0.

Here is an example configuration in which dynamic group quotas are assigned for
a single group and its subgroups.
\begin{verbatim}
  GROUP_QUOTA_DYNAMIC_group_econ = .6
  GROUP_QUOTA_DYNAMIC_group_econ.project1 = .2
  GROUP_QUOTA_DYNAMIC_group_econ.project2 = .15
  GROUP_QUOTA_DYNAMIC_group_econ.project3 = .2
\end{verbatim}
The sum of dynamic quotas for the subgroups is .55, 
which is less than 1.0.
If the pool has 100 slots, 
then the \Expr{project1} subgroup is assigned a quota that
equals (100)(.6)(.2) = 12 machines. 
The \Expr{project2} subgroup is assigned a quota that
equals (100)(.6)(.15) = 9 machines. 
The \Expr{project3} subgroup is assigned a quota that
equals (100)(.6)(.2) = 12 machines. 
The 60-33=27 machines unused by the subgroups are assigned
for use by job submitters in the parent \Expr{group\_econ} group.

If the calculated dynamic quota of the subgroups resulted in non integer
numbers of machines,
integer numbers of machines are assigned based on the truncation of the
non integer dynamic group quota.
The unused, surplus quota of machines resulting from
fractional remainders resulting from the truncation are summed
and assigned to the next higher level within the group hierarchy.

Here is another example configuration in which dynamic group quotas 
are assigned for a single group and its subgroups.
\begin{verbatim}
  GROUP_QUOTA_DYNAMIC_group_stat = .5
  GROUP_QUOTA_DYNAMIC_group_stat.project1 = .4
  GROUP_QUOTA_DYNAMIC_group_stat.project2 = .3
  GROUP_QUOTA_DYNAMIC_group_stat.project3 = .4
\end{verbatim}
In this case, the sum of dynamic quotas for the subgroups is 1.1, 
which is greater than 1.0 .
A warning is written to the log for the \Condor{negotiator} daemon,
and each of the subgroups will have their dynamic group quota scaled
for this example.
.4 becomes .4/1.1=.3636,
and .3 becomes .3/1.1=.2727 .
If the pool has 100 slots, 
then each of the \Expr{project1} and \Expr{project3} subgroups 
is assigned a dynamic quota of
(100)(.5)(.3636), which is 18.1818 machines. 
The \Expr{project2} subgroup is assigned a dynamic quota of
(100)(.5)(.2727), which is 13.6364 machines. 
The quota for each of \Expr{project1} and \Expr{project3} results 
in the truncated amount of 18 machines,
and \Expr{project2} results in the truncated amount of 13 machines,
with the 0.1818 + .6364 + .1818 = 1.0 remaining machine assigned to
job submitters in the parent group, \Expr{group\_stat}. 

\end{description}

