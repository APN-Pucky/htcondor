%%%%%%%%%%%%%%%%%%%%%%%%%%%%%%%%%%%%%%%%%%%%%%%%%%%%%%%%%%%%%%%%%%%%%%
\section{User Priorities in the Condor System}
\label{sec:UserPrio}
%%%%%%%%%%%%%%%%%%%%%%%%%%%%%%%%%%%%%%%%%%%%%%%%%%%%%%%%%%%%%%%%%%%%%%
For accounting purposes, each user is identified by ``username@uid\_domain''
so users have the same priority value even if they begin submitting from a 
different machine in the same domain, or even submit from multiple machines 
in the same domain.

The numerical priorities values assigned to users is inversely related to the 
``goodness'' of the priority, so a user with a numerical priority of 5 will get 
more resources than a user with a numerical priority of 50.  There are two 
priority values assigned to Condor users:
\begin{itemize}
	\item The Real User Priority (RUP), which measures resource usage of the 
		user, and
	\item The Effective User Priority (EUP), which determines the number of
		resources the user can get.
\end{itemize}
This section describes these two priorities and how they effect resource
allocations in Condor.  Documentation on configuring and controlling 
priorities may be found in section~\ref{sec:Negotiator-Config-File-Entries}.

\subsection{Real User Priority (RUP)}
The real user priority of a user measures the resource usage of the user 
through time.  Every user begins with a RUP of half (i.e., 0.5) and,
at steady state, the RUP of a user equilibrates to the number of resources 
used by that user.  Therefore, if a particular user continuously uses exactly 
ten resources for a long period of time, the RUP of that user stabilizes at 
ten.

However, if the user decreases the number of resources used, the RUP value of 
user drops (i.e., gets better).  The rate at which the priority value decays 
can be set by the macro \Macro{PRIORITY\_HALFLIFE}, which is a time period 
defined in seconds.   Intuitively, if the \Macro{PRIORITY\_HALFLIFE} in a pool 
is set to 86400 (i.e., one day) and if a user whose RUP was 10 removes all his 
jobs, the user's RUP would be 5 one day later, 2.5 two days hence, etc.

\subsection{Effective User Priority (EUP)}
The effective user priority of a user is used to determine how many resources 
that user can get.  The EUP of a user is always linearly related to the RUP
by a \emph{priority factor} which may be defined on a per-user basis.  Unless 
otherwise configured, the priority factor for all users is 1.0, and so the EUP 
is the same as the the RUP.  However, if required, the priority factors of
particular users (such as remote submitters) can be increased so that 
remaining users are served preferentially.

The number of resources that a user can claim is inversely related to the ratio 
between the EUPs of submitting users.  Therefore user $A$ with EUP 5 will get 
twice as many resources as user $B$ with EUP 10, and four times as many 
resources as user $C$ with EUP 20.  However, if $A$ does not use his ``quota''
of resources, the available resources are repartitioned and distributed among
remaining users in accordance to the inverse ratio rule.

Condor supplies mechanisms to directly support two scenarios in which EUP may
be useful:
\begin{description}
	\item[Nice users]  A job may be submitted with the parameter 
	\texttt{nice\_user} set to true in the submit command file.  A nice user
	job automatically gets its RUP boosted by the 
	\Macro{NICE\_USER\_PRIO\_FACTOR} priority factor specified in the 
	configuration file, leading to a (usually very large) EUP.  These jobs are 
	therefore equivalent to ``background jobs'' which use resources not used 
	by other users of Condor.

	\item[Remote Users]  The flocking feature of Condor (see 
	section~\ref{sec:Flocking}) allows the \Condor{schedd} to submit to more
	than one pool.  In such situations, one may have submitters from other
	domains submitting into the local pool.  It is often desirable to have
	Condor treat local users preferentially over such remote users.  If
	configured, Condor will boost the RUPs of remote users by 
	\Macro{REMOTE\_PRIO\_FACTOR}, specified in the configuration file.
\end{description}

The priority boost factors for individual users can be set with the 
\Opt{setfactor} option of \Condor{userprio}, for which documentation can
be found in section~\ref{man-condor-userprio}.

\subsection{Priorities and Preemption}
Priorities are used to ensure that users get their fair share of resources.  
The priority values are used at allocation time as discussed previously, and
the system additionally preempts machines (by performing a checkpoint and
vacate) and reallocates them to avoid priority inversion.

To ensure that preemptions do not lead to ``thrashing,'' a 
\Macro{PREEMPTION\_HOLD} expression may be defined to deny preemption.  This
expression is usually defined to deny preemption if the current running job
has been running there for a relatively short period of time, effectively
limiting the number of preemptions per resource per time interval.

\subsection{Priority Calculation}
This section may be skipped if the reader so feels, but for the curious,
we now describe Condor's priority calculation algorithm.

The RUP of a user $u$ at time $t$, $\pi_r(u,t)$, is calculated 
every time interval $\delta t$ using the formula 
$$\pi_r(u,t) = \beta\times\pi(u,t-\delta t) + (1-\beta)\times\rho(u,t)$$
where $\rho(u,t)$ is the number of resources used by user $u$ at time $t$,
and $\beta=0.5^{{\delta t}/h}$ ($h$ is the half life period set by 
\Macro{PRIORITY\_HALFLIFE}).  The EUP of user $u$ at time $t$, $\pi_e(u,t)$
is calculated by
$$\pi_e(u,t) = \pi_r(u,t)\times f(u,t)$$
where $f(u,t)$ is the priority boost factor for user $u$ at time $t$.

As mentioned previously, the RUP calculation is designed so that at steady
state, each user's RUP stabilizes at the number of resources used by that user. 
The definition of $\beta$ ensures that the calculation of $\pi_r(u,t)$ can be 
calculated over non-uniform time intervals $\delta t$ without affecting the 
calculation.  The time interval $\delta t$ varies due to events internal to 
the system, but Condor guarantees that unless the Central Manager machine is 
down, no matches will be unaccounted for due to this variance.
