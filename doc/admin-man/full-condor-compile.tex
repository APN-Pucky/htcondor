%%%%%%%%%%%%%%%%%%%%%%%%%%%%%%%%%%%%%%%%%%%%%%%%%%%%%%%%%%%%%%%%%%%%%%%%%%%
\subsection{\label{sec:full-condor-compile}Full Installation of
\condor{compile}} 
%%%%%%%%%%%%%%%%%%%%%%%%%%%%%%%%%%%%%%%%%%%%%%%%%%%%%%%%%%%%%%%%%%%%%%%%%%%

In order to take advantage of two major HTCondor features: checkpointing
and remote system calls, users need to relink
their binaries.  Programs that are not relinked for HTCondor can run 
under
HTCondor's vanilla universe. However, these jobs cannot
take checkpoints and migrate.

To relink programs with HTCondor, we provide the 
\Condor{compile} tool.  As installed by default, \Condor{compile} works
with the following commands: \Prog{gcc}, \Prog{g++}, \Prog{g77},
\Prog{cc}, \Prog{acc}, \Prog{c89}, \Prog{CC}, \Prog{f77},
\Prog{fort77}, \Prog{ld}.  
%On Solaris and Digital Unix, \Prog{f90} is
%also supported.  
See the \Cmd{\condor{compile}}{1} man page for details on
using \Condor{compile}.

\Condor{compile} can work transparently with all
commands on the system, including \Prog{make}.  
The basic idea here is to replace the system linker (\Prog{ld}) with
the HTCondor linker.  Then, when a program is to be linked, the HTCondor
linker figures out whether this binary will be for HTCondor, or for a
normal binary.  If it is to be a normal compile, the old \Prog{ld} is
called.  If this binary is to be linked for HTCondor,
the script
performs the necessary operations in order to prepare a binary that
can be used with HTCondor.  In order to differentiate between normal
builds and HTCondor builds, the user simply places 
\Condor{compile} before their build command, which sets the
appropriate environment variable that lets the HTCondor linker script
know it needs to do its magic.

In order to perform this full installation of \Condor{compile}, the
following steps need to be taken:
	
\begin{enumerate}
	\item Rename the system linker from \Prog{ld} to \Prog{ld.real}.
	\item Copy the HTCondor linker to the location of the previous 
\Prog{ld}.
	\item Set the owner of the linker to \Login{root}.
	\item Set the permissions on the new linker to 755.
\end{enumerate}

The actual commands to execute depend upon the platform.
The location of the system linker (\Prog{ld}), is as follows:
\begin{verbatim}
	Operating System              Location of ld (ld-path)
	Linux                         /usr/bin
\end{verbatim}
%	Solaris 2.X                   /usr/ccs/bin
%	OSF/1 (Digital Unix)          /usr/lib/cmplrs/cc

On these platforms, issue the following commands (as \Login{root}), where
\Prog{ld-path} is replaced by the path to the system's \Prog{ld}.
\begin{verbatim}
  mv /[ld-path]/ld /<ld-path>/ld.real
  cp /usr/local/condor/lib/ld /<ld-path>/ld
  chown root /<ld-path>/ld
  chmod 755 /<ld-path>/ld
\end{verbatim}

If you remove HTCondor from your system later on, linking will continue
to work, since the HTCondor linker will always default to compiling
normal binaries and simply call the real \Prog{ld}.  In the interest of
simplicity, it is recommended that you reverse the above changes by
moving your \Prog{ld.real} linker back to its former position as \Prog{ld},
overwriting the HTCondor linker.

\Note If you ever upgrade your operating system after performing a
full installation of \Condor{compile}, you will probably have to re-do
all the steps outlined above.
Generally speaking, new versions or patches of an operating system
might replace the system \Prog{ld} binary, which would undo the
full installation of \Condor{compile}.

