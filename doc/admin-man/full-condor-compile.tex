%%%%%%%%%%%%%%%%%%%%%%%%%%%%%%%%%%%%%%%%%%%%%%%%%%%%%%%%%%%%%%%%%%%%%%%%%%%
\subsection{\label{sec:full-condor-compile}Full Installation of
\condor{compile}} 
%%%%%%%%%%%%%%%%%%%%%%%%%%%%%%%%%%%%%%%%%%%%%%%%%%%%%%%%%%%%%%%%%%%%%%%%%%%

In order to take advantage of two major Condor features: checkpointing
and remote system calls, users of the Condor system need to relink
their binaries.  Programs that are not relinked for Condor can run in
Condor's ``vanilla'' universe just fine, however, they cannot
checkpoint and migrate, or run on machines without a shared filesystem.

To relink your programs with Condor, we provide a special tool,
\Condor{compile}.  As installed by default, \Condor{compile} works
with the following commands: \Prog{gcc}, \Prog{g++}, \Prog{g77},
\Prog{cc}, \Prog{acc}, \Prog{c89}, \Prog{CC}, \Prog{f77},
\Prog{fort77}, \Prog{ld}.  On Solaris and Digital Unix, \Prog{f90} is
also supported.  See the \Cmd{\condor{compile}}{1} man page for details on
using \Condor{compile}.

However, you can make \Condor{compile} work transparently with all
commands on your system whatsoever, including \Prog{make}.  

The basic idea here is to replace the system linker (\Prog{ld}) with
the Condor linker.  Then, when a program is to be linked, the condor
linker figures out whether this binary will be for Condor, or for a
normal binary.  If it is to be a normal compile, the old \Prog{ld} is
called.  If this binary is to be linked for condor, the script
performs the necessary operations in order to prepare a binary that
can be used with condor.  In order to differentiate between normal
builds and condor builds, the user simply places 
\Condor{compile} before their build command, which sets the
appropriate environment variable that lets the condor linker script
know it needs to do its magic.

In order to perform this full installation of \Condor{compile}, the
following steps need to be taken:
	
\begin{enumerate}
	\item Rename the system linker from ld to ld.real.
	\item Copy the condor linker to the location of the previous ld.
	\item Set the owner of the linker to root.
	\item Set the permissions on the new linker to 755.
\end{enumerate}

The actual commands that you must execute depend upon the system that you
are on.  The location of the system linker (\Prog{ld}), is as follows:
\begin{verbatim}
	Operating System              Location of ld (ld-path)
	Linux                         /usr/bin
	Solaris 2.X                   /usr/ccs/bin
	OSF/1 (Digital Unix)          /usr/lib/cmplrs/cc
\end{verbatim}

On these platforms, issue the following commands (as root), where
\Prog{ld-path} is replaced by the path to your system's \Prog{ld}.
\begin{verbatim}
        mv /[ld-path]/ld /[ld-path]/ld.real
        cp /usr/local/condor/lib/ld /[ld-path]/ld
        chown root /[ld-path]/ld
        chmod 755 /[ld-path]/ld
\end{verbatim}

If you remove Condor from your system latter on, linking will continue
to work, since the condor linker will always default to compiling
normal binaries and simply call the real ld.  In the interest of
simplicity, it is recommended that you reverse the above changes by
moving your ld.real linker back to it's former position as ld,
overwriting the condor linker.

\Note If you ever upgrade your operating system after performing a
full installation of \Condor{compile}, you will probably have to re-do
all the steps outlined above.
Generally speaking, new versions or patches of an operating system
might replace the system ld binary, which would undo the
full installation of \Condor{compile}.

