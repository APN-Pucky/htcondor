%%%%%%%%%%%%%%%%%%%%%%%%%%%%%%%%%%%%%%%%%%%%%%%%%%%%%%%%%%%%%%%%%%%%%%
\subsection{\label{sec:Absent-Ads}Absent Ads}
%%%%%%%%%%%%%%%%%%%%%%%%%%%%%%%%%%%%%%%%%%%%%%%%%%%%%%%%%%%%%%%%%%%%%%
\index{pool management!absent ads}

By default, Condor assumes that resources are transient: the collector 
will discard ads older than CLASSAD\_LIFETIME seconds.  (The default 
configuration is 15 minutes, so that the default UPDATE\_INTERVAL will 
pass three times before Condor forgets about a resource.)  In some 
pools, especially those with dedicated resources, this approach may make 
it unnecessarily difficult to determine what the composition of the pool 
``should be'', in the sense of which machines would be in the pool if 
Condor were properly functioning on all of them.

This assumption can be corrected by use of absent ads.  When a machine 
ad would otherwise expire (that is, CLASSAD\_LIFETIME seconds have 
passed), the collector evaluates ABSENT\_REQUIREMENTS against the ad.  If 
true, the ad will be saved persistently and marked as absent; this will 
cause it to appear in the output of 'condor\_status -absent'.  When the 
machine returns to the pool, its first update to the collector will 
invalidate the absent ad.

Absent ads, like offline ads, are stored to disk to ensure that they 
aren't forgotten even across collector crashes.  The configuration knob 
PERSISTENT\_AD\_LOG defines the file in which the ads are stored, and 
deprecates the older OFFLINE\_LOG.  Absent ads are retained on disk by 
the collector for ABSENT\_EXPIRE\_ADS\_AFTER seconds, for which a value of 
0 means forever, and defaults to thirty days.

Absent ads are only returned by the collector when using the -collector 
option to condor\_status, or when the Absent machine-ad attribute is 
mentioned on its command-line.  This renders absent ads invisible to the 
rest of the Condor infrastructure.
