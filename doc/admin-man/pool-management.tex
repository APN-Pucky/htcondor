%%%%%%%%%%%%%%%%%%%%%%%%%%%%%%%%%%%%%%%%%%%%%%%%%%%%%%%%%%%%%%%%%%%%%%
\section{\label{sec:Pool-Management}Pool Management}
%%%%%%%%%%%%%%%%%%%%%%%%%%%%%%%%%%%%%%%%%%%%%%%%%%%%%%%%%%%%%%%%%%%%%%
\index{pool management}

Condor provides administrative tools to help with
pool management.
This section describes some of these tasks.

All of the commands described in this section are subject to the
security policy chosen for the Condor pool.
As such, the commands must be either run from a
machine that has the proper authorization, 
or run by a user that is authorized to issue the commands.
Section~\ref{sec:Security} on
page~\pageref{sec:Security} details the implementation of 
security in Condor.

%%%%%%%%%%%%%%%%%%%%%%%%%%%%%%%%%%%%%%%%%%%%%%%%%%%%%%%%%%%%%%%%%%%%%%
\subsection{\label{sec:Pool-Upgrade}
Upgrading -- Installing a New Version on an Existing Pool}
%%%%%%%%%%%%%%%%%%%%%%%%%%%%%%%%%%%%%%%%%%%%%%%%%%%%%%%%%%%%%%%%%%%%%%
\index{pool management!installing a new version on an existing pool}
\index{installation!installing a new version on an existing pool}

An upgrade changes the running version of Condor
from the current installation to a newer version.
The safe method
to install and start running a newer version of Condor
in essence is:
shut down the current installation of Condor,
install the newer version,
and then restart Condor using the newer version.
To allow for falling back to the current version,
place the new version in a separate directory.
Copy the existing configuration files,
and modify the copy to point to and use the new version,
as well as incorporate any configuration variables that are new or changed
in the new version.
Set the \Env{CONDOR\_CONFIG} environment variable
to point to the new copy of the configuration,
so the new version of Condor will use the new configuration when restarted.

When upgrading from a version of Condor earlier than 6.8 to more recent version,
note that the configuration settings must be modified for security reasons.
Specifically, the \Macro{HOSTALLOW\_WRITE} configuration variable
must be explicitly changed,
or no jobs may be submitted,
and error messages will be issued by Condor tools.

Another way to upgrade leaves Condor running. 
Condor will automatically restart itself if the \Condor{master} binary
is updated,
and this method takes advantage of this. 
Download the newer version, placing it such that it does not 
overwrite the currently running version.
With the download will be a new set of configuration files;
update this new set with any specializations implemented in the currently
running version of Condor.
Then, modify the currently running installation by changing its
configuration such that the path to binaries points instead
to the new binaries.
One way to do that (under Unix) is to use a symbolic link that points 
to the current Condor installation directory (for example, \File{/opt/condor}).
Change the symbolic link to point to the new directory. 
If Condor is configured to locate its binaries via the symbolic link, 
then after the symbolic link changes, 
the \Condor{master} daemon notices the new binaries and restarts itself. 
How frequently it checks is controlled by the configuration variable 
\Macro{MASTER\_CHECK\_NEW\_EXEC\_INTERVAL}, which defaults 5 minutes.

When the \Condor{master} notices new binaries, 
it begins a graceful restart. 
On an execute machine, 
a graceful restart means that running jobs are preempted. 
Standard universe jobs will attempt to take a checkpoint. 
This could be a bottleneck if all machines in a large pool 
attempt to do this at the same time. 
If they do not complete within the cutoff time specified by the \MacroNI{KILL} 
policy expression (defaults to 10 minutes), 
then the jobs are killed without producing a checkpoint. 
It may be appropriate to increase this cutoff time, 
and a better approach may be to upgrade the pool in stages 
rather than all at once. 

For universes other than the standard universe, jobs are preempted. 
If jobs have been guaranteed a certain amount of uninterrupted run time 
with \MacroNI{MaxJobRetirementTime}, 
then the job is not killed until the specified amount of 
retirement time has been exceeded (which is 0 by default). 
The first step of killing the job is a soft kill signal, 
which can be intercepted by the job so that it can exit gracefully, 
perhaps saving its state. 
If the job has not gone away once the \MacroNI{KILL} expression fires 
(10 minutes by default), 
then the job is forcibly hard-killed. 
Since the graceful shutdown of jobs may rely on shared resources such as disks 
where state is saved, 
the same reasoning applies as for the standard universe: 
it may be appropriate to increase the cutoff time 
for large pools, 
and a better approach may be to upgrade the pool in stages 
to avoid jobs running out of time. 

Another time limit to be aware of is the configuration variable 
\MacroNI{SHUTDOWN\_GRACEFUL\_TIMEOUT}. 
This defaults to 30 minutes. 
If the graceful restart is not completed within this time, 
a fast restart ensues. 
This causes jobs to be hard-killed. 

%%%%%%%%%%%%%%%%%%%%%%%%%%%%%%%%%%%%%%%%%%%%%%%%%%%%%%%%%%%%%%%%%%%%%%
\subsection{\label{sec:Pool-Shutdown-and-Restart}
Shutting Down and Restarting a Condor Pool}
%%%%%%%%%%%%%%%%%%%%%%%%%%%%%%%%%%%%%%%%%%%%%%%%%%%%%%%%%%%%%%%%%%%%%%
\index{pool management!shutting down Condor}
\index{pool management!restarting Condor}

\begin{description}
\item[Shutting Down Condor]
There are a variety of ways to shut down all or parts of a Condor pool.
All utilize the \Condor{off} tool.

To stop a single execute machine from running jobs,
the \Condor{off} command specifies the machine by host name.
\begin{verbatim}
  condor_off -startd <hostname>
\end{verbatim}
A running \SubmitCmd{standard} universe job will be allowed to 
take a checkpoint before the job is killed.
A running job under another universe will be killed.
If it is instead desired that the machine stops running jobs
only after the currently executing job completes, the command is
\begin{verbatim}
  condor_off -startd -peaceful <hostname>
\end{verbatim}
Note that this waits indefinitely for the running job to finish,
before the \Condor{startd} daemon exits.

Th shut down all execution machines within the pool,
\begin{verbatim}
  condor_off -all -startd
\end{verbatim}

To wait indefinitely for each machine in the pool to finish its current
Condor job,
shutting down all of the execute machines as they no longer
have a running job,
\begin{verbatim}
  condor_off -all -startd -peaceful
\end{verbatim}

To shut down Condor on a machine from which jobs are submitted,
\begin{verbatim}
  condor_off -schedd <hostname>
\end{verbatim}

If it is instead desired that the submit machine shuts down
only after all jobs that are currently in the queue are finished,
first disable new submissions to the queue 
by setting the configuration variable
\begin{verbatim}
  MAX_JOBS_SUBMITTED = 0
\end{verbatim}
See instructions below in section~\ref{sec:Reconfigure-Pool} for how
to reconfigure a pool.
After the reconfiguration, the command to wait for all jobs to complete
and shut down the submission of jobs is
\begin{verbatim}
  condor_off -schedd -peaceful <hostname>
\end{verbatim}

Substitute the option \Opt{-all} for the host name,
if all submit machines in the pool are to be shut down.

\item[Restarting Condor, If Condor Daemons Are Not Running]
If Condor is not running,
perhaps because one of the \Condor{off} commands was used,
then starting Condor daemons back up depends on which part of
Condor is currently not running.

If no Condor daemons are running, then starting Condor is a matter
of executing the \Condor{master} daemon.
The \Condor{master} daemon will then invoke all other specified daemons
on that machine.
The \Condor{master} daemon executes on every machine that is to run Condor.

If a specific daemon needs to be started up, and the \Condor{master} daemon
is already running, then issue the command on the specific machine with
\begin{verbatim}
  condor_on -subsystem <subsystemname>
\end{verbatim}
where \verb@<subsystemname>@ is replaced by the daemon's subsystem
name.
Or, this command might be issued from another machine in the pool
(which has administrative authority) with
\begin{verbatim}
  condor_on <hostname> -subsystem <subsystemname>
\end{verbatim}
where \verb@<subsystemname>@ is replaced by the daemon's subsystem
name, and \verb@<hostname>@ is replaced by the host name of the
machine where this \Condor{on} command is to be directed.

\item[Restarting Condor, If Condor Daemons Are Running]
If Condor daemons are currently running, but need to be killed and
newly invoked,
the \Condor{restart} tool does this.
This would be the case for a new value of a configuration variable for
which using \Condor{reconfig} is inadequate.

To restart all daemons on all machines in the pool,
\begin{verbatim}
  condor_restart -all
\end{verbatim}

To restart all daemons on a single machine in the pool,
\begin{verbatim}
  condor_restart <hostname>
\end{verbatim}
where \verb@<hostname>@ is replaced by the host name of the
machine to be restarted.

\end{description}

%%%%%%%%%%%%%%%%%%%%%%%%%%%%%%%%%%%%%%%%%%%%%%%%%%%%%%%%%%%%%%%%%%%%%%
\subsection{\label{sec:Reconfigure-Pool}Reconfiguring a Condor Pool}
%%%%%%%%%%%%%%%%%%%%%%%%%%%%%%%%%%%%%%%%%%%%%%%%%%%%%%%%%%%%%%%%%%%%%%
\index{pool management!reconfiguration}

To change a global configuration variable and have all the
machines start to use the new setting, change the value within the file,
and send a \Condor{reconfig} command to each host.
Do this with a \emph{single} command,
\begin{verbatim}
  condor_reconfig -all
\end{verbatim}

If the global configuration file is not shared among all the machines,
as it will be if using a shared file system,
the change must be made to each copy of the global configuration file
before issuing the \Condor{reconfig} command.

Issuing a \Condor{reconfig} command is inadequate for some
configuration variables.
For those, a restart of Condor is required.
Those configuration variables that require a restart are listed in
section~\ref{sec:Macros-Requiring-Restart}
on page~\pageref{sec:Macros-Requiring-Restart}.
The manual page for \Condor{restart} is at
~\ref{man-condor-restart}.

%%%%%%%%%%%%%%%%%%%%%%%%%%%%%%%%%%%%%%%%%%%%%%%%%%%%%%%%%%%%%%%%%%%%%%
\subsection{\label{sec:Absent-Ads}Absent ClassAds}
%%%%%%%%%%%%%%%%%%%%%%%%%%%%%%%%%%%%%%%%%%%%%%%%%%%%%%%%%%%%%%%%%%%%%%
\index{pool management!absent ClassAds}
\index{absent ClassAd}
\index{ClassAd!absent ClassAd}

By default, HTCondor assumes that resources are transient: 
the \Condor{collector}
will discard ClassAds older than \Macro{CLASSAD\_LIFETIME} seconds.
Its default configuration value is 15 minutes, 
and as such, the default value for \Macro{UPDATE\_INTERVAL} will 
pass three times before HTCondor forgets about a resource.
In some pools, especially those with dedicated resources, 
this approach may make it unnecessarily difficult to determine 
what the composition of the pool ought to be, 
in the sense of knowing which machines would be in the pool,
if HTCondor were properly functioning on all of them.

This assumption of transient machines can be modified by 
the use of absent ClassAds.  
When a machine ClassAd would otherwise expire, 
the \Condor{collector} evaluates the configuration variable
\Macro{ABSENT\_REQUIREMENTS} against the machine ClassAd.
If \Expr{True}, 
the machine ClassAd will be saved in a persistent manner and 
be marked as absent;
this causes the machine to appear in the output of 
\Expr{condor\_status -absent}.
When the machine returns to the pool, 
its first update to the \Condor{collector} will 
invalidate the absent machine ClassAd.

Absent ClassAds, like offline ClassAds, 
are stored to disk to ensure that they are remembered,
even across \Condor{collector} crashes.
The configuration variable \Macro{PERSISTENT\_AD\_LOG}
defines the file in which the ClassAds are stored,
and replaces the no longer used variable \MacroNI{OFFLINE\_LOG}.
Absent ClassAds are retained on disk as maintained by 
the \Condor{collector} for a length of time in seconds defined by the
configuration variable \Macro{ABSENT\_EXPIRE\_ADS\_AFTER}.
A value of 0 for this variable means that the ClassAds are never discarded,
and the default value is thirty days.

Absent ClassAds are only returned by the \Condor{collector} when 
using the \Opt{-collector} option to \Condor{status},
or when the absent machine ClassAd attribute is mentioned on 
the \Condor{status} command line.  
This renders absent ClassAds invisible to the rest of the HTCondor infrastructure.

