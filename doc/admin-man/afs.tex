%%%%%%%%%%%%%%%%%%%%%%%%%%%%%%%%%%%%%%%%%%%%%%%%%%%%%%%%%%%%%%%%%%%%%%%%%%%
\subsection{\label{sec:Condor-AFS}Using Condor with AFS}
%%%%%%%%%%%%%%%%%%%%%%%%%%%%%%%%%%%%%%%%%%%%%%%%%%%%%%%%%%%%%%%%%%%%%%%%%%%

If you are using AFS at your site, be sure to read
section~\ref{sec:Shared-Filesystem-Config-File-Entries} on ``Shared
Filesystem Config Files Entries'' for details on configuring your
machines to interact with and use shared filesystems, AFS in
particular.

Condor does not currently have a way to authenticate itself to AFS.
This is true of the Condor daemons that would like to authenticate as
AFS user Condor, and the \Condor{shadow}, which would like to
authenticate as the user who submitted the job it is serving.  Since
neither of these things can happen yet, there are a number of special
things people who use AFS with Condor must do.  Some of this must be
done by the administrator(s) installing Condor.  Some of this must be
done by Condor users who submit jobs.

%%%%%%%%%%%%%%%%%%%%%%%%%%%%%%%%%%%%%%%%%%%%%%%%%%%%%%%%%%%%%%%%%%%%%%%%%%%
\subsubsection{\label{sec:Condor-AFS-Admin}AFS and Condor for Administrators}
%%%%%%%%%%%%%%%%%%%%%%%%%%%%%%%%%%%%%%%%%%%%%%%%%%%%%%%%%%%%%%%%%%%%%%%%%%%

The most important thing is that since the Condor daemons can't
authenticate to AFS, the \MacroNI{LOCAL\_DIR} (and it's subdirectories
like ``log'' and ``spool'') for each machine must be either writable
to unauthenticated users, or must not be on AFS.  The first option is
a \textbf{VERY} bad security hole so you should \textbf{NOT} have your
local directory on AFS.  If you've got NFS installed as well and want
to have your \MacroNI{LOCAL\_DIR} for each machine on a shared file
system, use NFS.  Otherwise, you should put the \MacroNI{LOCAL\_DIR} on
a local partition on each machine in your pool.  This means that you
should run \Condor{install} to install your release directory and
configure your pool, setting the \MacroNI{LOCAL\_DIR} parameter to some
local partition.  When that's complete, log into each machine in your
pool and run \Condor{init} to set up the local Condor directory.

The \MacroNI{RELEASE\_DIR}, which holds all the Condor binaries,
libraries and scripts can and probably should be on AFS.  None of the
Condor daemons need to write to these files, they just need to read
them.  So, you just have to make your \MacroNI{RELEASE\_DIR} world
readable and Condor will work just fine.  This makes it easier to
upgrade your binaries at a later date, which means that your users can find
the Condor tools in a consistent location on all the machines in your
pool, and that you can have the Condor config files in a centralized
location.  This is what we do at UW-Madison's CS department Condor
pool and it works quite well.

Finally, you might want to setup some special AFS groups to help your
users deal with Condor and AFS better (you'll want to read the section
below anyway, since you're probably going to have to explain this
stuff to your users).  Basically, if you can, create an AFS group that
contains all unauthenticated users but that is restricted to a given
host or subnet.  You're supposed to be able to make these host-based
ACLs with AFS, but we've had some trouble getting that working here at
UW-Madison.  What we have instead is a special group for all machines
in our department.  So, the users here just have to make their output
directories on AFS writable to any process running on any of our
machines, instead of any process on any machine with AFS on the
Internet.

%%%%%%%%%%%%%%%%%%%%%%%%%%%%%%%%%%%%%%%%%%%%%%%%%%%%%%%%%%%%%%%%%%%%%%%%%%%
\subsubsection{\label{sec:Condor-AFS-Users}AFS and Condor for Users}
%%%%%%%%%%%%%%%%%%%%%%%%%%%%%%%%%%%%%%%%%%%%%%%%%%%%%%%%%%%%%%%%%%%%%%%%%%%

The \Condor{shadow} process runs on the machine where you submitted
your Condor jobs and performs all file system access for your jobs.
Because this process isn't authenticated to AFS as the user who
submitted the job, it will not normally be able to write any output.
So, when you submit jobs, any directories where your job will be
creating output files will need to be world writable (to
non-authenticated AFS users).  In addition, if your program writes to
\File{stdout} or \File{stderr}, or you're using a user log for your
jobs, those files will need to be in a directory that's
world-writable.

Any input for your job, either the file you specify as input in your
submit file, or any files your program opens explicitly, needs to be
world-readable.

Some sites may have special AFS groups set up that can make this
unauthenticated access to your files less scary.  For example, there's
supposed to be a way with AFS to grant access to any unauthenticated
process on a given host.  That way, you only have to grant write
access to unauthenticated processes on your submit machine, instead of
any unauthenticated process on the Internet.  Similarly,
unauthenticated read access could be granted only to processes running
on your submit machine.  Ask your AFS administrators about the existence
of such AFS groups and details of how to use them.

The other solution to this problem is to just not use AFS at all.  If
you have disk space on your submit machine in a partition that is not
on AFS, you can submit your jobs from there.  While the \Condor{shadow}
is not authenticated to AFS, it does run with the effective UID of the
user who submitted the jobs.  So, on a local (or NFS) file system, the
\Condor{shadow} will be able to access your files normally, and you
won't have to grant any special permissions to anyone other than
yourself.  If the Condor daemons are not started as root however, the
shadow will not be able to run with your effective UID, and you'll
have a similar problem as you would with files on AFS.  See the
section on ``Running Condor as Non-Root'' for details.
    

