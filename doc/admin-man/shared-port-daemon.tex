%%%%%%%%%%%%%%%%%%%%%%%%%%%%%%%%%%%%%%%%%%%%%%%%%%%%%%%%%%%%%%%%%%%%%%%%%%%
\subsection{\label{sec:shared-port-daemon}Reducing Port Usage with the \Condor{shared\_port} Daemon}
%%%%%%%%%%%%%%%%%%%%%%%%%%%%%%%%%%%%%%%%%%%%%%%%%%%%%%%%%%%%%%%%%%%%%%%%%%%

\index{HTCondor daemon!condor\_shared\_port@\Condor{shared\_port}}
\index{daemon!condor\_shared\_port@\Condor{shared\_port}}
\index{condor\_shared\_port daemon}
The \Condor{shared\_port} is an optional daemon
responsible for creating a TCP listener port shared by all of the
HTCondor daemons for which the configuration variable
\Macro{USE\_SHARED\_PORT} is \Expr{True}.
The \Condor{master} will invoke the \Condor{shared\_port} daemon
if it is listed in \MacroNI{DAEMON\_LIST}.

The main purpose of the \Condor{shared\_port} daemon is to reduce the
number of ports that must be opened when HTCondor needs to be
accessible through a firewall.
This has a greater security benefit
than simply reducing the number of open ports.
Without the \Condor{shared\_port} daemon,
one can configure HTCondor to use a range of ports,
but since some HTCondor daemons are created dynamically, 
this full range of ports will not be in use by HTCondor at all times.
This implies that other non-HTCondor processes not intended to be exposed to
the outside network could unintentionally bind to ports in the range
intended for HTCondor,
unless additional steps are taken to control access to those ports.  
While the \Condor{shared\_port} daemon is running,
it is exclusively bound to its port, which means that other non-HTCondor
processes cannot accidentally bind to that port.

A secondary benefit of the \Condor{shared\_port} daemon
is that it helps address the scalability issues of a submit machine.
Without the \Condor{shared\_port} daemon,
approximately 2.1 ephemeral ports per running job are required,
and possibly more, depending on the rate of job completion.
There are only 64K ports in total,
and most standard Unix installations only allocate a subset of
these as ephemeral ports.
In practice, with long running jobs,
and with between 11K and 14K simultaneously running jobs,
port exhaustion has been observed in typical Linux installations.
After increasing the ephemeral port range as to as many as possible,
port exhaustion occurred between 20K and 25K running jobs.
Using the \Condor{shared\_port} daemon,
each running job requires fewer, approximately 1.1 ephemeral ports
on the submit node, if HTCondor on the submit node connects directly
to HTCondor on the execute node.
If the submit node connects via CCB to the execute
node, \emph{no} ports are required per running job; only the one port
allocated to the \Condor{shared\_port} daemon is used.

When CCB is utilized via setting the configuration variable
\Macro{CCB\_ADDRESS},
the \Condor{shared\_port} daemon registers with
the CCB server on behalf of all daemons sharing the port.
This means that it is not possible to individually enable or disable
CCB connectivity to daemons that are using the shared port;
they all effectively share the same setting,
and the \Condor{shared\_port} daemon handles all CCB connection
requests on their behalf.

HTCondor's authentication and authorization steps are unchanged by the
use of a shared port.  Each HTCondor daemon continues to operate
according to its configured policy.  Requests for connections to the
shared port are not authenticated or restricted by
the \Condor{shared\_port} daemon.
They are simply passed to the requested daemon,
which is then responsible for enforcing the security policy.

When the \Condor{master} is configured to use the shared port
by setting the configuration variable
\begin{verbatim}
  USE_SHARED_PORT = True
\end{verbatim}
the \Condor{shared\_port} daemon is treated specially. 
A command such as \Condor{off},
which shuts down all daemons except for the \Condor{master},
will also leave the \Condor{shared\_port} running.
This prevents the \Condor{master} from getting into a state
where it can no longer receive commands.

The \Condor{collector} daemon typically has its own port;
it uses 9618 by default.
However, it can be configured to use a shared port.
Since the address of the \Condor{collector} must be set in 
the configuration file,
it is necessary to specify the shared port socket name of 
the \Condor{collector},
so that connections to the shared port that are intended for 
the \Condor{collector} can be forwarded to it.
If the shared port number is 11000, a \Condor{collector} address using this
shared port could be configured:

\footnotesize
\begin{verbatim}
COLLECTOR_HOST = collector.host.name:11000?sock=collector
\end{verbatim}
\normalsize

This configuration assumes that the socket name used by 
the \Condor{collector} is \Expr{collector}.
The \Condor{collector} that runs on \Expr{collector.host.name}
will automatically choose this socket name if \MacroNI{COLLECTOR\_HOST}
is configured as in the example above.
If multiple \Condor{collector} daemons are started on the same
machine, the socket name can be explicitly set in the daemon arguments,
as in the example:

\begin{verbatim}
COLLECTOR_ARGS = -sock collector
\end{verbatim}

When the \Condor{collector} address is a shared port,
TCP updates will be automatically used instead of UDP.
Under Unix, this means that the
\Condor{collector} daemon should be configured to have enough file descriptors.
See section~\ref{sec:tcp-collector-update} for more information on using
TCP within HTCondor.

SOAP commands cannot be sent over a shared port.
However, a daemon may be configured to open a fixed, non-shared port,
in addition to using a shared port.
This is done both by setting
\Expr{USE\_SHARED\_PORT = True} and by specifying a fixed port for the daemon
using \verb|<SUBSYS>_ARGS = -p <portnum>|.

The TCP connections required to manage standard universe jobs do not
make use of shared ports.

