%%%%%%%%%%%%%%%%%%%%%%%%%%%%%%%%%%%%%%%%%%%%%%%%%%%%%%%%%%%%%%%%%%%%%%%%%%%
\subsection{\label{sec:shared-port-daemon}Reducing Port Usage with the \Condor{shared\_port} Daemon}
%%%%%%%%%%%%%%%%%%%%%%%%%%%%%%%%%%%%%%%%%%%%%%%%%%%%%%%%%%%%%%%%%%%%%%%%%%%

\index{Condor daemon!condor\_shared\_port@\Condor{shared\_port}}
\index{daemon!condor\_shared\_port@\Condor{shared\_port}}
\index{condor\_shared\_port daemon}
\index{configuration!condor\_shared\_port configuration variables}
\Condor{shared\_port} is an optional daemon that may be added to the
\Condor{master} daemon's \MacroNI{DAEMON\_LIST} on platforms that
support it (currently all Unix systems except for HPUX).  It is
responsible for creating a TCP listener port shared by all of the
Condor daemons for which \Macro{USE\_SHARED\_PORT} is true.

The main purpose of \Condor{shared\_port} is to reduce the
 number of ports that must be opened when condor needs to be
 accessible through a firewall.  This has a greater security benefit
 than simply reducing the number of open ports.  Without
 \Condor{shared\_port}, one can configure Condor to use a range of
 ports, but since some Condor daemons are created dynamically, this
 full range of ports will not be in use by Condor at all times.  This
 implies that other non-Condor processes not intended to be exposed to
 the outside network could unintentionally bind to ports in the range
 intended for Condor unless additional steps are taken to control
 access to those ports.  While \Condor{shared\_port} is running, it is
 exclusively bound to its port, which means that other non-Condor
 processes cannot accidentally bind to that port.

A secondary benefit of \Condor{shared\_port} is that it helps address
 scalability on the submit node.  Without \Condor{shared\_port}, about
 2.1 ephemeral ports per running job are required, possibly more
 depending on the rate of job completion.  There are only 64k ports in
 total and most standard Unix installations only allocate a subset of
 these as ephemeral ports.  In practice, with long running jobs, we
 have observed port exhaustion in typical Linux installations between
 11k and 14k simultaneously running jobs.  After increasing the
 ephemeral port range as much as possible, port exhaustion happened
 between 20k and 25k running jobs.  With \Condor{shared\_port}, each
 running job requires only about 1.1 ephemeral ports on the submit
 node if Condor on the submit node connects directly to Condor on the
 execute node.  If the submit node connects via CCB to the execute
 node, \emph{no} ports are required per running job; only the one port
 allocated to \Condor{shared\_port} is used.

When \Macro{CCB\_ADDRESS} is set, \Condor{shared\_port} registers with
 the CCB server on behalf of all daemons sharing the port.  This means
 it is not possible to individually enable or disable CCB connectivity
 to daemons that are using the shared port; they all effectively share
 the same setting and \Condor{shared\_port} handles all CCB connection
 requests on their behalf.

Condor's authentication and authorization steps are unchanged by the
 use of a shared port.  Each Condor daemon continues to operate
 according to its configured policy.  Requests for connections to the
 shared port are not authenticated or restricted by
 \Condor{shared\_port}.  They are simply passed to the requested daemon,
 which is then responsible for enforcing the security policy.

When \Condor{master} is configured to use the shared port
 (\Expr{USE\_SHARED\_PORT=True}), \Condor{shared\_port} is treated
 specially.  A command such as \Condor{off} which shuts down all
 daemons except for the master will also leave \Condor{shared\_port}
 running.  This prevents \Condor{master} from getting into a state
 where it can no longer receive commands.

The \Condor{collector} daemon typically has its own port (9618 by default).
However, it can be configured to use a shared port.  Since the address of
the collector must be set in the configuration file, it is necessary to
specify the shared port socket name of the collector so that connections
to the shared port that are intended for the collector can be forwarded
to it.  If the shared port number is 11000, a collector address using this
shared port could be configured as follows:

\begin{verbatim}
COLLECTOR_HOST = collector.host.name:11000?sock=collector
\end{verbatim}

This assumes that the socket name used by \Condor{collector} is `collector'.
The collector that runs on `collector.host.name' will automatically choose
this socket name.  If multiple collectors are being started on the same
machine, the socket name can be explicitly set in the daemon arguments:

\begin{verbatim}
COLLECTOR_ARGS = -sock collector
\end{verbatim}

When the collector address is a shared port, TCP updates will be
 automatically used instead of UDP.  Under Unix, this means that the
 \Condor{collector} daemon should be configured to have enough file
 descriptors.  See section~\ref{sec:tcp-collector-update} for more
 information.

SOAP commands cannot be sent over a shared port.  However, a daemon
 may be configured to open a fixed non-shared port in addition to
 using a shared port.  This is done by setting both
 \Expr{USE\_SHARED\_PORT=True} and specifying a fixed port for the daemon
 using \verb|<SUBSYS>_ARGS = -p <portnum>|.

The TCP connections required to manage standard universe jobs do not
 make use of shared ports.

