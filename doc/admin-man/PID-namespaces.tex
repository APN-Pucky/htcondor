%%%%%%%%%%%%%%%%%%%%%%%%%%%%%%%%%%%%%%%%%%%%%%%%%%%%%%%%%%%%%%%%%%%%%%%%%%%
\subsection{\label{sec:PIDNamespaces}Per Job PID Namespaces} 
%%%%%%%%%%%%%%%%%%%%%%%%%%%%%%%%%%%%%%%%%%%%%%%%%%%%%%%%%%%%%%%%%%%%%%%%%%%
\index{PID namespaces!per job}
\index{namespaces!per job PID namespaces}
\index{Linux kernel!per job PID namespaces}

Per job PID namespaces provide enhanced security of one process
tree from another through the isolation implemented by the namespace.
HTCondor may enable the use of per job PID namespaces for 
Linux version ? and more recent kernels.

Read about per job PID namespaces \URL{here}.

The needed isolation of jobs from the same user that execute on the
same machine as each other is already provided by the implementation
of slot users as described in section~\ref{sec:RunAsNobody}.
This is the recommended way to implement the prevention of interference
between more than one job submitted by a single user.
However, the use of a shared file system by slot users presents
issues in the ownership of files written by the jobs.

The per job PID namespace provides a way to handle the ownership
of files produced by jobs within a shared file system.
It also isolates the processes of a job within its PID namespace.
As a side effect and benefit, the clean up of processes for a job
within a PID namespace is enhanced. 
When the process with PID = 1 is killed, 
the operating system takes care of killing all child processes.

To enable the use of per job PID namespaces, 
set the configuration to include

\begin{verbatim}
  USE_PID_NAMESPACES = True
\end{verbatim}

This configuration variable defaults to \Expr{False},
thus the use of per job PID namespaces is disabled by default.
