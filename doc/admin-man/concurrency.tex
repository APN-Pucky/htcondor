%%%%%%%%%%%%%%%%%%%%%%%%%%%%%%%%%%%%%%%%%%%%%%%%%%%%%%%%%%%%%%%%%%%%%%%%%%%
\subsection{\label{sec:Concurrency-Limits}Concurrency Limits} 
%%%%%%%%%%%%%%%%%%%%%%%%%%%%%%%%%%%%%%%%%%%%%%%%%%%%%%%%%%%%%%%%%%%%%%%%%%%
\index{concurrency limits}

Condor's implementation of the mechanism called \Term{concurrency limits}
allows an administrator to define and set integer limits on
consumable resources.
These limits are utilized during matchmaking, preventing matches when
the resources are allocated.
Typical uses of this mechanism will include
the management of software licenses, database connections,
and any other consumable resource external to Condor.

Use of the concurrency limits mechanism requires configuration variables
to set distinct limits,
while jobs must identify the need for a specific resource.

In the configuration, a string must be chosen as a name for the
particular resource.
This name is used in the configuration of a \Condor{negotiator} daemon
variable that defines the concurrency limit, or integer quantity
available of this resource.
For example, assume that there are 3 licenses for the X software.
The configuration variable concurrency limit may be:
\begin{verbatim}
XSW_LIMIT = 3
\end{verbatim}
where \AdStr{XSW} is the invented name of this resource,
which is appended with the string \MacroNI{\_LIMIT}.
With this limit, a maximum of 3 jobs declaring that they need this
resource may be executed concurrently.

In addition to named limits, such as in the example named limit \MacroNI{XSW},
configuration may specify a concurrency limit for all resources
that are not covered by specifically-named limits.
The configuration variable \Macro{CONCURRENCY\_LIMIT\_DEFAULT} sets
this value.  For example,
\begin{verbatim}
CONCURRENCY_LIMIT_DEFAULT = 1
\end{verbatim}
sets a limit of 1 job in execution for any job that declares its
requirement for a resource that is not named in the configuration.
If \MacroNI{CONCURRENCY\_LIMIT\_DEFAULT} is omitted  from the
configuration, then no limits are placed on the number of 
concurrently executing jobs of resources for which there is no
specifically named concurrency limit. 

The job must declare its need for a resource by placing a command
in its submit description file or adding an attribute to the
job ClassAd.
In the submit description file, an example job that requires
the X software adds:
\begin{verbatim}
concurrency_limits = XSW
\end{verbatim}
This results in the job ClassAd attribute
\begin{verbatim}
ConcurrencyLimits = "XSW"
\end{verbatim}

The implementation of the job ClassAd attribute \Attr{ConcurrencyLimits}
has a more general implementation.
It is either a string or a string list.
A list contains items delimited by space characters and comma characters.
Therefore, a job that requires the 3 separate resources 
named as  \AdStr{XSW}, \AdStr{y}, and  \AdStr{Z}, 
will contain in its submit description file:
\begin{verbatim}
concurrency_limits = y,XSW,Z
\end{verbatim}

Note that the maximum for any given limit,
as specified with the configuration variable \MacroNI{<*>\_LIMIT},
is as strictly enforced \Bold{as possible}.
In the presence of preemption and dropped updates from
the \Condor{startd} daemon to the \Condor{collector} daemon,
it is possible for the limit to be exceeded.
Condor will never kill a job to free up a limit,
including the case where a limit maximum is exceeded. 
