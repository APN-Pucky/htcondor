%%%%%%%%%%%%%%%%%%%%%%%%%%%%%%%%%%%%%%%%%%%%%%%%%%%%%%%%%%%%%%%%%%%%%%
\subsection{\label{sec:PrivSep}Privilege Separation}
%%%%%%%%%%%%%%%%%%%%%%%%%%%%%%%%%%%%%%%%%%%%%%%%%%%%%%%%%%%%%%%%%%%%%%

Section~\ref{sec:uids} discusses why under most circumstances, it is
beneficial to run the Condor daemons as \Login{root}. In situations
where multiple users are involved or where Condor is responsible for
enforcing a machine owner's policy, running as \Login{root} is the 
\emph{only} way for Condor to do its job correctly and securely.

Unfortunately, this requirement of running Condor as \Login{root} is
at odds with a well-established goal of security-conscious
administrators: keeping the amount of software that runs with
superuser privileges to a reasonable minimum. Condor's nature as a
large distributed system that routinely communicates with potentially
untrusted components over the network further aggravates this goal.

The privilege separation (PrivSep) effort in Condor aims to minimize
the amount of code that needs \Login{root}-level access while still
giving Condor the tools it needs to work properly. Note that PrivSep
is currently only available for execute side functionality, and not
implemented on Windows.

In the PrivSep model, all logic in Condor that requires superuser
privilege is contained in a small component called the \emph{PrivSep
Kernel}. The Condor daemons, executing using an unprivileged account,
must explicitly request action from the PrivSep Kernel whenever
\Login{root}-level operations are needed.

To illustrate, imagine that an attacker has found an exploit in the
\Condor{startd} that allows for execution of arbitrary code on that
daemon's behalf. If the \Condor{startd} is running as \Login{root} on
all execute machines in a Condor pool, the attacker can achieve full
control over any machine where the exploit can be exercised.

If the \Condor{startd} were instead deployed to run in ``PrivSep
mode,'' the attacker could no longer take arbitrary action as
\Login{root}, but would have access the PrivSep Kernel. By placing
limits on what actions can performed via the PrivSep Kernel, however,
the attacker's sphere of influence can be contained. Refer ahead to
the ``PrivSep Kernel Interface'' section for a detailed description of
the services that the PrivSep Kernel provides to Condor, and how it
limits what \Login{root}-level actions are allowed. The following
section describes how to enable PrivSep for an execute-side Condor
installation.

%%%%%%%%%%%%%%%%%%%%%%%%%%%%%%%%%%%%%%%%%%%%%%%%%%%%%%%%%%%%%%%%%%%%%%
\subsubsection{PrivSep Configuration}
%%%%%%%%%%%%%%%%%%%%%%%%%%%%%%%%%%%%%%%%%%%%%%%%%%%%%%%%%%%%%%%%%%%%%%

The PrivSep Kernel is implemented as two programs:
\Condor{root\_switchboard} and \Condor{procd}. Both are contained in
the \File{sbin} directory of the Condor distribution. When Condor is
running in PrivSep mode, these will be the only two programs that run
with \Login{root} privilege.

Each of these binaries must be accessible on the file system via a
\emph{trusted path}. A trusted path ensures that no user (other than
\Login{root}) can alter what binary it refers to. To ensure that the
paths to these binaries are trusted, make sure that only
\Login{root}-owned directories are used, and that the permissions on
these directories deny write access to anyone but \Login{root}. The
binaries themselves must also be owned by \Login{root} and writable by
no one else. The \Condor{root\_switchboard} program additionally
should be installed with the setuid bit set. The following command
properly sets the permissions on the \Condor{root\_switchboard}
binary:
\begin{verbatim}
chmod 4755 /opt/condor/release/sbin/condor_root_switchboard
\end{verbatim}

The PrivSep Kernel has a configuration file that must be located at
\File{/etc/condor/privsep\_config}. The format of this file consists
of lines with ``\Code{key = value}'' pairs. Lines with only whitespace
or lines with ``\Code{\#}'' as the first non-whitespace character are
ignored.

Some configuration settings are interpreted as single values, while
others are interpreted as lists. To populate a list setting with
multiple values, use multiple configuration file lines with the same
key. For example, the following configures the \Code{valid-dirs}
setting as a list with two entries:
\begin{verbatim}
valid-dirs = /opt/condor/execute_1
valid-dirs = /opt/condor/execute_2
\end{verbatim}
It is an error to have multiple lines with the same key for a setting
that is not interpreted as a list.

Some settings more specifically require a list of UIDs or GIDs, and
these allow for a more specialized syntax. User and group IDs can be
specified either numerically or textually. In addition, list entries
can be given that specify a range of IDs using a ``\Code{-}''
character to separate the minimum and maximum IDs included. The
``\Code{*}'' character can be given on the right-hand side of such a
range to indicate that the range extends to the maximum possible
ID. Finally, multiple list entries may be given on a single
configuration file line using the \Code{:} character as a
delimiter. The following example shows how this specialized syntax can
be used to build up a complex list of IDs.
\begin{verbatim}
valid-target-uids = nobody : nfsuser1 : nfsuser2
valid-target-uids = condor_run_1 - condor_run_8
valid-target-uids = 800 - *
\end{verbatim}

The settings used to configure the PrivSep Kernel are listed
below. All settings are required.

\begin{itemize}

\item \Code{valid-caller-uids} and \Code{valid-caller-gids}. These
specify users and groups that will be allowed to request action from
the PrivSep Kernel. These typically will just be set to contain the
UID and primary GID that the Condor daemons will run as.

\item \Code{valid-target-uids} and \Code{valid-target-gids}. These
specify the users and groups that Condor will be allowed to act on
behalf of. These settings will need to include all users and groups
that Condor jobs may use on the given execute machine.

\item \Code{valid-dirs}. This specifies a list of directories that
Condor will be allowed to manage for the use of temporary job
files. Normally, this will only need to include Condor's
\Macro{EXECUTE} directory.

\item \Code{procd-executable}. A (trusted) full path to the
\Condor{procd} executable must be given.

\end{itemize}

An example of a full \File{privsep\_config} file is given below. This
file gives the \Login{condor} account access to the PrivSep
Kernel. Condor's use of this execute machine will be restricted to a
set of eight dedicated accounts along with the \Login{users}
group. Condor's \MacroNI{EXECUTE} directory and the \Condor{procd}
executable are also provided, as required.
\begin{verbatim}
valid-caller-uids = condor
valid-caller-gids = condor
valid-target-uids = condor_run_1 - condor_run_8
valid-target-gids = users : condor_run_1 - condor_run_8
valid-dirs = /opt/condor/local/execute
procd-executable = /opt/condor/release/sbin/condor_procd
\end{verbatim}

Once the PrivSep Kernel is properly installed and configured, Condor's
configuration must be updated to specify that PrivSep should be
used. The \Macro{PRIVSEP\_ENABLED} parameter is a boolean flag that
serves this purpose. In addition, Condor must be told where the
\Condor{root\_switchboard} binary is located using the
\Macro{PRIVSEP\_SWITCHBOARD} setting. The following example
illustrates:
\begin{verbatim}
PRIVSEP_ENABLED = True
PRIVSEP_SWITCHBOARD = $(SBIN)/condor_root_switchboard
\end{verbatim}

Finally, note that while the \Condor{procd} is in general an optional
component of Condor, it is required when PrivSep is in use. If
\MacroNI{PRIVSEP\_ENABLED} is true, the \Condor{procd} will be used
regardless of the \Macro{USE\_PROCD} setting.

%%%%%%%%%%%%%%%%%%%%%%%%%%%%%%%%%%%%%%%%%%%%%%%%%%%%%%%%%%%%%%%%%%%%%%
\subsubsection{PrivSep Kernel Interface}
%%%%%%%%%%%%%%%%%%%%%%%%%%%%%%%%%%%%%%%%%%%%%%%%%%%%%%%%%%%%%%%%%%%%%%

This section describes the \Login{root}-enabled operations that the
PrivSep Kernel makes available to Condor. The PrivSep Kernel's
interface is designed to provide only operations needed by Condor in
order to function properly. Each operation is further restricted based
on the PrivSep Kernel's configuration settings.

The following list describes each action that can be performed via the
PrivSep Kernel, along with the limitations enforced on how it may be
used. The terms \emph{valid target users}, \emph{valid target groups},
and \emph{valid directories} refer respectively to the settings for
\Code{valid-target-uids}, \Code{valid-target-gids}, and
\Code{valid-dirs} from the PrivSep Kernel's configuration.

\begin{itemize}

\item \emph{Make a directory as a user.} This operation simply creates
an empty directory as a regular user. The user must be a valid target
user, and the new directory's parent must be a valid directory.

\item \emph{Change ownership of a directory tree.} This operation
involves recursively changing ownership of all files and
subdirectories contained in a given directory. The directory's parent
must be a valid directory, and the new owner must either be a valid
target user or the user invoking the PrivSep Kernel.

\item \emph{Remove a directory tree.} This deletes a given directory,
including everything contained within. The directory's parent must be
a valid directory.

\item \emph{Execute a program as a user.} Condor can invoke the
PrivSep kernel to execute a program as a valid target user. The user's
primary group and any supplemental groups that it is a member of must
all be valid target groups. This operation may also include opening
files for standard input, output, and error before executing the
program.

\end{itemize}

After launching a program as a valid target user, the PrivSep Kernel
allows Condor limited control over its execution. The following
operations are supported on a progam started via the PrivSep Kernel:

\begin{itemize}

\item \emph{Get resource usage information.} This allows Condor to
gather usage statistics such as CPU time and memory image size. This
applies to the program's initial process and any of its descendents.

\item \emph{Signal the program.} Condor may ask that signals be sent
to the program's initial process as a notification mechanism.

\item \emph{Suspend and resume the program.} These operations send
\Code{SIGSTOP} or \Code{SIGCONT} signals to all processes that make up
the program.

\item \emph{Kill the process and all descendents.} Condor is allowed
to terminate the execution of the program or cleanup any processes
left behind when the program completes.

\end{itemize}

It is apparent that by sufficiently constraining the valid target
accounts and valid directories that the PrivSep Kernel will allow
access to, the ability of a compromised Condor daemon to do damage can
be considerably reduced.

