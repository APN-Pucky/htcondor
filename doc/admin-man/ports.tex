%%%%%%%%%%%%%%%%%%%%%%%%%%%%%%%%%%%%%%%%%%%%%%%%%%%%%%%%%%%%%%%%%%%%%%%%%%%
\subsection{\label{sec:Port-Details}Port Usage in HTCondor}
%%%%%%%%%%%%%%%%%%%%%%%%%%%%%%%%%%%%%%%%%%%%%%%%%%%%%%%%%%%%%%%%%%%%%%%%%%
\index{port usage}

%%%%%%%%%%%%%%%%%%%%%%%%%%%%%%%%%%%%%%%%%%%%%%%%%%%%%%%%%%%%%%%%%%%%%%%%%%%
\subsubsection{\label{sec:IPv4-Port-Specification}IPv4 Port Specification}
%%%%%%%%%%%%%%%%%%%%%%%%%%%%%%%%%%%%%%%%%%%%%%%%%%%%%%%%%%%%%%%%%%%%%%%%%%%
\index{IPv4 port specification}
\index{port usage!IPv4 port specification}
The general form for IPv4 port specification is 
\begin{verbatim}
<IP:port?param1name=value1&param2name=value2&param3name=value3&...> 
\end{verbatim}

These parameters and values are URL-encoded.
This means any special character is encoded with \Percent,
followed by two hexadecimal digits specifying the ASCII value.
Special characters are any non-alphanumeric character. 

HTCondor currently recognizes the following parameters with an
IPv4 port specification:
\begin{description}
\item[\Expr{CCBID}]
  Provides contact information for forming a CCB connection to a daemon, 
  or a space separated list, if the daemon is registered with more than 
  one CCB server.
  Each contact information is specified in the form of \verb|IP:port#ID|.
  Note that spaces between list items will be URL encoded by \verb|%20|.
\item[\Expr{PrivNet}]
  Provides the name of the daemon's private network.
  This value is specified in the configuration with 
  \MacroNI{PRIVATE\_NETWORK\_NAME}.
\item[\Expr{sock}]
  Provides the name of \Condor{shared\_port} daemon named socket.
\item[\Expr{PrivAddr}]
  Provides the daemon's private address in form of
  \Expr{IP:port}.
\end{description}

%%%%%%%%%%%%%%%%%%%%%%%%%%%%%%%%%%%%%%%%%%%%%%%%%%%%%%%%%%%%%%%%%%%%%%%%%%%
\subsubsection{\label{sec:Ports-Standard}Default Port Usage}
%%%%%%%%%%%%%%%%%%%%%%%%%%%%%%%%%%%%%%%%%%%%%%%%%%%%%%%%%%%%%%%%%%%%%%%%%%%

Every HTCondor daemon listens on a network port for incoming commands.
(Using \Condor{shared\_port}, this port may be shared between multiple
daemons.)
Most daemons listen on a dynamically assigned port.
In order to send a message,
HTCondor daemons and tools locate the correct port to use
by querying the \Condor{collector},
extracting the port number from the ClassAd.
One of the attributes included in every daemon's ClassAd is the full
IP address and port number upon which the daemon is listening.

To access the \Condor{collector} itself,
all HTCondor daemons and tools
must know the port number  where the \Condor{collector} is listening.
The \Condor{collector} is the only daemon with a well-known,
fixed port.
By default, HTCondor uses port 9618 for the \Condor{collector} daemon.
However, this port number can be changed (see below).

As an optimization for daemons and tools communicating with another
daemon that is running on the same host,
each HTCondor daemon can be configured to
write its IP address and port number into a well-known file.
The file names are controlled using the \MacroB{<SUBSYS>\_ADDRESS\_FILE}
configuration variables,
as described in section~\ref{param:SubsysAddressFile} on
page~\pageref{param:SubsysAddressFile}. 

\Note In the 6.6 stable series, and HTCondor versions earlier than
6.7.5, the \Condor{negotiator} also listened on a fixed, well-known
port (the default was 9614).
However, beginning with version 6.7.5, the \Condor{negotiator} behaves
like all other HTCondor daemons, and publishes its own ClassAd to the
\Condor{collector} which includes the dynamically assigned port 
the \Condor{negotiator} is listening on.
All HTCondor tools and daemons that need to communicate with the
\Condor{negotiator} will either use the
\Macro{NEGOTIATOR\_ADDRESS\_FILE} or will query the
\Condor{collector} for the \Condor{negotiator}'s ClassAd.

Sites that configure any checkpoint servers will introduce
other fixed ports into their network.
Each \Condor{ckpt\_server} will listen to 4 fixed ports: 5651, 5652,
5653, and 5654.
There is currently no way to configure alternative values for any of
these ports.


%%%%%%%%%%%%%%%%%%%%%%%%%%%%%%%%%%%%%%%%%%%%%%%%%%%%%%%%%%%%%%%%%%%%%%%%%%%
\subsubsection{\label{sec:Ports-NonStandard}Using 
a Non Standard, Fixed Port for the \Condor{collector}}
%%%%%%%%%%%%%%%%%%%%%%%%%%%%%%%%%%%%%%%%%%%%%%%%%%%%%%%%%%%%%%%%%%%%%%%%%%%
\index{port usage!nonstandard ports for central managers}
By default,
HTCondor uses port 9618 for the \Condor{collector} daemon.
To use a different port number for this daemon,
the configuration variables that tell HTCondor these communication
details are modified.
Instead of
\begin{verbatim}
CONDOR_HOST = machX.cs.wisc.edu
COLLECTOR_HOST = $(CONDOR_HOST)
\end{verbatim}
the configuration might be
\begin{verbatim}
CONDOR_HOST = machX.cs.wisc.edu
COLLECTOR_HOST = $(CONDOR_HOST):9650
\end{verbatim}

If a non standard port is defined, the same value of
\MacroNI{COLLECTOR\_HOST} (including the port) must be used for all
machines in the HTCondor pool.
Therefore, this setting should be modified in the global
configuration file (\File{condor\_config} file),
or the value must be duplicated across
all configuration files in the pool if a single configuration file
is not being shared.

When querying the \Condor{collector} for a remote pool that is running
on a non standard port, any HTCondor tool that accepts the \Opt{-pool}
argument can optionally be given a port number.  For example:
\footnotesize
\begin{verbatim}
        % condor_status -pool foo.bar.org:1234
\end{verbatim}
\normalsize


%%%%%%%%%%%%%%%%%%%%%%%%%%%%%%%%%%%%%%%%%%%%%%%%%%%%%%%%%%%%%%%%%%%%%%%%%%%
\subsubsection{\label{sec:Ports-Dynamic-Collector}Using 
a Dynamically Assigned Port for the \Condor{collector}}
%%%%%%%%%%%%%%%%%%%%%%%%%%%%%%%%%%%%%%%%%%%%%%%%%%%%%%%%%%%%%%%%%%%%%%%%%%%

On single machine pools, 
it is permitted to configure the
\Condor{collector} daemon
to use a dynamically assigned port,
as given out by the operating system.
This prevents port conflicts with other services on the same machine.
However, a dynamically assigned port is only to be used on
single machine HTCondor pools,
and only if the
\Macro{COLLECTOR\_ADDRESS\_FILE} 
configuration variable has also been defined.
This mechanism allows all of the HTCondor daemons and tools running on
the same machine to find the port upon which the \Condor{collector}
daemon is listening,
even when this port is not defined in the
configuration file and is not known in advance.

To enable the \Condor{collector} daemon to use a dynamically assigned port,
the port number is set to 0 in the \Macro{COLLECTOR\_HOST}
variable.
The \MacroNI{COLLECTOR\_ADDRESS\_FILE}
configuration variable must also be defined,
as it provides a known file where the IP address
and port information will be stored.
All HTCondor clients know to look at the
information stored in this file.
For example:
\footnotesize
\begin{verbatim}
COLLECTOR_HOST = $(CONDOR_HOST):0
COLLECTOR_ADDRESS_FILE = $(LOG)/.collector_address
\end{verbatim}
\normalsize

Configuration definition of \MacroNI{COLLECTOR\_ADDRESS\_FILE}
is in section~\ref{param:SubsysAddressFile} on
page~\pageref{param:SubsysAddressFile},
and
\MacroNI{COLLECTOR\_HOST}
is in
section~\ref{param:CollectorHost} on
page~\pageref{param:CollectorHost}.


%%%%%%%%%%%%%%%%%%%%%%%%%%%%%%%%%%%%%%%%%%%%%%%%%%%%%%%%%%%%%%%%%%%%%%%%%%%
\subsubsection{\label{sec:Ports-Firewalls}Restricting Port Usage to
 Operate with Firewalls}
%%%%%%%%%%%%%%%%%%%%%%%%%%%%%%%%%%%%%%%%%%%%%%%%%%%%%%%%%%%%%%%%%%%%%%%%%%%

\index{port usage!firewalls}
If an HTCondor pool is completely behind a firewall,
then no special consideration or port usage is needed.
However, if there is a firewall between the machines within
an HTCondor pool, then
configuration variables may be set to force the usage of
specific ports, and to utilize a specific range of ports.

By default,
HTCondor uses port 9618 for the \Condor{collector} daemon,
and dynamic (apparently random) ports for everything else.
See section~\ref{sec:Ports-Dynamic-Collector},
if a dynamically assigned port is desired for the
\Condor{collector} daemon.

All of the HTCondor daemons on a machine
may be configured to share a single
port.  See section~\ref{sec:Config-shared-port} for more information.

The configuration variables
\Macro{HIGHPORT} and \Macro{LOWPORT} facilitate setting a restricted
range of ports that HTCondor will use.
This may be useful when some machines are behind a firewall.
The configuration macros
\MacroNI{HIGHPORT} and \MacroNI{LOWPORT} 
will restrict dynamic ports to the range specified.
The configuration variables are fully defined
in section~\ref{sec:Network-Related-Config-File-Entries}.
All of these ports must be greater than 0 and less than 65,536.
Note that both \MacroNI{HIGHPORT} and \MacroNI{LOWPORT} must be at 
least 1024 for HTCondor version 6.6.8.
In general, use ports greater than 1024,
in order
to avoid port conflicts with standard services on the machine.
Another reason for using ports greater than 1024 is that
daemons and tools are often not run as \Login{root},
and only \Login{root} may listen to a port lower than 1024.
Also, the range must include enough ports that are not in use, 
or HTCondor cannot work.

The range of ports assigned may be restricted based on 
incoming (listening) and outgoing (connect) ports
with the configuration variables
\Macro{IN\_HIGHPORT},
\Macro{IN\_LOWPORT},
\Macro{OUT\_HIGHPORT}, and
\Macro{OUT\_LOWPORT}.
See section~\ref{sec:Network-Related-Config-File-Entries}
for complete definitions of these configuration variables.
A range of ports lower than 1024 for daemons
running as \Login{root} is appropriate for incoming ports,
but not for outgoing ports.
The use of ports below 1024 (versus above 1024)
has security implications; 
therefore, it is inappropriate to assign a range that crosses
the 1024 boundary.


\Note Setting \MacroNI{HIGHPORT} and \MacroNI{LOWPORT} will not
automatically force the \Condor{collector} to bind to a port within
the range.
The only way to control what port the \Condor{collector} uses is by
setting the \MacroNI{COLLECTOR\_HOST} (as described above).

The total number of ports needed depends on the size of the pool,
the usage of the machines within the pool (which machines
run which daemons),
and the number of jobs that may execute at one time.
Here we discuss how many ports are used by each
participant in the system.  This assumes that \Condor{shared\_port}
is not being used.  If it \emph{is} being used, then all daemons
can share a single incoming port.

The central manager of the pool needs
\Expr{5 + number of \Condor{schedd} daemons}
ports for outgoing connections and 2 ports for incoming connections for
daemon communication.

Each execute machine (those machines running a \Condor{startd} daemon)
requires
\Expr{ 5 + (5 * number of slots advertised by that machine)}
ports.
By default, the number of slots advertised
will equal the number of physical CPUs in that machine.

Submit machines (those machines running a \Condor{schedd} daemon)
require
\Expr{ 5 + (5 *  \MacroNI{MAX\_JOBS\_RUNNING})} ports.
The configuration variable \Macro{MAX\_JOBS\_RUNNING}
limits (on a per-machine basis, if desired)
the maximum number of jobs.
Without this configuration macro,
the maximum number of jobs that could be simultaneously
executing at one time
is a function of the number of reachable execute machines. 

Also be aware that \MacroNI{HIGHPORT} and \MacroNI{LOWPORT}
only impact dynamic port selection used by the HTCondor system,
and they do not impact port selection used by jobs submitted to HTCondor.
Thus, jobs submitted to HTCondor that may create
network connections may not work in a port restricted environment.
For this reason, specifying \MacroNI{HIGHPORT} and \MacroNI{LOWPORT}
is not going to produce the
expected results if a user submits MPI applications to be executed under
the parallel universe.

Where desired, a local
configuration for machines \emph{not} behind a firewall
can override the usage of \MacroNI{HIGHPORT} and \MacroNI{LOWPORT},
such that the ports used for these machines are not restricted.
This can be accomplished by adding the following to the
local configuration file of those machines \emph{not}
behind a firewall:
\begin{verbatim}
HIGHPORT = UNDEFINED
LOWPORT  = UNDEFINED
\end{verbatim}


If the maximum number of ports allocated using 
\MacroNI{HIGHPORT} and \MacroNI{LOWPORT}
is too few,
socket binding errors of the form
\footnotesize
\begin{verbatim}
failed to bind any port within <$LOWPORT> - <$HIGHPORT>
\end{verbatim}
\normalsize
are likely to appear repeatedly in log files.


%%%%%%%%%%%%%%%%%%%%%%%%%%%%%%%%%%%%%%%%%%%%%%%%%%%%%%%%%%%%%%%%%%%%%%%%%%%
\subsubsection{\label{sec:Ports-MultipleCollectors}Multiple Collectors}
%%%%%%%%%%%%%%%%%%%%%%%%%%%%%%%%%%%%%%%%%%%%%%%%%%%%%%%%%%%%%%%%%%%%%%%%%%%
\index{port usage!multiple collectors}
\Todo


%%%%%%%%%%%%%%%%%%%%%%%%%%%%%%%%%%%%%%%%%%%%%%%%%%%%%%%%%%%%%%%%%%%%%%%%%%%
\subsubsection{\label{sec:Ports-Conflicts}Port Conflicts}
%%%%%%%%%%%%%%%%%%%%%%%%%%%%%%%%%%%%%%%%%%%%%%%%%%%%%%%%%%%%%%%%%%%%%%%%%%%
\index{port usage!conflicts}
\Todo

