%%%%%%%%%%%%%%%%%%%%%%%%%%%%%%%%%%%%%%%%%%%%%%%%%%%%%%%%%%%%%%%%%%%%%%
\section{Introduction to Configuration}\label{sec:Configuring-HTCondor-Intro}
%%%%%%%%%%%%%%%%%%%%%%%%%%%%%%%%%%%%%%%%%%%%%%%%%%%%%%%%%%%%%%%%%%%%%%

\index{HTCondor!configuration-intro}
\index{configuration: introduction}

This section of the manual contains general information about HTCondor
configuration, relating to all parts of the HTCondor system.
If you're setting up an HTCondor pool, you should read
this section before you read the other configuration-related sections:

\begin{itemize}

\item Section~\ref{sec:Configuring-HTCondor-Templates} contains information
about configuration templates, which are now the preferred way to set
many configuration macros.

\item Section~\ref{sec:Configuring-HTCondor-Macros} contains information
about the hundreds of individual configuration macros.  In general,
it is best to try to achieve your desired configuration using
configuration templates
before resorting to setting individual configuration macros, but it is
sometimes necessary to set individual configuration macros.

\item The settings that control the policy under which HTCondor will start,
suspend, resume, vacate or kill jobs
are described in 
section~\ref{sec:Configuring-Policy} on Policy Configuration for the
\Condor{startd}. 
\end{itemize}

%%%%%%%%%%%%%%%%%%%%%%%%%%%%%%%%%%%%%%%%%%%%%%%%%%%%%%%%%%%%%%%%%%%%%%
\subsection{\label{sec:Intro-to-Config-Files}HTCondor
Configuration Files} 
%%%%%%%%%%%%%%%%%%%%%%%%%%%%%%%%%%%%%%%%%%%%%%%%%%%%%%%%%%%%%%%%%%%%%%

The HTCondor configuration files are used to customize how HTCondor
operates at a given site.  The basic configuration as shipped with
HTCondor can be used as a starting point, but most likely you will
want to modify that configuration to some extent.

Each HTCondor program will, as part of its initialization process,
configure itself by calling a library routine which parses the
various configuration files that might be used, including pool-wide,
platform-specific, and machine-specific configuration files.
Environment variables may also contribute to the configuration.

The result of configuration is a list of key/value pairs.
Each key is a configuration variable name,
and each value is a string literal
that may utilize macro substitution (as defined below).
Some configuration variables are evaluated by HTCondor as ClassAd
expressions; some are not.  Consult the documentation for each specific
case.  Unless otherwise noted, configuration values that are expected
to be numeric or boolean constants can be any valid ClassAd expression
of operators on constants.  Example:

\begin{verbatim}
MINUTE          = 60
HOUR            = (60 * $(MINUTE))
SHUTDOWN_GRACEFUL_TIMEOUT = ($(HOUR)*24)
\end{verbatim}

%%%%%%%%%%%%%%%%%%%%%%%%%%%%%%%%%%%%%%%%%%%%%%%%%%%%%%%%%%%%%%%%%%%%%%
\subsection{\label{sec:Ordering-Config-File}Ordered Evaluation to
Set the Configuration} 
%%%%%%%%%%%%%%%%%%%%%%%%%%%%%%%%%%%%%%%%%%%%%%%%%%%%%%%%%%%%%%%%%%%%%%
\index{configuration file!evaluation order}

Multiple files, as well as a program's environment variables,
determine the configuration.
The order in which attributes are defined is important, as later
definitions override earlier definitions.
The order in which the (multiple) configuration files are parsed 
is designed to ensure the security of the system.
Attributes which must be set a specific way 
must appear in the last file to be parsed.
This prevents both the naive and the malicious HTCondor user 
from subverting the system through its configuration.
The order in which items are parsed is:
\begin{enumerate}
\item a single initial configuration file, which has historically
been known as the global configuration file (see below);
\item other configuration files that are referenced and parsed due 
to specification within the single initial configuration file
(these files have historically been known as local configuration files);
\item if HTCondor daemons are \emph{not} running as \Login{root} on
Unix platforms,
the file \File{\$(HOME)/.condor/user\_config} if it exists,
or the file defined by configuration variable \Macro{USER\_CONFIG\_FILE};

if HTCondor daemons are \emph{not} running as Local System on 
Windows platforms,
the file \verb@%USERPROFILE\.condor\user_config@ if it exists,
or the file defined by configuration variable \Macro{USER\_CONFIG\_FILE};

\item specific environment variables whose names are
prefixed with \MacroNI{\_CONDOR\_} (note that these environment variables
directly define macro name/value pairs, not the names of configuration
files).
\end{enumerate}


Some HTCondor tools utilize environment variables to set their
configuration;
these tools search for specifically-named environment variables.
The variable names are prefixed by the string \MacroNI{\_CONDOR\_}
or \MacroNI{\_condor\_}.
The tools strip off the prefix, and utilize what remains
as configuration.
As the use of environment variables is the last within
the ordered evaluation, 
the environment variable definition is used.
The security of the system is not compromised,
as only specific variables are considered for definition
in this manner, not any environment variables with
the \MacroNI{\_CONDOR\_} prefix.

The location of the single initial configuration file differs on
Windows from Unix platforms. 
For Unix platforms, 
the location of the single initial configuration file starts at 
the top of the following list.
The first file that exists is used,
and then remaining possible file locations from this list become irrelevant.

\begin{enumerate}
   \item the file specified by the \Env{CONDOR\_CONFIG} environment variable.
   If there is a problem reading that file, HTCondor will print an error
   message and exit right away.
   \item \File{/etc/condor/condor\_config}
   \item \File{/usr/local/etc/condor\_config}
   \item \File{\Tilde condor/condor\_config}
\end{enumerate}

For Windows platforms, 
the location of the single initial configuration file is determined by 
the contents of the environment variable \Env{CONDOR\_CONFIG}.
If this environment variable is not defined, 
then the location is the registry value of 
\Expr{HKEY\_LOCAL\_MACHINE/Software/Condor/CONDOR\_CONFIG}. 

The single, initial configuration file may contain the specification 
of one or more other configuration files, referred to here as
local configuration files.
Since more than one file may contain a definition of the same variable,
and since the last definition of a variable sets the value, 
the parse order of these local configuration files
is fully specified here. 
In order: 
\begin{enumerate}
   \item The value of configuration variable \Macro{LOCAL\_CONFIG\_DIR} lists 
   one or more directories which contain configuration files. 
   The list is parsed from left to right. 
   The leftmost (first) in the list is parsed first. 
   Within each directory, 
   a lexicographical ordering by file name determines the ordering of 
   file consideration.
   \item The value of configuration variable \Macro{LOCAL\_CONFIG\_FILE} 
   lists one or more configuration files. 
   These listed files are parsed from  left to right.
   The leftmost (first) in the list is parsed first.
   \item If one of these steps changes the value (right hand side) of 
   \MacroNI{LOCAL\_CONFIG\_DIR}, 
   then \MacroNI{LOCAL\_CONFIG\_DIR} is processed for a second time,
    using the changed list of directories.
\end{enumerate}

The parsing and use of configuration files may be bypassed by
setting environment variable \Env{CONDOR\_CONFIG} with the
string \Expr{ONLY\_ENV}.
With this setting, there is no attempt to locate or read
configuration files.
This may be useful for testing where the environment contains
all needed information.

%%%%%%%%%%%%%%%%%%%%%%%%%%%%%%%%%%%%%%%%%%%%%%%%%%%%%%%%%%%%%%%%%%%%%%
\subsection{\label{sec:Config-File-Macros}Configuration File Macros} 
%%%%%%%%%%%%%%%%%%%%%%%%%%%%%%%%%%%%%%%%%%%%%%%%%%%%%%%%%%%%%%%%%%%%%%
%TEMP -- give example with default value

\index{macro!in configuration file}
\index{configuration file!macro definitions}

Macro definitions are of the form:
\begin{verbatim}
<macro_name> = <macro_definition>
\end{verbatim}

The macro name given on the left hand side of the definition is
a case insensitive identifier.
There may be white space between the macro name, the
equals sign (\verb@=@), and the macro definition.
The macro definition is a string literal that may utilize macro substitution.

Macro invocations are of the form: 
\begin{verbatim}
$(macro_name[:<default if macro_name not defined>])
\end{verbatim}

The colon and default are optional in a macro invocation.
Macro definitions may contain references to other macros, even ones
that are not yet defined, as long as they are eventually defined in
the configuration files.
All macro expansion is done after all configuration files have been parsed,
with the exception of macros that reference themselves.

\begin{verbatim}
A = xxx
C = $(A) 
\end{verbatim}
is a legal set of macro definitions, and the resulting value of 
\MacroNI{C} is
\Expr{xxx}.
Note that
\MacroNI{C} is actually bound to 
\MacroUNI{A}, not its value.

As a further example,
\begin{verbatim}
A = xxx
C = $(A)
A = yyy
\end{verbatim}
is also a legal set of macro definitions, and the resulting value of
\MacroNI{C} is \Expr{yyy}.  

A macro may be incrementally defined by invoking itself in its
definition.  For example,
\begin{verbatim}
A = xxx
B = $(A)
A = $(A)yyy
A = $(A)zzz
\end{verbatim}
is a legal set of macro definitions, and the resulting value of 
\MacroNI{A}
is \Expr{xxxyyyzzz}.
Note that invocations of a macro in
its own definition are immediately
expanded.
\MacroUNI{A} is immediately expanded in line 3 of the example.
If it were not, then the definition would be impossible to
evaluate.

Recursively defined macros such as
\begin{verbatim}
A = $(B)
B = $(A)
\end{verbatim}
are \emph{not} allowed.
They create definitions that HTCondor refuses to parse. 

A macro invocation where the macro name is not defined results
in a substitution of the empty string.
Consider the example
\begin{verbatim}
MAX_ALLOC_CPUS = $(NUMCPUS)-1
\end{verbatim}
If \MacroNI{NUMCPUS} is not defined, then this macro substitution
becomes
\begin{verbatim}
MAX_ALLOC_CPUS = -1
\end{verbatim}
The default value may help to avoid this situation.
The default value may be a literal
\begin{verbatim}
MAX_ALLOC_CPUS = $(NUMCPUS:4)-1
\end{verbatim}
such that if \MacroNI{NUMCPUS} is not defined, the result of macro
substitution becomes
\begin{verbatim}
MAX_ALLOC_CPUS = 4-1
\end{verbatim}
The default may be another macro invocation:
\begin{verbatim}
MAX_ALLOC_CPUS = $(NUMCPUS:$(DETECTED_CPUS))-1
\end{verbatim}
These default specifications are restricted such that 
a macro invocation with a default can not be nested inside of another default.  
An alternative way of stating this restriction is that there can
only be one colon character per line. 
The effect of nested defaults can be achieved by placing the
macro definitions on separate lines of the configuration.

All entries in a configuration file must have an operator,
which will be an equals sign (\verb@=@).
Identifiers are alphanumerics combined with the underscore character,
optionally with a subsystem name and a period as a prefix.
As a special case,
a line without an operator that begins with a left square bracket
will be ignored.
The following two-line example treats the first line as a comment,
and correctly handles the second line.
\begin{verbatim}
[HTCondor Settings]
my_classad = [ foo=bar ]
\end{verbatim}

% functionality added to version 6.7.13
To simplify pool administration,
any configuration variable name may be prefixed by
a subsystem 
(see the \MacroUNI{SUBSYSTEM} macro in 
section~\ref{sec:Pre-Defined-Macros}
for the list of subsystems)
and the period (\verb@.@) character.
For configuration variables defined this way,
the value is applied to the specific subsystem.
For example,
the ports that HTCondor may use can be restricted to a range 
using the \MacroNI{HIGHPORT} and \MacroNI{LOWPORT} configuration
variables.

\begin{verbatim}
  MASTER.LOWPORT   = 20000
  MASTER.HIGHPORT  = 20100
\end{verbatim}

Note that all configuration variables may utilize this syntax,
but nonsense configuration variables may result.
For example, it makes no sense to define
\begin{verbatim}
  NEGOTIATOR.MASTER_UPDATE_INTERVAL = 60
\end{verbatim}
since the \Condor{negotiator} daemon does not use the
\MacroNI{MASTER\_UPDATE\_INTERVAL} variable.

It makes little sense to do so, but HTCondor will configure
correctly with a definition such as
\begin{verbatim}
  MASTER.MASTER_UPDATE_INTERVAL = 60
\end{verbatim}
The \Condor{master} uses this configuration variable,
and the prefix of \MacroNI{MASTER.} causes this configuration
to be specific to the \Condor{master} daemon.

As of HTCondor version 8.1.1, evaluation works in the expected manner
when combining the definition of a macro with
use of a prefix that gives the subsystem name and a period. 
Consider the example
\begin{verbatim}
  FILESPEC = A
  MASTER.FILESPEC = B 
\end{verbatim}
combined with a later definition that incorporates \Expr{FILESPEC}
in a macro:
\begin{verbatim}
  USEFILE = mydir/$(FILESPEC)
\end{verbatim}
When the \Condor{master} evaluates variable \MacroNI{USEFILE},
it evaluates to \Expr{mydir/B}. 
Previous to HTCondor version 8.1.1, it evaluated to \Expr{mydir/A}.
When any other subsystem evaluates variable \MacroNI{USEFILE},
it evaluates to \Expr{mydir/A}.

% the local functionality added in 7.1.4
This syntax has been further expanded to allow for the
specification of a local name on the command line 
using the command line option
\begin{verbatim}
  -local-name <local-name>
\end{verbatim}
This allows multiple instances of a daemon to be run 
by the same \Condor{master} daemon,
each instance with its own local configuration variable.

The ordering used to look up a variable, called \verb@<parameter name>@:

\begin{enumerate}
\item \verb@<subsystem name>.<local name>.<parameter name>@

\item \verb@<local name>.<parameter name>@

\item \verb@<subsystem name>.<parameter name>@

\item \verb@<parameter name>@
\end{enumerate}

If this local name is not specified on the command line, 
numbers 1 and 2 are skipped.
As soon as the first match is found, the search is completed,
and the corresponding value is used.

This example configures a \Condor{master} to run 2 \Condor{schedd}
daemons.  The \Condor{master} daemon needs the configuration:
\begin{verbatim}
  XYZZY           = $(SCHEDD)
  XYZZY_ARGS      = -local-name xyzzy
  DAEMON_LIST     = $(DAEMON_LIST) XYZZY
  DC_DAEMON_LIST  = + XYZZY
  XYZZY_LOG       = $(LOG)/SchedLog.xyzzy
\end{verbatim}

Using this example configuration, the \Condor{master} starts up a
second \Condor{schedd} daemon, 
where this second \Condor{schedd} daemon is passed 
\OptArg{-local-name}{xyzzy}
on the command line.

Continuing the example,
configure the \Condor{schedd} daemon named \Attr{xyzzy}.
This \Condor{schedd} daemon will share all configuration variable
definitions with the other \Condor{schedd} daemon,
except for those specified separately.

\begin{verbatim}
  SCHEDD.XYZZY.SCHEDD_NAME = XYZZY
  SCHEDD.XYZZY.SCHEDD_LOG  = $(XYZZY_LOG)
  SCHEDD.XYZZY.SPOOL       = $(SPOOL).XYZZY
\end{verbatim}

Note that the example \MacroNI{SCHEDD\_NAME} and \MacroNI{SPOOL} are
specific to the \Condor{schedd} daemon, as opposed to a different daemon
such as the \Condor{startd}.
Other HTCondor daemons using this feature will
have different requirements for which parameters need to be
specified individually.  This example works for the \Condor{schedd},
and more local configuration can, and likely would be specified.

Also note that each daemon's log file must be specified individually,
and in two places: one specification is for use by the \Condor{master},
and the other is for use by the daemon itself.
In the example,
the \Attr{XYZZY} \Condor{schedd} configuration variable
\MacroNI{SCHEDD.XYZZY.SCHEDD\_LOG} definition references the
\Condor{master} daemon's \MacroNI{XYZZY\_LOG}.


%%%%%%%%%%%%%%%%%%%%%%%%%%%%%%%%%%%%%%%%%%%%%%%%%%%%%%%%%%%%%%%%%%%%%%
\subsection{\label{sec:Other-Syntax}Comments and Line Continuations}
%%%%%%%%%%%%%%%%%%%%%%%%%%%%%%%%%%%%%%%%%%%%%%%%%%%%%%%%%%%%%%%%%%%%%%

An HTCondor configuration file may contain comments and
line continuations.
A comment is any line beginning with a pound character (\verb@#@).
A continuation is any entry that continues across multiples lines.
Line continuation is accomplished by placing the backslash
character (\verb@/@) at the end of any line to be continued onto another.
Valid examples of line continuation are
\begin{verbatim}
  START = (KeyboardIdle > 15 * $(MINUTE)) && \
  ((LoadAvg - CondorLoadAvg) <= 0.3)
\end{verbatim}
and
\begin{verbatim}
  ADMIN_MACHINES = condor.cs.wisc.edu, raven.cs.wisc.edu, \
  stork.cs.wisc.edu, ostrich.cs.wisc.edu, \
  bigbird.cs.wisc.edu
  HOSTALLOW_ADMINISTRATOR = $(ADMIN_MACHINES)
\end{verbatim}

Where a line continuation character directly precedes a comment,
the entire comment line is ignored,
and the following line is used in the continuation. 
Line continuation characters within comments are ignored.

Both this example
\begin{verbatim}
  A = $(B) \
  # $(C)
  $(D)
\end{verbatim}
and this example
\begin{verbatim}
  A = $(B) \
  # $(C) \
  $(D)
\end{verbatim}
result in the same value for \verb@A@:
\begin{verbatim}
  A = $(B) $(D)
\end{verbatim}

%%%%%%%%%%%%%%%%%%%%%%%%%%%%%%%%%%%%%%%%%%%%%%%%%%%%%%%%%%%%%%%%%%%%%%
\subsection{\label{sec:Multi-Line-Values}Multi-Line Values}
%%%%%%%%%%%%%%%%%%%%%%%%%%%%%%%%%%%%%%%%%%%%%%%%%%%%%%%%%%%%%%%%%%%%%%

As of version 8.5.6, the value for a macro can comprise multiple
lines of text.
The syntax for this is as follows:
\begin{verbatim}
<macro_name> @=<tag>
<macro_definition lines>
@<tag>
\end{verbatim}

For example:
\begin{verbatim}
JOB_ROUTER_DEFAULTS @=jrd
  [
    requirements=target.WantJobRouter is True;
    MaxIdleJobs = 10;
    MaxJobs = 200;

    /* now modify routed job attributes */
    /* remove routed job if it goes on hold or stays idle for over 6 hours */
    set_PeriodicRemove = JobStatus == 5 ||
                        (JobStatus == 1 && (time() - QDate) > 3600*6);
    delete_WantJobRouter = true;
    set_requirements = true;
  ]
  @jrd
\end{verbatim}

Note that in this example, the square brackets are part of the
JOB\_ROUTER\_DEFAULTS value.

%%%%%%%%%%%%%%%%%%%%%%%%%%%%%%%%%%%%%%%%%%%%%%%%%%%%%%%%%%%%%%%%%%%%%%
\subsection{\label{sec:Program-Defined-Macros}Executing a Program to Produce Configuration Macros}
%%%%%%%%%%%%%%%%%%%%%%%%%%%%%%%%%%%%%%%%%%%%%%%%%%%%%%%%%%%%%%%%%%%%%%

Instead of reading from a file,
HTCondor can run a program to obtain configuration macros.
The vertical bar character (\Bar{}) as the last character defining
a file name provides the syntax necessary to tell 
HTCondor to run a program.
This syntax may only be used in the definition of
the \Env{CONDOR\_CONFIG} environment variable,
or the \Macro{LOCAL\_CONFIG\_FILE} configuration variable.

The command line for the program 
is formed by the characters preceding the vertical bar character.
The standard output of the program is parsed as a configuration 
file would be.

An example:
\begin{verbatim}
LOCAL_CONFIG_FILE = /bin/make_the_config|
\end{verbatim}

Program \Prog{/bin/make\_the\_config} is executed, and its output
is the set of configuration macros.

Note that either a program is executed to generate the
configuration macros or the configuration is read from 
one or more files.
The syntax uses space characters to separate command line elements,
if an executed program produces the configuration macros.
Space characters would otherwise separate the list of files.
This syntax does not permit distinguishing one from the other,
so only one may be specified.

(Note that the \Macro{include command} syntax (see below) is now
the preferred way to execute a program to generate configuration
macros.)

%%%%%%%%%%%%%%%%%%%%%%%%%%%%%%%%%%%%%%%%%%%%%%%%%%%%%%%%%%%%%%%%%%%%%%
\subsection{\label{sec:Config-Include}Including Configuration from Elsewhere}
%%%%%%%%%%%%%%%%%%%%%%%%%%%%%%%%%%%%%%%%%%%%%%%%%%%%%%%%%%%%%%%%%%%%%%
\index{configuration!INCLUDE syntax}
\index{INCLUDE configuration syntax}

%TEMP -- $(IsMaster), $(CondorIsAdmin) are not documented (see gittrac #5781)
% Note: I can't find IsMaster in the code (wenger 2016-09-28)

Externally defined configuration can be incorporated using the following
syntax:
\begin{verbatim}
  include [ifexist] : <file>
  include : <cmdline>|
  include [ifexist] command [into <cache-file>] : <cmdline>
\end{verbatim}

(Note that the \verb@ifexist@ and \verb@into@ options were
added in version 8.5.7.  Also note that the \verb@command@ option
must be specified in order to use the \verb@into@ option -- just
using the bar after \verb@<cmdline>@ will not work.)

In the file form of the \MacroNI{include} command, the \verb@<file>@
specification must describe a single file, the contents of which
will be parsed and incorporated into the configuration.
Unless the \verb@ifexist@ option is specified, the non-existence
of the file is a fatal error.

In the command line form of the \MacroNI{include} command (specified with
either the \verb@command@ option or by appending a bar (|) character
after the \verb@<cmdline>@ specification), the
\verb@<cmdline>@ specification must describe a command line
(program and arguments); the command line will be executed,
and the output will be parsed and incorporated into the configuration.

If the \verb@into@ option is not used, the command line
will be executed every time the configuration file is referenced.
This may well be undesirable, and can be avoided by using the
\verb@into@ option.
The \verb@into@ keyword must be followed by the full pathname of a file
into which to write the output of the command line. If that file exists, it
will be read and the command line will not be executed. If that file
does not exist, the output of the command line will be written into it
and then the cache file will be read and incorporated into the
configuration. If the command line produces no
output, a zero length file will be created. If the command line returns
a non-zero exit code, configuration will abort and the cache file will
not be created unless the \verb@ifexist@ keyword is also specified. 

The \MacroNI{include} key word is case insensitive.
There are \emph{no} requirements for white space characters surrounding
the colon character.

Consider the example
\begin{verbatim}
  FILE = config.$(FULL_HOSTNAME)
  include : $(LOCAL_DIR)/$(FILE)
\end{verbatim}
Values are acquired for configuration variables 
\MacroNI{FILE}, and \MacroNI{LOCAL\_DIR} by immediate evaluation, 
causing variable \MacroNI{FULL\_HOSTNAME} to also be immediately evaluated.
The resulting value forms a full path and file name.
This file is read and parsed.
The resulting configuration is incorporated into the current
configuration.
This resulting configuration may contain further nested \MacroNI{include} 
specifications, 
which are also parsed, evaluated, and incorporated.
Levels of nested \MacroNI{include}s are limited,
such that infinite nesting is discovered and thwarted,
while still permitting nesting.

Consider the further example
\begin{verbatim}
  SCRIPT_FILE = script.$(IP_ADDRESS)
  include : $(RELEASE_DIR)/$(SCRIPT_FILE) |
\end{verbatim}
In this example, the bar character at the end of the line 
causes a script to be invoked,
and the output of the script is incorporated into the current
configuration.
The same immediate parsing and evaluation occurs in this
case as when a file's contents are included. 

For pools that are transitioning to using this new syntax in configuration,
while still having some tools and daemons with HTCondor versions 
earlier than 8.1.6,
special syntax in the configuration will cause those daemons to
fail upon startup,
rather than continuing, but incorrectly parsing the new syntax.
Newer daemons will ignore the extra syntax.
Placing the \verb|@| character before the \MacroNI{include} key word
causes the older daemons to fail when they attempt to
parse this syntax.

Here is the same example, but with the syntax that causes older daemons
to fail when reading it.
\begin{verbatim}
  FILE = config.$(FULL_HOSTNAME)
  @include : $(LOCAL_DIR)/$(FILE)
\end{verbatim}
A daemon older than version 8.1.6 will fail to start.
Running an older \Condor{config\_val} identifies the \Expr{@include}
line as being bad.
A daemon of HTCondor version 8.1.6 or more recent sees:
\begin{verbatim}
  FILE = config.$(FULL_HOSTNAME)
  include : $(LOCAL_DIR)/$(FILE)
\end{verbatim}
and starts up successfully.

Here is an example using the new \verb@ifexist@ and \verb@into@ options:
\begin{verbatim}
  # stuff.pl writes "STUFF=1" to stdout
  include ifexist command into $(LOCAL_DIR)/stuff.config : perl $(LOCAL_DIR)/stuff.pl
\end{verbatim}

%%%%%%%%%%%%%%%%%%%%%%%%%%%%%%%%%%%%%%%%%%%%%%%%%%%%%%%%%%%%%%%%%%%%%%
\subsection{\label{sec:Config-Error_Warn}Reporting Errors and Warnings}
%%%%%%%%%%%%%%%%%%%%%%%%%%%%%%%%%%%%%%%%%%%%%%%%%%%%%%%%%%%%%%%%%%%%%%
\index{configuration!Error and warning syntax}
\index{Error and warning configuration syntax}

As of version 8.5.7, warning and error messages can be included
in HTCondor configuration files.

The syntax for warning and error messages is as follows:
\begin{verbatim}
  warning : <warning message>
  error : <error message>
\end{verbatim}

The warning and error messages will be printed when the configuration
file is used (when almost any HTCondor command is run, for example).
Error messages (unlike warnings) will prevent the
successful use of the configuration file.  This will, for example,
prevent a daemon from starting, and prevent \Condor{config\_val}
from returning a value.

Here's an example of using an error message in a configuration
file (combined with some of the new include features documented
above):
\begin{verbatim}
# stuff.pl writes "STUFF=1" to stdout
include command into $(LOCAL_DIR)/stuff.config : perl $(LOCAL_DIR)/stuff.pl
if ! defined stuff
  error : stuff is needed!
endif
\end{verbatim}

%%%%%%%%%%%%%%%%%%%%%%%%%%%%%%%%%%%%%%%%%%%%%%%%%%%%%%%%%%%%%%%%%%%%%%
\subsection{\label{sec:Config-IfElse}Conditionals in Configuration}
%%%%%%%%%%%%%%%%%%%%%%%%%%%%%%%%%%%%%%%%%%%%%%%%%%%%%%%%%%%%%%%%%%%%%%
\index{configuration!IF/ELSE syntax}
\index{IF/ELSE configuration syntax}
%% IMPORTANT:
%% This file contains prose to describe the semantics and syntax
%% of the if-else that can be used both in configuration and
%% in a submit description file.
%%
%% Please be careful when changing this prose, as it is included
%% in 2 different parts of the manual.  Do not include examples
%% specific to either a submit description file usage or a
%% configuration usage in this prose.

Conditional \verb@if@/\verb@else@ semantics
are available in a limited form.
The syntax:
\begin{verbatim}
  if <simple condition>
     <statement>
     . . .
     <statement>
  else
     <statement>
     . . .
     <statement>
  endif
\end{verbatim}

An \verb@else@ key word and statements are not required,
such that simple \verb@if@ semantics are implemented.
The  \verb@<simple condition>@ does not permit compound conditions.
It optionally contains the exclamation point character (\verb@!@)
to represent the not operation,
followed by 
\begin{itemize}
  \item the \verb@defined@ keyword followed by the name of a 
    variable.
    If the variable is defined, the statement(s) are
    incorporated into the expanded input.
    If the variable is \emph{not} defined, the statement(s) are
    not incorporated into the expanded input.
    As an example,
\begin{verbatim}
  if defined MY_UNDEFINED_VARIABLE
     X = 12
  else
     X = -1
  endif
\end{verbatim}
    results in \Expr{X = -1}, when \Expr{MY\_UNDEFINED\_VARIABLE} is
    \emph{not} yet defined.
  \item the \verb@version@ keyword, representing the version number of
    of the daemon or tool currently reading this conditional. 
    This keyword is followed by an HTCondor version number.
    That version number can be of the form \verb@x.y.z@ or \verb@x.y@.
    The version of the daemon or tool is compared to the specified
    version number.
    The comparison operators are
    \begin{itemize}
    \item \verb@==@ for equality. Current version 8.2.3 is equal to 8.2.
    \item \verb@>=@ to see if the current version number is greater than or
    equal to. Current version 8.2.3 is greater than 8.2.2,
    and current version 8.2.3 is greater than or equal to 8.2.
    \item \verb@<=@ to see if the current version number is less than or
    equal to. Current version 8.2.0 is less than 8.2.2, 
    and current version 8.2.3 is less than or equal to 8.2.
    \end{itemize}
    As an example,
\begin{verbatim}
  if version >= 8.1.6
     DO_X = True
  else
     DO_Y = True
  endif
\end{verbatim}
    results in defining \Expr{DO\_X} as \Expr{True} if the current
    version of the daemon or tool reading this if statement is 8.1.6
    or a more recent version.
  \item \verb@True@ or \verb@yes@ or the value 1.
    The statement(s) are incorporated.
  \item \verb@False@ or \verb@no@ or the value 0
    The statement(s) are \emph{not} incorporated.
  \item \verb@$(<variable>)@ may be used where the
    immediately evaluated value is a simple boolean value.
    A value that evaluates to the empty string is considered \verb@False@,
    otherwise a value that does not evaluate to a simple boolean value
    is a syntax error.
\end{itemize}

The syntax
\begin{verbatim}
  if <simple condition>
     <statement>
     . . .
     <statement>
  elif <simple condition>
     <statement>
     . . .
     <statement>
  endif
\end{verbatim}
is the same as syntax
\begin{verbatim}
  if <simple condition>
     <statement>
     . . .
     <statement>
  else
     if <simple condition>
        <statement>
        . . .
        <statement>
     endif
  endif
\end{verbatim}


%%%%%%%%%%%%%%%%%%%%%%%%%%%%%%%%%%%%%%%%%%%%%%%%%%%%%%%%%%%%%%%%%%%%%%
\subsection{\label{sec:admin-man-macros}Function Macros in Configuration}
%%%%%%%%%%%%%%%%%%%%%%%%%%%%%%%%%%%%%%%%%%%%%%%%%%%%%%%%%%%%%%%%%%%%%%
\index{configuration!function macros}
%% IMPORTANT:
%% This file contains prose to describe the semantics and syntax
%% of function macros that can be used both in configuration and
%% in a submit description file.
%%
%% Please be careful when changing this prose, as it is included
%% in 2 different parts of the manual.  Do not include examples
%% specific to either a submit description file usage or a
%% configuration usage in this prose.

\MoreTodo

A set of predefined functions increase flexibility.
Both submit description files and configuration files are read using
the same parser,
so these functions may be used in both submit description files and
configuration files.

Case is significant in the function's name, so use the
same letter case as given in these definitions.

\begin{description}

\item [\Expr{\$CHOICE(index, listname)} 
       or \Expr{\$CHOICE(index, item1, item2, \Dots)}]
An item within the list is returned.
The list is represented by a parameter name,
or the list items are the parameters.
The \Expr{index} parameter determines which item.
The first item in the list is at index 0.
If the index is out of bounds for the list contents, 
an error occurs.


\item [\Expr{\$ENV(environment-variable-name)}]
Evaluates to the value of environment variable 
\Expr{environment-variable-name}.
For example, 
\begin{verbatim}
  A = $ENV(HOME)
\end{verbatim}
binds \MacroNI{A} to the value of the \Env{HOME} environment variable.

\item [\Expr{\$F[pdnxq](filename)}]
One or more of the lower case letters may be combined to form the
function name and thus, its functionality.
Each letter operates on the \Expr{filename} in its own way.
\begin{itemize}
  \item \Expr{p} refers to the entire directory portion of \Expr{filename},
with a trailing slash or backslash character.
Whether a slash or backslash is used depends on the platform of
the machine.
The slash will be recognized on Linux platforms;
either a slash or backslash will be recognized on Windows platforms,
and the parser will use the same character specified.
  \item \Expr{d} refers to the last portion of the directory within the path,
if specified.
It will have a trailing slash or backslash, 
as appropriate to the platform of the machine.
The slash will be recognized on Linux platforms;
either a slash or backslash will be recognized on Windows platforms,
and the parser will use the same character specified.
  \item \Expr{n} refers to the file name at the end of any path,
but without any file name extension.
As an  example, the return value from \Expr{\$Fn(/tmp/simulate.exe)}
will be \Expr{simulate} (without the \File{.exe} extension).
  \item \Expr{x} refers to a file name extension, but without the associated
period (\Expr{.}).
As an  example, the return value from \Expr{\$Fn(/tmp/simulate.exe)}
will be \Expr{exe}.
  \item \Expr{q} causes the return value to be enclosed within double
quote marks.
\end{itemize}


\item [\Expr{\$INT(item-to-convert)}
       or \Expr{\$INT(item-to-convert, conversion-character)}]
Expands, evaluates, and returns a string version of 
\Expr{item-to-convert}.
The \Expr{conversion-character} is a C language conversion character.
If no \Expr{conversion-character} is specified, \verb@"%d"@ is used
as a conversion character.

\item [\Expr{\$RANDOM\_CHOICE(choice1, choice2, choice3, \Dots)}]
\index{\$RANDOM\_CHOICE() function macro}
A random choice of one of the parameters in the list of parameters is made.
For example,
if one of the integers 0-8 (inclusive) should be randomly
chosen:
\begin{verbatim}
  $RANDOM_CHOICE(0,1,2,3,4,5,6,7,8)
\end{verbatim}

\item [\Expr{\$RANDOM\_INTEGER(min, max [, step])}]
\index{\$RANDOM\_INTEGER()!in configuration}
A random integer within the range \verb@min@ and \verb@max@, inclusive,
is selected.
The optional \verb@step@ parameter
controls the stride within the range, and it defaults to the value 1.
For example, to randomly chose an even integer in the range 0-8 (inclusive):
\begin{verbatim}
  $RANDOM_INTEGER(0, 8, 2)
\end{verbatim}

\item [\Expr{\$REAL(item-to-convert)}
       or \Expr{\$REAL(item-to-convert, conversion-character)}]
Expands, evaluates, and returns a string version of 
\Expr{item-to-convert} for a floating point type.
The \Expr{conversion-character} is a C language conversion character.
If no \Expr{conversion-character} is specified, \verb@"%16G"@ is used
as a conversion character.

\item [\Expr{\$SUBSTR(name, start-index)}
       or \Expr{\$SUBSTR(name, start-index, length)}]
Expands \verb@name@ and returns a substring of it.
The first character of the string is at index 0.
The first character of the substring is at index \verb@start-index@.
If the optional \verb@length@ is not specified, 
then the substring includes characters up to the end of the string.
A negative value of  \verb@start-index@ works back from the end of the
string.
A negative value of \verb@length@ eliminates use of characters
from the end of the string.
Here are some examples that all assume
\begin{verbatim}
  Name = abcdef
\end{verbatim}
\begin{itemize}
  \item \Expr{\$SUBSTR(Name, 2)} is \Expr{cdef}.
  \item \Expr{\$SUBSTR(Name, 0, -2)} is \Expr{abcd}.
  \item \Expr{\$SUBSTR(Name, 1, 3)} is \Expr{bcd}.
  \item \Expr{\$SUBSTR(Name, -1)} is \Expr{f}.
  \item \Expr{\$SUBSTR(Name, 4, -3)} is the empty string, as there are
no characters in the substring for this request.
\end{itemize}

\end{description}


Environment references are not currently used in standard HTCondor
configurations.
However, they can sometimes be useful in custom configurations.

%%%%%%%%%%%%%%%%%%%%%%%%%%%%%%%%%%%%%%%%%%%%%%%%%%%%%%%%%%%%%%%%%%%%%%
\subsection{\label{sec:Macros-Requiring-Restart}Macros That Will Require a Restart When Changed}
%%%%%%%%%%%%%%%%%%%%%%%%%%%%%%%%%%%%%%%%%%%%%%%%%%%%%%%%%%%%%%%%%%%%%%
%TEMP -- we don't mention condor_reconfig before this...
\index{configuration change requiring a restart of HTCondor}
When any of the following listed configuration variables are changed,
HTCondor must be restarted.
Reconfiguration using \Condor{reconfig} will not be enough.

\begin{itemize}
  \item \verb@BIND_ALL_INTERFACES@
  \item \verb@FetchWorkDelay@
  \item \verb@MAX_NUM_CPUS@
  \item \verb@MAX_TRACKING_GID@
  \item \verb@MEMORY@
  \item \verb@MIN_TRACKING_GID@
  \item \verb@NETWORK_HOSTNAME@
  \item \verb@NETWORK_INTERFACE@
  \item \verb@NUM_CPUS@
  \item \verb@PREEMPTION_REQUIREMENTS_STABLE@
  \item \verb@PRIVSEP_ENABLED@
  \item \verb@PROCD_ADDRESS@
  \item \verb@SLOT_TYPE_<N>@
  \item \verb@OFFLINE_MACHINE_RESOURCE_<name>@
\end{itemize}

%%%%%%%%%%%%%%%%%%%%%%%%%%%%%%%%%%%%%%%%%%%%%%%%%%%%%%%%%%%%%%%%%%%%%%
\subsection{\label{sec:Pre-Defined-Macros}Pre-Defined Macros}
%%%%%%%%%%%%%%%%%%%%%%%%%%%%%%%%%%%%%%%%%%%%%%%%%%%%%%%%%%%%%%%%%%%%%%

\index{configuration!pre-defined macros}
\index{configuration file!pre-defined macros}
HTCondor provides pre-defined macros that help configure HTCondor.
Pre-defined macros are listed as \MacroUNI{macro\_name}.

This first set are entries whose values are determined at
run time and cannot be overwritten.  These are inserted automatically by
the library routine which parses the configuration files.
This implies that a change to the underlying value of any of these
variables will require a full restart of HTCondor in order to use
the changed value.
\begin{description}

\label{param:FullHostname}
\item[\MacroU{FULL\_HOSTNAME}]
  The fully qualified host name of the local machine,
  which is host name plus domain name.

\label{param:Hostname}
\item[\MacroU{HOSTNAME}]
  The host name of the local machine, \emph{without} a domain name.

\label{param:IpAddress}
\item[\MacroU{IP\_ADDRESS}]
  The ASCII string version of the local machine's ``most public'' IP address.
  This address may be IPv4 or IPv6, but the macro will always be set.

  HTCondor selects the ``most public'' address heuristically.  Your
  configuration should not depend on HTCondor picking any particular IP
  address for this macro; this macro's value may not even be one of the IP
  addresses HTCondor is configured to advertise.

label{param:IPv4Address}
\item[\MacroU{IPV4\_ADDRESS}]
  The ASCII string version of the local machine's ``most public'' IPv4
  address; unset if the local machine has no IPv4 address.

  See \MacroNI{IP\_ADDRESS} about ``most public''.

\label{param:IPv6Address}
\item[\MacroU{IPV6\_ADDRESS}]
  The ASCII string version of the local machine's ``most public'' IPv6
  address; unset if the local machine has no IPv6 address.

  See \MacroNI{IP\_ADDRESS} about ``most public''.

\label{param:IpAddressIsV6}
\item[\MacroU{IP\_ADDRESS\_IS\_V6}]
  A boolean which is true if and only if \Macro{IP\_ADDRESS} is an IPv6
  address.  Useful for conditonal configuration.

\label{param:Tilde}
\item[\MacroU{TILDE}]
  The full path to the
  home directory of the Unix user \Login{condor}, if such a user exists on the
  local machine.

  \label{sec:HTCondor-Subsystem-Names}
  \index{configuration file!subsystem names}
\label{param:Subsystem}
\item[\MacroU{SUBSYSTEM}]
  The subsystem
  name of the daemon or tool that is evaluating the macro.
  This is a unique string which identifies a given daemon within the
  HTCondor system.  The possible subsystem names are:

  \index{subsystem names}
  \index{macro!subsystem names}
  \begin{itemize}
  \label{list:subsystem names}
  \item \verb@C_GAHP@
  \item \verb@C_GAHP_WORKER_THREAD@
  \item \verb@CKPT_SERVER@
  \item \verb@COLLECTOR@
  \item \verb@DBMSD@
  \item \verb@DEFRAG@
  \item \verb@EC2_GAHP@
  \item \verb@GANGLIAD@
  \item \verb@GCE_GAHP@
  \item \verb@GRIDMANAGER@
  \item \verb@HAD@
  \item \verb@HDFS@
  \item \verb@JOB_ROUTER@
  \item \verb@KBDD@ 
  \item \verb@LEASEMANAGER@
  \item \verb@MASTER@
  \item \verb@NEGOTIATOR@
  \item \verb@REPLICATION@
  \item \verb@ROOSTER@
  \item \verb@SCHEDD@
  \item \verb@SHADOW@
  \item \verb@SHARED_PORT@
  \item \verb@STARTD@
  \item \verb@STARTER@
  \item \verb@SUBMIT@
  \item \verb@TOOL@
  \item \verb@TRANSFERER@
  \end{itemize}

\label{param:DetectedCpus}
\item[\MacroU{DETECTED\_CPUS}]
  The integer number of hyper-threaded CPUs, as given by
  \MacroUNI{DETECTED\_CORES}, when \MacroNI{COUNT\_HYPERTHREAD\_CPUS}
  is \Expr{True}.
  The integer number of physical (non hyper-threaded) CPUs, as given by
  \MacroUNI{DETECTED\_PHYSICAL\_CPUS}, when \Macro{COUNT\_HYPERTHREAD\_CPUS}
  is \Expr{False}.
  When \MacroNI{COUNT\_HYPERTHREAD\_CPUS}
  is \Expr{True}.

\label{param:DetectedPhysicalCpus}
\item[\MacroU{DETECTED\_PHYSICAL\_CPUS}]
  The integer number of physical (non hyper-threaded) CPUs.
  This will be equal the number of unique CPU IDs.

\end{description}

This second set of macros are entries whose default values are
determined automatically at run time but which can be overwritten.  
\index{configuration file!macros}
\begin{description}

\label{param:Arch}
\item[\MacroU{ARCH}]
  Defines the string
  used to identify the architecture of the local machine to HTCondor.
  The \Condor{startd} will advertise itself with this attribute so
  that users can submit binaries compiled for a given platform and
  force them to run on the correct machines.  \Condor{submit} will
  append a requirement to the job ClassAd that it must
  run on the same \MacroNI{ARCH} and \MacroNI{OPSYS} of the machine where
  it was submitted, unless the user specifies \MacroNI{ARCH} and/or
  \MacroNI{OPSYS} explicitly in their submit file.
  See the \Condor{submit} manual page
  on page~\pageref{man-condor-submit} for details.

\label{param:OpSys}
\item[\MacroU{OPSYS}]
  Defines the string used to identify the operating system
  of the local machine to HTCondor.
  If it is not defined in the configuration file, HTCondor will
  automatically insert the operating system of this machine as
  determined by \Prog{uname}.

\label{param:OpSysVer}
\item[\MacroU{OPSYS\_VER}]
  Defines the integer used to identify the operating system version number.

\label{param:OpSysAndVer}
\item[\MacroU{OPSYS\_AND\_VER}]
  Defines the string used prior to HTCondor version 7.7.2 as \MacroUNI{OPSYS}.

\label{param:UnameArch}
\item[\MacroU{UNAME\_ARCH}]
  The architecture as reported by \Prog{uname}(2)'s \Code{machine} field.
  Always the same as \MacroNI{ARCH} on Windows.

\label{param:UnameOpsys}
\item[\MacroU{UNAME\_OPSYS}]
  The operating system as reported by \Prog{uname}(2)'s \Code{sysname} field.
  Always the same as \MacroNI{OPSYS} on Windows.

\label{param:DetectedMemory}
\item[\MacroU{DETECTED\_MEMORY}]
  The amount of detected physical memory (RAM) in MiB.

\label{param:DetectedCores}
\item[\MacroU{DETECTED\_CORES}]
  The number of CPU cores that the operating system schedules.
  On machines that support hyper-threading, this will be 
  the number of hyper-threads.

\label{param:Pid}
\item[\MacroU{PID}]
  The process ID for the daemon or tool.

\label{param:Ppid}
\item[\MacroU{PPID}]
  The process ID of the parent process for the daemon or tool.

\label{param:Username}
\item[\MacroU{USERNAME}]
  The user name of the UID of the daemon or tool.
  For daemons started as root, but running under another UID
  (typically the user \Login{condor}), this will be the other UID.

\item[\MacroU{FILESYSTEM\_DOMAIN}]
  Defaults to the fully
  qualified host name of the machine it is evaluated on.  See
  section~\ref{sec:Shared-Filesystem-Config-File-Entries}, Shared
  File System Configuration File Entries for the full description of
  its use and under what conditions it could be desirable to change it.

\label{param:UIDDomain}
\item[\MacroU{UID\_DOMAIN}]
  Defaults to the fully
  qualified host name of the machine it is evaluated on.  See
  section~\ref{sec:Shared-Filesystem-Config-File-Entries} 
  for the full description of this configuration variable.

\end{description}

Since \MacroUNI{ARCH} and \MacroUNI{OPSYS} will automatically be set to the
correct values, we recommend that you do not overwrite them.
