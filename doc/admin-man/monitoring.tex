%%%%%%%%%%%%%%%%%%%%%%%%%%%%%%%%%%%%%%%%%%%%%%%%%%%%%%%%%%%%%%%%%%%%%%
\subsection{\label{sec:Monitoring}Monitoring}
%%%%%%%%%%%%%%%%%%%%%%%%%%%%%%%%%%%%%%%%%%%%%%%%%%%%%%%%%%%%%%%%%%%%%%
\index{pool management!Monitoring}
\index{Monitoring}

The \Condor{collector} collects information from all the other Condor daemons
in the pool.
Each daemon periodically updates the collector with a ClassAd.
The condor\_status command queries the \Condor{collector} to obtain a
snapshot of the current state of the pool.
However, monitoring systems track the state over time and optionally
alert the system administrator about exceptional conditions.

The most prevalent monitoring system on the Open Science Grid is ganglia.
(http://ganglia.info/)
Nagios is often used to provide alerts based on data from the ganglia
monitoring system.
(http://www.nagios.org/)
\Condor{gangliad} provides a efficient way to take information from
\Condor{collector} and supply it to the ganglia monitoring system.


%%%%%%%%%%%%%%%%%%%%%%%%%%%%%%%%%%%%%%%%%%%%%%%%%%%%%%%%%%%%%%%%%%%%%%
\subsubsection{\label{sec:monitor-ganglia}Ganglia}
%%%%%%%%%%%%%%%%%%%%%%%%%%%%%%%%%%%%%%%%%%%%%%%%%%%%%%%%%%%%%%%%%%%%%%
\index{Monitoring!with Ganglia}
\index{Ganglia monitoring}

In the past, system adminstrators have written scripts to extract data
using the condor\_status command and propogate the data to their local
monitoring system.

The \Condor{gangliad} streamlines getting data to the ganglia monitoring
system.
Updates to ganglia are done using the ganglia shared libraries for
efficiency.

If ganglia is already deployed in your pool, monitoring HTCondor can
be a simple as adding GANGLIAD to your DAEMON\_LIST on the host running
the ganglia gmetad.
If your pool does not use ganglia, it can still be monitored by a separate
server running ganglia.

The first decision is where to run the \Condor{gangliad}.
Running the \Condor{gangliad} on the host where your gmetad is running,
involves the least amount of configuration.
The default configuration works in this case.

It may be advantageous to run the \Condor{gangliad} on the \Condor{collector}
host.
On a large pool, there will be a large number of ClassAds.
This may help cut down on network traffic.
However, you may need to modify the \Macro{GANGLIA\_GSTAT\_COMMAND} to get the
list of monitored hosts from the master gmond.

By default, the \Condor{gangliad} will only propogate metrics to hosts
that are already monitoring by ganglia.
However, if you setup a ganglia host to monitor a pool not monitored by ganglia
or have a heterogenous pool where some hosts are not monitored, then you will
have to set \Macro{GANGLIA\_SEND\_DATA\_FOR\_ALL\_HOSTS} to \Expr{True}.
In this case, default graphs that ganglia provides will not be present.
However, the HTCondor metrics will appear.

On large pools, setting \Macro{GANGLIAD\_PER\_EXECUTE\_NODE\_METRICS} to false
will reduce the amount of data sent to ganglia.
The execute node data is the least important to monitor.

One can also limit the amount of data by setting the
\Macro{GANGLIAD\_REQUIREMNTS}.
Be aware, that aggregate sums over the entire pool will not be accurate
if you limit the ClassAd queried with a requirements expression.

An example of a metric definition:

\begin{verbatim}
[
  Name   = "JobsSubmitted";
  Desc   = "Number of jobs submitted";
  Units  = "jobs";
  TargetType = "Scheduler";
]
\end{verbatim}

  This file containing the default metrics is a good source of
  additional examples: 
  \File{\$(GANGLIAD\_METRICS\_CONFIG\_DIR)/00\_default\_metrics}.

  The recognized metric attributes are the following:

  \begin{description}

  \item[Name] The name of this metric.  Metrics published for the same
    machine must have unique names.

  \item[Value] A ClassAd expression that produces the value of this
    metric.  The default value is the value in the daemon ClassAd of the
    attribute with the same name as this metric.

  \item[Desc] A description of the metric.  This is displayed when the
    user holds the mouse over the metric's graph.

  \item[Verbosity] The integer verbosity level of this metric.  Metrics
    with a higher verbosity level than \Macro{GANGLIA\_VERBOSITY} will
    not be published.

  \item[TargetType] A string containing a comma-separated list of daemon
    ClassAd types this metric monitors.  The specified values should
    match \Attr{MyType} of the daemon ClassAd.  In addition, there are
    special values that may be included. \verb|"Machine_slot1"| may be
    specified to monitor the machine ClassAd for slot 1 only.  This is
    useful when monitoring machine-wide attributes.  The special
    value \verb|"ANY"| matches any type of ClassAd.

  \item[Requirements] If specified, this may be used to restrict which
    daemon ClassAds are monitored by this metric.  By default, all daemon
    ClassAds of the specified type are monitored.

  \item[Title] The graph title for this metric.  The default is the
    metric name.

  \item[Group] A string specifying the name of this metric's group.
    Metrics are arranged by group in the Ganglia web page.  The
    default is determined by the daemon type.  Metrics in different
    groups must still have unique names.

  \item[Cluster] A string specifying the cluster name for this metric.
    The default cluster name is taken from the configuration variable
    \Macro{GANGLIAD\_DEFAULT\_CLUSTER}.

  \item[Units] A string describing the units of this metric.

  \item[Scale] A scale factor. The value is multiplied by this amount.
    The scale factor is used when the value is not in the basic unit
    or human interpretable unit. For example, duty cycle is commonly
    expressed as a percent, but the HTCondor value ranges from 0 to 1.
    So, duty cycle is scaled by 100. Some metrics are reported in kbytes.
    Scaling by 1024 allows ganglia to pick the appropriate units (M bytes
    rather than k kbytes). When scaling by large numbers, converting to
    the \verb|"float"| type is recommended.

  \item[Derivative] A boolean value that specifies if Ganglia should
    graph the derivative of this metric.  Ganglia versions prior to
    3.4 do not support this.

  \item[Type] A string specifying the type of the metric.  Possible
    values are \verb|"double"|, \verb|"float"|, \verb|"int32"|,
    \verb|"uint32"|, \verb|"int16"|, \verb|"uint16"|,
    \verb|"int8"|, \verb|"uint8"|, and \verb|"string"|.
    The default is \verb|"string"| for string values,
    The default is \verb|"int32"| for integer values,
    The default is \verb|"float"| for real values,
    The default is \verb|"int8"| for boolean values.
    Integer values can be coerced to \verb|"float"| or \verb|"double"|.
    This is especially important for values stored internally as 64-bit
    values.

  \item[Regex] This string value specifies a regular expression that
    matches attributes to be monitored by this metric.  This is useful
    for dynamic attributes that cannot be enumerated in advance,
    because their names depend on dynamic information such as the
    users who are currently running jobs.  When this is specified, one
    metric per matching attribute is created.  The default metric name
    is the name of the matched attribute, and the default value is the
    value of that attribute.  As usual, the \Attr{Value} expression
    may be used when the raw attribute value needs to be manipulated
    before publication.  However, since the name of the attribute is
    not known in advance, a special ClassAd attribute in the daemon ClassAd
    is provided to allow the \Attr{Value} expression to refer to it.
    This special attribute is named \Attr{Regex}.  Another special
    feature is the ability to refer to text matched by regular
    expression groups defined by parentheses within the regular
    expression.  These may be substituted into the values of other
    string attributes such as \Attr{Name} and \Attr{Desc}.  This is
    done by putting macros in the string values.  \verb|"\\1"| is
    replaced by the first group, \verb|"\\2"| by the second group, and
    so on.

  \item[Aggregate] This string value specifies an aggregation function
    to apply, instead of publishing individual metrics for each daemon
    ClassAd.  Possible values are \verb|"sum"|, \verb|"avg"|, \verb|"max"|,
    and \verb|"min"|.

  \item[AggregateGroup] When an aggregate function has been specified,
    this string value specifies which aggregation group the current
    daemon ClassAd belongs to.  The default is the metric name.  This
    feature works like GROUP BY in SQL.  The aggregation function
    produces one result per value of \Attr{AggregateGroup}.  A single
    aggregate group would therefore be appropriate for a pool-wide
    metric.  Example use of aggregate grouping: if one wished to
    publish the sum of an attribute across different types of slot
    ClassAds, one could make the metric name an expression that is unique
    to each type.  The default \Attr{AggregateGroup} would be set
    accordingly.  Note that the assumption is still that the result
    is a pool-wide metric, so by default it is associated with the
    \Condor{collector} daemon's host.
    To group by machine and publish the result into
    the Ganglia page associated with each machine, one should make
    the \Attr{AggregateGroup} contain the machine name and override
    the default \Attr{Machine} attribute to be the daemon's machine
    name, rather than the \Condor{collector}'s.

  \item[Machine] The name of the host associated with this metric.  If
    \Macro{GANGLIAD\_DEFAULT\_MACHINE} is not specified, the default
    is taken from the \Attr{Machine} attribute of the daemon ClassAd.
    If the daemon name is of the form \verb|name@hostname|, this may
    indicate that there are multiple instances of HTCondor running on
    the same machine.  To avoid the metrics from these instances
    overwriting each other, the default machine name is set to the
    daemon name in this case.  For aggregate metrics, the default
    value of \Attr{Machine} will be the name of the \Condor{collector} host.

  \item[IP] A string containing the IP address of the host associated
    with this metric.  If \Macro{GANGLIAD\_DEFAULT\_IP} is not
    specified, the default is extracted from the \Attr{MyAddress}
    attribute of the daemon ClassAd.  This value must be unique for each
    machine published to Ganglia.  It need not be a valid IP address.
    If the value of \Attr{Machine} contains an \verb|"@"| sign, the
    default IP value will be set to the same value as \Attr{Machine}
    in order to make the IP value unique to each instance of HTCondor
    running on the same host.

  \end{description}
