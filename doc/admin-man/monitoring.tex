%%%%%%%%%%%%%%%%%%%%%%%%%%%%%%%%%%%%%%%%%%%%%%%%%%%%%%%%%%%%%%%%%%%%%%
\subsection{\label{sec:Monitoring}Monitoring}
%%%%%%%%%%%%%%%%%%%%%%%%%%%%%%%%%%%%%%%%%%%%%%%%%%%%%%%%%%%%%%%%%%%%%%
\index{pool management!monitoring}
\index{monitoring pools}
\index{pool monitoring}

Information that
the \Condor{collector} collects can be used to monitor a pool.
The \Condor{status} command can be used to display
snapshot of the current state of the pool.
Monitoring systems can be set up to track the state over time,
and they might go further, 
to alert the system administrator about exceptional conditions.

Prevalent in the Open Science Grid is the Ganglia monitoring system
(\URL{http://ganglia.info/}).
Nagios (\URL{http://www.nagios.org/})
is often used to provide alerts based on data from the Ganglia
monitoring system.
The \Condor{gangliad} daemon provides an efficient way to take information from
an HTCondor pool and supply it to the Ganglia monitoring system.


%%%%%%%%%%%%%%%%%%%%%%%%%%%%%%%%%%%%%%%%%%%%%%%%%%%%%%%%%%%%%%%%%%%%%%
\subsubsection{\label{sec:monitor-ganglia}Ganglia}
%%%%%%%%%%%%%%%%%%%%%%%%%%%%%%%%%%%%%%%%%%%%%%%%%%%%%%%%%%%%%%%%%%%%%%
\index{Monitoring!with Ganglia}
\index{Ganglia monitoring}

Support for Ganglia monitoring is integral to HTCondor.
The \Condor{gangliad} gathers up data as specified by its configuration,
and it streamlines getting that data to the Ganglia monitoring
system.
Updates sent to Ganglia are done using the Ganglia shared libraries for
efficiency.

If Ganglia is already deployed in the pool,
the monitoring of HTCondor is enabled by running the \Condor{gangliad} daemon
on a single machine within the pool.
If the machine chosen is the one running Ganglia's \Prog{gmetad},
then the HTCondor configuration consists of
adding \Expr{GANGLIAD} to the definition of configuration
variable \MacroNI{DAEMON\_LIST} on that machine.
It may be advantageous to run the \Condor{gangliad} daemon
on the same machine as is running the \Condor{collector} daemon,
because on a large pool with many ClassAds,
there is likely to be less network traffic.
If the \Condor{gangliad} daemon is to run on a different machine
than the one running Ganglia's \Prog{gmetad},
modify configuration variable \Macro{GANGLIA\_GSTAT\_COMMAND} to get the
list of monitored hosts from the master \Prog{gmond} program.

If the pool does not use Ganglia, 
the pool can still be monitored by a separate server running Ganglia.

By default, the \Condor{gangliad} will only propagate metrics to hosts
that are already monitored by Ganglia.
Set configuration variable \Macro{GANGLIA\_SEND\_DATA\_FOR\_ALL\_HOSTS} 
to \Expr{True} to set up
a Ganglia host to monitor a pool not monitored by Ganglia
or have a heterogeneous pool where some hosts are not monitored.
In this case, default graphs that Ganglia provides will not be present.
However, the HTCondor metrics will appear.

On large pools, 
setting configuration variable \Macro{GANGLIAD\_PER\_EXECUTE\_NODE\_METRICS}
to \Expr{False} will reduce the amount of data sent to Ganglia.
The execute node data is the least important to monitor.
One can also limit the amount of data by setting configuration variable
\Macro{GANGLIAD\_REQUIREMENTS}.
Be aware that aggregate sums over the entire pool will not be accurate
if this variable limits the ClassAds queried.

Metrics to be sent to Ganglia are specified in all files within the
directory specified by configuration variable 
\Macro{GANGLIAD\_METRICS\_CONFIG\_DIR}.
Each file in the directory is read,
and the format within each file is that of New ClassAds.
Here is an example of a single metric definition given as a New ClassAd:

\begin{verbatim}
[
  Name   = "JobsSubmitted";
  Desc   = "Number of jobs submitted";
  Units  = "jobs";
  TargetType = "Scheduler";
]
\end{verbatim}

A nice set of default metrics is in file:
\File{\$(GANGLIAD\_METRICS\_CONFIG\_DIR)/00\_default\_metrics}.

Recognized metric attribute names and their use:

  \begin{description}

  \item[Name] The name of this metric,  
    which corresponds to the ClassAd attribute name.
    Metrics published for the same machine must have unique names.

  \item[Value] A ClassAd expression that produces the value when evaluated.
    The default value is the value in the daemon ClassAd of the
    attribute with the same name as this metric.

  \item[Desc] A brief description of the metric.  This string is displayed 
    when the user holds the mouse over the Ganglia graph for the metric.

  \item[Verbosity] The integer verbosity level of this metric.  
    Metrics with a higher verbosity level than that specified by
    configuration variable \Macro{GANGLIA\_VERBOSITY} will not be published.

  \item[TargetType] A string containing a comma-separated list of daemon
    ClassAd types that this metric monitors.  The specified values should
    match the value of \Attr{MyType} of the daemon ClassAd.  
    In addition, there are
    special values that may be included. \verb|"Machine_slot1"| may be
    specified to monitor the machine ClassAd for slot 1 only.  This is
    useful when monitoring machine-wide attributes.  The special
    value \verb|"ANY"| matches any type of ClassAd.

  \item[Requirements] A boolean expression that may restrict how this
    metric is incorporated.  It defaults to \Expr{True}, which places
    no restrictions on the collection of this ClassAd metric.

  \item[Title] The graph title used for this metric.  The default is the
    metric name.

  \item[Group] A string specifying the name of this metric's group.
    Metrics are arranged by group within a Ganglia web page.  The
    default is determined by the daemon type.  Metrics in different
    groups must have unique names.

  \item[Cluster] A string specifying the cluster name for this metric.
    The default cluster name is taken from the configuration variable
    \Macro{GANGLIAD\_DEFAULT\_CLUSTER}.

  \item[Units] A string describing the units of this metric.

  \item[Scale] A scaling factor that is multiplied by the value of the
    \Attr{Value} attribute.
    The scale factor is used when the value is not in the basic unit
    or a human-interpretable unit. For example, duty cycle is commonly
    expressed as a percent, but the HTCondor value ranges from 0 to 1.
    So, duty cycle is scaled by 100. Some metrics are reported in Kbytes.
    Scaling by 1024 allows Ganglia to pick the appropriate units,
    such as number of bytes rather than number of Kbytes. 
    When scaling by large values, converting to
    the \verb|"float"| type is recommended.

  \item[Derivative] A boolean value that specifies if Ganglia should
    graph the derivative of this metric.  Ganglia versions prior to
    3.4 do not support this.

  \item[Type] A string specifying the type of the metric.  Possible
    values are \verb|"double"|, \verb|"float"|, \verb|"int32"|,
    \verb|"uint32"|, \verb|"int16"|, \verb|"uint16"|,
    \verb|"int8"|, \verb|"uint8"|, and \verb|"string"|.
    The default is \verb|"string"| for string values,
    The default is \verb|"int32"| for integer values,
    The default is \verb|"float"| for real values,
    The default is \verb|"int8"| for boolean values.
    Integer values can be coerced to \verb|"float"| or \verb|"double"|.
    This is especially important for values stored internally as 64-bit
    values.

  \item[Regex] This string value specifies a regular expression that
    matches attributes to be monitored by this metric.  This is useful
    for dynamic attributes that cannot be enumerated in advance,
    because their names depend on dynamic information such as the
    users who are currently running jobs.  When this is specified, one
    metric per matching attribute is created.  The default metric name
    is the name of the matched attribute, and the default value is the
    value of that attribute.  As usual, the \Attr{Value} expression
    may be used when the raw attribute value needs to be manipulated
    before publication.  However, since the name of the attribute is
    not known in advance, a special ClassAd attribute in the daemon ClassAd
    is provided to allow the \Attr{Value} expression to refer to it.
    This special attribute is named \Attr{Regex}.  Another special
    feature is the ability to refer to text matched by regular
    expression groups defined by parentheses within the regular
    expression.  These may be substituted into the values of other
    string attributes such as \Attr{Name} and \Attr{Desc}.  This is
    done by putting macros in the string values.  \verb|"\\1"| is
    replaced by the first group, \verb|"\\2"| by the second group, and
    so on.

  \item[Aggregate] This string value specifies an aggregation function
    to apply, instead of publishing individual metrics for each daemon
    ClassAd.  Possible values are \verb|"sum"|, \verb|"avg"|, \verb|"max"|,
    and \verb|"min"|.

  \item[AggregateGroup] When an aggregate function has been specified,
    this string value specifies which aggregation group the current
    daemon ClassAd belongs to.  The default is the metric \Attr{Name}.  This
    feature works like GROUP BY in SQL.  The aggregation function
    produces one result per value of \Attr{AggregateGroup}.  A single
    aggregate group would therefore be appropriate for a pool-wide
    metric.  Example use of aggregate grouping: if one wished to
    publish the sum of an attribute across different types of slot
    ClassAds, one could make the metric name an expression that is unique
    to each type.  The default \Attr{AggregateGroup} would be set
    accordingly.  Note that the assumption is still that the result
    is a pool-wide metric, so by default it is associated with the
    \Condor{collector} daemon's host.
    To group by machine and publish the result into
    the Ganglia page associated with each machine, one should make
    the \Attr{AggregateGroup} contain the machine name and override
    the default \Attr{Machine} attribute to be the daemon's machine
    name, rather than the \Condor{collector} daemon's.

  \item[Machine] The name of the host associated with this metric.  
    If configuration variable
    \Macro{GANGLIAD\_DEFAULT\_MACHINE} is not specified, 
    the default
    is taken from the \Attr{Machine} attribute of the daemon ClassAd.
    If the daemon name is of the form \verb|name@hostname|, this may
    indicate that there are multiple instances of HTCondor running on
    the same machine.  To avoid the metrics from these instances
    overwriting each other, the default machine name is set to the
    daemon name in this case.  For aggregate metrics, the default
    value of \Attr{Machine} will be the name of the \Condor{collector} host.

  \item[IP] A string containing the IP address of the host associated
    with this metric.  If \Macro{GANGLIAD\_DEFAULT\_IP} is not
    specified, the default is extracted from the \Attr{MyAddress}
    attribute of the daemon ClassAd.  This value must be unique for each
    machine published to Ganglia.  It need not be a valid IP address.
    If the value of \Attr{Machine} contains an \verb|"@"| sign, the
    default IP value will be set to the same value as \Attr{Machine}
    in order to make the IP value unique to each instance of HTCondor
    running on the same host.

  \end{description}
