%%%%%%%%%%%%%%%%%%%%%%%%%%%%%%%%%%%%%%%%%%%%%%%%%%%%%%%%%%%%%%%%%%%%%%
\subsection{\label{sec:EventD}
Condor Event Daemon}
%%%%%%%%%%%%%%%%%%%%%%%%%%%%%%%%%%%%%%%%%%%%%%%%%%%%%%%%%%%%%%%%%%%%%%

The Event Daemon is an administrative tool for scheduling events in a
Condor pool.
Every \Macro{EVENTD\_INTERVAL}, for each defined event, the event
daemon computes an estimate of the time required to complete or
prepare for the event.  If the time required is less than the time
between the next interval and the start of the event, the event daemon
activates the event.

Currently, this daemon supports SHUTDOWN events, which place machines
in the ``Owner'' state during scheduled times.
The daemon vacates jobs one at a time in anticipation of SHUTDOWN
events.
To determine the estimate of the time required to complete a SHUTDOWN
event, the ImageSize values for all running standard universe jobs are
totalled and then divided by the maximum bandwidth specified for this
event.
When a SHUTDOWN event is activated, the eventd contacts all startds
which match the given constraint and places them in ``shutdown'' mode.
All startds which are not running jobs will immediately transition to
the ``Owner'' state.
The eventd then sends a ``vacate'' command to the each startd in
the list which is currently running a job.
Once the job is vacated, the startd will immediately transition to the
``Owner'' state.

The \Condor{eventd} must run on a machine with ``administrator''
access to your pool.
See section~\ref{sec:Host-Security} on
page~\pageref{sec:Host-Security} for full details about IP/host-based
security in Condor.

%%%%%%%%%%%%%%%%%%%%%%%%%%%%%%%%%%%%%%%%%%%%%%%%%%%%%%%%%%%%%%%%%%%%%%
\subsubsection{\label{sec:EventD-Installation}
Installing the Event Daemon} 
%%%%%%%%%%%%%%%%%%%%%%%%%%%%%%%%%%%%%%%%%%%%%%%%%%%%%%%%%%%%%%%%%%%%%%

The \Condor{eventd} requires version 6.1.3 or later of the
\Condor{startd}.
So, you should first install either the latest version of the SMP
\Condor{startd} contrib module or the latest release of Condor version
6.1.

Then, download the \Condor{eventd} contrib module.
When you uncompress and untar the file, you'll have a directory that
contains a \File{eventd.tar}.
The \File{eventd.tar} acts much like the \File{release.tar} file from
a main release.
This archive contains these files:
\begin{verbatim}
	sbin/condor_eventd
	etc/examples/condor_config.local.eventd
\end{verbatim}
These are all new files, not found in the main release, so you can
safely untar the archive directly into your existing release
directory.
\File{\condor{eventd}} is the eventd binary.
The example config file is described below.

%%%%%%%%%%%%%%%%%%%%%%%%%%%%%%%%%%%%%%%%%%%%%%%%%%%%%%%%%%%%%%%%%%%%%%
\subsubsection{\label{sec:EventD-Configuration}
Configuring the Event Daemon} 
%%%%%%%%%%%%%%%%%%%%%%%%%%%%%%%%%%%%%%%%%%%%%%%%%%%%%%%%%%%%%%%%%%%%%%

The file \File{etc/examples/condor\_config.local.eventd} contains an
example \Condor{eventd} configuration.
To define events, you must first set the \Macro{EVENT\_LIST} macro.
This macro contains a list of macro names which define the individual
events.
The definition of individual events depends on the type of the event.
Currently, there is only one event type: SHUTDOWN.
The format for SHUTDOWN events is:
\begin{verbatim}
	SHUTDOWN DAY TIME DURATION BANDWIDTH CONSTRAINT RANK
\end{verbatim}
TIME and DURATION are specified in an hours:minutes format.
For example:
\begin{verbatim}
EVENT_LIST	= TestEvent, TestEvent2
TestEvent	= SHUTDOWN W 16:00 1:00 2.5 TestEventConstraint TestEventRank
TestEvent2	= SHUTDOWN F 14:00 0:30 6.0 TestEventConstraint2 TestEventRank
TestEventConstraint		= (Arch == "INTEL")
TestEventConstraint2		= (True)
TestEventRank			= (0 - ImageSize)
\end{verbatim}
In this example, the ``TestEvent'' is a SHUTDOWN type event, which
specifies that all machines whose startd ads match the constraint
(Arch == ``INTEL'') should be shutdown for one hour starting at
16:00 every Wednesday, and no more than 2.5 Mb/s of bandwidth
should be used to vacate jobs in anticipation of the shutdown
event.  According to the TestEventRank, jobs will be vacated in
reverse order of their ImageSize (larger jobs first, smaller jobs
last).  ``TestEvent2'' is a SHUTDOWN type event, which specifies
that all machines should be shutdown for 30 minutes starting at
14:00 every Friday, and no more than 6.0 Mb/s of bandwidth should
be used to vacate jobs in anticipation of the shutdown event.

Note that the \Macro{DAEMON\_LIST} macro (Described in
section~\ref{sec:Master-Config-File-Entries}) is defined in the
section of settings you may want to customize.
If you want the event daemon managed by the \Condor{master}, the
\Macro{DAEMON\_LIST} entry must contain both MASTER and EVENTD.
You should verify that this macro is set to run the correct daemons on
this machine.  By default, the list also includes SCHEDD and STARTD.

See section~\ref{sec:Eventd-Config-File-Entries} on
page~\pageref{sec:Eventd-Config-File-Entries} for a description of
optional event daemon parameters.

%%%%%%%%%%%%%%%%%%%%%%%%%%%%%%%%%%%%%%%%%%%%%%%%%%%%%%%%%%%%%%%%%%%%%%
\subsubsection{\label{sec:Spawn-EventD} 
Spawning the Event Daemon} 
%%%%%%%%%%%%%%%%%%%%%%%%%%%%%%%%%%%%%%%%%%%%%%%%%%%%%%%%%%%%%%%%%%%%%%

To spawn a event daemon once it is configured to run on a given
machine, all you have to do is restart Condor on that host to enable
the \Condor{master} to notice the new configuration.
You can do this by sending a \Condor{restart} command from any machine
with ``administrator'' access to your pool.
See section~\ref{sec:Host-Security} on
page~\pageref{sec:Host-Security} for full details about IP/host-based
security in Condor.

