%%%%%%%%%%%%%%%%%%%%%%%%%%%%%%%%%%%%%%%%%%%%%%%%%%%%%%%%%%%%%%%%%%%%%%
\subsection{\label{sec:EventD}
Installing the Condor Event Daemon}
%%%%%%%%%%%%%%%%%%%%%%%%%%%%%%%%%%%%%%%%%%%%%%%%%%%%%%%%%%%%%%%%%%%%%%

\index{daemon!eventd}
\index{event daemon}
\index{contrib module!event daemon}

The event daemon is an administrative tool for scheduling events in a
Condor pool.
Every \Macro{EVENTD\_INTERVAL}, for each defined event, the event
daemon (eventd) computes an estimate of the time required to complete or
prepare for the event.  If the time required is less than the time
between the next interval and the start of the event, the event daemon
activates the event.

Currently, this daemon supports \Macro{SHUTDOWN} events, which place machines
in the owner state during scheduled times.
The eventd causes machines to vacate jobs in an orderly fashion
in anticipation of \MacroNI{SHUTDOWN} events.
Scheduling this improves performance, because the machines
do not all attempt to checkpoint their jobs at the same time.
Instead, the eventd schedules checkpoint transfers according to
bandwidth limits defined in its configuration files.

When a \MacroNI{SHUTDOWN} event is activated, the eventd contacts all startd
daemons
that match constraints given in the configuration file,
and instructs them to shut down.
In response to this instruction,
the startd on any machine not running a job will immediately transition to
the owner state.
Any machine currently running a job will continue to run the
job, but will not start any new job.
The eventd then sends a vacate command to each startd
that is currently running a job.
Once the job is vacated, the startd transitions to the
owner state and remains in this state for the duration of the
\MacroNI{SHUTDOWN} event.

The \Condor{eventd} must run on a machine with \Macro{STARTD}
\Macro{ADMINISTRATOR} and \Macro{CONFIG} access to your pool.
See section~\ref{sec:Host-Security} on
page~\pageref{sec:Host-Security} for full details about IP/host-based
security in Condor.

%%%%%%%%%%%%%%%%%%%%%%%%%%%%%%%%%%%%%%%%%%%%%%%%%%%%%%%%%%%%%%%%%%%%%%
\subsubsection{\label{sec:EventD-Installation}
Installing the Event Daemon} 
%%%%%%%%%%%%%%%%%%%%%%%%%%%%%%%%%%%%%%%%%%%%%%%%%%%%%%%%%%%%%%%%%%%%%%

First, download the \Condor{eventd} contrib module.
Uncompress and untar the file, to have a directory that
contains a \File{eventd.tar}.
The \File{eventd.tar} acts much like the \File{release.tar} file from
a main release.
This archive contains the files:
\begin{verbatim}
	sbin/condor_eventd
	etc/examples/condor_config.local.eventd
\end{verbatim}
These are all new files, not found in the main release, so you can
safely untar the archive directly into your existing release
directory.
The file \File{\condor{eventd}} is the eventd binary.
The example configuration file is described below.

%%%%%%%%%%%%%%%%%%%%%%%%%%%%%%%%%%%%%%%%%%%%%%%%%%%%%%%%%%%%%%%%%%%%%%
\subsubsection{\label{sec:EventD-Configuration}
Configuring the Event Daemon} 
%%%%%%%%%%%%%%%%%%%%%%%%%%%%%%%%%%%%%%%%%%%%%%%%%%%%%%%%%%%%%%%%%%%%%%

The file \File{etc/examples/condor\_config.local.eventd} contains an
example configuration.
To define events, first set the \Macro{EVENT\_LIST} macro.
This macro contains a list of macro names which define the individual
events.
The definition of individual events depends on the type of the event.
The format for \Macro{SHUTDOWN} events is
\begin{verbatim}
	SHUTDOWN DAY TIME DURATION CONSTRAINT
\end{verbatim}
\verb@TIME@ and \verb@DURATION@ are specified in an hours:minutes
format.  \verb@DAY@ is a string of days, where \verb@M@ = Monday,
\verb@T@ = Tuesday, \verb@W@ = Wednesday, \verb@R@ = Thursday,
\verb@F@ = Friday, \verb@S@ = Saturday, and \verb@U@ = Sunday.  For
example, \verb@MTWRFSU@ would specify that the event occurs daily,
\verb@MTWRF@ would specify that the event occurs only on weekdays, and
\verb@SU@ would specificy that the event occurs only on weekends.

Two options can be specified to change the default behavior of
\MacroNI{SHUTDOWN} events.
If \MacroNI{\_RUNTIME} is appended to the \MacroNI{SHUTDOWN} event
specification, the startd shutdown configurations will not be
persistent.  If a machine reboots or a startd is restarted, the startd
will no longer be ``shutdown'' and may transition out of the owner state.
This is useful for reboot events, where the startd should leave the
shutdown state when the machine reboots.
If \MacroNI{\_STANDARD} is appended to the \MacroNI{SHUTDOWN} event
specification, the eventd will only consider standard universe jobs.
It will vacate only standard universe jobs and configure machines to
run only non-standard universe jobs during the shutdown event.
This is also useful for reboot events, where there is no point
vacating vanilla jobs before the machine is rebooted because
they are unable to checkpoint.
Reboot events are usually listed as \Macro{SHUTDOWN\_RUNTIME\_STANDARD}.

The following is an example event daemon configuration:
\index{event daemon!example configuration}
\begin{verbatim}
EVENT_LIST	= TestEvent, TestEvent2
TestEvent	= SHUTDOWN_RUNTIME W 2:00 1:00 TestEventConstraint
TestEvent2	= SHUTDOWN MTWRF 14:00 0:30 TestEventConstraint2
TestEventConstraint		= (Arch == "INTEL")
TestEventConstraint2	= (True)
\end{verbatim}

In this example, the \verb@TestEvent@ is a \Macro{SHUTDOWN\_RUNTIME}
type event, which
specifies that all machines whose startd ads match the constraint
\verb@Arch == "INTEL"@ should be shutdown for one hour (or until the
\Condor{startd} daemon restarts) starting at 2:00 every Wednesday.
\verb@TestEvent2@ is a \MacroNI{SHUTDOWN} type event, which specifies
that all machines should be shutdown for 30 minutes starting at
14:00 on weekdays.

The bandwidth limits used in the eventd's schedule are specified in
the file indicated by the \Macro{EVENTD\_CAPACITY\_INFO} parameter, and
any network routing information required to implement those limits is
specified in the file indicated by the \Macro{EVENTD\_ROUTING\_INFO}
parameter.
The \MacroNI{EVENTD\_CAPACITY\_INFO} file has the same
format as the \Macro{NETWORK\_CAPACITY\_INFO} file, described in
section~\ref{sec:Bandwidth-Alloc-Capinfo}.
The \MacroNI{EVENTD\_ROUTING\_INFO} file has the same
format as the \Macro{NETWORK\_ROUTING\_INFO} file, described in
section~\ref{sec:Bandwidth-Alloc-Routes}.

Note that the \Macro{DAEMON\_LIST} macro (described in
section~\ref{sec:Master-Config-File-Entries}) is defined in the
section of settings you may want to customize.
If you want the event daemon managed by the \Condor{master}, the
\MacroNI{DAEMON\_LIST} entry must contain both 
\Attr{MASTER} and \Attr{EVENTD}.
Verify that this macro is set to run the correct daemons on
this machine.  By default, the list also includes
\Attr{SCHEDD} and \Attr{STARTD}.

See section~\ref{sec:Eventd-Config-File-Entries} on
page~\pageref{sec:Eventd-Config-File-Entries} for a description of
optional event daemon parameters.

%%%%%%%%%%%%%%%%%%%%%%%%%%%%%%%%%%%%%%%%%%%%%%%%%%%%%%%%%%%%%%%%%%%%%%
\subsubsection{\label{sec:Start-EventD} 
Starting the Event Daemon} 
%%%%%%%%%%%%%%%%%%%%%%%%%%%%%%%%%%%%%%%%%%%%%%%%%%%%%%%%%%%%%%%%%%%%%%

To start an event daemon once it is configured to run on a given
machine, restart Condor on that given machine to enable
the \Condor{master} to notice the new configuration.
Send a \Condor{restart} command from any machine
with administrator access to your pool.
See section~\ref{sec:Host-Security} on
page~\pageref{sec:Host-Security} for full details about IP/host-based
security in Condor.

