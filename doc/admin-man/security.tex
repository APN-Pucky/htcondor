%%%%%%%%%%%%%%%%%%%%%%%%%%%%%%%%%%%%%%%%%%%%%%%%%%%%%%%%%%%%%%%%%%%%%%%%%%%
\section{\label{sec:Security}Security}
%%%%%%%%%%%%%%%%%%%%%%%%%%%%%%%%%%%%%%%%%%%%%%%%%%%%%%%%%%%%%%%%%%%%%%%%%%%
\index{security!in HTCondor|(}

Security in HTCondor is a broad issue, with many aspects to consider.
Because HTCondor's main purpose is to allow users to run arbitrary code
on large numbers of computers, it is important to try to limit who can
access an HTCondor pool and what privileges they have when using the
pool.  This section covers these topics.

There is a distinction between the
kinds of resource attacks HTCondor can defeat,
and the kinds of attacks HTCondor cannot defeat.
HTCondor cannot
prevent security breaches of users that can elevate their privilege to
the root or administrator account.
HTCondor does not run user jobs in
sandboxes (standard universe jobs are a partial exception to this),
so HTCondor cannot defeat all malicious actions by user jobs.
An example of a malicious job is
one that launches a distributed denial of service attack.
HTCondor assumes that users are trustworthy.
HTCondor can prevent unauthorized access to the HTCondor pool,
to help ensure that only trusted users have access to the pool.
In addition, HTCondor provides encryption and
integrity checking, to ensure that data (both HTCondor's data and user
jobs' data) has not been examined or tampered with while in transit.

Broadly speaking, the aspects of security in
HTCondor may be categorized and described: 

\begin{description}

\item[Users] Authorization or capability in an operating system is
based on a process owner.
Both those that submit jobs and HTCondor daemons
become process owners.
The HTCondor system prefers that HTCondor daemons are run as the user
\Login{root}, while other common operations are owned by a
user of HTCondor.
Operations that do not belong to either \Login{root} or an HTCondor user
are often owned by the \Login{condor} user.
See Section~\ref{sec:uids} for more detail.

\item[Authentication] 
Proper identification of a user is accomplished by the
process of authentication.
It attempts to distinguish between real users and impostors.
By default, HTCondor's authentication uses the user id (UID)
to determine identity,
but HTCondor can choose among a variety of authentication mechanisms,
including the stronger  authentication methods Kerberos and GSI.

\item[Authorization] Authorization specifies who is allowed
to do what.
Some users are allowed to submit jobs,
while other users are allowed administrative privileges over HTCondor itself.
HTCondor provides authorization on either a per-user or on
a per-machine basis.

\item[Privacy] HTCondor may encrypt data sent across the network, which
prevents others from viewing the data.
With persistence and sufficient computing power,
decryption is possible. 
HTCondor can encrypt the data sent for internal communication, 
as well as user data, such as files and executables.
Encryption operates on network
transmissions: unencrypted data is stored on disk.

\item[Integrity] 
The  \Term{man-in-the-middle} attack tampers with data
without the awareness of either side of the communication.
HTCondor's integrity check sends additional cryptographic data
to verify that network data transmissions have not been
tampered with.
Note that the integrity information is only for network
transmissions: data stored on disk does not have this integrity
information. 

\end{description}


%%%%%%%%%%%%%%%%%%%%%%%%%%%%%%%%%%%%%%%%%%%%%%%%%%%%%%%%%%%%%%%%%%%%%%
\subsection{\label{sec:Config-Security}HTCondor's Security Model}
%%%%%%%%%%%%%%%%%%%%%%%%%%%%%%%%%%%%%%%%%%%%%%%%%%%%%%%%%%%%%%%%%%%%%%

At the heart of HTCondor's security model is the notion that 
communications are subject to various security checks. 
A request from one HTCondor daemon to another may require
authentication to prevent subversion of the system.
A request from a user of HTCondor may need to be denied
due to the confidential nature of the request.
The security model handles these example situations
and many more.

Requests to HTCondor are
categorized into groups of \Term{access levels},
based on the type of operation requested.
The user of a specific request must be authorized at
the required access level.
For example,
executing the \Condor{status} command requires
the \DCPerm{READ} access level.
Actions that accomplish management tasks,
such as shutting down or restarting of a daemon
require an \DCPerm{ADMINISTRATOR} access level.
See
Section~\ref{sec:Security-Authorization}
for a full list of HTCondor's access levels and their meanings.

There are two sides to any communication or command invocation in HTCondor.
One side is identified as the \Term{client},
and the other side is identified as the \Term{daemon}.
The client is the party that initiates the command,
and the daemon is the party that processes the command and responds.
In some cases it is easy to distinguish the
client from the daemon, while in other cases it is not as easy.
HTCondor tools such as \Condor{submit} and \Condor{config\_val} are clients.
They send commands to daemons and act as clients in all their communications.
For example, the \Condor{submit} command communicates
with the \Condor{schedd}.
Behind the scenes, HTCondor daemons also communicate with each other;
in this case the daemon initiating the command plays the role of the client.
For instance,
the \Condor{negotiator} daemon acts as a client when contacting the
\Condor{schedd} daemon to initiate matchmaking.
Once a match has been found,
the \Condor{schedd} daemon acts as a client and contacts the
\Condor{startd} daemon.

HTCondor's security model
is implemented using configuration.
Commands in HTCondor are executed over TCP/IP network connections.
While network communication enables HTCondor to manage
resources that are distributed across an organization (or beyond),
it also brings in security challenges.
HTCondor must have ways of ensuring that commands are being sent
by trustworthy users.
Jobs that are operating on sensitive data must be allowed to use
encryption such that the data is not seen by outsiders.
Jobs may need assurance that data has not been tampered with.
These issues
can be addressed with HTCondor's authentication,
encryption, and integrity features.


%%%%%%%%%%%%%%%%%%%%%%%%%%%%%%%%%%%%%%%%%%%%%%%%%%%%%%%%%%%%%%%%%%%%%%
\subsubsection{\label{sec:Security-access-levels}Access Level Descriptions}
%%%%%%%%%%%%%%%%%%%%%%%%%%%%%%%%%%%%%%%%%%%%%%%%%%%%%%%%%%%%%%%%%%%%%%
\index{security!access levels}
Authorization is granted based on specified access levels.
This list describes each access level,
and provides examples of their usage.
The levels implement a partial hierarchy;  a higher level often implies
a \DCPerm{READ} or both a 
\DCPerm{WRITE} and a \DCPerm{READ} level of access as described.

\begin{description}

\item[\DCPerm{READ}] \label{sec-level-read} This access level
   can obtain or read information about HTCondor.
   Examples that require only \DCPerm{READ} access are
   viewing the status of the pool with \Condor{status}, 
   checking a job queue with \Condor{q},
   or viewing user priorities with \Condor{userprio}.
   \DCPerm{READ} access does not allow any
   changes, and it does not allow job submission.

\item[\DCPerm{WRITE}] \label{sec-level-write} This access level is
   required to send (write) information to HTCondor. Examples that
   require \DCPerm{WRITE} access are job submission with
   \Condor{submit} and advertising a machine so it appears in the pool
   (this is usually done automatically by the \Condor{startd} daemon).
   The \DCPerm{WRITE} level of access implies \DCPerm{READ} access. 

\item[\DCPerm{ADMINISTRATOR}] \label{sec-level-administrator} This
   access level has additional HTCondor
   administrator rights to the pool.  It includes the ability to
   change user priorities with the command \Condor{userprio},
   as well as the ability to turn HTCondor on and off
   (as with the commands \Condor{on} and \Condor{off}).
   The \DCPerm{ADMINISTRATOR} level of access implies both
   \DCPerm{READ} and \DCPerm{WRITE} access.

\item[\DCPerm{SOAP}] \label{sec-level-soap} This
   access level is required for the authorization of any party that will 
   use the Web Services (SOAP) interface to HTCondor.
   It is not a general access level to be used with the variety
   of configuration variables for authentication, encryption,
   and integrity checks.

\item[\DCPerm{CONFIG}] \label{sec-level-config} This access level is
   required to modify a daemon's configuration using
   the \Condor{config\_val} command.
   By default, this level of access can
   change any configuration parameters of an HTCondor pool,
   except those specified in
   the \File{condor\_config.root} configuration file.
   The \DCPerm{CONFIG} level of access implies \DCPerm{READ} access. 

\item[\DCPerm{OWNER}] \label{sec-level-owner} This level of access is
   required for commands that the owner of a machine (any local user)
   should be able to use, in addition to the HTCondor administrators.
   An example that requires the \DCPerm{OWNER} access level is
   the \Condor{vacate} command.
   The command causes the \Condor{startd} daemon to vacate any
   HTCondor job currently running on a machine.
   The owner of that machine should be able to cause the removal
   of a job running on the machine.

\item[\DCPerm{DAEMON}] \label{sec-level-daemon} This access level
   is used for commands that are internal to the operation of
   HTCondor.  An example of this internal operation is when the
   \Condor{startd} daemon sends
   its ClassAd updates to the \Condor{collector} daemon (which may be
   more specifically controlled by the \DCPerm{ADVERTISE\_STARTD}
   access level).
   Authorization at this access level should only be given to
   the user account under which the HTCondor daemons run.
   The \DCPerm{DAEMON} level of access implies both
   \DCPerm{READ} and \DCPerm{WRITE} access.  Any setting for this access
   level that is not defined will default to the corresponding setting
   in the \DCPerm{WRITE} access level.

\item[\DCPerm{NEGOTIATOR}] \label{sec-level-negotiator} This 
   access level is used specifically to verify that commands are
   sent by the \Condor{negotiator} daemon.
   The \Condor{negotiator} daemon runs on the central manager of
   the pool.
   Commands requiring this access
   level are the ones that tell the \Condor{schedd} daemon to begin
   negotiating, and those that tell an available \Condor{startd} daemon
   that it has been matched to a \Condor{schedd} with jobs to run.
   The \DCPerm{NEGOTIATOR} level of access implies \DCPerm{READ} access. 

\item[\DCPerm{ADVERTISE\_MASTER}] \label{sec-level-advertise-master} This
   access level is used specifically for commands used to advertise a
   \Condor{master} daemon to the collector.  Any setting for this access
   level that is not defined will default to the corresponding setting
   in the \DCPerm{DAEMON} access level.

\item[\DCPerm{ADVERTISE\_STARTD}] \label{sec-level-advertise-master} This
   access level is used specifically for commands used to advertise a
   \Condor{startd} daemon to the collector.  Any setting for this access
   level that is not defined will default to the corresponding setting
   in the \DCPerm{DAEMON} access level.

\item[\DCPerm{ADVERTISE\_SCHEDD}] \label{sec-level-advertise-master} This
   access level is used specifically for commands used to advertise a
   \Condor{schedd} daemon to the collector.  Any setting for this access
   level that is not defined will default to the corresponding setting
   in the \DCPerm{DAEMON} access level.

\item[\DCPerm{CLIENT}] \label{sec-level-client} This access level is
   different from all the others.  Whereas all of the other access levels
   refer to the security policy for accepting connections \emph{from} others,
   the \DCPerm{CLIENT} access level applies when an HTCondor daemon or tool is
   connecting \emph{to} some other HTCondor daemon.  In other words, it specifies
   the policy of the client that is initiating the operation, rather than
   the server that is being contacted.

\end{description}


The following is a list of registered commands that daemons will
accept.  The list is ordered by daemon.
For each daemon, the commands are grouped by the access level
required for a daemon to accept the command from a
given machine.

ALL DAEMONS:

\begin{description}
\item[\DCPerm{WRITE}]

  The command sent as a result of \Condor{reconfig} to reconfigure a daemon.
\end{description}

STARTD:

\begin{description}
\item[\DCPerm{WRITE}] 

All commands that relate to a \Condor{schedd} daemon claiming
  a machine, starting jobs there, or stopping those jobs.

The command that \Condor{checkpoint} sends to periodically checkpoint
  all running jobs.

\item[\DCPerm{READ}]

The command that \Condor{preen} sends to request the
  current state of the \Condor{startd} daemon.

\item[\DCPerm{OWNER}]
The command that \Condor{vacate} sends to cause
  any running jobs to stop running.

\item[\DCPerm{NEGOTIATOR}]
The command that the \Condor{negotiator} daemon sends to
  match a machine's \Condor{startd} daemon with a given \Condor{schedd}
  daemon.
\end{description}

NEGOTIATOR:

\begin{description}
\item[\DCPerm{WRITE}]
The command that initiates a new negotiation
  cycle. It is sent by the \Condor{schedd} when new jobs are submitted
  or a \Condor{reschedule} command is issued.

\item[\DCPerm{READ}]
The command that can retrieve the current state
  of user priorities in the pool, sent by the \Condor{userprio} command.

\item[\DCPerm{ADMINISTRATOR}]
The command that can set the current
  values of user priorities, sent as a result of the \Condor{userprio}
  command.
\end{description}

COLLECTOR:

\begin{description}

\item[\DCPerm{ADVERTISE\_MASTER}]
Commands that update the \Condor{collector} daemon with new \Condor{master} ClassAds.

\item[\DCPerm{ADVERTISE\_SCHEDD}]
Commands that update the \Condor{collector} daemon with new \Condor{schedd} ClassAds.

\item[\DCPerm{ADVERTISE\_STARTD}]
Commands that update the \Condor{collector} daemon with new \Condor{startd} ClassAds.

\item[\DCPerm{DAEMON}]
All other commands that update the \Condor{collector} daemon with new
ClassAds.  Note that the specific access levels such as
\DCPerm{ADVERTISE\_STARTD} default to the \DCPerm{DAEMON} settings,
which in turn defaults to \DCPerm{WRITE}.

\item[\DCPerm{READ}]
All commands that query the \Condor{collector} daemon for ClassAds.
\end{description}

SCHEDD: 

\begin{description}
\item[\DCPerm{NEGOTIATOR}]
The command that the \Condor{negotiator} sends to
  begin negotiating with this \Condor{schedd} to match its jobs with available
  \Condor{startds}.

\item[\DCPerm{WRITE}]
The command which \Condor{reschedule} sends to
  the \Condor{schedd} to get it to update the \Condor{collector} with a current ClassAd
  and begin a negotiation cycle.

  The commands which write information into the job queue (such as
  \Condor{submit} and \Condor{hold}).  
  Note that for most commands which attempt to write to the job queue, HTCondor
  will perform an additional user-level authentication step.  
  This additional user-level authentication prevents, for example, an
  ordinary user from removing a different user's jobs.

\item[\DCPerm{READ}]
The command from any
  tool to view the status of the job queue.  

 The commands that a \Condor{startd} sends to the \Condor{schedd} when
 the \Condor{schedd} daemon's claim is being preempted and also when the
 lease on the claim is renewed.  These operations only require
 \DCPerm{READ} access, rather than \DCPerm{DAEMON} in order to
 limit the level of trust that the \Condor{schedd} must have for the
 \Condor{startd}.  Success of these commands is only possible if the
 \Condor{startd} knows the secret claim id, so effectively, authorization for
 these commands is more specific than HTCondor's general security model
 implies.  The \Condor{schedd} \emph{automatically} grants the
 \Condor{startd} \DCPerm{READ} access for the duration of the claim.
 Therefore, if one desires to only authorize specific execute
 machines to run jobs, one must either limit which machines are
 allowed to advertise themselves to the pool (most common) or
 configure the \Condor{schedd}'s \Macro{ALLOW\_CLIENT} setting to only
 allow connections from the \Condor{schedd} to the trusted execute machines.

\end{description}

MASTER:  All commands are registered with \DCPerm{ADMINISTRATOR}
access:

\begin{description}
\item[restart] : Master restarts itself (and all its children)	
\item[off] : Master shuts down all its children
\item[off -master] : Master shuts down all its children and exits
\item[on] : Master spawns all the daemons it is configured to spawn
\end{description}

%%%%%%%%%%%%%%%%%%%%%%%%%%%%%%%%%%%%%%%%%%%%%%%%%%%%%%%%%%%%%%%%%%%%%%
\subsection{\label{sec:Security-Negotiation}Security Negotiation}
%%%%%%%%%%%%%%%%%%%%%%%%%%%%%%%%%%%%%%%%%%%%%%%%%%%%%%%%%%%%%%%%%%%%%%

Because of the wide range of environments and security demands necessary,
HTCondor must be flexible.
Configuration provides this flexibility.
The process by which HTCondor determines the security settings that will
be used when a connection is established is called
\Term{security negotiation}.
Security negotiation's primary purpose is to determine which
of the features of authentication, encryption, and integrity checking
will be enabled for a connection.
In addition, since HTCondor supports multiple
technologies for authentication and encryption,
security negotiation also
determines which technology is chosen for the connection.

%The issue of allowing flexible configuration of security options is
%complicated by the fact that HTCondor components that wish to interact
%may be configured to use different security options. This should not
%preclude the interaction from taking place. Pretend Frida, an
%aggressive cycle-scavenger, runs a variety of HTCondor tools from her
%workstation.  She would prefer not to use authentication as part of
%her interactions with HTCondor (because she thinks it's a waste of
%CPU). A \Condor{schedd} that she submits jobs to, however, refuses to
%accept any job unless it can be sure of the submitter's identity, and
%thus wishes to force authentication. Given the \Condor{schedd}'s
%policy, Frida would rather her jobs run even if she does have to
%authenticate first, so job submission should be allowed to commence
%(with authentication turned on, of course).

Security negotiation is a completely separate process from
matchmaking, and should not be confused with any specific function of
the \Condor{negotiator} daemon. 
Security negotiation occurs when one
HTCondor daemon or tool initiates communication with another HTCondor daemon,
to determine the security settings by which the communication will
be ruled.
The \Condor{negotiator} daemon does negotiation,
whereby queued jobs and available machines within a pool
go through the process of matchmaking (deciding out which
machines will run which jobs).

%%%%%%%%%%%%%%%%%%%%%%%%%%%%%%%%%%%%%%%%%%%%%%%%%%%%%%%%%%%%%%%%%%%%%%
\subsubsection{\label{sec:security-negotiation-features}Configuration}
%%%%%%%%%%%%%%%%%%%%%%%%%%%%%%%%%%%%%%%%%%%%%%%%%%%%%%%%%%%%%%%%%%%%%%

The configuration macro names that determine what features will be used during
client-daemon communication follow the pattern:
\begin{verbatim}
    SEC_<context>_<feature>
\end{verbatim}

The \verb@<feature>@
portion of the macro name determines which security feature's
policy is being set.
\verb@<feature>@ may be any one of
\begin{verbatim}
    AUTHENTICATION
    ENCRYPTION
    INTEGRITY
    NEGOTIATION
\end{verbatim}

The \verb@<context>@ component of the security policy macros can be
used to craft a fine-grained security policy based on the type of
communication taking place.
\verb@<context>@ may be any one of
\begin{verbatim}
    CLIENT
    READ
    WRITE
    ADMINISTRATOR
    CONFIG
    OWNER
    DAEMON
    NEGOTIATOR
    ADVERTISE_MASTER
    ADVERTISE_STARTD
    ADVERTISE_SCHEDD
    DEFAULT
\end{verbatim}

Any of these constructed configuration macros may be set to
any of the following values:
\begin{verbatim}
    REQUIRED
    PREFERRED
    OPTIONAL
    NEVER 
\end{verbatim}

Security negotiation resolves various client-daemon combinations
of desired security features in order to set a policy.

As an example, consider Frida the scientist.
Frida wants to avoid authentication when possible.
She sets
\begin{verbatim}
    SEC_DEFAULT_AUTHENTICATION = OPTIONAL
\end{verbatim}
The machine running the \Condor{schedd}
to which Frida will remotely submit jobs,
however,
is operated by a security-conscious system administrator who dutifully
sets:
\begin{verbatim}
    SEC_DEFAULT_AUTHENTICATION = REQUIRED
\end{verbatim}
When Frida submits her jobs, HTCondor's security negotiation determines
that authentication will be used, and allows the command to continue.
This example illustrates the point that the most restrictive
security policy sets the levels of security enforced.
There is actually more to the understanding of this scenario.
Some HTCondor commands, such as the use of \Condor{submit}
to submit jobs \emph{always} require
authentication of the submitter, no matter what the policy says. 
This is because the identity
of the submitter needs to be known in order to carry out the operation.
Others commands, such as \Condor{q}, do not always require
authentication, so in the above example, the server's policy would
force Frida's \Condor{q} queries to be authenticated, whereas a different
policy could allow \Condor{q} to happen without any authentication.


Whether or not security negotiation occurs depends on the
setting at both the client and daemon side of the 
configuration variable(s) defined by \MacroNI{SEC\_*\_NEGOTIATION}.
\MacroNI{SEC\_DEFAULT\_NEGOTIATION} is a variable representing
the entire set of configuration variables for \MacroNI{NEGOTIATION}.
For the client side setting,
the only definitions that make sense are \Expr{REQUIRED} and \Expr{NEVER}.
For the daemon side setting,
the \Expr{PREFERRED} value makes no sense.
Table~\ref{table:Sec-Negotiation} shows how security negotiation
resolves various client-daemon combinations of security negotiation policy
settings.
Within the table, Yes means the security negotiation will take place.
No means it will not.
Fail means that the policy settings are incompatible and the communication
cannot continue.

\begin{table}[tb]
\centering
\begin{tabular}{|c|c|c|c|c|}
  \hline
  \multicolumn{2}{|c|}{\hfill} & \multicolumn{3}{c|}{Daemon Setting} \\
  \cline{3-5}
  \multicolumn{2}{|c|}{\hfill} & NEVER & OPTIONAL & REQUIRED \\
  \hline
  \cline{2-5}
  Client & NEVER & No & No & Fail \\
  \cline{2-5}
  Setting & REQUIRED & Fail & Yes & Yes \\
  \hline
\end{tabular}
\caption{\label{table:Sec-Negotiation}Resolution of security negotiation.  }
\end{table}


Enabling authentication, encryption, and integrity checks is
dependent on security negotiation taking place.
The enabled security negotiation further sets the policy for
these other features.
Table~\ref{table:Sec-Resolution} shows how security features
are resolved for client-daemon combinations of security feature policy
settings.
Like Table~\ref{table:Sec-Negotiation},
Yes means the feature will be utilized.
No means it will not.
Fail implies incompatibility and the feature cannot be resolved.


\begin{table}[tb]
\centering
\begin{tabular}{|c|c|c|c|c|c|}
  \hline
  \multicolumn{2}{|c|}{\hfill} & \multicolumn{4}{c|}{Daemon Setting} \\
  \cline{3-6}
  \multicolumn{2}{|c|}{\hfill} & NEVER & OPTIONAL & PREFERRED & REQUIRED \\
  \hline
  \hfill & NEVER & No & No & No & Fail \\
  \cline{2-6}
  Client & OPTIONAL & No & No & Yes & Yes \\
  \cline{2-6}
  Setting & PREFERRED & No & Yes & Yes & Yes \\
  \cline{2-6}
  \hfill & REQUIRED & Fail & Yes & Yes & Yes \\
  \hline
\end{tabular}
\caption{\label{table:Sec-Resolution}Resolution of security features. }
\end{table}


The enabling of encryption and/or integrity checks is dependent on
authentication taking place.
The authentication provides a key exchange.
The key is needed for both encryption and integrity checks.

Setting \verb@SEC_CLIENT_<feature>@ determines the policy for all
outgoing commands.
The policy for incoming commands
(the daemon side of the communication)
takes a more fine-grained approach that implements a set of
access levels for the received command.
For example, it is desirable to have all incoming administrative
requests require authentication.
Inquiries on pool status may not be so restrictive.
To implement this, the administrator configures the policy:

\begin{verbatim}
SEC_ADMINISTRATOR_AUTHENTICATION = REQUIRED
SEC_READ_AUTHENTICATION          = OPTIONAL
\end{verbatim}

The \verb@DEFAULT@ value for \verb@<context>@ provides a way
to set a policy for all access levels
(\verb@READ@, \verb@WRITE@, etc.)
that do not have a specific configuration variable defined.
In addition, some access levels will default to the settings specified
for other access levels.  For example, \DCPerm{ADVERTISE\_STARTD}
defaults to \DCPerm{DAEMON}, and \DCPerm{DAEMON} defaults to
\DCPerm{WRITE}, which then defaults to the general \verb@DEFAULT@
setting.

%%%%%%%%%%%%%%%%%%%%%%%%%%%%%%%%%%%%%%%%%%%%%%%%%%%%%%%%%%%%%%%%%%%%%%
\subsubsection{\label{sec:security-negotiation-methods}Configuration for 
Security Methods}
%%%%%%%%%%%%%%%%%%%%%%%%%%%%%%%%%%%%%%%%%%%%%%%%%%%%%%%%%%%%%%%%%%%%%%

Authentication and encryption can each be accomplished by a variety
of methods or technologies.
Which method is utilized is determined during
security negotiation.

The configuration macros that determine the methods
to use for authentication and/or encryption are
\begin{verbatim}
SEC_<context>_AUTHENTICATION_METHODS
SEC_<context>_CRYPTO_METHODS
\end{verbatim}

These macros are defined by a comma or space delimited list of
possible methods to use.
Section \ref{sec:Security-Authentication} lists all implemented
authentication methods.
Section \ref{sec:Security-Encryption} lists all implemented
encryption methods.



%%%%%%%%%%%%%%%%%%%%%%%%%%%%%%%%%%%%%%%%%%%%%%%%%%%%%%%%%%%%%%%%%%%%%%
\subsection{\label{sec:Security-Authentication}Authentication}
%%%%%%%%%%%%%%%%%%%%%%%%%%%%%%%%%%%%%%%%%%%%%%%%%%%%%%%%%%%%%%%%%%%%%%
\index{authentication|(}
\index{security!authentication}

The client side of any communication uses one of two macros
to specify whether authentication is to occur:
\index{SEC\_DEFAULT\_AUTHENTICATION macro@\texttt{SEC\_DEFAULT\_AUTHENTICATION} macro}
\index{configuration macro!\texttt{SEC\_DEFAULT\_AUTHENTICATION}}
\index{SEC\_CLIENT\_AUTHENTICATION macro@\texttt{SEC\_CLIENT\_AUTHENTICATION} macro}
\index{configuration macro!\texttt{SEC\_CLIENT\_AUTHENTICATION}}
\begin{verbatim}
    SEC_DEFAULT_AUTHENTICATION
    SEC_CLIENT_AUTHENTICATION
\end{verbatim}

For the daemon side, there are a larger number of macros to specify whether
authentication is to take place, based upon the necessary
access level:
\index{SEC\_DEFAULT\_AUTHENTICATION macro@\texttt{SEC\_DEFAULT\_AUTHENTICATION} macro}
\index{configuration macro!\texttt{SEC\_DEFAULT\_AUTHENTICATION}}
\index{SEC\_READ\_AUTHENTICATION macro@\texttt{SEC\_READ\_AUTHENTICATION} macro}
\index{configuration macro!\texttt{SEC\_READ\_AUTHENTICATION}}
\index{SEC\_WRITE\_AUTHENTICATION macro@\texttt{SEC\_WRITE\_AUTHENTICATION} macro}
\index{configuration macro!\texttt{SEC\_WRITE\_AUTHENTICATION}}
\index{SEC\_ADMINISTRATOR\_AUTHENTICATION macro@\texttt{SEC\_ADMINISTRATOR\_AUTHENTICATION} macro}
\index{configuration macro!\texttt{SEC\_ADMINISTRATOR\_AUTHENTICATION}}
\index{SEC\_CONFIG\_AUTHENTICATION macro@\texttt{SEC\_CONFIG\_AUTHENTICATION} macro}
\index{configuration macro!\texttt{SEC\_CONFIG\_AUTHENTICATION}}
\index{SEC\_OWNER\_AUTHENTICATION macro@\texttt{SEC\_OWNER\_AUTHENTICATION} macro}
\index{configuration macro!\texttt{SEC\_OWNER\_AUTHENTICATION}}
\index{SEC\_DAEMON\_AUTHENTICATION macro@\texttt{SEC\_DAEMON\_AUTHENTICATION} macro}
\index{configuration macro!\texttt{SEC\_DAEMON\_AUTHENTICATION}}
\index{SEC\_NEGOTIATOR\_AUTHENTICATION macro@\texttt{SEC\_NEGOTIATOR\_AUTHENTICATION} macro}
\index{configuration macro!\texttt{SEC\_NEGOTIATOR\_AUTHENTICATION}}
\index{SEC\_ADVERTISE\_MASTER\_AUTHENTICATION macro@\texttt{SEC\_ADVERTISE\_MASTER\_AUTHENTICATION} macro}
\index{configuration macro!\texttt{SEC\_ADVERTISE\_MASTER\_AUTHENTICATION}}
\index{SEC\_ADVERTISE\_STARTD\_AUTHENTICATION macro@\texttt{SEC\_ADVERTISE\_STARTD\_AUTHENTICATION} macro}
\index{configuration macro!\texttt{SEC\_ADVERTISE\_STARTD\_AUTHENTICATION}}
\index{SEC\_ADVERTISE\_SCHEDD\_AUTHENTICATION macro@\texttt{SEC\_ADVERTISE\_SCHEDD\_AUTHENTICATION} macro}
\index{configuration macro!\texttt{SEC\_ADVERTISE\_SCHEDD\_AUTHENTICATION}}
\begin{verbatim}
    SEC_DEFAULT_AUTHENTICATION
    SEC_READ_AUTHENTICATION
    SEC_WRITE_AUTHENTICATION
    SEC_ADMINISTRATOR_AUTHENTICATION
    SEC_CONFIG_AUTHENTICATION
    SEC_OWNER_AUTHENTICATION
    SEC_DAEMON_AUTHENTICATION
    SEC_NEGOTIATOR_AUTHENTICATION
    SEC_ADVERTISE_MASTER_AUTHENTICATION
    SEC_ADVERTISE_STARTD_AUTHENTICATION
    SEC_ADVERTISE_SCHEDD_AUTHENTICATION
\end{verbatim}

As an example, the macro defined in the configuration file
for a daemon as
\begin{verbatim}
SEC_WRITE_AUTHENTICATION = REQUIRED
\end{verbatim}
signifies that the daemon must authenticate the client for
any communication that requires the \DCPerm{WRITE} access level.
If the daemon's configuration contains
\begin{verbatim}
SEC_DEFAULT_AUTHENTICATION = REQUIRED
\end{verbatim}
and does not contain any other security configuration for
\verb@AUTHENTICATION@, then this default defines the daemon's needs
for authentication over all access levels.
Where a specific macro is defined, the more specific value takes
precedence over the default definition.


If authentication is to be done, then the communicating parties
must negotiate a mutually acceptable method of
authentication to be used.
A list of acceptable methods may be provided by the client, using the
macros
\index{SEC\_DEFAULT\_AUTHENTICATION\_METHODS macro@\texttt{SEC\_DEFAULT\_AUTHENTICATION\_METHODS} macro}
\index{configuration macro!\texttt{SEC\_DEFAULT\_AUTHENTICATION\_METHODS}}
\index{SEC\_CLIENT\_AUTHENTICATION\_METHODS macro@\texttt{SEC\_CLIENT\_AUTHENTICATION\_METHODS} macro}
\index{configuration macro!\texttt{SEC\_CLIENT\_AUTHENTICATION\_METHODS}}
\begin{verbatim}
    SEC_DEFAULT_AUTHENTICATION_METHODS
    SEC_CLIENT_AUTHENTICATION_METHODS
\end{verbatim}
A list of acceptable methods may be provided by the daemon, using the
macros
\index{SEC\_DEFAULT\_AUTHENTICATION\_METHODS macro@\texttt{SEC\_DEFAULT\_AUTHENTICATION\_METHODS} macro}
\index{configuration macro!\texttt{SEC\_DEFAULT\_AUTHENTICATION\_METHODS}}
\index{SEC\_READ\_AUTHENTICATION\_METHODS macro@\texttt{SEC\_READ\_AUTHENTICATION\_METHODS} macro}
\index{configuration macro!\texttt{SEC\_READ\_AUTHENTICATION\_METHODS}}
\index{SEC\_WRITE\_AUTHENTICATION\_METHODS macro@\texttt{SEC\_WRITE\_AUTHENTICATION\_METHODS} macro}
\index{configuration macro!\texttt{SEC\_WRITE\_AUTHENTICATION\_METHODS}}
\index{SEC\_ADMINISTRATOR\_AUTHENTICATION\_METHODS macro@\texttt{SEC\_ADMINISTRATOR\_AUTHENTICATION\_METHODS} macro}
\index{configuration macro!\texttt{SEC\_ADMINISTRATOR\_AUTHENTICATION\_METHODS}}
\index{SEC\_DAEMON\_AUTHENTICATION\_METHODS macro@\texttt{SEC\_DAEMON\_AUTHENTICATION\_METHODS} macro}
\index{configuration macro!\texttt{SEC\_DAEMON\_AUTHENTICATION\_METHODS}}
\index{SEC\_CONFIG\_AUTHENTICATION\_METHODS macro@\texttt{SEC\_CONFIG\_AUTHENTICATION\_METHODS} macro}
\index{configuration macro!\texttt{SEC\_CONFIG\_AUTHENTICATION\_METHODS}}
\index{SEC\_OWNER\_AUTHENTICATION\_METHODS macro@\texttt{SEC\_OWNER\_AUTHENTICATION\_METHODS} macro}
\index{configuration macro!\texttt{SEC\_OWNER\_AUTHENTICATION\_METHODS}}
\index{SEC\_NEGOTIATOR\_AUTHENTICATION\_METHODS macro@\texttt{SEC\_NEGOTIATOR\_AUTHENTICATION\_METHODS} macro}
\index{configuration macro!\texttt{SEC\_NEGOTIATOR\_AUTHENTICATION\_METHODS}}
\index{SEC\_ADVERTISE\_MASTER\_AUTHENTICATION\_METHODS macro@\texttt{SEC\_ADVERTISE\_MASTER\_AUTHENTICATION\_METHODS} macro}
\index{configuration macro!\texttt{SEC\_ADVERTISE\_MASTER\_AUTHENTICATION\_METHODS}}
\index{SEC\_ADVERTISE\_STARTD\_AUTHENTICATION\_METHODS macro@\texttt{SEC\_ADVERTISE\_STARTD\_AUTHENTICATION\_METHODS} macro}
\index{configuration macro!\texttt{SEC\_ADVERTISE\_STARTD\_AUTHENTICATION\_METHODS}}
\index{SEC\_ADVERTISE\_SCHEDD\_AUTHENTICATION\_METHODS macro@\texttt{SEC\_ADVERTISE\_SCHEDD\_AUTHENTICATION\_METHODS} macro}
\index{configuration macro!\texttt{SEC\_ADVERTISE\_SCHEDD\_AUTHENTICATION\_METHODS}}
\begin{verbatim}
    SEC_DEFAULT_AUTHENTICATION_METHODS
    SEC_READ_AUTHENTICATION_METHODS
    SEC_WRITE_AUTHENTICATION_METHODS
    SEC_ADMINISTRATOR_AUTHENTICATION_METHODS
    SEC_CONFIG_AUTHENTICATION_METHODS
    SEC_OWNER_AUTHENTICATION_METHODS
    SEC_DAEMON_AUTHENTICATION_METHODS
    SEC_NEGOTIATOR_AUTHENTICATION_METHODS
    SEC_ADVERTISE_MASTER_AUTHENTICATION_METHODS
    SEC_ADVERTISE_STARTD_AUTHENTICATION_METHODS
    SEC_ADVERTISE_SCHEDD_AUTHENTICATION_METHODS
\end{verbatim}
The methods are
given as a comma-separated list of acceptable values.
These variables list the authentication methods that are available
to be used.
The ordering of the list defines preference;
the first item in the list indicates the highest preference.
As not all of the authentication methods work on Windows platforms, 
which ones do \emph{not} work on Windows are indicated in 
the following list of defined values:
\begin{verbatim}
    GSI       (not available on Windows platforms)
    SSL
    KERBEROS
    PASSWORD
    FS        (not available on Windows platforms)
    FS_REMOTE (not available on Windows platforms)
    NTSSPI
    CLAIMTOBE
    ANONYMOUS
\end{verbatim}

For example, a client may be configured with:
\begin{verbatim}
SEC_CLIENT_AUTHENTICATION_METHODS = FS, GSI
\end{verbatim}
and a daemon the client is trying to contact with:
\begin{verbatim}
SEC_DEFAULT_AUTHENTICATION_METHODS = GSI
\end{verbatim}

Security negotiation will determine that GSI authentication is the only
compatible choice. If there are multiple compatible authentication
methods, security negotiation will make a list of acceptable methods and
they will be tried in order until one succeeds. 

As another example, the macro
\begin{verbatim}
SEC_DEFAULT_AUTHENTICATION_METHODS = KERBEROS, NTSSPI
\end{verbatim}
indicates that either Kerberos or Windows authentication may be used,
but Kerberos is preferred over Windows.
Note that if the client and daemon agree that multiple authentication
methods may be used, then they are tried in turn. For instance, if
they both agree that Kerberos or NTSSPI may be used, then Kerberos
will be tried first, and if there is a failure for any reason, then
NTSSPI will be tried. 

An additional specialized method of authentication exists for
communication between the \Condor{schedd} and \Condor{startd}.  It is
especially useful when operating at large scale over high latency
networks or in situations where it is inconvenient to set up one of
the other methods of strong authentication between the submit and
execute daemons.  See the description of
\MacroNI{SEC\_ENABLE\_MATCH\_PASSWORD\_AUTHENTICATION} on
\pageref{param:SecEnableMatchPasswordAuthentication} for
details.

If the configuration for a machine does not define any variable
for \MacroNI{SEC\_<access-level>\_AUTHENTICATION},
then HTCondor uses a default value of \verb@OPTIONAL@.
Authentication will be required for
any operation which modifies the job queue,
such as \Condor{qedit} and \Condor{rm}.
If the configuration for a machine does not define any variable
for \MacroNI{SEC\_<access-level>\_AUTHENTICATION\_METHODS},
the default value for a Unix machine is \verb@FS@, \verb@KERBEROS@,
\verb@GSI@. 
This default value for a Windows machine is
\verb@NTSSPI@, \verb@KERBEROS@, \verb@GSI@. 

%%%%%%%%%%%%%%%%%%%%%%%%%%%%%%%%%%%%%%%%%%%%%%%%%%%%%%%%%%%%%%%%%%%%%%
\subsubsection{\label{sec:GSI-Authentication}GSI Authentication}
%%%%%%%%%%%%%%%%%%%%%%%%%%%%%%%%%%%%%%%%%%%%%%%%%%%%%%%%%%%%%%%%%%%%%%
\index{authentication!GSI}
The GSI (Grid Security Infrastructure) protocol provides
an avenue for HTCondor to do
PKI-based (Public Key Infrastructure) authentication using X.509
certificates. 
The basics of GSI are well-documented elsewhere, such as
\URL{http://www.globus.org/}. 

A simple introduction to this type of authentication
defines HTCondor's use of terminology,
and it illuminates the needed items that HTCondor must access to
do this authentication.
Assume that 
A authenticates to B.
In this example, A is the client, and B is the daemon within
their communication.
This example's one-way authentication implies that B
is verifying the identity of A,
using the certificate A provides,
and utilizing B's own set of trusted CAs (Certification Authorities).
Client A provides its certificate (or proxy) to daemon B.
B does two things:
B checks that the certificate is valid,
and B checks to see that the CA that signed A's certificate
is one that B trusts.

For the GSI authentication protocol,
an X.509 certificate is required.
\index{certificate!X.509}
Files with predetermined names hold a certificate,
a key, and optionally, a proxy.
A separate directory has one or more files that become the list of
trusted CAs.

Allowing HTCondor to do this GSI authentication
requires knowledge of the locations of
the client A's certificate and the daemon B's list of
trusted CAs.
When one side of the communication (as either client A or daemon B)
is an HTCondor daemon, these locations are determined
by configuration or by default locations.
When one side of the communication (as a client A)
is a user of HTCondor (the process owner of an HTCondor tool,
for example \Condor{submit}), these locations are determined by the
pre-set values of environment variables or by default locations.

\begin{description}
\item[GSI certificate locations for HTCondor daemons]

For an HTCondor daemon, the certificate may be a single host certificate,
\index{host certificate}
and all HTCondor daemons on the same machine may share the same certificate.
In some cases, the certificate can also be copied to other machines,
where local copies are necessary.
This may occur only in cases where a single host certificate can
match multiple host names, something that is beyond the scope of this
manual. 
The certificates must be protected by access rights to files,
since the password file is not encrypted.

The specification of the location of the necessary files
through configuration uses the following precedence.
\begin{enumerate}
\item
Configuration variable \Macro{GSI\_DAEMON\_DIRECTORY} gives the complete
path name to the directory that contains the certificate, key,
and directory with trusted CAs.
HTCondor uses this directory as follows in its construction of the following
configuration variables:
\footnotesize
\begin{verbatim}
GSI_DAEMON_CERT           = $(GSI_DAEMON_DIRECTORY)/hostcert.pem
GSI_DAEMON_KEY            = $(GSI_DAEMON_DIRECTORY)/hostkey.pem 
GSI_DAEMON_TRUSTED_CA_DIR = $(GSI_DAEMON_DIRECTORY)/certificates 
\end{verbatim}

\normalsize
Note that no proxy is assumed in this case.
\item
If the \MacroNI{GSI\_DAEMON\_DIRECTORY} is not defined, 
or when defined,
the location may be overridden with specific configuration
variables that specify the complete path and file name of 
the certificate with
  \begin{description}
  \item{\Macro{GSI\_DAEMON\_CERT}}
  \end{description}
the key with
  \begin{description}
  \item{\Macro{GSI\_DAEMON\_KEY}}
  \end{description}
a proxy with
  \begin{description}
  \item{\Macro{GSI\_DAEMON\_PROXY}}
  \end{description}
the complete path to the directory containing the list of trusted CAs with 
  \begin{description}
  \item{\Macro{GSI\_DAEMON\_TRUSTED\_CA\_DIR}}
  \end{description}
\item
The default location assumed is \File{/etc/grid-security}.
Note that this implemented by setting the value of  
\MacroNI{GSI\_DAEMON\_DIRECTORY}.
\end{enumerate}

When a daemon acts as the client within authentication,
the daemon needs a listing of those from which it
will accept certificates.
This is done with \MacroNI{GSI\_DAEMON\_NAME}.
This name is specified with the following format
\footnotesize
\begin{verbatim}
GSI_DAEMON_NAME = /X.509/name/of/server/1,/X.509/name/of/server/2,...
\end{verbatim}
\normalsize

HTCondor will also need a way to map an X.509 distinguished
name to an HTCondor user id.
There are two ways to accomplish this mapping.
For a first way to specify the mapping, see
section~\ref{sec:Security-Unified-Map-File}
to use HTCondor's unified map file.
The second way to do the mapping is within
an administrator-maintained GSI-specific file called an X.509 map file,
mapping from X.509 Distinguished Name (DN) to HTCondor user id.
It is similar to a Globus grid map file, except that it is only used for
mapping to a user id, not for authorization. 
If the user names in the
map file do not specify a domain for the user
(specification would appear as \verb|user@domain|),
then the value of \MacroNI{UID\_DOMAIN} is used.
Entries (lines) in the file each contain two items.
The first item in an entry is the 
X.509 certificate subject name, and it is enclosed in double quote marks
(using the character \verb@"@).
The second item is the HTCondor user id.
The two items in an entry are separated by tab or space character(s).
Here is an example of an entry in an X.509 map file.
Entries must be on a single line; this example is broken
onto two lines for formatting reasons.

\footnotesize
\begin{verbatim}
"/C=US/O=Globus/O=University of Wisconsin/
OU=Computer Sciences Department/CN=Alice Smith" asmith
\end{verbatim}
\normalsize

HTCondor finds the map file in one of three ways.
If the configuration variable \Macro{GRIDMAP} is defined,
it gives the full path name to the map file.
When not defined,
HTCondor looks for the map file in 
\begin{verbatim}
$(GSI_DAEMON_DIRECTORY)/grid-mapfile
\end{verbatim}
If \Macro{GSI\_DAEMON\_DIRECTORY} is not defined,
then the third place HTCondor looks for the map file is given by
\begin{verbatim}
/etc/grid-security/grid-mapfile
\end{verbatim}


\item[GSI certificate locations for Users]

The user specifies the location of a certificate, proxy, etc.
in one of two ways:
\begin{enumerate}
\item
Environment variables give the location of necessary items.

  \Env{X509\_USER\_PROXY} gives the path and file name of the proxy.
  This proxy will have been created using the \Prog{grid-proxy-init} program, 
  which will place the proxy in the \File{/tmp}
  directory with the file name being determined by the format:
  \begin{verbatim}
  /tmp/x509up_uXXXX
  \end{verbatim}
  The specific file name is given by substituting the \verb@XXXX@
  characters with the UID of the user.
  Note that when a valid proxy is used, the certificate and key locations
  are not needed. 

  \Env{X509\_USER\_CERT} gives the path and file name of the
  certificate. It is also used if a proxy location has been checked,
  but the proxy is no longer valid.  

  \Env{X509\_USER\_KEY} gives the path and file name of the
  key. Note that most keys are password encrypted, such that knowing
  the location could not lead to using the key. 

  \Env{X509\_CERT\_DIR} gives the path to the directory 
  containing the list of trusted CAs. 

\item
Without environment variables to give locations of necessary
certificate information,
HTCondor uses a default directory for the user.
This directory is given by 
\begin{verbatim}
$(HOME)/.globus
\end{verbatim}
\end{enumerate}

\item[Example GSI Security Configuration]

Here is an example portion of the configuration file that would
enable and require GSI authentication,
along with a minimal set of other variables to make it work. 

\footnotesize
\begin{verbatim}
SEC_DEFAULT_AUTHENTICATION = REQUIRED
SEC_DEFAULT_AUTHENTICATION_METHODS = GSI
SEC_DEFAULT_INTEGRITY = REQUIRED
GSI_DAEMON_DIRECTORY = /etc/grid-security
GRIDMAP = /etc/grid-security/grid-mapfile

# authorize based on user names produced by the map file
ALLOW_READ = *@cs.wisc.edu/*.cs.wisc.edu
ALLOW_DAEMON = condor@cs.wisc.edu/*.cs.wisc.edu
ALLOW_NEGOTIATOR = condor@cs.wisc.edu/condor.cs.wisc.edu, \
                   condor@cs.wisc.edu/condor2.cs.wisc.edu
ALLOW_ADMINISTRATOR = condor-admin@cs.wisc.edu/*.cs.wisc.edu

# condor daemon certificate(s) trusted by condor tools and daemons
# when connecting to other condor daemons
GSI_DAEMON_NAME = /C=US/O=Condor/O=UW/OU=CS/CN=condor@cs.wisc.edu

# clear out any host-based authorizations
# (unnecessary if you leave authentication REQUIRED,
#  but useful if you make it optional and want to
#  allow some unauthenticated operations, such as
#  ALLOW_READ = */*.cs.wisc.edu)
HOSTALLOW_READ =
HOSTALLOW_WRITE =
HOSTALLOW_NEGOTIATOR =
HOSTALLOW_ADMINISTRATOR =
\end{verbatim}
\normalsize

The
\MacroNI{SEC\_DEFAULT\_AUTHENTICATION} macro specifies that
authentication is required for all communications.
This single macro covers all communications, but could be
replaced with a set of macros that require authentication for
only specific communications.

The macro \MacroNI{GSI\_DAEMON\_DIRECTORY} is specified
to give
HTCondor a single place to find the daemon's certificate.
This path may be a directory on a shared file system such as AFS. 
Alternatively, this path name can point to 
local copies of the certificate stored
in a local file system.

The macro \MacroNI{GRIDMAP} specifies the file
to use for mapping GSI names to user names within HTCondor.  For example,
it might look like this:

\footnotesize
\begin{verbatim}
"/C=US/O=Condor/O=UW/OU=CS/CN=condor@cs.wisc.edu" condor@cs.wisc.edu
\end{verbatim}
\normalsize

Additional mappings would be needed for the users who submit jobs to
the pool or who issue administrative commands.

\end{description}

%%%%%%%%%%%%%%%%%%%%%%%%%%%%%%%%%%%%%%%%%%%%%%%%%%%%%%%%%%%%%%%%%%%%%%
\subsubsection{\label{sec:SSL-Authentication}SSL Authentication}
%%%%%%%%%%%%%%%%%%%%%%%%%%%%%%%%%%%%%%%%%%%%%%%%%%%%%%%%%%%%%%%%%%%%%%
\index{authentication!SSL}
SSL authentication is similar to GSI authentication,
but without GSI's delegation (proxy) capabilities.
SSL utilizes X.509 certificates.

All SSL authentication is mutual authentication in HTCondor.
This means that when SSL authentication is used and when one process
communicates with another, each process must be able to verify the
signature on the certificate presented by the other process.  
The process that initiates the connection is the client,
and the process that receives the connection is the server.
For example, when a \Condor{startd} daemon
authenticates with a \Condor{collector} daemon
to provide a machine ClassAd,
the \Condor{startd} daemon initiates the connection and acts as the client,
and the \Condor{collector} daemon acts as the server.

The names and locations of keys and certificates for clients,
servers, and the files used to specify trusted certificate authorities
(CAs) are defined by settings in the configuration files.
The contents of the files are identical in format
and interpretation to those used by
other systems which use SSL, such as Apache httpd.

The configuration variables 
\Macro{AUTH\_SSL\_CLIENT\_CERTFILE} and \Macro{AUTH\_SSL\_SERVER\_CERTFILE}
specify the file location
for the certificate file for the initiator and recipient of connections,
respectively.
Similarly, the configuration variables
\Macro{AUTH\_SSL\_CLIENT\_KEYFILE} and \Macro{AUTH\_SSL\_SERVER\_KEYFILE}
specify the locations for keys.

The configuration variables 
\Macro{AUTH\_SSL\_SERVER\_CAFILE} and \Macro{AUTH\_SSL\_CLIENT\_CAFILE}
each specify a path and file name, providing the location
of a file containing one or more
certificates issued by trusted certificate authorities.
Similarly,
\Macro{AUTH\_SSL\_SERVER\_CADIR} and \Macro{AUTH\_SSL\_CLIENT\_CADIR}
each specify a directory with one or more files,
each which may contain a single CA certificate.  The directories
must be prepared using the OpenSSL \Code{c\_rehash} utility.

%%%%%%%%%%%%%%%%%%%%%%%%%%%%%%%%%%%%%%%%%%%%%%%%%%%%%%%%%%%%%%%%%%%%%%
\subsubsection{\label{sec:Kerberos-Authentication}Kerberos Authentication}
%%%%%%%%%%%%%%%%%%%%%%%%%%%%%%%%%%%%%%%%%%%%%%%%%%%%%%%%%%%%%%%%%%%%%%
\index{authentication!Kerberos}
\index{Kerberos authentication}

If Kerberos is used for authentication,
then a mapping from a
Kerberos domain (called a realm) to an HTCondor UID domain is necessary.
There are two ways to accomplish this mapping.
For a first way to specify the mapping, see
section~\ref{sec:Security-Unified-Map-File}
to use HTCondor's unified map file.
A second way to specify the mapping defines
the configuration variable
\Macro{KERBEROS\_MAP\_FILE}
to define a path to an administrator-maintained Kerberos-specific 
map file.
The configuration syntax is
\begin{verbatim}
KERBEROS_MAP_FILE = /path/to/etc/condor.kmap
\end{verbatim}

Lines within this map file have the syntax
\begin{verbatim}
   KERB.REALM = UID.domain.name
\end{verbatim}

Here are two lines from a map file to use as an example:
\begin{verbatim}
   CS.WISC.EDU   = cs.wisc.edu
   ENGR.WISC.EDU = ee.wisc.edu
\end{verbatim}

If a \MacroNI{KERBEROS\_MAP\_FILE}
configuration variable is defined and set,
then all permitted realms must be explicitly mapped.
If no map file is specified, then HTCondor assumes that the
Kerberos realm is the same as the HTCondor UID domain.

\index{authentication!Kerberos principal}
The configuration variable
\Macro{KERBEROS\_SERVER\_PRINCIPAL}
defines the name of a Kerberos principal.
If \MacroNI{KERBEROS\_SERVER\_PRINCIPAL} is not defined,
then the default value used is \verb@host@.
A principal specifies a unique name to which a set of credentials
may be assigned.

HTCondor takes the specified (or default) principal and appends
a slash character, the host name, an '@' (at sign character),
and the Kerberos realm.
As an example, the configuration
\begin{verbatim}
KERBEROS_SERVER_PRINCIPAL = condor-daemon
\end{verbatim}
results in HTCondor's use of
\begin{verbatim}
condor-daemon/the.host.name@YOUR.KERB.REALM
\end{verbatim}
as the server principal.

Here is
an example of configuration settings that use Kerberos for
authentication and require authentication of all communications
of the write or administrator access level.
\footnotesize
\begin{verbatim}
SEC_WRITE_AUTHENTICATION                 = REQUIRED
SEC_WRITE_AUTHENTICATION_METHODS         = KERBEROS
SEC_ADMINISTRATOR_AUTHENTICATION         = REQUIRED
SEC_ADMINISTRATOR_AUTHENTICATION_METHODS = KERBEROS
\end{verbatim}
\normalsize

Kerberos authentication on Unix platforms
requires access to various files that
usually are only accessible by the root user.
At this time,
the only supported way to use KERBEROS authentication on Unix platforms
is to start daemons HTCondor as user \Login{root}.

%%%%%%%%%%%%%%%%%%%%%%%%%%%%%%%%%%%%%%%%%%%%%%%%%%%%%%%%%%%%%%%%%%%%%%
\subsubsection{\label{sec:Password-Authentication} Password Authentication}
%%%%%%%%%%%%%%%%%%%%%%%%%%%%%%%%%%%%%%%%%%%%%%%%%%%%%%%%%%%%%%%%%%%%%%

The password method provides mutual authentication through the use of
a shared secret.  This is often a good choice when strong security is
desired,
but an existing Kerberos or X.509 infrastructure is not in place.
Password authentication is available on both Unix and Windows.
It currently can only be used for daemon-to-daemon authentication.
The shared secret in this context is referred to as
the \Term{pool password}.

Before a daemon can use password authentication, the pool password
must be stored on the daemon's local machine.
On Unix, the password
will be placed in a file defined by the configuration variable
\Macro{SEC\_PASSWORD\_FILE}. This file will be accessible only by the
UID that HTCondor is started as.  On Windows, the same secure password
store that is used for user passwords will be used for the pool
password (see section \ref{sec:windows-sps}).

Under Unix, the password file can be generated by using the following
command to write directly to the password file:
\begin{verbatim}
condor_store_cred -f /path/to/password/file
\end{verbatim}

Under Windows (or under Unix), storing the pool password is done
with the \Opt{-c} option when using to \Condor{store\_cred} \Opt{add}.
Running
\begin{verbatim}
condor_store_cred -c add
\end{verbatim}
prompts for the pool password and store it on the local machine,
making it available for daemons to use in authentication. The
\Condor{master} must be running for this command to work.

In addition, storing the pool password to a given machine requires
\verb@CONFIG@-level access. For example, if the pool password should
only be set locally, and only by root, the following would be placed in
the global configuration file.
\begin{verbatim}
ALLOW_CONFIG = root@mydomain/$(IP_ADDRESS)
\end{verbatim}

It is also possible to set the pool password remotely, but this is
recommended only if it can be done over an encrypted channel.  This is
possible on Windows, for example, in an environment where common
accounts exist across all the machines in the pool. In this case,
\verb@ALLOW_CONFIG@ can be set to allow the HTCondor administrator (who
in this example has an account \Login{condor} common to all machines in
the pool) to set the password from the central manager as follows.
\begin{verbatim}
ALLOW_CONFIG = condor@mydomain/$(CONDOR_HOST)
\end{verbatim}
The HTCondor administrator then executes
\begin{verbatim}
condor_store_cred -c -n host.mydomain add
\end{verbatim}
from the central manager to store the password to a given machine.
Since the \Login{condor} account exists on both the central manager and
\verb@host.mydomain@, the NTSSPI authentication method can be used to
authenticate and encrypt the connection.  \Condor{store\_cred} will
warn and prompt for cancellation, if the channel is not encrypted for
whatever reason (typically because common accounts do not exist or
HTCondor's security is misconfigured).

When a daemon is authenticated using a pool password, its security
principle is \verb|condor_pool@$(UID_DOMAIN)|, where
\verb@$(UID_DOMAIN)@ is taken from the daemon's configuration.  The
\verb@ALLOW_DAEMON@ and \verb@ALLOW_NEGOTIATOR@ configuration
variables for authorization
should restrict access using this name. For example,
\begin{verbatim}
ALLOW_DAEMON = condor_pool@mydomain/*, condor@mydomain/$(IP_ADDRESS)
ALLOW_NEGOTIATOR = condor_pool@mydomain/$(CONDOR_HOST)
\end{verbatim}
This configuration allows remote \verb@DAEMON@-level and
\verb@NEGOTIATOR@-level access, if the pool password is known.  Local
daemons authenticated as \verb|condor@mydomain| are also allowed
access. This is done so local authentication can be done using
another method such as \verb@FS@.


\begin{description}
\item[Example Security Configuration Using Pool Password]
\index{security!sample configuration using pool password}
The following example configuration uses pool password authentication and
network message integrity checking for all communication between HTCondor
daemons.

\begin{verbatim}
SEC_PASSWORD_FILE = $(LOCK)/pool_password
SEC_DAEMON_AUTHENTICATION = REQUIRED
SEC_DAEMON_INTEGRITY = REQUIRED
SEC_DAEMON_AUTHENTICATION_METHODS = PASSWORD
SEC_NEGOTIATOR_AUTHENTICATION = REQUIRED
SEC_NEGOTIATOR_INTEGRITY = REQUIRED
SEC_NEGOTIATOR_AUTHENTICATION_METHODS = PASSWORD
SEC_CLIENT_AUTHENTICATION_METHODS = FS, PASSWORD, KERBEROS, GSI
ALLOW_DAEMON = condor_pool@$(UID_DOMAIN)/*.cs.wisc.edu, \
               condor@$(UID_DOMAIN)/$(IP_ADDRESS)
ALLOW_NEGOTIATOR = condor_pool@$(UID_DOMAIN)/negotiator.machine.name
\end{verbatim}


\item[Example Using Pool Password for \Condor{startd} Advertisement]
\index{security!sample configuration using pool password for startd advertisement}

One problem with the pool password method of authentication is that it
involves a single, shared secret.
This does not scale well with the addition of
remote users who flock to the local pool.
However, the pool password may still be used for authenticating portions
of the local pool, while others
(such as the remote \Condor{schedd} daemons involved in flocking)
are authenticated by other means.

In this example, only the \Condor{startd} daemons in the local pool
are required to have the
pool password when they advertise themselves to the \Condor{collector} daemon.

\begin{verbatim}
SEC_PASSWORD_FILE = $(LOCK)/pool_password
SEC_ADVERTISE_STARTD_AUTHENTICATION = REQUIRED
SEC_ADVERTISE_STARTD_INTEGRITY = REQUIRED
SEC_ADVERTISE_STARTD_AUTHENTICATION_METHODS = PASSWORD
SEC_CLIENT_AUTHENTICATION_METHODS = FS, PASSWORD, KERBEROS, GSI
ALLOW_ADVERTISE_STARTD = condor_pool@$(UID_DOMAIN)/*.cs.wisc.edu
\end{verbatim}

\end{description}

%%%%%%%%%%%%%%%%%%%%%%%%%%%%%%%%%%%%%%%%%%%%%%%%%%%%%%%%%%%%%%%%%%%%%%
\subsubsection{\label{sec:FS-Authentication}File System Authentication}
%%%%%%%%%%%%%%%%%%%%%%%%%%%%%%%%%%%%%%%%%%%%%%%%%%%%%%%%%%%%%%%%%%%%%%

\index{authentication!using a file system}
This form of authentication utilizes the ownership of a file
in the identity verification of a client.
A daemon authenticating a client requires the client to write
a file in a specific location (\File{/tmp}).
The daemon then checks the ownership of the file.
The file's ownership verifies the identity of the client.
In this way, the file system becomes the trusted authority.
This authentication method is only appropriate for clients and daemons
that are on the same computer. 

%%%%%%%%%%%%%%%%%%%%%%%%%%%%%%%%%%%%%%%%%%%%%%%%%%%%%%%%%%%%%%%%%%%%%%
\subsubsection{\label{sec:FSR-Authentication}File System Remote Authentication}
%%%%%%%%%%%%%%%%%%%%%%%%%%%%%%%%%%%%%%%%%%%%%%%%%%%%%%%%%%%%%%%%%%%%%%
\index{authentication!using a remote file system}
Like file system authentication,
this form of authentication utilizes the ownership of a file
in the identity verification of a client.
In this case,
a daemon authenticating a client requires the client to write
a file in a specific location,
but the location is not restricted to \File{/tmp}.
The location of the file is specified by the configuration
variable \Macro{FS\_REMOTE\_DIR}.

% %%%%%%%%%%%%%%%%%%%%%%%%%%%%%%%%%%%%%%%%%%%%%%%%%%%%%%%%%%%%%%%%%%%%%%
% \index{authentication!using a remote file system}
% \Todo

% %%%%%%%%%%%%%%%%%%%%%%%%%%%%%%%%%%%%%%%%%%%%%%%%%%%%%%%%%%%%%%%%%%%%%%
% \subsubsection{\label{sec:Passwd-Authentication}Password Authentication}
% %%%%%%%%%%%%%%%%%%%%%%%%%%%%%%%%%%%%%%%%%%%%%%%%%%%%%%%%%%%%%%%%%%%%%%
% \index{authentication!using a password}
% Authentication can be done interactively.
% This method of authentication can only be used for verifying the
% identity of a client.
% It requires a command line ?,
% and the client is prompted for a password.

%%%%%%%%%%%%%%%%%%%%%%%%%%%%%%%%%%%%%%%%%%%%%%%%%%%%%%%%%%%%%%%%%%%%%%
\subsubsection{\label{sec:NTSSPI-Authentication}Windows Authentication}
%%%%%%%%%%%%%%%%%%%%%%%%%%%%%%%%%%%%%%%%%%%%%%%%%%%%%%%%%%%%%%%%%%%%%%
\index{authentication!Windows}
This authentication is done only among Windows machines using
a proprietary method.
The Windows security interface SSPI is used to enforce NTLM
(NT LAN Manager).
The authentication is based on challenge and response, using the user's
password as a key.
This is similar to Kerberos.
The main difference 
is that Kerberos provides an access token that typically grants
access to an entire network, whereas NTLM authentication only 
verifies an identity to one machine at a time.
NTSSPI is best-used in a way similar to file system authentication in
Unix, and probably should not be used for authentication between two
computers. 

%%%%%%%%%%%%%%%%%%%%%%%%%%%%%%%%%%%%%%%%%%%%%%%%%%%%%%%%%%%%%%%%%%%%%%
\subsubsection{\label{sec:CLAIM-Authentication}Claim To Be Authentication}
%%%%%%%%%%%%%%%%%%%%%%%%%%%%%%%%%%%%%%%%%%%%%%%%%%%%%%%%%%%%%%%%%%%%%%
Claim To Be authentication accepts any identity claimed by the client.
As such, it does not authenticate.
It is included in HTCondor and in the list of authentication methods
for testing purposes only.

%%%%%%%%%%%%%%%%%%%%%%%%%%%%%%%%%%%%%%%%%%%%%%%%%%%%%%%%%%%%%%%%%%%%%%
\subsubsection{\label{sec:ANON-Authentication}Anonymous Authentication}
%%%%%%%%%%%%%%%%%%%%%%%%%%%%%%%%%%%%%%%%%%%%%%%%%%%%%%%%%%%%%%%%%%%%%%
Anonymous authentication causes authentication to be skipped entirely.
As such, it does not authenticate.
It is included in HTCondor and in the list of authentication methods
for testing purposes only.

\index{authentication|)}

%%%%%%%%%%%%%%%%%%%%%%%%%%%%%%%%%%%%%%%%%%%%%%%%%%%%%%%%%%%%%%%%%%%%%%
\subsection{\label{sec:Security-Unified-Map-File}The Unified Map File for Authentication}
%%%%%%%%%%%%%%%%%%%%%%%%%%%%%%%%%%%%%%%%%%%%%%%%%%%%%%%%%%%%%%%%%%%%%%
\index{security!unified map file}

HTCondor's unified map file allows the mappings
from authenticated names to an HTCondor canonical user name
to be specified as a single list within a single file. 
The location of the unified map file is defined 
by the configuration variable
\Macro{CERTIFICATE\_MAPFILE}; it specifies the
path and file name of the unified map file.
Each mapping is on its own line of the unified map file.
Each line contains 3 fields, separated by white space 
(space or tab characters):
\begin{enumerate}
\item{The name of the authentication method to which the mapping applies.}
\item{A regular expression representing the authenticated name
to be mapped.}
\item{The canonical HTCondor user name.}
\end{enumerate}

Allowable authentication method names are the same as used to define
any of the configuration variables \MacroNI{SEC\_*\_AUTHENTICATION\_METHODS},
as repeated here:
\begin{verbatim}
    GSI
    SSL
    KERBEROS
    PASSWORD
    FS
    FS_REMOTE
    NTSSPI
    CLAIMTOBE
    ANONYMOUS
\end{verbatim}

The fields that represent an authenticated name and the canonical
HTCondor user name may utilize regular expressions as defined 
by PCRE (Perl-Compatible Regular Expressions).
Due to this, more than one line (mapping) within the unified
map file may match.
Look ups are therefore defined to use the first mapping that
matches.

A regular expression may need to contain spaces, and in this case the
entire expression can be surrounded by double quote marks. If a
double quote character also needs to appear in such an expression, it
is preceded by a backslash.

The default behavior of HTCondor when no map file is specified is to
do the following mappings, with some additional logic noted below:
\begin{verbatim}
FS (.*) \1
FS_REMOTE (.*) \1
GSI (.*) GSS_ASSIST_GRIDMAP
SSL (.*) ssl@unmapped
KERBEROS ([^/]*)/?[^@]*@(.*) \1@\2
NTSSPI (.*) \1
CLAIMTOBE (.*) \1
PASSWORD (.*) \1
\end{verbatim}

For GSI (or SSL), the special name \MacroNI{GSS\_ASSIST\_GRIDMAP} instructs 
HTCondor to use the GSI grid map file (configured with \Macro{GRIDMAP}
as shown in section~\ref{sec:GSI-Authentication}) to do the mapping.
If no mapping can be found for GSI (with or without the use of
\MacroNI{GSS\_ASSIST\_GRIDMAP}), the user is mapped to \verb|gsi@unmapped|.

For Kerberos, if \Macro{KERBEROS\_MAP\_FILE} is specified, the domain portion
of the name is obtained by mapping the Kerberos realm to the value specified
in the map file, rather than just using the realm verbatim as the domain
portion of the condor user name.  See section~\ref{sec:Kerberos-Authentication}
for details.

\index{unauthenticated}
\index{unmapped}
If authentication did not happen or failed and was not required, then
the user is given the name \verb|unauthenticated@unmapped|.

With the integration of VOMS for GSI authentication,
the interpretation of the regular expression
representing the authenticated name may change.
First, the full serialized DN and FQAN are used in attempting a match.
If no match is found using the full DN and FQAN,
then the DN is then used on its own without the FQAN.
Using this, roles or user names from the VOMS attributes may be extracted
to be used as the target for mapping.
And, in this case the FQAN are verified, permitting reliance
on their authenticity.

%%%%%%%%%%%%%%%%%%%%%%%%%%%%%%%%%%%%%%%%%%%%%%%%%%%%%%%%%%%%%%%%%%%%%%
\subsection{\label{sec:Security-Encryption}Encryption}
%%%%%%%%%%%%%%%%%%%%%%%%%%%%%%%%%%%%%%%%%%%%%%%%%%%%%%%%%%%%%%%%%%%%%%
\index{security!encryption}
Encryption provides privacy support between two communicating parties.
Through configuration macros, both the client and the daemon
can specify whether encryption is required for further communication.

The client uses one of two macros to enable or disable encryption:
\index{SEC\_DEFAULT\_ENCRYPTION macro@\texttt{SEC\_DEFAULT\_ENCRYPTION} macro}
\index{configuration macro!\texttt{SEC\_DEFAULT\_ENCRYPTION}}
\index{SEC\_CLIENT\_ENCRYPTION macro@\texttt{SEC\_CLIENT\_ENCRYPTION} macro}
\index{configuration macro!\texttt{SEC\_CLIENT\_ENCRYPTION}}
\begin{verbatim}
    SEC_DEFAULT_ENCRYPTION
    SEC_CLIENT_ENCRYPTION
\end{verbatim}

For the daemon, there are seven macros to enable or disable encryption:
\index{SEC\_DEFAULT\_ENCRYPTION macro@\texttt{SEC\_DEFAULT\_ENCRYPTION} macro}
\index{configuration macro!\texttt{SEC\_DEFAULT\_ENCRYPTION}}
\index{SEC\_READ\_ENCRYPTION macro@\texttt{SEC\_READ\_ENCRYPTION} macro}
\index{configuration macro!\texttt{SEC\_READ\_ENCRYPTION}}
\index{SEC\_WRITE\_ENCRYPTION macro@\texttt{SEC\_WRITE\_ENCRYPTION} macro}
\index{configuration macro!\texttt{SEC\_WRITE\_ENCRYPTION}}
\index{SEC\_ADMINISTRATOR\_ENCRYPTION macro@\texttt{SEC\_ADMINISTRATOR\_ENCRYPTION} macro}
\index{configuration macro!\texttt{SEC\_ADMINISTRATOR\_ENCRYPTION}}
\index{SEC\_DAEMON\_ENCRYPTION macro@\texttt{SEC\_DAEMON\_ENCRYPTION} macro}
\index{configuration macro!\texttt{SEC\_DAEMON\_ENCRYPTION}}
\index{SEC\_CONFIG\_ENCRYPTION macro@\texttt{SEC\_CONFIG\_ENCRYPTION} macro}
\index{configuration macro!\texttt{SEC\_CONFIG\_ENCRYPTION}}
\index{SEC\_OWNER\_ENCRYPTION macro@\texttt{SEC\_OWNER\_ENCRYPTION} macro}
\index{configuration macro!\texttt{SEC\_OWNER\_ENCRYPTION}}
\index{SEC\_NEGOTIATOR\_ENCRYPTION macro@\texttt{SEC\_NEGOTIATOR\_ENCRYPTION} macro}
\index{configuration macro!\texttt{SEC\_OWNER\_ENCRYPTION}}
\index{SEC\_ADVERTISE\_MASTER\_ENCRYPTION macro@\texttt{SEC\_ADVERTISE\_MASTER\_ENCRYPTION} macro}
\index{configuration macro!\texttt{SEC\_ADVERTISE\_MASTER\_ENCRYPTION}}
\index{SEC\_ADVERTISE\_STARTD\_ENCRYPTION macro@\texttt{SEC\_ADVERTISE\_STARTD\_ENCRYPTION} macro}
\index{configuration macro!\texttt{SEC\_ADVERTISE\_STARTD\_ENCRYPTION}}
\index{SEC\_ADVERTISE\_SCHEDD\_ENCRYPTION macro@\texttt{SEC\_ADVERTISE\_SCHEDD\_ENCRYPTION} macro}
\index{configuration macro!\texttt{SEC\_ADVERTISE\_SCHEDD\_ENCRYPTION}}
\begin{verbatim}
    SEC_DEFAULT_ENCRYPTION
    SEC_READ_ENCRYPTION
    SEC_WRITE_ENCRYPTION
    SEC_ADMINISTRATOR_ENCRYPTION
    SEC_CONFIG_ENCRYPTION
    SEC_OWNER_ENCRYPTION
    SEC_DAEMON_ENCRYPTION
    SEC_NEGOTIATOR_ENCRYPTION
    SEC_ADVERTISE_MASTER_ENCRYPTION
    SEC_ADVERTISE_STARTD_ENCRYPTION
    SEC_ADVERTISE_SCHEDD_ENCRYPTION
\end{verbatim}

As an example, the macro defined in the configuration file
for a daemon as
\begin{verbatim}
SEC_CONFIG_ENCRYPTION = REQUIRED
\end{verbatim}
signifies that any communication that changes a daemon's configuration
must be encrypted.
If a daemon's configuration contains
\begin{verbatim}
SEC_DEFAULT_ENCRYPTION = REQUIRED
\end{verbatim}
and does not contain any other security configuration for
ENCRYPTION, then this default defines the daemon's needs
for encryption over all access levels.
Where a specific macro is present, its value takes
precedence over any default given.

If encryption is to be done, then the communicating parties
must find (negotiate) a mutually acceptable method of
encryption to be used.
A list of acceptable methods may be provided by the client, using the
macros
\index{SEC\_DEFAULT\_CRYPTO\_METHODS macro@\texttt{SEC\_DEFAULT\_CRYPTO\_METHODS} macro}
\index{configuration macro!\texttt{SEC\_DEFAULT\_CRYPTO\_METHODS}}
\index{SEC\_CLIENT\_CRYPTO\_METHODS macro@\texttt{SEC\_CLIENT\_CRYPTO\_METHODS} macro}
\index{configuration macro!\texttt{SEC\_CLIENT\_CRYPTO\_METHODS}}
\begin{verbatim}
    SEC_DEFAULT_CRYPTO_METHODS
    SEC_CLIENT_CRYPTO_METHODS
\end{verbatim}
A list of acceptable methods may be provided by the daemon, using the
macros
\index{SEC\_DEFAULT\_CRYPTO\_METHODS macro@\texttt{SEC\_DEFAULT\_CRYPTO\_METHODS} macro}
\index{configuration macro!\texttt{SEC\_DEFAULT\_CRYPTO\_METHODS}}
\index{SEC\_READ\_CRYPTO\_METHODS macro@\texttt{SEC\_READ\_CRYPTO\_METHODS} macro}
\index{configuration macro!\texttt{SEC\_READ\_CRYPTO\_METHODS}}
\index{SEC\_WRITE\_CRYPTO\_METHODS macro@\texttt{SEC\_WRITE\_CRYPTO\_METHODS} macro}
\index{configuration macro!\texttt{SEC\_WRITE\_CRYPTO\_METHODS}}
\index{SEC\_ADMINISTRATOR\_CRYPTO\_METHODS macro@\texttt{SEC\_ADMINISTRATOR\_CRYPTO\_METHODS} macro}
\index{configuration macro!\texttt{SEC\_ADMINISTRATOR\_CRYPTO\_METHODS}}
\index{SEC\_DAEMON\_CRYPTO\_METHODS macro@\texttt{SEC\_DAEMON\_CRYPTO\_METHODS} macro}
\index{configuration macro!\texttt{SEC\_DAEMON\_CRYPTO\_METHODS}}
\index{SEC\_CONFIG\_CRYPTO\_METHODS macro@\texttt{SEC\_CONFIG\_CRYPTO\_METHODS} macro}
\index{configuration macro!\texttt{SEC\_CONFIG\_CRYPTO\_METHODS}}
\index{SEC\_OWNER\_CRYPTO\_METHODS macro@\texttt{SEC\_OWNER\_CRYPTO\_METHODS} macro}
\index{configuration macro!\texttt{SEC\_OWNER\_CRYPTO\_METHODS}}
\index{SEC\_NEGOTIATOR\_CRYPTO\_METHODS macro@\texttt{SEC\_NEGOTIATOR\_CRYPTO\_METHODS} macro}
\index{configuration macro!\texttt{SEC\_NEGOTIATOR\_CRYPTO\_METHODS}}
\index{SEC\_ADVERTISE\_MASTER\_CRYPTO\_METHODS macro@\texttt{SEC\_ADVERTISE\_MASTER\_CRYPTO\_METHODS} macro}
\index{configuration macro!\texttt{SEC\_ADVERTISE\_MASTER\_CRYPTO\_METHODS}}
\index{SEC\_ADVERTISE\_STARTD\_CRYPTO\_METHODS macro@\texttt{SEC\_ADVERTISE\_STARTD\_CRYPTO\_METHODS} macro}
\index{configuration macro!\texttt{SEC\_ADVERTISE\_STARTD\_CRYPTO\_METHODS}}
\index{SEC\_ADVERTISE\_SCHEDD\_CRYPTO\_METHODS macro@\texttt{SEC\_ADVERTISE\_SCHEDD\_CRYPTO\_METHODS} macro}
\index{configuration macro!\texttt{SEC\_ADVERTISE\_SCHEDD\_CRYPTO\_METHODS}}
\begin{verbatim}
    SEC_DEFAULT_CRYPTO_METHODS
    SEC_READ_CRYPTO_METHODS
    SEC_WRITE_CRYPTO_METHODS
    SEC_ADMINISTRATOR_CRYPTO_METHODS
    SEC_CONFIG_CRYPTO_METHODS
    SEC_OWNER_CRYPTO_METHODS
    SEC_DAEMON_CRYPTO_METHODS
    SEC_NEGOTIATOR_CRYPTO_METHODS
    SEC_ADVERTISE_MASTER_CRYPTO_METHODS
    SEC_ADVERTISE_STARTD_CRYPTO_METHODS
    SEC_ADVERTISE_SCHEDD_CRYPTO_METHODS
\end{verbatim}

The methods are
given as a comma-separated list of acceptable values.
These variables list the encryption methods that are available
to be used.
The ordering of the list gives preference;
the first item in the list indicates the highest preference.
Possible values are
\begin{verbatim}
    3DES
    BLOWFISH
\end{verbatim}

%%%%%%%%%%%%%%%%%%%%%%%%%%%%%%%%%%%%%%%%%%%%%%%%%%%%%%%%%%%%%%%%%%%%%%
\subsection{\label{sec:Security-Integrity}Integrity}
%%%%%%%%%%%%%%%%%%%%%%%%%%%%%%%%%%%%%%%%%%%%%%%%%%%%%%%%%%%%%%%%%%%%%%
\index{security!integrity}

An integrity check assures that the messages between communicating parties
have not been tampered with.
Any change, such as addition, modification, or deletion can
be detected.
Through configuration macros, both the client and the daemon
can specify whether an integrity check is required of further communication.

The client uses one of two macros to enable or disable an integrity check:
\index{SEC\_DEFAULT\_INTEGRITY macro@\texttt{SEC\_DEFAULT\_INTEGRITY} macro}
\index{configuration macro!\texttt{SEC\_DEFAULT\_INTEGRITY}}
\index{SEC\_CLIENT\_INTEGRITY macro@\texttt{SEC\_CLIENT\_INTEGRITY} macro}
\index{configuration macro!\texttt{SEC\_CLIENT\_INTEGRITY}}
\begin{verbatim}
    SEC_DEFAULT_INTEGRITY
    SEC_CLIENT_INTEGRITY
\end{verbatim}

For the daemon, there are seven macros to enable or disable an integrity check:
\index{SEC\_DEFAULT\_INTEGRITY macro@\texttt{SEC\_DEFAULT\_INTEGRITY} macro}
\index{configuration macro!\texttt{SEC\_DEFAULT\_INTEGRITY}}
\index{SEC\_READ\_INTEGRITY macro@\texttt{SEC\_READ\_INTEGRITY} macro}
\index{configuration macro!\texttt{SEC\_READ\_INTEGRITY}}
\index{SEC\_WRITE\_INTEGRITY macro@\texttt{SEC\_WRITE\_INTEGRITY} macro}
\index{configuration macro!\texttt{SEC\_WRITE\_INTEGRITY}}
\index{SEC\_ADMINISTRATOR\_INTEGRITY macro@\texttt{SEC\_ADMINISTRATOR\_INTEGRITY} macro}
\index{configuration macro!\texttt{SEC\_ADMINISTRATOR\_INTEGRITY}}
\index{SEC\_DAEMON\_INTEGRITY macro@\texttt{SEC\_DAEMON\_INTEGRITY} macro}
\index{configuration macro!\texttt{SEC\_DAEMON\_INTEGRITY}}
\index{SEC\_CONFIG\_INTEGRITY macro@\texttt{SEC\_CONFIG\_INTEGRITY} macro}
\index{configuration macro!\texttt{SEC\_CONFIG\_INTEGRITY}}
\index{SEC\_OWNER\_INTEGRITY macro@\texttt{SEC\_OWNER\_INTEGRITY} macro}
\index{configuration macro!\texttt{SEC\_OWNER\_INTEGRITY}}
\index{SEC\_NEGOTIATOR\_INTEGRITY macro@\texttt{SEC\_NEGOTIATOR\_INTEGRITY} macro}
\index{configuration macro!\texttt{SEC\_NEGOTIATOR\_INTEGRITY}}
\index{SEC\_ADVERTISE\_MASTER\_INTEGRITY macro@\texttt{SEC\_ADVERTISE\_MASTER\_INTEGRITY} macro}
\index{configuration macro!\texttt{SEC\_ADVERTISE\_MASTER\_INTEGRITY}}
\index{SEC\_ADVERTISE\_STARTD\_INTEGRITY macro@\texttt{SEC\_ADVERTISE\_STARTD\_INTEGRITY} macro}
\index{configuration macro!\texttt{SEC\_ADVERTISE\_STARTD\_INTEGRITY}}
\index{SEC\_ADVERTISE\_SCHEDD\_INTEGRITY macro@\texttt{SEC\_ADVERTISE\_SCHEDD\_INTEGRITY} macro}
\index{configuration macro!\texttt{SEC\_ADVERTISE\_SCHEDD\_INTEGRITY}}
\begin{verbatim}
    SEC_DEFAULT_INTEGRITY
    SEC_READ_INTEGRITY
    SEC_WRITE_INTEGRITY
    SEC_ADMINISTRATOR_INTEGRITY
    SEC_CONFIG_INTEGRITY
    SEC_OWNER_INTEGRITY
    SEC_DAEMON_INTEGRITY
    SEC_NEGOTIATOR_INTEGRITY
    SEC_ADVERTISE_MASTER_INTEGRITY
    SEC_ADVERTISE_STARTD_INTEGRITY
    SEC_ADVERTISE_SCHEDD_INTEGRITY
\end{verbatim}

As an example, the macro defined in the configuration file
for a daemon as
\begin{verbatim}
SEC_CONFIG_INTEGRITY = REQUIRED
\end{verbatim}
signifies that any communication that changes a daemon's configuration
must have its integrity assured.
If a daemon's configuration contains
\begin{verbatim}
SEC_DEFAULT_INTEGRITY = REQUIRED
\end{verbatim}
and does not contain any other security configuration for
\verb@INTEGRITY@, then this default defines the daemon's needs
for integrity checks over all access levels.
Where a specific macro is present, its value takes
precedence over any default given.

A signed MD5 check sum is currently the only available method
for integrity checking.
Its use is implied whenever integrity checks occur.
If more methods are implemented, then there will be further
macros to allow both the client and the daemon to specify
which methods are acceptable.

%%%%%%%%%%%%%%%%%%%%%%%%%%%%%%%%%%%%%%%%%%%%%%%%%%%%%%%%%%%%%%%%%%%%%%
\subsection{\label{sec:Security-Authorization}Authorization}
%%%%%%%%%%%%%%%%%%%%%%%%%%%%%%%%%%%%%%%%%%%%%%%%%%%%%%%%%%%%%%%%%%%%%%
\index{security!authorization}
\index{authorization!for security}
\index{security!based on user authorization}

Authorization protects resource usage by granting or denying
access requests made to the resources.
It defines who is allowed to do what.

Authorization is defined in terms of users.
An initial implementation provided authorization
based on hosts (machines), while the current implementation
relies on user-based authorization.
Section~\ref{sec:Host-Security}
on Setting Up IP/Host-Based Security in HTCondor describes the
previous implementation.
This IP/Host-Based security still exists, and it can be used,
but significantly stronger and more flexible
security can be achieved with the newer
authorization based on fully qualified user names.
This section discusses user-based authorization. 

% format for the configuration macro index entries
%\index{ macro@\texttt{} macro}
%\index{configuration macro!\texttt{}}

The authorization portion of the security of an HTCondor pool is
based on a set of configuration macros.
The macros list which user will be authorized
to issue what request given a specific access level.
When a daemon is to be authorized, its user name is the 
login under which the daemon is executed.

These configuration macros define a set of users that will be
allowed to (or denied from) carrying out various HTCondor commands.
Each access level may have its own list of authorized users.
A complete list of the authorization macros:
\index{ALLOW\_READ macro@\texttt{ALLOW\_READ} macro}
\index{configuration macro!\texttt{ALLOW\_READ}}
\index{ALLOW\_WRITE macro@\texttt{ALLOW\_WRITE} macro}
\index{configuration macro!\texttt{ALLOW\_WRITE}}
\index{ALLOW\_ADMINISTRATOR macro@\texttt{ALLOW\_ADMINISTRATOR} macro}
\index{configuration macro!\texttt{ALLOW\_ADMINISTRATOR}}
\index{ALLOW\_SOAP macro@\texttt{ALLOW\_SOAP} macro}
\index{configuration macro!\texttt{ALLOW\_SOAP}}
\index{ALLOW\_CONFIG macro@\texttt{ALLOW\_CONFIG} macro}
\index{configuration macro!\texttt{ALLOW\_CONFIG}}
\index{ALLOW\_DAEMON macro@\texttt{ALLOW\_DAEMON} macro}
\index{configuration macro!\texttt{ALLOW\_DAEMON}}
\index{ALLOW\_OWNER macro@\texttt{ALLOW\_OWNER} macro}
\index{configuration macro!\texttt{ALLOW\_OWNER}}
\index{ALLOW\_NEGOTIATOR macro@\texttt{ALLOW\_NEGOTIATOR} macro}
\index{configuration macro!\texttt{ALLOW\_NEGOTIATOR}}
\index{DENY\_READ macro@\texttt{DENY\_READ} macro}
\index{configuration macro!\texttt{DENY\_READ}}
\index{DENY\_WRITE macro@\texttt{DENY\_WRITE} macro}
\index{configuration macro!\texttt{DENY\_WRITE}}
\index{DENY\_ADMINISTRATOR macro@\texttt{DENY\_ADMINISTRATOR} macro}
\index{configuration macro!\texttt{DENY\_ADMINISTRATOR}}
\index{DENY\_SOAP macro@\texttt{DENY\_SOAP} macro}
\index{configuration macro!\texttt{DENY\_SOAP}}
\index{DENY\_CONFIG macro@\texttt{DENY\_CONFIG} macro}
\index{configuration macro!\texttt{DENY\_CONFIG}}
\index{DENY\_DAEMON macro@\texttt{DENY\_DAEMON} macro}
\index{configuration macro!\texttt{DENY\_DAEMON}}
\index{DENY\_OWNER macro@\texttt{DENY\_OWNER} macro}
\index{configuration macro!\texttt{DENY\_OWNER}}
\index{DENY\_NEGOTIATOR macro@\texttt{DENY\_NEGOTIATOR} macro}
\index{configuration macro!\texttt{DENY\_NEGOTIATOR}}
\begin{verbatim}
    ALLOW_READ
    ALLOW_WRITE
    ALLOW_ADMINISTRATOR
    ALLOW_CONFIG
    ALLOW_SOAP
    ALLOW_OWNER
    ALLOW_NEGOTIATOR
    ALLOW_DAEMON
    DENY_READ
    DENY_WRITE
    DENY_ADMINISTRATOR
    DENY_SOAP
    DENY_CONFIG
    DENY_OWNER
    DENY_NEGOTIATOR
    DENY_DAEMON
\end{verbatim}

In addition, the following are used to control authorization of
specific types of HTCondor daemons when advertising themselves to the
pool.  If unspecified, these default to the broader
\MacroNI{ALLOW\_DAEMON} and \MacroNI{DENY\_DAEMON} settings.

\index{ALLOW\_ADVERTISE\_MASTER macro@\texttt{ALLOW\_ADVERTISE\_MASTER} macro}
\index{configuration macro!\texttt{ALLOW\_ADVERTISE\_MASTER}}
\index{ALLOW\_ADVERTISE\_STARTD macro@\texttt{ALLOW\_ADVERTISE\_STARTD} macro}
\index{configuration macro!\texttt{ALLOW\_ADVERTISE\_STARTD}}
\index{ALLOW\_ADVERTISE\_SCHEDD macro@\texttt{ALLOW\_ADVERTISE\_SCHEDD} macro}
\index{configuration macro!\texttt{ALLOW\_ADVERTISE\_SCHEDD}}
\index{DENY\_ADVERTISE\_MASTER macro@\texttt{DENY\_ADVERTISE\_MASTER} macro}
\index{configuration macro!\texttt{DENY\_ADVERTISE\_MASTER}}
\index{DENY\_ADVERTISE\_STARTD macro@\texttt{DENY\_ADVERTISE\_STARTD} macro}
\index{configuration macro!\texttt{DENY\_ADVERTISE\_STARTD}}
\index{DENY\_ADVERTISE\_SCHEDD macro@\texttt{DENY\_ADVERTISE\_SCHEDD} macro}
\index{configuration macro!\texttt{DENY\_ADVERTISE\_SCHEDD}}
\begin{verbatim}
    ALLOW_ADVERTISE_MASTER
    ALLOW_ADVERTISE_STARTD
    ALLOW_ADVERTISE_SCHEDD
    DENY_ADVERTISE_MASTER
    DENY_ADVERTISE_STARTD
    DENY_ADVERTISE_SCHEDD
\end{verbatim}


Each client side of a connection may also specify its own list of
trusted servers.  This is done using the following settings.  Note
that the FS and CLAIMTOBE authentication methods are not symmetric.
The client is authenticated by the server, but the server is not
authenticated by the client.  When the server is not authenticated to
the client, only the network address of the host may be authorized and
not the specific identity of the server.

\index{ALLOW\_CLIENT macro@\texttt{ALLOW\_CLIENT} macro}
\index{configuration macro!\texttt{ALLOW\_CLIENT}}
\index{DENY\_CLIENT macro@\texttt{DENY\_CLIENT} macro}
\index{configuration macro!\texttt{DENY\_CLIENT}}
\begin{verbatim}
  ALLOW_CLIENT
  DENY_CLIENT
\end{verbatim}

The names \MacroNI{ALLOW\_CLIENT} and \MacroNI{DENY\_CLIENT} should
be thought of as
``when I am acting as a client, these are the servers I allow or deny.''
It should \emph{not} be confused with the incorrect thought
``when I am the server, these are the clients I allow or deny.''

All authorization settings are defined by a comma-separated list of
fully qualified users.
Each fully qualified user
is described using the following format:
\begin{verbatim}
    username@domain/hostname
\end{verbatim}
The information to the left of the slash character describes
a user within a domain.
The information to the right of the slash character describes
one or more machines from which the user would be issuing a command. 
This host name may take the form of either a fully qualified host name
of the form
\begin{verbatim}
bird.cs.wisc.edu
\end{verbatim}
or an IP address
of the form
\begin{verbatim}
128.105.128.0
\end{verbatim}

An example is
\begin{verbatim}
zmiller@cs.wisc.edu/bird.cs.wisc.edu
\end{verbatim}

Within the format, wild card characters (the asterisk, *) are allowed.
The use of wild cards is limited to one wild card on either side
of the slash character.
A wild card character used in the host name is further limited
to come at the beginning of a fully qualified host name
or at the end of an IP address.
For example,
\begin{verbatim}
*@cs.wisc.edu/bird.cs.wisc.edu
\end{verbatim}
refers to any user that comes from \verb@cs.wisc.edu@,
where the command is originating from the machine
\verb@bird.cs.wisc.edu@.
Another valid example,
\begin{verbatim}
zmiller@cs.wisc.edu/*.cs.wisc.edu
\end{verbatim}
refers to commands coming from any machine within the 
\verb@cs.wisc.edu@ domain, and issued by \verb@zmiller@.
A third valid example,
\begin{verbatim}
*@cs.wisc.edu/*
\end{verbatim}
refers to commands coming from any user within the 
\verb@cs.wisc.edu@ domain
where the command is issued from any machine.
A fourth valid example,
\begin{verbatim}
*@cs.wisc.edu/128.105.*
\end{verbatim}
refers to commands coming from any user within the 
\verb@cs.wisc.edu@ domain
where the command is issued from machines within the network that match
the first two octets of the IP address.

If the set of machines is specified by an IP address,
then further specification using a net mask
identifies a physical set (subnet) of machines.
This physical set of machines is specified using the form
\begin{verbatim}
network/netmask
\end{verbatim}
The \verb@network@ is an IP address.
The net mask takes one of two forms.
It may be a decimal number which refers to the number of leading
bits of the IP address that are used in describing a subnet.
Or, the net mask may take the form of
\begin{verbatim}
a.b.c.d
\end{verbatim}
where \verb@a@,
\verb@b@,
\verb@c@, and
\verb@d@
are decimal numbers that each specify an 8-bit mask.
An example net mask is
\begin{verbatim}
255.255.192.0
\end{verbatim}
which specifies the bit mask
\begin{verbatim}
11111111.11111111.11000000.00000000
\end{verbatim}

A single complete example of a configuration variable that uses
a net mask is
\footnotesize
\begin{verbatim}
ALLOW_WRITE = joesmith@cs.wisc.edu/128.105.128.0/17
\end{verbatim}
\normalsize
User \verb@joesmith@ within the
\verb@cs.wisc.edu@ domain is given write authorization
when originating from machines that match their leftmost
17 bits of the IP address.

This flexible set of configuration macros could used to define
conflicting authorization.
Therefore, the following protocol defines the precedence of the
configuration macros.
\begin{description}
\item{1. }\MacroNI{DENY\_*} macros take precedence over \Macro{ALLOW\_* macros}
where there is a conflict.
This implies that if a specific user is both denied and granted authorization,
the conflict is resolved by denying access.
\item{2. }If macros are omitted, the default behavior is to grant
authorization for every user.
\end{description}

In addition, there are some hard-coded authorization rules that
 cannot be modified by configuration.
\begin{enumerate}

\index{unauthenticated}
\item Connections with a name matching \verb|*@unmapped| are not
 allowed to do any job management commands (e.g. submitting, removing,
 or modifying jobs).  This prevents these operations from being done
 by unauthenticated users and users who are authenticated but lacking
 a name in the map file.

\item To simplify flocking, the \Condor{schedd} automatically grants
 the \Condor{startd} \DCPerm{READ} access for the duration of a claim
 so that claim-related communications are possible.  The
 \Condor{shadow} grants the \Condor{starter} \DCPerm{DAEMON} access so
 that file transfers can be done.  The identity that is granted access
 in both these cases is the authenticated name (if available) and IP
 address of the \Condor{startd} when the \Condor{schedd} initially
 connects to it to request the claim.  It is important that only trusted
 \Condor{startd}s are allowed to publish themselves to the collector
 or that the \Condor{schedd}'s \MacroNI{ALLOW\_CLIENT} setting prevent
 it from allowing connections to \Condor{startd}s that it does not trust
 to run jobs.

\item When \Macro{SEC\_ENABLE\_MATCH\_PASSWORD\_AUTHENTICATION} is
 true, \verb|execute-side@matchsession| is automatically granted
 \DCPerm{READ} access to the \Condor{schedd} and \DCPerm{DAEMON}
 access to the \Condor{shadow}.

\end{enumerate}

%%%%%%%%%%%%%%%%%%%%%%%%%%%%%%%%%%%%%%%%%%%%%%%%%%%%%%%%%%%%%%%%%%%%%%
\subsubsection{\label{sec:Security-sample2} Example of Authorization Security Configuration}
%%%%%%%%%%%%%%%%%%%%%%%%%%%%%%%%%%%%%%%%%%%%%%%%%%%%%%%%%%%%%%%%%%%%%%

An example of the configuration variables for the user-side
authorization is derived from the necessary access levels
as described in
Section~\ref{sec:Security-access-levels}.

\footnotesize
\begin{verbatim}
ALLOW_READ            = *@cs.wisc.edu/*
ALLOW_WRITE           = *@cs.wisc.edu/*.cs.wisc.edu
ALLOW_ADMINISTRATOR   = condor-admin@cs.wisc.edu/*.cs.wisc.edu
ALLOW_CONFIG          = condor-admin@cs.wisc.edu/*.cs.wisc.edu
ALLOW_NEGOTIATOR      = condor@cs.wisc.edu/condor.cs.wisc.edu, \
                        condor@cs.wisc.edu/condor2.cs.wisc.edu
ALLOW_DAEMON          = condor@cs.wisc.edu/*.cs.wisc.edu

# Clear out any old-style HOSTALLOW settings:
HOSTALLOW_READ =
HOSTALLOW_WRITE =
HOSTALLOW_DAEMON =
HOSTALLOW_NEGOTIATOR =
HOSTALLOW_ADMINISTRATOR =
HOSTALLOW_OWNER =
\end{verbatim}
\normalsize


This example configuration authorizes
any authenticated user in the 
\verb@cs.wisc.edu@ domain to 
carry out a request that requires the 
\DCPerm{READ} access level
from any machine.
Any user in the 
\verb@cs.wisc.edu@ domain may 
carry out a request that requires the 
\DCPerm{WRITE} access level
from any machine in the
\verb@cs.wisc.edu@ domain.
Only the user called \verb@condor-admin@ may 
carry out a request that requires the 
\DCPerm{ADMINISTRATOR} access level
from any machine in the
\verb@cs.wisc.edu@ domain.
The administrator, logged into any machine within
the \verb@cs.wisc.edu@ domain is authorized at the
\DCPerm{CONFIG} access level.
Only the negotiator daemon, running as
\verb@condor@ on the two central managers
are authorized 
with the
\DCPerm{NEGOTIATOR} access level.
And, the last line of the example presumes that there is a
user called condor, and that the daemons have all been started
up as this user.
It authorizes only programs (which will be the daemons)
running as 
\verb@condor@ to
carry out requests that require the 
\DCPerm{DAEMON} access level,
where the commands originate from
any machine in the
\verb@cs.wisc.edu@ domain.

In the local configuration file for each host, the host's
owner should be authorized
as the owner of the machine.
An example of the entry in the local configuration file:
\footnotesize
\begin{verbatim}
ALLOW_OWNER           = username@cs.wisc.edu/hostname.cs.wisc.edu
\end{verbatim}
\normalsize
In this example the owner has a login of
\verb@username@, and the machine's name is represented by
\verb@hostname@.

%%%%%%%%%%%%%%%%%%%%%%%%%%%%%%%%%%%%%%%%%%%%%%%%%%%%%%%%%%%%%%%%%%%%%%
\subsubsection{\label{sec:Security-Debugging} Debugging Security Configuration}
%%%%%%%%%%%%%%%%%%%%%%%%%%%%%%%%%%%%%%%%%%%%%%%%%%%%%%%%%%%%%%%%%%%%%%

If the authorization policy denies a network request, an explanation
of why the request was denied is printed in the log file of the daemon
that denied the request.  The line in the log file contains the words
\verb|PERMISSION DENIED|.

To get HTCondor to generate a similar explanation of why requests are
accepted, add \Macro{D\_SECURITY} to the daemon's debug options (and
restart or reconfig the daemon).  The line in the log file for these
cases will contain the words \verb|PERMISSION GRANTED|.  If you do not
want to see a full explanation but just want to see when requests
are made, add \Macro{D\_COMMAND} to the daemon's debug options.

If the authorization policy makes use of host or domain names, then
be aware that HTCondor depends on DNS to map IP addresses to names.  The
security and accuracy of your DNS service is therefore a requirement.
Typos in DNS mappings are an occasional source of unexpected behavior.
If the authorization policy is not behaving as expected, carefully compare
the names in the policy with the host names HTCondor mentions in the
explanations of why requests are granted or denied.

%%%%%%%%%%%%%%%%%%%%%%%%%%%%%%%%%%%%%%%%%%%%%%%%%%%%%%%%%%%%%%%%%%%%%%
\subsection{\label{sec:Security-Sessions}Security Sessions}
%%%%%%%%%%%%%%%%%%%%%%%%%%%%%%%%%%%%%%%%%%%%%%%%%%%%%%%%%%%%%%%%%%%%%%
\index{security!sessions}
\index{sessions}

To set up and configure secure communications in HTCondor,
authentication, encryption, and integrity checks can be used.  
However, these come at a cost: performing strong authentication can
take a significant amount of time, and  generating the cryptographic
keys for encryption and integrity checks can take a significant amount
of processing power. 

The HTCondor system makes many network connections between different
daemons.  
If each one of these was to be authenticated,
and new keys were generated for each connection,
HTCondor would not be able to scale well.  
Therefore, HTCondor uses the concept of \Term{sessions} to cache
relevant security information for future use and greatly speed up the
establishment of secure communications between the various HTCondor
daemons.

A new session is established the first time a connection is made from one daemon to another.
Each session has a fixed lifetime after which it will expire and
a new session will need to be created again.
But while a valid session exists, it can be re-used as many times as
needed, thereby preventing the need to continuously re-establish secure connections.
Each entity of a connection will have access to a \Term{session key} that proves the
identity of the other entity on the opposing side of the connection.
This session key is exchanged securely using
a strong authentication method, such as Kerberos or GSI.
Other authentication methods, such as \MacroNI{NTSSPI},
\MacroNI{FS\_REMOTE},  \MacroNI{CLAIMTOBE}, and
\MacroNI{ANONYMOUS}, do not support secure key exchange.
An entity
listening on the wire may be able to impersonate the client or server
in a session that does not use a strong authentication method.

Establishing a secure session requires that either the encryption or the integrity options be enabled.
If the encryption capability is enabled, then the session will be restarted using the session key
as the encryption key.
If integrity capability is enabled, then the check sum includes the session key even
though it is not transmitted.
Without either of these two methods enabled,
it is possible for an
attacker to use an open session to make a connection to a daemon and
use that connection for nefarious purposes.
It is strongly recommended that if \emph{you have authentication turned
on, you should also turn on integrity and/or encryption}.

The configuration parameter \MacroNI{SEC\_DEFAULT\_NEGOTIATION} will allow
a user to set the default level of secure sessions in HTCondor.
Like other security settings, the possible values for this parameter can be
\verb@REQUIRED@, \verb@PREFERRED@, \verb@OPTIONAL@,
or \verb@NEVER@.
If you disable sessions and you have authentication turned
on, then most authentication (other than commands like
\Condor{submit}) will fail because HTCondor requires sessions when you
have security turned on. 
On the other hand, if you are not using strong security in HTCondor, but
you are relying on the default host-based security, turning off
sessions may be useful in certain situations. These might include debugging problems
with the security session management or slightly decreasing the memory
consumption of the daemons, which keep track of the sessions in use. 

Session lifetimes for specific daemons are already properly configured in the default installation
of HTCondor.
HTCondor tools such as \Condor{q} and \Condor{status} create a
session that expires after one minute. 
Theoretically they should not create a session at all,
because the
session cannot be reused between program invocations, but this is
difficult to do in the general case.
This allows a very small window of time for any possible attack,
and it helps
keep the memory footprint of running daemons down,
because they are not keeping track of all of the sessions.
The session durations may be manually tuned
by using macros in the configuration file,
but this is not recommended.


%%%%%%%%%%%%%%%%%%%%%%%%%%%%%%%%%%%%%%%%%%%%%%%%%%%%%%%%%%%%%%%%%%%%%%
\subsection{\label{sec:Host-Security}Host-Based Security in
HTCondor} 
%%%%%%%%%%%%%%%%%%%%%%%%%%%%%%%%%%%%%%%%%%%%%%%%%%%%%%%%%%%%%%%%%%%%%%
\index{security!host-based}

This section describes the mechanisms for setting up HTCondor's
host-based security.
This is now an outdated form of implementing security
levels for machine access.
It remains available and documented for purposes of backward compatibility.
If used at the same time as the user-based authorization,
the two specifications are merged together.

The host-based security paradigm allows control over which machines can
join an HTCondor pool, which machines can find out information about
your pool, and which machines within a pool can perform
administrative commands.  By default, HTCondor is configured to allow
anyone to view or join a pool. It is recommended that this parameter is changed
to only allow access from machines that you trust.

This section discusses how the host-based security works inside HTCondor.
It lists the different levels of access and what
parts of HTCondor use which levels.
There is a description of how to configure
a pool to grant or deny certain levels of access to various
machines.
Configuration examples and the settings of configuration variables
using the \Condor{config\_val} command complete this section.

Inside the HTCondor daemons or tools that use DaemonCore (see
section~\ref{sec:DaemonCore} for details), most
tasks are accomplished by sending commands to another HTCondor daemon.
These commands are represented by an integer value to specify which command
is being requested, followed
by any optional information that the protocol requires at that point
(such as a ClassAd, capability string, etc).
When the daemons start up,
they will register which commands they are willing to accept, what to
do with arriving commands, and the access level required for
each command.
When a command request is received by a daemon, HTCondor identifies the  access level
required and checks the IP address of the sender to verify that
it satisfies the allow/deny settings
from the configuration file.
If permission is granted, the command request is honored;
otherwise, the request will be aborted.
%% What does it mean for a command to be aborted?  Is it just
%% thrown away (ignored), or is a reply sent indicating failure?

Settings for the access levels in the global
configuration file will affect all the machines in the pool.
Settings in a local configuration file will only affect the specific machine.
The settings for a given machine determine what other hosts can send
commands to that machine.
If a machine foo is to be given
administrator access on machine bar, place foo in
bar's configuration file access list (not the other way around).


The following are the various access levels that commands within
HTCondor can be registered with:

\begin{description}

\item[\DCPerm{READ}] \label{dcperm:read} Machines with \DCPerm{READ}
   access can read information from the HTCondor daemons.  For example, they can
   view the status of the pool, see the job queue(s), and view user
   permissions.  \DCPerm{READ} access does not allow a machine to
   alter any information, and does not allow
   job submission. A machine listed
   with \DCPerm{READ} permission will be unable join an HTCondor pool; the machine can
   only view information about the pool.

\item[\DCPerm{WRITE}] \label{dcperm:write} Machines with
   \DCPerm{WRITE} access can write information to the HTCondor daemons.
   Most important for granting a machine with this access is that the machine
   will be able to join a pool since they are allowed to send ClassAd
   updates to the central manager.
   The machine can talk to the other machines
   in a pool in order to submit or run jobs.
   In addition, any machine with
   \DCPerm{WRITE} access can request the \Condor{startd} daemon to perform
   periodic checkpoints on an executing job. After the
   checkpoint is completed, the job will continue to execute and the
   machine will still be claimed by the original \Condor{schedd} daemon.
   This allows users on the machines where they submitted their jobs
   to use the \Condor{checkpoint} command to get their jobs to
   periodically checkpoint, even if the users do not have an account on the
   machine where the jobs execute.

   \textbf{IMPORTANT:} For a machine to join an HTCondor pool, the machine must
   have both \DCPerm{WRITE} permission \textbf{AND} \DCPerm{READ} permission.
   \DCPerm{WRITE} permission is not enough.

\item[\DCPerm{ADMINISTRATOR}] \label{dcperm:administrator} Machines
   with \DCPerm{ADMINISTRATOR} access are granted additional HTCondor
   administrator rights to the pool.  This includes the ability to
   change user priorities with the command \Condor{userprio},
   and the ability to turn HTCondor on and off
   (with the command \Code{\Condor{off} \Sinful{machine}}).
   It is recommended that few machines be granted administrator access in a pool;
   typically these are the machines that are used by HTCondor and system
   administrators as their primary workstations,
   or the machines running as the pool's central manager.

   \textbf{IMPORTANT:} Giving \DCPerm{ADMINISTRATOR} privileges to a machine
   grants administrator access for the pool to
   \textbf{ANY USER} on that machine. This includes any
   users who can run HTCondor jobs on that machine.
   It is recommended that \DCPerm{ADMINISTRATOR} access is granted with due diligence.

\item[\DCPerm{OWNER}] \label{dcperm:owner} This level of access is
   required for commands that the owner of a machine (any local user)
   should be able to use, in addition to the HTCondor administrators.
   For example, the \Condor{vacate} command causes the
   \Condor{startd} daemon to vacate any running HTCondor job.
   It requires \DCPerm{OWNER} permission,
   so that any user logged into a local machine
   can issue a \Condor{vacate} command.

\item[\DCPerm{NEGOTIATOR}] \label{dcperm:negotiator} This
   access level is used specifically to verify that commands are
   sent by the \Condor{negotiator} daemon.
   The \Condor{negotiator} daemon runs on the central manager of
   the pool.
   Commands requiring this access
   level are the ones that tell the \Condor{schedd} daemon to begin
   negotiating, and those that tell an available \Condor{startd} daemon
   that it has been matched to a \Condor{schedd} with jobs to run.

\item[\DCPerm{CONFIG}] \label{dcperm:config} This access level is
   required to modify a daemon's configuration using
   the \Condor{config\_val} command.
   By default, machines with this level of access are able
   to change any configuration parameter, except those specified in
   the \File{condor\_config.root} configuration file.
   Therefore, one should exercise extreme caution before
   granting this level of host-wide access.
   Because of the implications caused by \DCPerm{CONFIG} privileges,
   it is disabled by default for all hosts.

\item[\DCPerm{DAEMON}] \label{dcperm:daemon} This access level
   is used for commands that are internal to the operation of
   HTCondor.  An example of this internal operation is when the
   \Condor{startd} daemon sends
   its ClassAd updates to the \Condor{collector} daemon (which may be
   more specifically controlled by the \DCPerm{ADVERTISE\_STARTD}
   access level).
   Authorization at this access level should only be given to
   hosts that actually run HTCondor in your pool.
   The \DCPerm{DAEMON} level of access implies both
   \DCPerm{READ} and \DCPerm{WRITE} access.  Any setting for this access
   level that is not defined will default to the corresponding setting
   in the \DCPerm{WRITE} access level.

\item[\DCPerm{ADVERTISE\_MASTER}] \label{dcperm:advertise-master} This
   access level is used specifically for commands used to advertise a
   \Condor{master} daemon to the collector.  Any setting for this access
   level that is not defined will default to the corresponding setting
   in the \DCPerm{DAEMON} access level.

\item[\DCPerm{ADVERTISE\_STARTD}] \label{dcperm:advertise-startd} This
   access level is used specifically for commands used to advertise a
   \Condor{startd} daemon to the collector.  Any setting for this access
   level that is not defined will default to the corresponding setting
   in the \DCPerm{DAEMON} access level.

\item[\DCPerm{ADVERTISE\_SCHEDD}] \label{dcperm:advertise-schedd} This
   access level is used specifically for commands used to advertise a
   \Condor{schedd} daemon to the collector.  Any setting for this access
   level that is not defined will default to the corresponding setting
   in the \DCPerm{DAEMON} access level.

\item[\DCPerm{CLIENT}] \label{dcperm:client} This access level is
   different from all the others.  Whereas all of the other access levels
   refer to the security policy for accepting connections \emph{from} others,
   the \DCPerm{CLIENT} access level applies when an HTCondor daemon or tool is
   connecting \emph{to} some other HTCondor daemon.  In other words, it specifies
   the policy of the client that is initiating the operation, rather than
   the server that is being contacted.

\end{description}

\DCPerm{ADMINISTRATOR} and \DCPerm{NEGOTIATOR} access default to 
the central manager machine.
\DCPerm{OWNER} access defaults to the local machine, as well as
any machines
given with \DCPerm{ADMINISTRATOR} access.
\DCPerm{CONFIG} access is not granted to any machine
as its default.
These defaults are sufficient for most pools, and should not be changed without
a compelling reason.
If machines other than the default are to have to have \DCPerm{OWNER}
access, they probably should also have \DCPerm{ADMINISTRATOR} access.
By granting machines \DCPerm{ADMINISTRATOR} access, they
will automatically have \DCPerm{OWNER} access, given how
\DCPerm{OWNER} access is set within the configuration.

%%%%%%%%%%%%%%%%%%%%%%%%%%%%%%%%%%%%%%%%%%%%%%%%%%%%%%%%%%%%%%%%%%%%%%
\subsection{\label{sec:Example-Sec-Config}Examples of Security Configuration}
%%%%%%%%%%%%%%%%%%%%%%%%%%%%%%%%%%%%%%%%%%%%%%%%%%%%%%%%%%%%%%%%%%%%%%
\index{security!configuration examples} 

Here is a sample security configuration:
\footnotesize
\begin{verbatim}
ALLOW_ADMINISTRATOR = $(CONDOR_HOST)
ALLOW_OWNER = $(FULL_HOSTNAME), $(ALLOW_ADMINISTRATOR)
ALLOW_READ = *
ALLOW_WRITE = *
ALLOW_NEGOTIATOR = $(COLLECTOR_HOST)
ALLOW_NEGOTIATOR_SCHEDD = $(COLLECTOR_HOST), $(FLOCK_NEGOTIATOR_HOSTS)
ALLOW_WRITE_COLLECTOR = $(ALLOW_WRITE), $(FLOCK_FROM)
ALLOW_WRITE_STARTD    = $(ALLOW_WRITE), $(FLOCK_FROM)
ALLOW_READ_COLLECTOR  = $(ALLOW_READ), $(FLOCK_FROM)
ALLOW_READ_STARTD     = $(ALLOW_READ), $(FLOCK_FROM)
ALLOW_CLIENT = *
\end{verbatim}
\normalsize

This example configuration presumes that the \Condor{collector}
and \Condor{negotiator} daemons are running on the same machine.

For each access level, an ALLOW or a DENY may be added.
\begin{itemize}

\item If there is an ALLOW, it means "only allow these machines".  No
    ALLOW means allow anyone.

\item If there is a DENY, it means "deny these machines".  No DENY
    means deny nobody.

\item If there is both an ALLOW and a DENY, it means allow the
    machines listed in ALLOW except for the machines listed in DENY.

\item Exclusively for the \DCPerm{CONFIG} access,
    no ALLOW means allow no one.
    Note that this is different than the other ALLOW configurations.
    It is different to enable more stringent security where
    older configurations are used, since
    older configuration files would not have a 
    \DCPerm{CONFIG} configuration entry.
\end{itemize}

Multiple machine entries
in the configuration files
may be separated by either a space or a comma.
The machines may be listed by

\begin{itemize}
\item Individual host names, for example: \Expr{condor.cs.wisc.edu}
\item Individual IP address, for example: \Expr{128.105.67.29}
\item IP subnets (use a trailing \Expr{*}),
 for example: \Expr{144.105.*, 128.105.67.*}
\item Host names with a wild card \Expr{*} character
 (only one \Expr{*} is allowed per name),
 for example: \Expr{*.cs.wisc.edu, sol*.cs.wisc.edu}
\end{itemize}

To resolve an entry that falls into both allow and deny:
individual
machines have a higher order of precedence than wild card entries, and
host names with a wild card have a higher order of precedence than IP
subnets.
Otherwise, DENY has a higher order of precedence than ALLOW.
This is how most people would intuitively expect it to work.  

In addition, the above access levels may be specified on a
per-daemon basis, instead of machine-wide for all daemons.
Do this with the subsystem string (described in
section~\ref{sec:HTCondor-Subsystem-Names} on Subsystem Names),
which is one of: \Expr{STARTD}, \Expr{SCHEDD}, \Expr{MASTER}, \Expr{NEGOTIATOR},
or \Expr{COLLECTOR}.
For example, to grant different read access for the \Condor{schedd}:
\footnotesize
\begin{verbatim}
ALLOW_READ_SCHEDD = <list of machines>
\end{verbatim}
\normalsize


Here are more examples of configuration settings.
Notice that \DCPerm{ADMINISTRATOR} access is
only granted through an \Expr{ALLOW} setting to explicitly grant access to
a small number of machines.  We recommend this.

\begin{itemize}

\item Let any machine join the pool.
Only the central manager has
administrative access.
\footnotesize
\begin{verbatim}
ALLOW_ADMINISTRATOR = $(CONDOR_HOST)
ALLOW_OWNER = $(FULL_HOSTNAME), $(ALLOW_ADMINISTRATOR)
\end{verbatim}
\normalsize

\item Only allow machines at NCSA to join or view the pool.
The central manager is the only machine with \DCPerm{ADMINISTRATOR} access.
\footnotesize
\begin{verbatim}
ALLOW_READ = *.ncsa.uiuc.edu
ALLOW_WRITE = *.ncsa.uiuc.edu
ALLOW_ADMINISTRATOR = $(CONDOR_HOST)
ALLOW_OWNER = $(FULL_HOSTNAME), $(ALLOW_ADMINISTRATOR)
\end{verbatim}
\normalsize

\item Only allow machines at NCSA and the U of I Math department join the
pool, \emph{except do not} allow lab machines to do so.
Also, do not
allow the 177.55 subnet (perhaps this is the dial-in subnet).
Allow anyone to view pool statistics.  The machine named
bigcheese administers the pool (not the central manager).
\footnotesize
\begin{verbatim}
ALLOW_WRITE = *.ncsa.uiuc.edu, *.math.uiuc.edu
DENY_WRITE = lab-*.edu, *.lab.uiuc.edu, 177.55.*
ALLOW_ADMINISTRATOR = bigcheese.ncsa.uiuc.edu
ALLOW_OWNER = $(FULL_HOSTNAME), $(ALLOW_ADMINISTRATOR)
\end{verbatim}
\normalsize

\item Only allow machines at NCSA and UW-Madison's CS department to
view the pool.  Only NCSA machines and the machine raven.cs.wisc.edu can join
the pool.
Note: the machine raven.cs.wisc.edu has the read access it needs through the
wild card setting in \MacroNI{ALLOW\_READ}).
This example also shows
how to use the continuation character, \verb@\@, to continue a long list of 
machines onto multiple lines, making it more readable.
This works for all configuration file entries, not just host access entries.
\footnotesize
\begin{verbatim}
ALLOW_READ = *.ncsa.uiuc.edu, *.cs.wisc.edu
ALLOW_WRITE = *.ncsa.uiuc.edu, raven.cs.wisc.edu
ALLOW_ADMINISTRATOR = $(CONDOR_HOST), bigcheese.ncsa.uiuc.edu, \
                          biggercheese.uiuc.edu
ALLOW_OWNER = $(FULL_HOSTNAME), $(ALLOW_ADMINISTRATOR)
\end{verbatim}
\normalsize

\item Allow anyone except the military to view the status of the
pool, but only let machines at NCSA view the job queues.
Only NCSA machines can join the pool.
The central manager, bigcheese, and
biggercheese can perform most administrative functions.
However, only biggercheese can update user priorities.
\footnotesize
\begin{verbatim}
DENY_READ = *.mil
ALLOW_READ_SCHEDD = *.ncsa.uiuc.edu 
ALLOW_WRITE = *.ncsa.uiuc.edu
ALLOW_ADMINISTRATOR = $(CONDOR_HOST), bigcheese.ncsa.uiuc.edu, \
                          biggercheese.uiuc.edu
ALLOW_ADMINISTRATOR_NEGOTIATOR = biggercheese.uiuc.edu
ALLOW_OWNER = $(FULL_HOSTNAME), $(ALLOW_ADMINISTRATOR)
\end{verbatim}
\normalsize

\end{itemize}

%%%%%%%%%%%%%%%%%%%%%%%%%%%%%%%%%%%%%%%%%%%%%%%%%%%%%%%%%%%%%%%%%%%%%%
\subsection{\label{sec:Change-Sec-Config}Changing the Security Configuration}
%%%%%%%%%%%%%%%%%%%%%%%%%%%%%%%%%%%%%%%%%%%%%%%%%%%%%%%%%%%%%%%%%%%%%%
\index{security!changing the configuration}

A new security feature introduced in
HTCondor version 6.3.2 enables more fine-grained control over the
configuration settings that can be modified remotely with the
\Condor{config\_val} command.
The manual page for \Condor{config\_val} on
page~\pageref{man-condor-config-val} details how to use 
\Condor{config\_val} to modify configuration settings remotely. 
Since certain configuration attributes can have a large impact on the 
functioning of the HTCondor system and the security of the machines in an
HTCondor pool, it is important to restrict the ability to change
attributes remotely.

For each security access level described,
the HTCondor
administrator can define which configuration settings a host at that
access level is allowed to change.
Optionally, the administrator can define separate lists of settable
attributes for each HTCondor daemon, or the administrator
can define one list that is used by all daemons.

For each command that requests a change in configuration setting,
HTCondor searches all the different possible security access
levels to see which, if any, the request satisfies.
(Some hosts can qualify for multiple access levels. For example, any
host with \DCPerm{ADMINISTRATOR} permission probably has
\DCPerm{WRITE} permission also).
Within the qualified access level,
HTCondor searches for the list of attributes that may be modified.
If the request is covered by the list,
the request will be granted.
If not covered, the request will be refused.

The default configuration shipped with HTCondor is exceedingly
restrictive.
HTCondor users or administrators cannot set
configuration values from remote hosts with \Condor{config\_val}.
Enabling this feature requires a change to the
settings in the configuration file.
Use this security feature carefully.
Grant access only for attributes which you need to be able to modify
in this manner, and grant access only at the most restrictive
security level possible.

The most secure use of this feature allows HTCondor users to set
attributes in the configuration file which are not used by HTCondor
directly.
These are custom attributes published by various HTCondor
daemons with the \MacroNI{<SUBSYS>\_ATTRS} setting described in
section~\ref{param:SubsysAttrs} on page~\pageref{param:SubsysAttrs}.
It is secure to grant access only to modify attributes that are used by HTCondor
to publish information.
Granting access to modify
settings used to control the behavior of HTCondor is
not secure.
The goal is to
ensure no
one can use the power to change configuration attributes to compromise 
the security of your HTCondor pool.

The control lists are defined by configuration settings that contain 
\Macro{SETTABLE\_ATTRS} in their name.
The name of the control lists have the following form: 

\footnotesize
\begin{verbatim}
<SUBSYS>.SETTABLE_ATTRS_<PERMISSION-LEVEL>
\end{verbatim}
\normalsize

The two parts of this name that can vary are
the \verb@<PERMISSION-LEVEL>@ and the \verb@<SUBSYS>@.
The \verb@<PERMISSION-LEVEL>@ can be any of the security access levels
described earlier in this section.
Examples include \DCPerm{WRITE}, \DCPerm{OWNER}, and \DCPerm{CONFIG}.

The \verb@<SUBSYS>@ is an optional portion of the name. 
It can be used to
define separate rules for which configuration attributes can be set
for each kind of HTCondor daemon (for example, 
\Expr{STARTD}, \Expr{SCHEDD}, and \Expr{MASTER}).
There are many configuration settings that can be defined differently
for each daemon that use this \verb@<SUBSYS>@ naming convention.
See section~\ref{sec:HTCondor-Subsystem-Names} on
page~\pageref{sec:HTCondor-Subsystem-Names} for a list.
If there is no daemon-specific value for a given daemon, HTCondor will
look for \Macro{SETTABLE\_ATTRS\_<PERMISSION-LEVEL>}.

Each control list is defined by a comma-separated list of attribute
names which should be allowed to be modified.
The lists can contain wild cards characters (\verb@*@). 

Some examples of valid definitions of control lists with explanations:

\begin{itemize}

\item \begin{verbatim}SETTABLE_ATTRS_CONFIG = *\end{verbatim}
Grant unlimited access to modify configuration attributes
to any request that came from a machine in the \DCPerm{CONFIG} access
level. 
This was the default behavior before HTCondor version 6.3.2.

\item \begin{verbatim}SETTABLE_ATTRS_ADMINISTRATOR = *_DEBUG, MAX_*_LOG\end{verbatim} 
Grant access to change any configuration setting that ended
with \verb@_DEBUG@ (for example, \MacroNI{STARTD\_DEBUG}) and any
attribute that matched \verb@MAX_*_LOG@ (for example,
\MacroNI{MAX\_SCHEDD\_LOG}) to any host with \DCPerm{ADMINISTRATOR}
access. 

\item \begin{verbatim}STARTD.SETTABLE_ATTRS_OWNER = HasDataSet\end{verbatim}
Allows any request to modify the \MacroNI{HasDataSet} 
attribute that came from a host with \DCPerm{OWNER} access.
By default, \DCPerm{OWNER} covers any request originating from the
local host, plus any machines listed in the \DCPerm{ADMINISTRATOR}
level.
Therefore, any HTCondor job would qualify for OWNER access to the
machine where it is running. 
So, this setting would allow any process running on a given host,
including an HTCondor job, to modify the \MacroNI{HasDataSet} variable for
that host. 
\MacroNI{HasDataSet} is not used by HTCondor, it is an invented attribute
included in the \Macro{STARTD\_ATTRS} setting in order for this
example to make sense.

\end{itemize}

%%%%%%%%%%%%%%%%%%%%%%%%%%%%%%%%%%%%%%%%%%%%%%%%%%%%%%%%%%%%%%%%%%%%%%
\subsection{\label{sec:security-networks}Using 
HTCondor w/ Firewalls, Private Networks, and NATs}
%%%%%%%%%%%%%%%%%%%%%%%%%%%%%%%%%%%%%%%%%%%%%%%%%%%%%%%%%%%%%%%%%%%%%%

This topic is now addressed in more detail in
section~\ref{sec:Networking}, which explains network communication in
HTCondor.

%%%%%%%%%%%%%%%%%%%%%%%%%%%%%%%%%%%%%%%%%%%%%%%%%%%%%%%%%%%%%%%%%%%%%%
\subsection{\label{sec:uids}User Accounts in HTCondor on Unix Platforms}
%%%%%%%%%%%%%%%%%%%%%%%%%%%%%%%%%%%%%%%%%%%%%%%%%%%%%%%%%%%%%%%%%%%%%%

\index{UIDs in HTCondor|(}

On a Unix system,
UIDs (User IDentification numbers) form part of an operating system's
tools for maintaining access control.
Each executing program has a UID,
a unique identifier of a user executing the program.
This is also called the real UID.
\index{UID!real}
A common situation has one user executing the program owned
by another user.
Many system commands work this way, with a user (corresponding
to a person) executing a program belonging to (owned by) \Login{root}.
Since the program may require privileges that \Login{root} has which
the user does not have, a special bit in the program's
protection specification (a setuid bit) allows the program
to run with the UID of the program's owner, instead of the
user that executes the program.
This UID of the program's owner is called an effective UID.
\index{UID!effective}

%GID (group identification)
HTCondor works most smoothly when its daemons run as \Login{root}.
The daemons then have the ability to switch their 
effective UIDs at will.
When the daemons run as \Login{root},
they normally leave their effective UID and GID (Group IDentification)
to be those of user and group \Login{condor}.
This allows access to the log files without
changing the ownership of the log files.
It also allows access to these files when
the user \Login{condor}'s home directory resides on an NFS server.
\Login{root} can not normally access NFS files.

If there is no \Login{condor} user and group on the system, an
administrator can specify which UID and GID the HTCondor daemons should
use when they do not need root privileges in two ways:
either with the \Env{CONDOR\_IDS} environment variable or the
\Macro{CONDOR\_IDS} configuration variable.
In either case, the value should be the UID integer, followed by a
period, followed by the GID integer.
For example, if an HTCondor administrator does not want to create a
\Login{condor} user, and instead wants their HTCondor daemons to run as
the \Login{daemon} user (a common non-root user for system daemons to
execute as), the \Login{daemon} user's UID was 2, and group
\Login{daemon} had a GID of 2, the corresponding setting in the HTCondor
configuration file would be \Expr{CONDOR\_IDS = 2.2}.

On a machine where a job is submitted,
the \Condor{schedd} daemon
changes its effective UID to \Login{root}
such that it has the capability to start up a \Condor{shadow} daemon
for the job.
Before a \Condor{shadow} daemon is created,
the \Condor{schedd} daemon
switches back to \Login{root},
so that it can start up the \Condor{shadow} daemon with the (real) UID
of the user who submitted the job.
Since the \Condor{shadow} runs as the owner of the job,
all remote system calls are performed under the owner's UID
and GID.
This ensures that as the job executes,
it can access only files that its owner could access if the job
were running locally, without HTCondor.

On the machine where the job executes, the 
job runs either as the submitting user or as user \Login{nobody},
to help ensure that the job cannot access local resources or do harm.  
If the \Macro{UID\_DOMAIN} matches,
and the user exists as the same UID in password files
on both the submitting machine and on the execute machine,
the job will run as the submitting user.
If the user does not exist in the execute machine's
password file and \Macro{SOFT\_UID\_DOMAIN} is True,
then the job will run under the submitting user's UID anyway (as
defined in the submitting machine's password file).
If \MacroNI{SOFT\_UID\_DOMAIN} is False,
and \MacroNI{UID\_DOMAIN} matches,
and the user is not in the execute machine's password file,
then the job execution attempt will be aborted.

%%%%%%%%%%%%%%%%%%%%%%%%%%%%%%%%%%%%%%%%%%%%%%%%%%%%%%%%%%%%%%%%%%%%%%
\subsubsection{\label{sec:Non-Root}Running HTCondor as Non-Root}
%%%%%%%%%%%%%%%%%%%%%%%%%%%%%%%%%%%%%%%%%%%%%%%%%%%%%%%%%%%%%%%%%%%%%%

While we strongly recommend starting up the HTCondor daemons as \Login{root},
we understand that it is not always possible to do so.
The main problems of not running HTCondor daemons as \Login{root}
appear when one HTCondor installation is shared by many users on a
single machine, or if machines are set up to only execute
HTCondor jobs.  With a submit-only installation for a
single user, there is no need for or benefit from running as
\Login{root}.

The effects of HTCondor of running both with and without root access
are classified for each daemon:

\begin{description}

\item[\Condor{startd}] An HTCondor machine set up to execute jobs where the
   \Condor{startd} is not started as \Login{root}
   relies on the good will of the HTCondor users to agree to the policy
   configured for the \Condor{startd} to enforce for starting, suspending,
   vacating, and killing HTCondor jobs. 
   When the \Condor{startd} is started
   as \Login{root}, however, these policies may be enforced regardless of
   malicious users.  By running as \Login{root}, the HTCondor daemons run with a
   different UID than the HTCondor job. 
   The user's job is started as either the UID of the user who submitted
   it, or as user \Login{nobody}, depending on the \Macro{UID\_DOMAIN}
   settings.
   Therefore, the HTCondor job cannot do anything to the HTCondor daemons.
   Without starting the daemons as \Login{root}, all
   processes started by HTCondor, including the user's job, run with
   the same UID. Only \Login{root} can switch UIDs.
   Therefore, a user's job could kill the \Condor{startd} and
   \Condor{starter}. By doing so, the user's job avoids
   getting suspended or vacated.
   This is nice for the job, as it obtains unlimited access to the
   machine, but it is awful for the machine owner or administrator.
   If there is trust of the users submitting jobs to HTCondor,
   this might not be a concern.  
   However, to ensure that the policy chosen is
   enforced by HTCondor, the \Condor{startd} should be
   started as \Login{root}.

   In addition, some system information cannot be obtained without
   \Login{root} access on some platforms.
   As a result, when running without \Login{root} access,
   the \Condor{startd} must
   call other programs such as \Prog{uptime},
   to get this information.
   This is much less efficient than getting the
   information directly from the kernel,
   as is done when running as \Login{root}.
   On Linux, this information is available without root access,
   so it is not a concern on those platforms.

   If all of HTCondor cannot be run as \Login{root}, at least consider
   installing the \Condor{startd} as setuid root.  That
   would solve both problems.  Barring that,
   install it as a setgid sys or kmem program,
   depending on
   whatever group has read access to \File{/dev/kmem} on the system.
   That would solve the system information problem.

\item[\Condor{schedd}] The biggest problem with running the \Condor{schedd}
    without \Login{root} access is that the \Condor{shadow} processes which it
    spawns are stuck with the same UID that the \Condor{schedd} has.
    This requires users to go out of their way
    to grant write access to user or group that the \Condor{schedd} is run as
    for any files or directories their jobs write or create.
    Similarly, read access must be granted to their input files.

    Consider installing \Condor{submit} as a setgid \Login{condor}
    program so that at least the \File{stdout}, \File{stderr} and
    user log files get created with the right permissions.  If
    \Condor{submit} is a setgid program, it will automatically set
    its umask to 002 and create group-writable files.  This
    way, the simple case of a job that only writes to \File{stdout}
    and \File{stderr} will work.  If users have programs that open
    their own files, they will need to know and set the proper permissions
    on the directories they submit from.

\item[\Condor{master}] The \Condor{master} spawns both the
    \Condor{startd} and the \Condor{schedd}. To have both running
    as \Login{root}, have the \Condor{master} run as \Login{root}.
    This happens
    automatically if the \Condor{master} is started from boot scripts.

\item[\Condor{negotiator} and \Condor{collector}]
    There is no need to have either of these daemons running as \Login{root}.

\item[\Condor{kbdd}] On platforms that need the \Condor{kbdd},
    the \Condor{kbdd} must run as \Login{root}.  If it is
    started as any other user, it will not work.  Consider
    installing this program as a setuid root binary if the \Condor{master} will
    not be run as \Login{root}.
    Without the \Condor{kbdd},
    the \Condor{startd} has no way to monitor USB mouse or keyboard activity,
    although it will notice keyboard activity on ttys such as
    xterms and remote logins.

\end{description}

If HTCondor is not run as root, then choose almost any user name.
A common choice is to set up and use the \Login{condor} user; this
simplifies the setup,
because HTCondor will look for its configuration
files in the \Login{condor} user's directory.
If \Login{condor} is not selected,
then the configuration must be placed properly such that
HTCondor can find its configuration files. 

If users will be submitting jobs as a user different than the user
HTCondor is running as (perhaps you are running as the \Login{condor} user and
users are submitting as themselves), then users have to be careful to
only have file permissions properly set up to be accessible by the
user HTCondor is using. In practice, this means creating world-writable
directories for output from HTCondor jobs.  This creates a potential
security risk, in that any user on the machine where the job is
submitted can alter the data, remove it, or do other undesirable
things.  It is only acceptable in an environment where users can trust
other users.

Normally, users without \Login{root} access who wish to use HTCondor on their
machines create a \File{condor} home directory somewhere within their
own accounts and start up the daemons (to run with the UID of the
user).  As in the case where the daemons run as user \Login{condor}, there is
no ability to switch UIDs or GIDs.  The daemons run as the UID and GID
of the user who started them.  On a machine where jobs are submitted,
the \Condor{shadow} daemons all run as this same user. 
But, if other
users are using HTCondor on the machine in this environment, the
\Condor{shadow} daemons for these other users' jobs execute with the
UID of the user who started the daemons.  This is a security risk,
since the HTCondor job of the other user has access to all the files and
directories of the user who started the daemons.  Some installations
have this level of trust, but others do not.  Where this level of
trust does not exist, it is best to set up a \Login{condor} account and group,
or to have each user start up their own Personal HTCondor submit
installation.

When a machine is an execution site for an HTCondor job, the HTCondor job
executes with the UID of the user who started the \Condor{startd}
daemon.  This is also potentially a security risk, which is why we do
not recommend starting up the execution site daemons as a regular
user.  Use either \Login{root} or a user
such as \Login{condor} that exists
only to run HTCondor jobs.

%%%%%%%%%%%%%%%%%%%%%%%%%%%%%%%%%%%%%%%%%%%%%%%%%%%%%%%%%%%%%%%%%%%%%%
\subsubsection{\label{sec:RunAsNobody}Running Jobs as the Nobody User}
%%%%%%%%%%%%%%%%%%%%%%%%%%%%%%%%%%%%%%%%%%%%%%%%%%%%%%%%%%%%%%%%%%%%%%
\index{user nobody!potential security risk with jobs}
\index{UID!potential risk running jobs as user nobody}
\index{security!running jobs as user nobody}

Under Unix, HTCondor runs jobs either as the user that submitted the
jobs, or as the user called \Login{nobody}.  HTCondor uses user
\Login{nobody} if the value of the \MacroNI{UID\_DOMAIN} configuration
variable of the submitting and executing machines are different or if
\Macro{STARTER\_ALLOW\_RUNAS\_OWNER} is false or if the job ClassAd
contains \Attr{RunAsOwner}=False.  Under Windows, HTCondor by default
runs jobs under a dynamically created local account that exists for
the duration of the job, but it can optionally run the job as the user
account that owns the job if \Macro{STARTER\_ALLOW\_RUNAS\_OWNER} is
True and the job contains \Attr{RunAsOwner}=True.

When HTCondor cleans up after executing a vanilla universe job, it does
the best that it can by deleting all of the processes started by the
job.  During the life of the job, it also does its best to track the
CPU usage of all processes created by the job.  There are a variety of
mechanisms used by HTCondor to detect all such processes, but, in
general, the only foolproof mechanism is for the job to run under a
dedicated execution account (as it does under Windows by default).
With all other mechanisms, it is possible to fool HTCondor, and leave
processes behind after HTCondor has cleaned up.  In the case of a shared
account, such as the Unix user \Login{nobody}, it is possible for the
job to leave a lurker process lying in wait for the next job run as
\Login{nobody}.  The lurker process may prey maliciously on the next
\Login{nobody} user job, wreaking havoc.

HTCondor could prevent this problem by simply killing all processes run by
the \Login{nobody} user, but this would annoy many system administrators.
The \Login{nobody} user is often used for non-HTCondor system processes.
It may also be used by other HTCondor jobs running on the same machine, if
it is a multi-processor machine.

HTCondor provides a two-part solution to this difficulty.
First, create user accounts specifically for HTCondor to use instead
of user \Login{nobody}.
These can be low-privilege accounts,
as the \Login{nobody} user is.
Create one of these accounts for each job execution slot per computer,
so that distinct users can be used for concurrent processes.
This prevents malicious behavior between
processes running on distinct slots.
Section ~\ref{sec:Configuring-SMP} details slots.
For a sample machine with two compute slots,
create two users that are intended only to be used by HTCondor.
As an example, call them \Login{cndrusr1} and \Login{cndrusr2}.
Tell HTCondor about these users
with the \Macro{SLOT<N>\_USER} configuration variables,
where \MacroNI{<N>} is replaced with the slot number.
In this example:

\begin{verbatim}
   SLOT1_USER = cndrusr1
   SLOT2_USER = cndrusr2
\end{verbatim}

Then tell HTCondor that these accounts are intended only to be used by HTCondor,
so HTCondor can kill all the processes belonging to these users upon
job completion.
The configuration variable \Macro{DEDICATED\_EXECUTE\_ACCOUNT\_REGEXP}
is introduced and set to a regular expression that matches the account
names we have just created.

\begin{verbatim}
   DEDICATED_EXECUTE_ACCOUNT_REGEXP = cndrusr[0-9]+
\end{verbatim}

Finally, tell HTCondor not to run jobs as the job owner:

\begin{verbatim}
   STARTER_ALLOW_RUNAS_OWNER = False
\end{verbatim}

Notes:
\begin{enumerate}

\item{ Currently, none of these configuration settings apply to
standard universe jobs.  Normally, standard universe jobs do not
create additional processes. }

\item{
On Windows, \MacroNI{SLOT<N>\_USER} will only work if the credential
of the specified user is stored on the execute machine
using \Condor{store\_cred}.
See the \Condor{store\_cred}
manual page (in section~\ref{man-condor-store-cred})
for details of this command.  However, the default behavior in Windows
is to run jobs under a dynamically created dedicated execution account,
so just using the default behavior is sufficient to avoid problems
with lurker processes.
}
\item {
You can tell if the starter is in fact treating the account as a
dedicated account, because it will print a line such as the following
in its log file:

\begin{verbatim}
Tracking process family by login "cndrusr1"
\end{verbatim}
}
\end{enumerate}


%%%%%%%%%%%%%%%%%%%%%%%%%%%%%%%%%%%%%%%%%%%%%%%%%%%%%%%%%%%%%%%%%%%%%%
\subsubsection{\label{sec:DirOfJob}Working Directories for Jobs}
%%%%%%%%%%%%%%%%%%%%%%%%%%%%%%%%%%%%%%%%%%%%%%%%%%%%%%%%%%%%%%%%%%%%%%
\index{cwd!of jobs}
\index{current working directory}

Every executing process has a notion of its current working directory.
This is the directory that acts as the base for all file system
access. 
There are two current working directories for any HTCondor job:
one where the job is submitted and a
second where the job executes.
When a user submits a job,
the submit-side current working directory is
the same as for the user when the \Condor{submit} command
is issued.
The \SubmitCmd{initialdir} submit command may change this,
thereby allowing different jobs to have different working
directories.
This is useful when submitting large numbers of jobs.
This submit-side current working directory remains unchanged for the
entire life of a job. 
The submit-side current working directory is also 
the working directory of the \Condor{shadow} daemon.
This is particularly relevant for standard universe jobs,
since file system
access for the job goes through the \Condor{shadow} daemon, and
therefore all accesses behave as if they were executing without
HTCondor.

There is also an execute-side current working directory.
For standard universe jobs,
it is set to the
\File{execute} subdirectory of HTCondor's home directory.
This directory is world-writable, since an HTCondor job usually runs as user
\Login{nobody}.
Normally, standard universe jobs would never access this directory,
since all I/O system calls are passed back to the
\Condor{shadow} daemon on the submit machine.
In the event, however,
that a job crashes and creates a core dump file, the execute-side
current working directory needs to be accessible by the job
so that it can write the core file.
The core file is moved back to the submit machine,
and the \Condor{shadow} daemon is informed.
The \Condor{shadow} daemon
sends e-mail to the job owner announcing the core file, and provides a
pointer to where
the core file resides in the submit-side current working directory.

\index{UIDs in HTCondor|)}


%%%%%%%%%%%%%%%%%%%%%%%%%%%%%%%%%%%%%%%%%%%%%%%%%%%%%%%%%%%%%%%%%%%%%%
\subsection{\label{sec:PrivSep}Privilege Separation}
%%%%%%%%%%%%%%%%%%%%%%%%%%%%%%%%%%%%%%%%%%%%%%%%%%%%%%%%%%%%%%%%%%%%%%
\index{privilege separation}
\index{PrivSep (privilege separation)}

Section~\ref{sec:uids} discusses why, under most circumstances, it is
beneficial to run the Condor daemons as \Login{root}. In situations
where multiple users are involved or where Condor is responsible for
enforcing a machine owner's policy, running as \Login{root} is the 
\emph{only} way for Condor to do its job correctly and securely.

Unfortunately, this requirement of running Condor as \Login{root} is
at odds with a well-established goal of security-conscious
administrators: keeping the amount of software that runs with
superuser privileges to a minimum. Condor's nature as a
large distributed system that routinely communicates with potentially
untrusted components over the network further aggravates this goal.

The privilege separation (PrivSep) effort in Condor aims to minimize
the amount of code that needs \Login{root}-level access, while still
giving Condor the tools it needs to work properly. Note that PrivSep
is currently only available for execute side functionality, and is not
implemented on Windows.

In the PrivSep model, all logic in Condor that requires superuser
privilege is contained in a small component called 
the PrivSep Kernel.
The Condor daemons execute as an unprivileged account.
They explicitly request action from the PrivSep Kernel whenever
\Login{root}-level operations are needed.

The PrivSep model then prevents the following attack scenario.
In the attack scenario, an attacker has found an exploit in the
\Condor{startd} that allows for execution of arbitrary code on that
daemon's behalf. 
This gives the attacker \Login{root} access and therefore
control over any machine
on which the \Condor{startd} is running as \Login{root}
and the exploit can be exercised.
Under the PrivSep model,
the \Condor{startd} no longer runs as \Login{root}.
This prevents the attacker from taking arbitrary action as \Login{root}. 
Further, limits on requested actions from the PrivSep Kernel
contain and restrict the attacker's sphere of influence.

The following section describes the configuration necessary
to enable PrivSep for an execute-side Condor installation.
After this is a detailed description of
the services that the PrivSep Kernel provides to Condor, and how it
limits the allowed \Login{root}-level actions.

%%%%%%%%%%%%%%%%%%%%%%%%%%%%%%%%%%%%%%%%%%%%%%%%%%%%%%%%%%%%%%%%%%%%%%
\subsubsection{PrivSep Configuration}
%%%%%%%%%%%%%%%%%%%%%%%%%%%%%%%%%%%%%%%%%%%%%%%%%%%%%%%%%%%%%%%%%%%%%%

The PrivSep Kernel is implemented as two programs:
the \Condor{root\_switchboard} and the \Condor{procd}.
Both are contained in
the \File{sbin} directory of the Condor distribution. When Condor is
running in PrivSep mode, these are to be the only two Condor daemons that run
with \Login{root} privilege.

Each of these binaries must be accessible on the file system via a
\Term{trusted path}. A trusted path ensures that no user (other than
\Login{root}) can alter the binary or path to the binary referred to.
To ensure that the
paths to these binaries are trusted, use only
\Login{root}-owned directories, and set the permissions on
these directories to deny write access to all but \Login{root}. 
The binaries themselves must also be owned by \Login{root} 
and not writable by any other.
The \Condor{root\_switchboard} program additionally
is installed with the setuid bit set. The following command
properly sets the permissions on the \Condor{root\_switchboard}
binary:
\begin{verbatim}
chmod 4755 /opt/condor/release/sbin/condor_root_switchboard
\end{verbatim}

The PrivSep Kernel has its own configuration file.
This file  must be \File{/etc/condor/privsep\_config}.
The format of this file is different than 
a Condor configuration file.
It consists
of lines with ``\Code{key = value}'' pairs. 
Lines with only whitespace
or lines with ``\Code{\#}'' as the first non-whitespace character are ignored.

In the PrivSep Kernel configuration file,
some configuration settings are interpreted as single values, 
while others are interpreted as lists. 
To populate a list with multiple values,
use multiple lines with the same key.
For example, the following configures the \Code{valid-dirs}
setting as a list with two entries:
\begin{verbatim}
valid-dirs = /opt/condor/execute_1
valid-dirs = /opt/condor/execute_2
\end{verbatim}
It is an error to have multiple lines with the same key for a setting
that is not interpreted as a list.

Some PrivSep Kernel configuration file settings require a list of UIDs or GIDs,
and these allow for a more specialized syntax.
User and group IDs can be specified either numerically or textually.
Multiple list entries may be given on a single
line using the \Code{:} (colon) character as a delimiter.
In addition, list entries
may specify a range of IDs using a \Code{-} (dash)
character to separate the minimum and maximum IDs included.
The \Code{*} (asterisk) character on the right-hand side of such a
range indicates that the range extends to the maximum possible ID. 
The following example builds a complex list of IDs:
\begin{verbatim}
valid-target-uids = nobody : nfsuser1 : nfsuser2
valid-target-uids = condor_run_1 - condor_run_8
valid-target-uids = 800 - *
\end{verbatim}

If \Code{condor\_run\_1} maps to UID 701, and
\Code{condor\_run\_8} maps to UID 708, 
then this range specifies the 8 UIDs of 701 through 708 (inclusive).

The following settings are required to configure the PrivSep Kernel:
\begin{itemize}

\item \Code{valid-caller-uids} and \Code{valid-caller-gids}. These lists
specify users and groups that will be allowed to request action from
the PrivSep Kernel. The list typically will contain the
UID and primary GID that the Condor daemons will run as.

\item \Code{valid-target-uids} and \Code{valid-target-gids}. These lists
specify the users and groups that Condor will be allowed to act on
behalf of. The list will need to include IDs of all users and groups
that Condor jobs may use on the given execute machine.

\item \Code{valid-dirs}. This list specifies directories that
Condor will be allowed to manage for the use of temporary job
files. Normally, this will only need to include the value of Condor's
\MacroU{EXECUTE} directory. 
Note that the PrivSep Kernel does not
have access to Condor's configuration variables,
and therefore may not refer to them in this file.

\item \Code{procd-executable}. A (trusted) full path to the
\Condor{procd} executable.
Note that the PrivSep Kernel does not
have access to Condor's configuration variables,
and therefore may not refer to them in this file.

\end{itemize}

Here is an example of a full \File{privsep\_config} file.
This file gives the \Login{condor} account access to the PrivSep
Kernel. Condor's use of this execute machine will be restricted to a
set of eight dedicated accounts, along with the \Login{users}
group. Condor's \MacroUNI{EXECUTE} directory and the \Condor{procd}
executable are also specified, as required.
\begin{verbatim}
valid-caller-uids = condor
valid-caller-gids = condor
valid-target-uids = condor_run_1 - condor_run_8
valid-target-gids = users : condor_run_1 - condor_run_8
valid-dirs = /opt/condor/local/execute
procd-executable = /opt/condor/release/sbin/condor_procd
\end{verbatim}

Once the PrivSep Kernel is properly installed and configured, Condor's
configuration must be updated to specify that PrivSep should be
used. The Condor configuration variable \Macro{PRIVSEP\_ENABLED} 
is a boolean flag serving this purpose.
In addition, Condor must be told where the
\Condor{root\_switchboard} binary is located using the
\Macro{PRIVSEP\_SWITCHBOARD} setting. The following example
illustrates:
\begin{verbatim}
PRIVSEP_ENABLED = True
PRIVSEP_SWITCHBOARD = $(SBIN)/condor_root_switchboard
\end{verbatim}

Finally, note that while the \Condor{procd} is in general an optional
component of Condor, it is required when PrivSep is in use. If
\MacroNI{PRIVSEP\_ENABLED} is \Expr{True}, the \Condor{procd} will be used
regardless of the \Macro{USE\_PROCD} setting.
Details on these Condor configuration variables are in
section~\ref{sec:Config-PrivSep} for PrivSep variables
and 
section~\ref{sec:Procd-Config-File-Entries} for \Condor{procd} variables.

%%%%%%%%%%%%%%%%%%%%%%%%%%%%%%%%%%%%%%%%%%%%%%%%%%%%%%%%%%%%%%%%%%%%%%
\subsubsection{PrivSep Kernel Interface}
%%%%%%%%%%%%%%%%%%%%%%%%%%%%%%%%%%%%%%%%%%%%%%%%%%%%%%%%%%%%%%%%%%%%%%

This section describes the \Login{root}-enabled operations that the
PrivSep Kernel makes available to Condor. The PrivSep Kernel's
interface is designed to provide only operations needed by Condor in
order to function properly. Each operation is further restricted based
on the PrivSep Kernel's configuration settings.

The following list describes each action that can be performed via the
PrivSep Kernel, along with the limitations enforced on how it may be
used. The terms \emph{valid target users}, \emph{valid target groups},
and \emph{valid directories} refer respectively to the settings for
\Code{valid-target-uids}, \Code{valid-target-gids}, and
\Code{valid-dirs} from the PrivSep Kernel's configuration.

\begin{itemize}

\item \emph{Make a directory as a user.} This operation creates
an empty directory, owned by a user. The user must be a valid target
user, and the new directory's parent must be a valid directory.

\item \emph{Change ownership of a directory tree.} This operation
involves recursively changing ownership of all files and
subdirectories contained in a given directory. The directory's parent
must be a valid directory, and the new owner must either be a valid
target user or the user invoking the PrivSep Kernel.

\item \emph{Remove a directory tree.} This operation deletes a given directory,
including everything contained within. The directory's parent must be
a valid directory.

\item \emph{Execute a program as a user.} Condor can invoke the
PrivSep kernel to execute a program as a valid target user. The user's
primary group and any supplemental groups that it is a member of must
all be valid target groups. This operation may also include opening
files for standard input, output, and error before executing the
program.

\end{itemize}

After launching a program as a valid target user, the PrivSep Kernel
allows Condor limited control over its execution. The following
operations are supported on a program executed via the PrivSep Kernel:

\begin{itemize}

\item \emph{Get resource usage information.} This allows Condor to
gather usage statistics such as CPU time and memory image size. This
applies to the program's initial process and any of its descendants.

\item \emph{Signal the program.} Condor may ask that signals be sent
to the program's initial process as a notification mechanism.

\item \emph{Suspend and resume the program.} These operations send
\Code{SIGSTOP} or \Code{SIGCONT} signals to all processes that make up
the program.

\item \emph{Kill the process and all descendants.} Condor is allowed
to terminate the execution of the program or any processes
left behind when the program completes.

\end{itemize}

By sufficiently constraining the valid target
accounts and valid directories to which the PrivSep Kernel allows
access, the ability of a compromised Condor daemon to do damage can
be considerably reduced.



\index{security!in HTCondor|)}
