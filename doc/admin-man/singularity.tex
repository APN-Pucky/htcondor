%%%%%%%%%%%%%%%%%%%%%%%%%%%%%%%%%%%%%%%%%%%%%%%%%%%%%%%%%%%%%%%%%%%%%%
\section{Singularity Support}\label{sec:singularity-support}
%%%%%%%%%%%%%%%%%%%%%%%%%%%%%%%%%%%%%%%%%%%%%%%%%%%%%%%%%%%%%%%%%%%%%%

\index{installation!Singularity}
\index{Singularity}

Note:  This documentation is very basic and needs improvement!

Here's an example configuration file:

\begin{verbatim}
  # Only set if singularity is not in $PATH.
  #SINGULARITY = /opt/singularity/bin/singularity

  # Forces _all_ jobs to run inside singularity.
  SINGULARITY_JOB = true

  # Forces all jobs to use the CernVM-based image.
  SINGULARITY_IMAGE_EXPR = "/cvmfs/cernvm-prod.cern.ch/cvm3"

  # Maps $_CONDOR_SCRATCH_DIR on the host to /srv inside the image.
  SINGULARITY_TARGET_DIR = /srv

  # Writable scratch directories inside the image.  Auto-deleted after the job exits.
  MOUNT_UNDER_SCRATCH = /tmp, /var/tmp
\end{verbatim}

This provides the user with no opportunity to select a specific image.
Here are some changes to the above example to allow the user to specify an
image path:

\begin{verbatim}
  SINGULARITY_JOB = !isUndefined(TARGET.SingularityImage)
  SINGULARITY_IMAGE_EXPR = TARGET.SingularityImage
\end{verbatim}

Then, users could add the following to their submit file
(note the quoting):

\begin{verbatim}
  +SingularityImage = "/cvmfs/cernvm-prod.cern.ch/cvm3"
\end{verbatim}

Finally, let's pick an image based on the OS -- not the filename:

\begin{verbatim}
  SINGULARITY_JOB = (TARGET.DESIRED_OS isnt MY.OpSysAndVer) && ((TARGET.DESIRED_OS is "CentOS6") || (TARGET.DESIRED_OS is "CentOS7"))
  SINGULARITY_IMAGE_EXPR = (TARGET.DESIRED_OS is "CentOS6") ? "/cvmfs/cernvm-prod.cern.ch/cvm3" : "/cvmfs/cms.cern.ch/rootfs/x86_64/centos7/latest"
\end{verbatim}

Then, the user adds to their submit file:

\begin{verbatim}
  +DESIRED_OS="CentOS6"
\end{verbatim}

That would cause the job to run on the native host for CentOS6 hosts
and inside a CentOS6 Singularity container on CentOS7 hosts.
