%%%%%%%%%%%%%%%%%%%%%%%%%%%%%%%%%%%%%%%%%%%%%%%%%%%%%%%%%%%%%%%%%%%%%%
\subsection{Configuring The Startd for SMP Machines}
\label{sec:Configuring-SMP}
%%%%%%%%%%%%%%%%%%%%%%%%%%%%%%%%%%%%%%%%%%%%%%%%%%%%%%%%%%%%%%%%%%%%%%

This section describes how to configure the \Condor{startd} for SMP
(Symetric Multi-Processor) machines.
Beginning with Condor version 6.1, machines with more than one CPU can
be configured to run more than one job at a time.
As always, owners of the resources have great flexibility in defining
the policy under which multiple jobs may run, suspend, vacate, etc.  

%%%%%%%%%%%%%%%%%%%%%%%%%%%%%%%%%%%%%%%%%%%%%%%%%%%%%%%%%%%%%%%%%%%%%%
\subsubsection{How Shared Resources are Represented to Condor}
\label{sec:How-Resources-Represented}
%%%%%%%%%%%%%%%%%%%%%%%%%%%%%%%%%%%%%%%%%%%%%%%%%%%%%%%%%%%%%%%%%%%%%%

The way SMP machines are represented to the Condor system is that
the shared resources are broken up into individual \Term{virtual
machines} (``vm'') that can be matched with and claimed by users.
Each virtual machine is represented by an individual ``ClassAd''
(see the ClassAd reference, section~\ref{classad-refernce}, for
details). 
In this way, a single SMP machine will appear to the Condor system as
a collection of seperate virtual machines.  
So for example, if you had an SMP machine named
``vulture.cs.wisc.edu'', it would appear to Condor as multiple
machines, named ``vm1@vulture.cs.wisc.edu'',
``vm2@vulture.cs.wisc.edu'', and so on.

For now, the \Condor{startd} simply creates a seperate virtual machine
for every CPU.
All shared system resources (like RAM, disk space, swap space, etc)
are divided evenly among all the virtual machines
In future versions of Condor, you will be able to configure what
percentage or total value of each resource will go to each vm.  

%%%%%%%%%%%%%%%%%%%%%%%%%%%%%%%%%%%%%%%%%%%%%%%%%%%%%%%%%%%%%%%%%%%%%%
\subsubsection{Configuring Startd Policy for SMP Machines}
\label{sec:Config-SMP-Policy}
%%%%%%%%%%%%%%%%%%%%%%%%%%%%%%%%%%%%%%%%%%%%%%%%%%%%%%%%%%%%%%%%%%%%%%

\Note Be sure you have read and understand
section~\ref{sec:Configuring-Policy} on ``Configuring The Startd
Policy'' before you proceed with this section.

Each virtual machine from an SMP machine is treated as an independent
machine, with its own view of its machine state.
For now, a single set of policy expressions is in place for all
virtual machines simultaneously.  
Eventually, you will be able to explicitly specify seperate policies
for each one.
However, since you do have control over each virtual machine's view of
its own state, you can effectively have seperate policies for each
resource.

For example, you can configure how many of the virtual machines
``notice'' console or tty activity on the SMP as a whole.
Ones that aren't configured to notice any activity will report
ConsoleIdle and KeyboardIdle times from when the startd was started,
(plus a configurable number of seconds).
So, you can setup a 4 CPU machine with all the default startd policy
settings and with the keyboard and console ``connected'' to only one
resource offer.  
Assuming there isn't too much load average (see
section~\ref{sec:SMP-Load} below on ``Load Average for SMP
Machines''), only one resource offer will suspend or vacate its job
when the owner starts typing at their machine again.
The rest of the resource offers could be matched with jobs and leave
them running, even while the user was interactively using the
machine. 

Or, if you wish, you can configure all resources offers to notice all
tty and console activity.
In this case, if a machine owner came back to her machine, all the
currently running jobs would suspend or preempt (depending on your
policy expressions), all at the same time.

%%%%%%%%%%%%%%%%%%%%%%%%%%%%%%%%%%%%%%%%%%%%%%%%%%%%%%%%%%%%%%%%%%%%%%
\subsubsection{Load Average for SMP Machines}
\label{sec:SMP-Load}
%%%%%%%%%%%%%%%%%%%%%%%%%%%%%%%%%%%%%%%%%%%%%%%%%%%%%%%%%%%%%%%%%%%%%%

Most operating systems define the load average for an SMP machine as
the total load on all CPUs.
For example, if you have a 4 CPU machine with 3 CPU bound processes
running at the same time, the load would be 3.0
In Condor, we maintain this view of the total load average and publish
it in all resource classads as \Attr{TotalLoadAvg}.

However, we also define the ``per-CPU'' load average for SMP machines.
In this way, the model that each node on an SMP is a virtual machine,
totally seperate from the other nodes, can be maintained.
All of the default, single-CPU policy expressions can be used directly
on SMP machines, without modification, since the \Attr{LoadAvg} and
\Attr{CondorLoadAvg} attributes are the per-virtual machine versions,
not the total, SMP-wide versions.

The per-CPU load average on SMP machines is a number we basically
invented. 
There is no system call you can use to ask your operating system for
this value.
Here's how it works:

We already compute the load average generated by Condor on each
virtual machine.
We do this by close monitoring of all processes spawned by any of the
Condor daemons, even ones that are orphened and then inheritted by
init. 
This \Term{Condor load average} per virtual machine is reported as
\Attr{CondorLoadAvg} in all resource classads, and the total Condor
load average for the entire machine is reported as
\Attr{TotalCondorLoadAvg}. 
We also have the total, system-wide load average for the entire
machine (reported as \Attr{TotalLoadAvg}).
Basically, we walk through all the resource offers and assign out
portions of the total load average to each one. 
First, we assign out the known Condor load average to each node that
is generating any.  
If there's any load average left in the total system load, that's
considered \Term{owner load}.
Any virtual machines we already think are in the Owner state (like
ones that have keyboard activity, etc), are the first to get assigned
this owner load.
We hand out owner load in increments of at most 1.0, so generally
speaking, no virtual machine has a load average above 1.0.
If we run out of total load average before we run out of resource
offers, all the remaining offers think they have no load average at
all.
If, instead, we run out of resource offers and we still have owner
load left, we start assigning that load to Condor nodes, too, creating
individual nodes with a load average higher than 1.0.

%%%%%%%%%%%%%%%%%%%%%%%%%%%%%%%%%%%%%%%%%%%%%%%%%%%%%%%%%%%%%%%%%%%%%%
\subsubsection{New Config File Parameters for the SMP Startd}
\label{sec:New-SMP-Parameters}
%%%%%%%%%%%%%%%%%%%%%%%%%%%%%%%%%%%%%%%%%%%%%%%%%%%%%%%%%%%%%%%%%%%%%%

This section describes the config file settings that you might want to
customize on an SMP machine.  
These control each virtual machine's view of the system resources, as
described in section~\ref{sec:Config-SMP-Policy} above.

\begin{description}

\item[\Macro{VIRTUAL\_MACHINES\_CONNECTED\_TO\_CONSOLE}] 
\label{param:VirtualMachinesConnectedToConsole}
  This should be an integer which indicates how many of the virtual
  machines the startd is representing should be "connected" to the
  console (in other words, notice when there's console activity).
  This defaults to all virtual machines (N in a machine with N CPUs).
  This must be an integer ("N" isn't a valid setting, that's just used
  to represent the default).

\item[\Macro{VIRTUAL\_MACHINES\_CONNECTED\_TO\_KEYBOARD}]
\label{param:VirtualMachinesConnectedToKeyboard}
  This should be an integer which indicates how many of the virtual
  machines the startd is representing should be "connected" to the
  keyboard (for remote tty activity, as well as console activity).
  Defaults to 1.

\item[\Macro{DISCONNECTED\_KEYBOARD\_IDLE\_BOOST}]
\label{param:DisconnectedKeyboardIdleBoost}
  If there are virtual machines not connected to either the keyboard
  or the console, the corresponding idle time reported will be the
  time since the startd was spawned, plus the value of this parameter.
  It defaults to 1200 seconds (20 minutes). 
  We do this because if the virtual machine is configured not to care
  about keyboard activity, we want it to be available to Condor jobs
  as soon as the startd starts up, instead of having to wait for 15
  minutes or more (which is the default time a machine must be idle
  before Condor will start a job).
  If you don't want this boost, just set the value to 0.  
  If you change your START expression to require more than 15 minutes
  before a job starts, but you still want jobs to start right away on
  some of your SMP nodes, just increase this parameter.

\end{description}


%%%%%%%%%%%%%%%%%%%%%%%%%%%%%%%%%%%%%%%%%%%%%%%%%%%%%%%%%%%%%%%%%%%%%%
\subsubsection{Debug logging in the SMP Startd}
\label{sec:SMP-logging}
%%%%%%%%%%%%%%%%%%%%%%%%%%%%%%%%%%%%%%%%%%%%%%%%%%%%%%%%%%%%%%%%%%%%%%

This section describes how the startd handles its debug messages for
SMP machines.
In general, a given log message will either be something that is
machine-wide (like reporting the total system load average), or it
will be specific to a given virtual machine.
Any log entires specific to a virtual machine will have an extra
header printed out in the entry: \texttt{vm\#:}.  
So, for example, here's the output about system resources that are
being gathered (with \Dflag{FULLDEBUG} and \Dflag{LOAD} turned on) on
a 2 CPU machine with no Condor activity, and the keyboard connected to
both virtual machines:
\begin{verbatim}
11/25 18:15 Swap space: 131064
11/25 18:15 number of kbytes available for (/home/condor/execute): 1345063
11/25 18:15 Looking up RESERVED_DISK parameter
11/25 18:15 Reserving 5120 kbytes for file system
11/25 18:15 Disk space: 1339943
11/25 18:15 Load avg: 0.340000 0.800000 1.170000
11/25 18:15 Idle Time: user= 0 , console= 4 seconds
11/25 18:15 SystemLoad: 0.340   TotalCondorLoad: 0.000  TotalOwnerLoad: 0.340
11/25 18:15 vm1: Idle time: Keyboard: 0        Console: 4
11/25 18:15 vm1: SystemLoad: 0.340  CondorLoad: 0.000  OwnerLoad: 0.340
11/25 18:15 vm2: Idle time: Keyboard: 0        Console: 4
11/25 18:15 vm2: SystemLoad: 0.000  CondorLoad: 0.000  OwnerLoad: 0.000
11/25 18:15 vm1: State: Owner           Activity: Idle
11/25 18:15 vm2: State: Owner           Activity: Idle
\end{verbatim}

If, on the other hand, this machine only had one virtual machine
connected to the keyboard and console, and the other vm was running a
job, it might look something like this:
\begin{verbatim}
11/25 18:19 Load avg: 1.250000 0.910000 1.090000
11/25 18:19 Idle Time: user= 0 , console= 0 seconds
11/25 18:19 SystemLoad: 1.250   TotalCondorLoad: 0.996  TotalOwnerLoad: 0.254
11/25 18:19 vm1: Idle time: Keyboard: 0        Console: 0
11/25 18:19 vm1: SystemLoad: 0.254  CondorLoad: 0.000  OwnerLoad: 0.254
11/25 18:19 vm2: Idle time: Keyboard: 1496     Console: 1496
11/25 18:19 vm2: SystemLoad: 0.996  CondorLoad: 0.996  OwnerLoad: 0.000
11/25 18:19 vm1: State: Owner           Activity: Idle
11/25 18:19 vm2: State: Claimed         Activity: Busy
\end{verbatim}

As you can see, shared system resources are printed without the header
(like total swap space), which vm-specific messages (like the load
average or state of each vm,) get the special header appended.  
