%%%%%%%%%%%%%%%%%%%%%%%%%%%%%%%%%%%%%%%%%%%%%%%%%%%%%%%%%%%%%%%%%%%%%%
\subsection{\label{sec:Configuring-SMP}
Configuring The Startd for SMP Machines}
%%%%%%%%%%%%%%%%%%%%%%%%%%%%%%%%%%%%%%%%%%%%%%%%%%%%%%%%%%%%%%%%%%%%%%

This section describes how to configure the \Condor{startd} for SMP
(Symmetric Multi-Processor) machines.
Beginning with Condor version 6.1, machines with more than one CPU can
be configured to run more than one job at a time.
As always, owners of the resources have great flexibility in defining
the policy under which multiple jobs may run, suspend, vacate, etc.  

%%%%%%%%%%%%%%%%%%%%%%%%%%%%%%%%%%%%%%%%%%%%%%%%%%%%%%%%%%%%%%%%%%%%%%
\subsubsection{\label{sec:How-Resources-Represented}
How Shared Resources are Represented to Condor}
%%%%%%%%%%%%%%%%%%%%%%%%%%%%%%%%%%%%%%%%%%%%%%%%%%%%%%%%%%%%%%%%%%%%%%

The way SMP machines are represented to the Condor system is that
the shared resources are broken up into individual \Term{virtual
machines} (``VM'') that can be matched with and claimed by users.
Each virtual machine is represented by an individual ``ClassAd''
(see the ClassAd reference, section~\ref{classad-reference}, for
details). 
In this way, a single SMP machine will appear to the Condor system as
a collection of separate virtual machines.  
So for example, if you had an SMP machine named
``vulture.cs.wisc.edu'', it would appear to Condor as multiple
machines, named ``vm1@vulture.cs.wisc.edu'',
``vm2@vulture.cs.wisc.edu'', and so on.

You can configure how you want the \Condor{startd} to break up the
shared system resources into the different virtual machines.  
All shared system resources (like RAM, disk space, swap space, etc)
can either be divided evenly among all the virtual machines, with each
CPU getting its own virtual machine, or you can define your own
\Term{virtual machine types}, so that resources can be unevenly
partitioned.
The following section gives details on how to configure Condor to
divide the resources on an SMP machine into seperate virtual machines.

%%%%%%%%%%%%%%%%%%%%%%%%%%%%%%%%%%%%%%%%%%%%%%%%%%%%%%%%%%%%%%%%%%%%%%
\subsubsection{\label{sec:SMP-Divide}
Dividing System Resources in SMP Machines}
%%%%%%%%%%%%%%%%%%%%%%%%%%%%%%%%%%%%%%%%%%%%%%%%%%%%%%%%%%%%%%%%%%%%%%

This section describes the settings that allow you to define your own
virtual machine types and to control how many virtual machines of each
type are reported to Condor.

There are two main ways to go about dividing an SMP machine:

\begin{description}

\item[Define your own virtual machine types.]
  By defining your own types, you can specify what fraction of shared
  system resources (CPU, RAM, swap space and disk space) go to each
  virtual machine.
  Once you define your own types, you can control how many of each
  type are being reported at any given time.

\item[Evenly divide all resources.]  
  If you do not define your own types, the \Condor{startd} will
  automatically	partition your machine into virtual machines for you.
  It will do so by giving each VM a single CPU, and evenly dividing
  all shared resources among each CPU.
  With this default partitioning, you only have to specify how many
  VMs are reported at a time.
  By default, all VMs are reported to Condor.

\end{description}

Begining with Condor version 6.1.6, the number of each type being
reported can be changed at run-time, by issuing a simple reconfig to
the \Condor{startd} (sending a SIGHUP or using \Condor{reconfig}).
However, the definitions for the types themselves cannot be changed
with a reconfig.
If you change any VM type definitions, you must use ``\condor{restart}
-startd'' for that change to take effect.

%%%%%%%%%%%%%%%%%%%%%%%%%%%%%%%%%%%%%%%%%%%%%%%%%%%%%%%%%%%%%%%%%%%%%%
\subsubsection{\label{sec:VM-Type-Define}
Defining Virtual Machine Types}
%%%%%%%%%%%%%%%%%%%%%%%%%%%%%%%%%%%%%%%%%%%%%%%%%%%%%%%%%%%%%%%%%%%%%%

To define your own virtual machine types, you simply add config file
parameters that list how much of each system resource you want in the
given VM type.  You do this with settings of the form
\Macro{VIRTUAL\_MACHINE\_TYPE\_<N>}.
The \Sinful{N} is to be replaced with an integer, for example, 
\MacroNI{VIRTUAL\_MACHINE\_TYPE\_1}, which specifies the virtual 
machine type you're defining.
You will use this number later to configure how many VMs of this type
you want to advertise.

A type describes what share of the total system resources a given
virtual machine has available to it.

The type can be defined in a number of ways:
\begin{itemize}
  \item A simple fraction, such as ``1/4''
  \item A simple percentage, such as ``25\%''
  \item A comma-separated list of attributes, and a percentage,
	fraction, or value for each one.
\end{itemize}
If you just specify a fraction or percentage, that share of the total
system resources, including the number of cpus, will be used for each
virtual machine of this type.
However, if you specify the comma-seperated list, you can fine-tune
the amounts for specific attributes.

Some attributes, such as the number of CPUs and total amount of RAM in
the machine, do not change (unless the machine is turned off and more
chips are added to it).
For these two attributes, you can specify either absolute values, or
percentages of the total available amount.  
For example, in a machine with 128 megs of RAM, you could specify any
of the following to get the same effect: ``mem=64'', ``mem=1/2'', or
``mem=50\%''.
Other resources are dynamic, such as disk space and swap space.
For these, you must specify the percentage or fraction of the total
value that is alloted to each VM, instead of specifying absolute values.
As the total values of these resources change on your machine, each
VM will take its fraction of the total and report that as its
available amount.

All attribute names are case insensitive when defining VM types.
You can use as much or as little of each word as you'd like.
The attributes you can tune are:
\begin{itemize}
  \item cpus
  \item ram
  \item disk (must specify with a fraction or percentage)
  \item swap (must specify with a fraction or percentage)
\end{itemize}
In addition, the following names are equivalent: ``ram'' = ``memory''
and ``swap'' = ``virualmemory''.

Assume the host as 4 CPUs and 256 megs of RAM.
Here are some example VM type definitions, all of which are valid. 
Types 1-3 are all equivalent with each other, as are types 4-6

\MacroNI{VIRTUAL\_MACHINE\_TYPE\_1} = cpus=2, ram=128, swap=25\%, disk=1/2

\MacroNI{VIRTUAL\_MACHINE\_TYPE\_2} = cpus=1/2, memory=128, virt=25\%, disk=50\%

\MacroNI{VIRTUAL\_MACHINE\_TYPE\_3} = c=1/2, m=50\%, v=1/4, disk=1/2

\MacroNI{VIRTUAL\_MACHINE\_TYPE\_4} = c=25\%, m=64, v=1/4, d=25\%

\MacroNI{VIRTUAL\_MACHINE\_TYPE\_5} = 25\%

\MacroNI{VIRTUAL\_MACHINE\_TYPE\_6} = 1/4


%%%%%%%%%%%%%%%%%%%%%%%%%%%%%%%%%%%%%%%%%%%%%%%%%%%%%%%%%%%%%%%%%%%%%%
\subsubsection{\label{sec:Config-VM-Number}
Configuring the Number of Virtual Machines Reported}
%%%%%%%%%%%%%%%%%%%%%%%%%%%%%%%%%%%%%%%%%%%%%%%%%%%%%%%%%%%%%%%%%%%%%%

If you are not defining your own VM types, all you have to configure
is how many of the evenly divided VMs you want reported to Condor.
You do this by setting the \Macro{NUM\_VIRTUAL\_MACHINES} parameter.
You just supply the number of machines you want reported.
If you do not define this yourself, Condor will advertise all the CPUs
in your machines by default.

If you define your own types, things are slightly more complicated.  
Now, you must specify how many virtual machines of each type should be
reported.
You do this with settings of the form
\Macro{NUM\_VIRTUAL\_MACHINES\_TYPE\_<N>}.
The \Sinful{N} is to be replaced with an actual number, for example, 
\MacroNI{NUM\_VIRTUAL\_MACHINES\_TYPE\_1}. 


%%%%%%%%%%%%%%%%%%%%%%%%%%%%%%%%%%%%%%%%%%%%%%%%%%%%%%%%%%%%%%%%%%%%%%
\subsubsection{\label{sec:Config-SMP-Policy}
Configuring Startd Policy for SMP Machines}
%%%%%%%%%%%%%%%%%%%%%%%%%%%%%%%%%%%%%%%%%%%%%%%%%%%%%%%%%%%%%%%%%%%%%%

\Note Be sure you have read and understand
section~\ref{sec:Configuring-Policy} on ``Configuring The Startd
Policy'' before you proceed with this section.

Each virtual machine from an SMP is treated as an independent machine,
with its own view of its machine state.
For now, a single set of policy expressions is in place for all
virtual machines simultaneously.  
Eventually, you will be able to explicitly specify separate policies
for each one.
However, since you do have control over each virtual machine's view of
its own state, you can effectively have separate policies for each
resource.

For example, you can configure how many of the virtual machines
``notice'' console or tty activity on the SMP as a whole.
Ones that aren't configured to notice any activity will report
ConsoleIdle and KeyboardIdle times from when the startd was started,
(plus a configurable number of seconds).
So, you can setup a 4 CPU machine with all the default startd policy
settings and with the keyboard and console ``connected'' to only one
virtual machine.
Assuming there isn't too much load average (see
section~\ref{sec:SMP-Load} below on ``Load Average for SMP
Machines''), only one virtual machine will suspend or vacate its job
when the owner starts typing at their machine again.
The rest of the virtual machines could be matched with jobs and leave
them running, even while the user was interactively using the
machine. 

Or, if you wish, you can configure all virtual machines to notice all
tty and console activity.
In this case, if a machine owner came back to her machine, all the
currently running jobs would suspend or preempt (depending on your
policy expressions), all at the same time.

All of this is controlled with the config file parameters listed
below.    
These settings are fully described in
section~\ref{sec:Startd-Config-File-Entries} on
page~\pageref{sec:Startd-Config-File-Entries} which lists all the
configuration file settings for the \Condor{startd}.

\begin{itemize}
\item \Macro{VIRTUAL\_MACHINES\_CONNECTED\_TO\_CONSOLE}
\item \Macro{VIRTUAL\_MACHINES\_CONNECTED\_TO\_KEYBOARD}
\item \Macro{DISCONNECTED\_KEYBOARD\_IDLE\_BOOST}
\end{itemize}


%%%%%%%%%%%%%%%%%%%%%%%%%%%%%%%%%%%%%%%%%%%%%%%%%%%%%%%%%%%%%%%%%%%%%%
\subsubsection{\label{sec:SMP-Load}
Load Average for SMP Machines}
%%%%%%%%%%%%%%%%%%%%%%%%%%%%%%%%%%%%%%%%%%%%%%%%%%%%%%%%%%%%%%%%%%%%%%

Most operating systems define the load average for an SMP machine as
the total load on all CPUs.
For example, if you have a 4 CPU machine with 3 CPU-bound processes
running at the same time, the load would be 3.0
In Condor, we maintain this view of the total load average and publish
it in all resource ClassAds as \Attr{TotalLoadAvg}.

However, we also define the ``per-CPU'' load average for SMP machines.
In this way, the model that each node on an SMP is a virtual machine,
totally separate from the other nodes, can be maintained.
All of the default, single-CPU policy expressions can be used directly
on SMP machines, without modification, since the \Attr{LoadAvg} and
\Attr{CondorLoadAvg} attributes are the per-virtual machine versions,
not the total, SMP-wide versions.

The per-CPU load average on SMP machines is a number we basically
invented. 
There is no system call you can use to ask your operating system for
this value.
Here's how it works:

We already compute the load average generated by Condor on each
virtual machine.
We do this by close monitoring of all processes spawned by any of the
Condor daemons, even ones that are orphaned and then inherited by
init. 
This \Term{Condor load average} per virtual machine is reported as
\Attr{CondorLoadAvg} in all resource ClassAds, and the total Condor
load average for the entire machine is reported as
\Attr{TotalCondorLoadAvg}. 
We also have the total, system-wide load average for the entire
machine (reported as \Attr{TotalLoadAvg}).
Basically, we walk through all the virtual machines and assign out
portions of the total load average to each one. 
First, we assign out the known Condor load average to each node that
is generating any.  
If there's any load average left in the total system load, that's
considered \Term{owner load}.
Any virtual machines we already think are in the Owner state (like
ones that have keyboard activity, etc), are the first to get assigned
this owner load.
We hand out owner load in increments of at most 1.0, so generally
speaking, no virtual machine has a load average above 1.0.
If we run out of total load average before we run out of virtual
machines, all the remaining machines think they have no load average
at all.
If, instead, we run out of virtual machines and we still have owner
load left, we start assigning that load to Condor nodes, too, creating
individual nodes with a load average higher than 1.0.


%%%%%%%%%%%%%%%%%%%%%%%%%%%%%%%%%%%%%%%%%%%%%%%%%%%%%%%%%%%%%%%%%%%%%%
\subsubsection{\label{sec:SMP-logging}
Debug logging in the SMP Startd}
%%%%%%%%%%%%%%%%%%%%%%%%%%%%%%%%%%%%%%%%%%%%%%%%%%%%%%%%%%%%%%%%%%%%%%

This section describes how the startd handles its debug messages for
SMP machines.
In general, a given log message will either be something that is
machine-wide (like reporting the total system load average), or it
will be specific to a given virtual machine.
Any log entrees specific to a virtual machine will have an extra
header printed out in the entry: \texttt{vm\#:}.  
So, for example, here's the output about system resources that are
being gathered (with \Dflag{FULLDEBUG} and \Dflag{LOAD} turned on) on
a 2 CPU machine with no Condor activity, and the keyboard connected to
both virtual machines:
\begin{verbatim}
11/25 18:15 Swap space: 131064
11/25 18:15 number of kbytes available for (/home/condor/execute): 1345063
11/25 18:15 Looking up RESERVED_DISK parameter
11/25 18:15 Reserving 5120 kbytes for file system
11/25 18:15 Disk space: 1339943
11/25 18:15 Load avg: 0.340000 0.800000 1.170000
11/25 18:15 Idle Time: user= 0 , console= 4 seconds
11/25 18:15 SystemLoad: 0.340   TotalCondorLoad: 0.000  TotalOwnerLoad: 0.340
11/25 18:15 vm1: Idle time: Keyboard: 0        Console: 4
11/25 18:15 vm1: SystemLoad: 0.340  CondorLoad: 0.000  OwnerLoad: 0.340
11/25 18:15 vm2: Idle time: Keyboard: 0        Console: 4
11/25 18:15 vm2: SystemLoad: 0.000  CondorLoad: 0.000  OwnerLoad: 0.000
11/25 18:15 vm1: State: Owner           Activity: Idle
11/25 18:15 vm2: State: Owner           Activity: Idle
\end{verbatim}

If, on the other hand, this machine only had one virtual machine
connected to the keyboard and console, and the other vm was running a
job, it might look something like this:
\begin{verbatim}
11/25 18:19 Load avg: 1.250000 0.910000 1.090000
11/25 18:19 Idle Time: user= 0 , console= 0 seconds
11/25 18:19 SystemLoad: 1.250   TotalCondorLoad: 0.996  TotalOwnerLoad: 0.254
11/25 18:19 vm1: Idle time: Keyboard: 0        Console: 0
11/25 18:19 vm1: SystemLoad: 0.254  CondorLoad: 0.000  OwnerLoad: 0.254
11/25 18:19 vm2: Idle time: Keyboard: 1496     Console: 1496
11/25 18:19 vm2: SystemLoad: 0.996  CondorLoad: 0.996  OwnerLoad: 0.000
11/25 18:19 vm1: State: Owner           Activity: Idle
11/25 18:19 vm2: State: Claimed         Activity: Busy
\end{verbatim}

As you can see, shared system resources are printed without the header
(like total swap space), which VM-specific messages (like the load
average or state of each VM,) get the special header appended.  
