%%%%%%%%%%%%%%%%%%%%%%%%%%%%%%%%%%%%%%%%%%%%%%%%%%%%%%%%%%%%%%%%%%%%%%
\subsection{\label{sec:Configuring-SMP}
Configuring The \Condor{startd} for SMP Machines}
%%%%%%%%%%%%%%%%%%%%%%%%%%%%%%%%%%%%%%%%%%%%%%%%%%%%%%%%%%%%%%%%%%%%%%

\index{SMP machines!configuration|(}
\index{partitioning SMP machines}
This section describes how to configure the \Condor{startd} for SMP
(Symmetric Multi-Processor) machines.
Machines with more than one CPU may
be configured to run more than one job at a time.
As always, owners of the resources have great flexibility in defining
the policy under which multiple jobs may run, suspend, vacate, etc.  

%%%%%%%%%%%%%%%%%%%%%%%%%%%%%%%%%%%%%%%%%%%%%%%%%%%%%%%%%%%%%%%%%%%%%%
\subsubsection{\label{sec:How-Resources-Represented}
How Shared Resources are Represented to HTCondor}
%%%%%%%%%%%%%%%%%%%%%%%%%%%%%%%%%%%%%%%%%%%%%%%%%%%%%%%%%%%%%%%%%%%%%%

The way SMP machines are represented to the HTCondor system is that
the shared resources are broken up into individual \Term{slots}.
Each slot can be matched and claimed by users.
Each slot is represented by an individual ClassAd.
See the ClassAd reference in section~\ref{sec:classad-reference} for details. 
In this way, each SMP machine will appear to the HTCondor system as
a collection of separate slots.
As an example, an SMP machine named
\Expr{vulture.cs.wisc.edu} would appear to HTCondor as the
multiple machines, named \Expr{slot1@example.com},
\Expr{slot2@example.com},
\Expr{slot3@example.com}, and so on.

The way that the \Condor{startd} breaks up the
shared system resources into the different slots
is configurable.
All shared system resources,
such as RAM, disk space, swap space, or specialized custom resources,
can either be divided evenly among all the slots, 
with each CPU getting its own slot.
It is also possible to define \Term{slot types},
so that resources can be unevenly partitioned.  
Regardless of the partitioning scheme used, it is important
to remember the goal is to create a representative slot
ClassAd, to be used for matchmaking with jobs.  HTCondor does not
directly enforce slot shared resource allocations, and jobs
are free to oversubscribe to shared resources.

Consider an example where two slots are each defined with 50\Percent\ 
of available RAM.  
The resultant ClassAd for each slot will advertise one half the available RAM.
Users may submit jobs with RAM requirements that match these slots.
However, jobs run on either of these two slots are free to
consume more than 50\Percent\ of available RAM.  
HTCondor will not directly enforce a RAM utilization limit on either slot.
If a shared resource enforcement capability is needed, 
it is possible to write a \Condor{startd} policy that will evict 
a job that oversubscribes to shared resources, 
as described in section~\ref{sec:Config-SMP-Policy}.

The following sections detail on how to configure HTCondor to
divide the slot resources on an SMP machine.

%%%%%%%%%%%%%%%%%%%%%%%%%%%%%%%%%%%%%%%%%%%%%%%%%%%%%%%%%%%%%%%%%%%%%%
\subsubsection{\label{sec:SMP-Divide}
Dividing System Resources for Slots within Machines}
%%%%%%%%%%%%%%%%%%%%%%%%%%%%%%%%%%%%%%%%%%%%%%%%%%%%%%%%%%%%%%%%%%%%%%
\index{partitioning SMP machines}

The number of each slot type
reported can be changed at run time by issuing a reconfiguration
command to
the \Condor{startd} daemon by sending a SIGHUP or using \Condor{reconfig}.
However, the definitions for the types themselves cannot be changed
with reconfiguration.
To change any slot type definition, use \Condor{restart}
for that change to take effect.

There are two main ways to go about partitioning a machine:

\begin{description}

\item[Evenly divide all resources.]  
  Without individualized slot types, the \Condor{startd} will
  automatically	partition the machine into slots.
  It will do so by placing a single CPU in each slot, 
  evenly dividing all shared resources among the slots.
  With this default partitioning, the only specification becomes how many
  slots are reported at a time.
  By default, all slots are reported to HTCondor.

  The number of slots within the SMP machine is the only attribute
  that needs to be defined.
  Its definition is accomplished by setting the configuration
  variable \Macro{NUM\_SLOTS} to the integer number of slots desired.
  If variable \MacroNI{NUM\_SLOTS} is not defined,
  it defaults to the number of CPUs within the SMP machine.
  \MacroNI{NUM\_SLOTS} \emph{cannot} be used to make HTCondor advertise more
  slots than there are CPUs on the machine.
  To do that, use \Macro{NUM\_CPUS}.

\item[Define individualized slot types.]
  Individualized slot types may specify what fraction of shared
  system resources (CPU, RAM, swap space, disk space, and custom resources)
  go to each slot.
  Given these types, configuration may control how many of each
  type are reported at any given time.

\end{description}

To define individualized slot types, add configuration variables
that list how much of each system resource is for the given slot type.
Do this by defining configuration variables named with the form
\Macro{SLOT\_TYPE\_<N>}.
The \verb@<N>@ represents an integer,
so for example, there may be the variable named \MacroNI{SLOT\_TYPE\_1}.
Note that there may be multiple slots of each type.  The number created
is configured with \MacroNI{NUM\_SLOTS\_TYPE\_<N>} as described later in
this section.

The definition of variable \MacroNI{SLOT\_TYPE\_<N>}
is a comma and/or space separated list that
describes what share of the total system resources a given
slot has available to it.
These types can be defined by:
\begin{itemize}
  % \frac{1}{4} doesn't work
  %\item A simple fraction, such as \frac{1}{4}
  \item A simple fraction, such as 1/4
  \item A simple percentage, such as 25\Percent
  \item A comma-separated list of attributes, with a percentage,
	fraction, numerical value, or the key word \Expr{auto} for each one
  \item A comma-separated list, including a blanket value that serves
        as a default for any resources not explicitly specified in the list
\end{itemize}
A simple fraction or percentage causes a fractional allocation
of the total system resources.
This includes the number of CPUs.
The attributes that specify the number of CPUs
and the total amount of RAM in
the SMP machine do not change.
For these attributes, specify absolute values,
percentages of the total available amount, or \Expr{auto}.  
For example, in a machine with 128 Mbytes of RAM,
all the following definitions result in the same allocation amount.
\begin{verbatim}
  SLOT_TYPE_1 = mem=64

  SLOT_TYPE_1 = mem=1/2

  SLOT_TYPE_1 = mem=50%

  SLOT_TYPE_1 = mem=auto
\end{verbatim}

Other attributes are dynamic, such as disk space and swap space.
For these, specify a percentage or a fraction of the total
value that is allocated to each slot, instead of specifying absolute values.
As the total values of these resources change on the machine,
each slot will take its fraction of the total and report that as its
available amount.

The disk space allocated to each slot is taken from the disk partition
containing the slot's \MacroUNI{EXECUTE} directory. 
If every slot is in a different partition, 
then each one may be defined with up to
100\Percent\ for its disk share.
If some slots are in the same partition, 
then their total is not allowed to exceed 100\Percent.

The four predefined attribute names are case insensitive when defining 
slot types.
The first letter of the attribute name distinguishes between
these attributes.
The four attributes, with several examples of acceptable names for each:
\begin{itemize}
  \item Cpus, C, c, cpu 
  \item ram, RAM, MEMORY, memory, Mem, R, r, M, m
  \item disk, Disk, D, d
  \item swap, SWAP, S, s, VirtualMemory, V, v
\end{itemize}

As an example, consider a
host of 4 CPUs and 256 Mbytes of RAM.
Here are valid example slot type definitions. 
Types 1-3 are all equivalent to each other, as are types 4-6.
Note that this is not a real configuration, as all of these slot types together
add up to more than 100\Percent\ of the various system resources.

\begin{verbatim}
  SLOT_TYPE_1 = cpus=2, ram=128, swap=25%, disk=1/2

  SLOT_TYPE_2 = cpus=1/2, memory=128, virt=25%, disk=50%

  SLOT_TYPE_3 = c=1/2, m=50%, v=1/4, disk=1/2

  SLOT_TYPE_4 = c=25%, m=64, v=1/4, d=25%

  SLOT_TYPE_5 = 25%

  SLOT_TYPE_6 = 1/4
\end{verbatim}

The default value for each resource share is \Expr{auto}.  The share
may also be explicitly set to \Expr{auto}.  All slots with the value
\Expr{auto} for a given type of resource will evenly divide
whatever remains after subtracting out whatever was explicitly
allocated in other slot definitions.  For example, if one slot is
defined to use 10\Percent\ of the memory and the rest define it as
\Expr{auto} (or leave it undefined), then the rest of the slots will
evenly divide 90\Percent\ of the memory between themselves.

These two examples are equivalent to each other.
The disk share is set to \Expr{auto},
CPUs is 1, and everything else is 50\Percent:

\begin{verbatim}
SLOT_TYPE_1 = cpus=1, ram=1/2, swap=50%

SLOT_TYPE_1 = cpus=1, disk=auto, 50%
\end{verbatim}

The number of slots of each type is set with the
configuration variable
\Macro{NUM\_SLOTS\_TYPE\_<N>},
where \verb@<N>@ is the type as given in the
\MacroNI{SLOT\_TYPE\_<N>} variable.

Note that it is possible to set the configuration variables such
that they specify an impossible configuration.
If this occurs, the \Condor{startd} daemon fails after writing
a message to its log attempting to indicate the configuration
requirements that it could not implement.


In addition to the standard resources of CPUs, memory, disk, and swap,
the administrator may also define custom resources on 
a localized per-machine basis.
To implement this, 
a list of names of the local machine resources are defined using configuration 
variable \Macro{MACHINE\_RESOURCE\_NAMES}.
This example defines two resources,
a GPU and an actuator:
\begin{verbatim}
MACHINE_RESOURCE_NAMES = gpu, actuator
\end{verbatim}

The quantities of available resources are defined using configuration
variables of the form \Macro{MACHINE\_RESOURCE\_<name>},
where \Expr{<name>} is as defined by variable 
\MacroNI{MACHINE\_RESOURCE\_NAMES}, as shown in this example:
\begin{verbatim}
MACHINE_RESOURCE_gpu = 16
MACHINE_RESOURCE_actuator = 8
\end{verbatim}

Local machine resource names defined in this way may now be used in conjunction 
with \Macro{SLOT\_TYPE\_<N>}, using all the same syntax described
earlier in this section.
The following example demonstrates
the definition of static and partitionable slot types with local machine 
resources:
\begin{verbatim}
# declare one partitionable slot with half of the GPUs, 6 actuators, and
# 50% of all other resources:
SLOT_TYPE_1 = gpu=50%,actuator=6,50%
SLOT_TYPE_1_PARTITIONABLE = TRUE
NUM_SLOTS_TYPE_1 = 1

# declare two static slots, each with 25% of the GPUs, 1 actuator, and
# 25% of all other resources: 
SLOT_TYPE_2 = gpu=25%,actuator=1,25%
SLOT_TYPE_2_PARTITIONABLE = FALSE
NUM_SLOTS_TYPE_2 = 2
\end{verbatim}

A job may request these local machine resources using the 
syntax \SubmitCmd{request\_<name>}, 
as described in section~\ref{sec:SMP-dynamicprovisioning}.  
This example shows a portion of a submit description file 
that requests GPUs and an actuator:
\begin{verbatim}
universe = vanilla

# request two GPUs and one actuator:
request_gpu = 2
request_actuator = 1

queue
\end{verbatim}

The slot ClassAd will represent each local machine resource
with the following attributes:
\begin{description}
\item{\Attr{Total<name>}: the total quantity of the resource 
  identified by \Expr{<name>}}
\item{\Attr{Detected<name>}: the quantity detected of the resource
  identified by \Expr{<name>}; this attribute is
  currently equivalent to \Attr{Total<name>}}
\item{\Attr{TotalSlot<name>}: the quantity of the resource
  identified by \Expr{<name>} allocated to this slot}
\item{\Attr{<name>}: the amount of the resource
  identified by \Expr{<name>} available to be used on this slot}
\end{description}

From the example given, the \Expr{gpu} resource would be represented by
the ClassAd attributes
\Attr{TotalGpu}, \Attr{DetectedGpu}, \Attr{TotalSlotGpu}, and \Attr{Gpu}.
In the job ClassAd, 
the amount of the requested machine resource appears 
in a job ClassAd attribute named \Attr{Request<name>}.
For this example,
the two attributes will be \Attr{RequestGpu} and \Attr{RequestActuator}.

%%%%%%%%%%%%%%%%%%%%%%%%%%%%%%%%%%%%%%%%%%%%%%%%%%%%%%%%%%%%%%%%%%%%%%
\subsubsection{\label{sec:Config-SMP-Policy}
Configuring Startd Policy for SMP Machines}
%%%%%%%%%%%%%%%%%%%%%%%%%%%%%%%%%%%%%%%%%%%%%%%%%%%%%%%%%%%%%%%%%%%%%%

\index{configuration!SMP machines}
Section~\ref{sec:Configuring-Policy} details the Startd
Policy Configuration.
This section continues the discussion with respect to SMP machines.

Each slot within an SMP machine is treated as an
independent machine,
each with its own view of its machine state.
There is a single set of policy expressions for the SMP machine
as a whole.
This policy may consider the slot state(s) in its expressions.
This makes some policies easy to set, but it makes
other policies difficult or impossible to set.

An easy policy to set
configures how many of the slots
notice console or tty activity on the SMP as a whole.
Slots that are not configured to notice any activity will report
ConsoleIdle and KeyboardIdle times from when the
\Condor{startd} daemon was started,
(plus a configurable number of seconds).
With this, you can set up a multiple CPU machine with
the default policy
settings plus add that the keyboard and console noticed by only one
slot.
Assuming a reasonable load average (see
section~\ref{sec:SMP-Load} below on ``Load Average for SMP
Machines''), only the one slot will suspend or vacate its job
when the owner starts typing at their machine again.
The rest of the slots could be matched with jobs and leave
them running, even while the user was interactively using the
machine. 
If the default policy is used,
all slots notice
tty and console activity
and
currently running jobs would suspend or preempt.

This example policy is
controlled with the following configuration variables.
\begin{itemize}
\item \Macro{SLOTS\_CONNECTED\_TO\_CONSOLE}
\item \Macro{SLOTS\_CONNECTED\_TO\_KEYBOARD}
\item \Macro{DISCONNECTED\_KEYBOARD\_IDLE\_BOOST}
\end{itemize}

These configuration variables are fully described in
section~\ref{sec:Startd-Config-File-Entries} on
page~\pageref{sec:Startd-Config-File-Entries} which lists all the
configuration file settings for the \Condor{startd}.

% Karen's edit goes to this line.
% need discussion here about what canNOT be done given the single
% set  of policy expressions.
The configuration of slots allows each slot to advertise
its own machine ClassAd.
Yet, there is only one set of policy expressions for the SMP
machine as a whole.
This makes the implementation of certain types of policies impossible.
While evaluating the state of one slot (within the SMP machine),
the state of other slots (again within the SMP machine) are not
available.
Decisions for one slot cannot be based on what other machines within the SMP
are doing.

Specifically, the evaluation of a slot policy expression works in
the following way.
\begin{enumerate}
\item 
The configuration file specifies policy expressions that are shared among
all of the slots on the SMP machine.
\item 
Each slot reads the configuration file and sets up its own machine ClassAd.
\item 
Each slot is now separate from the others.  It has a
different state, a different machine ClassAd, and if there is a job
running, a separate job ad.
Each slot periodically
evaluates the policy expressions, changing its own state
as necessary.
This occurs independently of the other slots on the machine.
So, if the \Condor{startd} daemon is evaluating a policy expression
on a specific slot,
and the policy expression refers to \Attr{ProcID}, \Attr{Owner},
or any attribute from a job ad,
it \emph{always} refers to the ClassAd of the
job running on the specific slot.
\end{enumerate}

To set a different policy for the slots within an SMP machine,
a (\verb@SUSPEND@) policy will be of the form
\begin{verbatim}
SUSPEND = ( (SlotID == 1) && (PolicyForSlot1) ) || \
            ( (SlotID == 2) && (PolicyForSlot2) )
\end{verbatim}
where \verb@(PolicyForSlot1)@ and \verb@(PolicyForSlot2)@ are the
desired expressions for each slot.

%%%%%%%%%%%%%%%%%%%%%%%%%%%%%%%%%%%%%%%%%%%%%%%%%%%%%%%%%%%%%%%%%%%%%%
\subsubsection{\label{sec:SMP-Load}
Load Average for SMP Machines}
%%%%%%%%%%%%%%%%%%%%%%%%%%%%%%%%%%%%%%%%%%%%%%%%%%%%%%%%%%%%%%%%%%%%%%

Most operating systems define the load average for an SMP machine as
the total load on all CPUs.
For example, if you have a 4-CPU machine with 3 CPU-bound processes
running at the same time, the load would be 3.0
In HTCondor, we maintain this view of the total load average and publish
it in all resource ClassAds as \Attr{TotalLoadAvg}.

HTCondor also provides a per-CPU load average for SMP machines.
This nicely represents the model that each node on an SMP is a slot,
separate from the other nodes.
All of the default, single-CPU policy expressions can be used directly
on SMP machines, without modification, since the \Attr{LoadAvg} and
\Attr{CondorLoadAvg} attributes are the per-slot versions,
not the total, SMP-wide versions.

The per-CPU load average on SMP machines is an HTCondor invention. 
No system call exists to ask the operating system for this value.
HTCondor already computes the load average generated by HTCondor on each
slot.
It does this by close monitoring of all processes spawned by any of the
HTCondor daemons, even ones that are orphaned and then inherited by
\Prog{init}. 
This HTCondor load average per slot is reported as
the attribute
\Attr{CondorLoadAvg} in all resource ClassAds, and the total HTCondor
load average for the entire machine is reported as
\Attr{TotalCondorLoadAvg}. 
The total, system-wide load average for the entire
machine  is reported as \Attr{TotalLoadAvg}.
Basically, HTCondor walks through all the slots and assigns out
portions of the total load average to each one. 
First, HTCondor assigns the known HTCondor load average to each node that
is generating load.  
If there's any load average left in the total system load, it is
considered an owner load.
Any slots HTCondor believes are in the Owner state (like
ones that have keyboard activity), are the first to get assigned
this owner load.
HTCondor hands out owner load in increments of at most 1.0, so generally
speaking, no slot has a load average above 1.0.
If HTCondor runs out of total load average before it runs out of virtual
machines, all the remaining machines believe that they have no load average
at all.
If, instead, HTCondor runs out of slots and it still has owner
load remaining, HTCondor starts assigning that load to HTCondor nodes as
well,
giving individual nodes with a load average higher than 1.0.


%%%%%%%%%%%%%%%%%%%%%%%%%%%%%%%%%%%%%%%%%%%%%%%%%%%%%%%%%%%%%%%%%%%%%%
\subsubsection{\label{sec:SMP-logging}
Debug logging in the SMP Startd}
%%%%%%%%%%%%%%%%%%%%%%%%%%%%%%%%%%%%%%%%%%%%%%%%%%%%%%%%%%%%%%%%%%%%%%

This section describes how the \Condor{startd} daemon
handles its debugging messages for SMP machines.
In general, a given log message will either be something that is
machine-wide (like reporting the total system load average), or it
will be specific to a given slot.
Any log entrees specific to a slot have an extra
header printed out in the entry: \texttt{slot\#:}.  
So, for example, here's the output about system resources that are
being gathered (with \Dflag{FULLDEBUG} and \Dflag{LOAD} turned on) on
a 2-CPU machine with no HTCondor activity, and the keyboard connected to
both slots:
\begin{verbatim}
11/25 18:15 Swap space: 131064
11/25 18:15 number of Kbytes available for (/home/condor/execute): 1345063
11/25 18:15 Looking up RESERVED_DISK parameter
11/25 18:15 Reserving 5120 Kbytes for file system
11/25 18:15 Disk space: 1339943
11/25 18:15 Load avg: 0.340000 0.800000 1.170000
11/25 18:15 Idle Time: user= 0 , console= 4 seconds
11/25 18:15 SystemLoad: 0.340   TotalCondorLoad: 0.000  TotalOwnerLoad: 0.340
11/25 18:15 slot1: Idle time: Keyboard: 0        Console: 4
11/25 18:15 slot1: SystemLoad: 0.340  CondorLoad: 0.000  OwnerLoad: 0.340
11/25 18:15 slot2: Idle time: Keyboard: 0        Console: 4
11/25 18:15 slot2: SystemLoad: 0.000  CondorLoad: 0.000  OwnerLoad: 0.000
11/25 18:15 slot1: State: Owner           Activity: Idle
11/25 18:15 slot2: State: Owner           Activity: Idle
\end{verbatim}

If, on the other hand, this machine only had one slot
connected to the keyboard and console, and the other slot was running a
job, it might look something like this:
\begin{verbatim}
11/25 18:19 Load avg: 1.250000 0.910000 1.090000
11/25 18:19 Idle Time: user= 0 , console= 0 seconds
11/25 18:19 SystemLoad: 1.250   TotalCondorLoad: 0.996  TotalOwnerLoad: 0.254
11/25 18:19 slot1: Idle time: Keyboard: 0        Console: 0
11/25 18:19 slot1: SystemLoad: 0.254  CondorLoad: 0.000  OwnerLoad: 0.254
11/25 18:19 slot2: Idle time: Keyboard: 1496     Console: 1496
11/25 18:19 slot2: SystemLoad: 0.996  CondorLoad: 0.996  OwnerLoad: 0.000
11/25 18:19 slot1: State: Owner           Activity: Idle
11/25 18:19 slot2: State: Claimed         Activity: Busy
\end{verbatim}

As you can see, shared system resources are printed without the header
(like total swap space), and slot-specific messages (like the load
average or state of each slot) get the special header appended.  


%%%%%%%%%%%%%%%%%%%%%%%%%%%%%%%%%%%%%%%%%%%%%%%%%%%%%%%%%%%%%%%%%%%%%%
\subsubsection{\label{sec:SMP-exprs}
Configuring STARTD\_ATTRS on a per-slot basis}
%%%%%%%%%%%%%%%%%%%%%%%%%%%%%%%%%%%%%%%%%%%%%%%%%%%%%%%%%%%%%%%%%%%%%%

The \Macro{STARTD\_ATTRS} (and legacy \MacroNI{STARTD\_EXPRS}) settings
can be configured on a per-slot basis.
The \Condor{startd} daemon builds the list of items to
advertise by combining the lists in this order:
\begin{enumerate}
\item{\MacroNI{STARTD\_ATTRS}}
\item{\MacroNI{STARTD\_EXPRS}}
\item{\MacroNI{SLOT<N>\_STARTD\_ATTRS}}
\item{\MacroNI{SLOT<N>\_STARTD\_EXPRS}}
\end{enumerate}

For example, consider the following configuration:
\begin{verbatim}
STARTD_ATTRS = favorite_color, favorite_season
SLOT1_STARTD_ATTRS = favorite_movie
SLOT2_STARTD_ATTRS = favorite_song
\end{verbatim}

This will result in the \Condor{startd} ClassAd for
slot1 defining values for
\Attr{favorite\_color}, \Attr{favorite\_season},
and \Attr{favorite\_movie}.
slot2 will have values for
\Attr{favorite\_color}, \Attr{favorite\_season}, and \Attr{favorite\_song}.

Attributes themselves in the \Expr{STARTD\_ATTRS} list
can also be defined on a per-slot basis.
Here is another example:

\begin{verbatim}
favorite_color = "blue"
favorite_season = "spring"
STARTD_ATTRS = favorite_color, favorite_season
SLOT2_favorite_color = "green"
SLOT3_favorite_season = "summer"
\end{verbatim}

For this example, the \Condor{startd} ClassAds are
\begin{description}
\item{slot1}:
\begin{verbatim}
favorite_color = "blue"
favorite_season = "spring"
\end{verbatim}
\item{slot2}:
\begin{verbatim}
favorite_color = "green"
favorite_season = "spring"
\end{verbatim}
\item{slot3}:
\begin{verbatim}
favorite_color = "blue"
favorite_season = "summer"
\end{verbatim}
\end{description}

%%%%%%%%%%%%%%%%%%%%%%%%%%%%%%%%%%%%%%%%%%%%%%%%%%%%%%%%%%%%%%%%%%%%%%
\subsubsection{\label{sec:SMP-dynamicprovisioning}
Dynamic \Condor{startd} Provisioning: Dynamic Slots}
%%%%%%%%%%%%%%%%%%%%%%%%%%%%%%%%%%%%%%%%%%%%%%%%%%%%%%%%%%%%%%%%%%%%%%
\index{dynamic \Condor{startd} provisioning}
\index{slots!dynamic \Condor{startd} provisioning}
\index{slots!subdividing slots}
\index{dynamic slots}
\index{partitionable slots}

\Term{Dynamic provisioning},
also referred to as a partitionable \Condor{startd} or as dynamic slots,
allows users to mark slots as partitionable. 
This means that more than one job can occupy a single slot at any one time. 
Typically, slots have a fixed set of resources,
including the CPUs, memory and disk space. 
By partitioning the slot, 
these resources become more flexible and able to be better utilized.

Dynamic provisioning provides powerful configuration
possibilities, and so should be used with care. 
Specifically, while preemption occurs for each individual dynamic slot,
it cannot occur directly for the partitionable slot, 
or for groups of dynamic slots. 
For example, for a large number of jobs requiring 1GB of memory,
a pool might be split up into 1GB dynamic slots. 
In this instance a job requiring 2GB of memory will be starved
and unable to run.  A partial solution to this problem is provided
by \Condor{defrag}, which is discussed in section~\ref{sec:SMP-defrag}.

Here is an example that demonstrates how more than one job
can be matched to a single slot using dynamic provisioning.
In this example, slot1 has the following resources:
\begin{description}
  \item{cpu=10}
  \item{memory=10240}
  \item{disk=BIG}
\end{description}
Assume that JobA is allocated to this slot.
JobA includes the following requirements:
\begin{description}
  \item{cpu=3}
  \item{memory=1024}
  \item{disk=10240} 
\end{description}
The portion of the slot that is utilized is referred to as Slot1\_1,
and after allocation, the slot advertises that it has
the following resources still available:
\begin{description}
  \item{cpu=7}
  \item{memory=9216}
  \item{disk=BIG-10240}
\end{description}
As each new job is allocated to Slot1,
it breaks into Slot1\_1, Slot1\_2, etc., until the entire set of
available resources have been consumed by jobs.

To enable dynamic provisioning, 
set the \Macro{SLOT\_TYPE\_<N>\_PARTITIONABLE} configuration variable 
to \Expr{True}.
The string \MacroNI{N} within the configuration variable name
is the slot number.  For the most common cases the machine should
be configured for one slot, managing all the resources on the machine.
To do so, set the following configuration variables:

\begin{verbatim}
NUM_SLOTS=1
NUM_SLOTS_TYPE_1=1
SLOT_TYPE_1_PARTITIONABLE=true
\end{verbatim}

In a pool using dynamic provisioning, 
jobs can have extra, and desired, resources specified in the submit
description file:
\begin{description}
  \item{request\_cpus}
  \item{request\_memory}
  \item{request\_disk (in kilobytes)}
\end{description}

This example shows a portion of the job submit description file
for use when submitting a job to a pool with dynamic provisioning.
\begin{verbatim}
universe = vanilla

request_cpus = 3
request_memory = 1024
request_disk = 10240

queue 
\end{verbatim}

For each type of slot,
the original, partitionable slot and the new smaller, dynamic slots,
an attribute is added to identify it.
The original slot,
as defined at page~\pageref{PartitionableSlot-machine-attribute},
will have an attribute stating 
\begin{verbatim}
  PartitionableSlot = True
\end{verbatim}
and the dynamic slots will have an attribute, 
as defined at page~\pageref{DynamicSlot-machine-attribute},
\begin{verbatim}
  DynamicSlot = True
\end{verbatim}
These attributes may be used in a \MacroNI{START} expression for 
the purposes of creating detailed policies.

A partitionable slot will always appear as though it is not running a job.
It will eventually show as having no available resources, 
which will prevent further matching to new jobs.
Because it has been effectively broken up into smaller slots,
these will show as running jobs directly.
These dynamic slots can also be preempted in the same way as 
nonpartitioned slots.

%%%%%%%%%%%%%%%%%%%%%%%%%%%%%%%%%%%%%%%%%%%%%%%%%%%%%%%%%%%%%%%%%%%%%%
\subsubsection{\label{sec:SMP-resource-defaults}
Defaults for Partitionable Slot Sizes}
%%%%%%%%%%%%%%%%%%%%%%%%%%%%%%%%%%%%%%%%%%%%%%%%%%%%%%%%%%%%%%%%%%%%%%

If a job does not specify the required number of CPUs, amount of memory,
or disk space, there are ways for the administrator to set default
values for all of these parameters.

First, if any of these attributes are not set in the submit description file,
there are three variables in the configuration file that \condor{submit}
will use to fill in default values.  These are

\begin{description}
  \item{\Macro{JOB\_DEFAULT\_REQUESTMEMORY}}
  \item{\Macro{JOB\_DEFAULT\_REQUESTDISK}}
  \item{\Macro{JOB\_DEFAULT\_REQUESTCPUS}}
\end{description}

The value of these variables can be ClassAd expressions.  
The default values for these variables, should they not be set are

\begin{description}
  \item{\MacroNI{JOB\_DEFAULT\_REQUESTMEMORY} = \Expr{ifThenElse(MemoryUsage =!= UNDEFINED, MemoryUsage, 1)}}
  \item{\MacroNI{JOB\_DEFAULT\_REQUESTCPUS} = \Expr{1}}
  \item{\MacroNI{JOB\_DEFAULT\_REQUESTDISK} = \Expr{DiskUsage}}
\end{description}

Note that these default values are chosen such that 
jobs matched to partitionable slots function similar to static slots.

Once the job has been matched, 
and has made it to the execute machine, 
the \Condor{startd} has the ability to modify these 
resource requests before using them to size the
actual dynamic slots carved out of the partitionable slot.  
Clearly, for the job to work,
the \Condor{startd} daemon must create slots with at least 
as many resources as the job needs.  
However,
it may be valuable to create dynamic slots somewhat bigger 
than the job's request, 
as subsequent jobs may be more likely to reuse the newly created slot 
when the initial job is done using it.

The \Condor{startd} configuration variables which control this 
and their defaults are

\begin{description}
  \item{\Macro{MODIFY\_REQUEST\_EXPR\_REQUESTCPUS} = \Expr{quantize(RequestCpus, \{1\})}}
  \item{\Macro{MODIFY\_REQUEST\_EXPR\_REQUESTMEMORY} = \Expr{quantize(RequestMemory, \{TotalSlotMem / TotalSlotCpus / 4\}) }}
  \item{\Macro{MODIFY\_REQUEST\_EXPR\_REQUESTDISK} = \Expr{quantize(RequestDisk, \{1024\}) }}
\end{description}

%%%%%%%%%%%%%%%%%%%%%%%%%%%%%%%%%%%%%%%%%%%%%%%%%%%%%%%%%%%%%%%%%%%%%%
\subsubsection{\label{sec:SMP-defrag}
Defragmenting Dynamic Slots}
%%%%%%%%%%%%%%%%%%%%%%%%%%%%%%%%%%%%%%%%%%%%%%%%%%%%%%%%%%%%%%%%%%%%%%
\index{HTCondor daemon!condor\_defrag@\Condor{defrag}}
\index{daemon!condor\_defrag@\Condor{defrag}}
\index{condor\_defrag daemon}

When partitionable slots are used, some attention must be given to the
problem of the starvation of large jobs due to the fragmentation of resources.
The problem is that over time the machine resources may become
partitioned into slots suitable for running small jobs.
If a sufficient number of these slots do not happen to become idle at the
same time on a machine, then a large job will not be able to claim that
machine, even if the large job has a better priority than the small jobs.

One way of addressing the partitionable slot fragmentation problem is
to periodically drain all jobs from fragmented machines so that they
become defragmented.  
The \Condor{defrag} daemon implements a configurable policy for doing that.
To use this daemon,
\MacroNI{DEFRAG} must be added to \MacroNI{DAEMON\_LIST},
and the defragmentation policy must be configured.
Typically, only one instance of the \Condor{defrag} daemon would be
run per pool.  
It is a lightweight daemon that should not require a lot of system resources.

Here is an example configuration that puts the \Condor{defrag} daemon to work:

\begin{verbatim}
DAEMON_LIST = $(DAEMON_LIST) DEFRAG
DEFRAG_INTERVAL = 3600
DEFRAG_DRAINING_MACHINES_PER_HOUR = 1.0
DEFRAG_MAX_WHOLE_MACHINES = 20
DEFRAG_MAX_CONCURRENT_DRAINING = 10
\end{verbatim}

This example policy tells \Condor{defrag} to initiate draining 
jobs from 1 machine per hour,
but to avoid initiating new draining if there are 
20 completely defragmented machines or 10 machines in a draining state.
A full description of each configuration variable
used by the \Condor{defrag} daemon may be found in 
section~\ref{sec:Config-defrag}.

By default, when a machine is drained, existing jobs are gracefully evicted.
This means that each job will be allowed to use the remaining
time promised to it by \MacroNI{MaxJobRetirementTime}.  
If the job has not finished when the retirement time runs out, 
the job will be killed with a soft kill signal, 
so that it has an opportunity to save a checkpoint
(if the job supports this).
No new jobs will be allowed to start while the machine is draining.
To reduce unused time on the
machine caused by some jobs having longer retirement time than others,
the eviction of jobs with shorter retirement time is delayed until the
job with the longest retirement time needs to be evicted.

There is a trade off between reduced starvation and throughput.
Frequent draining of machines reduces the chance of starvation of
large jobs.  However, frequent draining reduces total throughput.
Some of the machine's resources may go unused during draining,
if some jobs finish before others.  
If jobs that cannot produce checkpoints are killed
because they run past the end of their retirement time during draining,
this also adds to the cost of draining.

To help gauge the costs of draining, the \Condor{startd} advertises
the accumulated time that was unused due to draining and the time
spent by jobs that were killed due to draining.  
These are advertised
respectively in the attributes \AdAttr{TotalMachineDrainingUnclaimedTime} and
\AdAttr{TotalMachineDrainingBadput}.  
The \Condor{defrag} daemon
averages these values across the pool and advertises the result in its
daemon ClassAd in the attributes \AdAttr{AvgDrainingBadput} and
\AdAttr{AvgDrainingUnclaimed}.  
Details of all attributes published by
the \Condor{defrag} daemon are described in
section~\ref{sec:Defrag-ClassAd-Attributes}.

The following command may be used to view the \Condor{defrag} daemon
ClassAd:

\begin{verbatim}
condor_status -l -any -constraint 'MyType == "Defrag"'
\end{verbatim}

\index{SMP machines!configuration|)}
