%%%%%%%%%%%%%%%%%%%%%%%%%%%%%%%%%%%%%%%%%%%%%%%%%%%%%%%%%%%%%%%%%%%%%%%%%%%
\subsection{\label{sec:tcp-collector-update}Using TCP to Send Updates to
the \Condor{collector}}
%%%%%%%%%%%%%%%%%%%%%%%%%%%%%%%%%%%%%%%%%%%%%%%%%%%%%%%%%%%%%%%%%%%%%%%%%%

\index{TCP}
\index{TCP!sending updates}
\index{UDP}
\index{UDP!lost datagrams}
\index{condor\_collector}

TCP sockets are reliable, connection-based sockets that guarantee
the delivery of any data sent.
However, TCP sockets are fairly expensive to establish, and there is more
network overhead involved in sending and receiving messages.

UDP sockets are datagrams, and are not reliable.
There is very little overhead in establishing or using a UDP socket,
but there is also no guarantee that the data will be delivered.
Typically, the lack of guaranteed delivery for UDP does not cause
problems for HTCondor.

HTCondor can be configured to use TCP
sockets to send updates to the \Condor{collector} instead of
UDP datagrams.
This feature is intended for sites where UDP updates are
lost because of the underlying network.
An example where this may happen is if the pool is comprised of
machines across a wide area network (WAN) where UDP packets are
observed to be frequently dropped.

To enable the use of TCP sockets, the following configuration
setting is used:

\begin{description}

\item[\Macro{UPDATE\_COLLECTOR\_WITH\_TCP}]
  When set to \Expr{True}, the HTCondor daemons to use TCP to
  update the \Condor{collector}, instead of the default UDP.
  Defaults to \Expr{False}.

\end{description}

When there are sufficient file descriptors, the \Condor{collector} leaves
established TCP sockets open, facilitating better performance.
Subsequent updates can reuse an already open socket.

Each HTCondor daemon will have 1 socket open to the \Condor{collector}.
So, in a pool with N machines, each of them running a \Condor{master},
\Condor{schedd}, and \Condor{startd}, the \Condor{collector} would
need at least 3*N file descriptors.  If the \Condor{collector} is also
acting as a CCB server, it will require an additional file descriptor for
each registered daemon.  In typical Unix installations,
the default number of file descriptors available to the \Condor{collector}
 is only 1024.
  This can be modified with a configuration setting such as the following:

\begin{verbatim}
COLLECTOR_MAX_FILE_DESCRIPTORS = 1600
\end{verbatim}

If there are not sufficient file descriptors for all of the daemons
sending updates to the \Condor{collector}, a warning will be printed in the
\Condor{collector} log file.  Look for the string
\Expr{file descriptor safety level exceeded}.

\Note At this time, \MacroNI{UPDATE\_COLLECTOR\_WITH\_TCP} only
affects the main \Condor{collector} for the site, not any sites that
a \Condor{schedd} might flock to.

