%%%%%%%%%%%%%%%%%%%%%%%%%%%%%%%%%%%%%%%%%%%%%%%%%%%%%%%%%%%%%%%%%%%%%%%%%%%
\subsection{\label{sec:tcp-collector-update}Using TCP to Send Updates to
the \Condor{collector}}
%%%%%%%%%%%%%%%%%%%%%%%%%%%%%%%%%%%%%%%%%%%%%%%%%%%%%%%%%%%%%%%%%%%%%%%%%%

\index{TCP}
\index{TCP!sending updates}
\index{UDP}
\index{UDP!lost datagrams}
\index{condor\_collector}

TCP sockets are reliable, connection-based sockets that guarantee
the delivery of any data sent.
However, TCP sockets are fairly expensive to establish, and there is more
network overhead involved in sending and receiving messages.

UDP sockets are datagrams, and are not reliable.
There is very little overhead in establishing or using a UDP socket,
but there is also no guarantee that the data will be delivered.
The lack of guaranteed delivery of UDP will negatively affect some pools,
particularly ones comprised of machines across a wide area network (WAN)
or highly-congested network links, 
where UDP packets are frequently dropped.

By default, HTCondor daemons will use TCP to send updates to the
\Condor{collector}, 
with the exception of the \Condor{collector} forwarding
updates to any \Condor{collector} daemons
specified in \MacroNI{CONDOR\_VIEW\_HOST},
where UDP is used.
These configuration variables control the protocol used:

\begin{description}

\item[\Macro{UPDATE\_COLLECTOR\_WITH\_TCP}]
  When set to \Expr{False}, the HTCondor daemons will use UDP to
  update the \Condor{collector}, instead of the default TCP.
  Defaults to \Expr{True}.

\item[\Macro{UPDATE\_VIEW\_COLLECTOR\_WITH\_TCP}]
  When set to \Expr{True}, the HTCondor collector will use TCP to
  forward updates to \Condor{collector} daemons specified 
  by \MacroNI{CONDOR\_VIEW\_HOST},
  instead of the default UDP.
  Defaults to \Expr{False}.

\item[\Macro{TCP\_UPDATE\_COLLECTORS}]
  A list of \Condor{collector} daemons which will be updated with TCP 
  instead of UDP,
  when \MacroNI{UPDATE\_COLLECTOR\_WITH\_TCP} or
  \MacroNI{UPDATE\_VIEW\_COLLECTOR\_WITH\_TCP} is set to \Expr{False}.

\end{description}

When there are sufficient file descriptors, the \Condor{collector} leaves
established TCP sockets open, facilitating better performance.
Subsequent updates can reuse an already open socket.

Each HTCondor daemon that sends updates to the \Condor{collector} will have
1 socket open to it.
So, in a pool with N machines, each of them running a \Condor{master},
\Condor{schedd}, and \Condor{startd}, the \Condor{collector} would
need at least 3*N file descriptors.  If the \Condor{collector} is also
acting as a CCB server, it will require an additional file descriptor for
each registered daemon.
In the default configuration,
the number of file descriptors available to the \Condor{collector}
is 10240.
For very large pools,  the number of descriptor can be modified with 
the configuration:

\begin{verbatim}
  COLLECTOR_MAX_FILE_DESCRIPTORS = 40960
\end{verbatim}

If there are insufficient file descriptors for all of the daemons
sending updates to the \Condor{collector}, 
a warning will be printed in the \Condor{collector} log file.  
The string
\AdStr{file descriptor safety level exceeded} identifies this warning.

