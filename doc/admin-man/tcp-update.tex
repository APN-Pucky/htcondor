%%%%%%%%%%%%%%%%%%%%%%%%%%%%%%%%%%%%%%%%%%%%%%%%%%%%%%%%%%%%%%%%%%%%%%%%%%%
\subsection{\label{sec:tcp-collector-update}
Using TCP to Send Updates to the \Condor{collector}}
%%%%%%%%%%%%%%%%%%%%%%%%%%%%%%%%%%%%%%%%%%%%%%%%%%%%%%%%%%%%%%%%%%%%%%%%%%

\index{TCP}
\index{TCP!sending updates}
\index{UDP}
\index{UDP!lost datagrams}
\index{condor\_collector}

TCP sockets are reliable, connection-based sockets that guarantee
the delivery of any data sent.
However, TCP sockets are fairly expensive to establish, and there is more
network overhead involved in sending and receiving messages.

UDP sockets are datagrams, and are not reliable.
There is very little overhead in establishing or using a UDP socket,
but there is also no guarantee that the data will be delivered.
All previous Condor versions used UDP sockets to send updates to
the \Condor{collector}, and this did not cause problems.

Beginning with version 6.5.0, Condor can be configured to use TCP
sockets to send updates to the \Condor{collector} instead of
UDP datagrams.
It is \emph{not} intended for most sites.
This feature is targeted at sites where UDP updates are
lost because of the underlying network.
Most Condor administrators that believe this is a good idea for
their site are wrong.
Do not enable this feature just because it sounds like a good idea.
The only cases where an administrator would want this feature are if
the ClassAd updates are consistently not getting to the
\Condor{collector}.
An example where this may happen is if the pool is comprised of
machines across a wide area network (WAN) where UDP packets are
frequently dropped.

Configuration variables are set to enable the use of TCP sockets.
There are two variables that an
administrator must define to enable this feature:

\begin{itemize}

\item[\Macro{UPDATE\_COLLECTOR\_WITH\_TCP}]
  When set to TRUE, the Condor daemons to use TCP to
  update the \Condor{collector}, instead of the default UDP.
  Defaults to FALSE.

\item[\Macro{COLLECTOR\_SOCKET\_CACHE\_SIZE}] 
  Specifies the number of TCP sockets cached at the \Condor{collector}.
  The default value for this setting is 0, with no cache enabled.

\end{itemize}

The use of a cache allows Condor to leave established TCP sockets open,
facilitating much better performance.
Subsequent updates can reuse an already open socket.
The work to establish a TCP connection may be lengthy,
including authentication and setting up encryption.
Therefore, Condor requires that
a socket cache be defined if TCP updates are to be used.
TCP updates will be refused by the \Condor{collector} daemon
if a cache is not enabled.

Each Condor daemon will have 1 socket open to the \Condor{collector}.
So, in a pool with N machines, each of them running a \Condor{master},
\Condor{schedd}, and \Condor{startd}, the \Condor{collector} would
need a socket cache that has at least 3*N entries.
Machines running Personal Condor in the pool need
an additional two entries (for the \Condor{master} and
\Condor{schedd}) for each Personal Condor installation.

Every cache entry utilizes a file descriptor within the
\Condor{collector} daemon.
Therefore, be careful not to define a cache that
is larger than the number of file descriptors the underlying operating
system allocates for a single process.

\Note At this time, \MacroNI{UPDATE\_COLLECTOR\_WITH\_TCP}, only
affects the main \Condor{collector} for the site, not any sites that
a \Condor{schedd} might flock to.

