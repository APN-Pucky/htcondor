%%%%%%%%%%%%%%%%%%%%%%%%%%%%%%%%%%%%%%%%%%%%%%%%%%%%%%%%%%%%%%%%%%%%%%%%%%%
\subsection{\label{sec:tcp-collector-update}
Using TCP to Send Collector Updates}
%%%%%%%%%%%%%%%%%%%%%%%%%%%%%%%%%%%%%%%%%%%%%%%%%%%%%%%%%%%%%%%%%%%%%%%%%%

\index{TCP}
\index{TCP!Updates}
\index{UDP}
\index{UDP!Updates being lost}
\index{collector}

TCP sockets are reliable, connection-based sockets that guarantee
delivery of any data sent.
However, they are fairly expensive to establish, and there is more
network overhead involved in sending and receiving messages.
UDP sockets are ``datagrams'', and are not reliable.
There is very little overhead in establishing or using a UDP socket,
but there is no guarantee that the data will be delivered.

In all previous versions, Condor used UDP sockets to send updates to
the \Condor{collector}, and there were never any problems.
Beginning with version 6.5.0, Condor can be configured to use TCP
connections to send its updates.
This feature was added for sites where the usual UDP updates were
getting lost because of the underlying network.
It is NOT intended for most sites.
Even if a Condor administrator thinks this would be a good idea for
their site, they are probably wrong.
They should not enable this feature just because it sounds like a good
idea.
The only cases where an administrator would want this feature are if
the ClassAd updates are consistently not getting to the
\Condor{collector}, for example, because the pool is comprised of
machines across a wide area network (WAN) where UDP packets are
frequently dropped.

There are two settings in the \File{condor\_config} file that an
administrator must define to enable this feature:

\begin{itemize}

\item[\Macro{UPDATE\_COLLECTOR\_WITH\_TCP}]
  When set to TRUE, this enables the Condor daemons to use TCP to
  update the \Condor{collector}, instead of the default UDP.
  Defaults to FALSE.

\item \Macro{COLLECTOR\_SOCKET\_CACHE\_SIZE}] 
  Enables a cache of TCP sockets at the \Condor{collector}.
  The default value for this setting is 0, with no cache enabled.

\end{itemize}

A cache provides much better performance, since once all the work is
done to establish a TCP connection (including authentication and
possibly setting up encryption), the socket between a given daemon and
the collector can remain open, and subsequent updates can reuse the
already open socket.
If a socket cache is not defined, TCP updates will be refused by
the collector.
Therefore, if an administrator enables
\MacroNI{UPDATE\_COLLECTOR\_WITH\_TCP}, they must also define
\MacroNI{COLLECTOR\_SOCKET\_CACHE\_SIZE}.

Each Condor daemon will have 1 socket open to the \Condor{collector}.
So, in a pool with N machines, each of them running a \Condor{master},
\Condor{schedd}, and \Condor{startd}, the \Condor{collector} would
need a socket cache that has at least 3*N entries.
If there are machines with ``personal Condor'' in the your pool, the
cache would need another 2 entries (for the \Condor{master} and
\Condor{schedd}) for each personal Condor installation.
Every entry in this cache uses another file descriptor in the
\Condor{collector}.
Therefore, administrators should be careful not to define a cache that
is larger than the number of file descriptors the underlying operating
system allocates for a single process.

\Note At this time, \MacroNI{UPDATE\_COLLECTOR\_WITH\_TCP}, only
effects the main \Condor{collector} for your site, not any sites that
a \Condor{schedd} might flock to.

