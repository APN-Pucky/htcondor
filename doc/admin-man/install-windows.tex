%%%%%%%%%%%%%%%%%%%%%%%%%%%%%%%%%%%%%%%%%%%%%%%%%%%%%%%%%%%%%%%%%%%%%%
\subsection{\label{sec:Windows-Install}Installation on Windows}
%%%%%%%%%%%%%%%%%%%%%%%%%%%%%%%%%%%%%%%%%%%%%%%%%%%%%%%%%%%%%%%%%%%%%%

\index{installation!Windows|(}
\index{Windows!installation|(}
This section contains the instructions for 
installing the Windows version of Condor.  
The install program will set up a slightly customized configuration
file that may be further customized after the installation has completed.

Please read the copyright and disclaimer information in 
section~\ref{sec:license} on
page~\pageref{sec:license} of the manual.
Installation and
use of Condor is acknowledgment that you have read and agree to the
terms.

Be sure that the Condor tools are of the same version
as the daemons installed.
The Condor executable for distribution is packaged in
a single file named similar to:
\begin{verbatim}
  condor-7.4.3-winnt50-x86.msi
\end{verbatim}
\index{Windows!installation!initial file size}
This file is approximately 80 Mbytes in size, and it may be
removed once Condor is fully installed.

Before installing Condor, please consider joining the condor-world mailing
list.  Traffic on this list is kept to an absolute minimum.  It is only
used to announce new releases of Condor.
To subscribe, follow the directions given at
\URL{http://www.cs.wisc.edu/condor/mail-lists/}.

For any installation, Condor services are installed and run as the
Local System account.
Running the Condor services as any other account (such as a domain user) 
is not supported and could be problematic.
 
\subsubsection{Installation Requirements}

\begin{itemize}

\item Condor for Windows requires Windows 2000 SP4, Windows XP SP2, 
or a more recent version.

\item 300 megabytes of free disk space is recommended.  Significantly more 
disk space could be desired to be able to run jobs with large data files.

\item Condor for Windows will operate on either an NTFS or FAT file system.  However, for security purposes, NTFS is preferred.

\item Condor for Windows requires the Visual C++ 2008 C runtime library.
\end{itemize}

%%%%%%%%%%%%%%%%%%%%%%%%%%%%%%%%%%%%%%%%%%%%%%%%%%%%%%%%%%%%%%%%%%%%%%
\subsubsection{\label{sec:NT-Preparing-to-Install}Preparing to Install
Condor under Windows } 
%%%%%%%%%%%%%%%%%%%%%%%%%%%%%%%%%%%%%%%%%%%%%%%%%%%%%%%%%%%%%%%%%%%%%%

\index{Windows!installation!preparation}
Before installing the Windows version of Condor,
there are two major
decisions to make about the basic layout of the pool.

\begin{enumerate}
\item What machine will be the central manager?
\item Is there enough disk space for Condor?
\end{enumerate}

If the answers to these questions are already known,
skip to the Windows Installation Procedure section below,
section~\ref{sec:nt-install-procedure} on
page~\pageref{sec:nt-install-procedure}.
If unsure, read on.

\begin{itemize} 

%%%%%%%%%%%%%%%%%%%%%%%%%%%%%%%%%%%%%%%%%%%%%%%%%%%%%%%%%%%%%%%%%%%%%%
\item{What machine will be the central manager?}
%%%%%%%%%%%%%%%%%%%%%%%%%%%%%%%%%%%%%%%%%%%%%%%%%%%%%%%%%%%%%%%%%%%%%%

One machine in your pool must be the central manager.
This is the
centralized information repository for the Condor pool and is also the
machine that matches available machines with waiting
jobs.  If the central manager machine crashes, any currently active
matches in the system will keep running, but no new matches will be
made.  Moreover, most Condor tools will stop working.  Because of the
importance of this machine for the proper functioning of Condor, we
recommend installing it on a machine that is likely to stay up all the
time, or at the very least, one that will be rebooted quickly if it
does crash.  Also, because all the services will send updates (by
default every 5 minutes) to this machine, it is advisable to consider
network traffic and network layout when choosing the central
manager.

Install Condor on the central manager before installing
on the other machines within the pool.

%%%%%%%%%%%%%%%%%%%%%%%%%%%%%%%%%%%%%%%%%%%%%%%%%%%%%%%%%%%%%%%%%%%%%%
\item{Is there enough disk space for Condor?}
%%%%%%%%%%%%%%%%%%%%%%%%%%%%%%%%%%%%%%%%%%%%%%%%%%%%%%%%%%%%%%%%%%%%%%

\index{Windows!installation!required disk space}
The Condor release directory takes up a fair amount of space.
The size requirement for the release
directory is approximately 250 Mbytes.
Condor itself, however, needs space to store all of the jobs and their
input files.  If there will be large numbers of jobs,
consider installing Condor on a volume with a large amount
of free space.

\end{itemize}


%%%%%%%%%%%%%%%%%%%%%%%%%%%%%%%%%%%%%%%%%%%%%%%%%%%%%%%%%%%%%%%%%%%%%%
\subsubsection{\label{sec:nt-install-procedure}
Installation Procedure Using the MSI Program}
%%%%%%%%%%%%%%%%%%%%%%%%%%%%%%%%%%%%%%%%%%%%%%%%%%%%%%%%%%%%%%%%%%%%%%

% condor MUST be run as local system
% 
%  root == administrator
%  to install, must be running with administrator privileges
%  the kernel runs as == local system

Installation of Condor must be done by a user with administrator privileges.
After installation, the Condor services will be run under 
the local system account.
When Condor is running a user job, however, 
it will run that user job with normal user permissions.

Download Condor, and start the installation process by running the installer. 
The Condor installation is completed by answering questions 
and choosing options within the following steps.

\begin{description}
\item[If Condor is already installed.]

     If Condor has been previously installed,
     a dialog box will appear before the installation of Condor proceeds.
     The question asks if you wish to preserve your current
     Condor configuration files.
     Answer yes or no, as appropriate.
	 
     If you answer yes, your configuration files will not be changed, 
     and you will proceed to the point where the new binaries will be installed.

     If you answer no, then there will be a second question
     that asks if you want to use answers
     given during the previous installation
     as default answers.

\item[STEP 1: License Agreement.]

     The first step in installing Condor
     is a welcome screen and license agreement.
     You are reminded that it is best to run the installation
     when no other Windows programs are running.
     If you need to close other Windows programs, it is safe to cancel the
     installation and close them.
     You are asked to agree to the license.
     Answer yes or no.  If you should disagree with the License, the
     installation will not continue.

     Also fill in name and company information,
     or use the defaults as given.

\item[STEP 2: Condor Pool Configuration.]

     The Condor configuration needs to be set based upon
     if this is a new pool or to join an existing one.
     Choose the appropriate radio button.

     For a new pool, enter a chosen name for the pool.
     To join an existing pool, 
     enter the host name of the central manager of the pool.

\item[STEP 3: This Machine's Roles.] 

     Each machine within a Condor pool may either
     submit jobs or execute submitted jobs, or both
     submit and execute jobs.
     A check box determines if this machine will be a submit point for
     the pool.

     A set of radio buttons determines the ability and configuration of
     the ability to execute jobs.
     There are four choices:
     \begin{description}
     \item[Do not run jobs on this machine.]
     This machine will not execute Condor jobs.
     \item[Always run jobs and never suspend them.]
     \item[Run jobs when the keyboard has been idle for 15 minutes.]
     \item[Run jobs when the keyboard has been idle for 15 minutes,
and the CPU is idle.]
     \end{description}

     For testing purposes, it is often helpful to use the always run Condor
     jobs option. 

     For a machine that is to execute jobs and the choice is one of
the last two in the list,
Condor needs to further know what to do with the currently running jobs.
There are two choices:
     \begin{description}
     \item[Keep the job in memory and continue when the machine meets
the condition chosen for when to run jobs.]
     \item[Restart the job on a different machine.]
     \end{description}

     This choice involves a trade off.
     Restarting the job on a different machine is less intrusive
     on the workstation owner than leaving the job in memory for a later time.
     A suspended job left in memory will require swap space,
     which could be a scarce resource.
     Leaving a job in memory, however, has the benefit that accumulated
     run time is not lost for a partially completed job.

\item[STEP 4: The Account Domain.]

% not really used right now.  "Things that suck about NT."
% UNIX has 2 domains:  file system domain and user-ID domain
% NT has only 1:  a combination, and so going back to letter
% drives, things get screwed up.

     Enter the machine's accounting (or UID) domain.
     On this version of Condor for Windows, this setting is only used for user
     priorities (see section~\ref{sec:UserPrio} on
     page~\pageref{sec:UserPrio}) and to form a default e-mail address for
     the user.

\item[STEP 5: E-mail Settings.]

     Various parts of Condor will send e-mail to a Condor administrator
     if something goes wrong and requires human attention.
     Specify the e-mail address and the SMTP relay host
     of this administrator.  Please pay close attention to this e-mail,
     since it will indicate problems in the Condor pool.

\item[STEP 6: Java Settings.]
     In order to run jobs in the \SubmitCmd{java} universe,
     Condor must have the path to the jvm executable on the machine.
     The installer will search for and list the jvm path, if it finds one.
     If not, enter the path.
     To disable use of the \SubmitCmd{java} universe,
     leave the field blank.

\item[STEP 7: Host Permission Settings.]
     Machines within the Condor pool will need various types of 
     access permission. 
     The three categories of permission are read, write, and administrator.
     Enter the machines or domain to be given access permissions,
     or use the defaults provided.
     Wild cards and macros are permitted.

     \begin{description}
     \item[Read]
     Read access allows a machine to obtain information about
     Condor such as the status of machines in the pool and the job queues.
     All machines in the pool should be given read access. 
     In addition, giving read access to *.cs.wisc.edu 
     will allow the Condor team to obtain information about
     the Condor pool, in the event that debugging is needed.
     \item[Write]
     All machines in the pool should be given write access. 
     It allows the machines you specify to send information to your
     local Condor daemons, for example, to start a Condor job.
     Note that for a machine to join the Condor pool, 
     it must have both read and write access to all of the machines in the pool.
     \item[Administrator]
     A machine with administrator access will be allowed more
     extended permission to do things such as
     change other user's priorities, modify the job queue,
     turn Condor services on and off, and restart Condor.
     The central manager should be given administrator access
     and is the default listed.
     This setting is granted to the entire machine, 
     so care should be taken not to make this too open.
     \end{description}

     For more details on these access permissions, and others that can be
     manually changed in your configuration file, please
     see the section titled Setting Up IP/Host-Based Security in Condor
     in section section~\ref{sec:Host-Security}
     on page~\pageref{sec:Host-Security}.

\item[STEP 8: VM Universe Setting.]
     A radio button determines whether 
     this machine will be configured to run \SubmitCmd{vm} universe jobs 
     utilizing VMware.
     In addition to having the VMware Server installed, Condor also needs
     \Prog{Perl} installed.
     The resources available for \SubmitCmd{vm} universe jobs can be tuned 
     with these settings, or the defaults listed may be used.

     \begin{description}
     \item[Version]
     Use the default value, as only one version is currently supported.
     \item[Maximum Memory]
     The maximum memory that each virtual machine is permitted to use 
     on the target machine.
     \item[Maximum Number of VMs]
     The number of virtual machines that can be run in parallel 
     on the target machine.
     \item[Networking Support]
     The VMware instances can be configured to use network support.
     There are four options in the pull-down menu.
          \begin{itemize}
          \item None: No networking support.
          \item NAT: Network address translation.
          \item Bridged: Bridged mode.
          \item NAT and Bridged: Allow both methods.
          \end{itemize}
     \item[Path to Perl Executable]
     The path to the \Prog{Perl} executable.
     \end{description}

\item[STEP 9: HDFS Settings.]
     A radio button enables support for 
     the Hadoop Distributed File System (HDFS).
     When enabled, a further radio button specifies 
     either name node or data node mode.

     Running HDFS requires Java to be installed,
     and Condor must know where the installation is.
     Running HDFS in data node mode also requires the installation of Cygwin,
     and the path to the Cygwin directory must be added to the 
     global \Env{PATH} environment variable.

     HDFS has several configuration options that must be filled in to be used.
     \begin{description}
     \item[Primary Name Node]
     The full host name of the primary name node.
     \item[Name Node Port]
     The port that the name node is listening on.
     \item[Name Node Web Port]
     The port the name node's web interface is bound to.
     It should be different from the name node's main port.
     \end{description}

\item[STEP 10: Choose Setup Type]

\index{Windows!installation!location of files}
The next step is where the destination of the Condor files will be
decided.
We recommend that Condor be installed in the location shown as the default 
in the install choice:
\verb@C:\Condor@. This is due to several hard coded
paths in scripts and configuration files.
Clicking on the Custom choice permits changing the installation directory.

Installation on the local disk is chosen for several reasons.
The Condor services run as local system, and within Microsoft Windows, 
local system has no network privileges.
Therefore, for Condor to operate, 
Condor should be installed on a local hard drive,
as opposed to a network drive (file server).

The second reason for installation on the local disk is that
the Windows usage of drive letters has implications for where
Condor is placed.
The drive letter used must be not change, even when different users are
logged in.
Local drive letters do not change under normal operation of Windows.

While it is strongly discouraged, 
it may be possible to place Condor on a hard drive that is not local,
if a dependency is added to the service control manager
such that Condor starts after the required file services
are available.

%  !! goes in C:/condor   (default)
%  !! advice is really should go on local hard drive,
%  as opposed to a network drive (also called file server)
%  Because,
%    1. Condor runs as local system, and accesses to a network
%      drive can't be authenticated  -- local system has
%      no network privileges.
%    2.  it is likely that you don't have this set up:
%    (and you need it to make it work)
%    you can add a dependency in the service control manager
%    that condor should start after the file services are
%    available
%    3. drive letters are "system-wide"
%    Must have dedicated letter (for all users), that remains
%    intact for all time, or condor won't know where
%    things are and can't get access (without its "letter")

\end{description}


%%%%%%%%%%%%%%%%%%%%%%%%%%%%%%%%%%%%%%%%%%%%%%%%%%%%%%%%%%%%%%%%%%%%%%
\subsubsection{\label{sec:nt-unattended-install-procedure}
Unattended Installation Procedure Using the Included Set Up Program}
%%%%%%%%%%%%%%%%%%%%%%%%%%%%%%%%%%%%%%%%%%%%%%%%%%%%%%%%%%%%%%%%%%%%%%

\index{Windows!installation!unattended install}
This section details how to run the Condor for Windows installer in an
unattended batch mode.
This mode is one that occurs completely from the command prompt,
without the GUI interface.

The Condor for Windows installer uses the Microsoft Installer (MSI)
technology, and it can be configured for unattended installs analogous
to any other ordinary MSI installer.

The following is a sample batch file that is used to set all the
properties necessary for an unattended install.

\begin{verbatim}
@echo on
set ARGS=
set ARGS=NEWPOOL="N"
set ARGS=%ARGS% POOLNAME=""
set ARGS=%ARGS% RUNJOBS="C"
set ARGS=%ARGS% VACATEJOBS="Y"
set ARGS=%ARGS% SUBMITJOBS="Y"
set ARGS=%ARGS% CONDOREMAIL="you@yours.com"
set ARGS=%ARGS% SMTPSERVER="smtp.localhost"
set ARGS=%ARGS% HOSTALLOWREAD="*"
set ARGS=%ARGS% HOSTALLOWWRITE="*"
set ARGS=%ARGS% HOSTALLOWADMINISTRATOR="$(IP_ADDRESS)"
set ARGS=%ARGS% INSTALLDIR="C:\Condor"
set ARGS=%ARGS% POOLHOSTNAME="$(IP_ADDRESS)"
set ARGS=%ARGS% ACCOUNTINGDOMAIN="none"
set ARGS=%ARGS% JVMLOCATION="C:\Windows\system32\java.exe"
set ARGS=%ARGS% USEVMUNIVERSE="N"set ARGS=%ARGS% VMMEMORY="128"
set ARGS=%ARGS% VMMAXNUMBER="$(NUM_CPUS)"
set ARGS=%ARGS% VMNETWORKING="N"
set ARGS=%ARGS% USEHDFS="N"
set ARGS=%ARGS% NAMENODE=""
set ARGS=%ARGS% HDFSMODE="HDFS_NAMENODE"
set ARGS=%ARGS% HDFSPORT="5000"
set ARGS=%ARGS% HDFSWEBPORT="4000"
 
msiexec /qb /l* condor-install-log.txt /i condor-7.1.0-winnt50-x86.msi %ARGS%
\end{verbatim}

Each property corresponds to answers that would have been
supplied while running an interactive installer.
The following is a brief explanation of each property
as it applies to unattended installations:

\begin{description}
\item [NEWPOOL = $<$ Y \Bar\ N $>$]
determines whether the installer will create a new pool with the target
machine as the central manager.

\item [POOLNAME]
sets the name of the pool, if a new pool is to be created. Possible values
are either the name or the empty string \verb@""@.

\item [RUNJOBS = $<$ N \Bar\ A \Bar\ I \Bar\ C $>$]
determines when Condor will run jobs. This can be set to:
\begin{itemize}
\item Never run jobs (N)
\item Always run jobs (A)
\item Only run jobs when the keyboard and mouse are Idle (I)
\item Only run jobs when the keyboard and mouse are idle and the CPU
usage is low (C)
\end{itemize}

\item [VACATEJOBS = $<$ Y \Bar\ N $>$]
determines what Condor should do when it has to stop the execution of
a user job. When set to Y, Condor will vacate the job and start
it somewhere else if possible. When set to N, Condor will merely
suspend the job in memory and wait for the machine to become available
again. 

\item[SUBMITJOBS  = $<$ Y \Bar\ N $>$]
will cause the installer to configure the machine as a submit
node when set to Y. 

\item[CONDOREMAIL]
sets the e-mail address of the Condor administrator. Possible values are
an e-mail address or the empty string \verb@""@.

\item[HOSTALLOWREAD]
is a list of host names that are allowed to issue READ commands to
Condor daemons. This value should be set in accordance with the
\Macro{HOSTALLOW\_READ} setting in the configuration file, as described in
section~\ref{sec:Host-Security} on page~\pageref{sec:Host-Security}.

\item[HOSTALLOWWRITE]
is a list of host names that are allowed to issue WRITE commands to
Condor daemons. This value should be set in accordance with the
\Macro{HOSTALLOW\_WRITE} setting in the configuration file, as described in
section~\ref{sec:Host-Security} on page~\pageref{sec:Host-Security}.

\item[HOSTALLOWADMINISTRATOR]
is a list of host names that are allowed to issue ADMINISTRATOR commands to
Condor daemons. This value should be set in accordance with the
\Macro{HOSTALLOW\_ADMINISTRATOR} setting in the configuration file, 
as described in
section~\ref{sec:Host-Security} on page~\pageref{sec:Host-Security}.

\item[INSTALLDIR]
defines the path to the directory where Condor will be installed. 

\item[POOLHOSTNAME]
defines the host name of the pool's central manager. 

\item[ACCOUNTINGDOMAIN] 
defines the accounting (or UID) domain the target machine will be in.

\item[JVMLOCATION]
defines the path to Java virtual machine on the target machine.

\item[SMTPSERVER]
defines the host name of the SMTP server that the target machine is to
use to send e-mail.

\item [VMMEMORY]
an integer value that defines the maximum memory each VM run on the target
machine.

\item [VMMAXNUMBER]
an integer value that defines the number of VMs that can be run in parallel
on the target machine.

\item [VMNETWORKING = $<$ N \Bar\ A \Bar\ B \Bar\ C $>$]
determines if VM Universe can use networking. This can be set to:
\begin{itemize}
\item None (N)
\item NAT (A)
\item Bridged (B)
\item NAT and Bridged (C)
\end{itemize}

\item [USEVMUNIVERSE = $<$ Y \Bar\ N $>$]
will cause the installer to enable VM Universe jobs on the target machine.


\item[PERLLOCATION]
defines the path to \Prog{Perl} on the target machine. This is required in
order to use the \SubmitCmd{vm} universe.

\item[USEHDFS = $<$ Y \Bar\ N $>$]
determines if HDFS is run.

\item[HDFSMODE $<$ HDFS\_DATANODE \Bar\ HDFS\_NAMENODE $>$]
sets the mode HDFS runs in.

\item[NAMENODE]
sets the host name of the primary name node.

\item[HDFSPORT]
sets the port number that the primary name node listens to.

\item [HDFSWEBPORT]
sets the port number that the name node web interface is bound to.
\end {description}

After defining each of these properties for the MSI installer, the
installer can be started with the \Prog{msiexec} command. The following
command starts the installer in unattended mode, and it dumps a journal of
the installer's progress to a log file:
\footnotesize
\begin{verbatim}
msiexec /qb /lxv* condor-install-log.txt /i condor-7.2.2-winnt50-x86.msi [property=value] ... 
\end{verbatim}
\normalsize

More information on the features of \Prog{msiexec}
can be found at Microsoft's website at
\URL{http://www.microsoft.com/resources/documentation/windows/xp/all/proddocs/en-us/msiexec.mspx}.

%%%%%%%%%%%%%%%%%%%%%%%%%%%%%%%%%%%%%%%%%%%%%%%%%%%%%%%%%%%%%%%%%%%%%%
\subsubsection{\label{sec:NT-Manual-Install}Manual Installation Condor on Windows}
%%%%%%%%%%%%%%%%%%%%%%%%%%%%%%%%%%%%%%%%%%%%%%%%%%%%%%%%%%%%%%%%%%%%%%

\index{Windows!manual install}
If you are to install Condor on many different machines, you may wish
to use some other mechanism to install Condor on additional machines
rather than running the Setup program described above on each machine.

\Warn This is for advanced users only!  All others should use the Setup program described above. 

Here is a brief overview of how to install Condor manually without using the provided GUI-based setup program:

\begin{description}
\item [The Service]
The service that Condor will install is called "Condor".  The Startup
Type is Automatic.  The service should log on as System Account, but
\Bold{do not enable} "Allow Service to Interact with Desktop".  The
program that is run is \Condor{master.exe}.

The Condor service can be installed and removed using the
\File{sc.exe} tool, which is included in Windows XP and Windows 2003
Server. The tool is also available as part of the Windows 2000
Resource Kit.

Installation can be done as follows:
\begin{verbatim}
sc create Condor binpath= c:\condor\bin\condor_master.exe
\end{verbatim}

To remove the service, use:
\begin{verbatim}
sc delete Condor
\end{verbatim}

\item [The Registry]
Condor uses a few registry entries in its operation.  The key that Condor
uses is HKEY\_LOCAL\_MACHINE/Software/Condor.  The values that Condor puts
in this registry key serve two purposes.
\begin{enumerate}
\item The values of CONDOR\_CONFIG and RELEASE\_DIR are used for Condor
to start its service.

CONDOR\_CONFIG should point to the \File{condor\_config} file.  In this version
of Condor, it \Bold{must} reside on the local disk.

RELEASE\_DIR should point to the directory where Condor is installed.  This
is typically
\verb@C:\Condor@, and again, this \Bold{must} reside on the
local disk.

\item The other purpose is storing the entries from the last installation
so that they can be used for the next one.
\end{enumerate}

\item [The File System]
The files that are needed for Condor to operate are identical to the Unix
version of Condor, except that executable files end in \File{.exe}.  For
example the on Unix one of the files is \File{condor\_master} and on Condor
the corresponding file is \File{condor\_master.exe}.

These files currently must reside on the local disk for a variety of reasons.
Advanced Windows users might be able to put the files on remote resources.
The main concern is twofold.  First, the files must be there when the service
is started.  Second, the files must always be in the same spot (including
drive letter), no matter who is logged into the machine.  

Note also that when installing manually, you will need to create the
directories that Condor will expect to be present given your
configuration. This normally is simply a matter of creating the
\File{log}, \File{spool}, and \File{execute} directories.

\end{description}


%%%%%%%%%%%%%%%%%%%%%%%%%%%%%%%%%%%%%%%%%%%%%%%%%%%%%%%%%%%%%%%%%%%%%%
\subsubsection{\label{nt-installed-now-what}
Starting Condor Under Windows After Installation}
%%%%%%%%%%%%%%%%%%%%%%%%%%%%%%%%%%%%%%%%%%%%%%%%%%%%%%%%%%%%%%%%%%%%%%
\index{Windows!starting the Condor service}
\index{starting Condor!Windows platforms}

After the installation of Condor is completed, the Condor service
must be started.  If you used the GUI-based setup program to install
Condor, the Condor service should already be started.  If you installed
manually, Condor must
be started by hand, or you can simply reboot. \Note The Condor service
will start automatically whenever you reboot your machine.

To start Condor by hand:
\begin{enumerate}
\item From the Start menu, choose Settings.
\item From the Settings menu, choose Control Panel.
\item From the Control Panel, choose Services.
\item From Services, choose Condor, and Start.
\end{enumerate}

Or, alternatively you can enter the following command from a command prompt:
\begin{verbatim}
         net start condor
\end{verbatim}

\index{Windows!Condor daemon names}
Run the Task Manager (Control-Shift-Escape) to check that Condor
services are running.  The following tasks should
be running:  
\begin{itemize}
\item \Condor{master.exe}
\item \Condor{negotiator.exe}, if this machine is a central manager.
\item \Condor{collector.exe}, if this machine is a central manager.
\item \Condor{startd.exe}, if you indicated that this Condor node should start jobs
\item \Condor{schedd.exe}, if you indicated that this Condor node should submit jobs
to the Condor pool.
\end{itemize}

Also, you should now be able to open up a new cmd (DOS prompt) window, and
the Condor bin directory should be in your path, so you can issue the normal
Condor commands, such as \Condor{q} and \Condor{status}.

\index{installation!Windows|)}
\index{Windows!installation|)}

%%%%%%%%%%%%%%%%%%%%%%%%%%%%%%%%%%%%%%%%%%%%%%%%%%%%%%%%%%%%%%%%%%%%%%
\subsubsection{\label{nt-running-now-what}
Condor is Running Under Windows ... Now What?}
%%%%%%%%%%%%%%%%%%%%%%%%%%%%%%%%%%%%%%%%%%%%%%%%%%%%%%%%%%%%%%%%%%%%%%

Once Condor services are running, try submitting test jobs.
Example 2 within section~\ref{sec:sample-submit-files} 
on page~\pageref{sec:sample-submit-files} presents a vanilla
universe job.
