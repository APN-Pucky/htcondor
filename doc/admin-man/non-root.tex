%%%%%%%%%%%%%%%%%%%%%%%%%%%%%%%%%%%%%%%%%%%%%%%%%%%%%%%%%%%%%%%%%%%%%%
\subsection{\label{sec:Non-Root}Running Condor as Non-Root}
%%%%%%%%%%%%%%%%%%%%%%%%%%%%%%%%%%%%%%%%%%%%%%%%%%%%%%%%%%%%%%%%%%%%%%

While we strongly recommend starting up the Condor daemons as root, we
understand that that's not always possible.  The main problems this
causes are if you've got one Condor installation shared by many users
on a single machine, or if you're setting up your machines to execute
Condor jobs.  If you're just setting up a submit-only installation for
a single user, there's no need for (or benefit from) running as
root.  

What follows are the details of what effect running without root
access has on the various parts of Condor:

\begin{description}

\item[\Condor{startd}] If you're setting up a machine to run Condor
   jobs and don't start the \Condor{startd} as root, you're basically
   relying on the goodwill of your Condor users to agree to the policy
   you configure the startd to enforce as far as starting, suspending,
   vacating and killing Condor jobs under certain conditions.  If you
   run as root, however, you can enforce these policies regardless of
   malicious users.  By running as root, the Condor daemons run with a
   different UID than the Condor job that gets started (since the
   user's job is started as either the UID of the user who submitted
   it, or as user ``nobody'', depending on the \Macro{UID\_DOMAIN}
   settings).  Therefore, the Condor job cannot do anything to the
   Condor daemons.  If you don't start the daemons as root, all
   processes started by Condor, including the end user's job, run with
   the same UID (since you can't switch UIDs unless you're root).
   Therefore, a user's job could just kill the \Condor{startd} and
   \Condor{starter} as soon as it starts up and by doing so, avoid
   getting suspended or vacated when a user comes back to the machine.
   This is nice for the user, since they get unlimited access to the
   machine, but awful for the machine owner or administrator.  If you
   trust the users submitting jobs to Condor, this might not be a
   concern.  However, to ensure that the policy you choose is
   effectively enforced by Condor, the \Condor{startd} should be
   started as root.

   In addition, some system information cannot be obtained without
   root access on some platforms (such as load average on IRIX).  As a
   result, when we're running without root access, the startd has to
   call other programs (for example, ``uptime'') to get this
   information.  This is much less efficient than getting the
   information directly from the kernel (which is what we do if we're
   running as root).  On Linux and Solaris, we can get this
   information directly without root access, so this is not a concern
   on those platforms.

   If you can't have all of Condor running as root, at least consider
   whether you can install the Condor{startd} as setuid root.  That
   would solve both of these problems.  If you can't do that, you
   could also install it as a setgid sys or kmem program (depending on
   whatever group has read access to \File{/dev/kmem} on your system)
   and that would at least solve the system information problem.

\item[\Condor{schedd}] The biggest problem running the schedd
    without root access is that the \Condor{shadow} processes which it
    spawns are stuck with the same UID the \Condor{schedd} has.  This
    means that users submitting their jobs have to go out of their way
    to grant write access to user or group condor (or whoever the
    schedd is running as) for any files or directories their jobs
    write or create.  Similarly, read access must be granted to their
    input files.

    You might consider installing \Condor{submit} as a setgid condor
    program so that at least the \File{stdout}, \File{stderr} and
    \File{UserLog} files get created with the right permissions.  If
    \Condor{submit} is a setgid program, it will automatically set
    it's umask to 002, so that creates group-writable files.  This
    way, the simple case of a job that just writes to \File{stdout}
    and \File{stderr} will work.  If users have programs that open
    their own files, they'll have to know to set the right permissions
    on the directories they submit from.

\item[\Condor{master}] The \Condor{master} is what spawns the
    \Condor{startd} and \Condor{schedd}, so if want them both running
    as root, you should have the master run as root.  This happens
    automatically if you start the master from your boot scripts.

\item[\Condor{negotiator}]
\item[\Condor{collector}] There is no need to have either of these
daemons running as root.

\item[\Condor{kbdd}] On platforms that need the \Condor{kbdd} (Digital
    Unix and IRIX) the \Condor{kbdd} has to run as root.  If it is
    started as any other user, it will not work.  You might consider
    installing this program as a setuid root binary if you can't run
    the \Condor{master} as root.  Without the \Condor{kbdd}, the
    startd has no way to monitor mouse activity at all, and the only
    keyboard activity it will notice is activity on ttys (such as
    xterms, remote logins, etc).

\end{description}