%%%%%%%%%%%%%%%%%%%%%%%%%%%%%%%%%%%%%%%%%%%%%%%%%%%%%%%%%%%%%%%%%%%%%%
\subsection{\label{sec:Install-MPI-Condor}
Installing MPI Support in Condor} 
%%%%%%%%%%%%%%%%%%%%%%%%%%%%%%%%%%%%%%%%%%%%%%%%%%%%%%%%%%%%%%%%%%%%%%

\index{contrib module!MPI}
\index{MPI contrib module}
\index{intallation!MPI contrib module}

For complete documentation on using MPI in Condor, see the section
entitled ``Running MPICH jobs in Condor'' in the version 6.1 manual.
This manual can be found at
\Url{http://www.cs.wisc.edu/condor/manual/v6.1}.

To install the MPI contrib module, all you have to do is download
to appropriate binary module for whatever platform(s) you plan to use
for MPI jobs in Condor.
Once you have downloaded each module, uncompressed and untarred it, you
will be left with a directory that contains a \File{mpi.tar},
\File{README} and so on.
The \File{mpi.tar} acts much like the \File{release.tar} file for a
main release. 
It contains all the binaries and supporting files you would install in
your release directory:
\begin{verbatim}
        sbin/condor_shadow.v61
        sbin/condor_starter.v61
        sbin/rsh
\end{verbatim}

Since these files do not exist in a main release, you can safely untar
the \File{mpi.tar} directly into your release directory, and you're
done installing the MPI contrib module.
Again, see the 6.1 manual for instructions on how to use MPI in
Condor.

