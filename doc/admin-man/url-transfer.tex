%%%%%%%%%%%%%%%%%%%%%%%%%%%%%%%%%%%%%%%%%%%%%%%%%%%%%%%%%%%%%%%%%%%%%%
\subsection{\label{sec:URL-transfer}
Enabling the Transfer of Files Specified by a URL}
%%%%%%%%%%%%%%%%%%%%%%%%%%%%%%%%%%%%%%%%%%%%%%%%%%%%%%%%%%%%%%%%%%%%%%
\index{file transfer mechanism!input file specified by URL}
\index{URL file transfer}

A vanilla universe's job input files may be specified by a URL,
causing the execute machine to retrieve the files.
This differs from the normal file transfer mechanism,
in which transfers are from the machine where the job is submitted
to the machine where the job is executed.

The transfers are accomplished a \Term{plug-in},
an executable or shell script that handles the task of file transfer.
More than one plug-in may be specified.
The plug-ins must be installed and available on every execute machine 
that may run a job specifying an input file with a URL.

URL transfers are enabled by default in the configuration 
of execute machines.
Disabling URL transfers is accomplished by setting
\footnotesize
\begin{verbatim}
ENABLE_URL_TRANSFERS = FALSE
\end{verbatim}
\normalsize

In addition, a comma separated list giving the absolute path and name
of all available plug-ins is specified as in the example:
\footnotesize
\begin{verbatim}
FILETRANSFER_PLUGINS = /opt/condor/plugins/wget-plugin, \
                       /opt/condor/plugins/hdfs-plugin, \
                       /opt/condor/plugins/custom-plugin
\end{verbatim}
\normalsize

The \Condor{starter} invokes all listed plug-ins to determine their 
capabilities. Each may handle one or more protocols (scheme names).
The plug-in's response to invocation identifies which protocols
it can handle.
When a URL transfer is specified by a job,
the \Condor{starter} invokes the proper one to do the transfer.
If more than one plugin is capable of handling a particular protocol,
then the last one within the list given by \MacroNI{FILETRANSFER\_PLUGINS}
is used.

Condor assumes that all plug-ins will respond in specific
ways.
To determine the capabilities of the plug-ins as to which protocols
they handle,
the \Condor{starter} daemon invokes each plug-in giving it the
command line argument \Opt{-classad}.
In response to invocation with this command line argument,
the plug-in must respond with an output of three ClassAd attributes. 
The first two are fixed:
\footnotesize
\begin{verbatim}
PluginVersion = "0.1"
PluginType = "FileTransfer"
\end{verbatim}
\normalsize

The third ClassAd attribute is \Attr{SupportedMethods}. 
This attribute is a string containing a comma separated list of the
protocols that the plug-in handles.
So, for example:
\footnotesize
\begin{verbatim}
SupportedMethods = "http,ftp,file"
\end{verbatim}
\normalsize
would identify that the three protocols described by \verb@http@,
\verb@ftp@, and \verb@ftp@ are supported.
These strings will match the protocol specification as given
within a URL in a \SubmitCmd{transfer\_input\_files} command 
in a submit description file for a job.

When a job specifies a URL transfer,
the plug-in is invoked, without the command line argument \Opt{-classad}.
It will instead be given two other command line arguments.
The first will be the URL of the file to retrieve.
The second will be the absolute path identifying where to place the
transferred file.
The plug-in is expected to do the transfer,
exiting with status 0 if the transfer was successful, 
and a non-zero status if the transfer was \emph{not} successful.

Note that this functionality is not limited to a predefined set
of protocols.
New ones can be invented.
As an invented example,
the \verb@zkm@ transfer type writes random bytes to a file.
The plug-in that handles \verb@zkm@ transfers would respond to 
invocation with the \Opt{-classad} command line argument with:
\footnotesize
\begin{verbatim}
PluginVersion = "0.1"
PluginType = "FileTransfer"
SupportedMethods = "zkm"
\end{verbatim}
\normalsize
And, then when a job requested that this plug-in be invoked,
for the invented example:
\footnotesize
\begin{verbatim}
transfer_intput_files = zkm://128/r-data
\end{verbatim}
\normalsize
the plug-in will be invoked with a first command line argument
of \verb@zkm://128/r-data@ and a second command line argument giving
the full path along with the file name \File{r-data} as the location
for the plug-in to write 128 bytes of random data.
