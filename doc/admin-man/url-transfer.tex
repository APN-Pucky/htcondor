%%%%%%%%%%%%%%%%%%%%%%%%%%%%%%%%%%%%%%%%%%%%%%%%%%%%%%%%%%%%%%%%%%%%%%
\subsection{\label{sec:URL-transfer}
Enabling the Transfer of Files Specified by a URL}
%%%%%%%%%%%%%%%%%%%%%%%%%%%%%%%%%%%%%%%%%%%%%%%%%%%%%%%%%%%%%%%%%%%%%%
\index{file transfer mechanism!input file specified by URL}
\index{file transfer mechanism!output file(s) specified by URL}
\index{URL file transfer}

Because staging data on the submit machine is not always efficient,
HTCondor permits input files to be transferred
from a location specified by a URL;
likewise, output files may be transferred
to a location specified by a URL.
All transfers (both input and output) are accomplished by invoking 
a \Term{plug-in},
an executable or shell script that handles the task of file transfer.

For transferring input files,
URL specification is limited to jobs running under the vanilla universe 
and to a vm universe VM image file.
The execute machine retrieves the files.
This differs from the normal file transfer mechanism,
in which transfers are from the machine where the job is submitted
to the machine where the job is executed.
Each file to be transferred by specifying a URL, causing a
plug-in to be invoked, is specified separately in the job submit
description file with the command \SubmitCmd{transfer\_input\_files};
see section~\ref{sec:file-transfer-by-URL} for details.

For transferring output files,
either the entire output sandbox, which are all files produced or
modified by the job as it executes, or a subset of these files,
as specified by the submit description file command 
\SubmitCmd{transfer\_output\_files} are transferred to the
directory specified by the URL.
The URL itself is specified in the separate submit description file command 
\SubmitCmd{output\_destination};
see section~\ref{sec:file-transfer-by-URL} for details.
The plug-in is invoked once for each output file to be transferred.

Configuration identifies the availability of the one or more plug-in(s).
The plug-ins must be installed and available on every execute machine 
that may run a job which might specify a URL, either for input or for output.

URL transfers are enabled by default in the configuration 
of execute machines.
Disabling URL transfers is accomplished by setting
\footnotesize
\begin{verbatim}
ENABLE_URL_TRANSFERS = FALSE
\end{verbatim}
\normalsize

A comma separated list giving the absolute path and name
of all available plug-ins is specified as in the example:
\footnotesize
\begin{verbatim}
FILETRANSFER_PLUGINS = /opt/condor/plugins/wget-plugin, \
                       /opt/condor/plugins/hdfs-plugin, \
                       /opt/condor/plugins/custom-plugin
\end{verbatim}
\normalsize

The \Condor{starter} invokes all listed plug-ins to determine their 
capabilities. Each may handle one or more protocols (scheme names).
The plug-in's response to invocation identifies which protocols
it can handle.
When a URL transfer is specified by a job,
the \Condor{starter} invokes the proper one to do the transfer.
If more than one plugin is capable of handling a particular protocol,
then the last one within the list given by \MacroNI{FILETRANSFER\_PLUGINS}
is used.

HTCondor assumes that all plug-ins will respond in specific
ways.
To determine the capabilities of the plug-ins as to which protocols
they handle,
the \Condor{starter} daemon invokes each plug-in giving it the
command line argument \Opt{-classad}.
In response to invocation with this command line argument,
the plug-in must respond with an output of three ClassAd attributes. 
The first two are fixed:
\footnotesize
\begin{verbatim}
PluginVersion = "0.1"
PluginType = "FileTransfer"
\end{verbatim}
\normalsize

The third ClassAd attribute is \Attr{SupportedMethods}. 
This attribute is a string containing a comma separated list of the
protocols that the plug-in handles.
So, for example
\footnotesize
\begin{verbatim}
SupportedMethods = "http,ftp,file"
\end{verbatim}
\normalsize
would identify that the three protocols described by \verb@http@,
\verb@ftp@, and \verb@file@ are supported.
These strings will match the protocol specification as given
within a URL in a \SubmitCmd{transfer\_input\_files} command
or within a URL in an \SubmitCmd{output\_destination} command 
in a submit description file for a job.

When a job specifies a URL transfer,
the plug-in is invoked, without the command line argument \Opt{-classad}.
It will instead be given two other command line arguments.
For the transfer of input file(s),
the first will be the URL of the file to retrieve
and the second will be the absolute path identifying where to place the
transferred file.
For the transfer of output file(s),
the first will be the absolute path on the local machine of the file
to transfer,
and the second will be the URL of the directory and file name
at the destination.

The plug-in is expected to do the transfer,
exiting with status 0 if the transfer was successful, 
and a non-zero status if the transfer was \emph{not} successful.
When \emph{not} successful, the job is placed on hold,
and the job ClassAd attribute \Attr{HoldReason} will be set as
appropriate for the job.
The job ClassAd attribute \Attr{HoldReasonSubCode} will be set to
the exit status of the plug-in.

As an example of the transfer of a subset of output files,
assume that the submit description file contains 
\footnotesize
\begin{verbatim}
output_destination = url://server/some/directory/
transfer_output_files = foo, bar, qux
\end{verbatim}
\normalsize
HTCondor invokes the plug-in that handles the \Expr{url} protocol
three times.
The directory delimiter
(\verb@/@ on Unix, and \verb@\@ on Windows) 
is appended to the destination URL,
such that the three (Unix) invocations of the plug-in will appear similar to
\footnotesize
\begin{verbatim}
url_plugin /path/to/local/copy/of/foo url://server/some/directory//foo
url_plugin /path/to/local/copy/of/bar url://server/some/directory//bar
url_plugin /path/to/local/copy/of/qux url://server/some/directory//qux
\end{verbatim}
\normalsize

Note that this functionality is not limited to a predefined set
of protocols.
New ones can be invented.
As an invented example,
the \verb@zkm@ transfer type writes random bytes to a file.
The plug-in that handles \verb@zkm@ transfers would respond to 
invocation with the \Opt{-classad} command line argument with:
\footnotesize
\begin{verbatim}
PluginVersion = "0.1"
PluginType = "FileTransfer"
SupportedMethods = "zkm"
\end{verbatim}
\normalsize
And, then when a job requested that this plug-in be invoked,
for the invented example:
\footnotesize
\begin{verbatim}
transfer_input_files = zkm://128/r-data
\end{verbatim}
\normalsize
the plug-in will be invoked with a first command line argument
of \verb@zkm://128/r-data@ and a second command line argument giving
the full path along with the file name \File{r-data} as the location
for the plug-in to write 128 bytes of random data.

The transfer of output files in this manner was introduced 
in HTCondor version 7.6.0. 
Incompatibility and inability to function will result if the executables
for the \Condor{starter} and \Condor{shadow} are versions earlier
than HTCondor version 7.6.0.
Here is the expected behavior for these cases that 
cannot be backward compatible. 
\begin{itemize}
\item
If the \Condor{starter} version is earlier than 7.6.0,
then regardless of the \Condor{shadow} version,
transfer of output files, as identified in the submit description
file with the command \SubmitCmd{output\_destination} is ignored.
The files are transferred back to the submit machine.
\item
If the \Condor{starter} version is 7.6.0 or later,
but the  \Condor{shadow} version is earlier than 7.6.0,
then the \Condor{starter} will attempt to send the command to the
\Condor{shadow}, but the \Condor{shadow} will ignore the command.
No files will be transferred, and the job will be placed on hold.
\end{itemize}
