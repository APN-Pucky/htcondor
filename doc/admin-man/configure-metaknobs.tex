%TEMPTEMP -- say that part of why we have meta-knobs is because a lot of things require multiple knobs to be set in coordination with each other
%%%%%%%%%%%%%%%%%%%%%%%%%%%%%%%%%%%%%%%%%%%%%%%%%%%%%%%%%%%%%%%%%%%%%%
\section{\label{sec:Configuring-HTCondor-Templates}Configuration Templates}
%%%%%%%%%%%%%%%%%%%%%%%%%%%%%%%%%%%%%%%%%%%%%%%%%%%%%%%%%%%%%%%%%%%%%%

\index{HTCondor!configuration-templates}
\index{configuration: templates}

TEMPTEMP -- often need to set a number of knobs in concert with each
other; templates do that automatically

%%%%%%%%%%%%%%%%%%%%%%%%%%%%%%%%%%%%%%%%%%%%%%%%%%%%%%%%%%%%%%%%%%%%%%
\subsection{\label{sec:Config-Metaknobs}Configuration Templates: Using
Predefined Sets of Configuration}
%%%%%%%%%%%%%%%%%%%%%%%%%%%%%%%%%%%%%%%%%%%%%%%%%%%%%%%%%%%%%%%%%%%%%%
\index{configuration!USE syntax}
\index{USE configuration syntax}
Predefined sets of configuration may be identified and incorporated
into the configuration using the syntax
\begin{verbatim}
  use <category name> : <template name> 
\end{verbatim}

The \MacroNI{use} key word is case insensitive.
There are \emph{no} requirements for white space characters surrounding
the colon character.
More than one \MacroNI{<template name>} identifier may be placed within
a single \MacroNI{use} line. 
Separate the names by a space character. 
There is no mechanism by which the administrator may define their
own custom \MacroNI{<category name>} or \MacroNI{<template name>}.

Each predefined \MacroNI{<category name>} has a fixed, case insensitive
name for the sets of configuration that are predefined. 
Placement of a \MacroNI{use} line in the configuration brings in
the predefined configuration it identifies.

There are four \MacroNI{<category name>} values.
Within a category, a predefined, case insensitive name identifies
the set of configuration it incorporates. 
\begin{description}

\label{usecategory:ROLE}
\item[\MacroNI{ROLE}]
  Describes configuration for the various roles that a machine might
  play within an HTCondor pool. The configuration will identify which
  daemons are running on a machine.
  \begin{itemize}
    \item \texttt{Personal}
    Settings needed for when a single machine is the entire pool.
    \item \texttt{Submit}
    Settings needed to allow this machine to submit jobs to the pool.
    May be combined with \texttt{Execute} and \texttt{CentralManager} roles.
    \item \texttt{Execute}
    Settings needed to allow this machine to execute jobs.
    May be combined with \texttt{Submit} and \texttt{CentralManager} roles.
    \item \texttt{CentralManager}
    Settings needed to allow this machine to act as the central manager
    for the pool.
    May be combined with \texttt{Submit} and \texttt{Execute} roles.
  \end{itemize}

\label{usecategory:FEATURE}
\item[\MacroNI{FEATURE}]
  Describes configuration for implemented features. 
  \begin{itemize}
    \item \texttt{Remote\_Runtime\_Config}
    Enables the use of \Condor{config\_val} \Opt{-rset} to the machine with
    this configuration.
    Note that there are security implications for use of this configuration,
    as it potentially permits the arbitrary modification of configuration.
    Variable \Macro{SETTABLE\_ATTRS\_CONFIG} must also be defined.
    \item \texttt{Remote\_Config}
    Enables the use of \Condor{config\_val} \Opt{-set} to the machine with
    this configuration.
    Note that there are security implications for use of this configuration,
    as it potentially permits the arbitrary modification of configuration.
    Variable \Macro{SETTABLE\_ATTRS\_CONFIG} must also be defined.
    \item \texttt{VMware}
    Enables use of the vm universe with VMware virtual machines.
    Note that this feature depends on Perl. 
    \item \texttt{GPUs}
    Sets configuration based on detection with the \Condor{gpu\_discovery}
    tool, and defines a custom resource using the name \Expr{GPUs}.
    Supports both OpenCL and CUDA if detected. 
  \end{itemize}

\label{usecategory:POLICY}
\item[\MacroNI{POLICY}]
  Describes configuration for the circumstances under which
  machines choose to run jobs.
  \begin{itemize}
    \item \texttt{Always\_Run\_Jobs}
    Always start jobs and run them to completion, without consideration of
    \Condor{negotiator} generated preemption or suspension.
    This is the default policy, and it is intended to be used with dedicated
    resources.
    If this policy is used together with the \texttt{Limit\_Job\_Runtimes}
    policy,
    order the specification by placing this \texttt{Always\_Run\_Jobs} 
    policy first. 
    \item \texttt{UWCS\_Desktop}
    This was the default policy before HTCondor version 8.1.6.
    It is intended to be used with desktop machines not exclusively running
    HTCondor jobs.
    It injects \Expr{UWCS} into the name of some configuration variables.
    \item \texttt{Desktop}
    An updated and reimplementation of the \texttt{UWCS\_Desktop} policy,
    but \emph{without} the \Expr{UWCS} naming of some configuration variables.
    \item \texttt{Limit\_Job\_Runtimes}
    Limits running jobs to a maximum of 24 hours using preemption.
    To set the limit to a different amount of time, 
    define configuration variable \MacroNI{MAX\_JOB\_RUNTIME}
    with the desired limit in seconds;
    place this definition of \MacroNI{MAX\_JOB\_RUNTIME} \emph{after} the
    \Expr{use POLICY : Limit\_Job\_Runtimes} line within the configuration
    file.  
    If this policy is used together with the \texttt{Always\_Run\_Jobs} policy,
    order the specification by placing this \texttt{Limit\_Job\_Runtimes} 
    policy second. 
	\item \texttt{Preempt\_If\_Cpus\_Exceeded}
	If the startd observes the number of CPU cores used by the job exceed
	the number of cores in the slot by more than 0.8 on average over the past
	minute, preempt the job immediately
	ignoring any job retirement time.
	\item \texttt{Hold\_If\_Cpus\_Exceeded}
	If the startd observes the number of CPU cores used by the job exceed
	the number of cores in the slot by more than 0.8 on average over the past
	minute, immediately place the job on hold
	ignoring any job retirement time.  The job will go on hold with a reasonable
	hold reason in job attribute \Attr{HoldReason} and a value of 101 in job
	attribute \Attr{HoldReasonCode}.  The hold reason and code can be customized by
	specifying \MacroNI{HOLD\_REASON\_CPU\_EXCEEDED} and
	\MacroNI{HOLD\_SUBCODE\_CPU\_EXCEEDED} respectively.
	\item \texttt{Preempt\_If\_Memory\_Exceeded}
	If the startd observes the memory usage of the job exceed
	the memory provisioned in the slot, preempt the job immediately
	ignoring any job retirement time.
	\item \texttt{Hold\_If\_Memory\_Exceeded}
	If the startd observes the memory usage of the job exceed
	the memory provisioned in the slot,
	immediately place the job on hold
	ignoring any job retirement time.
	The job will go on hold with a reasonable
	hold reason in job attribute \Attr{HoldReason} and a value of 102 in job
	attribute \Attr{HoldReasonCode}.  The hold reason and code can be customized by
	specifying \MacroNI{HOLD\_REASON\_MEMORY\_EXCEEDED} and
	\MacroNI{HOLD\_SUBCODE\_MEMORY\_EXCEEDED} respectively.
  \end{itemize}

\label{usecategory:SECURITY}
\item[\MacroNI{SECURITY}]
  Describes configuration for an implemented security model.
  \begin{itemize}
    \item \texttt{Host\_Based}
    The default security model used.
    Do \emph{not} combine with \texttt{User\_Based} security.
    \item \texttt{User\_Based}
    Grants permissions to an administrator and uses 
    \texttt{With\_Authentication}.
    Do \emph{not} combine with \texttt{Host\_Based} security.
    \item \texttt{With\_Authentication}
    Requires both authentication and integrity checks.
    \item \texttt{Strong}
    Requires authentication, encryption, and integrity checks.
  \end{itemize}

\end{description}

For pools that are transitioning to using this new syntax in configuration,
while still having some tools and daemons with HTCondor versions 
earlier than 8.1.6,
special syntax in the configuration will cause those daemons to
fail upon start up,
rather than use the new, but misinterpreted, syntax. 
Newer daemons will ignore the extra syntax.
Placing the \verb|@| character before the \MacroNI{use} key word
causes the older daemons to fail when they attempt to
parse this syntax.

As an example, consider the \Condor{startd} as it starts up.
A \Condor{startd} previous to HTCondor version 8.1.6 fails to start
when it sees:
\begin{verbatim}
@use feature : GPUs
\end{verbatim}
Running an older \Condor{config\_val} also identifies the \Expr{@use}
line as being bad.
A \Condor{startd} of HTCondor version 8.1.6 or more recent sees
\begin{verbatim}
use feature : GPUs
\end{verbatim}
