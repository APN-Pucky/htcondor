%%%%%%%%%%%%%%%%%%%%%%%%%%%%%%%%%%%%%%%%%%%%%%%%%%%%%%%%%%%%%%%%%%%%%%
\section{\label{sec:Configuring-HTCondor-Templates}Configuration Templates}
%%%%%%%%%%%%%%%%%%%%%%%%%%%%%%%%%%%%%%%%%%%%%%%%%%%%%%%%%%%%%%%%%%%%%%

\index{HTCondor!configuration-templates}
\index{configuration: templates}

Achieving certain behaviors in an HTCondor pool often requires
setting the values of a number of configuration macros in concert
with each other.  We have added configuration templates as a way
to do this more easily, at a higher level, without having to
explicitly set each individual configuration macro.

Configuration templates are pre-defined; users cannot define their
own templates.

Note that the value of an individual configuration macro that is
set by a configuration template can be overridden by setting that
configuration macro later in the configuration.

Detailed information about configuration templates (such as the
macros they set) can be obtained using the \Condor{config\_val}
\texttt{use} option (see~\ref{man-condor-config-val}).
(This document does not contain such information because the
\Condor{config\_val} command is a better way to obtain it.)

%%%%%%%%%%%%%%%%%%%%%%%%%%%%%%%%%%%%%%%%%%%%%%%%%%%%%%%%%%%%%%%%%%%%%%
\subsection{\label{sec:Config-Templates}Configuration Templates: Using
Predefined Sets of Configuration}
%%%%%%%%%%%%%%%%%%%%%%%%%%%%%%%%%%%%%%%%%%%%%%%%%%%%%%%%%%%%%%%%%%%%%%
\index{configuration!USE syntax}
\index{USE configuration syntax}
Predefined sets of configuration can be identified and incorporated
into the configuration using the syntax
\begin{verbatim}
  use <category name> : <template name> 
\end{verbatim}

The \MacroNI{use} key word is case insensitive.
There are \emph{no} requirements for white space characters surrounding
the colon character.
More than one \MacroNI{<template name>} identifier may be placed within
a single \MacroNI{use} line. 
Separate the names by a space character. 
There is no mechanism by which the administrator may define their
own custom \MacroNI{<category name>} or \MacroNI{<template name>}.

Each predefined \MacroNI{<category name>} has a fixed, case insensitive
name for the sets of configuration that are predefined. 
Placement of a \MacroNI{use} line in the configuration brings in
the predefined configuration it identifies.

As of version 8.5.6, some of the configuration templates take arguments
(as described below).

%%%%%%%%%%%%%%%%%%%%%%%%%%%%%%%%%%%%%%%%%%%%%%%%%%%%%%%%%%%%%%%%%%%%%%
\subsection{\label{sec:Config-Templates-Available}Available
Configuration Templates}

There are four \MacroNI{<category name>} values.
Within a category, a predefined, case insensitive name identifies
the set of configuration it incorporates. 
\begin{description}

\label{usecategory:ROLE}
\item[\MacroNI{ROLE category}]
  Describes configuration for the various roles that a machine might
  play within an HTCondor pool. The configuration will identify which
  daemons are running on a machine.
  \begin{itemize}
    \item \texttt{Personal}

    Settings needed for when a single machine is the entire pool.

    \item \texttt{Submit}

    Settings needed to allow this machine to submit jobs to the pool.
    May be combined with \texttt{Execute} and \texttt{CentralManager} roles.

    \item \texttt{Execute}

    Settings needed to allow this machine to execute jobs.
    May be combined with \texttt{Submit} and \texttt{CentralManager} roles.

    \item \texttt{CentralManager}

    Settings needed to allow this machine to act as the central manager
    for the pool.
    May be combined with \texttt{Submit} and \texttt{Execute} roles.
  \end{itemize}

\label{usecategory:FEATURE}
\item[\MacroNI{FEATURE category}]
  Describes configuration for implemented features. 
  \begin{itemize}
    \item \texttt{Remote\_Runtime\_Config}

    Enables the use of \Condor{config\_val} \Opt{-rset} to the machine with
    this configuration.
    Note that there are security implications for use of this configuration,
    as it potentially permits the arbitrary modification of configuration.
    Variable \Macro{SETTABLE\_ATTRS\_CONFIG} must also be defined.

    \item \texttt{Remote\_Config}

    Enables the use of \Condor{config\_val} \Opt{-set} to the machine with
    this configuration.
    Note that there are security implications for use of this configuration,
    as it potentially permits the arbitrary modification of configuration.
    Variable \Macro{SETTABLE\_ATTRS\_CONFIG} must also be defined.

    \item \texttt{VMware}

    Enables use of the vm universe with VMware virtual machines.
    Note that this feature depends on Perl. 

    \item \texttt{GPUs}

    Sets configuration based on detection with the \Condor{gpu\_discovery}
    tool, and defines a custom resource using the name \Expr{GPUs}.
    Supports both OpenCL and CUDA, if detected.  Automatically includes
    the \texttt{GPUsMonitor} feature.

    \item \texttt{GPUsMonitor}

    Also adds configuration to report the usage of NVidia GPUs.

    \item \texttt{Monitor( resource\_name, mode, period, executable, metric[, metric]+ )}

    Configures a custom machine resource monitor with the given name, mode,
    period, executable, and metrics.  See \ref{sec:daemon-classad-hooks}
    for the definitions of these terms.

    \item \texttt{PartitionableSlot( slot\_type\_num [, allocation] )}

   	Sets up a partitionable slot of the specified slot type number
	and allocation (defaults for slot\_type\_num and allocation
	are 1 and 100\% respectively).
	See~\ref{sec:SMP-Divide} for information on partitionalble
	slot policies.
	
    \item \texttt{AssignAccountingGroup( map\_filename )}
	Sets up a \Condor{schedd} job transform that assigns an accounting group to
	each job as it is submitted. The accounting is determined by mapping the
	Owner attribute of the job using the given map file.

    \item \texttt{ScheddUserMapFile( map\_name, map\_filename )}
	Defines a \Condor{schedd} usermap named map\_name using the given map file.

    \item \texttt{SetJobAttrFromUserMap( dst\_attr, src\_attr, map\_name [, map\_filename] )}
	Sets up a \Condor{schedd} job transform that sets the dst\_attr attribute of each job
	as it is submitted. The value of dst\_attr is determined by mapping the src\_attr of the job
	using the usermap named map\_name.  If the optional map\_filename argument is specifed, then
	this metaknob also defines a \Condor{schedd} usermap named map\_Name using the given map file.

    \item \texttt{StartdCronOneShot( job\_name, exe [, hook\_args] )}

	Create a one-shot \Condor{startd} job hook.
	(See~\ref{sec:daemon-classad-hooks} for more information about job hooks.)

    \item \texttt{StartdCronPeriodic( job\_name, period, exe [, hook\_args] )}

	Create a periodic-shot \Condor{startd} job hook.
	(See~\ref{sec:daemon-classad-hooks} for more information about job hooks.)

    \item \texttt{StartdCronContinuous( job\_name, exe [, hook\_args] )}

	Create a (nearly) continuous \Condor{startd} job hook.
	(See~\ref{sec:daemon-classad-hooks} for more information about job hooks.)

    \item \texttt{ScheddCronOneShot( job\_name, exe [, hook\_args] )}

	Create a one-shot \Condor{schedd} job hook.
	(See~\ref{sec:daemon-classad-hooks} for more information about job hooks.)

    \item \texttt{ScheddCronPeriodic( job\_name, period, exe [, hook\_args] )}

	Create a periodic-shot \Condor{schedd} job hook.
	(See~\ref{sec:daemon-classad-hooks} for more information about job hooks.)

    \item \texttt{ScheddCronContinuous( job\_name, exe [, hook\_args] )}

	Create a (nearly) continuous \Condor{schedd} job hook.
	(See~\ref{sec:daemon-classad-hooks} for more information about job hooks.)

    \item \texttt{OneShotCronHook( STARTD\_CRON | SCHEDD\_CRON, job\_name,
	hook\_exe [,hook\_args] )}

	Create a one-shot job hook.
	(See~\ref{sec:daemon-classad-hooks} for more information about job hooks.)

    \item \texttt{PeriodicCronHook( STARTD\_CRON | SCHEDD\_CRON , job\_name,
	period, hook\_exe [,hook\_args] )}

	Create a periodic job hook.
	(See~\ref{sec:daemon-classad-hooks} for more information about job hooks.)

    \item \texttt{ContinuousCronHook( STARTD\_CRON | SCHEDD\_CRON ,
	job\_name, hook\_exe [,hook\_args] )}

	Create a (nearly) continuous job hook.
	(See~\ref{sec:daemon-classad-hooks} for more information about job hooks.)

	%TEMP -- not sure if we need to document this, because we don't
	% seem to use it anymore. wenger 2016-11-10
    % \item \texttt{TestingMode\_Policy\_Values}
\
    \item \texttt{UWCS\_Desktop\_Policy\_Values}

	Configuration values used in the \texttt{UWCS\_DESKTOP} policy.
	(Note that these values were previously in the parameter table;
	configuration that uses these values will have to use the
	\texttt{UWCS\_Desktop\_Policy\_Values} template.  For example,
	\texttt{POLICY : UWCS\_Desktop} uses the
	\texttt{FEATURE : UWCS\_Desktop\_Policy\_Values} template.)
  \end{itemize}

\label{usecategory:POLICY}
\item[\MacroNI{POLICY category}]
  Describes configuration for the circumstances under which
  machines choose to run jobs.
  \begin{itemize}

    \item \texttt{Always\_Run\_Jobs}

    Always start jobs and run them to completion, without consideration of
    \Condor{negotiator} generated preemption or suspension.
    This is the default policy, and it is intended to be used with dedicated
    resources.
    If this policy is used together with the \texttt{Limit\_Job\_Runtimes}
    policy,
    order the specification by placing this \texttt{Always\_Run\_Jobs} 
    policy first. 

    \item \texttt{UWCS\_Desktop}

    This was the default policy before HTCondor version 8.1.6.
    It is intended to be used with desktop machines not exclusively running
    HTCondor jobs.
    It injects \Expr{UWCS} into the name of some configuration variables.

    \item \texttt{Desktop}

    An updated and reimplementation of the \texttt{UWCS\_Desktop} policy,
    but \emph{without} the \Expr{UWCS} naming of some configuration variables.

    \item \texttt{Limit\_Job\_Runtimes( limit\_in\_seconds )}

    Limits running jobs to a maximum of the specified time using preemption.
	(The default limit is 24 hours.)
    If this policy is used together with the \texttt{Always\_Run\_Jobs} policy,
    order the specification by placing this \texttt{Limit\_Job\_Runtimes} 
    policy second. 

	\item \texttt{Preempt\_If\_Cpus\_Exceeded}

	If the startd observes the number of CPU cores used by the job exceed
	the number of cores in the slot by more than 0.8 on average over the past
	minute, preempt the job immediately
	ignoring any job retirement time.

	\item \texttt{Hold\_If\_Cpus\_Exceeded}

	If the startd observes the number of CPU cores used by the job exceed
	the number of cores in the slot by more than 0.8 on average over the past
	minute, immediately place the job on hold
	ignoring any job retirement time.  The job will go on hold with a reasonable
	hold reason in job attribute \Attr{HoldReason} and a value of 101 in job
	attribute \Attr{HoldReasonCode}.  The hold reason and code can be customized by
	specifying \MacroNI{HOLD\_REASON\_CPU\_EXCEEDED} and
	\MacroNI{HOLD\_SUBCODE\_CPU\_EXCEEDED} respectively.

	Standard universe jobs can't be held by startd policy expressions,
	so this metaknob automatically ignores them.

	\item \texttt{Preempt\_If\_Memory\_Exceeded}

	If the startd observes the memory usage of the job exceed
	the memory provisioned in the slot, preempt the job immediately
	ignoring any job retirement time.

	\item \texttt{Hold\_If\_Memory\_Exceeded}

	If the startd observes the memory usage of the job exceed
	the memory provisioned in the slot,
	immediately place the job on hold
	ignoring any job retirement time.
	The job will go on hold with a reasonable
	hold reason in job attribute \Attr{HoldReason} and a value of 102 in job
	attribute \Attr{HoldReasonCode}.  The hold reason and code can be customized by
	specifying \MacroNI{HOLD\_REASON\_MEMORY\_EXCEEDED} and
	\MacroNI{HOLD\_SUBCODE\_MEMORY\_EXCEEDED} respectively.

	Standard universe jobs can't be held by startd policy expressions,
	so this metaknob automatically ignores them.

	\item \texttt{Preempt\_If( policy\_variable )}

	Preempt jobs according to the specified policy.
	\texttt{policy\_variable} must be the name of a configuration macro
	containing an expression that evaluates to \Expr{True} if the job
	should be preempted.

	See an example here: ~\ref{sec:Config-Template-Examples}.

	\item \texttt{Want\_Hold\_If( policy\_variable, subcode, reason\_text )}

	Add the given policy to the \MacroNI{WANT\_HOLD} expression; if the
	\MacroNI{WANT\_HOLD} expression is defined, \texttt{policy\_variable}
	is prepended to the existing expression; otherwise
	\MacroNI{WANT\_HOLD} is simply set to the value of the
	texttt{policy\_variable} macro.

	Standard universe jobs can't be held by startd policy expressions,
	so this metaknob automatically ignores them.

	See an example here: ~\ref{sec:Config-Template-Examples}.

	\item \texttt{Startd\_Publish\_CpusUsage}

	Publish the number of CPU cores being used by the job into
	to slot ad as attribute \Attr{CpusUsage}. This value will
	be the average number of cores used by the job over the
	past minute, sampling every 5 seconds.
  \end{itemize}

\label{usecategory:SECURITY}
\item[\MacroNI{SECURITY category}]
  Describes configuration for an implemented security model.
  \begin{itemize}
    \item \texttt{Host\_Based}

    The default security model (based on IPs and DNS names).
    Do \emph{not} combine with \texttt{User\_Based} security.

    \item \texttt{User\_Based}

    Grants permissions to an administrator and uses 
    \texttt{With\_Authentication}.
    Do \emph{not} combine with \texttt{Host\_Based} security.

    \item \texttt{With\_Authentication}

    Requires both authentication and integrity checks.

    \item \texttt{Strong}

    Requires authentication, encryption, and integrity checks.
  \end{itemize}

\end{description}

%%%%%%%%%%%%%%%%%%%%%%%%%%%%%%%%%%%%%%%%%%%%%%%%%%%%%%%%%%%%%%%%%%%%%%
\subsection{\label{sec:Config-Template-Transition-Syntax}Configuration
Template Transition Syntax}
%%%%%%%%%%%%%%%%%%%%%%%%%%%%%%%%%%%%%%%%%%%%%%%%%%%%%%%%%%%%%%%%%%%%%%

For pools that are transitioning to using this new syntax in configuration,
while still having some tools and daemons with HTCondor versions 
earlier than 8.1.6,
special syntax in the configuration will cause those daemons to
fail upon start up,
rather than use the new, but misinterpreted, syntax. 
Newer daemons will ignore the extra syntax.
Placing the \verb|@| character before the \MacroNI{use} key word
causes the older daemons to fail when they attempt to
parse this syntax.

As an example, consider the \Condor{startd} as it starts up.
A \Condor{startd} previous to HTCondor version 8.1.6 fails to start
when it sees:
\begin{verbatim}
@use feature : GPUs
\end{verbatim}
Running an older \Condor{config\_val} also identifies the \Expr{@use}
line as being bad.
A \Condor{startd} of HTCondor version 8.1.6 or more recent sees
\begin{verbatim}
use feature : GPUs
\end{verbatim}

%%%%%%%%%%%%%%%%%%%%%%%%%%%%%%%%%%%%%%%%%%%%%%%%%%%%%%%%%%%%%%%%%%%%%%
\subsection{\label{sec:Config-Template-Examples}Configuration Template
Examples}
%%%%%%%%%%%%%%%%%%%%%%%%%%%%%%%%%%%%%%%%%%%%%%%%%%%%%%%%%%%%%%%%%%%%%%

\begin{itemize}
\item Preempt a job if its memory usage exceeds the requested memory:
    \begin{verbatim}
	MEMORY_EXCEEDED = (isDefined(MemoryUsage) && MemoryUsage > RequestMemory)
	use POLICY : PREEMPT_IF(MEMORY_EXCEEDED)
    \end{verbatim}

\item Put a job on hold if its memory usage exceeds the requested memory:
    \begin{verbatim}
	MEMORY_EXCEEDED = (isDefined(MemoryUsage) && MemoryUsage > RequestMemory)
	use POLICY : WANT_HOLD_IF(MEMORY_EXCEEDED, 102, memory usage exceeded request_memory)
    \end{verbatim}

\item Update dynamic GPU information every 15 minutes:

    \begin{verbatim}
	use FEATURE : StartdCronPeriodic(DYNGPU, 15*60, $(LOCAL_DIR)\dynamic_gpu_info.pl, $(LIBEXEC)\condor_gpu_discovery -dynamic)
    \end{verbatim}

where \texttt{dynamic\_gpu\_info.pl}  is a simple perl script that strips
off the DetectedGPUs line from texttt{condor\_gpu\_discovery}:

    \begin{verbatim}
	#!/usr/bin/env perl
	my @attrs = `@ARGV`;
	for (@attrs) {
		next if ($_ =~ /^Detected/i);
		print $_;
	}
    \end{verbatim}
\end{itemize}
