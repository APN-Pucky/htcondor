%% IMPORTANT:
%% This file contains prose to describe the semantics and syntax
%% of the if-else that can be used both in configuration and
%% in a submit description file.
%%
%% Please be careful when changing this prose, as it is included
%% in 2 different parts of the manual.  Do not include examples
%% specific to either a submit description file usage or a
%% configuration usage in this prose.

Conditional \verb@if@/\verb@else@ semantics
are available in a limited form.
The syntax:
\begin{verbatim}
  if <simple condition>
     <statement>
     . . .
     <statement>
  else
     <statement>
     . . .
     <statement>
  endif
\end{verbatim}

An \verb@else@ key word and statements are not required,
such that simple \verb@if@ semantics are implemented.
The  \verb@<simple condition>@ does not permit compound conditions.
It optionally contains the exclamation point character (\verb@!@)
to represent the not operation,
followed by 
\begin{itemize}
  \item the \verb@defined@ keyword followed by the name of a 
    variable.
    If the variable is defined, the statement(s) are
    incorporated into the expanded input.
    If the variable is \emph{not} defined, the statement(s) are
    not incorporated into the expanded input.
    As an example,
\begin{verbatim}
  if defined MY_UNDEFINED_VARIABLE
     X = 12
  else
     X = -1
  endif
\end{verbatim}
    results in \Expr{X = -1}, when \Expr{MY\_UNDEFINED\_VARIABLE} is
    \emph{not} yet defined.
  \item the \verb@version@ keyword, representing the version number of
    of the daemon or tool currently reading this conditional. 
    This keyword is followed by an HTCondor version number.
    That version number can be of the form \verb@x.y.z@ or \verb@x.y@.
    The version of the daemon or tool is compared to the specified
    version number.
    The comparison operators are
    \begin{itemize}
    \item \verb@==@ for equality. Current version 8.2.3 is equal to 8.2.
    \item \verb@>=@ to see if the current version number is greater than or
    equal to. Current version 8.2.3 is greater than 8.2.2,
    and current version 8.2.3 is greater than or equal to 8.2.
    \item \verb@<=@ to see if the current version number is less than or
    equal to. Current version 8.2.0 is less than 8.2.2, 
    and current version 8.2.3 is less than or equal to 8.2.
    \end{itemize}
    As an example,
\begin{verbatim}
  if version >= 8.1.6
     DO_X = True
  else
     DO_Y = True
  endif
\end{verbatim}
    results in defining \Expr{DO\_X} as \Expr{True} if the current
    version of the daemon or tool reading this if statement is 8.1.6
    or a more recent version.
  \item \verb@True@ or \verb@yes@ or the value 1.
    The statement(s) are incorporated.
  \item \verb@False@ or \verb@no@ or the value 0
    The statement(s) are \emph{not} incorporated.
  \item \verb@$(<variable>)@ may be used where the
    immediately evaluated value is a simple boolean value.
    A value that evaluates to the empty string is considered \verb@False@,
    otherwise a value that does not evaluate to a simple boolean value
    is a syntax error.
\end{itemize}

The syntax
\begin{verbatim}
  if <simple condition>
     <statement>
     . . .
     <statement>
  elif <simple condition>
     <statement>
     . . .
     <statement>
  endif
\end{verbatim}
is the same as syntax
\begin{verbatim}
  if <simple condition>
     <statement>
     . . .
     <statement>
  else
     if <simple condition>
        <statement>
        . . .
        <statement>
     endif
  endif
\end{verbatim}
