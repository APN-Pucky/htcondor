%%%%%%%%%%%%%%%%%%%%%%%%%%%%%%%%%%%%%%%%%%%%%%%%%%%%%%%%%%%%%%%%%%%%%%%%%%%
\subsection{\label{sec:ipv6}Running HTCondor on an IPv6 Network Stack}
%%%%%%%%%%%%%%%%%%%%%%%%%%%%%%%%%%%%%%%%%%%%%%%%%%%%%%%%%%%%%%%%%%%%%%%%%%
\index{IPv6|(}

HTCondor has limited support for running on IPv6 networks.

Current Limitations

\begin{itemize}
    \item{Microsoft Windows platforms are \emph{not} supported.}
    \item{Mixed IPv4/IPv6 pools are \emph{not} supported.}
    \item{Security policies cannot use IP addresses, only host names.
If using \Expr{NO\_DNS=TRUE}, the host names are reformatted IP addresses,
and can be matched against those. }
    \item{\MacroNI{NETWORK\_INTERFACE} \emph{must be set} to a 
specific IPv6 address. 
It is not possible to use multiple IPv6 interfaces on a single computer.}
    \item{There must be valid IPv6 (AAAA) DNS and reverse DNS records for 
the computers. 
Setting the configuration \Expr{NO\_DNS=TRUE} removes this limitation.}
\end{itemize}

Enabling IPv6

\begin{itemize}
    \item{In the configuration, set \Expr{ENABLE\_IPV6 = True} 
    and \Expr{ENABLE\_IPV4 = False}.}
    \item{Specify the IPv6 interface to use. 
Do \emph{not} put square brackets ([]) around this address.
As an example,
\begin{verbatim}
NETWORK_INTERFACE = 2607:f388:1086:0:21b:24ff:fedf:b520
\end{verbatim}
}
\end{itemize}

Additional Notes

Specification of  \MacroNI{CONDOR\_HOST} or \MacroNI{COLLECTOR\_HOST}
as an IP address \emph{must} place the address, 
but \emph{not} the port, in square brackets. 
Host names may be specified. 
For example:

\begin{verbatim}
CONDOR_HOST =[2607:f388:1086:0:21e:68ff:fe0f]:6462
# This configures the collector to listen on the non-standard port 5332.
COLLECTOR_HOST =[2607:f388:1086:0:21e:68ff:fe0f:6462]:5332
\end{verbatim}

Because IPv6 addresses are not currently supported in HTCondor's 
security settings, 
\MacroUNI{CONDOR\_HOST} or \MacroUNI{COLLECTOR\_HOST} will not
be permitted in the security configuration, 
to specify an IP address.

When using the configuration variable \MacroNI{NO\_DNS},
IPv6 addresses are turned into host names by taking the IPv6 address, 
changing colons to dashes, and appending \MacroUNI{DEFAULT\_DOMAIN\_NAME}. 
So, 
\begin{verbatim}
2607:f388:1086:0:21b:24ff:fedf:b520
\end{verbatim}
becomes 
\begin{verbatim}
2607-f388-1086-0-21b-24ff-fedf-b520.example.com 
\end{verbatim}
assuming
\begin{verbatim}
DEFAULT_DOMAIN_NAME=example.com
\end{verbatim}

\index{IPv6|)}
