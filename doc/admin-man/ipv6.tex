%%%%%%%%%%%%%%%%%%%%%%%%%%%%%%%%%%%%%%%%%%%%%%%%%%%%%%%%%%%%%%%%%%%%%%%%%%%
\subsection{\label{sec:ipv6}Running HTCondor on an IPv6 Network Stack}
%%%%%%%%%%%%%%%%%%%%%%%%%%%%%%%%%%%%%%%%%%%%%%%%%%%%%%%%%%%%%%%%%%%%%%%%%%
\index{IPv6|(}

HTCondor supports running on IPv6 networks, and has limited support for
running on mixed-mode networks.

To enable IPv6 networking, set \Macro{ENABLE\_IPV6} to \Expr{True}.  To
prevent HTCondor from using IPv4 networking, set \Macro{ENABLE\_IPV4} to
\Expr{False}.  HTCondor uses only IPv4 by default.

If you enable both IPv4 and IPv6 networking, HTCondor will run in mixed
mode.  In this mode, HTCondor determine which protocol to use to contact
a daemon in a number of different ways, depending on how it obtained the
daemon's address.  Generally speaking, HTCondor will obtain a daemon's
address in one of three ways:

\begin{itemize}
\item{on the command line or in configuration;}
\item{from a daemon to which it is connected;}
\item{or from the HTCondor collector, including when a daemon name is specified on the command line.}
\end{itemize}

HTCondor determines the protocol to use as follows:

\begin{itemize}
\item{If given an address literal, HTCondor will use the same protocol.  If
given a name, HTCondor will resolve the name (note that resolution occurs
differently when \Macro{NO\_DNS} is enabled) and use the protocol of the address
returned first.}
\item{If an HTCondor daemon sends its address to another daemon, it will send
its address on the protocol it is using to contact that other daemon.}
\item{Thus, other daemons (and tools which accept daemon names) will contact the
daemon on the same protocol as it contacted the collector.}
\end{itemize}

In practice, this means that an HTCondor pool's central manager (collector
and negotiator) and submit nodes (schedds) must have both IPv4 and IPv6
addresses for both IPv4-only and IPv6-only execute nodes (startds) to
function properly.

\subsubsection{IPv6 and Host-Based Security}

You may freely intermix IPv6 and IPv4 address literals.  You may also specify
IPv6 netmasks as a legal IPv6 address followed by a slash followed by the
number of bits in the mask; or as the prefix of a legal IPv6 address followed
by two colons followed by an asterisk.  The latter is entirely equivalent to the
former, except that it only allows you to (implicitly) specify mask bits
in groups of sixteen.  For example, \texttt{fe8f:1234::/60} and
\texttt{fe8f:1234::*} specify the same network mask.

The HTCondor security subsystem resolves names in the ALLOW and DENY
lists and uses all of the resulting IP addresses.  Thus, to allow or deny
IPv6 addresses, the names must have IPv6 DNS entries (AAAA records), or
\MacroNI{NO\_DNS} must be enabled.

\subsubsection{IPv6 Address Literals}

When you specify an IPv6 address and a port number simultaneously, you
must separate the IPv6 address from the port number by placing square
brackets around the address.  For instance:

\begin{verbatim}
COLLECTOR_HOST = [2607:f388:1086:0:21e:68ff:fe0f:6462]:5332
\end{verbatim}

If you do not (or may not) specify a port, do not use the square brackets.
For instance:

\begin{verbatim}
NETWORK_INTERFACE = 1234:5678::90ab
\end{verbatim}

\subsubsection{IPv6 without DNS}

When using the configuration variable \Macro{NO\_DNS},
IPv6 addresses are turned into host names by taking the IPv6 address,
changing colons to dashes, and appending \MacroUNI{DEFAULT\_DOMAIN\_NAME}.
So,
\begin{verbatim}
2607:f388:1086:0:21b:24ff:fedf:b520
\end{verbatim}
becomes
\begin{verbatim}
2607-f388-1086-0-21b-24ff-fedf-b520.example.com
\end{verbatim}
assuming
\begin{verbatim}
DEFAULT_DOMAIN_NAME=example.com
\end{verbatim}

\index{IPv6|)}
