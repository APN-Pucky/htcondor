%%%%%%%%%%%%%%%%%%%%%%%%%%%%%%%%%%%%%%%%%%%%%%%%%%%%%%%%%%%%%%%%%%%%%%%%%%%
\subsection{\label{sec:Config-Dedicated-Jobs}
Configuring Condor for Running Dedicated Jobs} 
%%%%%%%%%%%%%%%%%%%%%%%%%%%%%%%%%%%%%%%%%%%%%%%%%%%%%%%%%%%%%%%%%%%%%%%%%%

\index{MPI}
\index{MPI!Condor configuration}
\index{dedicated machines}

Beginning with Condor version 6.3.0, users can submit applications to
Condor which require multiple resources, and cannot be preempt by
the machine's user.

Condor's unique solution to this problem involves a combination of
\index{scheduling!opportunistic}
\index{opportunistic scheduling}
\Term{opportunistic scheduling} and \Term{dedicated scheduling} within
a single system.
\Term{Opportunistic scheduling} involves placing jobs on non-dedicated
resources under the assumption that the resources might not be
available for the entire duration of the jobs.
\index{scheduling!dedicated}
\index{dedicated scheduling}
\Term{Dedicated scheduling} assumes the constant availability of
resources to compute fixed schedules.
In other words, dedicated scheduling involves placing jobs on
resources where it is assumed that the job can run to completion
without interruption.

This section describes how to configure a Condor pool for scheduling
and running parallel jobs on dedicated resources.
The user manual, section~\ref{sec:MPI} on page~\pageref{sec:MPI}
contains information on how to submit MPI jobs under Condor.


%%%%%%%%%%%%%%%%%%%%%%%%%%%%%%%%%%%%%%%%%%%%%%%%%%%%%%%%%%%%%%%%%%%%%%
\subsubsection{\label{sec:Overview-Dedicated-Condor}
Overview of how Condor Manages Dedicated Jobs and Resources}
%%%%%%%%%%%%%%%%%%%%%%%%%%%%%%%%%%%%%%%%%%%%%%%%%%%%%%%%%%%%%%%%%%%%%%

To support dedicated applications, a Condor administrator must
configure some resources in the pool to be dedicated resources.
These resources are controlled by a dedicated scheduler, a
single machine within the pool that runs a \Condor{schedd} daemon.
In general, there is no limit on the number of dedicated schedulers within
a Condor pool.
However, each dedicated resource may only be managed by a single
dedicated scheduler.
Therefore, running multiple dedicated schedulers in a single pool
results in a greater fragmentation of dedicated resources.
This can create a situation where jobs will not run, because the jobs
can not get needed resources.  

After a \Condor{schedd} daemon has been selected as the dedicated scheduler
for the pool and resources are configured to be
dedicated, users submit MPI jobs to that \Condor{schedd}.
When an idle MPI job is found in the queue, the dedicated scheduler
performs its own scheduling algorithm to find appropriate
resources for the job.
The dedicated scheduler claims the resources and uses them to
service the MPI job.
When a resource can no longer be used to serve dedicated jobs, it is
allowed to run opportunistic jobs.


%%%%%%%%%%%%%%%%%%%%%%%%%%%%%%%%%%%%%%%%%%%%%%%%%%%%%%%%%%%%%%%%%%%%%%
\subsubsection{\label{sec:Setup-Dedicated-Scheduler}
Selecting and Setting up your Dedicated Scheduler}
%%%%%%%%%%%%%%%%%%%%%%%%%%%%%%%%%%%%%%%%%%%%%%%%%%%%%%%%%%%%%%%%%%%%%%

We recommend that you select a single host to act as the
dedicated scheduler.
This is the host from which all users submit their MPI jobs.
If you have a dedicated cluster of compute nodes and a single
front-end machine from which users are supposed to submit jobs, that
machine would be a perfect choice for your dedicated
scheduler.
If your pool does not have an obvious choice for a submit machine,
choose a host that all of your users can log into, and one
that is likely to be up and running all the time.
All of the Condor's other resource requirements for a submit node apply to
this machine, such as having enough disk space in the spool
directory to hold jobs (see section~\ref{sec:Preparing-to-Install} on
page~\pageref{sec:Preparing-to-Install} for details on these issues). 

Once you have selected a machine to serve as the dedicated scheduler,
ensure that the machine is running version of the \Condor{schedd}
and \Condor{shadow} daemons that support MPI jobs.
These versions must be the same,
and they should be at least 6.3.0.
The default configuration files with Condor version 6.3.0 include all
required settings.

%%%%%%%%%%%%%%%%%%%%%%%%%%%%%%%%%%%%%%%%%%%%%%%%%%%%%%%%%%%%%%%%%%%%%%
\subsubsection{\label{sec:Configure-Dedicated-Resource}
Configuration for Dedicated Resources} 
%%%%%%%%%%%%%%%%%%%%%%%%%%%%%%%%%%%%%%%%%%%%%%%%%%%%%%%%%%%%%%%%%%%%%%

To configure a dedicated resource under a given scheduler, the
resource owner or administrator sets a few lines in the
\Condor{startd}'s configuration file.
Starting with Condor version 6.3.0, all of these settings are supplied
in an example local configuration file called
\File{condor\_config.local.dedicated.resource} which can be found in
the \File{etc} directory once you unpack the Condor binaries.

Each dedicated resource advertises a special attribute in its ClassAd
that says which dedicated scheduler it is willing to be managed by.
This is accomplished by modifying the following lines within the local
configuration file for any dedicated resource:

\begin{verbatim}
DedicatedScheduler = "DedicatedScheduler@full.host.name"
STARTD_EXPRS = $(STARTD_EXPRS), DedicatedScheduler
\end{verbatim}

Substitute the real host name of the dedicated scheduler
machine. 

All dedicated resources must have policy expressions which allow for
dedicated jobs to always run and to never be evicted.
The resource must also be configured to prefer jobs from the dedicated 
scheduler over all other jobs.
A dedicated resource in Condor is simply configured so that the
dedicated scheduler of its choice has the highest rank.
See section~\ref{sec:Configuring-Policy} on
page~\pageref{sec:Configuring-Policy} 
for more details on Condor's policy expressions. 

It is worth noting that Condor puts no other requirements on a
resource for it to be considered dedicated.  
If the owners of desk-top workstations were willing to allow their
machines to be configured in this way, those workstations would be
dedicated resources in Condor, and would behave exactly like the nodes
in a Beowulf compute cluster.

To aid in the definition of the policy expressions, the dedicated
scheduler adds an attribute to all resource request ClassAds it
generates, the \Attr{Scheduler} attribute.
This attribute identifies each ClassAd as a request of a particular
dedicated scheduler.
For example, if your dedicated scheduler were running on a host named
front-end.cs.wisc.edu, the \Attr{Scheduler} attribute for all jobs
submitted from there would be
DedicatedScheduler@front-end.cs.wisc.edu. 

The owners of the resources can easily define separate policies for
dedicated and opportunistic jobs, simply by including two cases in
each policy expression, one case for when the Scheduler attribute
identifies the request as one belonging to the preferred dedicated
scheduler, and one for if the Scheduler attribute is not defined or
points to a different scheduler.

In the following sections, we will discuss a couple of different
policy scenarios you might want to use for your dedicated resources
and give you the exact policy expressions to put in your configuration
file to implement them.
The configuration settings for each scenario are provided in the
\File{condor\_config.local.dedicated.resource} file.

\Note You can configure different resources in your pool to have
different dedicated policies.
For example, you might have a cluster of machines in racks which have
no interactive user and which can always run jobs, along-side desk-top
machines that are willing to run dedicated jobs when necessary, but
which will still preempt and evict non-dedicated jobs if the machine
is being used by its owner.  
Both of these policy scenarios are discussed below, and both might be
present in a single pool.
In other words, the following policy scenarios are specific to a given
machine, not to a whole pool.


%%%%%%%%%%%%%%%%%%%%%%%%%%%%%%%%%%%%%%%%%%%%%%%%%%%%%%%%%%%%%%%%%%%%%%
\subsubsection{\label{sec:Configure-Dedicated-Only-Policy}
Policy Scenario One: Run Only Dedicated Jobs}
%%%%%%%%%%%%%%%%%%%%%%%%%%%%%%%%%%%%%%%%%%%%%%%%%%%%%%%%%%%%%%%%%%%%%%

One possible scenario for the use of dedicated resources
is to only allow dedicated jobs to run on them.  
This is the most basic policy for dedicated resources.
To enact this policy, the following expressions are used in
the configuration file:

\begin{verbatim}
START     = Scheduler =?= $(DedicatedScheduler)
SUSPEND   = False
CONTINUE  = True
PREEMPT   = False
KILL      = False
WANT_SUSPEND   = False
WANT_VACATE    = False
RANK      = Scheduler =?= $(DedicatedScheduler)
\end{verbatim}

The \Macro{START} expression specifies that the \Attr{Scheduler}
attribute in the job ClassAd must match the string specified for
the \Attr{DedicatedScheduler} attribute in the machine ClassAd.
The \Macro{RANK} expression specifies that a job with the
\Attr{Scheduler} attribute appropriately defined will have the highest
rank, which will prevent any other jobs from preempting it based on
user priorities.
The rest of the expressions disable all of the \Condor{startd} daemon's
regular policies for evicting jobs when keyboard and CPU activity is
discovered on the machine.


%%%%%%%%%%%%%%%%%%%%%%%%%%%%%%%%%%%%%%%%%%%%%%%%%%%%%%%%%%%%%%%%%%%%%%
\subsubsection{\label{sec:Configure-Dedicated-Ex2}
Policy Scenario Two: Running Dedicated and Opportunistic Jobs}
%%%%%%%%%%%%%%%%%%%%%%%%%%%%%%%%%%%%%%%%%%%%%%%%%%%%%%%%%%%%%%%%%%%%%%

While the first example works nicely for dedicated jobs,
it can 
lead to poor resource utilization if there are not enough dedicated
jobs to keep the dedicated machines busy.  
A more sophisticated strategy allows 
the machines to run non-dedicated jobs when no dedicated jobs exist.
The machine is
configured to prefer dedicated jobs, yet
run opportunistic jobs if no dedicated jobs are available.
Note that those jobs that do not require a dedicated resource
are executed as if they were dedicated jobs.

To implement this,
configure machines as dedicated resources.
Then, modify the \MacroNI{START} expression to be:

\begin{verbatim}
START = True
\end{verbatim}

%%%%%%%%%%%%%%%%%%%%%%%%%%%%%%%%%%%%%%%%%%%%%%%%%%%%%%%%%%%%%%%%%%%%%%
\subsubsection{\label{sec:Configure-Dedicated-Ex3}
Policy Scenario Three: Running Dedicated Jobs on Desk-Top Resources}
%%%%%%%%%%%%%%%%%%%%%%%%%%%%%%%%%%%%%%%%%%%%%%%%%%%%%%%%%%%%%%%%%%%%%%

A third policy example allows both dedicated and non-dedicated
jobs.
It assumes resources are not
configured to prefer or always run dedicated jobs.
These desk-top machines have a \MacroNI{START} expression that
takes the machine owner's usage into account for non-dedicated
jobs.
The machine does not preempt jobs that must run on dedicated
resources,
while it will preempt other jobs based on a previously set
policy for running jobs.
So, the default pool policy is used for starting and
stopping non-dedicated jobs, but dedicated jobs always start 
and are not preempted.

Allowing both dedicated and opportunistic jobs on the resources
requires that an opportunistic policy is already defined.
In the local configuration file for resources with this hybrid policy,
a second case is added to policy expressions that overrides
the initial policy expression specifically for
dedicated jobs.
The following represent the only settings that need to be modified
to implement this policy. 

\begin{verbatim}
SUSPEND    = Scheduler =!= $(DedicatedScheduler) && ($(SUSPEND))
PREEMPT    = Scheduler =!= $(DedicatedScheduler) && ($(PREEMPT))
RANK_FACTOR    = 1000000
RANK   = (Scheduler =?= $(DedicatedScheduler) * $(RANK_FACTOR)) + $(RANK)
START  = (Scheduler =?= $(DedicatedScheduler)) || ($(START))
\end{verbatim}

\Note For everything to work, you MUST set Macro{RANK\_FACTOR} to be a
larger value than the maximum value your existing rank expression
could possibly evaluate to.
\Macro{RANK} is just a floating point value, so there is no harm in
having a value that is very large. 

%%%%%%%%%%%%%%%%%%%%%%%%%%%%%%%%%%%%%%%%%%%%%%%%%%%%%%%%%%%%%%%%%%%%%%
\subsubsection{\label{sec:Configure-Dedicated-Preemption}
Preemption with Dedicated Jobs}
%%%%%%%%%%%%%%%%%%%%%%%%%%%%%%%%%%%%%%%%%%%%%%%%%%%%%%%%%%%%%%%%%%%%%%

The dedicated scheduler can optionally preempt running MPI jobs in
favor of higher priority MPI jobs in its queue.  Note that this is
different from preemption in non-MPI universes, and MPI jobs cannot
be preempted either by a machine's user pressing a key or by other means.

By default, the dedicated scheduler will never preempt running jobs.
Two configuration file items control dedicated preemption: 
\Macro{SCHEDD\_PREEMPTION\_REQUIREMENTS} and \Macro{SCHEDD\_PREEMPTION\_RANK}.
These have no default value, so if either are unset, preemption will
never occur. \MacroNI{SCHEDD\_PREEMPTION\_REQUIREMENTS} must evaluate to true
for a machine to be a candidate for preemption.  If more machines are 
candidates for preemption than needed to satisfy a higher priority job, the
preemptible machines are sorted by \MacroNI{SCHEDD\_PREEMPTION\_RANK}, and
only the highest ranked machines are taken.

Note that preempting one node of a running MPI job requires killing
the entire job on all of its nodes.  So, when preemption happens, it
may end up freeing more machines than strictly speaking are needed.
Also, as condor cannot checkpoint MPI jobs, preempted jobs will be re-run
from the beginning.  Thus, the administrator should be careful when
enabling dedicated preemption.  The following example shows how to
enable dedicated preemption.

\begin{verbatim}
STARTD_JOB_EXPRS = JobPrio
SCHEDD_PREEMPTION_REQUIREMENTS = (My.JobPrio < Target.JobPrio)
SCHEDD_PREEMPTION_RANK = 0.0
\end{verbatim}

In this case, preemption is enabled by the user job priority. If a set
of machines is running a job at user priority 5, and the user submits
a new job at user priority 10, the running job will be preempted for
the new job.  The old job is put back in the queue, and will start running
from the beginning when the new job completes.
