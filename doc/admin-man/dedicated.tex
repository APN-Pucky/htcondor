%%%%%%%%%%%%%%%%%%%%%%%%%%%%%%%%%%%%%%%%%%%%%%%%%%%%%%%%%%%%%%%%%%%%%%%%%%%
\subsection{\label{sec:Config-Dedicated-Jobs}
Condor's Dedicated Scheduling} 
%%%%%%%%%%%%%%%%%%%%%%%%%%%%%%%%%%%%%%%%%%%%%%%%%%%%%%%%%%%%%%%%%%%%%%%%%%
\index{dedicated scheduling}
\index{MPI application!under the dedicated scheduler}

The dedicated scheduler is a part of the \Condor{schedd} that handles 
the scheduling of parallel jobs that require more than one machine
concurrently running per job.  
MPI applications are a common use for the dedicated scheduler, 
but parallel applications which do not require MPI can also be run 
with the dedicated scheduler.
All jobs which use the parallel universe are routed to the dedicated scheduler
within the \Condor{schedd} they were submitted to.  
A default Condor installation
does not configure a dedicated scheduler; 
the administrator must designate one or more \Condor{schedd} daemons
to perform as dedicated scheduler.

%%%%%%%%%%%%%%%%%%%%%%%%%%%%%%%%%%%%%%%%%%%%%%%%%%%%%%%%%%%%%%%%%%%%%%
\subsubsection{\label{sec:Setup-Dedicated-Scheduler}
Selecting and Setting Up a Dedicated Scheduler}
%%%%%%%%%%%%%%%%%%%%%%%%%%%%%%%%%%%%%%%%%%%%%%%%%%%%%%%%%%%%%%%%%%%%%%

We recommend that you select a single machine within a 
Condor pool to act as the dedicated scheduler.
This becomes the machine from upon which all users submit their 
parallel universe jobs.
The perfect choice for the dedicated scheduler 
is the single, front-end machine for
a dedicated cluster of compute nodes.
For the pool without an obvious choice for a submit machine,
choose a machine that all users can log into, as well as one
that is likely to be up and running all the time.
All of Condor's other resource requirements for a submit machine apply to
this machine, such as having enough disk space in the spool
directory to hold jobs. See section~\ref{sec:Preparing-to-Install} on
page~\pageref{sec:Preparing-to-Install} for details on these issues. 

%%%%%%%%%%%%%%%%%%%%%%%%%%%%%%%%%%%%%%%%%%%%%%%%%%%%%%%%%%%%%%%%%%%%%%
\subsubsection{\label{sec:Configure-Dedicated-Resource}
Configuration Examples for Dedicated Resources} 
%%%%%%%%%%%%%%%%%%%%%%%%%%%%%%%%%%%%%%%%%%%%%%%%%%%%%%%%%%%%%%%%%%%%%%

Each machine may have its own policy for the execution of jobs.
This policy is set by configuration.
Each machine with aspects of its configuration that are dedicated
identifies the dedicated scheduler.
And, the ClassAd representing a job to be executed on
one or more of these dedicated machines includes an identifying attribute.
An example configuration file with the following various policy settings
is \File{/etc/condor\_config.local.dedicated.resource}.

Each dedicated machine defines the configuration variable
\Macro{DedicatedScheduler}, which identifies
the dedicated scheduler it is managed by.
The local configuration file for any dedicated resource contains
a modified form of

\begin{verbatim}
DedicatedScheduler = "DedicatedScheduler@full.host.name"
STARTD_ATTRS = $(STARTD_ATTRS), DedicatedScheduler
\end{verbatim}

Substitute the host name of the dedicated scheduler
machine for the string "\verb@full.host.name@". 

If running personal Condor, the name of the scheduler includes
the user name it was started as, so the configuration appears as:

\begin{verbatim}
DedicatedScheduler = "DedicatedScheduler@username@full.host.name"
STARTD_ATTRS = $(STARTD_ATTRS), DedicatedScheduler
\end{verbatim}

All dedicated resources must have policy expressions which allow for
jobs to always run, but not be preempted.
The resource must also be configured to prefer jobs from the dedicated 
scheduler over all other jobs.
Therefore, configuration gives
the dedicated scheduler of choice the highest rank.
It is worth noting that Condor puts no other requirements on a
resource for it to be considered dedicated.  

Job ClassAds from the dedicated scheduler 
contain the attribute \Attr{Scheduler}.
The attribute is defined by a string of the form 
\begin{verbatim}
Scheduler = "DedicatedScheduler@full.host.name"
\end{verbatim}
The host name of the dedicated scheduler
substitutes for the string \verb@full.host.name@. 

Different resources in the pool may have different dedicated policies
by varying the local configuration.

\begin{description}
\item[Policy Scenario: Machine Runs Only Jobs That Require Dedicated Resources]
%%%%%%%%%%%%%%%%%%%%%%%%%%%

One possible scenario for the use of a dedicated resource
is to only run jobs that require the dedicated resource.
To enact this policy, the configure with the following expressions:

\begin{verbatim}
START     = Scheduler =?= $(DedicatedScheduler)
SUSPEND   = False
CONTINUE  = True
PREEMPT   = False
KILL      = False
WANT_SUSPEND   = False
WANT_VACATE    = False
RANK      = Scheduler =?= $(DedicatedScheduler)
\end{verbatim}

The \Macro{START} expression specifies that a job with the \Attr{Scheduler}
attribute must match the string corresponding
\Attr{DedicatedScheduler} attribute in the machine ClassAd.
The \Macro{RANK} expression specifies that this same job 
(with the \Attr{Scheduler} attribute)
has the highest rank.
This prevents other jobs from preempting it based on user priorities.
The rest of the expressions disable all of the \Condor{startd} daemon's
regular policies for evicting jobs when keyboard and CPU activity is
discovered on the machine.


\item[Policy Scenario: Run Both Jobs That Do and Do Not Require Dedicated Resources]
%%%%%%%%%%%%%%%%%%%%%%%%%%%

While the first example works nicely for jobs requiring
dedicated resources,
it can 
lead to poor utilization of the dedicated machines.  
A more sophisticated strategy allows 
the machines to run other jobs, when no jobs that
require dedicated resources exist.
The machine is
configured to prefer jobs that require dedicated resources,
but not prevent others from running.

To implement this,
configure the machine as a dedicated resource (as above)
modifying only the \MacroNI{START} expression:

\begin{verbatim}
START = True
\end{verbatim}

\item[Policy Scenario: Adding Desk-Top Resources To The Mix]
%%%%%%%%%%%%%%%%%%%%%%%%%%%

A third policy example allows all jobs.
These desk-top machines use a preexisting \MacroNI{START} expression that
takes the machine owner's usage into account for some jobs.
The machine does not preempt jobs that must run on dedicated
resources,
while it will preempt other jobs based on a previously set
policy.
So, the default pool policy is used for starting and
stopping jobs, while jobs that require a dedicated resource always start 
and are not preempted.

The \MacroNI{START}, \MacroNI{SUSPEND}, \MacroNI{PREEMPT}, and
\MacroNI{RANK} policies are set in the global configuration.
Locally, the configuration is modified to this hybrid policy
by adding a second case.

\begin{verbatim}
SUSPEND    = Scheduler =!= $(DedicatedScheduler) && ($(SUSPEND))
PREEMPT    = Scheduler =!= $(DedicatedScheduler) && ($(PREEMPT))
RANK_FACTOR    = 1000000
RANK   = (Scheduler =?= $(DedicatedScheduler) * $(RANK_FACTOR)) \
               + $(RANK)
START  = (Scheduler =?= $(DedicatedScheduler)) || ($(START))
\end{verbatim}

Define \Macro{RANK\_FACTOR} to be a
larger value than the maximum value possible for the existing rank expression.
\Macro{RANK} is just a floating point value, so there is no harm in
having a value that is very large. 


\item[Policy Scenario: Parallel Scheduling Groups]
%%%%%%%%%%%%%%%%%%%%%%%%%%%

In some parallel environments, machines are divided into groups, and
jobs should not cross groups of machines -- that is, all the nodes of a parallel
job should be allocated to machines within the same group.
The most common example is a pool of machines using infiniband switches.
Each switch
might connect 16 machines, and a pool might have 160 machines on 10 switches.
If the infiniband switches are not routed to each other, each job must run 
on machines connected to the same switch.  

The dedicated scheduler's 
parallel scheduling groups features supports jobs that must not 
cross group boundaries.
Define a group by having each machine within a group
set the configuration variable 
\Macro{ParallelSchedulingGroup} with a string that is a unique name for
the group.
The submit description file for a parallel universe job which
must not cross group boundaries contains 
\begin{verbatim}
+WantParallelSchedulingGroups = True
\end{verbatim}

The dedicated scheduler enforces the allocation to within a group.
\end{description}

%%%%%%%%%%%%%%%%%%%%%%%%%%%%%%%%%%%%%%%%%%%%%%%%%%%%%%%%%%%%%%%%%%%%%%
\subsubsection{\label{sec:Configure-Dedicated-Preemption}
Preemption with Dedicated Jobs}
%%%%%%%%%%%%%%%%%%%%%%%%%%%%%%%%%%%%%%%%%%%%%%%%%%%%%%%%%%%%%%%%%%%%%%

The dedicated scheduler can optionally preempt running MPI jobs in
favor of higher priority MPI jobs in its queue.  Note that this is
different from preemption in non-parallel universes, and MPI jobs cannot
be preempted either by a machine's user pressing a key or by other means.

By default, the dedicated scheduler will never preempt running MPI jobs.
Two configuration file items control dedicated preemption: 
\Macro{SCHEDD\_PREEMPTION\_REQUIREMENTS} and \Macro{SCHEDD\_PREEMPTION\_RANK}.
These have no default value, so if either are not defined, preemption will
never occur. \MacroNI{SCHEDD\_PREEMPTION\_REQUIREMENTS} must evaluate to 
\Expr{True}
for a machine to be a candidate for this kind of preemption.
If more machines are 
candidates for preemption than needed to satisfy a higher priority job, the
machines are sorted by \MacroNI{SCHEDD\_PREEMPTION\_RANK}, and
only the highest ranked machines are taken.

Note that preempting one node of a running MPI job requires killing
the entire job on all of its nodes.  So, when preemption happens, it
may end up freeing more machines than strictly speaking are needed.
Also, as Condor cannot produce checkpoints for MPI jobs,
preempted jobs will be re-run, starting again from the beginning.
Thus, the administrator should be careful when
enabling dedicated preemption.  The following example shows how to
enable dedicated preemption.

\begin{verbatim}
STARTD_JOB_EXPRS = JobPrio
SCHEDD_PREEMPTION_REQUIREMENTS = (My.JobPrio < Target.JobPrio)
SCHEDD_PREEMPTION_RANK = 0.0
\end{verbatim}

In this case, preemption is enabled by the user job priority. If a set
of machines is running a job at user priority 5, and the user submits
a new job at user priority 10, the running job will be preempted for
the new job.  The old job is put back in the queue, and will begin again
from the beginning when assigned to a new set of machines.

%%%%%%%%%%%%%%%%%%%%%%%%%%%%%%%%%%%%%%%%%%%%%%%%%%%%%%%%%%%%%%%%%%%%%%
\subsubsection{\label{sec:Configure-Dedicated-Groups}
Grouping dedicated nodes into parallel scheduling groups}
%%%%%%%%%%%%%%%%%%%%%%%%%%%%%%%%%%%%%%%%%%%%%%%%%%%%%%%%%%%%%%%%%%%%%%

In some parallel environments, machines are divided into groups, and
jobs should not cross groups of machines -- that is, all the nodes of a parallel
job should be allocated to machines in the same group.  The most common
example is a pool of machine using infiniband switches.  Each switch
might connect 16 machines, and a pool might have 160 machines on 10 switches.
If the infiniband switches are not routed to each other, each job must run 
on machines connected to the same switch.  The dedicated scheduler's 
parallel scheduling groups features supports this operation.

Each \Condor{startd} must define which group it belongs to by setting the 
\Macro{ParallelSchedulingGroup} variable in the configuration file, and 
advertising it into the machine ClassAd.  
The value of this variable is a string,
which should be the same for all \Condor{startd} daemons in a given group.
The property must be advertised in the \Condor{startd} ClassAd
by appending \Macro{ParallelSchedulingGroup}
to the \Macro{STARTD\_ATTRS} configuration variable.
Then, parallel jobs which want to be scheduled by group declare this
by setting \Expr{+WantParallelSchedulingGroups = True}
in their submit description file. 
