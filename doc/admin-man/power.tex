%%%%%%%%%%%%%%%%%%%%%%%%%%%%%%%%%%%%%%%%%%%%%%%%%%%%%%%%%%%%%%%%%%%%%%
\section{\label{sec:power-man}Power Management}
%%%%%%%%%%%%%%%%%%%%%%%%%%%%%%%%%%%%%%%%%%%%%%%%%%%%%%%%%%%%%%%%%%%%%%
\index{power management|(}
\index{green computing|(}
\index{offline machine}

Condor supports placing machines in low power states.
A machine in the low power state is identified as being offline.
Power setting decisions are based upon  
Condor configuration.

Power conservation is relevant when machines are not in heavy use,
or when there are known periods of low activity within the pool.

%%%%%%%%%%%%%%%%%%%%%%%%%%%%%%%%%%%%%%%
\subsection{Entering a Low Power State}
%%%%%%%%%%%%%%%%%%%%%%%%%%%%%%%%%%%%%%%
\index{power management!entering a low power state}

By default, Condor does not do power management.
When desired, the ability to place a machine into a low
power state is accomplished through configuration.
This occurs when all slots on a machine agree that a low power state
is desired.

A slot's readiness to hibernate is determined by the 
evaluating the \Macro{HIBERNATE} configuration variable
(see section~\ref{param:Hibernate} on page~\pageref{param:Hibernate})
within the context of the slot.
Readiness is evaluated at fixed intervals, 
as determined by the \Macro{HIBERNATE\_CHECK\_INTERVAL} configuration variable.
A non-zero value of this variable enables the power management facility.
It is an integer value representing seconds,
and it need not be a small value.
There is a trade off between the extra time not at a low power state
and the unnecessary computation of readiness.  

To put the machine in a low power state rapidly
after it has become idle, consider checking each slot's state frequently,
as in the example configuration:

\begin{verbatim}
HIBERNATE_CHECK_INTERVAL = 20
\end{verbatim}

This checks each slot's readiness every 20 seconds.
A more common value for frequency of checks is 300 (5 minutes).
A value of 300 loses some degree of granularity,
but it is more reasonable as machines are likely to be put 
in to a low power state after a few hours, rather than minutes.
 
A slot's readiness or willingness to enter a low power state is 
determined by the \MacroNI{HIBERNATE} expression. 
Because this expression is evaluated in the context of each slot,
and not on the machine as a whole, 
any one slot can veto a change of power state.  
The \MacroNI{HIBERNATE} expression may reference a wide array of variables.
Possibilities include the change in power state if 
none of the slots are claimed, or if the slots are not in the
Owner state.

Here is a concrete example.
Assume that the \MacroNI{START} expression is not set to always be \Expr{True}.
This permits an easy determination whether or not
the machine is in an Unclaimed state through the use of
an auxiliary macro called \MacroNI{ShouldHibernate}.

\footnotesize
\begin{verbatim}
TimeToWait 	= (2 * $(HOUR))
ShouldHibernate = ( (KeyboardIdle > $(StartIdleTime)) \
                    && $(CPUIdle) \
                    && ($(StateTimer) > $(TimeToWait)) )
\end{verbatim}
\normalsize

This macro evaluates to \Expr{True} if the following are all \Expr{True}:
\begin{itemize}
\item The keyboard has been idle long enough.
\item The CPU is idle.
\item The slot has been Unclaimed for more than 2 hours.
\end{itemize}

The sample \MacroNI{HIBERNATE} expression 
that enters the power state called \verb@"RAM"@, if
\MacroNI{ShouldHibernate} evaluates to \Expr{True},
and remains in its current state otherwise is

\footnotesize
\begin{verbatim}
HibernateState 	= "RAM"
HIBERNATE = ifThenElse($(ShouldHibernate), $(HibernateState), "NONE" )
\end{verbatim}
\normalsize

If any slot returns \verb@"NONE"@,
that slot vetoes the decision to enter a low power state.
Only when values returned by all slots are all non-zero 
is there a decision to enter a low power state.
If all agree to enter the low power state, but differ in which state to enter,
then the largest magnitude value is chosen. 


%%%%%%%%%%%%%%%%%%%%%%%%%%%%%%%%%%%%%%%
\subsection{Returning From a Low Power State}
%%%%%%%%%%%%%%%%%%%%%%%%%%%%%%%%%%%%%%%
\index{power management!leaving a low power state}

The Condor command line tool \Condor{power} may wake a machine
from a low power state by 
sending a UDP Wake On LAN (WOL) packet.  See the \Condor{power} manual page on
page~\pageref{man-condor-power}.

\index{Condor daemon!condor\_rooster@\Condor{rooster}}
\index{daemon!condor\_rooster@\Condor{rooster}}
\index{condor\_rooster daemon}
To automatically call \Condor{power} under specific conditions,
\Condor{rooster} may be used.  The configuration options for
\Condor{rooster} are described in section~\ref{sec:Config-rooster}.

%%%%%%%%%%%%%%%%%%%%%%%%%%%%%%%%%%%%%%%
\subsection{Keeping a ClassAd for a Hibernating Machine}
%%%%%%%%%%%%%%%%%%%%%%%%%%%%%%%%%%%%%%% 

A pool's \Condor{collector} daemon can be configured to keep a 
persistent ClassAd entry for each machine, once it has entered hibernation.
This is required by \Condor{rooster} so that it can evaluate the
\Macro{UNHIBERNATE} expression of the offline machines.

To do this, define a log file using the \Macro{OFFLINE\_LOG}
configuration variable.
See section~\ref{param:OfflineLog} on
page~\pageref{param:OfflineLog} for the definition.
An optional expiration time for each ClassAd can
be specified with \Macro{OFFLINE\_EXPIRE\_ADS\_AFTER}.
The timing begins from the time the hibernating machine's ClassAd enters
the \Condor{collector} daemon.
See section~\ref{param:OfflineExpireAdsAfter} on
page~\pageref{param:OfflineExpireAdsAfter} for the definition.

%%%%%%%%%%%%%%%%%%%%%%%%%%%%%%%%%%%%%%%
\subsection{Linux Platform Details}
%%%%%%%%%%%%%%%%%%%%%%%%%%%%%%%%%%%%%%%
\index{power management!Linux platform details}

Depending on the Linux distribution and version,
there are three 
methods for controlling a machine's power state.
The methods:
\begin{enumerate}
\item \Prog{pm-utils} is a set of command line tools which can be used to
  detect and switch power states.
  In Condor, this is defined by the string \verb@"pm-utils"@.
\item The directory in the virtual file system \File{/sys/power} 
  contains virtual files that can be used to detect and set the power states.
  In Condor, this is defined by the string \verb@"/sys"@.
\item The directory in the virtual file system \File{/proc/acpi} 
  contains virtual files that can be used to detect and set the power states.
  In Condor, this is defined by the string \verb@"/proc"@.
\end{enumerate}

By default, the Condor attempts to detect the method
to use in the order shown.  The first method detected as usable
on the system is chosen.

This ordered detection may be bypassed,
to use a specified method instead by setting the configuration
variable \MacroNI{LINUX\_HIBERNATION\_METHOD} with
one of the defined strings.
This variable is defined in 
section~\ref{param:LinuxHibernationMethod} on
page~\pageref{param:LinuxHibernationMethod}.
If no usable methods are detected or the method specified by
\MacroNI{LINUX\_HIBERNATION\_METHOD} is either not detected or invalid,
hibernation is disabled.

The details of this selection process, and the final method selected
can be logged via enabling \MacroNI{D\_FULLDEBUG} in the relevant
subsystem's log configuration.


%%%%%%%%%%%%%%%%%%%%%%%%%%%%%%%%%%%%%%%
\subsection{Windows Platform Details}
%%%%%%%%%%%%%%%%%%%%%%%%%%%%%%%%%%%%%%%
\index{power management!Windows platform troubleshooting}

If after a suitable amount of time,
a Windows machine has not entered the expected power state,
then Condor is having difficulty exercising the operating system's
low power capabilities.  
While the cause will be specific to the machine's hardware,
it may also be due to improperly configured software.  
For hardware difficulties,
the likely culprit is the configuration within the machine's BIOS,
for which Condor can offer little guidance.
For operating system difficulties,
the Vista \Prog{powercfg} tool can be used to discover the available 
power states on the machine.
The following command demonstrates how to
list all of the supported power states of the machine:

\begin{verbatim}
> powercfg -A
The following sleep states are available on this system: 
Standby (S3) Hibernate Hybrid Sleep
The following sleep states are not available on this system:
Standby (S1)
        The system firmware does not support this standby state.
Standby (S2)
        The system firmware does not support this standby state.
\end{verbatim}

Note that the \MacroNI{HIBERNATE} expression is written in terms of the 
S$n$ state, where $n$ is the value evaluated from the expression.

This tool can also be used to enable and disable other sleep states.
This example turns hibernation on.

\begin{verbatim}
> powercfg -h on
\end{verbatim}

If this tool is insufficient for configuring the machine in the manner required,
the \Prog{Power Options} control panel application offers
the full extent of the machine's power management abilities.
Windows 2000 and XP lack the \Prog{powercfg} program,
so all configuration must be done via the \Prog{Power Options}
control panel application.

\index{green computing|)}
\index{power management|)}
