%%%%%%%%%%%%%%%%%%%%%%%%%%%%%%%%%%%%%%%%%%%%%%%%%%%%%%%%%%%%%%%%%%%%%%
\section{\label{sec:power-man}Power Management}
%%%%%%%%%%%%%%%%%%%%%%%%%%%%%%%%%%%%%%%%%%%%%%%%%%%%%%%%%%%%%%%%%%%%%%

\index{power managment|(}
\index{installation!power managment}
\MoreTodo

On the Windows platform,
Condor can be configured to place a machine in a low power state.  
The details of the machine's power capabilities are in the
configuration. 

This section discusses a starting point for users interested in 
experimenting with power conservation within their Condor pool.
Power conservation is relevant when machines are not in heavy use,
or when there are known periods of low activity within the pool.

%%%%%%%%%%%%%%%%%%%%%%%%%%%%%%%%%%%%%%%
\subsection{Configuration to Enter a Low Power State}
%%%%%%%%%%%%%%%%%%%%%%%%%%%%%%%%%%%%%%%

By default, Condor's power management is turned off.
To enable it is a matter
of configuring the time interval that determines when slots should be queried
for their readiness to hibernate.  A slot's readiness is determined by the 
evaluating the \Macro{HIBERNATE} expression in the context of the slot.  We 
will return to this point shortly.

To start power management set the \Macro{HIBERNATE\_CHECK\_INTERVAL} to a
non-zero value.  This value is interpreted in seconds, so it need not 
be a small number: it is a trade off between accuracy of measurement and 
unnecessary computation.  

For instance, if we want to put the machine in a low power state very rapidly
after it has become idle, we may want to check the each slot's state quite
frequently:

\begin{verbatim}
HIBERNATE_CHECK_INTERVAL = 20
\end{verbatim}

This would make Condor check each slot's readiness every 20 seconds.  This
may be to small a value in practice. A more common value would be:

\begin{verbatim}
HIBERNATE_CHECK_INTERVAL = 300
\end{verbatim}

This checks the slots every 5 minutes.  This setting, while losing some 
degree of granularity, is more reasonable as machine are likely to be put 
in to a low power state after a few hours, rather than minutes.
 
A slot's readiness|or \textit{willingness}|to enter a low power state is 
determined by the \MacroNI{HIBERNATE} expression.  As was mentioned above, this 
expression is evaluated in the context of each slot on the machine, and not 
on the machine itself; meaning that any one slot can veto a change of power 
state.  

Because \MacroNI{HIBERNATE} is an expression, it can be made to include a wide
array of variables.  A change in power state might occur, for instance, if 
none of the slots are claimed, or if the slots are not in the ``Owner'' state.
These are just a few of the options available.

As an example, let's assume that the \Macro{START} expression is not set to
always be trivially be true.  This allows us to easily determine if
we are in an ``Unclaimed'' state.  The following expression evaluates
to True if:

\begin{itemize}
\item The keyboard has been idle long enough, and
\item the cpu is idle, and
\item the slot haves been ``Unclaimed'' for more than 2 hours:
\end{itemize}

\begin{verbatim}
ShouldHibernate = ((KeyboardIdle > $(StartIdleTime)) \
                        && $(CPUIdle) \
                               && ($(StateTimer) > (2 * $(HOUR)))
\end{verbatim}

Note that we have not defined the \MacroNI{HIBERNATE} expression here, only an 
auxiliary macro to help us with the actual definition.  Using this macro we 
can create a clean hibernation expression:

\begin{verbatim}
HIBERNATE = ifThenElse($(ShouldHibernate), 3, 0)
\end{verbatim} %$ -- just there to fix emacs' bad dollar parsing

The actual hibernation expression says: We should enter power state $3$,
i.e. sleep (or standby), if \Code{ShouldHibernate} evaluates to true; 
otherwise, enter power state $0$|which is a no-op: it tells Condor to leave 
the machine in it's current state. 

Here we see can clearly see how one slot can veto the decision of all the 
other slots.  If any slot returns $0$, Condor will not put the machine in 
to a low power state.  If the numbers returned by the slots are all non-zero, 
but differ, then the largest value is used as the representative power 
state, and Condor will attempt to put the machine in to this power state.

%%%%%%%%%%%%%%%%%%%%%%%%%%%%%%%%%%%%%%%
\subsection{Returning from a low power state}
%%%%%%%%%%%%%%%%%%%%%%%%%%%%%%%%%%%%%%%
\Todo

%%%%%%%%%%%%%%%%%%%%%%%%%%%%%%%%%%%%%%%
\subsection{Troubleshooting}
%%%%%%%%%%%%%%%%%%%%%%%%%%%%%%%%%%%%%%%

If after a suitable amount of time the machine has not entered the expected 
power state, then Condor is having difficulty exercising the OS's low power 
capabilities.  While the cause of this can be very specific to the machine's
hardware, it may also be due to improperly configured software.  
With regards to hardware, most likely the configuration needs to be dealt with
in the machine's BIOS, which we can offer little guidance, due to the 
large variety of hardware out there.  In the case of the OS, we can offer
a few hints:

On Vista the \Prog{powercfg} tool can be used to discover the available 
power states on the machine.  The following example illustrates how to
list all of the supported power states of the machine:

\begin{verbatim}
> powercfg -A
The following sleep states are available on this system: 
Standby (S3) Hibernate Hybrid Sleep
The following sleep states are not available on this system:
Standby (S1)
        The system firmware does not support this standby state.
Standby (S2)
        The system firmware does not support this standby state.
\end{verbatim}

% TODO(?): maybe move this and re-work it into the body of the 
% admin-doc proper or into the HIBERNATE expression's description
Note that the \MacroNI{HIBERNATE} expression is written in terms of the 
S$n$ state, where $n$ is the number returned by evaluating expression.

This tool can also be used to enable and disable other sleep states.  As 
an example, the following will turn hibernation on:

\begin{verbatim}
powercfg -h on
\end{verbatim}

If this tool is insufficient for configuring the machine in the manner that 
is required, the \textit{Power Options} control panel application offers
the full extent of the machine's power management abilities.

On Windows 2000/XP, unfortunately, lacks the \Prog{powercfg} so all 
configuration must be done via the \textit{Power Options} control panel 
application.

\index{power managment|)}
