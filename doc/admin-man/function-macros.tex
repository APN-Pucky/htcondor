%% IMPORTANT:
%% This file contains prose to describe the semantics and syntax
%% of function macros that can be used both in configuration and
%% in a submit description file.
%%
%% Please be careful when changing this prose, as it is included
%% in 2 different parts of the manual.  Do not include examples
%% specific to either a submit description file usage or a
%% configuration usage in this prose.

\MoreTodo

A set of predefined functions increase flexibility.
Both submit description files and configuration files are read using
the same parser,
so these functions may be used in both submit description files and
configuration files.

Case is significant in the function's name, so use the
same letter case as given in these definitions.

\begin{description}

\item [\Expr{\$CHOICE()}]

\item [\Expr{\$ENV()}]
References to the HTCondor process's environment are allowed in the
configuration files.
Environment references use the \Macro{ENV} macro and are of the form:
\begin{verbatim}
  $ENV(environment_variable_name)
\end{verbatim}
For example, 
\begin{verbatim}
  A = $ENV(HOME)
\end{verbatim}
binds \MacroNI{A} to the value of the \Env{HOME} environment variable.
Environment references are not currently used in standard HTCondor
configurations.
However, they can sometimes be useful in custom configurations.

\item [\Expr{\$F[pdnxq]()}]

\item [\Expr{\$INT()}]

\item [\Expr{\$RANDOM\_CHOICE(choice1, choice2, choice3, \Dots)}]
\index{\$RANDOM\_CHOICE() function macro}
A random choice of one of the parameters in the list of parameters is made.
For example,
if one of the integers 0-8 (inclusive) should be randomly
chosen:
\begin{verbatim}
  $RANDOM_CHOICE(0,1,2,3,4,5,6,7,8)
\end{verbatim}

\item [\Expr{\$RANDOM\_INTEGER(min, max [, step])}]
\index{\$RANDOM\_INTEGER()!in configuration}
A random integer within the range \verb@min@ and \verb@max@, inclusive,
is selected.
The optional \verb@step@ parameter
controls the stride within the range, and it defaults to the value 1.
For example, to randomly chose an even integer in the range 0-8 (inclusive):
\begin{verbatim}
  $RANDOM_INTEGER(0, 8, 2)
\end{verbatim}

\item [\Expr{\$REAL()}]

\item [\Expr{\$SUBSTR()}]

\end{description}
