%% IMPORTANT:
%% This file contains prose to describe the semantics and syntax
%% of function macros that can be used both in configuration and
%% in a submit description file.
%%
%% Please be careful when changing this prose, as it is included
%% in 2 different parts of the manual.  Do not include examples
%% specific to either a submit description file usage or a
%% configuration usage in this prose.

\MoreTodo

A set of predefined functions increase flexibility.
Both submit description files and configuration files are read using
the same parser,
so these functions may be used in both submit description files and
configuration files.

Case is significant in the function's name, so use the
same letter case as given in these definitions.

\begin{description}

\item [\Expr{\$CHOICE(index, listname)} 
       or \Expr{\$CHOICE(index, item1, item2, \Dots)}]
An item within the list is returned.
The list is represented by a parameter name,
or the list items are the parameters.
The \Expr{index} parameter determines which item.
The first item in the list is at index 0.
If the index is out of bounds for the list contents, 
an error occurs.


\item [\Expr{\$ENV(environment\_variable\_name)}]
For example, 
\begin{verbatim}
  A = $ENV(HOME)
\end{verbatim}
binds \MacroNI{A} to the value of the \Env{HOME} environment variable.

\item [\Expr{\$F[pdnxq](filename)}]
One or more of the lower case letters may be combined to form the
function name and thus, its functionality.
Each letter operates on the \Expr{filename} in its own way.
\begin{itemize}
  \item \Expr{p} refers to the entire directory portion of \Expr{filename},
with a trailing slash or backslash character.
Whether a slash or backslash is used depends on the platform of
the machine.
The slash will be recognized on Linux platforms;
either a slash or backslash will be recognized on Windows platforms,
and the parser will use the same character specified.
  \item \Expr{d} refers to the last portion of the directory within the path,
if specified.
It will have a trailing slash or backslash, 
as appropriate to the platform of the machine.
The slash will be recognized on Linux platforms;
either a slash or backslash will be recognized on Windows platforms,
and the parser will use the same character specified.
  \item \Expr{n} refers to the file name at the end of any path,
but without any file name extension.
As an  example, the return value from \Expr{\$Fn(/tmp/simulate.exe)}
will be \Expr{simulate} (without the \File{.exe} extension).
  \item \Expr{x} refers to a file name extension, but without the associated
period (\Expr{.}).
As an  example, the return value from \Expr{\$Fn(/tmp/simulate.exe)}
will be \Expr{exe}.
  \item \Expr{q} causes the return value to be enclosed within double
quote marks.
\end{itemize}


\item [\Expr{\$INT(item-to-convert)}
       or \Expr{\$INT(item-to-convert, conversion-character)}]
Expands, evaluates, and returns a string version of 
\Expr{item-to-convert}.
The \Expr{conversion-character} is a C language conversion character.
If no \Expr{conversion-character} is specified, \verb@"%d"@ is used
as a conversion character.

\item [\Expr{\$RANDOM\_CHOICE(choice1, choice2, choice3, \Dots)}]
\index{\$RANDOM\_CHOICE() function macro}
A random choice of one of the parameters in the list of parameters is made.
For example,
if one of the integers 0-8 (inclusive) should be randomly
chosen:
\begin{verbatim}
  $RANDOM_CHOICE(0,1,2,3,4,5,6,7,8)
\end{verbatim}

\item [\Expr{\$RANDOM\_INTEGER(min, max [, step])}]
\index{\$RANDOM\_INTEGER()!in configuration}
A random integer within the range \verb@min@ and \verb@max@, inclusive,
is selected.
The optional \verb@step@ parameter
controls the stride within the range, and it defaults to the value 1.
For example, to randomly chose an even integer in the range 0-8 (inclusive):
\begin{verbatim}
  $RANDOM_INTEGER(0, 8, 2)
\end{verbatim}

\item [\Expr{\$REAL(item-to-convert)}
       or \Expr{\$REAL(item-to-convert, conversion-character)}]
Expands, evaluates, and returns a string version of 
\Expr{item-to-convert} for a floating point type.
The \Expr{conversion-character} is a C language conversion character.
If no \Expr{conversion-character} is specified, \verb@"%16G"@ is used
as a conversion character.

\item [\Expr{\$SUBSTR()}]

\end{description}
