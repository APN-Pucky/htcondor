%%%%%%%%%%%%%%%%%%%%%%%%%%%%%%%%%%%%%%%%%%%%%%%%%%%%%%%%%%%%%%%%%%%%%%%%%%%%%%%%
\subsection{\label{sec:Dynamic-Deployment}Dynamic Deployment}
%%%%%%%%%%%%%%%%%%%%%%%%%%%%%%%%%%%%%%%%%%%%%%%%%%%%%%%%%%%%%%%%%%%%%%%%%%%%%%%%
\index{dynamic deployment}
\index{deployment commands}

Dynamic deployment is a mechanism that allows rapid, automated
installation and start up of Condor resources on a given machine.
In this way any machine can be added to a Condor pool.
The dynamic
deployment tool set also provides tools to remove a machine from the
pool, without leaving residual effects on the machine such as leftover
installations, log files, and working directories.

\index{Condor commands!condor\_cold\_start}
Installation and start up is provided by \Condor{cold\_start}.
The \Condor{cold\_start} program determines the operating system and
architecture of the target machine, and transfers the correct
installation package from an ftp, http, or grid ftp site.
After transfer, it
installs Condor and creates a local working
directory for Condor to run in.  As a last step, \Condor{cold\_start}
begins running Condor in a manner which allows for later easy and reliable
shut down.

\index{Condor commands!condor\_cold\_stop}
The program that reliably shuts down and uninstalls a previously
dynamically installed Condor instance is \Condor{cold\_stop}.
\Condor{cold\_stop} begins by safely and reliably shutting off the
running Condor installation.  It ensures that Condor has
completely shut down before continuing, and optionally ensures that
there are no queued jobs at the site.
Next, \Condor{cold\_stop}
removes and optionally archives the Condor working directories,
including the \File{log} directory. 
These archives can be stored to a
mounted file system or to a grid ftp site.
As a last step,
\Condor{cold\_stop} uninstalls the Condor executables and libraries.
The end result is that the machine resources are left unchanged after
a dynamic deployment of Condor leaves.

%%%%%%%%%%%%%%%%%%%%%%%%%%%%%%%%%%%%%%%%%%%%%%%%%%%%%%%%%%%%%%%%%%%%%%%%%%%%%%%%
\subsubsection{Configuration and Usage}
%%%%%%%%%%%%%%%%%%%%%%%%%%%%%%%%%%%%%%%%%%%%%%%%%%%%%%%%%%%%%%%%%%%%%%%%%%%%%%%%

\index{dynamic deployment!configuration}
Dynamic deployment is designed for the expert Condor user
and administrator.
Tool design choices were made for functionality,
not ease-of-use.

Like every installation of Condor, a dynamically deployed installation
relies on a configuration.
To add a target
machine to a previously created Condor pool,
the global configuration file for that pool is a good starting point.
Modifications to that configuration can be made in a separate, 
local configuration file used in the dynamic deployment.
The global configuration file must
be placed on an ftp, http, grid ftp, or file server 
accessible by \Condor{cold\_start}.  The local configuration file
is to be on a file system accessible by the target machine.
There are some specific configuration variables that may be set for
dynamic deployment.  
A list of executables and directories which must be present
for Condor to start on the target machine may be set with
the configuration variables \Macro{DEPLOYMENT\_REQUIRED\_EXECS} and
\Macro{DEPLOYMENT\_REQUIRED\_DIRS}. 
If defined and the comma-separated list of executables or directories are
not present, then \Condor{cold\_start} exits with error.
Note this does not affect what is installed, only
whether start up is successful. 

A list of executables and directories which are recommended to be present
for Condor to start on the target machine may be set with
the configuration variables \Macro{DEPLOYMENT\_RECOMMENDED\_EXECS} and
\Macro{DEPLOYMENT\_RECOMMENDED\_DIRS}. 
If defined and the comma-separated lists of executables or directories are
not present, then \Condor{cold\_start} prints a warning message
and continues.
Here is a portion of the configuration relevant to
a dynamic deployment of a Condor submit node:

\footnotesize
\begin{verbatim}
DEPLOYMENT_REQUIRED_EXECS    = MASTER, SCHEDD, PREEN, STARTER, \
                               STARTER_STANDARD, SHADOW, \
                               SHADOW_STANDARD, GRIDMANAGER, GAHP, CONDOR_GAHP
DEPLOYMENT_REQUIRED_DIRS     = SPOOL, LOG, EXECUTE
DEPLOYMENT_RECOMMENDED_EXECS = CREDD
DEPLOYMENT_RECOMMENDED_DIRS  = LIB, LIBEXEC
\end{verbatim}
\normalsize

Additionally, the user must
specify which Condor services will be started.  This is done through
the \MacroNI{DAEMON\_LIST} configuration variable.  Another excerpt
from a dynamic submit node deployment configuration:

\footnotesize
\begin{verbatim}
DAEMON_LIST  = MASTER, SCHEDD
\end{verbatim}
\normalsize

Finally, the location
of the dynamically installed Condor executables is tricky to set,
since the location is unknown before installation.
Therefore,
the variable \Macro{DEPLOYMENT\_RELEASE\_DIR} is defined in the environment.
It corresponds to the location of the dynamic Condor installation.
If, as is often the case, 
the configuration file specifies the location of Condor executables in
relation to the \MacroNI{RELEASE\_DIR} variable, the configuration can
be made dynamically deployable by setting \MacroNI{RELEASE\_DIR} to
\MacroNI{DEPLOYMENT\_RELEASE\_DIR} as 

\footnotesize
\begin{verbatim}
RELEASE_DIR = $(DEPLOYMENT_RELEASE_DIR)
\end{verbatim}
\normalsize

In addition to setting up the configuration, the user must also
determine where the installation package will reside.
The installation package can be in either tar or 
gzipped tar form, and may
reside on a ftp, http, grid ftp, or file server.  
Create this installation package by tar'ing up the binaries and libraries
needed, and place them on the appropriate server.
The binaries can be tar'ed in a flat structure or within \File{bin} and
\File{sbin}.  Here is a list of files to give an example
structure for a dynamic deployment of the \Condor{schedd} daemon.

\footnotesize
\begin{verbatim}
% tar tfz latest-i686-Linux-2.4.21-37.ELsmp.tar.gz
bin/
bin/condor_config_val
bin/condor_q
sbin/
sbin/condor_preen
sbin/condor_shadow.std
sbin/condor_starter.std
sbin/condor_schedd
sbin/condor_master
sbin/condor_gridmanager
sbin/gt4_gahp
sbin/gahp_server
sbin/condor_starter
sbin/condor_shadow
sbin/condor_c-gahp
sbin/condor_off 
\end{verbatim}
\normalsize
