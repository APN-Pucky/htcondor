%%%%%%%%%%%%%%%%%%%%%%%%%%%%%%%%%%%%%%%%%%%%%%%%%%%%%%%%%%%%%%%%%%%%%%
\subsection{Installing a Checkpoint Server}
\label{sec:Ckpt-Server}
%%%%%%%%%%%%%%%%%%%%%%%%%%%%%%%%%%%%%%%%%%%%%%%%%%%%%%%%%%%%%%%%%%%%%%

The Checkpoint Server is a daemon that can be installed on a server to
handle all of the checkpoints that a Condor pool will create.
This machine should have a large amount of disk space available, and
should have a fast connection to your machines.

\Note It is a good idea to pick a very stable machine for your checkpoint
server.
If the checkpoint server crashes, the Condor system will continue to
operate.
While the Condor system will recover from a checkpoint server crash
as best it can, there are two problems that can (and will) occur.

The first problem is that if the checkpoint server is not functioning,
when jobs need to checkpoint, they will go back to the local schedd's.
While this is desirable, the problem is that with a checkpoint server
people may not have enough local disk space to store the checkpoint
files.
Hence, when the checkpoint files get transmitted back, they will not
be able to be kept because the disk is full.

The second problem is that when the jobs wish to start, if they cannot
be retrieved from the checkpoint server, they will be retrieved from the
local schedd.
If there is no checkpoint file on the local schedd, the job will be reloaded
from scratch.
It is advisable to shut your pool down before bringing your checkpoint
server down.

To install a checkpoint server, copy the file \File{condor\_config.local.checkpoint.server}
to your machines local configuration file (e.g. \File{froth.local}).
There is one setting that you must change, and that is \Macro{CKPT\_SERVER\_DIR}.
The \Macro{CKPT\_SERVER\_DIR} defines where your checkpoint files should be located.
It is better if this is on a very fast local file system (preferably a RAID).
The speed of this file system will have a direct impact on the speed at which your
checkpoint files can be retrieved from the remote machines.

The other optional settings are:
\begin{description}

\item[\Macro{DAEMON\_LIST}] As described in section \ref{sec:Master-Config-File-Entries},
the DAEMON\_LIST entry must have MASTER and CKPT\_SERVER.
Add STARTD if you want to allow jobs to run on your checkpoint server.
Similarly, add SCHEDD if you would like to submit jobs from your checkpoint server.

\item[\Macro{CKPT\_SERVER\_LOG}] The CKPT\_SERVER\_LOG is where the
checkpoint server log gets put.

\item[\Macro{MAX\_CKPT\_SERVER\_LOG}] Use this item to configure the
size of the checkpoint server log before it is rotated.

\item[\Macro{CKPT\_SERVER\_DEBUG}] The amount of information you would like printed
in your logfile.
Currently, the only debug level supported is
\Dflag{ALWAYS}.

\end{description}
