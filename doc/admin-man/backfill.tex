%%%%%%%%%%%%%%%%%%%%%%%%%%%%%%%%%%%%%%%%%%%%%%%%%%%%%%%%%%%%%%%%%%%%%%%%%%%
\subsection{\label{sec:Backfill}Configuring HTCondor for Running Backfill Jobs} 
%%%%%%%%%%%%%%%%%%%%%%%%%%%%%%%%%%%%%%%%%%%%%%%%%%%%%%%%%%%%%%%%%%%%%%%%%%

\index{Backfill}

HTCondor can be configured to run
backfill jobs whenever the \Condor{startd} has no other work to
perform.
These jobs are considered the lowest possible priority, but when
machines would otherwise be idle, the resources can be put to good 
use.

Currently, HTCondor only supports using the Berkeley Open Infrastructure
for Network Computing (BOINC) to provide the backfill jobs.
More information about BOINC is available at
\URL{http://boinc.berkeley.edu}.

The rest of this section provides an overview of how backfill jobs
work in HTCondor, details for configuring the policy for when backfill
jobs are started or killed, and details on how to configure HTCondor to
spawn the BOINC client to perform the work.


%%%%%%%%%%%%%%%%%%%%%%%%%%%%%%%%%%%%%%%%%%%%%%%%%%%%%%%%%%%%%%%%%%%%%%
\subsubsection{\label{sec:Backfill-Overview}Overview of Backfill jobs
in HTCondor}
%%%%%%%%%%%%%%%%%%%%%%%%%%%%%%%%%%%%%%%%%%%%%%%%%%%%%%%%%%%%%%%%%%%%%%

\index{Backfill!Overview}

Whenever a resource controlled by HTCondor is in the Unclaimed/Idle
state, it is totally idle; neither the interactive user nor an HTCondor
job is performing any work.
Machines in this state can be configured to enter the \Term{Backfill}
state, which allows the resource to attempt a background
computation to keep itself busy until other work arrives (either a 
user returning to use the machine interactively, or a normal HTCondor
job).
Once a resource enters the Backfill state, the \Condor{startd} will
attempt to spawn another program, called a \Term{backfill client}, to
launch and manage the backfill computation.
When other work arrives, the \Condor{startd} will kill the backfill
client and clean up any processes it has spawned, freeing the machine
resources for the new, higher priority task.
More details about the different states an HTCondor resource can enter
and all of the possible transitions between them are described in
section~\ref{sec:Configuring-Policy} beginning on
page~\pageref{sec:Configuring-Policy}, especially
sections~\ref{sec:States}, \ref{sec:Activities}, and
\ref{sec:State-and-Activity-Transitions}.

At this point, the only backfill system supported by HTCondor is BOINC. 
The \Condor{startd} has the ability to start and stop the BOINC client
program at the appropriate times, but otherwise provides no additional
services to configure the BOINC computations themselves.
Future versions of HTCondor might provide additional functionality to
make it easier to manage BOINC computations from within HTCondor.
For now, the BOINC client must be manually installed and configured
outside of HTCondor on each backfill-enabled machine.


%%%%%%%%%%%%%%%%%%%%%%%%%%%%%%%%%%%%%%%%%%%%%%%%%%%%%%%%%%%%%%%%%%%%%%
\subsubsection{\label{sec:Backfill-Policy}Defining the Backfill Policy}
%%%%%%%%%%%%%%%%%%%%%%%%%%%%%%%%%%%%%%%%%%%%%%%%%%%%%%%%%%%%%%%%%%%%%%

\index{Backfill!Defining HTCondor policy}

There are a small set of policy expressions that determine if a
\Condor{startd} will attempt to spawn a backfill client at all, and if so,
to control the transitions in to and out of the Backfill state.
This section briefly lists these expressions.
More detail can be found in
section~\ref{sec:Startd-Config-File-Entries} on
page~\pageref{sec:Startd-Config-File-Entries}.

\begin{description}

\item[\Macro{ENABLE\_BACKFILL}] A boolean value to determine if any
  backfill functionality should be used.
  The default value is \Expr{False}.

\item[\Macro{BACKFILL\_SYSTEM}] A string that defines what backfill
  system to use for spawning and managing backfill computations.
  Currently, the only supported string is \AdStr{BOINC}.
  
\item[\Macro{START\_BACKFILL}] A boolean expression to control if an
  HTCondor resource should start a backfill client.
  This expression is only evaluated when the machine is in the Unclaimed/Idle
  state and the \MacroNI{ENABLE\_BACKFILL} expression is \Expr{True}.

\item[\Macro{EVICT\_BACKFILL}] A boolean expression that is evaluated
  whenever an HTCondor resource is in the Backfill state.
  A value of \Expr{True} indicates the machine should immediately kill the
  currently running backfill client and any other spawned processes,
  and return to the Owner state.

\end{description}

The following example shows a possible configuration to enable
backfill:

\footnotesize
\begin{verbatim}
# Turn on backfill functionality, and use BOINC
ENABLE_BACKFILL = TRUE
BACKFILL_SYSTEM = BOINC

# Spawn a backfill job if we've been Unclaimed for more than 5
# minutes 
START_BACKFILL = $(StateTimer) > (5 * $(MINUTE))

# Evict a backfill job if the machine is busy (based on keyboard
# activity or cpu load)
EVICT_BACKFILL = $(MachineBusy)
\end{verbatim}
\normalsize


%%%%%%%%%%%%%%%%%%%%%%%%%%%%%%%%%%%%%%%%%%%%%%%%%%%%%%%%%%%%%%%%%%%%%%
\subsubsection{\label{sec:Backfill-BOINC-overview}Overview of the
 BOINC system}
%%%%%%%%%%%%%%%%%%%%%%%%%%%%%%%%%%%%%%%%%%%%%%%%%%%%%%%%%%%%%%%%%%%%%%

\index{Backfill!BOINC Overview}

The BOINC system is a distributed computing environment for solving
large scale scientific problems.
A detailed explanation of this system is beyond the scope of this
manual.
Thorough documentation about BOINC is available at their website:
\URL{http://boinc.berkeley.edu}.
However, a brief overview is provided here for sites interested in
using BOINC with HTCondor to manage backfill jobs. 

BOINC grew out of the relatively famous SETI@home computation, where
volunteers installed special client software, in the form of a
screen saver, that contacted a centralized server to download work
units.
Each work unit contained a set of radio telescope data and the
computation tried to find patterns in the data, a sign of intelligent
life elsewhere in the universe, hence the name: "Search for Extra
Terrestrial Intelligence at home".
BOINC is developed by the Space Sciences Lab at the University of
California, Berkeley, by the same people who created SETI@home.
However, instead of being tied to the specific radio telescope
application, BOINC is a generic infrastructure by which many different
kinds of scientific computations can be solved.
The current generation of SETI@home now runs on top of BOINC, along
with various physics, biology, climatology, and other applications.

The basic computational model for BOINC and the original SETI@home is
the same: volunteers install BOINC client software,
called the \Prog{boinc\_client},
which runs whenever the machine would otherwise be idle.
However, the BOINC installation on any given machine must be
configured so that it knows what computations to work for
instead of always working on a hard coded computation.
The BOINC terminology for a computation is a \Term{project}.
A given BOINC client can be configured to donate all of its cycles to
a single project, or to split the cycles between projects so that, on
average, the desired percentage of the computational power is
allocated to each project.
Once the \Prog{boinc\_client} starts running, 
it attempts to contact a centralized server for
each project it has been configured to work for.
The BOINC software downloads the appropriate platform-specific
application binary and some work units from the central server for
each project.
Whenever the client software completes a given work unit, it once
again attempts to connect to that project's central server to upload
the results and download more work.

BOINC participants must register at the centralized server for each
project they wish to donate cycles to.
The process produces a unique identifier so that the work performed by
a given client can be credited to a specific user.
BOINC keeps track of the work units completed by each user, so that
users providing the most cycles get the highest rankings, 
and therefore, bragging rights.

Because BOINC already handles the problems of distributing the
application binaries for each scientific computation, the work units,
and compiling the results, it is a perfect system for managing
backfill computations in HTCondor.
Many of the applications that run on top of BOINC produce their own
application-specific checkpoints, so even if the
\Prog{boinc\_client} is killed, 
for example, when an HTCondor job arrives
at a machine, or if the interactive user returns,
an entire work unit will not necessarily be lost.


%%%%%%%%%%%%%%%%%%%%%%%%%%%%%%%%%%%%%%%%%%%%%%%%%%%%%%%%%%%%%%%%%%%%%%
\subsubsection{\label{sec:Backfill-BOINC-install}Installing the BOINC client
software}
%%%%%%%%%%%%%%%%%%%%%%%%%%%%%%%%%%%%%%%%%%%%%%%%%%%%%%%%%%%%%%%%%%%%%%

\index{Backfill!BOINC Installation}

In HTCondor \VersionNotice, 
the \Prog{boinc\_client} must be manually downloaded, 
installed and configured outside of HTCondor.
%Hopefully in future versions, the HTCondor package will include the
%\Prog{boinc\_client}, and there will be a way to automatically install
%and configure the BOINC software together with HTCondor.
Download the \Prog{boinc\_client} executables at 
\URL{http://boinc.berkeley.edu/download.php}.

Once the BOINC client software has been downloaded, the
\Prog{boinc\_client} binary should be placed in a location where the
HTCondor daemons can use it.
The path will be specified with the HTCondor configuration variable
\Macro{BOINC\_Executable}.

Additionally, a local directory on each machine should be created
where the BOINC system can write files it needs.
This directory must not be shared by multiple instances of the BOINC
software. This is the same restriction as placed on
the \File{spool} or \File{execute} directories used by HTCondor.
The location of this directory is defined by
\Macro{BOINC\_InitialDir}.
The directory must be writable by whatever user the
\Prog{boinc\_client} will run as.
This user is either the same as the user the HTCondor daemons are
running as, if HTCondor is not running as root, or a user defined via
the \Macro{BOINC\_Owner} configuration variable.

Finally, HTCondor administrators wishing to use BOINC for backfill jobs
must create accounts at the various BOINC projects they want to donate
cycles to.
The details of this process vary from project to project.
Beware that this step must be done manually, as the 
\Prog{boinc\_client} can not automatically
register a user at a given project, 
unlike the more fancy GUI version
of the BOINC client software which many users run as a screen saver. 
For example, to configure machines to perform work for the
Einstein@home project (a physics experiment run by the University of
Wisconsin at Milwaukee), HTCondor administrators should go to
\URL{http://einstein.phys.uwm.edu/create\_account\_form.php}, fill in
the web form, and generate a new Einstein@home identity.
This identity takes the form of a project URL (such as
http://einstein.phys.uwm.edu) followed by an \Term{account key},
which is a long string of letters and numbers that is used as a unique
identifier. 
This URL and account key will be needed when configuring HTCondor to use
BOINC for backfill computations.


%%%%%%%%%%%%%%%%%%%%%%%%%%%%%%%%%%%%%%%%%%%%%%%%%%%%%%%%%%%%%%%%%%%%%%
\subsubsection{\label{sec:Backfill-BOINC-HTCondor}Configuring the BOINC client
under HTCondor}
%%%%%%%%%%%%%%%%%%%%%%%%%%%%%%%%%%%%%%%%%%%%%%%%%%%%%%%%%%%%%%%%%%%%%%

\index{Backfill!BOINC Configuration in HTCondor}

After the \Prog{boinc\_client}
has been installed on a given machine, 
the BOINC projects to join have been selected, 
and a unique project account key has been created for each project,
the HTCondor configuration needs to be modified.

Whenever the \Condor{startd} decides to spawn the \Prog{boinc\_client}
to perform backfill computations,
it will spawn a \Condor{starter} to directly launch and monitor the
\Prog{boinc\_client} program.
This \Condor{starter} is just like the one used to invoke any other HTCondor
jobs.
In fact, the argv[0] of the \Prog{boinc\_client} will be renamed to
\Prog{condor\_exec}, as described in section~\ref{sec:renaming-argv} on 
page~\pageref{sec:renaming-argv}.

This \Condor{starter} reads
values out of the HTCondor configuration files to define the job it
should run, as opposed to getting these values from a job ClassAd
in the case of a normal HTCondor job.
All of the configuration variables names for variables to control things 
such as the path to
the \Prog{boinc\_client} binary to use, the command-line arguments,
and the initial working directory, are prefixed with the string
\AdStr{BOINC\_}.
Each of these variables is described as either a required or an
optional configuration variable. 

Required configuration variables:

\begin{description}

\item[\Macro{BOINC\_Executable}] \label{param:BoincExecutable} The
  full path and executable name of the \Prog{boinc\_client} binary to use.

\item[\Macro{BOINC\_InitialDir}] \label{param:BoincInitialDir} The
  full path to the local directory where BOINC should run.

\item[\Macro{BOINC\_Universe}] \label{param:BoincUniverse} The HTCondor
  universe used for running the \Prog{boinc\_client} program.
  This \emph{must} be set to \Expr{vanilla} for BOINC to work under
  HTCondor.

\item[\Macro{BOINC\_Owner}] \label{param:BoincOwner} What user the
  \Prog{boinc\_client} program should be run as.
  This variable is only used if the HTCondor daemons are running as root.
  In this case, the \Condor{starter} must be told what user identity
  to switch to before invoking the \Prog{boinc\_client}.
  This can be any valid user on the local system, but it must have
  write permission in whatever directory is specified by
  \MacroNI{BOINC\_InitialDir}.

\end{description}

Optional configuration variables:

\begin{description}

\item[\Macro{BOINC\_Arguments}] \label{param:BoincArguments}
  Command-line arguments that should be passed to the
  \Prog{boinc\_client} program.
  For example, one way to specify the BOINC project to join is to use 
  the \Opt{--attach\_project} argument to specify a project URL and
  account key.
  For example:

\footnotesize
\begin{verbatim}
BOINC_Arguments = --attach_project http://einstein.phys.uwm.edu [account_key] 
\end{verbatim}
\normalsize

\item[\Macro{BOINC\_Environment}] \label{param:BoincEnvironment}
  Environment variables that should be set for the
  \Prog{boinc\_client}.

\item[\Macro{BOINC\_Output}] \label{param:BoincOutput} Full path to
  the file where \File{stdout} from the \Prog{boinc\_client} should be
  written.
  If this variable is not defined, \File{stdout} will be discarded.

\item[\Macro{BOINC\_Error}] \label{param:BoincError} Full path to
  the file where \File{stderr} from the \Prog{boinc\_client} should be
  written.
  If this macro is not defined, \File{stderr} will be discarded.

\end{description}


The following example shows one possible usage of these settings:

\footnotesize
\begin{verbatim}
# Define a shared macro that can be used to define other settings.
# This directory must be manually created before attempting to run
# any backfill jobs.
BOINC_HOME = $(LOCAL_DIR)/boinc

# Path to the boinc_client to use, and required universe setting
BOINC_Executable = /usr/local/bin/boinc_client
BOINC_Universe = vanilla

# What initial working directory should BOINC use?
BOINC_InitialDir = $(BOINC_HOME)

# Where to place stdout and stderr
BOINC_Output = $(BOINC_HOME)/boinc.out
BOINC_Error = $(BOINC_HOME)/boinc.err
\end{verbatim}
\normalsize

If the HTCondor daemons reading this configuration are running as root,
an additional variable must be defined:

\footnotesize
\begin{verbatim}
# Specify the user that the boinc_client should run as:
BOINC_Owner = nobody
\end{verbatim}
\normalsize

In this case, HTCondor would spawn the \Prog{boinc\_client} as
\Login{nobody}, so the directory specified in \MacroUNI{BOINC\_HOME}
would have to be writable by the \Login{nobody} user.

A better choice would probably be to create a separate user account
just for running BOINC jobs, so that the local BOINC installation is
not writable by other processes running as \Login{nobody}.
Alternatively, the \MacroNI{BOINC\_Owner} could be set to
\Login{daemon}. 

\noindent \Bold{Attaching to a specific BOINC project}

There are a few ways to attach an HTCondor/BOINC installation to a given
BOINC project:
\begin{itemize}

\item Use the \Opt{--attach\_project} argument to the \Prog{boinc\_client}
  program, defined via the \MacroNI{BOINC\_Arguments} variable.
  The \Prog{boinc\_client} will only accept a single
  \Opt{--attach\_project} argument, so this method can only be used to
  attach to one project.

\item The \Prog{boinc\_cmd} command-line tool can perform various
  BOINC administrative tasks, including attaching to a BOINC project.
  Using \Prog{boinc\_cmd}, the appropriate argument to use is called
  \Opt{--project\_attach}.
  Unfortunately, the \Prog{boinc\_client} must be running for
  \Prog{boinc\_cmd} to work, so this method can only be used once the
  HTCondor resource has entered the Backfill state and has spawned the
  \Prog{boinc\_client}. 
  
\item Manually create account files in the local BOINC directory.
  Upon start up, the \Prog{boinc\_client} will scan its local directory
  (the directory specified with \MacroNI{BOINC\_InitialDir})
  for files of the form \File{account\_[URL].xml}, for example,
  \File{account\_einstein.phys.uwm.edu.xml}. 
  Any files with a name that matches this convention will be read and
  processed.
  The contents of the file define the project URL and the
  authentication key.
  The format is:

\footnotesize
\begin{verbatim}
<account>
  <master_url>[URL]</master_url>
  <authenticator>[key]</authenticator>
</account>
\end{verbatim}
\normalsize

For example: 

\footnotesize
\begin{verbatim}
<account>
  <master_url>http://einstein.phys.uwm.edu</master_url>
  <authenticator>aaaa1111bbbb2222cccc3333</authenticator>
</account>
\end{verbatim}
\normalsize

Of course, the \verb@<authenticator>@ tag would use the real
authentication key returned when the account was created at a given
project.

These account files can be copied to the local BOINC directory on all
machines in an HTCondor pool, so administrators can either distribute
them manually, or use symbolic links to point to a shared file
system. 

\end{itemize}

In the two cases of using command-line arguments for
\Prog{boinc\_client} or running the \Prog{boinc\_cmd} tool,
BOINC will write out the resulting account file to the local BOINC directory
on the machine, and then future invocations of the
\Prog{boinc\_client} will already be attached to the appropriate
project(s).
%  This link is no longer valid, and on the boinc web page,
%  there does not seem to any other links that substitute.
%More information about participating in multiple BOINC projects can be
%found at \URL{http://boinc.berkeley.edu/multiple\_projects.php}.

%%%%%%%%%%%%%%%%%%%%%%%%%%%%%%%%%%%%%%%%%%%%%%%%%%%%%%%%%%%%%%%%%%%%%%
\subsubsection{\label{sec:Backfill-BOINC-Windows}BOINC on Windows}
%%%%%%%%%%%%%%%%%%%%%%%%%%%%%%%%%%%%%%%%%%%%%%%%%%%%%%%%%%%%%%%%%%%%%%

The Windows version of BOINC has multiple installation methods.
The preferred method of installation for use with HTCondor is the 
Shared Installation method.
Using this method gives all users access to the executables.
During the installation process 
\begin{enumerate}
\item
Deselect the option which makes BOINC the default screen saver
\item
Deselect the option which runs BOINC on start up.
\item
Do not launch BOINC at the conclusion of the installation.
\end{enumerate}

There are three major differences from the Unix version
to keep in mind when dealing with the Windows installation:

\begin{enumerate}
\item
The Windows executables have different names from the Unix versions.  
The Windows client is called \Prog{boinc.exe}.
Therefore, the configuration variable \Macro{BOINC\_Executable} 
is written:

\footnotesize
\begin{verbatim}
BOINC_Executable = C:\PROGRA~1\BOINC\boinc.exe
\end{verbatim}
\normalsize

The Unix administrative tool \Prog{boinc\_cmd} 
is called \Prog{boinccmd.exe} on Windows.

  
\item
When using BOINC on Windows, the configuration variable
\Macro{BOINC\_InitialDir} will not be respected fully.
To work around this difficulty,
pass the BOINC home directory directly to the BOINC application
via the \Macro{BOINC\_Arguments} configuration variable.
For Windows, rewrite the argument line as:

\footnotesize
\begin{verbatim}
BOINC_Arguments = --dir $(BOINC_HOME) \
          --attach_project http://einstein.phys.uwm.edu [account_key] 
\end{verbatim}
\normalsize

As a consequence of setting the BOINC home directory, some projects may 
fail with the authentication error:
\footnotesize
\begin{verbatim}
Scheduler request failed: Peer 
certificate cannot be authenticated 
with known CA certificates.
\end{verbatim}
\normalsize

To resolve this issue,
copy the \File{ca-bundle.crt} file
from the BOINC installation directory
to \File{\$(BOINC\_HOME)}.
This file appears to be project and machine independent,
and it can therefore be distributed as part of an 
automated HTCondor installation.

\item
The \Macro{BOINC\_Owner} configuration variable behaves differently
on Windows than it does on Unix.
Its value may take one of two forms: 
\begin{itemize}
\item 
\verb@domain\user@
\item 
\verb@user@ This form assumes that the user exists in the local domain 
(that is, on the computer itself).
\end{itemize}

Setting this option causes the addition of the job attribute
\begin{verbatim}
RunAsUser = True
\end{verbatim}
to the backfill client.
This further implies that the configuration variable
\Macro{STARTER\_ALLOW\_RUNAS\_OWNER} be set to \Expr{True}
to insure that the local \Condor{starter} be able to run jobs in this 
manner.
For more information on the \Attr{RunAsUser} attribute, 
see section~\ref{sec:windows-run-as-owner}. 
For more information on the the \MacroNI{STARTER\_ALLOW\_RUNAS\_OWNER} 
configuration variable, see 
section~\ref{param:StarterAllowRunAsOwner}.

\end{enumerate}
