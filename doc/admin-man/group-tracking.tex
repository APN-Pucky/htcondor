%%%%%%%%%%%%%%%%%%%%%%%%%%%%%%%%%%%%%%%%%%%%%%%%%%%%%%%%%%%%%%%%%%%%%%%%%%%
\subsection{\label{sec:GroupTracking}Group ID-Based Process Tracking} 
%%%%%%%%%%%%%%%%%%%%%%%%%%%%%%%%%%%%%%%%%%%%%%%%%%%%%%%%%%%%%%%%%%%%%%%%%%%

One function that HTCondor often must perform is keeping track of all
processes created by a job. This is done so that HTCondor can provide
resource usage statistics about jobs, and also so that HTCondor can properly
clean up any processes that jobs leave behind when they exit.

In general, tracking process families is difficult to do reliably.
By default HTCondor uses a combination of process parent-child
relationships, process groups, and information that HTCondor places in a
job's environment to track process families on a best-effort
basis. This usually works well, but it can falter for certain
applications or for jobs that try to evade detection.

Jobs that run with a user account dedicated for HTCondor's use
can be reliably tracked, since all HTCondor needs to do is look for all
processes running using the given account. Administrators must specify
in HTCondor's configuration what accounts can be considered dedicated
via the \Macro{DEDICATED\_EXECUTE\_ACCOUNT\_REGEXP} setting. See
Section~\ref{sec:RunAsNobody} for further details.

Ideally, jobs can be reliably tracked regardless of the user account
they execute under. This can be accomplished with group ID-based
tracking. This method of tracking requires that a range of dedicated
\emph{group} IDs (GID) be set aside for HTCondor's use. The number of GIDs
that must be set aside for an execute machine is equal to its number
of execution slots. GID-based tracking is only available on Linux, and
it requires that HTCondor either runs as \Login{root} or uses privilege
separation (see Section~\ref{sec:PrivSep}).

GID-based tracking works by placing a dedicated GID in the
supplementary group list of a job's initial process. Since modifying
the supplementary group ID list requires
\Login{root} privilege, the job will not be able to create processes
that go unnoticed by HTCondor.

Once a suitable GID range has been set aside for process tracking,
GID-based tracking can be enabled via the
\Macro{USE\_GID\_PROCESS\_TRACKING} parameter. The minimum and maximum
GIDs included in the range are specified with the
\Macro{MIN\_TRACKING\_GID} and \Macro{MAX\_TRACKING\_GID}
settings. For example, the following would enable GID-based tracking
for an execute machine with 8 slots.
\begin{verbatim}
USE_GID_PROCESS_TRACKING = True
MIN_TRACKING_GID = 750
MAX_TRACKING_GID = 757
\end{verbatim}

If the defined range is too small, such that there is not a GID available
when starting a job,
then the \Condor{starter} will fail as it tries to start the job.
An error message will be logged stating that there are no more tracking GIDs.

GID-based process tracking requires use of the \Condor{procd}. If
\MacroNI{USE\_GID\_PROCESS\_TRACKING} is true, the \Condor{procd} will
be used regardless of the \Macro{USE\_PROCD} setting.  Changes to
\MacroNI{MIN\_TRACKING\_GID} and \MacroNI{MAX\_TRACKING\_GID} require
a full restart of HTCondor.

%%%%%%%%%%%%%%%%%%%%%%%%%%%%%%%%%%%%%%%%%%%%%%%%%%%%%%%%%%%%%%%%%%%%%%%%%%%
\subsection{\label{sec:CGroupTracking}Cgroup-Based Process Tracking} 
%%%%%%%%%%%%%%%%%%%%%%%%%%%%%%%%%%%%%%%%%%%%%%%%%%%%%%%%%%%%%%%%%%%%%%%%%%%
\index{cgroup based process tracking}

A new feature in Linux kernels version 2.6.24 and more recent kernels
allows HTCondor to
more accurately and safely manage jobs composed of sets of processes.
This Linux feature is called Control Groups, or cgroups for short, and 
it is available starting with RHEL 6, Debian 6, and related distributions.  
Documentation about Linux kernel support for cgroups can be found in
the Documentation directory in the kernel source code distribution.
Another good reference is 
\URL{http://docs.redhat.com/docs/en-US/Red\_Hat\_Enterprise\_Linux/6/html/Resource\_Management\_Guide/index.html}
Even if cgroup support is built into the kernel, 
many distributions do not install the cgroup tools by default.
In order to use cgroups, 
the tools must be installed.  
On RPM-based systems, these can be installed with the command

\begin{verbatim}
yum install libcgroup\*
\end{verbatim}

Starting with HTCondor version 7.7.0, 
the \Condor{starter} daemon can optionally use cgroups
to accurately track all the processes started by a job, 
even when quickly-exiting parent processes spawn many child processes.
As with the GID-based tracking, 
this is only implemented when a \Condor{procd} daemon is running.
The HTCondor team recommends enabling this feature on Linux platforms 
that support it.
When cgroup tracking is enabled, 
HTCondor is able to report a much more accurate
measurement of the physical memory used by a set of processes.

Kernel cgroups are named in a virtual file system hierarchy. 
HTCondor will put each
running job on the execute node in a separate cgroup, 
named using the job's attributes by \Expr{job\_<ClusterId>\_<ProcId>},
where \Expr{<ClusterId>} is replaced by 
the job ClassAd attribute \Attr{ClusterId},
and \Expr{<ProcId>} is replaced by 
the job ClassAd attribute \Attr{ProcId}.
These directories will be under a base directory named 
by the HTCondor configuration variable \Macro{BASE\_CGROUP}.  
This variable has no default value, so if the variable is not set,
cgroup tracking will not be used.  
Unless there is a need for integration of HTCondor jobs with other
cgroup-based tracking, 
a good choice for \MacroNI{BASE\_CGROUP} location might be \File{/condor}. 

HTCondor itself will not mount the virtual cgroup file systems.  
This can either be done by hand at each system reboot, 
by the \Prog{cgconfig} service 
which reads a file called \File{/etc/cgconfig.conf}, 
or automatically by the \Prog{systemd} service 
on systems which use \Prog{systemd} instead of \Prog{init}.

Here is an example of the contents of file \File{cgconfig.conf}:

\begin{verbatim}
mount {
        cpuacct = /mnt/cgroups/cpuacct;
        memory  = /mnt/cgroups/memory;
        freezer = /mnt/cgroups/freezer;
        blkio   = /mnt/cgroups/blkio;
}

group condor {
        cpuacct {}
        memory {}
        freezer {}
        blkio {}
}
\end{verbatim}

If the mount command shows that no cgroup file systems are mounted, 
then either the by hand method or the \Prog{cgconfig} service 
will need to mount the four controllers which HTCondor needs:
cpuacct, memory, freezer and blkio.  

Once cgroup-based tracking is configured, 
usage should be invisible to the user and administrator.  
The \Condor{procd} log, as defined by configuration variable
\MacroNI{PROCD\_LOG}, 
will mention that it is using this method, 
but no user visible changes should occur,
other than the impossibility of a quickly-forking process escaping from the
control of the \Condor{starter},
and the more accurate reporting of memory usage.
