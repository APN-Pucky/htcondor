%%%%%%%%%%%%%%%%%%%%%%%%%%%%%%%%%%%%%%%%%%%%%%%%%%%%%%%%%%%%%%%%%%%%%%%%%%%
\subsection{\label{sec:GroupTracking}Group ID-Based Process Tracking} 
%%%%%%%%%%%%%%%%%%%%%%%%%%%%%%%%%%%%%%%%%%%%%%%%%%%%%%%%%%%%%%%%%%%%%%%%%%%

One function that HTCondor often must perform is keeping track of all
processes created by a job. This is done so that HTCondor can provide
resource usage statistics about jobs, and also so that HTCondor can properly
clean up any processes that jobs leave behind when they exit.

In general, tracking process families is difficult to do reliably.
By default HTCondor uses a combination of process parent-child
relationships, process groups, and information that HTCondor places in a
job's environment to track process families on a best-effort
basis. This usually works well, but it can falter for certain
applications or for jobs that try to evade detection.

Jobs that run with a user account dedicated for HTCondor's use
can be reliably tracked, since all HTCondor needs to do is look for all
processes running using the given account. Administrators must specify
in HTCondor's configuration what accounts can be considered dedicated
via the \Macro{DEDICATED\_EXECUTE\_ACCOUNT\_REGEXP} setting. See
Section~\ref{sec:RunAsNobody} for further details.

Ideally, jobs can be reliably tracked regardless of the user account
they execute under. This can be accomplished with group ID-based
tracking. This method of tracking requires that a range of dedicated
\emph{group} IDs (GID) be set aside for HTCondor's use. The number of GIDs
that must be set aside for an execute machine is equal to its number
of execution slots. 
GID-based tracking is only available on Linux, 
and it requires that HTCondor daemons run as \Login{root}.

GID-based tracking works by placing a dedicated GID in the
supplementary group list of a job's initial process. Since modifying
the supplementary group ID list requires
\Login{root} privilege, the job will not be able to create processes
that go unnoticed by HTCondor.

Once a suitable GID range has been set aside for process tracking,
GID-based tracking can be enabled via the
\Macro{USE\_GID\_PROCESS\_TRACKING} parameter. The minimum and maximum
GIDs included in the range are specified with the
\Macro{MIN\_TRACKING\_GID} and \Macro{MAX\_TRACKING\_GID}
settings. For example, the following would enable GID-based tracking
for an execute machine with 8 slots.
\begin{verbatim}
USE_GID_PROCESS_TRACKING = True
MIN_TRACKING_GID = 750
MAX_TRACKING_GID = 757
\end{verbatim}

If the defined range is too small, such that there is not a GID available
when starting a job,
then the \Condor{starter} will fail as it tries to start the job.
An error message will be logged stating that there are no more tracking GIDs.

GID-based process tracking requires use of the \Condor{procd}. If
\MacroNI{USE\_GID\_PROCESS\_TRACKING} is true, the \Condor{procd} will
be used regardless of the \Macro{USE\_PROCD} setting.  Changes to
\MacroNI{MIN\_TRACKING\_GID} and \MacroNI{MAX\_TRACKING\_GID} require
a full restart of HTCondor.

%%%%%%%%%%%%%%%%%%%%%%%%%%%%%%%%%%%%%%%%%%%%%%%%%%%%%%%%%%%%%%%%%%%%%%%%%%%
\subsection{\label{sec:CGroupTracking}Cgroup-Based Process Tracking} 
%%%%%%%%%%%%%%%%%%%%%%%%%%%%%%%%%%%%%%%%%%%%%%%%%%%%%%%%%%%%%%%%%%%%%%%%%%%
\index{cgroup based process tracking}

A new feature in Linux version 2.6.24 allows HTCondor to more accurately 
and safely manage jobs composed of sets of processes.  This Linux 
feature is called Control Groups, or cgroups for short, and it is 
available starting with RHEL 6, Debian 6, and related distributions.  
Documentation about Linux kernel support for cgroups can be found 
in the Documentation directory in the kernel source code distribution.
Another good reference is 
\URL{http://docs.redhat.com/docs/en-US/Red\_Hat\_Enterprise\_Linux/6/html/Resource\_Management\_Guide/index.html}
Even if cgroup support is built into the kernel, 
many distributions do not install the cgroup tools by default.

The interface between the kernel cgroup functionality is via a (virtual) 
file system.  When the condor\_master starts on a Linux system with
cgroup support in the kernel, it checks to see if cgroups are mounted,
and if not, it will try to mount the cgroup virtual
filesystem onto the directory /cgroup.

If your Linux distribution uses \Prog{systemd}, 
it will mount the cgroup file system, 
and the only remaining item is to set configuration variable
\Macro{BASE\_CGROUP}, as described below.


%RHEL6 and related distributions that support the \Prog{cgconfig} service 
%will mount them at boot time.

%On RPM-based systems, these can be installed with the command

%\begin{verbatim}
%  yum install libcgroup\*
%\end{verbatim}

%After these tools are installed, the \Prog{cgconfig} service needs to be
%running.  It parses the \File{/etc/cgconfig.conf} file, and makes
%appropriate mounts under \File{/cgroup}.  Before starting the \Prog{cgconfig}
%service, you will need to edit the file \File{/etc/cgconfig.conf} to
%add a group specific to HTCondor.
%
%Here is an example of the contents of file \File{/etc/cgconfig.conf} with
%appropriate values for the \texttt{htcondor} group:

%\begin{verbatim}
%mount {
%        cpu	= /cgroup/cpu;
%        cpuset	= /cgroup/cpuset;
%        cpuacct = /cgroup/cpuacct;
%        memory  = /cgroup/memory;
%        freezer = /cgroup/freezer;
%        blkio   = /cgroup/blkio;
%}

%group htcondor {
%	cpu {}
%        cpuacct {}
%        memory {}
%        freezer {}
%        blkio {}
%}
%\end{verbatim}

On Debian based systems, the memory cgroup controller is often not 
on by default, and needs to be enabled with a boot time option.

This setting needs to be inherited
down to the per-job cgroup with the following commands in \File{rc.local}:

\begin{verbatim}
/usr/sbin/cgconfigparser -l /etc/cgconfig.conf
/bin/echo 1 > /sys/fs/cgroup/htcondor/cgroup.clone_children
\end{verbatim}


%Also for Debian, add the following field to group \texttt{htcondor}:

%\begin{verbatim}
%cpuset {
%	cpusets.mems = 0;
%%}
%\end{verbatim}

%After the \File{/etc/cgconfig.conf} file has had the \texttt{htcondor} group
%added to it, add and start the \Prog{cgconfig} service by running

%\begin{verbatim}
%chkconfig --add cgconfig
%service cgconfig start
%\end{verbatim}

When cgroups are correctly configured and running, the virtual file system
mounted on \File{/cgroup} should have several subdirectories under it, 
and there should an \File{htcondor} subdirectory under the directory 
\File{/cgroup/cpu}.

The \Condor{starter} daemon uses cgroups by default on Linux systems
to accurately track all the processes started by a job, 
even when quickly-exiting parent processes spawn many child processes.
As with the GID-based tracking, this is only implemented when a 
\Condor{procd} daemon is running.  

Kernel cgroups are named in a virtual file system hierarchy. 
HTCondor will put each running job on the execute node in a distinct cgroup.
The name of this cgroup is the name of the execute directory for 
that \Condor{starter}, with slashes replaced by underscores, 
followed by the name and number of the slot.  
So, for the memory controller, 
a job running on slot1 would have its cgroup located at
\File{/cgroup/memory/htcondor/condor\_var\_lib\_condor\_execute\_slot1/}.  
The \File{tasks} file in this directory will contain a list 
of all the processes in this cgroup, and
many other files in this directory have useful information about resource usage
of this cgroup.  See the kernel documentation for full details.

Once cgroup-based tracking is configured, 
usage should be invisible to the user and administrator.  
The \Condor{procd} log, as defined by configuration variable
\MacroNI{PROCD\_LOG}, 
will mention that it is using this method, 
but no user visible changes should occur,
other than the impossibility of a quickly-forking process escaping from the
control of the \Condor{starter},
and the more accurate reporting of memory usage.
