%%%%%%%%%%%%%%%%%%%%%%%%%%%%%%%%%%%%%%%%%%%%%%%%%%%%%%%%%%%%%%%%%%%%%%
\subsection{\label{sec:Flocking}
Flocking: Configuring a Schedd to Submit to Multiple Pools}
%%%%%%%%%%%%%%%%%%%%%%%%%%%%%%%%%%%%%%%%%%%%%%%%%%%%%%%%%%%%%%%%%%%%%%

The \Condor{schedd} may be configured to submit jobs to more than one
pool---this is known as flocking. If Condor pool A can send jobs to
Condor pool B, then we say that A flocks to B. Flocking can be one
way, such as A flocking to B, or it can be set up in both directions. 

To configure flocking, you normally need to set just two configuration
variables. Assume you have the situation where pool A is flocking to
pool B. In pool A, set \Macro{FLOCK\_TO} to the host name of the
central manager of pool B. You could set a list of host names, if you
were flocking to multiple pools. In pool B, set \Macro{FLOCK\_FROM}
to the names of all the hosts from pool A that might flock to pool
B. If you don't wish to list all of the hosts, you can use a wildcard
to allow multiple hosts. For example, you could say use
``*.cs.wisc.edu'' to allow all hosts from the cs.wisc.edu domain. 

If you wish to also allow flocking from pool B to pool A, you can
simply set up flocking in the other direction.

When you flock to another pool, you will not attempt to flock a
particular job unless you cannot currently run it in your pool. Jobs
that are run in another pool can only be standard universe jobs, and
they are run as user ``nobody''.

\MacroUNI{HOSTALLOW\_NEGOTIATOR\_SCHEDD} (see
section~\ref{param:HostAllow}) must also be configured to allow
negotiators from all of the \MacroU{FLOCK\_NEGOTIATOR\_HOSTS} to
contact the schedd.  Please make sure the \MacroUNI{NEGOTIATOR\_HOST}
is first in the \MacroUNI{HOSTALLOW\_NEGOTIATOR\_SCHEDD} list.  This
is the default configuration, so it will be correct if you haven't
modified it.

% This was the old text for this section. I believe it to be pretty 
% outdated, but I didn't want to get rid of it yet. --Alain 11-Oct-01
% The \Condor{schedd} may be configured to submit jobs to more than one
% pool.
% In the default configuration, the \Condor{schedd} contacts the
% Central Manager specified by the \Macro{CONDOR\_HOST} macro (described
% in section~\ref{sec:Condor-wide-Config-File-Entries} on
% page~\pageref{sec:Condor-wide-Config-File-Entries})
% to locate execute machines
% available to run jobs in its queue.
% However, the
% \Macro{FLOCK\_NEGOTIATOR\_HOSTS} and \Macro{FLOCK\_COLLECTOR\_HOSTS}
% macros (described in
% section~\ref{sec:Schedd-Config-File-Entries} on
% page~\pageref{sec:Schedd-Config-File-Entries}) may 
% be used to specify additional 
% Central Managers for the \Condor{schedd} to contact.
% When the local
% pool does not satisfy all job requests, the \Condor{schedd} will try
% the pools specified by these macros in turn until all jobs are
% satisfied.

% \MacroUNI{HOSTALLOW\_NEGOTIATOR\_SCHEDD} (see
% section~\ref{param:HostAllow}) must also be configured to allow
% negotiators from all of the \MacroU{FLOCK\_NEGOTIATOR\_HOSTS} to contact the schedd.
% Please make sure the \MacroUNI{NEGOTIATOR\_HOST} is first in the 
% \MacroUNI{HOSTALLOW\_NEGOTIATOR\_SCHEDD} list.
% Similarly, the central managers of the remote pools must be configured
% to listen to requests from this schedd.
