%%%%%%%%%%%%%%%%%%%%%%%%%%%%%%%%%%%%%%%%%%%%%%%%%%%%%%%%%%%%%%%%%%%%%%
\section{\label{sec:Configuring-Condor}Configuration}
%%%%%%%%%%%%%%%%%%%%%%%%%%%%%%%%%%%%%%%%%%%%%%%%%%%%%%%%%%%%%%%%%%%%%%

\index{Condor!configuration}
\index{configuration}

This section describes how to configure all parts of the Condor
system.  General information about the configuration
files and their syntax is followed by a description of
settings that affect all
Condor daemons and tools.
The 
settings that control the policy under which Condor will start,
suspend, resume, vacate or kill jobs
are described in 
section~\ref{sec:Configuring-Policy} on Startd Policy Configuration. 

%%%%%%%%%%%%%%%%%%%%%%%%%%%%%%%%%%%%%%%%%%%%%%%%%%%%%%%%%%%%%%%%%%%%%%
\subsection{\label{sec:Intro-to-Config-Files}Introduction to
Configuration Files} 
%%%%%%%%%%%%%%%%%%%%%%%%%%%%%%%%%%%%%%%%%%%%%%%%%%%%%%%%%%%%%%%%%%%%%%

The Condor configuration files are used to customize how Condor
operates at a given site.  The basic configuration as shipped with
Condor works well for most sites.

Each Condor program will, as part of its initialization process,
configure itself by calling a library routine which parses the
various configuration files that might be used including pool-wide,
platform-specific, and machine-specific configuration files.
Environment variables may also contribute to the configuration.

The result of configuration is a list of key/value pairs.
Each key is a configuration variable name,
and each value is a string literal
that may utilize macro substitution (as defined below).
Note that the string literal value portion of a pair is not an expression,
and therefore it is not evaluated.
Those configuration variables that express the policy for
starting and stopping of jobs appear as expressions in the
configuration file.
However, these expressions (for configuration) are string literals.
At appropriate times,
Condor daemons and tools use these strings as expressions,
parsing them in order to do evaluation.


%%%%%%%%%%%%%%%%%%%%%%%%%%%%%%%%%%%%%%%%%%%%%%%%%%%%%%%%%%%%%%%%%%%%%%
\subsubsection{\label{sec:Ordering-Config-File}Ordered Evaluation to
Set the Configuration} 
%%%%%%%%%%%%%%%%%%%%%%%%%%%%%%%%%%%%%%%%%%%%%%%%%%%%%%%%%%%%%%%%%%%%%%
\index{configuration file!evaluation order}

Multiple files, as well as a program's environment variables
determine the configuration.
The order in which attributes are defined is important, as later
definitions override existing definitions.
The order in which the (multiple) configuration files are parsed 
is designed to ensure the security of the system.
Attributes which must be set a specific way 
must appear in the last file to be parsed.
This prevents both the naive and the malicious Condor user 
from subverting the system through its configuration.
The order in which items are parsed is
\begin{enumerate}
\item global configuration file
\item local configuration file
\item specific environment variables prefixed with \MacroNI{\_CONDOR\_}
\end{enumerate}

The locations for these files are as given in
section~\ref{sec:Config-File-Locations} on
page~\pageref{sec:Config-File-Locations}.

Some Condor tools utilize environment variables to set their
configuration.
These tools search for specifically-named environment variables.
The variables are prefixed by the string \MacroNI{\_CONDOR\_}
or \MacroNI{\_condor\_}.
The tools strip off the prefix, and utilize what remains
as configuration.
As the use of environment variables is the last within
the ordered evaluation, 
the environment variable definition is used.
The security of the system is not compromised,
as only specific variables are considered for definition
in this manner, not any environment variables with
the \MacroNI{\_CONDOR\_} prefix.


%%%%%%%%%%%%%%%%%%%%%%%%%%%%%%%%%%%%%%%%%%%%%%%%%%%%%%%%%%%%%%%%%%%%%%
\subsubsection{\label{sec:Config-File-Macros}Configuration File Macros} 
%%%%%%%%%%%%%%%%%%%%%%%%%%%%%%%%%%%%%%%%%%%%%%%%%%%%%%%%%%%%%%%%%%%%%%

\index{macro!in configuration file}
\index{configuration file!macro definitions}

Macro definitions are of the form:
\begin{verbatim}
<macro_name> = <macro_definition>
\end{verbatim}

\Note There must be white space between the macro name, the
``='' sign, and the macro definition.

Macro invocations are of the form: 
\begin{verbatim}
$(macro_name)
\end{verbatim}

Macro definitions may contain references to other macros, even ones
that are not yet defined (as long as they are eventually defined in
the configuration files).
All macro expansion is done after all configuration files have been parsed
(with the exception of macros that reference themselves, described
below). 

\begin{verbatim}
A = xxx
C = $(A) 
\end{verbatim}
is a legal set of macro definitions, and the resulting value of 
\MacroNI{C} is
\Expr{xxx}.
Note that
\MacroNI{C} is actually bound to 
\MacroUNI{A}, not its value.

As a further example,
\begin{verbatim}
A = xxx
C = $(A)
A = yyy
\end{verbatim}
is also a legal set of macro definitions, and the resulting value of
\MacroNI{C} is \Expr{yyy}.  

A macro may be incrementally defined by invoking itself in its
definition.  For example,
\begin{verbatim}
A = xxx
B = $(A)
A = $(A)yyy
A = $(A)zzz
\end{verbatim}
is a legal set of macro definitions, and the resulting value of 
\MacroNI{A}
is \Expr{xxxyyyzzz}.
Note that invocations of a macro in
its own definition are immediately
expanded.
\MacroUNI{A} is immediately expanded in line 3 of the example.
If it were not, then the definition would be impossible to
evaluate.

Recursively defined macros such as
\begin{verbatim}
A = $(B)
B = $(A)
\end{verbatim}
are not allowed.
They create definitions that Condor refuses to parse. 

% commented out in 2005, as it is too old, and is now confusing
%\Note Condor used to distinguish between ``macros'' and ``expressions''
%in its config files.
%Beginning with Condor version 6.1.13, this distinction has been
%removed.
%For backward compatibility, you can still use ``:'' instead of ``=''
%in your config files, and these attributes will just be treated as
%macros.

All entries in a configuration file must have an operator,
which will be an equals sign (\verb@=@).
%or a colon character (\verb@:@).
Identifiers are alphanumerics combined with the underscore character,
optionally with a subsystem name and a period as a prefix.
As a special case,
a line without an operator that begins with a left square bracket
will be ignored.
The following two-line example treats the first line as a comment,
and correctly handles the second line.
\begin{verbatim}
[Condor Settings]
my_classad = [ foo=bar ]
\end{verbatim}

% functionality added to version 6.7.13
To simplify pool administration,
any configuration variable name may be prefixed by
a subsystem 
(see the \MacroUNI{SUBSYSTEM} macro in 
section~\ref{sec:Pre-Defined-Macros}
for the list of subsystems)
and the period (\verb@.@) character.
For configuration variables defined this way,
the value is applied to the specific subsystem.
For example,
the ports that Condor may use can be restricted to a range 
using the \MacroNI{HIGHPORT} and \MacroNI{LOWPORT} configuration
variables.
If the range of intended ports is different for specific
daemons, this syntax may be used.
\begin{verbatim}
  MASTER.LOWPORT   = 20000
  MASTER.HIGHPORT  = 20100
  NEGOTIATOR.LOWPORT   =  22000 
  NEGOTIATOR.HIGHPORT  =  22100
\end{verbatim}

Note that all configuration variables may utilize this syntax,
but nonsense configuration variables may result.
For example, it makes no sense to define
\begin{verbatim}
  NEGOTIATOR.MASTER_UPDATE_INTERVAL = 60
\end{verbatim}
since the \Condor{negotiator} daemon does not use the
\MacroNI{MASTER\_UPDATE\_INTERVAL} variable.

It makes little sense to do so, but Condor will configure
correctly with a definition such as
\begin{verbatim}
  MASTER.MASTER_UPDATE_INTERVAL = 60
\end{verbatim}
The \Condor{master} uses this configuration variable,
and the prefix of \MacroNI{MASTER.} causes this configuration
to be specific to the \Condor{master} daemon.


%%%%%%%%%%%%%%%%%%%%%%%%%%%%%%%%%%%%%%%%%%%%%%%%%%%%%%%%%%%%%%%%%%%%%%
\subsubsection{\label{sec:Other-Syntax}Comments and Line Continuations}
%%%%%%%%%%%%%%%%%%%%%%%%%%%%%%%%%%%%%%%%%%%%%%%%%%%%%%%%%%%%%%%%%%%%%%

A Condor configuration file may contain comments and
line continuations.
A comment is any line beginning with a ``\#'' character.
A continuation is any entry that continues across multiples lines.
Line continuation is accomplished by placing the ``$\backslash$''
character at the end of any line to be continued onto another.
Valid examples of line continuation are
\begin{verbatim}
  START = (KeyboardIdle > 15 * $(MINUTE)) && \
  ((LoadAvg - CondorLoadAvg) <= 0.3)
\end{verbatim}
and
\begin{verbatim}
  ADMIN_MACHINES = condor.cs.wisc.edu, raven.cs.wisc.edu, \
  stork.cs.wisc.edu, ostrich.cs.wisc.edu, \
  bigbird.cs.wisc.edu
  HOSTALLOW_ADMIN = $(ADMIN_MACHINES)
\end{verbatim}

Note that a line continuation character may currently be used within
a comment, so the following example does \emph{not} set the
configuration variable \MacroNI{FOO}:
\begin{verbatim}
  # This comment includes the following line, so FOO is NOT set \
  FOO = BAR
\end{verbatim}
It is a poor idea to use this functionality, as it is likely to
stop working in future Condor releases.

%%%%%%%%%%%%%%%%%%%%%%%%%%%%%%%%%%%%%%%%%%%%%%%%%%%%%%%%%%%%%%%%%%%%%%
\subsubsection{\label{sec:Program-Defined-Macros}Executing a Program to Produce Configuration Macros}
%%%%%%%%%%%%%%%%%%%%%%%%%%%%%%%%%%%%%%%%%%%%%%%%%%%%%%%%%%%%%%%%%%%%%%

Instead of reading from a file,
Condor may run a program to obtain configuration macros.
The vertical bar character (\Bar) as the last character defining
a file name provides the syntax necessary to tell 
Condor to run a program.
This syntax may only be used in the definition of
the \Env{CONDOR\_CONFIG} environment variable,
or the \Macro{LOCAL\_CONFIG\_FILE} configuration variable.

The command line for the program 
is formed by the characters preceding the vertical bar character.
The standard output of the program is parsed as a configuration 
file would be.

An example:
\begin{verbatim}
LOCAL_CONFIG_FILE = /bin/make_the_config|
\end{verbatim}

Program \Prog{/bin/make\_the\_config} is executed, and its output
is the set of configuration macros.

Note that either a program is executed to generate the
configuration macros or the configuration is read from 
one or more files.
The syntax uses space characters to separate command line elements,
if an executed program produces the configuration macros.
Space characters would otherwise separate the list of files.
This syntax does not permit distinguishing one from the other,
so only one may be specified.

%%%%%%%%%%%%%%%%%%%%%%%%%%%%%%%%%%%%%%%%%%%%%%%%%%%%%%%%%%%%%%%%%%%%%%
\subsubsection{\label{sec:Pre-Defined-Macros}Pre-Defined Macros}
%%%%%%%%%%%%%%%%%%%%%%%%%%%%%%%%%%%%%%%%%%%%%%%%%%%%%%%%%%%%%%%%%%%%%%

\index{configuration!pre-defined macros}
Condor provides pre-defined macros that help configure Condor.
Pre-defined macros are listed as \MacroUNI{macro\_name}.

This first set are entries whose values are determined at
run time and cannot be overwritten.  These are inserted automatically by
the library routine which parses the configuration files.
\index{configuration file!pre-defined macros}
\begin{description}
  
\item[\MacroU{FULL\_HOSTNAME}] \label{param:FullHostname}
  The
  fully qualified hostname of the local machine (hostname plus domain
  name).
  
\item[\MacroU{HOSTNAME}] \label{param:Hostname}
  The hostname of the local machine (no domain name).
  
\item[\MacroU{IP\_ADDRESS}] \label{param:IpAddress}
  The ASCII string version of the local machine's IP address.

\item[\MacroU{TILDE}] \label{param:Tilde}
  The full path to the
  home directory of the Unix user condor, if such a user exists on the
  local machine.

  \label{sec:Condor-Subsystem-Names}
  \index{configuration file!subsystem names}
\item[\MacroU{SUBSYSTEM}] \label{param:Subsystem}
  The subsystem
  name of the daemon or tool that is evaluating the macro.
  This is a unique string which identifies a given daemon within the
  Condor system.  The possible subsystem names are:

  \index{subsystem names}
  \index{macro!subsystem names}
  \begin{itemize}
  \item \verb@STARTD@
  \item \verb@SCHEDD@
  \item \verb@MASTER@
  \item \verb@COLLECTOR@
  \item \verb@NEGOTIATOR@
  \item \verb@KBDD@ 
  \item \verb@SHADOW@
  \item \verb@STARTER@
  \item \verb@CKPT_SERVER@
  \item \verb@SUBMIT@
  \item \verb@GRIDMANAGER@
  \item \verb@TOOL@
  \item \verb@HAD@
  \item \verb@REPLICATION@
  \item \verb@QUILL@
  \item \verb@DBMSD@
  \item \verb@JOB_ROUTER@
    \label{list:subsystem names}
  \end{itemize}

\end{description}

This second set of macros are entries whose default values are
determined automatically at runtime but which can be overwritten.  
\index{configuration file!macros}
\begin{description}

\item[\MacroU{ARCH}] \label{param:Arch}
  Defines the string
  used to identify the architecture of the local machine to Condor.
  The \Condor{startd} will advertise itself with this attribute so
  that users can submit binaries compiled for a given platform and
  force them to run on the correct machines.  \Condor{submit} will
  append a requirement to the job ClassAd that it must
  run on the same \MacroNI{ARCH} and \MacroNI{OPSYS} of the machine where
  it was submitted, unless the user specifies \MacroNI{ARCH} and/or
  \MacroNI{OPSYS} explicitly in their submit file.  See the
  the \Condor{submit} manual page
  on page~\pageref{man-condor-submit} for details.

\item[\MacroU{OPSYS}] \label{param:OpSys}
  Defines the string used to identify the operating system
  of the local machine to Condor.
  If it is not defined in the configuration file, Condor will
  automatically insert the operating system of this machine as
  determined by \Prog{uname}.

\item[\MacroU{UNAME\_ARCH}] \label{param:UnameArch}
  The architecture as reported by \Prog{uname}(2)'s \Code{machine} field.
  Always the same as \MacroNI{ARCH} on Windows.

\item[\MacroU{UNAME\_OPSYS}] \label{param:UnameOpsys}
  The operating system as reported by \Prog{uname}(2)'s \Code{sysname} field.
  Always the same as \MacroNI{OPSYS} on Windows.

\item[\MacroU{PID}] \label{param:Pid}
  The process ID for the daemon or tool.

\item[\MacroU{PPID}] \label{param:Ppid}
  The process ID of the parent process for the daemon or tool.

\item[\MacroU{USERNAME}] \label{param:Username}
  The user name of the UID of the daemon or tool.
  For daemons started as root, but running under another UID
  (typically the user condor), this will be the other UID.

\item[\MacroU{FILESYSTEM\_DOMAIN}]
  \label{param:FilesystemDomain}
  Defaults to the fully
  qualified hostname of the machine it is evaluated on.  See
  section~\ref{sec:Shared-Filesystem-Config-File-Entries}, Shared
  File System Configuration File Entries for the full description of
  its use and under what conditions you would want to change it.

\item[\MacroU{UID\_DOMAIN}]
  \label{param:UIDDomain}
  Defaults to the fully
  qualified hostname of the machine it is evaluated on.  See
  section~\ref{sec:Shared-Filesystem-Config-File-Entries} 
  for the full description of this configuration variable.

\end{description}

Since \MacroUNI{ARCH} and \MacroUNI{OPSYS} will automatically be set to the
correct values, we recommend that you do not overwrite them.
Only do so if you know what you are doing.



%%%%%%%%%%%%%%%%%%%%%%%%%%%%%%%%%%%%%%%%%%%%%%%%%%%%%%%%%%%%%%%%%%%%%%
\subsection{\label{sec:Config-File-Special}The Special Configuration Macros
\$ENV(), \$RANDOM\_CHOICE(), and  \$RANDOM\_INTEGER()} 
%%%%%%%%%%%%%%%%%%%%%%%%%%%%%%%%%%%%%%%%%%%%%%%%%%%%%%%%%%%%%%%%%%%%%%

\index{configuration file!\$ENV definition}
\index{\$ENV!in configuration file}
References to the Condor process's environment are allowed in the
configuration files.
Environment references use the \Macro{ENV} macro and are of the form:
\begin{verbatim}
  $ENV(environment_variable_name)
\end{verbatim}
For example, 
\begin{verbatim}
  A = $ENV(HOME)
\end{verbatim}
binds \MacroNI{A} to the value of the HOME environment variable.
Environment references are not currently used in standard Condor
configurations.
However, they can sometimes be useful in custom configurations.

\index{\$RANDOM\_CHOICE()!in configuration}
This same syntax is used in the \Macro{RANDOM\_CHOICE()} macro to
allow a random choice of a parameter
within a configuration file.
These references are of the form:
\begin{verbatim}
  $RANDOM_CHOICE(list of parameters)
\end{verbatim}
This allows a random choice within the parameter list to be made
at configuration time.  Of the list of parameters, one is
chosen when encountered during configuration.  For example,
if one of the integers 0-8 (inclusive) should be randomly
chosen, the macro usage is
\begin{verbatim}
  $RANDOM_CHOICE(0,1,2,3,4,5,6,7,8)
\end{verbatim}

\index{\$RANDOM\_INTEGER()!in configuration}
The \Macro{RANDOM\_INTEGER()} macro is similar to the \MacroNI{RANDOM\_CHOICE()}
macro, and is used to select a random integer within a configuration file.
References are of the form:
\begin{verbatim}
  $RANDOM_INTEGER(min, max [, step])
\end{verbatim}
A random integer within the range \verb@min@ and \verb@max@, inclusive,
is selected at configuration time.
The optional \verb@step@ parameter
controls the stride within the range, and it defaults to the value 1.
For example, to randomly chose an even integer in the range 0-8 (inclusive),
the macro usage is
\begin{verbatim}
  $RANDOM_INTEGER(0, 8, 2)
\end{verbatim}

See section~\ref{sec:randomintegerusage} on
page~\pageref{sec:randomintegerusage}
for an actual use of this specialized macro.
%%%%%%%%%%%%%%%%%%%%%%%%%%%%%%%%%%%%%%%%%%%%%%%%%%%%%%%%%%%%%%%%%%%%%%
\subsection{\label{sec:Condor-wide-Config-File-Entries}Condor-wide Configuration File Entries} 
%%%%%%%%%%%%%%%%%%%%%%%%%%%%%%%%%%%%%%%%%%%%%%%%%%%%%%%%%%%%%%%%%%%%%%

\index{configuration!Condor-wide configuration variables}

This section describes settings which affect all parts of the Condor
system. 
Other system-wide settings can be found in
section~\ref{sec:Network-Related-Config-File-Entries} on
``Network-Related Configuration File Entries'', and
section~\ref{sec:Shared-Filesystem-Config-File-Entries} on ``Shared
File System Configuration File Entries''. 

\begin{description}
  
\item[\Macro{CONDOR\_HOST}] \label{param:CondorHost} This macro may be
  used to define the \MacroUNI{NEGOTIATOR\_HOST} and is used to define the
  \MacroUNI{COLLECTOR\_HOST} macro.  Normally the \Condor{collector}
  and \Condor{negotiator} would run on the same machine.  If for some
  reason they were not run on the same machine,
  \MacroUNI{CONDOR\_HOST} would not be needed.  Some
  of the host-based security macros use \MacroUNI{CONDOR\_HOST} by
  default.  See section~\ref{sec:Host-Security}, on Setting up
  IP/host-based security in Condor for details.
  
\item[\Macro{COLLECTOR\_HOST}] \label{param:CollectorHost} The
  hostname of the machine where the \Condor{collector} is running for
  your pool.  Normally, it is defined relative to
  the \MacroUNI{CONDOR\_HOST}
  macro.  There is no default value for this macro;
  \MacroNI{COLLECTOR\_HOST} must be defined for the pool to work
  properly.

  In addition to defining the hostname, this setting can optionally be
  used to specify the network port of the \Condor{collector}.
  The port is separated from the hostname by a colon ('\verb@:@').
  For example,
  \begin{verbatim}
    COLLECTOR_HOST = $(CONDOR_HOST):1234
  \end{verbatim}
  If no port is specified, the default port of 9618 is used.
  Using the default port is recommended for most sites.
  It is only changed if there is a conflict with another
  service listening on the same network port.
  For more information about specifying a non-standard port for the
  \Condor{collector} daemon,
  see section~\ref{sec:Ports-NonStandard} on
  page~\pageref{sec:Ports-NonStandard}.


\item[\Macro{NEGOTIATOR\_HOST}] \label{param:NegotiatorHost} 
  This configuration variable is no longer used.
  The port where the \Condor{negotiator} is listening is normally
  dynamically allocated since version 6.7.4.

  For pools running 6.7.3 and older versions: The
  host name of the machine where the \Condor{negotiator} is running for
  the pool.
  Normally, it is defined relative to the \MacroUNI{CONDOR\_HOST}
  macro.  There is no default value for this macro;
  \MacroNI{NEGOTIATOR\_HOST} must be defined for the pool to work
  properly.
  This variable may also be used to optionally define a network port for
  the \Condor{negotiator} daemon, as explained for the
  \MacroNI{COLLECTOR\_HOST} variable.

\item[\Macro{CONDOR\_VIEW\_HOST}] \label{param:CondorViewHost} The
  host name of the machine, optionally appended by a colon and the
  port number, where the CondorView server is running.
  This service is optional, and requires additional configuration 
  to enable it.  There is no default value for
  \MacroNI{CONDOR\_VIEW\_HOST}.  If \MacroNI{CONDOR\_VIEW\_HOST} is not
  defined, no CondorView server is used.
  See section~\ref{sec:Contrib-CondorView-Install} on
  page~\pageref{sec:Contrib-CondorView-Install} for more details.

\item[\Macro{SCHEDD\_HOST}] \label{param:ScheddHost} The
  hostname of the machine where the \Condor{schedd} is running for
  your pool.  This is the host that queues submitted jobs.  Note that,
  in most condor installations, there is a \Condor{schedd} running on
  each host from which jobs are submitted.  The default value of
  \Macro{SCHEDD\_HOST} is the current host.  For most pools, this
  macro is not defined.

\item[\Macro{RELEASE\_DIR}] \label{param:ReleaseDir} The full path to
  the Condor release directory, which holds the \File{bin},
  \File{etc}, \File{lib}, and \File{sbin} directories.  Other macros
  are defined relative to this one.  There is no default value for
  \Macro{RELEASE\_DIR}.

\item[\Macro{BIN}] \label{param:Bin} This directory points to the
  Condor directory where user-level programs are installed.  It is
  usually defined relative to the \MacroUNI{RELEASE\_DIR} macro.
  There is no default value for \Macro{BIN}.
  
\item[\Macro{LIB}] \label{param:Lib} This directory points to the
  Condor directory where libraries used to link jobs for Condor's
  standard universe are stored.  The \Condor{compile} program uses
  this macro to find these libraries, so it must be defined for
  \Condor{compile} to function.  \MacroUNI{LIB} is usually defined
  relative to the \MacroUNI{RELEASE\_DIR} macro, and has no default
  value.

\item[\Macro{LIBEXEC}] \label{param:LibExec} This directory points
  to the Condor directory where support commands that Condor
  needs will be placed.
  Do not add this directory to a user or system-wide path.

\item[\Macro{INCLUDE}] \label{param:Include} This directory points
  to the Condor directory where header files reside.
  \MacroUNI{INCLUDE} would usually be defined relative to
  the \MacroUNI{RELEASE\_DIR} configuration macro.
  There is no default value, but
  if defined, it can make inclusion of necessary header files
  for compilation of programs (such as those programs
  that use \File{libcondorapi.a})
  easier through the use of \Condor{config\_val}.

\item[\Macro{SBIN}] \label{param:Sbin} This directory points to the
  Condor directory where Condor's system binaries (such as the
  binaries for the Condor daemons) and administrative tools are
  installed.  Whatever directory \MacroU{SBIN} points to ought
  to be in the \Env{PATH} of users acting as Condor
  administrators.  \Macro{SBIN} has no default value.

\item[\Macro{LOCAL\_DIR}] \label{param:LocalDir} The location of the
  local Condor directory on each machine in your pool.  One common
  option is to use the condor user's home directory which may be
  specified with \MacroUNI{TILDE}.  There is no default value for
  \Macro{LOCAL\_DIR}.  For example:
  \begin{verbatim}
    LOCAL_DIR = $(tilde)
  \end{verbatim}
  
  On machines with a shared file system, where either the
  \MacroUNI{TILDE} directory or another directory you want to use is
  shared among all machines in your pool, you might use the
  \MacroUNI{HOSTNAME} macro and have a directory with many
  subdirectories, one for each machine in your pool, each named by
  host names.  For example:
  \begin{verbatim}
    LOCAL_DIR = $(tilde)/hosts/$(hostname)      
  \end{verbatim}
  or:
  \begin{verbatim}
    LOCAL_DIR = $(release_dir)/hosts/$(hostname)
  \end{verbatim}
  
\item[\Macro{LOG}] \label{param:Log} Used to specify the
  directory where each Condor daemon writes its log files.  The names
  of the log files themselves are defined with other macros, which use
  the \MacroUNI{LOG} macro by default.  The log directory also acts as
  the current working directory of the Condor daemons as the run, so
  if one of them should produce a core file for any reason, it would
  be placed in the directory defined by this macro.  \MacroNI{LOG} is
  required to be defined.  Normally, \MacroUNI{LOG} is defined in
  terms of \MacroUNI{LOCAL\_DIR}.
  
\item[\Macro{SPOOL}] \label{param:Spool} The spool directory is where
  certain files used by the \Condor{schedd} are stored, such as the
  job queue file and the initial executables of any jobs that have
  been submitted.  In addition, for systems not using a checkpoint
  server, all the checkpoint files from jobs that have been submitted
  from a given machine will be store in that machine's spool
  directory.  Therefore, you will want to ensure that the spool
  directory is located on a partition with enough disk space.  If a
  given machine is only set up to execute Condor jobs and not submit
  them, it would not need a spool directory (or this macro defined).
  There is no default value for \Macro{SPOOL}, and the \Condor{schedd}
  will not function without it \Macro{SPOOL} defined.  Normally,
  \MacroUNI{SPOOL} is defined in terms of \MacroUNI{LOCAL\_DIR}.
  
\item[\Macro{EXECUTE}] \label{param:Execute} This directory acts as
  a place to create the scratch directory of any Condor job that is executing
  on
  the local machine.  The scratch directory is the destination of
  any input files that were specified for transfer.  It also serves
  as the job's working directory if the job is using file transfer
  mode and no other working directory was specified.
  If a given machine is set up to only submit
  jobs and not execute them, it would not need an execute directory,
  and this macro need not be defined.  There is no default value for
  \MacroNI{EXECUTE}, and the \Condor{startd} will not function if
  \MacroNI{EXECUTE} is undefined.  Normally, \MacroUNI{EXECUTE} is
  defined in terms of \MacroUNI{LOCAL\_DIR}.  To customize the execute
  directory independently for each batch slot, use \MacroNI{SLOTx\_EXECUTE}.

\item[\Macro{SLOTx\_EXECUTE}] \label{param:SlotXExecute} Specifies an
  execute directory for use by a specific batch slot.  (\emph{x}
  should be the number of the batch slot, such as 1, 2, 3, etc.)  This
  execute directory serves the same purpose as \Macro{EXECUTE}, but it
  allows you to configure the directory independently for each batch
  slot.  Having slots each using a different partition would be
  useful, for example, in preventing one job from filling up the same
  disk that other jobs are trying to write to.  If this parameter is
  undefined for a given batch slot, it will use \MacroNI{EXECUTE} as
  the default.  Note that each slot will advertise \AdAttr{TotalDisk}
  and \AdAttr{Disk} for the partition containing its execute
  directory.

\item[\Macro{LOCAL\_CONFIG\_FILE}] \label{param:LocalConfigFile}
  Identifies the
  location of the local, machine-specific configuration
  file for each machine
  in the pool.  The two most common choices would be putting this
  file in the \MacroUNI{LOCAL\_DIR}, or putting all
  local configuration files for the pool in a shared directory, each one
  named by host name.  For example,
  \begin{verbatim}
    LOCAL_CONFIG_FILE = $(LOCAL_DIR)/condor_config.local
  \end{verbatim}
  or,
  \begin{verbatim}
    LOCAL_CONFIG_FILE = $(release_dir)/etc/$(hostname).local
  \end{verbatim}
  or, not using the release directory
  \begin{verbatim}
    LOCAL_CONFIG_FILE = /full/path/to/configs/$(hostname).local
  \end{verbatim}
  
  The value of \MacroUNI{LOCAL\_CONFIG\_FILE} is treated as a list of files,
  not a
  single file.  The items in the list are delimited by either commas
  or space characters.
  This allows the specification of multiple files as
  the local configuration file, each one processed in the
  order given (with parameters set in later files overriding values
  from previous files).  This allows the use of one global
  configuration file for multiple platforms in the pool, defines a
  platform-specific configuration file for each platform, and uses a
  local configuration file for each machine. 
  If the list of files is changed in one of the later read files, the new list
  replaces the old list, but any files that have already been processed
  remain processed, and are removed from the new list if they are present
  to prevent cycles.
  See section~\ref{sec:Program-Defined-Macros} on 
  page~\pageref{sec:Program-Defined-Macros} for directions on
  using a program to generate the configuration macros that would
  otherwise reside in one or more files as described here.
  If \MacroNI{LOCAL\_CONFIG\_FILE} is not defined, no local configuration
  files are processed.  For more information on this, see
  section~\ref{sec:Multiple-Platforms} about Configuring Condor for
  Multiple Platforms on page~\pageref{sec:Multiple-Platforms}.

\item[\Macro{REQUIRE\_LOCAL\_CONFIG\_FILE}] \label{param:RequireLocalConfigFile}
  A boolean value that defaults to \Expr{True}.
  When \Expr{True}, Condor exits with an error,
  if any file listed in \MacroNI{LOCAL\_CONFIG\_FILE} cannot be read.
  A value of \Expr{False} allows local configuration files to be missing.
  This is most useful for sites that have 
  both large numbers of machines in the pool and a local configuration file
  that uses the \MacroUNI{HOSTNAME} macro in its definition.
  Instead of having an empty file for every host
  in the pool, files can simply be omitted.

\item[\Macro{LOCAL\_CONFIG\_DIR}] \label{param:LocalConfigDir} 
  Beginning in Condor 6.7.18, a directory may be used as a container for 
  local configuration files. 
  The files found in the directory are sorted into lexicographical order, and 
  then each file is treated as though it was listed in 
  \MacroNI{LOCAL\_CONFIG\_FILE}. 
  \MacroNI{LOCAL\_CONFIG\_DIR} is processed before any files listed in 
  \MacroNI{LOCAL\_CONFIG\_FILE}, and is checked again after processing
  the \MacroNI{LOCAL\_CONFIG\_FILE} list. 
  It is a list of directories, and each directory is processed in the order
  it appears in the list. 
  The process is not recursive, so any directories found inside the directory
  being processed are ignored. 

\item[\Macro{CONDOR\_IDS}] \label{param:CondorIds}
  The User ID (UID) and Group ID (GID) pair that the Condor daemons
  should run as, if the daemons are spawned as root.
  This value can also be specified in the \Env{CONDOR\_IDS}
  environment variable.
  If the Condor daemons are not started as root, then neither this
  \MacroNI{CONDOR\_IDS} configuration macro nor the \Env{CONDOR\_IDS}
  environment variable are used.
  The value is given by two integers, separated by a period.  For
  example, \verb@CONDOR_IDS = 1234.1234@.
  If this pair is not specified in either the configuration file or in the
  environment, and the Condor daemons are spawned as root,
  then Condor will
  search for a \verb@condor@ user on the system, and run as that user's
  UID and GID.
  See section~\ref{sec:uids} on UIDs in Condor for more details.

\item[\Macro{CONDOR\_ADMIN}] \label{param:CondorAdmin} The email
  address that Condor will send mail to if something goes wrong in
  your pool.  For example, if a daemon crashes, the \Condor{master}
  can send an \Term{obituary} to this address with the last few lines
  of that daemon's log file and a brief message that describes what
  signal or exit status that daemon exited with.  There is no default
  value for \Macro{CONDOR\_ADMIN}.
  
\item[\Macro{CONDOR\_SUPPORT\_EMAIL}] \label{param:CondorSupportEmail}
  The email address to be included at the bottom of all email Condor
  sends out under the label ``Email address of the local Condor
  administrator:''.  
  This is the address where Condor users at your site should send
  their questions about Condor and get technical support.
  If this setting is not defined, Condor will use the address
  specified in \MacroNI{CONDOR\_ADMIN} (described above).

\item[\Macro{MAIL}] \label{param:Mail} The full path to a mail
  sending program that uses \Opt{-s} to specify a subject for the
  message.  On all platforms, the default shipped with Condor should
  work.  Only if you installed things in a non-standard location on
  your system would you need to change this setting.  There is no
  default value for \MacroNI{MAIL}, and the \Condor{schedd} will not
  function unless \MacroNI{MAIL} is defined.

\item[\Macro{RESERVED\_SWAP}] \label{param:ReservedSwap} Determines
  how much swap space you want to reserve for your own machine.
  Condor will not start up more \Condor{shadow} processes if the
  amount of free swap space on your machine falls below this level.
  \MacroNI{RESERVED\_SWAP} is specified in megabytes.  The default value
  of \MacroNI{RESERVED\_SWAP} is 5 megabytes.

\item[\Macro{RESERVED\_DISK}] \label{param:ReservedDisk} Determines
  how much disk space you want to reserve for your own machine.  When
  Condor is reporting the amount of free disk space in a given
  partition on your machine, it will always subtract this amount.  An
  example is the \Condor{startd}, which advertises the amount of free
  space in the \MacroUNI{EXECUTE} directory.  The default value of
  \Macro{RESERVED\_DISK} is zero.
  
\item[\Macro{LOCK}] \label{param:Lock} Condor needs to create
  lock files to synchronize access to various log files.  Because of
  problems with network file systems and file locking over
  the years, we \emph{highly} recommend that you put these lock
  files on a local partition on each machine.  If you do not have your
  \MacroUNI{LOCAL\_DIR} on a local partition, be sure to change this
  entry.

  Whatever user or group Condor is running as needs to have
  write access to this directory.  If you are not running as root, this
  is whatever user you started up the \Condor{master} as.  If you are
  running as root, and there is a condor account, it is most
  likely condor.
  Otherwise, it is whatever you set in the \Env{CONDOR\_IDS}
  \index{environment variables!CONDOR\_IDS@\texttt{CONDOR\_IDS}}
  \index{CONDOR\_IDS@\texttt{CONDOR\_IDS}!environment variable}
  environment variable, or whatever you define in the
  \MacroNI{CONDOR\_IDS} setting in the Condor config files.
  See section~\ref{sec:uids} on UIDs in Condor for details.

  If no value for \MacroNI{LOCK} is provided, the value of \MacroNI{LOG}
  is used.


\item[\Macro{HISTORY}] \label{param:History} Defines the
  location of the Condor history file, which stores information about
  all Condor jobs that have completed on a given machine.  This macro
  is used by both the \Condor{schedd} which appends the information
  and \Condor{history}, the user-level program used to view
  the history file.
  This configuration macro is given the default value of
  \File{\$(SPOOL)/history} in the default configuration.
  If not defined,
  no history file is kept.
  % PKK
  % Described in default config file: YES
  % Defined in the default config file: YES 
  % Default definition in config file: $(SPOOL)/history
  % Result if not defined or RHS is empty: no history file is kept.

\item[\Macro{ENABLE\_HISTORY\_ROTATION}] \label{param:EnableHistoryRotation} 
  If this is defined to be true, then the
  history file will be rotated. If it is false, then it will not be
  rotated, and it will grow indefinitely, to the limits allowed by the
  operating system. If this is not defined, it is assumed to be
  true. The rotated files will be stored in the same directory as the
  history file. 

\item[\Macro{MAX\_HISTORY\_LOG}] \label{param:MaxHistoryLog}
  Defines the maximum size for the history file, in bytes. It defaults
  to 20MB. This parameter is only used if history file rotation is
  enabled. 

\item[\Macro{MAX\_HISTORY\_ROTATIONS}] \label{param:MaxHistoryRotations}
  When history file rotation is turned on, this controls how many
  backup files there are. It default to 2, which means that there may
  be up to three history files (two backups, plus the history file
  that is being currently written to). When the history file is
  rotated, and this rotation would cause the number of backups to be
  too large, the oldest file is removed. 

\item[\Macro{MAX\_JOB\_QUEUE\_LOG\_ROTATIONS}]
\label{param:MaxJobQueueLogRotations}
  The schedd periodically rotates the job queue database file in order
  to save disk space.  This option controls how many rotated files are
  saved.  It defaults to 1, which means there may be up to two history
  files (the previous one, which was rotated out of use, and the current one
  that is being written to).  When the job queue file is rotated,
  and this rotation would cause the number of backups to be larger
  the the maximum specified, the oldest file is removed.  The primary
  reason to save one or more rotated job queue files is if you are
  using Quill, and you want to ensure that Quill keeps an accurate history
  of all events logged in the job queue file.  Quill keeps track of where
  it last left off when reading logged events, so when the file is rotated,
  Quill will resume reading from where it last left off, provided that
  the rotated file still exists.  If Quill finds that it needs to read
  events from a rotated file that has been deleted, it will be forced to
  skip the missing events and resume reading in the next chronological job
  queue file that can be found.  Such an event should not lead to
  an inconsistency in Quill's view of the current queue contents, but it
  would create a inconsistency in Quill's record of the history of the
  job queue.

\item[\Macro{DEFAULT\_DOMAIN\_NAME}] \label{param:DefaultDomainName}
  If you do not use a fully qualified name in file \File{/etc/hosts}
  (or NIS, etc.) for either your official hostname or as an
  alias, Condor would not normally be able to use fully qualified names
  in places that it wants to.  You can set this macro to the
  domain to be appended to your hostname, if changing your host
  information is not a good option.  This macro must be set in the
  global configuration file (not the \MacroUNI{LOCAL\_CONFIG\_FILE}.
  The reason for this is that the special \MacroUNI{FULL\_HOSTNAME}
  macro is used by the configuration file code in Condor needs
  to know the full hostname.  So, for \MacroUNI{DEFAULT\_DOMAIN\_NAME} to
  take effect, Condor must already have read in its value.  However,
  Condor must set the \MacroUNI{FULL\_HOSTNAME} special macro since you
  might use that to define where your local configuration file is.  After
  reading the global configuration file, Condor figures out the right values
  for \MacroUNI{HOSTNAME} and \MacroUNI{FULL\_HOSTNAME} and inserts them
  into its configuration table.
  % PKK
  % Described in default config file: YES
  % Defined in the default config file: NO
  % Default definition in config file: bogus value
  % Result if not defined or RHS is empty: autodiscovered domain, if any, is used.

\item[\Macro{NO\_DNS}] \label{param:NoDNS}
  A boolean value that defaults to \Expr{False}.
  When \Expr{True}, Condor constructs host names using the host's IP address
  together with the value defined for \MacroNI{DEFAULT\_DOMAIN\_NAME}. 

\item[\Macro{CM\_IP\_ADDR}] \label{param:CMIPAddr}
  If neither \MacroNI{COLLECTOR\_HOST} nor 
  \MacroNI{COLLECTOR\_IP\_ADDR} macros are defined, then this
  macro will be used to determine the IP address of the central
  manager (collector daemon).
  This macro is defined by an IP address.
  % PKK
  % Described in default config file: NO
  % Defined in the default config file: NO
  % Default definition in config file: N/A
  % Result if not defined or RHS is empty: Condor performs above algorithm

\item[\Macro{EMAIL\_DOMAIN}] \label{param:EmailDomain}
  By default, if a user does not specify \AdAttr{notify\_user} in the
  submit description file, any email Condor sends about that job will
  go to "username@UID\_DOMAIN".
  If your machines all share a common UID domain (so that you would
  set \MacroNI{UID\_DOMAIN} to be the same across all machines in your
  pool), but email to user@UID\_DOMAIN is not the right place for
  Condor to send email for your site, you can define the default
  domain to use for email.
  A common example would be to set \MacroNI{EMAIL\_DOMAIN} to the fully
  qualified hostname of each machine in your pool, so users submitting
  jobs from a specific machine would get email sent to
  user@machine.your.domain, instead of user@your.domain.  
  You would do this by setting \MacroNI{EMAIL\_DOMAIN} to
  \MacroUNI{FULL\_HOSTNAME}. 
  In general, you should leave this setting commented out unless two
  things are true: 1) \MacroNI{UID\_DOMAIN} is set to your domain, not
  \MacroUNI{FULL\_HOSTNAME}, and 2) email to user@UID\_DOMAIN will not 
  work. 
  % PKK
  % Described in default config file: YES
  % Defined in the default config file: NO
  % Default definition in config file: bogus
  % Result if not defined or RHS is empty: 
  %	Condor will try to use the notify_user attribute email in the job ad.
  %	If that is not present, then it will use the UID_DOMAIN embedded in
  %	the job ad.
  %	If that is not present, then it will use the UID_DOMAIN found in the
  %	config file.
  %	If that is not present, then I suspect there is a bug and the code will
  %	segfault!!! (This needs fixing...)
  
\item[\Macro{CREATE\_CORE\_FILES}] \label{param:CreateCoreFiles}
  Defines whether or not Condor daemons are to
  create a core file in the \Macro{LOG} directory
  if something really bad happens.  It is
  used to set
  the resource limit for the size of a core file.  If not defined,
  it leaves in place whatever limit was in effect
  when the Condor daemons (normally the \Condor{master}) were started.
  This allows Condor to inherit the default system core file generation
  behavior at startup.  For Unix operating systems, this behavior can
  be inherited from the parent shell, or specified in a shell script
  that starts Condor.
  If this parameter is set and \Expr{True}, the limit is increased to
  the maximum.  If it is set to \Expr{False}, the limit is set at 0
  (which means that no core files are created).  Core files
  greatly help the Condor developers debug any problems you might be
  having.  By using the parameter, you do not have to worry about
  tracking down where in your boot scripts you need to set the core
  limit before starting Condor. You set the parameter
  to whatever behavior you want Condor to enforce.  This parameter
  defaults to undefined to allow the initial operating system default
  value to take precedence, 
  and is commented out in the default configuration file. 
  % PKK
  % Described in default config file: YES
  % Defined in the default config file: NO
  % Default definition in config file: bogus
  % Result if not defined or RHS is empty: shell's default corelimit size applies

\item[\Macro{CKPT\_PROBE}] \label{param:CkptProbe}
  Defines the path and executable name of the helper process Condor will use to
  determine information for the \Attr{CheckpointPlatform} attribute
  in the machine's ClassAd. 
  The default value is \File{\$(LIBEXEC)/condor\_ckpt\_probe}.

\item[\Macro{ABORT\_ON\_EXCEPTION}] \label{param:AbortOnException}
  When Condor programs detect a fatal internal exception, they
  normally log an error message and exit.  If you have turned on
  \Macro{CREATE\_CORE\_FILES}, in some cases you may also want to turn
  on \Macro{ABORT\_ON\_EXCEPTION} so that core files are generated
  when an exception occurs.  Set the following to True if that is what
  you want.

\item[\Macro{Q\_QUERY\_TIMEOUT}] \label{param:QQueryTimeout}
  Defines the timeout (in seconds) that \Condor{q} uses when trying to
  connect to the \Condor{schedd}.  Defaults to 20 seconds.
  % PKK
  % Described in default config file: NO
  % Defined in the default config file: NO
  % Default definition in config file: N/A
  % Result if not defined or RHS is empty: defaults to 20 seconds.

\item[\Macro{DEAD\_COLLECTOR\_MAX\_AVOIDANCE\_TIME}]
\label{param:DeadCollectorMaxAvoidanceTime} Defines the interval of time
  (in seconds) between checks for a failed primary \Condor{collector} daemon.
  If connections to the dead primary \Condor{collector} take very
  little time to fail, new attempts to query the primary \Condor{collector} may
  be more frequent than the specified maximum avoidance time.
  The default value equals one hour.
  This variable has relevance to flocked jobs, as it defines 
  the maximum time they may be reporting to the primary \Condor{collector}
  without the \Condor{negotiator} noticing.

\item[\Macro{PASSWD\_CACHE\_REFRESH}]
  \label{param:PasswdCacheRefresh}
  Condor can cause NIS servers to become overwhelmed by queries for uid
  and group information in large pools. In order to avoid this problem,
  Condor caches UID and group information internally. This integer value allows
  pool administrators to specify (in seconds) how long Condor should wait
  until refreshes a cache entry. The default is set to 300 seconds, or
  5 minutes, plus a random number of seconds between 0 and 60 to avoid
  having lots of processes refreshing at the same time.
  This means that if a pool administrator updates the user
  or group database (for example, \File{/etc/passwd} or \File{/etc/group}),
  it can take up
  to 6 minutes before Condor will have the updated information. This
  caching feature can be disabled by setting the refresh interval to
  0. In addition, the cache can also be flushed explicitly by running
  the command
  \begin{verbatim}
    condor_reconfig -full
  \end{verbatim}
  This configuration variable has no effect on Windows.
  % PKK
  % Described in default config file: NO
  % Defined in the default config file: NO
  % Default definition in config file: N/A
  % Result if not defined or RHS is empty: 300 seconds
\item[\Macro{SYSAPI\_GET\_LOADAVG}] \label{param:SysapiGetLoadavg}
  If set to False, then Condor will not attempt to compute the load average
  on the system, and instead will always report the system load average
  to be 0.0.  Defaults to True.

\item[\Macro{NETWORK\_MAX\_PENDING\_CONNECTS}] \label{param:NetworkMaxPendingConnects}
  This specifies a limit to the maximum number of simultaneous network
  connection attempts.  This is primarily relevant to \Condor{schedd},
  which may try to connect to large numbers of startds when claiming
  them.  The negotiator may also connect to large numbers of startds
  when initiating security sessions used for sending MATCH messages.  On
  Unix, the default for this parameter is eighty percent of the process file
  descriptor limit.  On windows, the default is 1600.

\item[\Macro{WANT\_UDP\_COMMAND\_SOCKET}] \label{param:WantUDPCommandSocket}
  This setting, added in version 6.9.5, controls if Condor daemons
  should create a UDP command socket in addition to the TCP command
  socket (which is required).
  The default is \Expr{True}, and modifying it requires restarting all
  Condor daemons, not just a \Condor{reconfig} or SIGHUP.

  Normally, updates sent to the \Condor{collector} use UDP, in
  addition to certain keep alive messages and other non-essential
  communication.
  However, in certain situations, it might be desirable to disable the
  UDP command port (for example, to reduce the number of ports
  represented by a GCB broker, etc).

  Unfortunately, due to a limitation in how these command sockets are
  created, it is not possible to define this setting on a per-daemon
  basis, for example, by trying to set
  \MacroNI{STARTD.WANT\_UDP\_COMMAND\_SOCKET}.
  At least for now, this setting must be defined machine wide to
  function correctly.

  If this setting is set to true on a machine running a
  \Condor{collector}, the pool should be configured to use TCP updates
  to that collector (see section~\ref{sec:tcp-collector-update} on
  page~\pageref{sec:tcp-collector-update} for more information).

\end{description}

% Default value check end here: PKK 8 Jan 2004

%%%%%%%%%%%%%%%%%%%%%%%%%%%%%%%%%%%%%%%%%%%%%%%%%%%%%%%%%%%%%%%%%%%%%%%%%%%
\subsection{\label{sec:Daemon-Logging-Config-File-Entries}Daemon Logging Configuration File Entries} 
%%%%%%%%%%%%%%%%%%%%%%%%%%%%%%%%%%%%%%%%%%%%%%%%%%%%%%%%%%%%%%%%%%%%%%%%%%%

\index{configuration!daemon logging configuration variables}
These entries control how and where the Condor daemons write to log
files.  Many of the entries in this section represents multiple
macros. There is one for each subsystem (listed
in section~\ref{sec:Condor-Subsystem-Names}).
The macro name for each substitutes \MacroNI{<SUBSYS>} with the name
of the subsystem corresponding to the daemon.
\begin{description}
  
\item[\MacroB{<SUBSYS>\_LOG}] \label{param:SubsysLog}
\index{SUBSYS\_LOG macro@\texttt{<SUBSYS>\_LOG} macro}
  The name of
  the log file for a given subsystem.  For example,
  \MacroUNI{STARTD\_LOG} gives the location of the log file for
  \Condor{startd}.

\item[\Macro{MAX\_<SUBSYS>\_LOG}] \label{param:MaxSubsysLog} Controls
  the maximum length in bytes to which a
  log will be allowed to grow.  Each log file will grow to the
  specified length, then be saved to a file with the suffix
  \File{.old}.  The \File{.old}
  files are overwritten each time the log is saved, thus the maximum
  space devoted to logging for any one program will be twice the
  maximum length of its log file.  A value of 0 specifies that the
  file may grow without bounds.  The default is 1 Mbyte.

\item[\Macro{TRUNC\_<SUBSYS>\_LOG\_ON\_OPEN}]
  \label{param:TruncSubsysLogOnOpen}  If this macro is defined and set
  to \Expr{True}, the affected log will be truncated and started from an
  empty file with each invocation of the program.  Otherwise, new
  invocations of the program will append to the previous log
  file.  By default this setting is \Expr{False} for all daemons. 

\item[\MacroB{<SUBSYS>\_LOCK}] \label{param:SubsysLock} 
\index{SUBSYS\_LOCK macro@\texttt{<SUBSYS>\_LOCK} macro}
This macro
  specifies the lock file used to synchronize append operations to the
  log file for this subsystem.  It must be a separate file from the
  \MacroUNI{<SUBSYS>\_LOG} file, since the \MacroUNI{<SUBSYS>\_LOG} file may be
  rotated and you want to be able to synchronize access across log
  file rotations.  A lock file is only required for log files which
  are accessed by more than one process.  Currently, this includes
  only the \MacroNI{SHADOW} subsystem.  This macro is defined relative
  to the \MacroUNI{LOCK} macro.

\item[\Macro{FILE\_LOCK\_VIA\_MUTEX}] \label{param:FileLockViaMutex} 
  This macro setting only works on Win32 -- it is ignored on Unix.  If set
  to be \Expr{True}, then log locking is implemented via a kernel mutex
  instead of via file locking.  On Win32, mutex access is FIFO, while
  obtaining a file lock is non-deterministic.  Thus setting to \Expr{True}
  fixes problems on Win32 where processes (usually shadows) could starve
  waiting for a lock on a log file.  Defaults to \Expr{True} on Win32, and is
  always \Expr{False} on Unix.

\item[\Macro{ENABLE\_USERLOG\_LOCKING}] \label{param:EnableUserlogLocking}
  When \Expr{True} (the default value),
  a user's job log (as specified in a submit description file)
  will be locked before being written to.
  If \Expr{False}, Condor will not lock the file before writing.

\item[\Macro{TOUCH\_LOG\_INTERVAL}] \label{param:TouchLogInterval}
  The time interval in seconds between when daemons touch
  their log files.  The change in last modification time for the
  log file is useful when a daemon restarts after failure or shut down.
  The last modification date is printed, and it provides an upper bound
  on the length of time that the daemon was not running.
  Defaults to 60 seconds.

\item[\Macro{LOGS\_USE\_TIMESTAMP}] \label{param:LogsUseTimestamp}
  This macro controls how the current time is formatted at the start of
  each line in the daemon log files. When \Expr{True}, the Unix time is
  printed (number of seconds since 00:00:00 UTC, January 1, 1970).
  When \Expr{False} (the default value), the time is printed like so:
  \Expr{<Month>/<Day> <Hour>:<Minute>:<Second>} in the local timezone.

\item[\MacroB{<SUBSYS>\_DEBUG}] \label{param:SubsysDebug}
\index{SUBSYS\_DEBUG macro@\texttt{<SUBSYS>\_DEBUG} macro}
  All of the
  Condor daemons can produce different levels of output depending on
  how much information is desired.  The various levels of
  verbosity for a given daemon are determined by this macro.  All
  daemons have the default level \Dflag{ALWAYS}, and log messages for
  that level will be printed to the daemon's log, regardless of this
  macro's setting.  Settings are a comma- or space-separated list
  of the following values:

  \begin{description}
    \label{list:debug-level-description}

  \item[\Dflag{ALL}] \label{dflag:all}
    \index{SUBSYS\_DEBUG macro levels@\texttt{<SUBSYS>\_DEBUG} macro levels!D\_ALL@\texttt{D\_ALL}}
    This flag turns on \emph{all} debugging output by enabling all of the debug
    levels at once.  There is no need to list any other debug levels in addition
    to \Dflag{ALL}; doing so would be redundant.  Be warned: this will
    generate
    about a \emph{HUGE} amount of output.
    To obtain a higher
    level of output than the default, consider using \Dflag{FULLDEBUG} before
    using this option.

  \item[\Dflag{FULLDEBUG}] \label{dflag:fulldebug}
    \index{SUBSYS\_DEBUG macro levels@\texttt{<SUBSYS>\_DEBUG} macro levels!D\_FULLDEBUG@\texttt{D\_FULLDEBUG}}
    This level
    provides verbose output of a general nature into the log files.  
    Frequent log messages for very specific debugging
    purposes would be excluded. In those cases, the messages would
    be viewed by having that another flag and \Dflag{FULLDEBUG} both
    listed in the configuration file.

  \item[\Dflag{DAEMONCORE}] \label{dflag:daemoncore} 
    \index{SUBSYS\_DEBUG macro levels@\texttt{<SUBSYS>\_DEBUG} macro levels!D\_DAEMONCORE@\texttt{D\_DAEMONCORE}}
    Provides log
    file entries specific to DaemonCore, such as
    timers the daemons have set and the commands that are registered.
    If both \Dflag{FULLDEBUG} and \Dflag{DAEMONCORE} are set,
    expect \emph{very} verbose output.

  \item[\Dflag{PRIV}] \label{dflag:priv}
    \index{SUBSYS\_DEBUG macro levels@\texttt{<SUBSYS>\_DEBUG} macro levels!D\_PRIV@\texttt{D\_PRIV}}
    This flag provides log
    messages about the \Term{privilege state} switching that the daemons
    do.  See section~\ref{sec:uids} on UIDs in Condor for details.

  \item[\Dflag{COMMAND}] \label{dflag:command}
    \index{SUBSYS\_DEBUG macro levels@\texttt{<SUBSYS>\_DEBUG} macro levels!D\_COMMAND@\texttt{D\_COMMAND}}
    With this flag set, any
    daemon that uses DaemonCore will print out a log message
    whenever a command comes in.  The name and integer of the command,
    whether the command was sent via UDP or TCP, and where
    the command was sent from are all logged.  
    Because the messages about the command used by \Condor{kbdd} to
    communicate with the \Condor{startd} whenever there is activity on
    the X server, and the command used for keep-alives are both only
    printed with \Dflag{FULLDEBUG} enabled, it is best if this setting
    is used for all daemons.

  \item[\Dflag{LOAD}] \label{dflag:load}
    \index{SUBSYS\_DEBUG macro levels@\texttt{<SUBSYS>\_DEBUG} macro levels!D\_LOAD@\texttt{D\_LOAD}}
    The \Condor{startd} keeps track
    of the load average on the machine where it is running.  Both the
    general system load average, and the load average being generated by
    Condor's activity there are determined.
    With this flag set, the \Condor{startd}
    will log a message with the current state of both of these
    load averages whenever it computes them.  This flag only affects the
    \Condor{startd}.

  \item[\Dflag{KEYBOARD}] \label{dflag:keyboard} 
    \index{SUBSYS\_DEBUG macro levels@\texttt{<SUBSYS>\_DEBUG} macro levels!D\_KEYBOARD@\texttt{D\_KEYBOARD}}
    With this flag set, the \Condor{startd} will print out a log message
    with the current values for remote and local keyboard idle time.
    This flag affects only the \Condor{startd}.

  \item[\Dflag{JOB}] \label{dflag:job}
    \index{SUBSYS\_DEBUG macro levels@\texttt{<SUBSYS>\_DEBUG} macro levels!D\_JOB@\texttt{D\_JOB}}
    When this flag is set, the
    \Condor{startd} will send to its log file the contents of any
    job ClassAd that the \Condor{schedd} sends to claim the
    \Condor{startd} for its use.  This flag affects only the
    \Condor{startd}.
    
  \item[\Dflag{MACHINE}] \label{dflag:machine}
    \index{SUBSYS\_DEBUG macro levels@\texttt{<SUBSYS>\_DEBUG} macro levels!D\_MACHINE@\texttt{D\_MACHINE}}
    When this flag is set,
    the \Condor{startd} will send to its log file the contents of
    its resource ClassAd when the \Condor{schedd} tries to claim the
    \Condor{startd} for its use.  This flag affects only the
    \Condor{startd}.

  \item[\Dflag{SYSCALLS}] \label{dflag:syscalls}
    \index{SUBSYS\_DEBUG macro levels@\texttt{<SUBSYS>\_DEBUG} macro levels!D\_SYSCALLS@\texttt{D\_SYSCALLS}}
    This flag is used to
    make the \Condor{shadow} log remote syscall requests and return
    values.  This can help track down problems a user is having with a
    particular job by providing the system calls the job is
    performing. If any are failing, the reason for the
    failure is given.  The \Condor{schedd} also uses this flag for the server
    portion of the queue management code.  With \Dflag{SYSCALLS}
    defined in \MacroNI{SCHEDD\_DEBUG} there will be verbose logging of all
    queue management operations the \Condor{schedd} performs.  

  \item[\Dflag{MATCH}] \label{dflag:match}
    \index{SUBSYS\_DEBUG macro levels@\texttt{<SUBSYS>\_DEBUG} macro levels!D\_MATCH@\texttt{D\_MATCH}}
    When this flag is
    set, the \Condor{negotiator} logs a message for every match.

  \item[\Dflag{NETWORK}] \label{dflag:network}
    \index{SUBSYS\_DEBUG macro levels@\texttt{<SUBSYS>\_DEBUG} macro levels!D\_NETWORK@\texttt{D\_NETWORK}}
    When this flag is set,
    all Condor daemons will log a message on every TCP accept, connect,
    and close, and on every UDP send and receive.  This flag is not
    yet fully supported in the \Condor{shadow}.

  \item[\Dflag{HOSTNAME}] \label{dflag:hostname}
    \index{SUBSYS\_DEBUG macro levels@\texttt{<SUBSYS>\_DEBUG} macro levels!D\_HOSTNAME@\texttt{D\_HOSTNAME}}
    When this flag is set, the Condor daemons and/or tools will print
    verbose messages explaining how they resolve host names, domain
    names, and IP addresses.
    This is useful for sites that are having trouble getting Condor to
    work because of problems with DNS, NIS or other host name resolving
    systems in use.

  \item[\Dflag{CKPT}] \label{dflag:ckpt}
    \index{SUBSYS\_DEBUG macro levels@\texttt{<SUBSYS>\_DEBUG} macro levels!D\_CKPT@\texttt{D\_CKPT}}
    When this flag is set,
    the Condor process checkpoint support code, which is linked into a STANDARD 
    universe user job, will output some low-level details about the checkpoint
    procedure into the \MacroUNI{SHADOW\_LOG}.

  \item[\Dflag{SECURITY}] \label{dflag:security}
    \index{SUBSYS\_DEBUG macro levels@\texttt{<SUBSYS>\_DEBUG} macro levels!D\_SECURITY@\texttt{D\_SECURITY}}
    This flag will enable debug messages pertaining to the setup of 
    secure network communication, 
    including messages for the negotiation of a socket 
    authentication mechanism, the management of a session key cache.
    and messages about the authentication process itself.  See
    section~\ref{sec:Config-Security} for more information about
    secure communication configuration.

  \item[\Dflag{PROCFAMILY}] \label{dflag:procfamily}
    \index{SUBSYS\_DEBUG macro levels@\texttt{<SUBSYS>\_DEBUG} macro levels!D\_PROCFAMILY@\texttt{D\_PROCFAMILY}}
    Condor often times needs to manage an entire family of processes, (that
    is, a 
    process and all descendants of that process).  This debug flag will 
    turn on debugging output for the management of families of processes.

  \item[\Dflag{ACCOUNTANT}] \label{dflag:accountant}
    \index{SUBSYS\_DEBUG macro levels@\texttt{<SUBSYS>\_DEBUG} macro levels!D\_ACCOUNTANT@\texttt{D\_ACCOUNTANT}}
    When this flag is set,
    the \Condor{negotiator} will output debug messages relating to the computation
    of user priorities (see section~\ref{sec:UserPrio}).

  \item[\Dflag{PROTOCOL}] \label{dflag:protocol}
    \index{SUBSYS\_DEBUG macro levels@\texttt{<SUBSYS>\_DEBUG} macro levels!D\_PROTOCOL@\texttt{D\_PROTOCOL}}
    Enable debug messages relating to the protocol for Condor's matchmaking and
    resource claiming framework.
    
  \item[\Dflag{PID}] \label{dflag:pid}
    \index{SUBSYS\_DEBUG macro levels@\texttt{<SUBSYS>\_DEBUG} macro levels!D\_PID@\texttt{D\_PID}}
    This flag is different from the other flags, because it is
    used to change the formatting of all log messages that are printed,
    as opposed to specifying what kinds of messages should be printed.
    If \Dflag{PID} is set, Condor will always print out the process
    identifier (PID) of the process writing each line to the log file.
    This is especially helpful for Condor daemons that can fork
    multiple helper-processes (such as the \Condor{schedd} or
    \Condor{collector}) so the log file will clearly show which thread
    of execution is generating each log message.
    
  \item[\Dflag{FDS}] \label{dflag:fds}
    \index{SUBSYS\_DEBUG macro levels@\texttt{<SUBSYS>\_DEBUG} macro levels!D\_FDS@\texttt{D\_FDS}}
    This flag is different from the other flags, because it is
    used to change the formatting of all log messages that are printed,
    as opposed to specifying what kinds of messages should be printed.
    If \Dflag{FDS} is set, Condor will always print out the file descriptor
    that the open of the log file was allocated by the operating system.
    This can be helpful in debugging Condor's use of system file
    descriptors as it will generally track the number of file descriptors
    that Condor has open.

    
  \end{description}

\item[\Macro{ALL\_DEBUG}] \label{param:AllDebug} Used to make all subsystems
  share a debug flag. Set the parameter \MacroNI{ALL\_DEBUG}
  instead of changing all of the individual parameters.  For example,
  to turn on all debugging in all subsystems, set
  \verb$ALL_DEBUG = D_ALL$.

\item[\Macro{TOOL\_DEBUG}] \label{param:ToolDebug} Uses the same
  values (debugging levels) as \MacroNI{<SUBSYS>\_DEBUG} to
  describe the amount of debugging information sent to \File{stderr} 
  for Condor tools.

\item[\Macro{SUBMIT\_DEBUG}] \label{param:SubmitDebug} Uses the same
  values (debugging levels) as \MacroNI{<SUBSYS>\_DEBUG} to
  describe the amount of debugging information sent to \File{stderr} 
  for \Condor{submit}.

\end{description}

Log files may optionally be specified per debug level as follows:
\begin{description}

\item[\MacroB{<SUBSYS>\_<LEVEL>\_LOG}] \label{param:SubsysLevelLog}
\index{SUBSYS\_LEVEL\_LOG macro@\texttt{<SUBSYS>\_<LEVEL>\_LOG} macro}
  This is
  the name of a log file for messages at a specific debug level for a
  specific subsystem.  If the debug level is included in
  \MacroUNI{<SUBSYS>\_DEBUG}, then all messages of this debug level will be
  written both to the \MacroUNI{<SUBSYS>\_LOG} file and the
  \MacroUNI{<SUBSYS>\_<LEVEL>\_LOG} file.  For example,
  \MacroUNI{SHADOW\_SYSCALLS\_LOG} specifies a log file for all remote
  system call debug messages.

\item[\Macro{MAX\_<SUBSYS>\_<LEVEL>\_LOG}] \label{param:MaxSubsysLevelLog}
  Similar to \Macro{MAX\_<SUBSYS>\_LOG}.

\item[\Macro{TRUNC\_<SUBSYS>\_<LEVEL>\_LOG\_ON\_OPEN}]
  \label{param:TruncSubsysLevelLogOnOpen} Similar to
  \Macro{TRUNC\_<SUBSYS>\_LOG\_ON\_OPEN}.

\end{description}

The following macros control where and what is written to the 
event log,
a file that receives job user log events, 
but across all users and user's jobs.

\begin{description}

\item[\Macro{EVENT\_LOG}] \label{param:EventLog}
  The full path and file name of the event log.
  There is no default value for this variable,
  so no event log will be written, if not defined.

\item[\Macro{MAX\_EVENT\_LOG}] \label{param:MaxEventLog}
  Controls the maximum length in bytes to which the event log
  will be allowed to grow. The log file will grow to the specified length,
  then be saved to a file with the suffix .old.
  The .old  files are overwritten each time the log is saved.
  A value of 0 specifies that the file may grow without bounds.
  The default is 1 Mbyte.
  
\item[\Macro{EVENT\_LOG\_USE\_XML}] \label{param:EventLogUseXML}
  A boolean value that defaults to \Expr{False}.
  When \Expr{True}, events are logged in XML format.

\item[\Macro{EVENT\_LOG\_JOB\_AD\_INFORMATION\_ATTRS}]
  \label{param:EventLogJobAdInformationAttrs}
  A comma-separated list of job ClassAd attributes,
  whose evaluated values form a new event, the JobAdInformationEvent.
  This new event is placed in the event log in addition to each logged event.

\end{description}

%%%%%%%%%%%%%%%%%%%%%%%%%%%%%%%%%%%%%%%%%%%%%%%%%%%%%%%%%%%%%%%%%%%%%%%%%%%
\subsection{\label{sec:DaemonCore-Config-File-Entries}DaemonCore Configuration File Entries} 
%%%%%%%%%%%%%%%%%%%%%%%%%%%%%%%%%%%%%%%%%%%%%%%%%%%%%%%%%%%%%%%%%%%%%%%%%%%

\index{configuration!DaemonCore configuration variables}
Please read section~\ref{sec:DaemonCore} for details
on DaemonCore.  There are certain configuration file settings that
DaemonCore uses which affect all Condor daemons (except the checkpoint
server, shadow, and starter, none of which use DaemonCore yet).
\begin{description}

\item[\Macro{HOSTALLOW\Dots}] \label{param:HostAllow} All
  macros that begin with either \Macro{HOSTALLOW} or
  \Macro{HOSTDENY} are settings for Condor's host-based security.
  See section~\ref{sec:Host-Security} on Setting up
  IP/host-based security in Condor for details on these
  macros and how to configure them.

\item[\Macro{ENABLE\_RUNTIME\_CONFIG}]
  \label{param:EnableRuntimeConfig}
  The \Condor{config\_val} tool has an option \Opt{-rset} for
  dynamically setting runtime configuration values (which only effect
  the in-memory configuration variables).
  Because of the potential security implications of this feature, by
  default, Condor daemons will not honor these requests.
  To use this functionality, Condor administrators must specifically
  enable it by setting \MacroNI{ENABLE\_RUNTIME\_CONFIG} to \Expr{True}, and
  specify what configuration variables can be changed using the
  \MacroNI{SETTABLE\_ATTRS\Dots} family of configuration options
  (described below).
  Defaults to \Expr{False}.

\item[\Macro{ENABLE\_PERSISTENT\_CONFIG}]
  \label{param:EnablePersistentConfig}
  The \Condor{config\_val} tool has a \Opt{-set} option for
  dynamically setting persistent configuration values.
  These values override options in the normal Condor configuration
  files.
  Because of the potential security implications of this feature, by
  default, Condor daemons will not honor these requests.
  To use this functionality, Condor administrators must specifically
  enable it by setting \MacroNI{ENABLE\_PERSISTENT\_CONFIG} to \Expr{True},
  creating a directory where the Condor daemons will hold these
  dynamically-generated persistent configuration files (declared using
  \MacroNI{PERSISTENT\_CONFIG\_DIR}, described below) and specify what
  configuration variables can be changed using the
  \MacroNI{SETTABLE\_ATTRS\Dots} family of configuration options
  (described below).
  Defaults to \Expr{False}.

\item[\Macro{PERSISTENT\_CONFIG\_DIR}]
  \label{param:PersistentConfigDir}
  Directory where daemons should store dynamically-generated
  persistent configuration files (used to support
  \Condor{config\_val} \Opt{-set})
  This directory should \Bold{only} be writable by root, or the user
  the Condor daemons are running as (if non-root).
  There is no default, administrators that wish to use this
  functionality must create this directory and define this setting.
  This directory must not be shared by multiple Condor installations,
  though it can be shared by all Condor daemons on the same host.
  Keep in mind that this directory should not be placed on an NFS
  mount where ``root-squashing'' is in effect, or else Condor daemons
  running as root will not be able to write to them.
  A directory (only writable by root) on the local file system is
  usually the best location for this directory.

\item[\Macro{SETTABLE\_ATTRS\Dots}] \label{param:SettableAttrs} All
  macros that begin with \Macro{SETTABLE\_ATTRS} or
  \MacroNI{<SUBSYS>\_SETTABLE\_ATTRS} are settings used to restrict the 
  configuration values that can be changed using the \Condor{config\_val} 
  command.
  Section~\ref{sec:Host-Security} on Setting up
  IP/Host-Based Security in Condor for details on these
  macros and how to configure them.  
  In particular, section~\ref{sec:Host-Security}
  on page~\pageref{sec:Host-Security} contains details specific to
  these macros.

\item[\Macro{SHUTDOWN\_GRACEFUL\_TIMEOUT}]
  \label{param:ShutdownGracefulTimeout} Determines how long
  Condor will allow daemons try their graceful shutdown methods
  before they do a hard shutdown.  It is defined in terms of seconds.
  The default is 1800 (30 minutes).

  % This macro existed for 6.2.x, but does not exist any more.
  %\item[\Macro{AUTHENTICATION\_METHODS}]\label{param:AuthenticationMethods}
  %  There are many instances when the Condor system needs to authenticate
  %  the identity of the user.  For instance, when a job is submitted with
  %  \Condor{submit}, Condor needs to authenticate the user so that the job
  %  goes into the queue and runs with the proper credentials.  The
  %  \MacroNI{AUTHENTICATION\_METHODS} parameter is a list of
  %  permitted authentication methods.  The list is ordered by
  %  preference.  The actual authentication method used is the first method
  %  in this list that both the server and client are able to perform.
  %  Possible values are:
  %  \begin{itemize}
  %	\item NTSSPI Use NT's standard LAN-MANAGER challenge-response protocol.
  %	\Note This is the default method used on Windows NT.
  %	\item FS Use the filesystem to authenticate the user.  
  %		The server requests the client to create a specified temporary
  %		file, then the server verifies the ownership of that file. \Note
  %		This is the default method used on Unix systems.
  %	\item FS\_REMOTE Use a shared filesystem to authenticate the user.
  %		This is useful for submitting jobs to a remote \Condor{schedd}.
  %		Similar to FS authentication, except the temporary file to be
  %		created by the user must be on a shared filesystem (AFS, NFS, etc.)
  %		If the client's submit description file does not define the 
  %		command \Opt{rendezvousdir}, the \Opt{initialdir} value is used 
  %		as the default directory in which to create the temporary file.
  %		\Note Normal AFS issues apply here: Condor must be able to write
  %		to the directory used.
  % commented out by Karen for 6.4.3
  %       \item GSS Use Generic Security Services, which is implemented in Condor 
  %		with X.509 certificates. See section~\ref{sec:X509-Authentication}.
  %		These X.509 certificates are compatible with the Globus system from
  %		Argonne National Labs.
  %	\item CLAIMTOBE The server should simply trust the client.  
  %		\Note You had better trust all users who have access to your Condor
  %		pool if you enable CLAIMTOBE authentication.
  %  \end{itemize}

\item[\MacroB{<SUBSYS>\_ADDRESS\_FILE}]
  \label{param:SubsysAddressFile}
  \index{SUBSYS\_ADDRESS\_FILE macro@\texttt{<SUBSYS>\_ADDRESS\_FILE} macro}
  \index{NEGOTIATOR\_ADDRESS\_FILE macro@\texttt{NEGOTIATOR\_ADDRESS\_FILE} macro}
  \index{configuration macro!\texttt{NEGOTIATOR\_ADDRESS\_FILE}}
  \index{COLLECTOR\_ADDRESS\_FILE macro@\texttt{COLLECTOR\_ADDRESS\_FILE} macro}
  \index{configuration macro!\texttt{COLLECTOR\_ADDRESS\_FILE}}
  A complete path to a file that is to contain an
  IP address and port number for a daemon. 
  Every Condor daemon that uses
  DaemonCore has a command port where commands are sent.
  The IP/port of the daemon is put in that daemon's ClassAd,
  so that other machines in the pool can query the
  \Condor{collector} (which listens on a well-known port)
  to find the address of a given daemon on a given machine.
  When tools and daemons are all executing on the same
  single machine, communications do not require a query of the
  \Condor{collector} daemon.
  Instead, they look in a file on the local disk
  to find the IP/port.
  This macro causes daemons to write the
  IP/port of their command socket to a specified file.
  In this way,
  local tools will continue to operate,
  even if the machine running the \Condor{collector} crashes.
  Using this file will also generate
  slightly less network traffic in the pool,
  since tools including \Condor{q} and
  \Condor{rm} do not need to send any messages over the network to
  locate the \Condor{schedd} daemon.
  This macro is not necessary for the \Condor{collector} 
  daemon, since its command socket is at a well-known port.  
  
  The macro is named by substituting \MacroNI{<SUBSYS>}
  with the appropriate subsystem string as defined in
  section~\ref{sec:Condor-Subsystem-Names}.
  
\item[\MacroB{<SUBSYS>\_DAEMON\_AD\_FILE}]
  \label{param:SubsysDaemonAdFile}
  \index{SUBSYS\_DAEMON\_AD\_FILE macro@\texttt{<SUBSYS>\_DAEMON\_AD\_FILE} macro}
  A complete path to a file that is to contain the ClassAd for a daemon.
  When the daemon sends a ClassAd describing itself to the
  \Condor{collector}, it will also place a copy of the ClassAd in this
  file. Currently, this setting only works for the \Condor{schedd}
  (that is \Macro{SCHEDD\_DAEMON\_AD\_FILE}) and is required for Quill.

\item[\MacroB{<SUBSYS>\_ATTRS} or \label{param:SubsysExprs}
  \MacroB{<SUBSYS>\_EXPRS}] \label{param:SubsysAttrs}
  \index{SUBSYS\_ATTRS macro@\texttt{<SUBSYS>\_ATTRS} macro}
  \index{SUBSYS\_EXPRS macro@\texttt{<SUBSYS>\_EXPRS} macro}
  Allows any DaemonCore daemon to advertise arbitrary
  expressions from the configuration file in its ClassAd.  Give the
  comma-separated list of entries from the configuration file you want in the
  given daemon's ClassAd.
  Frequently used to add attributes to machines so that the
  machines can discriminate between other machines in a job's 
  \Opt{rank} and \Opt{requirements}.

  The macro is named by substituting \MacroNI{<SUBSYS>}
  with the appropriate subsystem string as defined in
  section~\ref{sec:Condor-Subsystem-Names}.

  \MacroNI{<SUBSYS>\_EXPRS} is a historic setting that functions identically to
  \MacroNI{<SUBSYS>\_ATTRS}. Use \MacroNI{<SUBSYS>\_ATTRS}.

  \Note The \Condor{kbdd} does not send
  ClassAds now, so this entry does not affect it.  The
  \Condor{startd}, \Condor{schedd}, \Condor{master}, and
  \Condor{collector} do send ClassAds, so those would be valid
  subsystems to set this entry for.
  
  \MacroNI{SUBMIT\_EXPRS} not part of the \MacroNI{<SUBSYS>\_EXPRS}, it is
  documented in section~\ref{sec:Submit-Config-File-Entries}

  Because of the different syntax of the configuration
  file and ClassAds, a little extra work is required to get a
  given entry into a ClassAd.  In particular, ClassAds require quote
  marks (") around strings.  Numeric values and boolean expressions
  can go in directly.  
  For example, if the \Condor{startd} is to advertise a string macro, a numeric
  macro, and a boolean expression, do something similar to:

  \begin{verbatim}
    STRING = This is a string 
    NUMBER = 666
    BOOL1 = True
    BOOL2 = CurrentTime >= $(NUMBER) || $(BOOL1)
    MY_STRING = "$(STRING)"
    STARTD_ATTRS = MY_STRING, NUMBER, BOOL1, BOOL2
  \end{verbatim}

\item[\Macro{DAEMON\_SHUTDOWN}] \label{param:DaemonShutdown}
  Starting with Condor version 6.9.3, whenever a daemon is about to
  publish a ClassAd update to the \Condor{collector}, it will evaluate
  this expression.
  If it evaluates to \Expr{True}, the daemon will gracefully shut itself down,
  exit with the exit code 99,
  and will not be restarted by the \Condor{master} (as if it sent
  itself a \Condor{off} command).
  The expression is evaluated in the context of the ClassAd that is
  being sent to the \Condor{collector}, so it can reference any
  attributes that can be seen with
  \verb@condor_status -long [-daemon_type]@ (for example,
  \verb@condor_status -long [-master]@ for the \Condor{master}).
  Since each daemon's ClassAd will contain different attributes,
  administrators should define these shutdown expressions specific to
  each daemon, for example:
  \begin{verbatim}
    STARTD.DAEMON_SHUTDOWN = when to shutdown the startd
    MASTER.DAEMON_SHUTDOWN = when to shutdown the master
  \end{verbatim}
  Normally, these expressions would not be necessary, so if not
  defined, they default to FALSE.
  One possible use case is for Condor glide-in, to have the
  \Condor{startd} shut itself down if it has not been claimed by a job
  after a certain period of time.

  \Note This functionality does not work in conjunction with Condor's
  high-availability support (see section~\ref{sec:high-availability}
  on page~\pageref{sec:high-availability} for more information).
  If you enable high-availability for a particular daemon, you should
  not define this expression.

\item[\Macro{DAEMON\_SHUTDOWN\_FAST}] \label{param:DaemonShutdownFast}
  Identical to \MacroNI{DAEMON\_SHUTDOWN} (defined above), except the
  daemon will use the fast shutdown mode (as if it sent itself a
  \Condor{off} command using the \Opt{-fast} option).

\item[\Macro{USE\_CLONE\_TO\_CREATE\_PROCESSES}] \label{param:UseCloneToCreateProcesses}
  This setting controls how a Condor daemon creates a new process under
  certain versions of Linux. If set to \Expr{True} (the default value),
  the \Expr{clone} system call is used. Otherwise, the \Expr{fork} system
  call is used. \Expr{clone} provides scalability improvements for daemons
  using a large amount of memory (e.g. a \Condor{schedd} with a lot of
  jobs in the queue). Currently, the use of \Expr{clone} is available on
  Linux systems other than IA-64, but not when GCB is enabled.

\end{description}

%%%%%%%%%%%%%%%%%%%%%%%%%%%%%%%%%%%%%%%%%%%%%%%%%%%%%%%%%%%%%%%%%%%%%%%%%%%
\subsection{\label{sec:Network-Related-Config-File-Entries}Network-Related Configuration File Entries}
%%%%%%%%%%%%%%%%%%%%%%%%%%%%%%%%%%%%%%%%%%%%%%%%%%%%%%%%%%%%%%%%%%%%%%%%%%%
\index{configuration!network-related configuration variables}

More information about networking in Condor can be found in
section~\ref{sec:Networking} on page~\pageref{sec:Networking}.

\begin{description}

\item[\Macro{BIND\_ALL\_INTERFACES}] \label{param:BindAllInterfaces}
  For systems with multiple network interfaces, if this configuration
  setting is set to FALSE, Condor will only bind network sockets to 
  the IP address specified with
  \MacroNI{NETWORK\_INTERFACE} (described below).  If set to TRUE,
  the default value, condor will listen on all interfaces.
  However, currently Condor is still only able to advertise a single
  IP address, even if it is listening on multiple interfaces.  By
  default, it will advertise the IP address of the network interface
  used to contact the collector, since this is the most likely to be
  accessible to other processes which query information from the same
  collector.
  More information about using this setting can be found in
  section~\ref{sec:Using-BindAllInterfaces} on
  page~\pageref{sec:Using-BindAllInterfaces}. 

\item[\Macro{NETWORK\_INTERFACE}] \label{param:NetworkInterface}
  For systems with multiple network interfaces, if this configuration
  setting is not defined, Condor binds all network sockets to first
  interface found.
  To bind to a specific network interface other than the
  first one, this \MacroNI{NETWORK\_INTERFACE} should be set to 
  the IP address to use.
  When \MacroNI{BIND\_ALL\_INTERFACES} is set to \Expr{True} (the
  default), this
  setting simply controls what IP address a given Condor host will
  advertise.
  More information about configuring Condor on machines with multiple
  network interfaces can be found in
  section~\ref{sec:Multiple-Interfaces} on
  page~\pageref{sec:Multiple-Interfaces}.
  % PKK
  % Described in default config file: NO
  % Defined in the default config file: NO
  % Default definition in config file: N/A
  % Result if not defined or RHS is empty: Condor binds to first interface found

\item[\Macro{PRIVATE\_NETWORK\_NAME}] \label{param:PrivateNetworkName}
  If two Condor daemons are trying to communicate with each other, and
  they both belong to the same private network, this setting will
  allow them to communicate directly using the private network
  interface, instead of having to use the Generic Connection Broker
  (GCB) or to go through a public IP address.
  Each private network should be assigned a unique network name.
  This string can have any form, but it must be unique for a
  particular private network.
  If another Condor daemon or tool is configured with the same
  \MacroNI{PRIVATE\_NETWORK\_NAME}, it will attempt to contact this
  daemon using the \Attr{PrivateIpAddr} attribute from the classified
  ad.
  Even for sites using GCB, this is an important optimization, since
  it means that two daemons on the same network can communicate
  directly, without having to go through the GCB broker.
  If GCB is enabled, and the \MacroNI{PRIVATE\_NETWORK\_NAME} is
  defined, the \Attr{PrivateIpAddr} will be defined automatically.
  Otherwise, you can specify a particular private IP address to use by
  defining the \MacroNI{PRIVATE\_NETWORK\_INTERFACE} setting
  (described below).
  There is no default for this setting.
  % DWW
  % Described in default config file: NO
  % Defined in the default config file: NO
  % Default definition in config file: N/A
  % Result if not defined or RHS is empty: No change in behavior.

\item[\Macro{PRIVATE\_NETWORK\_INTERFACE}] \label{param:PrivateNetworkInterface}
  For systems with multiple network interfaces, if this configuration
  setting and \MacroNI{PRIVATE\_NETWORK\_NAME} are both defined,
  Condor daemons will advertise some additional attributes in their
  ClassAds to help other Condor daemons and tools in the same private
  network to communicate directly.
  The \MacroNI{PRIVATE\_NETWORK\_INTERFACE} defines what IP address a
  given multi-homed machine should use for the private network.
  If another Condor daemon or tool is configured with the same
  \MacroNI{PRIVATE\_NETWORK\_NAME}, it will attempt to contact this
  daemon using the IP address specified here.
  Sites using the Generic Connection Broker (GCB) only need to define
  the \MacroNI{PRIVATE\_NETWORK\_NAME}, and the 
  \MacroNI{PRIVATE\_NETWORK\_INTERFACE} will be defined automatically.
  Unless GCB is enabled, there is no default for this setting.
  % DWW
  % Described in default config file: NO
  % Defined in the default config file: NO
  % Default definition in config file: N/A
  % Result if not defined or RHS is empty: No change in behavior.

\item[\Macro{HIGHPORT}] \label{param:HighPort}
  Specifies an upper limit of given port numbers for Condor to use,
  such that Condor is restricted to a range of port numbers.
  If this macro is not explicitly specified, then Condor will
  not restrict the port numbers that it uses. Condor will use
  system-assigned port numbers.
  For this macro to work, both \MacroNI{HIGHPORT} and
  \MacroNI{LOWPORT} (given below) must be defined.
  % PKK
  % Described in default config file: YES
  % Defined in the default config file: NO
  % Default definition in config file: bogus
  % Result if not defined or RHS is empty: Condor uses any ports available,
  % regardless of the LOWPORT setting

\item[\Macro{LOWPORT}] \label{param:LowPort}
  Specifies a lower limit of given port numbers for Condor to use,
  such that Condor is restricted to a range of port numbers.
  If this macro is not explicitly specified, then Condor will
  not restrict the port numbers that it uses. Condor will use
  system-assigned port numbers.
  For this macro to work, both \MacroNI{HIGHPORT} (given above) and
  \MacroNI{LOWPORT} must be defined.
  % PKK
  % Described in default config file: YES
  % Defined in the default config file: NO
  % Default definition in config file: bogus
  % Result if not defined or RHS is empty: Condor uses any ports available,
  % regardless of the HIGHPORT setting

\item[\Macro{IN\_LOWPORT}] \label{param:InLowPort}
  An integer value that specifies a lower limit of given port numbers
  for Condor to use on incoming connections (ports for listening),
  such that Condor is restricted to a range of port numbers.
  This range implies the use of both \MacroNI{IN\_LOWPORT} and
  \MacroNI{IN\_HIGHPORT}.
  A range of port numbers less than 1024 may be used for daemons 
  running as root.
  Do not specify \MacroNI{IN\_LOWPORT} in combination with 
  \MacroNI{IN\_HIGHPORT} such that the range crosses the port 1024
  boundary.
  Applies only to Unix machine configuration.
  Use of \MacroNI{IN\_LOWPORT} and \MacroNI{IN\_HIGHPORT} overrides
  any definition of \MacroNI{LOWPORT} and \MacroNI{HIGHPORT}.

\item[\Macro{IN\_HIGHPORT}] \label{param:InHighPort}
  An integer value that specifies an upper limit of given port numbers
  for Condor to use on incoming connections (ports for listening),
  such that Condor is restricted to a range of port numbers.
  This range implies the use of both \MacroNI{IN\_LOWPORT} and
  \MacroNI{IN\_HIGHPORT}.
  A range of port numbers less than 1024 may be used for daemons 
  running as root.
  Do not specify \MacroNI{IN\_LOWPORT} in combination with 
  \MacroNI{IN\_HIGHPORT} such that the range crosses the port 1024
  boundary.
  Applies only to Unix machine configuration.
  Use of \MacroNI{IN\_LOWPORT} and \MacroNI{IN\_HIGHPORT} overrides
  any definition of \MacroNI{LOWPORT} and \MacroNI{HIGHPORT}.

\item[\Macro{OUT\_LOWPORT}] \label{param:OutLowPort}
  An integer value that specifies a lower limit of given port numbers
  for Condor to use on outgoing connections,
  such that Condor is restricted to a range of port numbers.
  This range implies the use of both \MacroNI{OUT\_LOWPORT} and
  \MacroNI{OUT\_HIGHPORT}.
  A range of port numbers less than 1024 is inappropriate, as
  not all daemons and tools will be run as root.
  Applies only to Unix machine configuration.
  Use of \MacroNI{OUT\_LOWPORT} and \MacroNI{OUT\_HIGHPORT} overrides
  any definition of \MacroNI{LOWPORT} and \MacroNI{HIGHPORT}.

\item[\Macro{OUT\_HIGHPORT}] \label{param:OutHighPort}
  An integer value that specifies an upper limit of given port numbers
  for Condor to use on outgoing connections,
  such that Condor is restricted to a range of port numbers.
  This range implies the use of both \MacroNI{OUT\_LOWPORT} and
  \MacroNI{OUT\_HIGHPORT}.
  A range of port numbers less than 1024 is inappropriate, as
  not all daemons and tools will be run as root.
  Applies only to Unix machine configuration.
  Use of \MacroNI{OUT\_LOWPORT} and \MacroNI{OUT\_HIGHPORT} overrides
  any definition of \MacroNI{LOWPORT} and \MacroNI{HIGHPORT}.

\item[\Macro{UPDATE\_COLLECTOR\_WITH\_TCP}]
  \label{param:UpdateCollectorWithTcp}
  If your site needs to use TCP connections to send ClassAd updates to
  your collector (which it almost certainly does NOT), set to \Expr{True}
  to enable this feature.
  Please read section~\ref{sec:tcp-collector-update} on ``Using TCP to
  Send Collector Updates'' on page~\pageref{sec:tcp-collector-update}
  for more details and a discussion of when this
  functionality is needed. 
  At this time, this setting only affects the main \Condor{collector}
  for the site, not any sites that a \Condor{schedd} might flock to. 
  If enabled, also define
  \Macro{COLLECTOR\_SOCKET\_CACHE\_SIZE} at the central manager, so
  that the collector will accept TCP connections for updates, and will
  keep them open for reuse.
  Defaults to \Expr{False}.
  % PKK
  % Described in default config file: YES
  % Defined in the default config file: NO
  % Default definition in config file: bogus
  % Result if not defined or RHS is empty: disable the feature of TCP updates to
  % collectors
  % WARNING: The code also looks for UPDATE_COLLECTORS_WITH_TCP, which
  % means the SAME THING as UPDATE_COLLECTOR_WITH_TCP

\item[\Macro{TCP\_UPDATE\_COLLECTORS}]
  \label{param:TcpUpdateCollectors}
  The list of collectors which will be updated with TCP instead of UDP.
  Please read section~\ref{sec:tcp-collector-update} on ``Using TCP to
  Send Collector Updates'' on page~\pageref{sec:tcp-collector-update}
  for more details and a discussion of when a site needs this
  functionality. 
  If not defined, no collectors use TCP instead of UDP.
  % PKK
  % Described in default config file: NO
  % Defined in the default config file: NO
  % Default definition in config file: N/A
  % Result if not defined or RHS is empty: no collectors are updated with TCP
  % WARNING: Look in Version 6.5.2 version history for more information about
  % this particular entry.

\item[\MacroB{<SUBSYS>\_TIMEOUT\_MULTIPLIER}]
  \index{SUBSYS\_TIMEOUT\_MULTIPLIER macro@\texttt{<SUBSYS>\_TIMEOUT\_MULTIPLIER} macro}
  \label{param:SubsysTimeoutMultiplier}
  An integer value that defaults to 1.
  This value multiplies configured timeout values
  for all targeted subsystem communications,
  thereby increasing the time until a timeout occurs.
  This configuration variable is intended for use by developers for
  debugging purposes, where communication timeouts interfere.

\item[\Macro{NONBLOCKING\_COLLECTOR\_UPDATE}]
  \label{param:NonblockingCollectorUpdate}
  A boolean value that defaults to \Expr{True}.
  When \Expr{True}, the establishment of TCP connections
  to the \Condor{collector} daemon
  for a security-enabled pool are done in a nonblocking manner.

\item[\Macro{NEGOTIATOR\_USE\_NONBLOCKING\_STARTD\_CONTACT}]
  \label{param:NegotiatorUseNonblockingStartdContact}
  A boolean value that defaults to \Expr{True}.
  When \Expr{True}, the establishment of TCP connections
  from the \Condor{negotiator} daemon to the \Condor{startd} daemon
  for a security-enabled pool are done in a nonblocking manner.

\end{description}

The following settings are specific to enabling Generic Connection
Brokering or GCB in your Condor pool.
More information about GCB and how to configure it can be found in
section~\ref{sec:GCB} on page~\pageref{sec:GCB}.

\begin{description}

\item[\Macro{NET\_REMAP\_ENABLE}] 
  \label{param:NetRemapEnable}
  A boolean variable, that when defined to \Expr{True}, enables a network 
  remapping service for Condor.
  The service to use is controlled by \MacroNI{NET\_REMAP\_SERVICE}.
  This boolean value defaults to \Expr{False}.

\item[\Macro{NET\_REMAP\_SERVICE}]
  \label{param:NetRemapService}
  If \MacroNI{NET\_REMAP\_ENABLE} is
  defined to \Expr{True}, this setting controls what network remapping
  service should be used.
  Currently, the only value supported is \verb@GCB@.
  The default is undefined.

\item[\Macro{NET\_REMAP\_INAGENT}]
  \label{param:NetRemapInagent}
  A comma or space-separated list of IP addresses for GCB brokers.
  Upon start up, the \Condor{master} chooses one at random from
  among the working brokers in the list.
  There is no default if not defined.

\item[\Macro{NET\_REMAP\_ROUTE}]
  \label{param:NetRemapRoute}
  Hosts with the GCB network remapping service enabled that would like
  to use a GCB routing table 
  GCB broker specify
  the full path to their routing table with this setting.
  There is no default value if undefined.

\item[\Macro{MASTER\_WAITS\_FOR\_GCB\_BROKER}]
  \label{param:MasterWaitsForGCBBroker}
  A boolean value that defaults to \Expr{True}.
  This variable determines the behavior of the \Condor{master}
  with GCB enabled.
  With no GCB broker working upon either the start up of the \Condor{master}, 
  or once the \Condor{master} has successfully communicated with a
  GCB broker, but the communication fails,
  if \MacroNI{MASTER\_WAITS\_FOR\_GCB\_BROKER} is \Expr{True},
  the \Condor{master} waits while attempting to find a
  working GCB broker.
  With no GCB broker working upon the start up of the \Condor{master}, 
  if \MacroNI{MASTER\_WAITS\_FOR\_GCB\_BROKER} is \Expr{False},
  the \Condor{master} fails and exits, without restarting.
  Once the \Condor{master} has successfully communicated with a
  GCB broker, but the communication fails,
  if \MacroNI{MASTER\_WAITS\_FOR\_GCB\_BROKER} is \Expr{False},
  the \Condor{master} kills all its children, exits, and restarts.

  The set up task of \Condor{glidein} explicitly sets
  \MacroNI{MASTER\_WAITS\_FOR\_GCB\_BROKER} to \Expr{False} in the
  configuration file it produces.

\end{description}

%%%%%%%%%%%%%%%%%%%%%%%%%%%%%%%%%%%%%%%%%%%%%%%%%%%%%%%%%%%%%%%%%%%%%%%%%%%
\subsection{\label{sec:Shared-Filesystem-Config-File-Entries}Shared File System Configuration File Macros} 
%%%%%%%%%%%%%%%%%%%%%%%%%%%%%%%%%%%%%%%%%%%%%%%%%%%%%%%%%%%%%%%%%%%%%%%%%%%
\index{configuration!shared file system configuration variables}

These macros control how Condor interacts with various shared and
network file systems.  If you are using AFS as your shared file system,
be sure to read section~\ref{sec:Condor-AFS} on Using Condor with
AFS.
For information on submitting jobs under shared file systems,
see
section~\ref{sec:shared-fs}.
\begin{description}

\item[\Macro{UID\_DOMAIN}] \label{param:UidDomain}
  The \MacroNI{UID\_DOMAIN} macro
  is used to decide under which user to run jobs.
  If the \MacroUNI{UID\_DOMAIN}
  on the submitting machine is different than
  the \MacroUNI{UID\_DOMAIN}
  on the machine that runs a job, then Condor runs
  the job as the user \Login{nobody}.
  For example, if the submit machine has
  a \MacroUNI{UID\_DOMAIN} of
  flippy.cs.wisc.edu, and the machine where the job will execute
  has a \MacroUNI{UID\_DOMAIN} of
  cs.wisc.edu, the job will run as user \Login{nobody}, because
  the two \MacroUNI{UID\_DOMAIN}s are not the same.
  If the \MacroUNI{UID\_DOMAIN}
  is the same on both the submit and execute machines,
  then Condor will run the job as the user that submitted the job.

  A further check attempts to assure that the submitting
  machine can not lie about its \MacroNI{UID\_DOMAIN}.
  Condor compares the 
  submit machine's claimed value for \MacroNI{UID\_DOMAIN}
  to its fully qualified name.
  If the two do not end the same, then the submit machine
  is presumed to be lying about its \MacroNI{UID\_DOMAIN}.
  In this case, Condor will run the job as user \Login{nobody}.
  For example, a job submission to the Condor pool at the UW Madison
  from flippy.example.com, claiming a \MacroNI{UID\_DOMAIN} of
  of cs.wisc.edu,
  will run the job as the user \Login{nobody}.

  Because of this verification,
  \MacroUNI{UID\_DOMAIN} must be a real domain name.
  At the Computer Sciences department
  at the UW Madison, we set the \MacroUNI{UID\_DOMAIN}
  to be cs.wisc.edu to
  indicate that whenever someone submits from a department machine, we
  will run the job as the user who submits it.

  Also see \MacroNI{SOFT\_UID\_DOMAIN}
  below for information about one more check
  that Condor performs before running a job as a given user.

  A few details:

  An administrator could set \MacroNI{UID\_DOMAIN}
  to *. This will match all domains,
  but it is a gaping security hole. It is not recommended.

  An administrator can also leave \MacroNI{UID\_DOMAIN} undefined.
  This will force Condor to always run jobs as user \Login{nobody}.
  Running standard universe jobs as user \Login{nobody} enhances
  security and should cause no problems, because the jobs use remote
  I/O to access all of their files.
  However, if vanilla jobs are run as
  user \Login{nobody}, then files that need to be accessed by the job will need
  to be marked as world readable/writable so the user \Login{nobody} can access
  them.

  When Condor sends e-mail about a job, Condor sends the e-mail to
  %% This is the wrong LaTeX  macro to use, but we don't have a correct one.
  \File{user@\$(UID\_DOMAIN)}.
  If \MacroNI{UID\_DOMAIN}
  is undefined, the e-mail is sent to \File{user@submitmachinename}.


\item[\Macro{TRUST\_UID\_DOMAIN}]
  \label{param:TrustUidDomain}
  As an added security precaution when Condor is about to spawn a job,
  it ensures that the \MacroNI{UID\_DOMAIN} of a given
  submit machine is a substring of that machine's fully-qualified
  host name.
  However, at some sites, there may be multiple UID spaces that do
  not clearly correspond to Internet domain names.
  In these cases, administrators may wish to use names to describe the
  UID domains which are not substrings of the host names of the
  machines.
  For this to work, Condor must not do this regular security check.
  If the \MacroNI{TRUST\_UID\_DOMAIN} setting is defined to \Expr{True},
  Condor will not perform this test, and will trust whatever
  \MacroNI{UID\_DOMAIN} is presented by the submit machine when trying
  to spawn a job, instead of making sure the submit machine's host name
  matches the \MacroNI{UID\_DOMAIN}.
  When not defined, the default is \Expr{False},
  since it is more secure to perform this test. 
  % PKK
  % Described in default config file: YES
  % Defined in the default config file: NO
  % Default definition in config file: bogus
  % Result if not defined or RHS is empty: this entry is considered False

\item[\Macro{SOFT\_UID\_DOMAIN}] \label{param:SoftUidDomain}
  A boolean variable that defaults to \Expr{False} when not defined.
  When Condor is about to run a job as a particular user 
  (instead of as user \Login{nobody}),
  it verifies that the UID given for the user is in the
  password file and actually matches the given user name.
  However, under installations that do not have every user
  in every machine's password file,
  this check will fail and the execution attempt will be aborted.
  To cause Condor not to do
  this check, set this configuration variable to \Expr{True}.
  Condor will then run the job under the user's UID.

\item[\Macro{SLOTx\_USER}] \label{param:SlotXUser}
  The name of a user for Condor to use instead of
  user nobody,
  as part of a solution that plugs a security hole whereby
  a lurker process can prey on a subsequent job run as user name nobody. 
  \MacroNI{x} is an integer associated with slots.
  On Windows, \MacroNI{SLOTx\_USER}
  will only work if the credential of the specified
  user is stored on the execute machine using \Condor{store\_cred}.
  See Section~\ref{sec:RunAsNobody} for more information.

\item[\Macro{STARTER\_ALLOW\_RUNAS\_OWNER}]
  \label{param:StarterAllowRunAsOwner}
  This is a boolean expression (evaluated with the job ad as the
  target) that determines whether the job may run under the job owner's
  account (true) or whether it will run as \MacroNI{SLOTx\_USER} or
  nobody (false).  In Unix, this defaults to true.  In windows, it
  defaults to false.  The job ClassAd may also contain an attribute
  \Attr{RunAsOwner} which is logically ANDed with the starter's
  boolean value.  Under Unix, if the job does not specify it, this
  attribute defaults to true.  Under windows, it defaults to false.
  In Unix, if the \Attr{UidDomain} of the machine and job do not
  match, then there is no possibility to run the job as the owner
  anyway, so, in that case, this setting has no effect.  See
  Section~\ref{sec:RunAsNobody} for more information.

\item[\Macro{DEDICATED\_EXECUTE\_ACCOUNT\_REGEXP}]
  \label{param:DedicatedExecuteAccountRegexp}
  This is a regular expression (i.e. a string matching pattern) that
  matches the account name(s) that are dedicated to running condor
  jobs on the execute machine and which will never be used for more
  than one job at a time.  The default matches no account name.  If
  you have configured \MacroNI{SLOTx\_USER} to be a \emph{different}
  account for each Condor slot, and no non-condor processes will ever be
  run by these accounts, then this pattern should match the names of
  all \MacroNI{SLOTx\_USER} accounts.  Jobs run under a dedicated
  execute account are reliably tracked by Condor, whereas other jobs,
  may spawn processes that Condor fails to detect.  Therefore, a
  dedicated execution account provides more reliable tracking of CPU
  usage by the job and it also guarantees that when the job exits, no
  ``lurker'' processes are left behind.  When the job exits, condor
  will attempt to kill all processes owned by the dedicated execution
  account.  Example:

\begin{verbatim}
SLOT1_USER = cndrusr1
SLOT2_USER = cndrusr2
STARTER_ALLOW_RUNAS_OWNER = False
DEDICATED_EXECUTE_ACCOUNT_REGEXP = cndrusr[0-9]+
\end{verbatim}

  You can tell if the starter is in fact treating the account as a
  dedicated account, because it will print a line such as the following
  in its log file:

\begin{verbatim}
Tracking process family by login "cndrusr1"
\end{verbatim}


\item[\Macro{EXECUTE\_LOGIN\_IS\_DEDICATED}]
  \label{param:ExecuteLoginIsDedicated}
  This configuration setting is deprecated because it cannot handle the
  case where some jobs run as dedicated accounts and some do not.  Use
  \MacroNI{DEDICATED\_EXECUTE\_ACCOUNT\_REGEXP} instead.

  A boolean value that defaults to \Expr{False}.  When \Expr{True},
  Condor knows that all jobs are being run by dedicated execution
  accounts (whether they are running as the job owner or as nobody or as
  \MacroNI{SLOTx\_USER}).  Therefore, when the job exits, all processes
  running under the same account will be killed.

\item[\Macro{FILESYSTEM\_DOMAIN}] \label{param:FilesystemDomain}
  The \MacroNI{FILESYSTEM\_DOMAIN}
  macro is an arbitrary string that is used to decide if
  two machines (a submitting machine and an execute machine) share a
  file system.
  Although the macro name contains the word ``DOMAIN'',
  the macro is not required to be a domain name. 
  It often is a domain name.

  % NO LONGER TRUE
  % Vanilla Unix jobs currently require a shared file system in order to
  % share any data files or see the output of the program.
  % Condor decides if there is a shared filesystem by comparing the values
  % of 
  % \MacroUNI{FILESYSTEM\_DOMAIN}
  % of both the submitting and execute machines.
  % If the values are the same,
  % Condor assume there is a shared file system.
  % Condor implements the check
  % by extending the Requirements for your job.
  % You can see these requirements by using the \oArg{-v} argument
  % to \Condor{submit}.

  Note that this implementation is not ideal: machines may share some
  file systems but not others. Condor currently has no way to express
  this automatically. You can express the need to use a
  particular file system by adding additional attributes to your machines
  and submit files, similar to the example given in 
  Frequently Asked Questions, 
  section~\ref{sec:FAQ} on
  how to run jobs only on machines that have 
  certain software packages.

  Note that if you do not set 
  \MacroUNI{FILESYSTEM\_DOMAIN}, Condor defaults
  to setting the macro's value to be the fully qualified hostname
  of the local machine.
  Since each machine will have a different
  \MacroUNI{FILESYSTEM\_DOMAIN},
  they will not be considered to have shared file systems.

  
  % no longer used, and gone from the sample config file as of 5/30/03.
  %\item[\Macro{HAS\_AFS}] \label{param:HasAfs} Set this macro to \Expr{True} if
  %  all the machines you plan on adding in your pool can all access a
  %  common set of AFS fileservers.  Otherwise, set it to \Expr{False}.
  
\item[\Macro{RESERVE\_AFS\_CACHE}] \label{param:ReserveAfsCache} If
  your machine is running AFS and the AFS cache lives on the same
  partition as the other Condor directories, and you want Condor to
  reserve the space that your AFS cache is configured to use, set this
  macro to \Expr{True}.  It defaults to \Expr{False}.
  
\item[\Macro{USE\_NFS}] \label{param:UseNfs} This macro influences
  how Condor jobs running in the standard universe access their
  files.  Condor will redirect the file I/O requests
  of standard universe jobs to be executed on the machine which
  submitted the job.  Because of this, as a Condor job migrates around
  the network, the file system always appears to be identical to the
  file system where the job was submitted.  However, consider the case
  where a user's data files are sitting on an NFS server. The machine
  running the user's program will send all I/O over the network to the
  machine which submitted the job, which in turn sends all the I/O
  over the network a second time back to the NFS file server. Thus,
  all of the program's I/O is being sent over the network twice.
  
  If this macro to \Expr{True}, then Condor will attempt to
  read/write files without redirecting I/O back to the submitting
  machine if both the submitting machine and the machine running the job
  are both accessing the same NFS servers (\emph{if} they are both in the
  same \MacroUNI{FILESYSTEM\_DOMAIN} and in the same \MacroUNI{UID\_DOMAIN},
  as described above).  The result is I/O performed by Condor standard
  universe jobs is only sent over the network once.  
  While sending all file operations over the network twice might sound
  really bad, unless you are operating over networks where bandwidth
  as at a very high premium, practical experience reveals that this
  scheme offers very little real performance gain.  There are also
  some (fairly rare) situations where this scheme can break down.
  
  Setting \MacroUNI{USE\_NFS} to \Expr{False} is always safe.  It may result
  in slightly more network traffic, but Condor jobs are most often heavy
  on CPU and light on I/O.  It also ensures that a remote
  standard universe Condor job will always use Condor's remote system
  calls mechanism to reroute I/O and therefore see the exact same
  file system that the user sees on the machine where she/he submitted
  the job.
  
  Some gritty details for folks who want to know: If the you set
  \MacroUNI{USE\_NFS} to \Expr{True}, and the \MacroUNI{FILESYSTEM\_DOMAIN} of
  both the submitting machine and the remote machine about to execute
  the job match, and the \MacroUNI{FILESYSTEM\_DOMAIN} claimed by the
  submit machine is indeed found to be a subset of what an inverse
  lookup to a DNS (domain name server) reports as the fully qualified
  domain name for the submit machine's IP address (this security
  measure safeguards against the submit machine from lying),
  \emph{then} the job will access files using a local system call,
  without redirecting them to the submitting machine (with
  NFS).  Otherwise, the system call will get routed back to the
  submitting machine using Condor's remote system call mechanism.
  \Note When submitting a vanilla job, \Condor{submit} will, by default,
  append requirements to the Job ClassAd that specify the machine to run
  the job must be in the same \MacroUNI{FILESYSTEM\_DOMAIN} and the same
  \MacroUNI{UID\_DOMAIN}.

\item[\Macro{IGNORE\_NFS\_LOCK\_ERRORS}] \label{param:IgnoreNFSLockErrors}
  When set to \Expr{True}, all errors related to file locking errors from
  NFS are ignored.
  Defaults to \Expr{False}, not ignoring errors.
  
\item[\Macro{USE\_AFS}] \label{param:UseAfs} If your machines have AFS,
  this
  macro determines whether Condor will use remote system calls for
  standard universe jobs to send I/O requests to the submit machine,
  or if it should use local file access on the execute machine (which
  will then use AFS to get to the submitter's files).  Read the
  setting above on \MacroUNI{USE\_NFS} for a discussion of why you might
  want to use AFS access instead of remote system calls.
  
  One important difference between \MacroUNI{USE\_NFS} and
  \MacroUNI{USE\_AFS} is the AFS cache.  With \MacroUNI{USE\_AFS} set to
  \Expr{True}, the remote Condor job executing on some machine will start
  modifying the AFS cache, possibly evicting the machine owner's
  files from the cache to make room for its own.  Generally speaking,
  since we try to minimize the impact of having a Condor job run on a
  given machine, we do not recommend using this setting.

  While sending all file operations over the network twice might sound
  really bad, unless you are operating over networks where bandwidth
  as at a very high premium, practical experience reveals that this
  scheme offers very little real performance gain.  There are also
  some (fairly rare) situations where this scheme can break down.
  
  Setting \MacroUNI{USE\_AFS} to \Expr{False} is always safe.  It may result
  in slightly more network traffic, but Condor jobs are usually heavy
  on CPU and light on I/O.  \Expr{False} ensures that a remote
  standard universe Condor job will always see the exact same
  file system that the user on sees on the machine where he/she
  submitted the job.  Plus, it will ensure that the machine where the
  job executes does not have its AFS cache modified as a result of
  the Condor job being there.  
  
  However, things may be different at your site, which is why the
  setting is there.

\end{description}

%%%%%%%%%%%%%%%%%%%%%%%%%%%%%%%%%%%%%%%%%%%%%%%%%%%%%%%%%%%%%%%%%%%%%%%%%%%
\subsection{\label{Checkpoint-Server-Config-File-Entries}Checkpoint Server Configuration File Macros} 
%%%%%%%%%%%%%%%%%%%%%%%%%%%%%%%%%%%%%%%%%%%%%%%%%%%%%%%%%%%%%%%%%%%%%%%%%%%

\index{configuration!checkpoint server configuration variables}
These macros control whether or not Condor uses a checkpoint server.
If you are using a checkpoint server, this section
describes the settings that the checkpoint server itself needs
defined.  A checkpoint server is installed
separately. It is not included in the main Condor binary
distribution or installation procedure.  See
section~\ref{sec:Ckpt-Server} on Installing a Checkpoint Server
for details on installing and running a checkpoint server for your
pool.

\Note If you are setting up a machine to join the UW-Madison CS
Department Condor pool, you \emph{should} configure the machine to
use a checkpoint server, and use ``condor-ckpt.cs.wisc.edu'' as the
checkpoint server host (see below).

\begin{description}
  
\item[\Macro{CKPT\_SERVER\_HOST}] \label{param:CkptServerHost} The
  hostname of a checkpoint server.

\item[\Macro{STARTER\_CHOOSES\_CKPT\_SERVER}]
  \label{param:StarterChoosesCkptServer} If this parameter is \Expr{True} 
  or undefined on
  the submit machine, the checkpoint server specified by
  \MacroUNI{CKPT\_SERVER\_HOST} on the execute machine is used.  If it is
  \Expr{False} on the submit machine, the checkpoint server
  specified by \MacroUNI{CKPT\_SERVER\_HOST} on the submit machine is
  used.
  
\item[\Macro{CKPT\_SERVER\_DIR}] \label{param:CkptServerDir} The
  checkpoint server needs this macro defined to the full path of the
  directory the server should use to store checkpoint files.
  Depending on the size of your pool and the size of the jobs your
  users are submitting, this directory (and its subdirectories) might
  need to store many Mbytes of data.

\item[\Macro{USE\_CKPT\_SERVER}] \label{param:UseCkptServer} A boolean
  which determines if you want a given submit machine to use a
  checkpoint server if one is available.  If a
  checkpoint server isn't available or \MacroNI{USE\_CKPT\_SERVER} is set to
  False, checkpoints will be written to the local \MacroUNI{SPOOL} directory on
  the submission machine.

\item[\Macro{MAX\_DISCARDED\_RUN\_TIME}]
  \label{param:MaxDiscardedRunTime} If the shadow is unable to read a
  checkpoint file from the checkpoint server, it keeps trying only if
  the job has accumulated more than this many seconds of CPU usage.
  Otherwise, the job is started from scratch.  Defaults to 3600 (1
  hour). This setting is only used if \MacroUNI{USE\_CKPT\_SERVER} is
  \Expr{True}.

\item[\Macro{CKPT\_SERVER\_CHECK\_PARENT\_INTERVAL}]
  \label{param:CkptServerCheckParentInterval}  This is the number of
  seconds between checks to see whether the parent of the checkpoint
  server (i.e. the \Condor{master}) has died.  If the parent has died,
  the checkpoint server shuts itself down.  The default is 120 seconds.
  A setting of 0 disables this check.

\end{description}


%%%%%%%%%%%%%%%%%%%%%%%%%%%%%%%%%%%%%%%%%%%%%%%%%%%%%%%%%%%%%%%%%%%%%%%%%%%
\subsection{\label{sec:Master-Config-File-Entries}\condor{master} Configuration File Macros} 
%%%%%%%%%%%%%%%%%%%%%%%%%%%%%%%%%%%%%%%%%%%%%%%%%%%%%%%%%%%%%%%%%%%%%%%%%%%

\index{configuration!condor\_master configuration variables}
These macros control the \Condor{master}.
\begin{description}
  
\item[\Macro{DAEMON\_LIST}] \label{param:DaemonList} This macro
  determines what daemons the \Condor{master} will start and keep its
  watchful eyes on.  The list is a comma or space separated list of
  subsystem names (listed in
  section~\ref{sec:Condor-Subsystem-Names}).  For example,
  \begin{verbatim}
    DAEMON_LIST = MASTER, STARTD, SCHEDD
  \end{verbatim}

  \Note This configuration variable cannot be changed 
  by using \Condor{reconfig} or 
  by sending a SIGHUP.
  To change this configuration variable, restart the
  \Condor{master} daemon
  by using \Condor{restart}.
  Only then will the change take effect.

  \Note On your central manager, your \MacroUNI{DAEMON\_LIST}
  will be different from your regular pool, since it will include
  entries for the \Condor{collector} and \Condor{negotiator}.  
  
  \Note On machines running Digital Unix, your
  \MacroUNI{DAEMON\_LIST} will also include KBDD, for the
  \Condor{kbdd}, which is a special daemon that runs to monitor
  keyboard and mouse activity on the console.  It is only with this
  special daemon that we can acquire this information on those
  platforms. 

\item[\Macro{DC\_DAEMON\_LIST}] \label{param:DCDaemonList} This macro
  lists the daemons in \MacroNI{DAEMON\_LIST} which use the Condor
  DaemonCore library.  The \Condor{master} must differentiate between
  daemons that use DaemonCore and those that don't so it uses the
  appropriate inter-process communication mechanisms.  This list
  currently includes all Condor daemons except the checkpoint server
  by default.
  
\item[\MacroB{<SUBSYS>}] \label{param:SUBSYS}
  \index{SUBSYS macro@\texttt{<SUBSYS>} macro}
  Once you have defined which
  subsystems you want the \Condor{master} to start, you must provide
  it with the full path to each of these binaries.  For example:
  \begin{verbatim}
    MASTER          = $(SBIN)/condor_master
    STARTD          = $(SBIN)/condor_startd
    SCHEDD          = $(SBIN)/condor_schedd
  \end{verbatim}
  These are most often defined relative to the \MacroUNI{SBIN} macro.

  The macro is named by substituting \MacroNI{<SUBSYS>}
  with the appropriate subsystem string as defined in
  section~\ref{sec:Condor-Subsystem-Names}.

\item[\Macro{DAEMONNAME\_ENVIRONMENT}] \label{param:DaemonNameEnvironment}
  For each subsystem defined in \MacroNI{DAEMON\_LIST}, you may specify
  changes to the environment that daemon is started with by setting
  \MacroNI{DAEMONNAME\_ENVIRONMENT}, where \MacroNI{DAEMONNAME} is the name of
  a daemon listed in \MacroNI{DAEMON\_LIST}. It should use the same syntax
  for specifying the environment as the environment specification in
  a \Condor{submit} file (see page~\pageref{man-condor-submit-environment}).
  For example, if you wish to redefine the
  \Env{TMP} and \Env{CONDOR\_CONFIG} environment variables seen by the
  \Condor{schedd}, you could place the following in the config file:
  \begin{verbatim}
    SCHEDD_ENVIRONMENT = "TMP=/new/value CONDOR_CONFIG=/special/config"
  \end{verbatim}
  When the \Condor{schedd} was started by the \Condor{master}, it would
  see the specified values of \Env{TMP} and \Env{CONDOR\_CONFIG}.

\item[\MacroB{<SUBSYS>\_ARGS}] \label{param:SubsysArgs}
  \index{SUBSYS\_ARGS macro@\texttt{<SUBSYS>\_ARGS} macro}
  This macro allows
  the specification of additional command line arguments for any
  process spawned by the \Condor{master}.  List the desired arguments
  using the same syntax as the arguments specification in a
  \Condor{submit} submit file (see
  page~\pageref{man-condor-submit-arguments}), with one exception: do
  not escape double-quotes when using the old-style syntax (this is
  for backward compatibility).  Set the arguments for a specific
  daemon with this macro, and the macro will affect only that
  daemon. Define one of these for each daemon the \Condor{master} is
  controlling.  For example, set \MacroUNI{STARTD\_ARGS} to specify
  any extra command line arguments to the \Condor{startd}.

  The macro is named by substituting \MacroNI{<SUBSYS>}
  with the appropriate subsystem string as defined in
  section~\ref{sec:Condor-Subsystem-Names}.

\item[\Macro{PREEN}] \label{param:Preen} In addition to the daemons
  defined in \MacroUNI{DAEMON\_LIST}, the \Condor{master} also starts up
  a special process, \Condor{preen} to clean out junk files that have
  been left laying around by Condor.  This macro determines where the
  \Condor{master} finds the \Condor{preen} binary.
  Comment out this macro, and \Condor{preen} will not run.

\item[\Macro{PREEN\_ARGS}] \label{param:PreenArgs}
  Controls how \Condor{preen} behaves by allowing the specification
  of command-line arguments.
  This macro works as \MacroUNI{<SUBSYS>\_ARGS} does.
  The difference is that you must specify this macro for
  \Condor{preen} if you want it to do anything.
  \Condor{preen} takes action only
  because of command line arguments.
  \Opt{-m} means you want e-mail about files \Condor{preen} finds that it
  thinks it should remove.
  \Opt{-r} means you want \Condor{preen} to actually remove these files.

\item[\Macro{PREEN\_INTERVAL}] \label{param:PreenInterval} This macro
  determines how often \Condor{preen} should be started.  It is
  defined in terms of seconds and defaults to 86400 (once a day).

\item[\Macro{PUBLISH\_OBITUARIES}] \label{param:PublishObituaries}
  When a daemon crashes, the \Condor{master} can send e-mail to the
  address specified by \MacroUNI{CONDOR\_ADMIN} with an obituary letting
  the administrator know that the daemon died, the cause of
  death (which signal or exit status it exited with), and
  (optionally) the last few entries from that daemon's log file.  If
  you want obituaries, set this macro to \Expr{True}.

\item[\Macro{OBITUARY\_LOG\_LENGTH}] \label{param:ObituaryLogLength}
  This macro controls how many lines
  of the log file are part of obituaries.  This macro has a default
  value of 20 lines.

\item[\Macro{START\_MASTER}] \label{param:StartMaster} If this setting
  is defined and set to \Expr{False} when the \Condor{master} starts up, the first
  thing it will do is exit.  This appears strange, but perhaps you
  do not want Condor to run on certain machines in your pool, yet
  the boot scripts for your entire pool are handled by a centralized
  This is
  an entry you would most likely find in a local configuration file,
  not a global configuration file.

\item[\Macro{START\_DAEMONS}] \label{param:StartDaemons} This macro
  is similar to the \MacroUNI{START\_MASTER} macro described above.
  However, the \Condor{master} does not exit; it does not start any
  of the daemons listed in the \MacroUNI{DAEMON\_LIST}.
  The daemons may be started at a later time with a \Condor{on}
  command.

\item[\Macro{MASTER\_UPDATE\_INTERVAL}]
  \label{param:MasterUpdateInterval} This macro determines how often
  the \Condor{master} sends a ClassAd update to the
  \Condor{collector}.  It is defined in seconds and defaults to 300
  (every 5 minutes).
  
\item[\Macro{MASTER\_CHECK\_NEW\_EXEC\_INTERVAL}]
  \label{param:MasterCheckNewExecInterval} This
  macro controls how often the \Condor{master} checks the timestamps
  of the running daemons.  If any daemons have been modified, the
  master restarts them.  It is defined in seconds and defaults to 300
  (every 5 minutes).

\item[\Macro{MASTER\_NEW\_BINARY\_DELAY}]
  \label{param:MasterNewBinaryDelay} Once the \Condor{master} has
  discovered a new binary, this macro controls how long it waits
  before attempting to execute the new binary.  This delay exists
  because the \Condor{master} might notice a new binary while it
  is in the process of being copied,
  in which case trying to execute it yields
  unpredictable results.  The entry is defined in seconds and
  defaults to 120 (2 minutes).

\item[\Macro{SHUTDOWN\_FAST\_TIMEOUT}]
  \label{param:ShutdownFastTimeout} This macro determines the maximum
  amount of time daemons are given to perform their
  fast shutdown procedure before the \Condor{master} kills them
  outright.  It is defined in seconds and defaults to 300 (5 minutes).

\item[\Macro{MASTER\_BACKOFF\_CONSTANT} and
  \Macro{MASTER\_<name>\_BACKOFF\_CONSTANT}]
  \label{param:MasterBackoffConstant}
  When a daemon crashes, \Condor{master} uses an exponential back off
  delay before restarting it; see the discussion at the end of this
  section for a detailed discussion on how these parameters work together.
  These settings define the constant value of the expression used to
  determine how long to wait before starting the daemon again (and,
  effectively becomes the initial backoff time).  It is an integer in
  units of seconds, and defaults to 9 seconds.

  \MacroUNI{MASTER\_<name>\_BACKOFF\_CONSTANT} is the daemon-specific
  form of \MacroNI{MASTER\_BACKOFF\_CONSTANT}; if this daemon-specific
  macro is not defined for a specific daemon, the non-daemon-specific
  value will used.

\item[\Macro{MASTER\_BACKOFF\_FACTOR} and
      \Macro{MASTER\_<name>\_BACKOFF\_FACTOR}]
  \label{param:MasterBackoffFactor}
  When a daemon crashes, \Condor{master} uses an exponential back off
  delay before restarting it; see the discussion at the end of this
  section for a detailed discussion on how these parameters work together.
  This setting is the base of the
  exponent used to determine how long to wait before starting the
  daemon again.  It defaults to 2 seconds.

  \MacroUNI{MASTER\_<name>\_BACKOFF\_FACTOR} is the daemon-specific
  form of \MacroNI{MASTER\_BACKOFF\_FACTOR}; if this daemon-specific
  macro is not defined for a specific daemon, the non-daemon-specific
  value will used.

\item[\Macro{MASTER\_BACKOFF\_CEILING} and
      \Macro{MASTER\_<name>\_BACKOFF\_CEILING}]
  \label{param:MasterBackoffCeiling}
  When a daemon crashes, \Condor{master} uses an exponential back off
  delay before restarting it; see the discussion at the end of this
  section for a detailed discussion on how these parameters work together.
  This entry determines the maximum amount of time you want the master
  to wait between attempts to start a given daemon.
  (With 2.0 as the \MacroUNI{MASTER\_BACKOFF\_FACTOR},
  1 hour is obtained in 12 restarts).  It is defined in terms of
  seconds and defaults to 3600 (1 hour).

  \MacroUNI{MASTER\_<name>\_BACKOFF\_CEILING} is the daemon-specific
  form of \MacroNI{MASTER\_BACKOFF\_CEILING}; if this daemon-specific
  macro is not defined for a specific daemon, the non-daemon-specific
  value will used.

\item[\Macro{MASTER\_RECOVER\_FACTOR} and
      \Macro{MASTER\_<name>\_RECOVER\_FACTOR}]
  \label{param:MasterRecoverFactor}  A macro to set how long a daemon 
  needs to run without crashing before it is considered \emph{recovered}.
  Once a
  daemon has recovered, the number of restarts is reset, so the
  exponential back off returns to its initial state.  
  The macro is defined in
  terms of seconds and defaults to 300 (5 minutes).

  \MacroUNI{MASTER\_<name>\_RECOVER\_FACTOR} is the daemon-specific
  form of \MacroNI{MASTER\_RECOVER\_FACTOR}; if this daemon-specific
  macro is not defined for a specific daemon, the non-daemon-specific
  value will used.

\end{description}

When a daemon crashes, \Condor{master} will restart the daemon after a
delay (a back off).
The length of this delay is based on how many times it has been
restarted, and gets larger after each crashes. 
The equation for calculating this backoff time is
given by: $$t = c + k^n$$ where $t$ is the calculated time, $c$ is
the constant defined by \MacroUNI{MASTER\_BACKOFF\_CONSTANT}, $k$ is
the ``factor'' defined by \MacroUNI{MASTER\_BACKOFF\_FACTOR}, and $n$
is the number of restarts already attempted (0 for the first restart,
1 for the next, etc.).

With default values, after the first crash, the delay would be $t = 9
+ 2.0^0$, giving 10 seconds (remember, $n = 0$).  If the daemon keeps
crashing, the delay increases.

For example, take the \MacroUNI{MASTER\_BACKOFF\_FACTOR} (which defaults
to 2.0) to the power the number of times the daemon has restarted, and add
\MacroUNI{MASTER\_BACKOFF\_CONSTANT} (which defaults to 9).
Thus:

 $1^{st}$ crash:  $n = 0$, so: $t = 9 + 2^0 = 9 + 1 = 10\ seconds$

 $2^{nd}$ crash:  $n = 1$, so: $t = 9 + 2^1 = 9 + 2 = 11\ seconds$

 $3^{rd}$ crash:  $n = 2$, so: $t = 9 + 2^2 = 9 + 4 = 13\ seconds$

    ...

 $6^{th}$ crash:  $n = 5$, so: $t = 9 + 2^5 = 9 + 32 = 41\ seconds$

    ...

 $9^{th}$ crash:  $n = 8$, so: $t = 9 + 2^8 = 9 + 256 = 265\ seconds$

And, after the 13 crashes, it would be:

 $13^{th}$ crash:  $n = 12$, so: $t = 9 + 2^{12} = 9 + 4096 = 4105\ seconds$

This is bigger than the \MacroUNI{MASTER\_BACKOFF\_CEILING}, which
defaults to 3600, so the daemon would really be restarted after only
3600 seconds, not 4105.
The \Condor{master} tries again every hour (since the numbers would
get larger and would always be capped by the ceiling).
Eventually, imagine that daemon finally started and did not crash.
This might happen if, for example, an administrator reinstalled
an accidentally deleted binary after receiving e-mail about
the daemon crashing.
If it stayed alive for
\MacroUNI{MASTER\_RECOVER\_FACTOR} seconds (defaults to 5 minutes),
the count of how many restarts this daemon has performed is reset to
0.

The moral of the example is that 
the defaults work quite well, and you probably 
will not want to change them for any reason.
\begin{description}

\item[\Macro{MASTER\_NAME}] \label{param:MasterName}
  Defines a unique name given for a \Condor{master} daemon on a machine.
  Defaults to the fully qualified host name.
  If more than one \Condor{master} is running on the same host (for
  example, because of multiple Personal Condor installations running
  as different users) the \MacroNI{MASTER\_NAME} for each
  \Condor{master} should be defined to uniquely identify the separate
  daemons. 

  If the \MacroNI{MASTER\_NAME} contains more than a host name,
  it must
  have the form \verb$identifying-string@full.host.name$.
  If the string specified with \MacroNI{MASTER\_NAME} already includes
  an \verb$@$ sign, Condor will replace whatever follows the \verb$@$
  sign with the fully qualified host name of the local machine.
  If the string does not include an \verb$@$ sign,
  Condor will append one, followed by the host name.
  The \verb$identifying-string$ portion can contain any
  alphanumeric ASCII characters or punctuation marks except \verb$@$
  (which is used to delimit the name from the host name).
  We recommend that the string does not contain the \verb$:$
  character, since that might cause problems with certain tools.
  In the example of many Personal Condor installations on the same
  host, the user name that each \Condor{master} is executing as
  is, by convention,
  the \verb$identifying-string$.
  This is easily accomplished by setting
  \verb@MASTER_NAME = $(USERNAME)@ in the
  configuration file.

  If the \MacroNI{MASTER\_NAME} setting is used, and the
  \Condor{master} is configured to spawn a \Condor{schedd},
  the name
  defined with \MacroNI{MASTER\_NAME} takes precedence over the
  \Macro{SCHEDD\_NAME} setting (see section~\ref{param:ScheddName} on
  page~\pageref{param:ScheddName}). 
  Since Condor makes the assumption that there is only one
  instance of the \Condor{startd} running on a machine,
  the \MacroNI{MASTER\_NAME} is not automatically propagated to the
  \Condor{startd}.
  However, in situations where multiple \Condor{startd} daemons are
  running on the same host (for example, when using \Condor{glidein}),
  the \MacroNI{STARTD\_NAME} should be set to uniquely identify 
  the \Condor{startd} daemons
  (this is done automatically in the case of \Condor{glidein}).

  If a Condor daemon (master, schedd or startd) has been given a
  unique name, all Condor tools that need to contact that daemon can
  be told what name to use via the \Opt{-name} command-line option.


\item[\Macro{MASTER\_ATTRS}] \label{param:MasterExprs} This macro is
  described in section~\ref{param:SubsysExprs} as
  \MacroNI{<SUBSYS>\_ATTRS}.

\item[\Macro{MASTER\_DEBUG}] \label{param:MasterDebug} This macro
  is described in section~\ref{param:SubsysDebug} as
  \MacroNI{<SUBSYS>\_DEBUG}.

\item[\Macro{MASTER\_ADDRESS\_FILE}] \label{param:MasterAddressFile}
  This macro is described in
  section~\ref{param:SubsysAddressFile} as
  \MacroNI{<SUBSYS>\_ADDRESS\_FILE}. 

\item[\Macro{SECONDARY\_COLLECTOR\_LIST}]
  \label{param:SecondaryCollectorList} This macro has been removed
  as of Condor version 6.9.3.
  Use the \Macro{COLLECTOR\_HOST} configuration variable, which may define a
  list of \Condor{collector} daemons.

\item[\Macro{ALLOW\_ADMIN\_COMMANDS}]
  \label{param:AllowAdminCommands} If set to NO for a given host, this
  macro disables administrative commands, such as 
  \Condor{restart}, \Condor{on}, and \Condor{off}, to that host.

\item[\Macro{MASTER\_INSTANCE\_LOCK}] \label{param:MasterInstanceLock}
  Defines the name of a file for the \Condor{master} daemon
  to lock in order to prevent multiple \Condor{master}s
  from starting.
  This is useful when using shared file systems like NFS which do
  not technically support locking in the case where the lock files
  reside on a local disk.
  If this macro is not defined, the default file name will be
  \File{\$(LOCK)/InstanceLock}.
  \File{\$(LOCK)} can instead be defined to
  specify the location of all lock files, not just the 
  \Condor{master}'s \File{InstanceLock}.
  If \File{\$(LOCK)} is undefined, then the master log itself is locked.

\item[\Macro{ADD\_WINDOWS\_FIREWALL\_EXCEPTION}]
  \label{param:AddWindowsFirewallException} When set to \Expr{False}, the
  \Condor{master} will not automatically add Condor to the Windows
  Firewall list of trusted applications. Such trusted applications can
  accept incoming connections without interference from the firewall. This
  only affects machines running Windows XP SP2 or higher. The default
  is \Expr{True}.

\item[\Macro{WINDOWS\_FIREWALL\_FAILURE\_RETRY}]
  \label{param:WindowsFirewallFailureRetry} 
  An integer value (default value is 60) that represents
  the number of times the \Condor{master} will retry to add
  firewall exceptions.
  When a Windows machine boots
  up, Condor starts up by default as well. Under certain conditions, the
  \Condor{master} may have difficulty adding exceptions to the Windows
  Firewall because of a delay in other services starting up.
  Examples of services that may possibly be slow are the 
  SharedAccess service, the Netman service, or the Workstation service.
  This configuration variable allows administrators to set the number of
  times (once every 10 seconds) that the \Condor{master} will retry
  to add firewall exceptions. A value of 0 means that Condor will
  retry indefinitely.

\item[\Macro{USE\_PROCESS\_GROUPS}]
  \label{param:UseProcessGroups} 
  A boolean value that defaults to \Expr{True}.  When \Expr{False},
  Condor daemons on Unix machines will \emph{not} create new sessions
  or process groups. Condor uses processes groups to help it track the
  descendants of processes it creates. This can cause problems when
  Condor is run under another job execution system (e.g. Condor Glidein).

\end{description}

%%%%%%%%%%%%%%%%%%%%%%%%%%%%%%%%%%%%%%%%%%%%%%%%%%%%%%%%%%%%%%%%%%%%%%%%%%%
\subsection{\label{sec:Startd-Config-File-Entries}\condor{startd}
Configuration File Macros}
%%%%%%%%%%%%%%%%%%%%%%%%%%%%%%%%%%%%%%%%%%%%%%%%%%%%%%%%%%%%%%%%%%%%%%%%%%%

\index{configuration!condor\_startd configuration variables}
\Note If you are running Condor on a multi-CPU machine, be sure
to also read section~\ref{sec:Configuring-SMP} on
page~\pageref{sec:Configuring-SMP} which describes how to set up and
configure Condor on SMP machines.

These settings control general operation of the \Condor{startd}.
Examples using these configuration macros,
as well as further explanation is found in
section~\ref{sec:Configuring-Policy} on
Configuring The Startd Policy.

\begin{description}

\item[\Macro{START}] \label{param:Start}  A boolean expression
  that, when \Expr{True}, indicates that the machine is willing
  to start running a Condor job.
  \MacroNI{START} is considered when the \Condor{negotiator} daemon
  is considering evicting the job to replace it with one that will
  generate a better rank for the \Condor{startd} daemon,
  or a user with a higher priority.

\item[\Macro{SUSPEND}] \label{param:Suspend}  A boolean expression
   that, when \Expr{True}, causes Condor to suspend running
   a Condor job.
   The machine may still be claimed, but the job makes no further
   progress, and Condor does not generate a load on the machine.

\item[\Macro{PREEMPT}] \label{param:Preempt}   A boolean expression
   that, when \Expr{True}, causes Condor to stop a currently
   running job.

\item[\Macro{CONTINUE}] \label{param:Continue}  A boolean expression
   that, when \Expr{True}, causes Condor to continue the execution
   of a suspended job.

\item[\Macro{KILL}] \label{param:Kill}  A boolean expression
   that, when \Expr{True}, causes Condor to immediately stop the
   execution of a currently running job, without delay, and
   without taking the time to produce a checkpoint (for a standard
   universe job).

\item[\Macro{RANK}] \label{param:Rank}  A floating point value
   that Condor uses to compare potential jobs.
   A larger value for a specific job ranks that job above
   others with lower values for \MacroNI{RANK}.

\item[\Macro{IS\_VALID\_CHECKPOINT\_PLATFORM}] \label{param:IsValidCheckpointPlatform} A boolean expression that is logically anded with the
   with the \MacroNI{START} expression to limit which machines a
   standard universe job may continue execution on once they have
   produced a checkpoint.
   The default expression is

   \footnotesize
   \begin{verbatim}
   IS_VALID_CHECKPOINT_PLATFORM =
   (
     ( (TARGET.JobUniverse == 1) == FALSE) ||
   
     (
       (MY.CheckpointPlatform =!= UNDEFINED) &&
       (
         (TARGET.LastCheckpointPlatform =?= MY.CheckpointPlatform) ||
         (TARGET.NumCkpts == 0)
       )
     )
   )
   \end{verbatim}
   \normalsize

\item[\Macro{WANT\_SUSPEND}] \label{param:WantSuspend}  A boolean expression
   that, when \Expr{True}, tells Condor to evaluate the \MacroNI{SUSPEND} expression.

\item[\Macro{WANT\_VACATE}] \label{param:WantVacate}  A boolean expression
   that, when \Expr{True}, defines that a preempted
   Condor job is to be vacated, instead of killed.

\item[\Macro{IS\_OWNER}] \label{param:IsOwner}  A boolean expression that
   defaults to being defined as
\begin{verbatim}
        IS_OWNER = (START =?= FALSE)
\end{verbatim}
   Used to describe the state of the machine with respect to its use
   by its owner.
   Job ClassAd attributes are not used in defining \MacroNI{IS\_OWNER},
   as they would be \Expr{Undefined}.
\end{description}




\begin{description}

\item[\Macro{STARTER}] \label{param:Starter}  This macro holds the
  full path to the \Condor{starter} binary that the \Condor{startd} should 
  spawn.
  It is normally defined relative to \MacroUNI{SBIN}.
  
\item[\Macro{POLLING\_INTERVAL}] \label{param:PollingInterval} When a
  \Condor{startd} enters the claimed state, this macro determines how often
  the state of the machine is polled to check the need to suspend, resume,
  vacate or kill the job.  It is defined in terms of seconds and defaults to
  5.
  
\item[\Macro{UPDATE\_INTERVAL}] \label{param:UpdateInterval}
  Determines how often the \Condor{startd} should send a ClassAd update
  to the \Condor{collector}.  The \Condor{startd} also sends update on any
  state or activity change, or if the value of its \Expr{START} expression
  changes.  See section~\ref{sec:States} on \Condor{startd}
  states, section~\ref{sec:Activities} on \Condor{startd}
  Activities, and section~\ref{sec:Start-Expr} on \Condor{startd}
  \Expr{START} expression for details on states, activities, and the
  \Expr{START} expression.  This macro is defined in
  terms of seconds and defaults to 300 (5 minutes).
  
\item[\Macro{MAXJOBRETIREMENTTIME}] \label{param:MaxJobRetirementTime}
  An integer value representing the number of seconds a preempted job
  will be allowed to run before being evicted. The default value of 0
  (when the configuration variable is not present) implements the
  expected policy that there is no retirement time.  See
  \MacroNI{MAXJOBRETIREMENTTIME} in
  section~\ref{sec:State-Expression-Summary} for further explanation.

\item[\Macro{CLAIM\_WORKLIFE}] \label{param:ClaimWorklife} If
  provided, this expression specifies the number of seconds after
  which a claim will stop accepting additional jobs.  By default, once
  the negotiator gives a schedd a claim to a slot, the schedd will
  keep running jobs on that slot as long as it has more jobs with
  matching requirements, without returning the slot to the unclaimed
  state and renegotiating for machines.  Once \MacroNI{CLAIM\_WORKLIFE}
  expires, any existing job may continue to run as usual, but once it
  finishes or is preempted, the claim is closed.
  This may be useful if you want to force periodic renegotiation of
  resources without preemption having to occur.  For example, if you
  have some low-priority jobs which should never be interrupted with
  kill signals, you could prevent them from being killed with
  \Expr{MaxJobRetirementTime}, but now high-priority jobs may have to
  wait in line when they match to a machine that is busy running one of
  these uninterruptible jobs.  You can prevent the high-priority jobs
  from ever matching to such a machine by using a rank expression in the
  job or in the negotiator's rank expressions, but then the low-priority
  claim will never be interrupted; it can keep running more jobs.  The
  solution is to use \Expr{CLAIM\_WORKLIFE} to force the claim to stop
  running additional jobs after a certain amount of time.
  The default value for \Expr{CLAIM\_WORKLIFE} is -1, which is treated
  as an infinite claim worklife, so claims may be held indefinitely
  (as long as they are not preempted and the schedd does not
  relinquish them, of course).  A value of 0 has the effect of not allowing
  more than one job to run per claim, since it immediately expires after the
  first job starts running.

\item[\Macro{MAX\_CLAIM\_ALIVES\_MISSED}]
  \label{param:MaxClaimAlivesMissed} The \Condor{schedd} sends periodic updates
  to each \Condor{startd} as a keep alive (see the description of
  \Macro{ALIVE\_INTERVAL} on page~\pageref{param:AliveInterval}).  
  If the \Condor{startd} does not receive any keep alive messages, it assumes
  that something has gone wrong with the \Condor{schedd} and that the resource
  is not being effectively used.
  Once this happens, the \Condor{startd} considers the claim to have timed out,
  it releases the claim, and starts advertising itself as available
  for other jobs.
  Because these keep alive messages are sent via UDP, they are
  sometimes dropped by the network.
  Therefore, the \Condor{startd} has some tolerance for missed keep alive
  messages, so that in case a few keep alives are lost, the \Condor{startd}
  will not immediately release the claim.
  This setting controls how many keep alive messages can be missed
  before the \Condor{startd} considers the claim no longer valid.
  The default is 6.

\item[\Macro{STARTD\_HAS\_BAD\_UTMP}] \label{param:StartdHasBadUtmp}
  When the \Condor{startd} is computing the idle time of all the
  users of the machine (both local and remote), it checks the
  \File{utmp} file to find all the currently active ttys, and only
  checks access time of the devices associated with active logins.
  Unfortunately, on some systems, \File{utmp} is unreliable, and the
  \Condor{startd} might miss keyboard activity by doing this.  So, if your
  \File{utmp} is unreliable, set this macro to \Expr{True} and the
  \Condor{startd} will check the access time on all tty and pty devices.
  
\item[\Macro{CONSOLE\_DEVICES}] \label{param:ConsoleDevices} This
  macro allows the \Condor{startd} to monitor console (keyboard and mouse)
  activity by checking the access times on special files in
  \File{/dev}.  Activity on these files shows up as 
  \AdAttr{ConsoleIdle}
  time in the \Condor{startd}'s ClassAd.  Give a comma-separated list of
  the names of devices considered the console, without the
  \File{/dev/} portion of the pathname.  The defaults vary from
  platform to platform, and are usually correct.  

  One possible exception to this is on Linux, where
  we use ``mouse'' as
  one of the entries.  Most Linux installations put in a
  soft link from \File{/dev/mouse} that points to the appropriate
  device (for example, \File{/dev/psaux} for a PS/2 bus mouse, or
  \File{/dev/tty00} for a serial mouse connected to com1).  However,
  if your installation does not have this soft link, you will either
  need to put it in (you will be glad you did), or change this
  macro to point to the right device. 
  
  Unfortunately, there are no such devices on Digital Unix
  (don't be fooled by \File{/dev/keyboard0}; the kernel does not
  update the access times on these devices), so this macro is not
  useful in these cases, and we must use the \Condor{kbdd} to get this
  information by connecting to the X server.
  
\item[\Macro{STARTD\_JOB\_EXPRS}] \label{param:StartdJobExprs} When
  the machine is claimed by a remote user, the \Condor{startd} can also advertise
  arbitrary attributes from the job ClassAd in the machine
  ClassAd.
  List the attribute names to be advertised.  \Note Since
  these are already ClassAd expressions, do not do anything
  unusual with strings.   
  This setting defaults to ``JobUniverse''.

\item[\Macro{STARTD\_ATTRS}] \label{param:StartdAttrs} This macro is
  described in section~\ref{param:SubsysAttrs} as
  \MacroNI{<SUBSYS>\_ATTRS}.

\item[\Macro{STARTD\_DEBUG}] \label{param:StartdDebug} This macro
  (and other settings related to debug logging in the \Condor{startd}) is
  described in section~\ref{param:SubsysDebug} as
  \MacroNI{<SUBSYS>\_DEBUG}.

\item[\Macro{STARTD\_ADDRESS\_FILE}] \label{param:StartdAddressFile}
  This macro is described in
  section~\ref{param:SubsysAddressFile} as
  \MacroNI{<SUBSYS>\_ADDRESS\_FILE} 

\item[\Macro{STARTD\_SHOULD\_WRITE\_CLAIM\_ID\_FILE}] \label{param:StartdShouldWriteClaimIdFile}
  Starting with version 6.7.10, the \Condor{startd} can be configured
  to write out the \Attr{ClaimId} for the next available claim on all
  slots to separate files.
  This boolean attribute controls whether the \Condor{startd} should
  write these files.
  The default value is true.
  
\item[\Macro{STARTD\_CLAIM\_ID\_FILE}] \label{param:StartdClaimIdFile}
  This macro controls what file names are used if the above
  \MacroNI{STARTD\_SHOULD\_WRITE\_CLAIM\_ID\_FILE} is true.  By
  default, Condor will write the ClaimId into a file in the
  \MacroU{LOG} directory called \File{.startd\_claim\_id.slotX}, where
  \verb@X@ is the value of \Attr{SlotID}, the integer that
  identifies a given slot on the system, or \verb@1@ on a
  single-slot machine.
  If you define your own value for this setting, you should provide a
  full path, and Condor will automatically append the \verb@.slotX@
  portion of the file name.

\item[\Macro{NUM\_CPUS}] \label{param:NumCpus}
  This macro can be used to ``lie'' to the \Condor{startd} about how many CPUs
  your machine has.
  If you set this, it will override Condor's automatic computation of
  the number of CPUs in your machine, and Condor will use whatever
  integer you specify here. 
  In this way, you can allow multiple Condor jobs to run on a
  single-CPU machine by having that machine treated like an SMP
  machine with multiple CPUs, which could have different Condor jobs
  running on each one.
  Or, you can have an SMP machine advertise more slots than
  it has CPUs.
  However, using this parameter will hurt the performance of the jobs,
  since you would now have multiple jobs running on the same CPU,
  competing with each other.
  The option is only meant for people who specifically want this
  behavior and know what they are doing.  
  It is disabled by default.

  \Note This setting cannot be changed with a simple reconfig (either
  by sending a SIGHUP or using \Condor{reconfig}.
  If you change this, you must restart the \Condor{startd} for the
  change to take effect (by using ``\condor{restart} -startd'').

  \Note If you use this setting on a given machine, you should
  probably advertise that fact in the machine's ClassAd by using the
  \MacroNI{STARTD\_ATTRS} setting (described above).
  This way, jobs submitted in your pool could specify that they did or
  did not want to be matched with machines that were only really
  offering ``fractional CPUs''.

\item[\Macro{MAX\_NUM\_CPUS}] \label{param:MaxNumCpus}
  This macro will cap the number of CPUs detected by Condor on a machine.
  If you set \MacroNI{NUM\_CPUS} this cap is ignored.
  If it is set to zero, there is no cap. 
  If it is not defined in the config file, it defaults to zero and there is
  no cap. 

  \Note This setting cannot be changed with a simple reconfig (either
  by sending a SIGHUP or using \Condor{reconfig}.
  If you change this, you must restart the \Condor{startd} for the
  change to take effect (by using ``\condor{restart} -startd'').

\item[\Macro{COUNT\_HYPERTHREAD\_CPUS}] \label{param:CountHyperthreadCpus}
  This macro controls how Condor sees hyper threaded
  processors. When set to \Expr{True} (the default), it includes virtual CPUs in
  the default value of \MacroNI{NUM\_CPUS}. On dedicated cluster nodes, 
  counting virtual CPUs can sometimes improve total throughput at the expense 
  of individual job speed. However, counting them on desktop workstations can
  interfere with interactive job performance.

\item[\Macro{MEMORY}] \label{param:Memory}
  Normally, Condor will automatically detect the amount of physical
  memory available on your machine.  Define \MacroNI{MEMORY} to tell
  Condor how much physical memory (in MB) your machine has, overriding
  the value Condor computes automatically.

\item[\Macro{RESERVED\_MEMORY}] \label{param:ReservedMemory}
  How much memory would you like reserved from Condor?  By default,
  Condor considers all the physical memory of your machine as
  available to be used by Condor jobs.  If \MacroNI{RESERVED\_MEMORY} is
  defined, Condor subtracts it from the amount of memory it advertises
  as available.

\item[\Macro{STARTD\_NAME}] \label{param:StartdName}
  Used to give an alternative name in the \Condor{startd}'s
  class ad.
  This esoteric configuration macro might be used in the situation
  where there are two \Condor{startd} daemons running on one machine,
  and each reports to the same \Condor{collector}.
  Different names will distinguish the two daemons.
  See the description of \MacroNI{MASTER\_NAME} in
  section~\ref{param:MasterName} on page~\pageref{param:MasterName}
  for a description of valid Condor daemon names.

\item[\Macro{RUNBENCHMARKS}] \label{param:RunBenchmarks}
  Specifies when to run benchmarks.
  When the machine is in the Unclaimed state and this expression
  evaluates to \Expr{True}, benchmarks will be run.
  If RunBenchmarks is specified and set to anything other than \Expr{False},
  additional benchmarks will be run when the \Condor{startd} initially starts.
  To disable startup benchmarks, set \MacroNI{RunBenchmarks} to \Expr{False},
  or comment it out of the configuration file.

\item[\Macro{DedicatedScheduler}] \label{param:DedicatedScheduler}
  A string that identifies the dedicated scheduler.
  See section~\ref{sec:Configure-Dedicated-Resource}
  on page~\pageref{sec:Configure-Dedicated-Resource} for details.

\item[\Macro{STARTD\_NOCLAIM\_SHUTDOWN}] \label{param:StartdNoclaimShutdown}
  The number of seconds to run without receiving a claim before
  shutting Condor down on this machine.  Defaults to unset, which
  means to never shut down.  This is primarily intended for \condor{glidein}.
  Use in other situations is not recommended.

\end{description}

These macros control if the \Condor{startd} daemon should perform
backfill computations whenever resources would otherwise be idle.  
See section~\ref{sec:Backfill} on page~\pageref{sec:Backfill} on
Configuring Condor for Running Backfill Jobs for details.

\begin{description}

\item[\Macro{ENABLE\_BACKFILL}] \label{param:EnableBackfill} A boolean
  value that, when \Expr{True}, indicates that the machine is willing
  to perform backfill computations when it would otherwise be idle.
  This is not a policy expression that is evaluated, it is a simple
  \Expr{True} or \Expr{False}.
  This setting controls if any of the other backfill-related
  expressions should be evaluated.
  The default is \Expr{False}.

\item[\Macro{BACKFILL\_SYSTEM}] \label{param:BackfillSystem} A string
  that defines what backfill system to use for spawning and managing
  backfill computations.
  Currently, the only supported value for this is \AdStr{BOINC}, which
  stands for the \Term{Berkeley Open Infrastructure for Network
  Computing}.
  See \URL{http://boinc.berkeley.edu} for more information about
  BOINC.
  There is no default value, administrators must define this.
  
\item[\Macro{START\_BACKFILL}] \label{param:StartBackfill} A boolean
  expression that is evaluated whenever a Condor resource is in the
  Unclaimed/Idle state and the \MacroNI{ENABLE\_BACKFILL} expression
  is \Expr{True}.  
  If \MacroNI{START\_BACKFILL} evaluates to \Expr{True}, the machine
  will enter the Backfill state and attempt to spawn a backfill
  computation. 
  This expression is analogous to the \Macro{START} expression that
  controls when a Condor resource is available to run normal Condor
  jobs.
  The default value is \Expr{False} (which means do not spawn a
  backfill job even if the machine is idle and
  \MacroNI{ENABLE\_BACKFILL} expression is \Expr{True}).
  For more information about policy expressions and the Backfill
  state, see section~\ref{sec:Configuring-Policy} beginning on
  page~\pageref{sec:Configuring-Policy}, especially
  sections~\ref{sec:States}, \ref{sec:Activities}, and
  \ref{sec:State-and-Activity-Transitions}.

\item[\Macro{EVICT\_BACKFILL}] \label{param:EvictBackfill} A boolean
  expression that is evaluated whenever a Condor resource is in the
  Backfill state which, when \Expr{True}, indicates the machine should
  immediately kill the currently running backfill computation and
  return to the Owner state.
  This expression is a way for administrators to define a policy where
  interactive users on a machine will cause backfill jobs to be
  removed.
  The default value is \Expr{False}.
  For more information about policy expressions and the Backfill
  state, see section~\ref{sec:Configuring-Policy} beginning on
  page~\pageref{sec:Configuring-Policy}, especially
  sections~\ref{sec:States}, \ref{sec:Activities}, and
  \ref{sec:State-and-Activity-Transitions}.

\end{description}


These macros only apply to the \Condor{startd} daemon when it is running on an
SMP machine. 
See section~\ref{sec:Configuring-SMP} on
page~\pageref{sec:Configuring-SMP} on Configuring The Startd for 
SMP Machines for details.

\begin{description}

\item[\Macro{STARTD\_RESOURCE\_PREFIX}] 
\label{param:StartdResourcePrefix}
  A string which specifies what prefix to give the unique Condor
  resources that are advertised on SMP machines.
  Previously, Condor used the term \Term{virtual machine} to describe
  these resources, so the default value for this setting was ``vm''.
  However, to avoid confusion with other kinds of virtual machines
  (the ones created using tools like VMware or Xen), the old
  \Term{virtual machine} terminology has been changed, and we now use
  the term \Term{slot}.
  Therefore, the default value of this prefix is now ``slot''.
  If sites want to keep using ``vm'', or prefer something other
  ``slot'', this setting enables sites to define what string the
  \Condor{startd} will use to name the individual resources on an SMP
  machine.

\item[\Macro{SLOTS\_CONNECTED\_TO\_CONSOLE}] 
\label{param:SlotsConnectedToConsole}
  An integer which indicates how many of the machine slots
  the \Condor{startd} is representing should be "connected" to the
  console (in other words, notice when there's console activity).
  This defaults to all slots (N in a machine with N CPUs).

\item[\Macro{SLOTS\_CONNECTED\_TO\_KEYBOARD}]
\label{param:SlotsConnectedToKeyboard}
  An integer which indicates how many of the machine slots
  the \Condor{startd} is representing should be "connected" to the
  keyboard (for remote tty activity, as well as console activity).
  Defaults to 1.

\item[\Macro{DISCONNECTED\_KEYBOARD\_IDLE\_BOOST}]
\label{param:DisconnectedKeyboardIdleBoost}
  If there are slots not connected to either the keyboard
  or the console, the corresponding idle time reported will be the
  time since the \Condor{startd} was spawned, plus the value of this macro.
  It defaults to 1200 seconds (20 minutes). 
  We do this because if the slot is configured not to care
  about keyboard activity, we want it to be available to Condor jobs
  as soon as the \Condor{startd} starts up, instead of having to wait for 15
  minutes or more (which is the default time a machine must be idle
  before Condor will start a job).
  If you do not want this boost, set the value to 0.  
  If you change your START expression to require more than 15 minutes
  before a job starts, but you still want jobs to start right away on
  some of your SMP nodes, increase this macro's value.

\item[\Macro{STARTD\_SLOT\_ATTRS}]
\label{param:StartdSlotAttrs}
  The list of ClassAd attribute names that should be shared across all
  slots on the same machine.
  This setting was formerly know as \Macro{STARTD\_VM\_ATTRS} or
  \Macro{STARTD\_VM\_EXPRS} (before version 6.9.3).
  For each attribute in the list, the attribute's value is taken from
  each slot's machine ClassAd and placed into the machine
  ClassAd of all the other slots within the machine.
  For example, if the configuration file for a 2-slot machine
  contains
\begin{verbatim}
        STARTD_SLOT_ATTRS = State, Activity, EnteredCurrentActivity
\end{verbatim}
  then the machine ClassAd for both slots will contain
  attributes that will be of the form:
\begin{verbatim}
     slot1_State = "Claimed"
     slot1_Activity = "Busy"
     slot1_EnteredCurrentActivity = 1075249233
     slot2_State = "Unclaimed"
     slot2_Activity = "Idle"
     slot2_EnteredCurrentActivity = 1075240035
\end{verbatim}


\end{description}

The following settings control the number of slots reported
for a given SMP host, and what attributes each one has.  
They are only needed if you do not want to have an SMP machine report
to Condor with a separate slot for each CPU, with all
shared system resources evenly divided among them.
Please read section~\ref{sec:SMP-Divide} on
page~\pageref{sec:SMP-Divide} for details on how to properly configure
these settings to suit your needs.

\Note You can only change the number of each type of slot
the \Condor{startd} is reporting with a simple reconfig (such as
sending a SIGHUP signal, or using the \Condor{reconfig} command).
You cannot change the definition of the different slot
types with a reconfig.  
If you change them, you must restart the \Condor{startd} for the
change to take effect (for example, using 
\Code{condor\_restart -startd}).

\Note Prior to version 6.9.3, any settings that included the term
``slot'' used to use ``virtual machine'' or ``vm''.
If you're looking for information about one of these older settings,
search for the corresponding attribute names using ``slot'', instead.

\begin{description}

\item[\Macro{MAX\_SLOT\_TYPES}]
\label{param:MaxSlotTypes}
  The maximum number of different slot types.  
  Note: this is the maximum number of different \emph{types}, not of
  actual slots.
  Defaults to 10.  
  (You should only need to change this setting if you define more than
  10 separate slot types, which would be pretty rare.)

\item[\Macro{SLOT\_TYPE\_<N>}]
\label{param:SlotTypeN}
  This setting defines a given slot type, by specifying
  what part of each shared system resource (like RAM, swap space, etc)
  this kind of slot gets.  This setting has \emph{no} effect unless you also
  define \MacroNI{NUM\_SLOTS\_TYPE\_<N>}.
  N can be any integer from 1 to the value of
  \MacroUNI{MAX\_SLOT\_TYPES}, such as
  \MacroNI{SLOT\_TYPE\_1}. 
  The format of this entry can be somewhat complex, so please refer to
  section~\ref{sec:SMP-Divide} on page~\pageref{sec:SMP-Divide} for
  details on the different possibilities.

\item[\Macro{NUM\_SLOTS\_TYPE\_<N>}]
\label{param:NumSlotsTypeN}
  This macro controls how many of a given slot type
  are actually reported to Condor.
  There is no default.

\item[\Macro{NUM\_SLOTS}]
\label{param:NumSlots}
  If your SMP machine is being evenly divided, and the slot
  type settings described above are not being used, this
  macro controls how many slots will be reported.  
  The default is one slot for each CPU.
  This setting can be used to reserve some CPUs on an SMP which would
  not be reported to the Condor pool. You cannot use this parameter to
  make Condor advertise more slots than there are CPUs on the machine.
  To do that, use \Macro{NUM\_CPUS}.

\item[\Macro{ALLOW\_VM\_CRUFT}]
\label{param:AllowVMCruft}
  A boolean value that Condor sets and uses internally, currently
  defaulting to \Expr{True}.  When \Expr{True},
  Condor looks for configuration variables named with the
  previously used string \MacroNI{VM} after searching unsuccessfully
  for variables named with the currently used string \MacroNI{SLOT}.
  When \Expr{False}, Condor does \emph{not} look for variables named
  with the previously used string \MacroNI{VM} after searching
  unsuccessfully for the string \MacroNI{SLOT}. 

\end{description}

The following macros describe the \Prog{cron} capabilities of Condor.
The \Prog{cron} mechanism is used to run executables (called
modules) directly from the \Condor{startd} daemon.
The output from modules
is incorporated into the machine ClassAd generated by the
\Condor{startd}.  These capabilities are used in Hawkeye, but can be
used in other situations as well.

These configuration macros are divided into three sets.
The three sets occurred as the functionality and usage of
Condor's \Prog{cron} capabilities evolved.
The first set applies to both new and older macros and syntax.
The second set applies to the new macros and syntax.
The third set applies only to the older (and outdated) macros and syntax.


This first set of configuration macros applies to both new
and older macros and syntax.
\begin{description}

\item[\Macro{STARTD\_CRON\_NAME}]
\label{param:StartdCronName}
  Defines a logical name to be used in the formation of related
  configuration macro names. While
  not required, this macro makes other macros
  more readable and maintainable.  A common example is
\begin{verbatim}
   STARTD_CRON_NAME = HAWKEYE
\end{verbatim}
  This example allows the naming of other related macros
  to contain the string \verb@"HAWKEYE"@ in their name.

\item[\Macro{STARTD\_CRON\_CONFIG\_VAL}]
\label{param:StartdCronConfigVal}
  This configuration variable can be used to specify the
  \Condor{config\_val} program which the modules (jobs) should use to
  get configuration information from the daemon.  If this is provided,
  a environment variable by the same name with the same value will be
  passed to all modules.

  If \MacroNI{STARTD\_CRON\_NAME}
  is defined, then this configuration macro name is changed from
  \MacroNI{STARTD\_CRON\_CONFIG\_VAL} to
  \MacroNI{\MacroUNI{STARTD\_CRON\_NAME}\_CONFIG\_VAL}.  Example:

\begin{verbatim}
  HAWKEYE_CONFIG_VAL = /usr/local/condor/bin/condor_config_val
\end{verbatim}

\item[\Macro{STARTD\_CRON\_AUTOPUBLISH}]
\label{param:StartdCronAutopublish}
   Optional setting that determines if the \Condor{startd} should
   automatically publish a new update to the \Condor{collector} after
   any of the cron modules produce output.
   Beware that enabling this setting can greatly increase the network
   traffic in a Condor pool, especially when many modules are
   executed, or if the period in which they run is short.
   There are three possible (case insensitive) values for this
   setting: 
   \begin{description}
     \item[\texttt{Never}] This default value causes the
     \Condor{startd} to not automatically publish updates based on
     any cron modules. Instead, updates rely on the usual behavior for sending
     updates, which is periodic, based on the \Macro{UPDATE\_INTERVAL}
     configuration setting, or whenever a given slot
     changes state.
     \item[\texttt{Always}] Causes the \Condor{startd} to always send a new
     update to the \Condor{collector} whenever any module exits.
     \item[\texttt{If\_Changed}] Causes the \Condor{startd} to only send a
     new update to the \Condor{collector} if the output produced by a
     given module is different than the previous output of the
     same module.
     The only exception is the \Attr{LastUpdate} attribute
     (automatically set for all cron modules to be the timestamp when
     the module last ran), which is ignored when
     \MacroNI{STARTD\_CRON\_AUTOPUBLISH} is set to \verb@If_Changed@.
   \end{description}
   Beware that \MacroNI{STARTD\_CRON\_AUTOPUBLISH} does not honor the
   \MacroNI{STARTD\_CRON\_NAME} setting described above.
   Even if \MacroNI{STARTD\_CRON\_NAME} is defined,
   \MacroNI{STARTD\_CRON\_AUTOPUBLISH} will have the same name.

\end{description}

The following 
second set of configuration macros applies only to the new
macros and syntax.
This set is to be used for all new applications.
\begin{description}

\item[\Macro{STARTD\_CRON\_JOBLIST}]
\label{param:StartdCronJobList}
  This configuration variable is defined by a whitespace separated
  list of job names (called modules) to run.
  Each of these is the logical name of the module.  This name
  must be unique (no two modules may have the same name).

  If \MacroNI{STARTD\_CRON\_NAME}
  is defined, then this configuration macro name is changed from
  \MacroNI{STARTD\_CRON\_JOBLIST} to
  \MacroNI{\MacroUNI{STARTD\_CRON\_NAME}\_JOBLIST}.

\item[\Macro{STARTD\_CRON\_<ModuleName>\_PREFIX}]
\label{param:StartdCronModulePrefix}
    Specifies a string which is prepended by
    Condor to all attribute names that the module generates.  For
    example, if a prefix is ``xyz\_'', and an individual attribute is
    named ``abc'', the resulting attribute would be ``xyz\_abc''.
    Although it can be quoted, the prefix can contain only
    alpha-numeric characters.

    If \MacroNI{STARTD\_CRON\_NAME}
    is defined, then this configuration macro name is changed from
    \MacroNI{STARTD\_CRON\_<ModuleName>\_PREFIX} to
    \MacroNI{\MacroUNI{STARTD\_CRON\_NAME}\_<ModuleName>\_PREFIX}.

\item[\Macro{STARTD\_CRON\_<ModuleName>\_EXECUTABLE}]
\label{param:StartdCronModuleExecutable}
    Used to specify the full path to the
    executable to run for this module.  Note that multiple modules may
    specify the same executable (although they need to have different
    names).

    If \MacroNI{STARTD\_CRON\_NAME}
    is defined, then this configuration macro name is changed from
    \MacroNI{STARTD\_CRON\_<ModuleName>\_EXECUTABLE} to
    \MacroNI{\MacroUNI{STARTD\_CRON\_NAME}\_<ModuleName>\_EXECUTABLE}.

\item[\Macro{STARTD\_CRON\_<ModuleName>\_PERIOD}]
\label{param:StartdCronModulePeriod}
    The period specifies time intervals at
    which the module should be run.
    For periodic modules, this
    is the time interval that passes between starting the execution
    of the module.
    The value may be specified in seconds (append value with the
    character 's'), in minutes (append value with the character 'm'),
    or in hours (append value with the character 'h').
    As an example, 5m starts the execution of the module every five minutes.
    If no character is appended to the value, seconds are used as a default.
    For ``Wait For Exit'' mode, the value has a different meaning; in
    this case 
    the period specifies the length of time after the module ceases execution
    before it is restarted.
    The minimum valid value of the period is 1 second.

    If \MacroNI{STARTD\_CRON\_NAME}
    is defined, then this configuration macro name is changed from
    \MacroNI{STARTD\_CRON\_<ModuleName>\_PERIOD} to
    \MacroNI{\MacroUNI{STARTD\_CRON\_NAME}\_<ModuleName>\_PERIOD}.


\item[\Macro{STARTD\_CRON\_<ModuleName>\_MODE}]
\label{param:StartdCronModuleMode}
    Used to specify the ``Mode'' in which the module operates.
    Legal values are ``WaitForExit'' and ``Periodic'' (the
    default).  

    If \MacroNI{STARTD\_CRON\_NAME}
    is defined, then this configuration macro name is changed from
    \MacroNI{STARTD\_CRON\_<ModuleName>\_MODE} to
    \MacroNI{\MacroUNI{STARTD\_CRON\_NAME}\_<ModuleName>\_MODE}.

    The default ``Periodic'' mode is used for most modules.  In
    this mode, the module is expected to be started by the
    \Condor{startd} daemon, gather and publish its data, and then
    exit.

    The ``WaitForExit'' mode is used to specify a module
    which runs in the ``Wait For Exit'' mode.
    In this mode, the \Condor{startd} daemon
    interprets the ``period'' differently.  In this case, it
    refers to the amount of time to wait after the module exits
    before restarting it.  With a value of 1, the module is kept
    running nearly continuously.

    In general, ``Wait For Exit'' mode is for modules that produce
    a periodic stream of updated data, but it can be used for other
    purposes, as well.


\item[\Macro{STARTD\_CRON\_<ModuleName>\_RECONFIG}]
\label{param:StartdCronModuleReconfig}
    The ``ReConfig'' macro is used to specify whether a module
    can handle HUP signals, and should be sent a HUP signal when
    the \Condor{startd} daemon is reconfigured.  The module is
    expected to reread its configuration at that time.  A value
    of ``True'' enables this setting, and ``False'' disables it.

    If \MacroNI{STARTD\_CRON\_NAME}
    is defined, then this configuration macro name is changed from
    \MacroNI{STARTD\_CRON\_<ModuleName>\_RECONFIG} to
    \MacroNI{\MacroUNI{STARTD\_CRON\_NAME}\_<ModuleName>\_RECONFIG}.

\item[\Macro{STARTD\_CRON\_<ModuleName>\_KILL}]
\label{param:StartdCronModuleKill}
    The ``Kill'' macro is applicable on for modules running in the
    ``Periodic'' mode.  Possible values are ``True'' and ``False'' (the
    default).

    If \MacroNI{STARTD\_CRON\_NAME}
    is defined, then this configuration macro name is changed from
    \MacroNI{STARTD\_CRON\_<ModuleName>\_KILL} to
    \MacroNI{\MacroUNI{STARTD\_CRON\_NAME}\_<ModuleName>\_KILL}.

    This macro controls the behavior of the \Condor{startd} when it
    detects that the module's executable is still running when it is time
    to start the module for a run. 	If enabled, the
    \Condor{startd} will kill and restart the process in this
    condition.  If not enabled, the existing process is allowed to
    continue running. 

\item[\Macro{STARTD\_CRON\_<ModuleName>\_ARGS}]
\label{param:StartdCronModuleArgs}
    The command line arguments to pass to the module to be executed. 

    If \MacroNI{STARTD\_CRON\_NAME}
    is defined, then this configuration macro name is changed from
    \MacroNI{STARTD\_CRON\_<ModuleName>\_ARGS} to
    \MacroNI{\MacroUNI{STARTD\_CRON\_NAME}\_<ModuleName>\_ARGS}.

\item[\Macro{STARTD\_CRON\_<ModuleName>\_ENV}]
\label{param:StartdCronModuleEnv}
    The environment string to pass to the module.
    The syntax is the same as that of 
    \MacroNI{DAEMONNAME\_ENVIRONMENT} in ~\ref{param:DaemonNameEnvironment}.

    If \MacroNI{STARTD\_CRON\_NAME}
    is defined, then this configuration macro name is changed from
    \MacroNI{STARTD\_CRON\_<ModuleName>\_ENV} to
    \MacroNI{\MacroUNI{STARTD\_CRON\_NAME}\_<ModuleName>\_ENV}.

\item[\Macro{STARTD\_CRON\_<ModuleName>\_CWD}]
\label{param:StartdCronModuleCwd}
    The working directory in which to start the module.

    If \MacroNI{STARTD\_CRON\_NAME}
    is defined, then this configuration macro name is changed from
    \MacroNI{STARTD\_CRON\_<ModuleName>\_CWD} to
    \MacroNI{\MacroUNI{STARTD\_CRON\_NAME}\_<ModuleName>\_CWD}.

\item[\Macro{STARTD\_CRON\_<ModuleName>\_OPTIONS}]
\label{param:StartdCronModuleOptions}
    A colon separated list of options. 
    Not all combinations of options make sense;
    when a nonsense combination is listed,
    the last one in the list is followed.

    If \MacroNI{STARTD\_CRON\_NAME}
    is defined, then this configuration macro name is changed from
    \MacroNI{STARTD\_CRON\_<ModuleName>\_OPTIONS} to
    \MacroNI{\MacroUNI{STARTD\_CRON\_NAME}\_<ModuleName>\_OPTIONS}.

   \begin{itemize}

	\item The ``WaitForExit'' option enables the ``Wait For Exit''
	mode (see above).

	\item The ``ReConfig'' option enables the ``Reconfig'' setting
	(see above).

	\item The ``NoReConfig'' option disables the ``Reconfig'' setting
	(see above).

	\item The ``Kill'' option enables the ``Kill'' setting (see
	above).

	\item The ``NoKill'' option disables the ``Kill'' setting (see
	above).

    \end{itemize}

\end{description}
Here is a complete configuration example that uses Hawkeye.
\begin{verbatim}
# Hawkeye Job Definitions
STARTD_CRON_NAME = HAWKEYE

# Job 1
HAWKEYE_JOBLIST = job1
HAWKEYE_job1_PREFIX = prefix_
HAWKEYE_job1_EXECUTABLE = $(MODULES)/job1
HAWKEYE_job1_PERIOD = 5m
HAWKEYE_job1_MODE = WaitForExit
HAWKEYE_job1_KILL = false
HAWKEYE_job1_ARGS =-foo -bar
HAWKEYE_job1_ENV = xyzzy=somevalue

# Job 2
HAWKEYE_JOBLIST = $(HAWKEYE_JOBLIST) job2
HAWKEYE_job2_PREFIX = prefix_
HAWKEYE_job2_EXECUTABLE = $(MODULES)/job2
HAWKEYE_job2_PERIOD = 1h
HAWKEYE_job2_ENV = lwpi=somevalue
\end{verbatim}


The following 
third set of configuration macros applies only to older
macros and syntax.
This set is documented for completeness and backwards
compatibility.
Do not use these configuration macros for any new application.
Future releases of Condor may disable the use of this set.

\begin{description}

\item[\Macro{STARTD\_CRON\_JOBS}]
\label{param:StartdCronJobs}
  The list of the modules to execute.  In Hawkeye, this is usually
  named HAWKEYE\_JOBS.
  This configuration variable is defined by
  a whitespace or newline separated list of jobs (called modules) to run,
  where each module is specified using the format
\begin{verbatim}
   modulename:prefix:executable:period[:options]
\end{verbatim}
  Each of these fields can be surrounded by matching quote characters
  (single quote or double quote, but they must match).  This allows
  colon and whitespace characters to be specified.  For example, the
  following specifies an executable name with a colon and a space in it:
\begin{verbatim}
   foo:foo_:"c:/some dir/foo.exe":10m
\end{verbatim}
  These individual fields are described below: 
    \begin{itemize}

    \item \MacroNI{modulename} The logical name of the module.  This 
    must be unique (no two modules may have the same name).  See 
    \MacroNI{STARTD\_CRON\_JOBLIST}

    \item \MacroNI{prefix}
    See \MacroNI{STARTD\_CRON\_<ModuleName>\_PREFIX}

    \item \MacroNI{executable}
    See \MacroNI{STARTD\_CRON\_<ModuleName>\_EXECUTABLE}

    \item \MacroNI{period}
    See \MacroNI{STARTD\_CRON\_<ModuleName>\_PERIOD}

   \item Several options are available. Using more than one
   of these options for one module does not make sense.  If this happens,
   the last one in the list is followed. 
   See \MacroNI{STARTD\_CRON\_<ModuleName>\_OPTIONS}

	\begin{itemize}
	\item The ``Continuous'' option is used to specify a module
	which runs in continuous mode (as described above).  See the
	``WaitForExit'' and ``ReConfig'' options which replace
	``Continuous''.

	This option is now deprecated, and its functionality has been
	replaced by the new ``WaitForExit'' and ``ReConfig'' options,
	which together implement the capabilities of ``Continuous''.
	This option will be removed from a future version of Condor.

	\item The ``WaitForExit'' option

	See the discussion of ``WaitForExit'' in
	\MacroNI{STARTD\_CRON\_<ModuleName>\_OPTIONS} above.

	\item The ``ReConfig'' option

	See the discussion of ``ReConfig in
	\MacroNI{STARTD\_CRON\_<ModuleName>\_OPTIONS} above.

	\item The `NoReConfig'' option

	See the discussion of ``NoReConfig in
	\MacroNI{STARTD\_CRON\_<ModuleName>\_OPTIONS} above.

	\item The ``Kill'' option

	See the discussion of ``Kill'' in
	\MacroNI{STARTD\_CRON\_<ModuleName>\_OPTIONS} above.

	\item The ``NoKill'' option

	See the discussion of ``NoKill'' in
	\MacroNI{STARTD\_CRON\_<ModuleName>\_OPTIONS} above.

    \end{itemize}
	
  \end{itemize}
\Note The configuration file parsing logic will strip whitespace from
the beginning and end of continuation lines.  Thus, a job list like
below will be misinterpreted and will not work as expected:
\begin{verbatim}
# Hawkeye Job Definitions
HAWKEYE_JOBS   =\
    JOB1:prefix_:$(MODULES)/job1:5m:nokill\
    JOB2:prefix_:$(MODULES)/job1_co:1h
HAWKEYE_JOB1_ARGS =-foo -bar
HAWKEYE_JOB1_ENV = xyzzy=somevalue
HAWKEYE_JOB2_ENV = lwpi=somevalue
\end{verbatim}
Instead, write this as below:
\begin{verbatim}
# Hawkeye Job Definitions
HAWKEYE_JOBS   =

# Job 1
HAWKEYE_JOBS = $(HAWKEYE_JOBS) JOB1:prefix_:$(MODULES)/job1:5m:nokill
HAWKEYE_JOB1_ARGS =-foo -bar
HAWKEYE_JOB1_ENV = xyzzy=somevalue

# Job 2
HAWKEYE_JOBS = $(HAWKEYE_JOBS) JOB2:prefix_:$(MODULES)/job2:1h
HAWKEYE_JOB2_ENV = lwpi=somevalue
\end{verbatim}


\end{description}

The following macros control the optional computation of resource
availability statistics in the \Condor{startd}.

\begin{description}

\item[\Macro{STARTD\_COMPUTE\_AVAIL\_STATS}]
\label{param:StartdComputeAvailStats}
  A boolean that determines if the \Condor{startd} computes resource
  availability statistics.  The default is False.

If \Macro{STARTD\_COMPUTE\_AVAIL\_STATS} = True, the \Condor{startd} will
define the following ClassAd attributes for resources:

\begin{description}
\item[\AdAttr{AvailTime}]
\index{ClassAd machine attribute!AvailTime}
  The proportion of the time (between 0.0 and 1.0)
  that this resource has been in a state other than Owner.
\item[\AdAttr{LastAvailInterval}]
\index{ClassAd machine attribute!LastAvailInterval}
  The duration (in seconds) of the last period between Owner states.
\end{description}

The following attributes will also be included if the resource is
not in the Owner state:

\begin{description}
\item[\AdAttr{AvailSince}]
\index{ClassAd machine attribute!AvailSince}
  The time at which the resource last left the
  Owner state.  Measured in the number of seconds since the
  epoch (00:00:00 UTC, Jan 1, 1970).
\item[\AdAttr{AvailTimeEstimate}]
\index{ClassAd machine attribute!AvailTimeEstimate}
  Based on past history, an estimate
  of how long the current period between Owner states will last.
\end{description}

\item[\Macro{STARTD\_AVAIL\_CONFIDENCE}]
\label{param:StartdAvailConfidence}
  A floating point number representing the confidence level of the
  \Condor{startd} daemon's AvailTime estimate.
  By default, the estimate is based on
  the 80th percentile of past values
  (that is, the value is initially set to 0.8).

\item[\Macro{STARTD\_MAX\_AVAIL\_PERIOD\_SAMPLES}]
\label{param:StartdMaxAvailPeriodSamples}
  An integer that limits the number of samples of past available
  intervals stored by the \Condor{startd} to limit memory and disk consumption.
  Each sample requires 4 bytes of memory and approximately 10 bytes of
  disk space.

\end{description}


The following configuration variables support java universe jobs.

\begin{description}
\item[\Macro{JAVA}]
\label{param:Java}
  The full path to the Java interpreter (the Java Virtual Machine).

\item[\Macro{JAVA\_MAXHEAP\_ARGUMENT}]
\label{param:JavaMaxheapArgument}
  An incomplete command line argument to the Java interpreter
  (the Java Virtual Machine)
  to specify the switch name for the Maxheap Argument.
  Condor uses it to construct the maximum heap size
  for the Java Virtual Machine. 
  For example, the value for the Sun JVM is \Arg{-Xmx}.

\item[\Macro{JAVA\_CLASSPATH\_ARGUMENT}]
\label{param:JavaClasspathArgument}
  The command line argument to the Java interpreter (the Java Virtual Machine)
  that specifies the Java Classpath.
  Classpath is a Java-specific term that denotes the list of
  locations (\File{.jar} files and/or directories)
  where the Java interpreter can
  look for the Java class files that a Java program requires.

\item[\Macro{JAVA\_CLASSPATH\_SEPARATOR}]
\label{param:JavaClasspathSeparator}
  The single character used to delimit constructed entries in the
  Classpath for the given operating system and Java Virtual Machine.
  If not defined, the operating system is queried for its default
  Classpath separator.

\item[\Macro{JAVA\_CLASSPATH\_DEFAULT}]
\label{param:JavaClasspathDefault}
  A list of path names to \File{.jar} files to be added to the Java Classpath 
  by default.
  The comma and/or space character delimits list entries.

\item[\Macro{JAVA\_EXTRA\_ARGUMENTS}]
\label{param:JavaExtraArguments}
  A list of additional arguments to be passed to the Java executable.
\end{description}

These macros control the system of job hooks invoked by the
\Condor{startd} to optionally fetch work.
See section~\ref{sec:job-hooks} on page~\pageref{sec:job-hooks} on
``Job Hooks'' for more details.

\begin{description}

\item[\Macro{SLOTN\_JOB\_HOOK\_KEYWORD}]
\label{param:SlotNJobHookKeyword}
  The keyword used to define which set of hooks a particular
  compute slot should invoke.
  Note that the ``N'' in ``SLOTN'' should be replaced with the slot
  identification number, for example, on slot1, this setting would be
  called \MacroNI{[SLOT1\_JOB\_HOOK\_KEYWORD}.
  There is no default keyword.
  Sites that wish to use these job hooks must explicitly define the
  keyword (and the corresponding hook paths).

\item[\Macro{STARTD\_JOB\_HOOK\_KEYWORD}]
\label{param:StartdJobHookKeyword}
  The keyword used to define which set of hooks a particular
  \Condor{startd} should invoke.
  This setting is only used if a slot-specific keyword is not defined
  for a given compute slot.
  There is no default keyword.
  Sites that wish to use these job hooks must explicitly define the
  keyword (and the corresponding hook paths).

\item[\Macro{HOOK\_FETCH\_WORK}]
\label{param:HookFetchWork}
  The full path to the program to invoke whenever the \Condor{startd}
  wants to fetch work.
  The actual configuration setting must be prefixed with a hook keyword.
  There is no default.

\item[\Macro{HOOK\_REPLY\_CLAIM}]
\label{param:HookReplyClaim}
  The full path to the program to invoke whenever the \Condor{startd}
  finishes fetching a job and decides what to do with it.
  The actual configuration setting must be prefixed with a hook keyword.
  There is no default.

\item[\Macro{HOOK\_EVICT\_CLAIM}]
\label{param:HookEvictClaim}
  The full path to the program to invoke whenever the \Condor{startd}
  needs to evict a fetched claim.
  The actual configuration setting must be prefixed with a hook keyword.
  There is no default.

\item[\Macro{FetchWorkDelay}]
\label{param:FetchWorkDelay}
  An expression that defines the number of seconds that the
  \Condor{startd} should wait after an invocation of
  \Macro{HOOK\_FETCH\_WORK} completes before the hook should be
  invoked again.
  The expression is evaluated in the context of the slot ClassAd, and
  the ClassAd of the currently running job (if any).
  The expression must evaluate to an integer.
  If not defined, the \Condor{startd} will wait 300 seconds (five
  minutes) between attempts to fetch work.
  For more information about this expression, see
  section~\ref{sec:job-hooks-fetch-work-delay} on
  page~\pageref{sec:job-hooks-fetch-work-delay}.

\end{description}

These macros control the power management capabilities of the 
\Condor{startd} to optionally put the machine in to a low power state.
See section~\ref{sec:power-man} on page~\pageref{sec:power-man} on
``Power Management'' for more details.

\begin{description}

\item[\Macro{HIBERNATE\_CHECK\_INTERVAL}]
\label{param:HibernateCheckInterval}
  Determines how often the \Condor{startd} should check if the 
  machine is ready to enter a low power state.  If the value is 
  of \MacroNI{HIBERNATE\_CHECK\_INTERVAL} is 0, then the check is 
  disabled; otherwise, the \MacroNI{HIBERNATE} expression will be 
  evaluated in the context of each slot at the given interval.  
  This macro is defined in terms of seconds and defaults to 0,
  if used, a value $\ge300$ (5 minutes) is recommended.

  There is one special case when this interval is ignored: when the 
  machine has just returned from a low power state (excluding 
  shutdown (5)).  In order to avoid machines from volleying between 
  a running state and a low power state, an hour of uptime is enforced
  after a machine has been woken.  After this time has passed, regular
  checks will resume.

\item[\Macro{HIBERNATE}]
\label{param:Hibernate}
  An integer expression that, when non-zero, causes Condor to put the 
  machine into a low power state given by the evaluation of the 
  expression.  The following values are supported:

  \begin{itemize}
  \item[] 0 - No-op: do not enter a low power state
  \item[] 1 - On Windows, see 3
  \item[] 2 - On Windows, see 3
  \item[] 3 - Sleep (standby)
  \item[] 4 - Hibernate
  \item[] 5 - Shutdown (soft-off)
  \end{itemize}
  
  The \MacroNI{HIBERNATE} expression is written in terms of the S-states
  as defined in the Advanced Configuration and Power Interface 
  (ACPI) specification.  The S-states take the form S$n$, where $n$ is 
  an integer in the range $0$ to $5$, inclusive.  The number that results 
  from evaluating the expression determines which S-state to enter. The 
  $n$ from S$n$ notation was adopted because at this junction in time 
  it appears to be the standard naming scheme for power states on several
  popular Operating Systems, including various flavors of Windows and Linux
  distributions.

  Since this expression is evaluated in the context of each slot on the
  machine, any one slot has veto power over the other slots.  If the 
  evaluation of \MacroNI{HIBERNATE} in one slot evaluates to 0, then
  the machine will not be placed into a low power state.  On the other
  hand, if all slots evaluate to a non-zero value, but differ in value, 
  then the largest value is used as the representative power state.

\end{description}

%%%%%%%%%%%%%%%%%%%%%%%%%%%%%%%%%%%%%%%%%%%%%%%%%%%%%%%%%%%%%%%%%%%%%%%%%%%
\subsection{\label{sec:Schedd-Config-File-Entries}\condor{schedd}
Configuration File Entries}
%%%%%%%%%%%%%%%%%%%%%%%%%%%%%%%%%%%%%%%%%%%%%%%%%%%%%%%%%%%%%%%%%%%%%%%%%%%

\index{configuration!condor\_schedd configuration variables}
These macros control the \Condor{schedd}.
\begin{description}

\item[\Macro{SHADOW}] \label{param:Shadow} This macro determines the
  full path of the \Condor{shadow} binary that the \Condor{schedd}
  spawns.  It is normally defined in terms of \MacroUNI{SBIN}. 
  
\item[\Macro{START\_LOCAL\_UNIVERSE}] \label{param:StartLocalUniverse}
  A boolean value that defaults to \Expr{True}.
  The \Condor{schedd} uses this macro to determine whether to start
  a \SubmitCmd{local} universe job. 
  At intervals determined by \MacroNI{SCHEDD\_INTERVAL}, 
  the \Condor{schedd} daemon evaluates this macro
  for each idle \SubmitCmd{local} universe job that it has.
  For each job, if the \MacroNI{START\_LOCAL\_UNIVERSE} 
  macro is \Expr{True}, then the job's \Macro{Requirements} expression
  is evaluated. If both conditions are met, then the job is allowed
  to begin execution. 
  
  The following example only allows 10 \SubmitCmd{local} universe jobs to
  execute concurrently. The attribute \Attr{TotalLocalJobsRunning}
  is supplied by \Condor{schedd}'s ClassAd:
  
  \footnotesize
  \begin{verbatim}
    START_LOCAL_UNIVERSE = TotalLocalJobsRunning < 10
  \end{verbatim}
  \normalsize
  
\item[\Macro{STARTER\_LOCAL}] \label{param:StarterLocal}
  The complete path and executable name of the \Condor{starter} to
  run for \SubmitCmd{local} universe jobs.  This variable's value
  is defined in the initial configuration provided with Condor as
  \footnotesize
  \begin{verbatim}
  STARTER_LOCAL = $(SBIN)/condor_starter
  \end{verbatim}
  \normalsize
  This variable would only be modified or hand added into 
  the configuration for a pool to be upgraded from one
  running a version of Condor that existed before the
  \SubmitCmd{local} universe to one that includes the
  \SubmitCmd{local} universe, but without utilizing the newer, provided
  configuration files.

\item[\Macro{START\_SCHEDULER\_UNIVERSE}] \label{param:StartSchedulerUniverse}
  A boolean value that defaults to \Expr{True}.
  The \Condor{schedd} uses this macro to determine whether to start
  a \SubmitCmd{scheduler} universe job. 
  At intervals determined by \MacroNI{SCHEDD\_INTERVAL}, 
  the \Condor{schedd} daemon evaluates this macro
  for each idle \SubmitCmd{scheduler} universe job that it has.
  For each job, if the \MacroNI{START\_SCHEDULER\_UNIVERSE} 
  macro is \Expr{True}, then the job's \Macro{Requirements} expression
  is evaluated. If both conditions are met, then the job is allowed
  to begin execution. 
  
  The following example only allows 10 \SubmitCmd{scheduler} universe jobs to
  execute concurrently. The attribute \Attr{TotalSchedulerJobsRunning}
  is supplied by \Condor{schedd}'s ClassAd:
  
  \footnotesize
  \begin{verbatim}
    START_SCHEDULER_UNIVERSE = TotalSchedulerJobsRunning < 10
  \end{verbatim}
  \normalsize
  
  
\item[\Macro{MAX\_JOBS\_RUNNING}] \label{param:MaxJobsRunning} This
  macro limits the number of processes spawned by a given
  \Condor{schedd}, for all job universes except the 
  grid universe.  See section~\ref{sec:Choosing-Universe}.
  This includes, but is not limited to \Condor{shadow} processes,
  and scheduler universe processes, including \Condor{dagman}.
  The actual
  number of \Condor{shadow}s may be less if you have reached
  your \MacroUNI{RESERVED\_SWAP} limit.
  This macro has a default value of 200.

\item[\Macro{MAX\_JOBS\_SUBMITTED}] \label{param:MaxJobsSubmitted}
  This integer value limits the number of jobs permitted in 
  a \Condor{schedd} daemon's queue. Submission of a new cluster
  of jobs fails, if the total number of jobs would exceed this limit. 
  The default value for this variable is the largest positive
  integer value.

\item[\Macro{MAX\_SHADOW\_EXCEPTIONS}]
  \label{param:MaxShadowExceptions} This macro controls the maximum
  number of times that \Condor{shadow} processes can have a fatal
  error (exception) before the \Condor{schedd} will relinquish
  the match associated with the dying shadow.  Defaults to 5.

\item[\Macro{MAX\_CONCURRENT\_DOWNLOADS}]
  \label{param:MaxConcurrentDownloads} This specifies the maximum
  number of simultaneous transfers of output files from execute
  machines to the submit machine.  The limit applies to all jobs
  submitted from the same \Condor{schedd}.  The default is 10.  A
  setting of 0 means unlimited transfers.  This limit currently does
  not apply to grid universe jobs or standard universe jobs, and it
  also does not apply to streaming output files.  When the limit is
  reached, additional transfers will queue up and wait before
  proceeding.

\item[\Macro{MAX\_CONCURRENT\_UPLOADS}]
  \label{param:MaxConcurrentUploads} This specifies the maximum
  number of simultaneous transfers of input files from the submit
  machine to execute machines.  The limit applies to all jobs
  submitted from the same \Condor{schedd}.  The default is 10.  A
  setting of 0 means unlimited transfers.  This limit currently does
  not apply to grid universe jobs or standard universe jobs.  When the
  limit is reached, additional transfers will queue up and wait before
  proceeding.

\item[\Macro{SCHEDD\_QUERY\_WORKERS}] \label{param:ScheddQueryWorkers}
  This specifies the maximum number of concurrent sub-processes that
  the \Condor{schedd} will spawn to handle queries.  The setting is
  ignored in Windows.  In Unix, the default is 3.  If the limit is
  reached, the next query will be handled in the \Condor{schedd}'s main
  process.

\item[\Macro{SCHEDD\_INTERVAL}] \label{param:ScheddInterval} This
  macro determines the maximum interval for both how often the
  \Condor{schedd} sends a ClassAd update to the \Condor{collector} and
  how often the \Condor{schedd} daemon evaluates jobs.  It is defined
  in terms of seconds and defaults to 300 (every 5 minutes).

\item[\Macro{SCHEDD\_INTERVAL\_TIMESLICE}]
\label{param:ScheddIntervalTimeslice} The bookkeeping done by the
\Condor{schedd} takes more time when there are large numbers of jobs
in the job queue.  However, when it is not too expensive to do this
bookkeeping, it is best to keep the collector up to date with the
latest state of the job queue.  Therefore, this macro is used to
adjust the bookkeeping interval so that it is done more frequently
when the cost of doing so is relatively small, and less frequently
when the cost is high.  The default is 0.05, which means the schedd
will adapt its bookkeeping interval to consume no more than 5\% of the
total time available to the schedd.  The lower bound is configured by
\Macro{SCHEDD\_MIN\_INTERVAL} (default 5 seconds), and the upper bound
is configured by \Macro{SCHEDD\_INTERVAL} (default 300 seconds).


% \item[\Macro{REAL\_TIME\_JOB\_SUSPEND\_UPDATES}] 
%  \label{param:RealTimeJobSuspendUpdates}
%  If set to \Expr{True},
%  then the \Condor{shadow} will immediately update the
%  \Condor{schedd} upon the suspension or resumption of a job.
%  This allows \Condor{q} to show a
%  job in a suspended state in its default output.
%  In the \Expr{ST} column, there will be an \Expr{S} instead of an
%  \Expr{R} when the job is running, but suspended.

%  This attribute's real time connotation is currently applied only
%  to jobs in the vanilla, standard, and java universes.
%  Other universes may display a suspension state where applicable,
%  but the information may be stale.

%  The default value and when not present in the configuration
%  file is \Expr{False}.
  
\item[\Macro{JOB\_START\_COUNT}] \label{param:JobStartCount} This
  macro works together with the \Macro{JOB\_START\_DELAY} macro to
  throttle job starts.  The default and minimum values for this
  integer configuration variable are both 1.

\item[\Macro{JOB\_START\_DELAY}] \label{param:JobStartDelay}
  This integer-valued macro works together with the
  \Macro{JOB\_START\_COUNT} macro
  to throttle job starts.  The  \Condor{schedd} daemon starts
  \MacroUNI{JOB\_START\_COUNT} jobs at a time, then delays for
  \MacroUNI{JOB\_START\_DELAY} seconds before starting the next set of jobs.
  This delay prevents a sudden, large load on resources required by
  the jobs during their startup phase.
  The resulting job start rate
  averages as fast as
  (\MacroUNI{JOB\_START\_COUNT}/\MacroUNI{JOB\_START\_DELAY}) jobs/second.
  This configuration variable is also used during the graceful shutdown of the
  \Condor{schedd} daemon.
  During graceful shutdown, this macro determines the wait time in
  between requesting each \Condor{shadow} daemon to gracefully shut down.  
  It is defined in terms of seconds and defaults to 0, which means jobs
  will be started as fast as possible.  If you wish to throttle the rate
  of specific types of jobs, you can use the job attribute
  \AdAttr{NextJobStartDelay}.

\item[\Macro{MAX\_NEXT\_JOB\_START\_DELAY}] \label{param:MaxNextJobStartDelay}
  An integer number of seconds representing the maximum allowed value
  of the job ClassAd attribute \AdAttr{NextJobStartDelay}.  It defaults to 600,
  which is 10 minutes.

\item[\Macro{JOB\_IS\_FINISHED\_INTERVAL}] \label{param:JobIsFinishedInterval}
  The \Condor{schedd} maintains a list of jobs that are ready to permanently
  leave the job queue, e.g. they have completed or been removed.  This
  integer-valued macro specifies a delay in seconds to place between the
  taking jobs permanently out of the queue.  The default value is 0, which
  tells the \Condor{schedd} to not impose any delay.  
  
\item[\Macro{ALIVE\_INTERVAL}] \label{param:AliveInterval} This
  macro determines how often the \Condor{schedd} should send a keep
  alive message to any \Condor{startd} it has claimed.
  When the \Condor{schedd} claims a \Condor{startd}, it tells the \Condor{startd} how often it is
  going to send these messages.
  If the \Condor{startd} does not receive any of these keep alive messages
  during a certain period of time (defined via
  \Macro{MAX\_CLAIM\_ALIVES\_MISSED}, described on
  page~\pageref{param:MaxClaimAlivesMissed})
  the \Condor{startd} releases the claim, and the \Condor{schedd} no longer pays for
  the resource (in terms of user priority in the system).
  The macro is defined in terms of seconds and defaults to 300 (every
  5 minutes).

\item[\Macro{REQUEST\_CLAIM\_TIMEOUT}]
  \label{param:RequestClaimTimeout} This macro sets the time (in
  seconds) that the \Condor{schedd} will wait for a claim to be granted by the
  \Condor{startd}.  The default is 30 minutes.  This is only likely to matter
  if the \Condor{startd} has an existing claim and it takes a long time for the
  existing claim to be preempted due to \Expr{MaxJobRetirementTime}.
  Once a request times out, the \Condor{schedd} will simply begin the process
  of finding a machine for the job all over again.

\item[\Macro{SHADOW\_SIZE\_ESTIMATE}] \label{param:ShadowSizeEstimate}
  This macro sets the estimated virtual memory size of each
  \Condor{shadow} process.  Specified in kilobytes.  The default
  varies from platform to platform.

\item[\Macro{SHADOW\_RENICE\_INCREMENT}]
  \label{param:ShadowReniceIncrement} When the \Condor{schedd} spawns a new
  \Condor{shadow}, it can do so with a \Term{nice-level}.  A
  nice-level is a Unix mechanism that allows users to assign their own
  processes a lower priority so that the processes run with less
  priority than other tasks on the machine.  The value can be any
  integer between 0 and 19, with a value of 19 being the lowest
  priority.  It defaults to 0.

\item[\Macro{SCHED\_UNIV\_RENICE\_INCREMENT}]
  \label{param:SchedUnivReniceIncrement} Analogous to 
  \MacroNI{JOB\_RENICE\_INCREMENT} and
  \MacroNI{SHADOW\_RENICE\_INCREMENT}, scheduler universe jobs can be
  given a nice-level.  The value can be any integer between 0 and 19,
  with a value of 19 being the lowest priority.  It defaults to 0.

\item[\Macro{QUEUE\_CLEAN\_INTERVAL}] \label{param:QueueCleanInterval}
  The \Condor{schedd} maintains the job queue on a given machine.  It does so
  in a persistent way such that if the \Condor{schedd} crashes, it can recover
  a valid state of the job queue.  The mechanism it uses is a
  transaction-based log file (the \File{job\_queue.log} file,
  not the \File{SchedLog} file).  This file contains an initial
  state of the job queue, and a series of transactions that were
  performed on the queue (such as new jobs submitted, jobs completing,
  and checkpointing).  Periodically, the \Condor{schedd} will go through
  this log, truncate all the transactions and create a new file with
  containing only the new initial state of the log.
  This is a somewhat expensive operation,
  but it speeds up when the \Condor{schedd} restarts since there are
  fewer transactions it has to play to figure out what state the job
  queue is really in.  This macro determines how often the \Condor{schedd}
  should rework this queue to cleaning it up.  It is defined in terms of
  seconds and defaults to 86400 (once a day). 
  
\item[\Macro{WALL\_CLOCK\_CKPT\_INTERVAL}] \label{param:WallClockCkptInterval}
  The job queue contains a counter for each job's ``wall clock'' run
  time, i.e., how long each job has executed so far.  This counter is
  displayed by \Condor{q}.  The counter is updated when the job is
  evicted or when the job completes.  When the \Condor{schedd} crashes, the run
  time for jobs that are currently running will not be added to the
  counter (and so, the run time counter may become smaller than the
  CPU time counter).  The \Condor{schedd} saves run time ``checkpoints''
  periodically for running jobs so if the \Condor{schedd} crashes, only run
  time since the last checkpoint is lost.  This macro controls how
  often the \Condor{schedd} saves run time checkpoints.  It is defined in terms
  of seconds and defaults to 3600 (one hour).  A value of 0 will
  disable wall clock checkpoints.

\item[\Macro{ALLOW\_REMOTE\_SUBMIT}] \label{param:AllowRemoteSubmit}
  Starting with Condor Version 6.0, users can run \Condor{submit} on
  one machine and actually submit jobs to another machine in the
  pool.  This is called a \Term{remote submit}.  Jobs submitted in
  this way are entered into the job queue owned by the Unix user
  nobody.
  This macro determines whether this is allowed.
  It defaults to \Expr{False}.


\item[\Macro{QUEUE\_ALL\_USERS\_TRUSTED}]. \label{param:QueueAllUsersTrusted}
  Defaults to False. If set to True, then unauthenticated users are allowed
  to write to the queue, and also we always trust whatever the \Attr{Owner}
  value is set to be by the client in the job ad. This was added so users
  can continue to use the SOAP web-services interface over HTTP (w/o
  authenticating) to submit jobs in a secure, controlled environment -- for
  instance, in a portal setting.
     
\item[\Macro{QUEUE\_SUPER\_USERS}] \label{param:QueueSuperUsers} This
  macro determines what user names on a given machine have
  \Term{super-user access} to the job queue, meaning that they can
  modify or delete the job ClassAds of other users.  (Normally, you
  can only modify or delete ClassAds from the job queue that you own).
  Whatever user name corresponds with the UID that Condor is running as
  (usually the Unix user condor) will automatically be included in this list
  because that is needed for Condor's proper functioning.  See
  section~\ref{sec:uids} on UIDs in Condor for more details on
  this.  By default, we give root the ability to remove other
  user's jobs, in addition to user condor.
      
\item[\Macro{SCHEDD\_LOCK}] \label{param:ScheddLock} This macro
  specifies what lock file should be used for access to the
  \File{SchedLog} file.  It must be a separate file from the
  \File{SchedLog}, since the \File{SchedLog} may be rotated and
  synchronization across log file rotations
  is desired.
  This macro is defined relative to the \MacroUNI{LOCK} macro.

\item[\Macro{SCHEDD\_NAME}] \label{param:ScheddName}
  A unique name given for a \Condor{schedd} daemon on a machine.
  Defaults to the fully qualified hostname of the machine where the
  \Condor{schedd} is running.
  However, this configuration macro is used to uniquely identify
  \Condor{schedd} ClassAds if more than one \Condor{schedd} is running
  on the same host, for example, with many Personal Condor
  installations running as different users on the same machine.
  In that case, the recommended form for \MacroNI{SCHEDD\_NAME} is
  \verb$username@full.host.name$, where \verb$username$ is the
  user that a given \Condor{schedd} is running as. 

  See the description of \MacroNI{MASTER\_NAME} in
  section~\ref{param:MasterName} on page~\pageref{param:MasterName}
  for a description of valid Condor daemon names.
  Also, note that if the \MacroNI{MASTER\_NAME} setting is defined for
  the \Condor{master} that spawned a given \Condor{schedd}, that name
  will take precedence over whatever is defined in
  \MacroNI{SCHEDD\_NAME}. 

\item[\Macro{SCHEDD\_ATTRS}] \label{param:ScheddAttrs} This macro is
  described in section~\ref{param:SubsysExprs} as
  \MacroNI{<SUBSYS>\_ATTRS}.

\item[\Macro{SCHEDD\_DEBUG}] \label{param:ScheddDebug} This macro
  (and other settings related to debug logging in the \Condor{schedd}) is
  described in section~\ref{param:SubsysDebug} as
  \MacroNI{<SUBSYS>\_DEBUG}.

\item[\Macro{SCHEDD\_ADDRESS\_FILE}] \label{param:ScheddAddressFile}
  This macro is described in
  section~\ref{param:SubsysAddressFile} as
  \MacroNI{<SUBSYS>\_ADDRESS\_FILE}. 

\item[\Macro{SCHEDD\_EXECUTE}] \label{param:ScheddExecute}
  A directory to use as a temporary sandbox for local universe jobs.
  Defaults to \File{\MacroUNI{SPOOL}/execute}.

\item[\Macro{FLOCK\_NEGOTIATOR\_HOSTS}] \label{param:FlockNegotiatorHosts} 
  This macro defines a list of negotiator host names (not including the
  local \MacroUNI{NEGOTIATOR\_HOST} machine) for pools in which the
  \Condor{schedd} should attempt to run jobs.  Hosts in the list should be in
  order of preference.  The \Condor{schedd} will only send a request to a
  central manager in the list if the local pool and pools earlier in
  the list are not satisfying all the job requests.
  \MacroUNI{HOSTALLOW\_NEGOTIATOR\_SCHEDD} (see
  section~\ref{param:HostAllow}) must also be configured to allow
  negotiators from all of the \MacroUNI{FLOCK\_NEGOTIATOR\_HOSTS} to
  contact the \Condor{schedd}.  Please make sure the
  \MacroUNI{NEGOTIATOR\_HOST} is first in the
  \MacroUNI{HOSTALLOW\_NEGOTIATOR\_SCHEDD} list.  Similarly, the
  central managers of the remote pools must be configured to listen to
  requests from this \Condor{schedd}.

\item[\Macro{FLOCK\_COLLECTOR\_HOSTS}] \label{param:FLockCollectorHosts}
  This macro defines a list of collector host names for pools in which
  the \Condor{schedd} should attempt to run jobs.  The
  collectors must be specified in order, corresponding to the
  \MacroUNI{FLOCK\_NEGOTIATOR\_HOSTS} list.  In the typical case, where each pool
  has the collector and negotiator running on the same machine,
  \MacroUNI{FLOCK\_COLLECTOR\_HOSTS} should have the same definition as
  \MacroUNI{FLOCK\_NEGOTIATOR\_HOSTS}.

% No longer needed as of 6.5.2.  Now done automatically.
%\item[\Macro{FLOCK\_VIEW\_SERVERS}] \label{param:FlockViewServers}
% This macro defines a list of hostnames where the condor-view server
% is running in the pools to which you want your jobs to flock.  The
% order of this list must correspond to the order of the
% \MacroUNI{FLOCK\_COLLECTOR\_HOSTS} and
% \MacroUNI{FLOCK\_NEGOTIATOR\_HOSTS} lists.  List items may be empty
% for pools which don't use a separate condor-view server.
% \MacroUNI{FLOCK\_VIEW\_SERVER} may be left undefined if no remote
% pools use separate condor-view servers.  Note: It is required that
% the same hostname does not appear twice in the
% \MacroUNI{FLOCK\_VIEW\_SERVERS} list and that the
% \MacroUNI{CONDOR\_VIEW\_HOST} does not appear in the
% \MacroUNI{FLOCK\_VIEW\_SERVERS} list.

\item[\Macro{NEGOTIATE\_ALL\_JOBS\_IN\_CLUSTER}]
  \label{param:NegotiateAllJobsInCluster}
  If this macro is set to False (the default), when the \Condor{schedd} fails
  to start an idle job, it will not try to start any other
  idle jobs in the same cluster during that negotiation cycle.  This
  makes negotiation much more efficient for large job clusters.
  However, in some cases other jobs in the cluster can be started even
  though an earlier job can't.  For example, the jobs' requirements
  may differ, because of different disk space, memory, or
  operating system requirements.  Or, machines may be willing to run
  only some jobs in the cluster, because their requirements reference
  the jobs' virtual memory size or other attribute.  Setting this
  macro to True will force the \Condor{schedd} to try to start all idle jobs in
  each negotiation cycle.  This will make negotiation cycles last
  longer, but it will ensure that all jobs that can be started will be
  started.

\item[\Macro{PERIODIC\_EXPR\_INTERVAL}]
  \label{param:PeriodicExprInterval} This macro determines the minimum period,
  in seconds, between evaluation of periodic job control expressions,
  such as periodic\_hold, periodic\_release, and periodic\_remove,
  given by the user in a Condor submit file. By default, this value is
  60 seconds.  A value of 0 prevents the \Condor{schedd} from
  performing the periodic evaluations.

\item[\Macro{PERIODIC\_EXPR\_TIMESLICE}]
  \label{param:PeriodicExprTimeslice} This macro is used to adapt the
  frequency with which the \Condor{schedd} evaluates periodic job
  control expressions.  When the job queue is very large, the cost of
  evaluating all of the ClassAds is high, so in order for the
  \Condor{schedd} to continue to perform well, it makes sense to
  evaluate these expressions less frequently.  The default time slice
  is 0.01, so the \Condor{schedd} will set the interval between
  evaluations so that it spends only 1\% of its time in this activity.
  The lower bound for the interval is configured by
  \Macro{PERIODIC\_EXPR\_INTERVAL} (default 60 seconds).

\item[\Macro{SYSTEM\_PERIODIC\_HOLD}]
  \label{param:SystemPeriodicHold} This expression behaves identically
  to the job expression \AdAttr{periodic\_hold}, but it is evaluated by
  the \Condor{schedd} daemon individually for each job in the queue.
  It defaults to \Expr{False}.
  When \Expr{True}, it causes the job to stop running and go on hold.
  Here is an
  example that puts jobs on hold if they have been restarted too many
  times, have an unreasonably large virtual memory \Attr{ImageSize}, or have
  unreasonably large disk usage for an invented environment.

\footnotesize
\begin{verbatim}
SYSTEM_PERIODIC_HOLD = \
  (JobStatus == 1 || JobStatus == 2) && \
  (JobRunCount > 10 || ImageSize > 3000000 || DiskUsage > 10000000)
\end{verbatim}
\normalsize

\item[\Macro{SYSTEM\_PERIODIC\_RELEASE}]
  \label{param:SystemPeriodicRelease} This expression behaves identically
  to the job expression \AdAttr{periodic\_release}, but it is evaluated by
  the \Condor{schedd} daemon individually for each job in the queue.
  It defaults to \Expr{False}.
  When \Expr{True}, it causes a held job to return to the idle state.
  Here is an example
  that releases jobs from hold if they have tried to run less than 20
  times, have most recently been on hold for over 20 minutes, and have
  gone on hold due to ``Connection timed out'' when trying to execute
  the job, because the file system containing the job's executable is
  temporarily unavailable.

\footnotesize
\begin{verbatim}
SYSTEM_PERIODIC_RELEASE = \
  (JobRunCount < 20 && CurrentTime - EnteredCurrentStatus > 1200 ) && ( \
    (HoldReasonCode == 6 && HoldReasonSubCode == 110) \
  )
\end{verbatim} 
\normalsize


\item[\Macro{SYSTEM\_PERIODIC\_REMOVE}]
  \label{param:SystemPeriodicRemove} This expression behaves identically
  to the job expression \AdAttr{periodic\_remove}, but it is evaluated by
  the \Condor{schedd} daemon individually for each job in the queue.
  It defaults to \Expr{False}.
  When \Expr{True}, it causes the job to be removed from the queue.
  Here is an example
  that removes jobs which have been on hold for 30 days:

\footnotesize
\begin{verbatim}
SYSTEM_PERIODIC_REMOVE = \
  (JobStatus == 5 && CurrentTime - EnteredCurrentStatus > 3600*24*30)
\end{verbatim}
\normalsize

\item[\Macro{SCHEDD\_ASSUME\_NEGOTIATOR\_GONE}]
  \label{param:ScheddAssumeNegotiatorGone} This macro determines the period,
  in seconds, that the \Condor{schedd} will wait for the \Condor{negotiator} to
  initiate a negotiation cycle before the schedd will simply try to claim
  any local \Condor{startd}.  This allows for a machine that is acting as
  both a submit and execute node to run jobs locally if it cannot
  communicate with the central manager.  The default value, if not
  specified, is 4 x \MacroUNI{NEGOTIATOR\_INTERVAL}.  If
  \MacroUNI{NEGOTIATOR\_INTERVAL} is not defined, then 
  \MacroNI{SCHEDD\_ASSUME\_NEGOTIATOR\_GONE} will default to 1200 (20
  minutes).

\item[\Macro{SCHEDD\_ROUND\_ATTR\_<xxxx>}]
  \label{param:ScheddRoundAttr} This is used to round off attributes in
  the job ClassAd so that similar jobs may be grouped together for
  negotiation purposes.  There are two cases.  One is that a
  percentage such as 25\% is specified.  In this case, the value of
  the attribute named \verb@<xxxx>\@ in the job ClassAd will be
  rounded up to the next multiple of the specified percentage of the
  values order of magnitude.  For example, a setting of 25\% will
  cause a value near 100 to be rounded up to the next multiple of 25
  and a value near 1000 will be rounded up to the next multiple of
  250.  The other case is that an integer, such as 4, is specified
  instead of a percentage.  In this case, the job attribute is rounded
  up to the specified number of decimal places.
  Replace \verb@<xxxx>@ with the name of the attribute to round, and set this
  macro equal to the number of decimal places to round up.  For example, to
  round the value of job ClassAd attribute \Attr{foo}  up to the nearest
  100, set 
\begin{verbatim}
        SCHEDD_ROUND_ATTR_foo = 2
\end{verbatim}
  When the schedd rounds up an attribute value, it will save the raw 
  (un-rounded) actual value in an attribute with the same name appended
  with ``\_RAW".  So in the above example, the raw value will be stored
  in attribute \Attr{foo\_RAW} in the job ClassAd.
  The following are set by default:
\begin{verbatim}
        SCHEDD_ROUND_ATTR_ImageSize = 25%
        SCHEDD_ROUND_ATTR_ExecutableSize = 25%
        SCHEDD_ROUND_ATTR_DiskUsage = 25%
        SCHEDD_ROUND_ATTR_NumCkpts = 4
\end{verbatim}
  Thus, an ImageSize near 100MB will be rounded up to the next
  multiple of 25MB.  If your batch slots have less
  memory or disk than the rounded values, it may be necessary to
  reduce the amount of rounding, because the job requirements
  will not be met.

\item[\Macro{SCHEDD\_BACKUP\_SPOOL}]
  \label{param:ScheddBackupSpool} This macro is used to enable the
  \Condor{schedd} to make a backup of the job queue as it starts.  If
  set to ``True'', the \Condor{schedd} will create host specific a
  backup of the current spool file to the spool directory.  This
  backup file will be overwritten each time the \Condor{schedd}
  starts.  \Macro{SCHEDD\_BACKUP\_SPOOL} defaults to ``False''.

\item[\Macro{MPI\_CONDOR\_RSH\_PATH}]
  \label{param:MPICondorRshPath} The complete path to the
  special version of \Prog{rsh} that is required to spawn MPI
  jobs under Condor.
  \MacroU{LIBEXEC} is the proper value for this 
  configuration variable, required when running MPI dedicated jobs.

\item[\Macro{SCHEDD\_PREEMPTION\_REQUIREMENTS}]
  \label{param:ScheddPreemptionRequirements}
  This boolean expression is
  utilized only for machines allocated by a dedicated scheduler.
  When \Expr{True}, a machine becomes a candidate for job preemption.
  This configuration variable has no default;
  when not defined, preemption will never be considered.

\item[\Macro{SCHEDD\_PREEMPTION\_RANK}]
  \label{param:ScheddPreemptionRank}
  This floating point value is
  utilized only for machines allocated by a dedicated scheduler.
  It is evaluated in context of a job ClassAd,
  and it represents a machine's preference for running a job.
  This configuration variable has no default;
  when not defined, preemption will never be considered.

\item[\Macro{ParallelSchedulingGroup}]
  \label{param:ParallelSchedulingGroup}
  For parallel jobs which must be assigned within a group
  of machines (and not cross group boundaries),
  this configuration variable identifies members of a group. 
  Each machine within a group sets this configuration variable with 
  a string that identifies the group.

\item[\Macro{PER\_JOB\_HISTORY\_DIR}]
  \label{param:PerJobHistoryDir}
  If set to a directory writable by the Condor user, when a job
  leaves the \Condor{schedd}'s queue, a copy of its ClassAd will
  be written in that directory.  The files are named ``history.''
  with the job's cluster and process number appended.  For
  example, job 35.2 will result in a file named ``history.35.2''.
  Condor does not rotate or delete the files, so without an
  external entity to clean the directory it can grow very large.
  This option defaults to being unset.  When not set, no such
  files are written.

\item[\Macro{DEDICATED\_SCHEDULER\_USE\_FIFO}]
  \label{param:DedicatedSchedulerUseFifo}
  When this parameter is set to true (the default), parallel and mpi
  universe jobs will be scheduled in a first-in, first-out manner.
  When set to false, parallel and mpi jobs are scheduled using a
  best-fit algorithm. Using the best-fit algorithm is not recommended,
  as it can cause starvation.

\end{description}

%%%%%%%%%%%%%%%%%%%%%%%%%%%%%%%%%%%%%%%%%%%%%%%%%%%%%%%%%%%%%%%%%%%%%%%%%%%
\subsection{\label{sec:Shadow-Config-File-Entries}\condor{shadow}
Configuration File Entries}
%%%%%%%%%%%%%%%%%%%%%%%%%%%%%%%%%%%%%%%%%%%%%%%%%%%%%%%%%%%%%%%%%%%%%%%%%%%

\index{configuration!condor\_shadow configuration variables}
These settings affect the \Condor{shadow}.
\begin{description}

\item[\Macro{SHADOW\_LOCK}] \label{param:ShadowLock} This macro
  specifies the lock file to be used for access to the
  \File{ShadowLog} file.  It must be a separate file from the
  \File{ShadowLog}, since the \File{ShadowLog} may be rotated 
  and you want to synchronize access across log file rotations.
  This macro is defined relative to the \MacroUNI{LOCK} macro.

\item[\Macro{SHADOW\_DEBUG}] \label{param:ShadowDebug} This macro
  (and other settings related to debug logging in the shadow) is
  described in section~\ref{param:SubsysDebug} as
  \MacroNI{<SUBSYS>\_DEBUG}.

\item[\Macro{SHADOW\_QUEUE\_UPDATE\_INTERVAL}]
\label{param:ShadowQueueUpdateInterval}
The amount of time (in seconds) between ClassAd updates that the
\Condor{shadow} daemon sends to the \Condor{schedd} daemon.
Defaults to 900 (15 minutes).

\item[\Macro{SHADOW\_LAZY\_QUEUE\_UPDATE}]
\label{param:ShadowLazyQueueUpdate}
  This boolean macro specifies if the \Condor{shadow} should
  immediately update the job queue for certain attributes (at this
  time, it only effects the \AdAttr{NumJobStarts} and
  \AdAttr{NumJobReconnects} counters) or if it should wait and only
  update the job queue on the next periodic update.
  There is a trade-off between performance and the semantics of these
  attributes, which is why the behavior is controlled by a
  configuration macro.
  If the \Condor{shadow} do not use a lazy update, and immediately
  ensures the changes to the job attributes are written to the job
  queue on disk, the semantics for the attributes are very solid
  (there's only a tiny chance that the counters will be out of sync
  with reality), but this introduces a potentially large performance
  and scalability problem for a busy \Condor{schedd}.
  If the \Condor{shadow} uses a lazy update, there's no additional cost
  to the \Condor{schedd}, but it means that \Condor{q} and Quill won't
  immediately see the changes to the job attributes, and if the
  \Condor{shadow} happens to crash or be killed during that time, the
  attributes are never incremented.
  Given that the most obvious usage of these counter attributes is for
  the periodic user policy expressions (which are evaluated directly
  by the \Condor{shadow} using its own copy of the job's classified
  ad, which is immediately updated in either case), and since the
  additional cost for aggressive updates to a busy \Condor{schedd}
  could potentially cause major problems, the default is \Expr{True}
  to do lazy, periodic updates.

\item[\Macro{COMPRESS\_PERIODIC\_CKPT}]
  \label{param:CompressPeriodicCkpt} This boolean macro specifies
  whether the shadow should instruct applications to compress periodic
  checkpoints (when possible).  The default is \Expr{False}.

\item[\Macro{COMPRESS\_VACATE\_CKPT}]
  \label{param:CompressVacateCkpt} This boolean macro specifies
  whether the shadow should instruct applications to compress vacate
  checkpoints (when possible).  The default is \Expr{False}.

\item[\Macro{PERIODIC\_MEMORY\_SYNC}] \label{param:PeriodicMemorySync}
  This boolean macro specifies whether the shadow should instruct
  applications to commit dirty memory pages to swap space during a
  periodic checkpoint.  The default is \Expr{False}.  This potentially
  reduces the number of dirty memory pages at vacate time, thereby
  reducing swapping activity on the remote machine.

\item[\Macro{SLOW\_CKPT\_SPEED}] \label{param:SlowCkptSpeed}  This
  macro specifies the speed at which vacate checkpoints should be
  written, in kilobytes per second.  If zero (the default), vacate
  checkpoints are written as fast as possible.  Writing vacate
  checkpoints slowly can avoid overwhelming the remote machine with
  swapping activity.

\item[\Macro{SHADOW\_JOB\_CLEANUP\_RETRY\_DELAY}]
 \label{param:ShadowJobCleanupRetryDelay}
 This is an integer specifying the number of seconds to wait between tries
 to commit the final update to the job ClassAd in the \Condor{schedd}'s
 job queue.  The default is 30.

\item[\Macro{SHADOW\_MAX\_JOB\_CLEANUP\_RETRIES}]
 \label{param:ShadowMaxJobCleanupRetries}
 This is an integer specifying the number of times to try committing
 the final update to the job ClassAd in the \Condor{schedd}'s
 job queue.  The default is 5.

\end{description}

%%%%%%%%%%%%%%%%%%%%%%%%%%%%%%%%%%%%%%%%%%%%%%%%%%%%%%%%%%%%%%%%%%%%%%%%%%%
\subsection{\label{sec:Starter-Config-File-Entries}\condor{starter}
Configuration File Entries}
%%%%%%%%%%%%%%%%%%%%%%%%%%%%%%%%%%%%%%%%%%%%%%%%%%%%%%%%%%%%%%%%%%%%%%%%%%%

\index{configuration!condor\_starter configuration variables}
These settings affect the \Condor{starter}.
\begin{description}

\item[\Macro{EXEC\_TRANSFER\_ATTEMPTS}] \label{param:ExecTransferAttempts}
  Sometimes due to a router misconfiguration, kernel bug, or other Act
  of God network problem, the transfer of the initial checkpoint from
  the submit machine to the execute machine will fail midway through.
  This parameter allows a retry of the transfer a certain number of times
  that must be equal to or greater than 1. If this parameter is not
  specified, or specified incorrectly, then it will default to three.
  If the transfer of the initial executable fails every attempt, then
  the job goes back into the idle state until the next renegotiation
  cycle.

  \Note: This parameter does not exist in the NT starter.

\item[\Macro{JOB\_RENICE\_INCREMENT}] \label{param:JobReniceIncrement}
  When the \Condor{starter} spawns a Condor job, it can do so with a
  \Term{nice-level}.
  A nice-level is a
  Unix mechanism that allows users to assign their own processes a lower 
  priority, such that these processes do not interfere with interactive
  use of the machine.
  For machines with lots
  of real memory and swap space, such that the only scarce resource is CPU time,
  use this macro in conjunction with a policy that
  allows Condor to always start jobs on the machines. 
  Condor jobs would always run,
  but interactive response on the machines would never suffer.
  A user most likely will not notice Condor is
  running jobs.  See section~\ref{sec:Configuring-Policy} on
  Startd Policy Configuration for more details on setting up a
  policy for starting and stopping jobs on a given machine.

  The integer value is
  set by the \Condor{starter} daemon for each job just before the
  job runs.
  The range of allowable values are integers in the range of 0 to 19
  (inclusive),
  with a value of 19 being the lowest priority.  
  If the integer value is outside this range,
  then on a Unix machine, a value greater than 19 is auto-decreased to 19;
  a value less than 0 is treated as 0.
  For values outside this range, a Windows machine ignores the value
  and uses the default instead.
  The default value is 10, which maps to the idle priority class on
  a Windows machine.

\item[\Macro{STARTER\_LOCAL\_LOGGING}]
  \label{param:StarterLocalLogging} This macro determines whether the
  starter should do local logging to its own log file, or send debug
  information back to the \Condor{shadow} where it will end up in the
  ShadowLog.  It defaults to \Expr{True}.

\item[\Macro{STARTER\_DEBUG}] \label{param:StarterDebug} This setting
  (and other settings related to debug logging in the starter) is
  described above in section~\ref{param:SubsysDebug} as
  \MacroUNI{<SUBSYS>\_DEBUG}.

\item[\Macro{STARTER\_UPDATE\_INTERVAL}] \label{param:StarterUpdateInterval}
The amount of time (in seconds) between ClassAd updates that the
\Condor{starter} daemon sends to the \Condor{shadow} and \Condor{startd}
daemons.  Defaults to 300 (5 minutes).

\item[\Macro{USER\_JOB\_WRAPPER}] \label{param:UserJobWrapper} 
  The full path to an executable or script.
  This macro
  allows an administrator to specify a wrapper script to handle the
  execution of all user jobs.  
  If specified, Condor never directly executes a job, but instead
  invokes the program specified by this macro.
  The command-line arguments passed to this program will include the
  full-path to the actual user job which should be executed, followed by all
  the command-line parameters to pass to the user job.
  This wrapper program must ultimately replace its image with the user job;
  in other words,
  it must \Procedure{exec} the user job, not \Procedure{fork} it.
  For instance, if the wrapper program is a C/Korn shell script, the
  last line of execution should be:
\begin{verbatim}
        exec $*
\end{verbatim}
  This can potentially lose information about the arguments.
  Any argument with embedded whitespace will be split into multiple
  arguments.
  For example the argument "argument one" will become the two arguments
  "argument" and "one".
  For Bourne type shells (sh, bash, ksh),
  the following preserves the arguments:
\begin{verbatim}
        exec "$@"
\end{verbatim}
  For the C type shells (csh, tcsh), the following preserves the
  arguments:
\begin{verbatim}
        exec $*:q
\end{verbatim}


  For Windows machines, the wrapper will either be
  a batch script (with a file extension of \File{.bat} or \File{.cmd})
  or an executable (with a file extension of \File{.exe} or \File{.com}).

\item[\Macro{USE\_VISIBLE\_DESKTOP}] \label{param:UseVisibleDesktop} 
  This setting is only meaningful on Windows machines.  If True, Condor will
  allow the job to create windows on the desktop of the execute machine and
  interact with the job.  This is particularly useful for debugging why an
  application will not run under Condor.  If False, Condor uses the default
  behavior of creating a new, non-visible desktop to run the job on.
  See section~\ref{sec:platform-windows} for details on how Condor 
  interacts with the desktop.

\item[\Macro{STARTER\_JOB\_ENVIRONMENT}] \label{param:StarterJobEnvironment}
  This macro sets the default environment inherited by jobs.  The syntax is
  the same as the syntax for environment settings in the job submit file
  (see page~\pageref{man-condor-submit-environment}).
  If the same environment variable is assigned by this macro and by the user
  in the submit file, the user's setting takes precedence.

\item[\Macro{JOB\_INHERITS\_STARTER\_ENVIRONMENT}]
\label{param:JobInheritsStarterEnvironment} 
A boolean value that defaults to \Expr{False}.
When \Expr{True},
it causes jobs to inherit all environment variables from 
the \Condor{starter}.
This is useful for glidein jobs that need to
access environment variables from the batch system running the glidein
daemons.
When both the user job and \MacroNI{STARTER\_JOB\_ENVIRONMENT} define
an environment variable that is in the \Condor{starter}'s
environment, the user job's definition takes precedence.
This variable does not apply to standard universe jobs.

\item[\Macro{STARTER\_UPLOAD\_TIMEOUT}]
\label{param:StarterUploadTimeout} 
An integer value that specifies the network communication timeout to use
when transferring files back to the submit machine.  The default value is
200.  You may need to increase this value if the disk on the submit machine
cannot keep up with large bursts of activity, such as many jobs all
completing at the same time.

\end{description}

%%%%%%%%%%%%%%%%%%%%%%%%%%%%%%%%%%%%%%%%%%%%%%%%%%%%%%%%%%%%%%%%%%%%%%%%%%%
\subsection{\label{sec:Submit-Config-File-Entries}\condor{submit}
Configuration File Entries}
%%%%%%%%%%%%%%%%%%%%%%%%%%%%%%%%%%%%%%%%%%%%%%%%%%%%%%%%%%%%%%%%%%%%%%%%%%%
\index{configuration!condor\_submit configuration variables}
\begin{description}
\item[\Macro{DEFAULT\_UNIVERSE}] \label{param:DefaultUniverse}
The universe under which a job is executed may be specified in the submit
description file.
If it is not specified in the submit description file, then
this variable specifies the universe (when defined).
If the universe is not specified in the submit description
file, and if this variable is not defined, then
the default universe for a job will be the standard universe.
\end{description}

If you want \Condor{submit} to automatically append an expression to
the \AdAttr{Requirements} expression or \AdAttr{Rank} expression of 
jobs at your site use the following macros:
\begin{description}
  
\item[\Macro{APPEND\_REQ\_VANILLA}] \label{param:AppendReqVanilla}
  Expression to be appended to vanilla job requirements.
  
\item[\Macro{APPEND\_REQ\_STANDARD}] \label{param:AppendReqStandard}
  Expression to be appended to standard job requirements.

\item[\Macro{APPEND\_REQUIREMENTS}] \label{param:AppendReq}
  Expression to be appended to any type of universe jobs. 
  However, if \MacroNI{APPEND\_REQ\_VANILLA} or \MacroNI{APPEND\_REQ\_STANDARD}
  is defined, then ignore the \MacroNI{APPEND\_REQUIREMENTS} for those
  universes.

\item[\Macro{APPEND\_RANK}] \label{param:AppendRank}
  Expression to be appended to job rank.  \MacroNI{APPEND\_RANK\_STANDARD} or
    \MacroNI{APPEND\_RANK\_VANILLA} will override this setting if defined.

\item[\Macro{APPEND\_RANK\_STANDARD}] \label{param:AppendRankStandard}
  Expression to be appended to standard job rank.

\item[\Macro{APPEND\_RANK\_VANILLA}] \label{param:AppendRankVanilla}
  Expression to append to vanilla job rank.

\end{description}

\Note The \Macro{APPEND\_RANK\_STANDARD} and 
\Macro{APPEND\_RANK\_VANILLA} macros were called
\Macro{APPEND\_PREF\_STANDARD} and
\Macro{APPEND\_PREF\_VANILLA} in previous versions of Condor.

In addition, you may provide default \AdAttr{Rank} expressions if your users
do not specify their own with:

\begin{description}

\item[\Macro{DEFAULT\_RANK}] \label{param:DefaultRank}
  Default rank expression for any job that does not specify
  its own rank expression in the submit description file.  
  There is no default value, such that when undefined,
  the value used will be 0.0.

\item[\Macro{DEFAULT\_RANK\_VANILLA}] \label{param:DefaultRankVanilla}
  Default rank for vanilla universe jobs.  
  There is no default value, such that when undefined,
  the value used will be 0.0.
  When both \MacroNI{DEFAULT\_RANK} and \MacroNI{DEFAULT\_RANK\_VANILLA}
  are defined, the value for \MacroNI{DEFAULT\_RANK\_VANILLA} is
  used for vanilla universe jobs.

\item[\Macro{DEFAULT\_RANK\_STANDARD}] \label{param:DefaultRankStandard}
  Default rank for standard universe jobs.
  There is no default value, such that when undefined,
  the value used will be 0.0.
  When both \MacroNI{DEFAULT\_RANK} and \MacroNI{DEFAULT\_RANK\_STANDARD}
  are defined, the value for \MacroNI{DEFAULT\_RANK\_STANDARD} is
  used for standard universe jobs.

\item[\Macro{DEFAULT\_IO\_BUFFER\_SIZE}] \label{param:DefaultBufferSize}
  Condor keeps a buffer of recently-used data for each file an
  application opens.  This macro specifies the default maximum number
  of bytes to be buffered for each open file at the executing machine.
  The \Condor{status} \MacroNI{buffer\_size} command will override this
  default.  If this macro is undefined, a default size of 512 KB will
  be used.

\item[\Macro{DEFAULT\_IO\_BUFFER\_BLOCK\_SIZE}] 
  \label{param:DefaultBufferBlockSize} When buffering is enabled,
  Condor will attempt to consolidate small read and write operations
  into large blocks.  This macro specifies the default block size
  Condor will use.  The \Condor{status} \MacroNI{buffer\_block\_size}
  command will override this default.  If this macro is undefined, a
  default size of 32 KB will be used.

\item[\Macro{SUBMIT\_SKIP\_FILECHECK}] \label{param:SubmitSkipFilecheck}
  If True, \Condor{submit} behaves as if the \Opt{-d} 
  command-line option is used.
  This tells \Condor{submit} to disable file permission checks when
  submitting a job.
  This can significantly decrease the amount of time required to submit
  a large group of jobs.
  The default value is False.

\item[\Macro{WARN\_ON\_UNUSED\_SUBMIT\_FILE\_MACROS}]
  \label{param:WarnOnUnusedSubmitFileMacros}  A boolean variable that
  defaults to \Expr{True}.  When \Expr{True}, \Condor{submit}
  performs checks on the job's submit description file contents
  for commands that define a macro, but do not use the macro within
  the file.
  A warning is issued, but job submission continues.
  A definition of a new macro occurs when the lhs of a command is not
  a known submit command.  This check may help spot spelling errors
  of known submit commands.

\item[\Macro{SUBMIT\_SEND\_RESCHEDULE}] \label{param:SubmitSendReschedule}
  A boolean expression that when False, prevents \Condor{submit} from
  automatically sending a \Condor{reschedule} command as it completes.
  The \Condor{reschedule} command causes the \Condor{schedd} daemon
  to start searching for machines with which to match the submitted
  jobs.  When True, this step always occurs.
  In the case that the machine where the job(s) are submitted is
  managing a huge number of jobs (thousands or tens of thousands),
  this step would hurt performance in such a way that it became
  an obstacle to scalability.
  The default value is True.

\item[\Macro{SUBMIT\_EXPRS}] \label{param:SubmitExprs}
  The given comma-separated, named expressions are inserted into all 
  the job ClassAds that \Condor{submit} creates.  This is equivalent
  to the ``+'' syntax in submit files.  See the
  the \Condor{submit} manual page
  on page~\pageref{man-condor-submit} for details on using the ``+''
  syntax to add attributes to the job ClassAd.
  Attributes defined in the submit description file with ``+'' will
  override attributes defined in the config file with
  \MacroNI{SUBMIT\_EXPRS}. 

\item[\Macro{LOG\_ON\_NFS\_IS\_ERROR}] \label{param:LogOnNfsIsError}
  A boolean value that controls whether \Condor{submit} prohibits
  job submit files with user log files on NFS.  If
  \MacroNI{LOG\_ON\_NFS\_IS\_ERROR} is set to \Expr{True}, such
  submit files will be rejected.  If \MacroNI{LOG\_ON\_NFS\_IS\_ERROR}
  is set to \Expr{False},
  the job will be submitted.  If not defined,
  \MacroNI{LOG\_ON\_NFS\_IS\_ERROR} defaults to \Expr{False}.

\item[\Macro{SUBMIT\_MAX\_PROCS\_IN\_CLUSTER}]
  \label{param:SubmitMaxProcsInCluster}
  An integer value that limits the maximum number of jobs that would
  be assigned within a single cluster.  Job submissions that would exceed
  the defined value fail, issuing an error message, and with no jobs
  submitted.
  The default value is 0, which does not limit the number of jobs
  assigned a single cluster number.

\end{description}

%%%%%%%%%%%%%%%%%%%%%%%%%%%%%%%%%%%%%%%%%%%%%%%%%%%%%%%%%%%%%%%%%%%%%%%%%%%
\subsection{\label{sec:Preen-Config-File-Entries}\condor{preen}
Configuration File Entries}
%%%%%%%%%%%%%%%%%%%%%%%%%%%%%%%%%%%%%%%%%%%%%%%%%%%%%%%%%%%%%%%%%%%%%%%%%%%

\index{configuration!condor\_preen configuration variables}
These macros affect \Condor{preen}.

\begin{description}

\item[\Macro{PREEN\_ADMIN}] \label{param:PreenAdmin}  This macro
  sets the e-mail address where \Condor{preen} will send e-mail (if
  it is configured to send email at all... see the entry for
  \MacroNI{PREEN}).  Defaults to \MacroUNI{CONDOR\_ADMIN}.

\item[\Macro{VALID\_SPOOL\_FILES}] \label{param:ValidSpoolFiles}  This
  macro contains a (comma or space separated) list of files that
  \Condor{preen} considers valid files to find in the \MacroUNI{SPOOL}
  directory. There is no default value. \Condor{preen} will add to the
  list files and directories that are normally present in the
  \MacroUNI{SPOOL} directory.
  
\item[\Macro{INVALID\_LOG\_FILES}] \label{param:InvalidLogFiles} This
  macro contains a (comma or space separated) list of files that
  \Condor{preen} considers invalid files to find in the \MacroUNI{LOG}
  directory.  There is no default value.

\end{description}


%%%%%%%%%%%%%%%%%%%%%%%%%%%%%%%%%%%%%%%%%%%%%%%%%%%%%%%%%%%%%%%%%%%%%%%%%%%
\subsection{\label{sec:Collector-Config-File-Entries}\condor{collector}
Configuration File Entries}
%%%%%%%%%%%%%%%%%%%%%%%%%%%%%%%%%%%%%%%%%%%%%%%%%%%%%%%%%%%%%%%%%%%%%%%%%%%

\index{configuration!condor\_collector configuration variables}
These macros affect the \Condor{collector}.
\begin{description}
  
\item[\Macro{CLASSAD\_LIFETIME}] \label{param:ClassadLifetime} This
  macro determines the default maximum age for ClassAds collected by the
  \Condor{collector}.  ClassAd older than the maximum age are
  discarded by the \Condor{collector} as stale.

  If present, the ClassAd attribute ``ClassAdLifetime'' specifies the
  ad's lifetime in seconds.  If ``ClassAdLifetime'' is not present in
  the ad, the \Condor{collector} will use the value of
  \MacroUNI{CLASSAD\_LIFETIME}.  The macro is defined in terms of
  seconds, and defaults to 900 (15 minutes).
  
\item[\Macro{MASTER\_CHECK\_INTERVAL}]
  \label{param:MasterCheckInterval}  This macro defines how often the
  collector should check for machines that have ClassAds from some
  daemons, but not from the \Condor{master} (\Term{orphaned daemons})
  and send e-mail about it.  It is defined in seconds and 
  defaults to 10800 (3 hours).

\item[\Macro{COLLECTOR\_REQUIREMENTS}]
  \label{param:CollectorRequirements} A boolean expression
  that filters out unwanted ClassAd updates.  The
  expression is evaluated for ClassAd updates that have 
  passed through enabled security authorization checks.
  The default behavior when this expression is not
  defined is to allow all ClassAd updates to take place.
  If \Expr{False}, a ClassAd update will be rejected.

  Stronger security mechanisms are the better way to
  authorize or deny updates to the \Condor{collector}.
  This configuration variable exists to help those that
  use host-based security, and
  do not trust all processes that run on the hosts in the pool.
  This configuration variable may be used to throw out ClassAds that
  should not be allowed.  For example, for
  \Condor{startd} daemons that run on a fixed port,
  configure this expression to ensure that 
  only machine ClassAds advertising the expected
  fixed port are accepted.  As a convenience, before evaluating the
  expression, some basic sanity checks are performed on the ClassAd to
  ensure that all of the ClassAd attributes used by Condor to contain
  IP:port information are consistent.  To validate this
  information, the attribute to check is \AdAttr{TARGET.MyAddress}.
 

\item[\Macro{CLIENT\_TIMEOUT}] \label{param:ClientTimeout} Network
  timeout that the \Condor{collector} uses when talking to any daemons
  or tools that are sending it a ClassAd update.
  It is defined in seconds and defaults to 30.
  
\item[\Macro{QUERY\_TIMEOUT}] \label{param:QueryTimeout} Network
  timeout when talking to anyone doing a query. It is defined in seconds
  and defaults to 60.
  
\item[\Macro{CONDOR\_DEVELOPERS}] \label{param:CondorDevelopers}
  By default,
  Condor will send e-mail once per week to this address with the output
  of the \Condor{status} command, which lists how many machines
  are in the pool and how many are running jobs.  The default
  value of \Email{condor-admin@cs.wisc.edu} will send this report to
  the Condor Team developers at the University of Wisconsin-Madison.
  The Condor Team uses
  these weekly status messages in order to have some idea as to how
  many Condor pools exist in the world.  We appreciate
  getting the reports, as this is one way we can convince funding
  agencies that Condor is being used in the real world.  
  If you do not wish this information to be sent to the Condor Team,
  explicitly set the value to \Expr{NONE} to disable this feature,
  or replace the
  address with a desired location.  
  If undefined (commented out) in the configuration file, Condor follows
  its default behavior.

\item[\Macro{COLLECTOR\_NAME}] \label{param:CollectorName}
  This macro is used to specify a short description of your pool.
  It should be about 20 characters long.  For example, the name of the
  UW-Madison Computer Science Condor Pool is ``UW-Madison CS''.  
  While this macro might seem similar to \MacroNI{MASTER\_NAME} or
  \MacroNI{SCHEDD\_NAME}, it is totally unrelated.
  Those settings are used to unique identify (and locate) a specific
  set of Condor daemons if there are more than one running on the same
  machine.
  The \MacroNI{COLLECTOR\_NAME} setting is just used as a
  human-readable string to describe the pool, which is included in the
  updates set to the \MacroNI{CONDOR\_DEVELOPERS\_COLLECTOR} (see
  below). 

\item[\Macro{CONDOR\_DEVELOPERS\_COLLECTOR}]
  \label{param:CondorDevelopersCollector} By default, every pool sends
  periodic updates to a central \Condor{collector} at UW-Madison with
  basic information about the status of your pool.  This includes only
  the number of total machines, the number of jobs submitted, the
  number of machines running jobs, the host name of your central
  manager, and the \MacroUNI{COLLECTOR\_NAME} specified above.  These
  updates help the Condor Team see how Condor is being used around the world.
  By default, they will be sent to \File{condor.cs.wisc.edu}.
  If you do not want
  these updates to be sent from your pool,
  explicitly set this macro to \Expr{NONE}. 
  If undefined (commented out) in the configuration file, Condor follows
  its default behavior.

\item[\Macro{COLLECTOR\_SOCKET\_BUFSIZE}] 
  \label{param:CollectorSocketBufsize} This specifies the buffer size, in
  bytes, reserved for \Condor{collector} network UDP sockets.  The default is
  10240000, or a ten megabyte buffer.  This is a healthy size, even for a large
  pool.  The larger this value, the less likely the \Condor{collector} will
  have stale information about the pool due to dropping update packets.  If
  your pool is small or your central manager has very little RAM, considering
  setting this parameter to a lower value (perhaps 256000 or 128000).

\Note For some Linux distributions, it may be necessary to configure
a larger value than the default; this parameter is
/proc/sys/net/core/rmem\_max .  You can see the values that the
\Condor{collector} actually used by enabling D\_FULLDEBUG for the
collector and looking at the log line that looks like this:

Reset OS socket buffer size to 2048k (UDP), 255k (TCP).

\item[\Macro{COLLECTOR\_TCP\_SOCKET\_BUFSIZE}]

  \label{param:CollectorTcpSocketBufsize} This specifies the TCP buffer
  size, in  bytes, reserved for \Condor{collector} network sockets.  The
  default is 131072, or a 128 kilobyte buffer.  This is a healthy size, even
  for a large pool.  The larger this value, the less likely the
  \Condor{collector} will have stale information about the pool due to
  dropping update packets.  If your pool is small or your central
  manager has very little RAM, considering setting this parameter to a
  lower value (perhaps 65536 or 32768).

\Note See the note for \Macro{COLLECTOR\_SOCKET\_BUFSIZE}.

\item[\Macro{COLLECTOR\_SOCKET\_CACHE\_SIZE}] 
  \label{param:CollectorSocketCacheSize} 
  If your site wants to use TCP connections to send ClassAd updates to
  the collector, you must use this setting to enable a cache of TCP
  sockets (in addition to enabling
  \Macro{UPDATE\_COLLECTOR\_WITH\_TCP}). 
  Please read section~\ref{sec:tcp-collector-update} on ``Using TCP to
  Send Collector Updates'' on page~\pageref{sec:tcp-collector-update}
  for more details and a discussion of when you would need this
  functionality. 
  If you do not enable a socket cache, TCP updates will be refused by
  the collector.
  The default value for this setting is 0, with no cache enabled.   
  If you lower this number, you must run \Condor{restart} and not just
  \Condor{reconfig} for the change to take effect.

\item[\Macro{KEEP\_POOL\_HISTORY}] \label{param:KeepPoolHistory}
  This boolean macro is used to decide if the collector will write
  out statistical information about the pool to history files. The default
  is \Expr{False}. The location, size and frequency of history logging is controlled
  by the other macros.

\item[\Macro{POOL\_HISTORY\_DIR}] \label{param:PoolHistoryDir}
  This macro sets the name of the directory where the history
  files reside (if history logging is enabled).
  The default is the \File{SPOOL} directory.

\item[\Macro{POOL\_HISTORY\_MAX\_STORAGE}]
  \label{param:PoolHistoryMaxStorage} 
  This macro sets the maximum combined size of the history files.
  When the size of the history files is close to this limit, the oldest
  information will be discarded.
  Thus, the larger this parameter's value is, the larger the time
  range for which history will be available.  The default value is
  10000000 (10 Mbytes).

\item[\Macro{POOL\_HISTORY\_SAMPLING\_INTERVAL}]
  \label{param:PoolHistorySamplingInterval}
  This macro sets the interval, in seconds, between samples for
  history logging purposes. 
  When a sample is taken, the collector goes through the information
  it holds, and summarizes it.
  The information is written to the history file once for each 4
  samples.
  The default (and recommended) value is 60 seconds. Setting this
  macro's value too low will increase the load on the collector,
  while setting it to high will produce less precise statistical
  information.

\item[\Macro{COLLECTOR\_DAEMON\_STATS}]
  \label{param:CollectorDaemonStats}
  This macro controls whether or not the Collector keeps update
  statistics on incoming updates.  The default value is True.  If
  this option is enabled, the collector will insert several attributes
  into ClassAds that it stores and sends.  ClassAds without the
  ``UpdateSequenceNumber'' and ``DaemonStartTime'' attributes will not
  be counted, and will not have attributes inserted (all modern Condor
  daemons which publish ClassAds publish these attributes).

  The attributes inserted are ``UpdatesTotal'', ``UpdatesSequenced'',
  and ``UpdatesLost''.  ``UpdatesTotal'' is the total number of
  updates (of this ad type) the Collector has received from this host.
  ``UpdatesSequenced'' is the number of updates that the Collector
  could have as lost.  In particular, for the first update from a
  daemon it is impossible to tell if any previous ones have been lost
  or not.  ``UpdatesLost'' is the number of updates that the Collector
  has detected as being lost.

\item[\Macro{COLLECTOR\_STATS\_SWEEP}]
  \label{param:CollectorStatsSweep} This value specifies the number of
  seconds between sweeps of the \Condor{collector}'s per-daemon update
  statistics.  Records for daemons which have not reported in this amount
  of time are purged in order to save memory.  The default is two days.
  It is unlikely that you would ever need to adjust this.

\item[\Macro{COLLECTOR\_DAEMON\_HISTORY\_SIZE}]
  \label{param:CollectorDaemonHistorySize} This macro controls the
  size of the published update history that the Collector inserts into
  the ClassAds it stores and sends.  The default value is 128, which
  means that history is stored and published for the latest 128
  updates.  This macro is ignored if \MacroU{COLLECTOR\_DAEMON\_STATS}
  is not enabled.

  If this has a non-zero value, the Collector will insert
  ``UpdatesHistory'' into the ClassAd (similar to ``UpdatesTotal''
  above).  ``UpdatesHistory'' is a hexadecimal string which represents
  a bitmap of the last \Macro{COLLECTOR\_DAEMON\_HISTORY\_SIZE}
  updates.  The most significant bit (MSB) of the bitmap represents the
  most recent update, and the least significant bit (LSB) represents
  the least recent.  A value of zero means that the update was not
  lost, and a value of 1 indicates that the update was detected as
  lost.

  For example, if the last update was not lost, the previous lost, and
  the previous two not, the bitmap would be 0100, and the matching hex
  digit would be ``4''.  Note that the MSB can never be marked as lost
  because its loss can only be detected by a non-lost update (a
  ``gap'' is found in the sequence numbers).  Thus, UpdatesHistory =
  "0x40" would be the history for the last 8 updates.  If the next
  updates are all successful, the values published, after each update,
  would be: 0x20, 0x10, 0x08, 0x04, 0x02, 0x01, 0x00.

\item[\Macro{COLLECTOR\_CLASS\_HISTORY\_SIZE}]
  \label{param:CollectorClassHistorySize} This macro controls the
  size of the published update history that the Collector inserts into
  the Collector ClassAds it produces.  The default value is zero.

  If this has a non-zero value, the Collector will insert
  ``UpdatesClassHistory'' into the Collector ClassAd (similar to
  ``UpdatesHistory'' above).  These are added ``per class'' of
  ClassAd, however.  The classes refer to the ``type'' of ClassAds
  (i.e. ``Start'').  Additionally, there is a ``Total'' class created
  which represents the history of all ClassAds that this Collector
  receives.

  Note that the collector always publishes Lost, Total and Sequenced
  counts for all ClassAd ``classes''.  This is similar to the
  statistics gathered if \MacroU{COLLECTOR\_DAEMON\_STATS} is enabled.

 \item[\Macro{COLLECTOR\_QUERY\_WORKERS}]
  \label{param:CollectorQueryWorkers} This macro sets the maximum
  number of ``worker'' processes that the Collector can have.  When
  receiving a query request, the UNIX Collector will ``fork'' a new
  process to handle the query, freeing the main process to handle
  other requests.  When the number of outstanding ``worker'' processes
  reaches this maximum, the request is handled by the main process.
  This macro is ignored on Windows, and its default value is zero.
  The default configuration, however, has this set to 16.

\item[\Macro{COLLECTOR\_DEBUG}] \label{param:CollectorDebug} This
  macro (and other macros related to debug logging in the collector)
  is described in section~\ref{param:SubsysDebug} as
  \MacroNI{<SUBSYS>\_DEBUG}.

\end{description}

%%%%%%%%%%%%%%%%%%%%%%%%%%%%%%%%%%%%%%%%%%%%%%%%%%%%%%%%%%%%%%%%%%%%%%%%%%%
\subsection{\label{sec:Negotiator-Config-File-Entries}\condor{negotiator}
Configuration File Entries}
%%%%%%%%%%%%%%%%%%%%%%%%%%%%%%%%%%%%%%%%%%%%%%%%%%%%%%%%%%%%%%%%%%%%%%%%%%%
\index{configuration!condor\_negotiator configuration variables}

These macros affect the \Condor{negotiator}.
\begin{description}
  
\item[\Macro{NEGOTIATOR\_INTERVAL}] \label{param:NegotiatorInterval}
  Sets how often the negotiator starts a negotiation cycle.  It is defined
  in seconds and defaults to 300 (5 minutes).
  
\item[\Macro{NEGOTIATOR\_CYCLE\_DELAY}] \label{param:NegotiatorCycleDelay}
  An integer value that represents the minimum number of seconds
  that must pass before a new negotiation cycle may start.
  The default value is 20.
  \MacroNI{NEGOTIATOR\_CYCLE\_DELAY} is intended only for use by
  Condor experts.

\item[\Macro{NEGOTIATOR\_TIMEOUT}] \label{param:NegotiatorTimeout}
  Sets the timeout that the negotiator uses on its network connections
  to the \Condor{schedd} and \Condor{startd}s.  It is defined in seconds and defaults to 30.
  
\item[\Macro{PRIORITY\_HALFLIFE}] \label{param:PriorityHalfLife} This
  macro defines the half-life of the user priorities.  See
  section~\ref{sec:user-priority-explained}
  on User Priorities for details.  It is defined in seconds and defaults
  to 86400 (1 day).

\item[\Macro{DEFAULT\_PRIO\_FACTOR}] \label{param:DefaultPrioFactor} 
  This macro sets the priority factor for local users. See
  section~\ref{sec:user-priority-explained}
  on User Priorities for details.  Defaults to 1.

\item[\Macro{NICE\_USER\_PRIO\_FACTOR}] \label{param:NiceUserPrioFactor} 
  This macro sets the priority factor for nice users. See
  section~\ref{sec:user-priority-explained}
  on User Priorities for details.  Defaults to 10000000.

\item[\Macro{REMOTE\_PRIO\_FACTOR}] \label{param:RemotePrioFactor} 
  This macro defines the priority factor for remote users (users who
  who do not belong to the accountant's local domain - see
  below). See section~\ref{sec:user-priority-explained}
  on User Priorities for details.  Defaults to 10000.

\item[\Macro{ACCOUNTANT\_LOCAL\_DOMAIN}] \label{param:AccountantLocalDomain} 
  This macro is used to decide if a user is local or remote. A user
  is considered to be in the local domain if the UID\_DOMAIN matches
  the value of this macro. Usually, this macro is set
  to the local UID\_DOMAIN. If it is not defined, all users are considered
  local.

\item[\Macro{MAX\_ACCOUNTANT\_DATABASE\_SIZE}] 
  \label{param:MaxAccountantDatabaseSize}
  This macro defines the maximum size (in bytes) that the accountant
  database log file can reach before it is truncated (which re-writes
  the file in a more compact format).
  If, after truncating, the file is larger than one half the maximum
  size specified with this macro, the maximum size will be
  automatically expanded.
  The default is 1 megabyte (1000000).

\item[\Macro{NEGOTIATOR\_DISCOUNT\_SUSPENDED\_RESOURCES}] \label{param:NegotiatorDiscountSuspendedResources} 
   This macro tells the negotiator to not count resources that are suspended
   when calculating the number of resources a user is using. 
   Defaults to false, that is, a user is still charged for a resource even
   when that resource has suspended the job.

\item[\Macro{NEGOTIATOR\_SOCKET\_CACHE\_SIZE}]
  \label{param:NegotiatorSocketCacheSize} This macro defines the
  maximum number of sockets that the negotiator keeps in its
  open socket cache.  Caching open sockets makes the negotiation
  protocol more efficient by eliminating the need for socket
  connection establishment for each negotiation cycle.  The default is
  currently 16.  To be effective, this parameter should be set to a
  value greater than the number of \Condor{schedd}s submitting jobs to the
  negotiator at any time.  If you lower this number, you must run
  \Condor{restart} and not just \Condor{reconfig} for the change to
  take effect.

\item[\Macro{NEGOTIATOR\_INFORM\_STARTD}]
  \label{param:NegotiatorInformStartd}
  Boolean setting that controls if the \Condor{negotiator} should
  inform the \Condor{startd} when it has been matched with a job.
  The default is \Expr{True}.
  When this is set to \Expr{False}, the \Condor{startd} will never
  enter the Matched state, and will go directly from Unclaimed to
  Claimed.
  Because this notification is done via UDP, if a pool is configured
  so that the execute hosts do not create UDP command sockets (see the
  \Macro{WANT\_UDP\_COMMAND\_SOCKET} setting described in
  section~\ref{param:WantUDPCommandSocket} on
  page~\pageref{param:WantUDPCommandSocket} for details), the
  \Condor{negotiator} should be configured not to attempt to contact
  these \Condor{startds} by configuring this setting to \Expr{False}.

\item[\Macro{NEGOTIATOR\_PRE\_JOB\_RANK}]
  \label{param:NegotiatorPreJobRank} Resources that match a request
  are first sorted by this expression.  If there are any ties in the
  rank of the top choice, the top resources are sorted by the
  user-supplied rank in the job ClassAd, then by
  \MacroNI{NEGOTIATOR\_POST\_JOB\_RANK}, then by
  \MacroNI{PREEMPTION\_RANK} (if the match would cause preemption and
  there are still any ties in the top choice).  \verb@MY@ refers to
  attributes of the machine ClassAd and \verb@TARGET@ refers to the
  job ClassAd.  The purpose of the pre job rank is to allow the pool
  administrator to override any other rankings, in order to optimize
  overall throughput.  For example, it is commonly used to minimize
  preemption, even if the job rank prefers a machine that is busy.  If
  undefined, this expression has no effect on the ranking of matches.
  The standard configuration file shipped with Condor specifies an
  expression to steer jobs away from busy resources:

\begin{verbatim}
NEGOTIATOR_PRE_JOB_RANK = RemoteOwner =?= UNDEFINED
\end{verbatim}

\item[\Macro{NEGOTIATOR\_POST\_JOB\_RANK}]
  \label{param:NegotiatorPostJobRank}
  Resources that match a request are first sorted by
  \MacroNI{NEGOTIATOR\_PRE\_JOB\_RANK}.  If there are any ties in the
  rank of the top choice, the top resources are sorted by the
  user-supplied rank in the job ClassAd, then by
  \MacroNI{NEGOTIATOR\_POST\_JOB\_RANK}, then by
  \MacroNI{PREEMPTION\_RANK} (if the match would cause preemption and
  there are still any ties in the top choice).  \verb@MY@ refers to
  attributes of the machine ClassAd and \verb@TARGET@ refers to the
  job ClassAd.  The purpose of the post job rank is to allow the pool
  administrator to choose between machines that the job ranks equally.
  The default value is undefined, which causes this rank to have no
  effect on the ranking of matches.  The following example expression
  steers jobs toward faster machines and tends to fill a cluster of
  multi-processors by spreading across all machines before filling up
  individual machines.  In this example, the expression is chosen to
  have no effect when preemption would take place, allowing control to
  pass on to \MacroNI{PREEMPTION\_RANK}.

\begin{verbatim}
UWCS_NEGOTIATOR_POST_JOB_RANK = \
 (RemoteOwner =?= UNDEFINED) * (KFlops - VirtualMachineID)
\end{verbatim}


\item[\Macro{PREEMPTION\_REQUIREMENTS}]
  \label{param:PreemptionRequirements} When considering user priorities, the negotiator will not preempt
  a job running on a given machine unless the
  \MacroNI{PREEMPTION\_REQUIREMENTS} expression evaluates to \Expr{True} and the
  owner of the idle job has a better priority than the owner of the
  running job. 
  The \MacroNI{PREEMPTION\_REQUIREMENTS} expression is evaluated within the
  context of the candidate machine ClassAd and the candidate idle job
  ClassAd; thus the \verb@MY@ scope prefix refers to the machine ClassAd,
  and the \verb@TARGET@ scope prefix refers to the ClassAd of the idle
  (candidate) job.  
  If not explicitly set in the Condor configuration file, the default value
  for this expression is \Expr{True}.
  Note that this setting does not
  influence other potential causes of preemption, such as startd
  \MacroNI{RANK}, or \MacroNI{PREEMPT} expressions.  See
  section \ref{sec:Disabling Preemption} for a general discussion of
  limiting preemption.

\item[\Macro{PREEMPTION\_RANK}] \label{param:PreemptionRank} Resources
  that match a request are first sorted by
  \MacroNI{NEGOTIATOR\_PRE\_JOB\_RANK}.  If there are any ties in the
  rank of the top choice, the top resources are sorted by the
  user-supplied rank in the job ClassAd, then by
  \MacroNI{NEGOTIATOR\_POST\_JOB\_RANK}, then by
  \MacroNI{PREEMPTION\_RANK} (if the match would cause preemption and
  there are still any ties in the top choice).  \verb@MY@ refers to
  attributes of the machine ClassAd and \verb@TARGET@ refers to the
  job ClassAd.  This expression is used to rank machines that the job
  and the other negotiation expressions rank the same.  For example,
  if the job has no preference, it is usually preferable to preempt a
  job with a small \AdAttr{ImageSize} instead of a job with a large
  \AdAttr{ImageSize}.  The default is to rank all preemptable matches
  the same.  However, the negotiator will always prefer to match the
  job with an idle machine over a preemptable machine, if none of the
  other ranks express a preference between them.

\item[\Macro{NEGOTIATOR\_DEBUG}] \label{param:NegotiatorDebug} This macro
  (and other settings related to debug logging in the negotiator) is
  described in section~\ref{param:SubsysDebug} as \MacroNI{<SUBSYS>\_DEBUG}.

\item[\Macro{NEGOTIATOR\_MAX\_TIME\_PER\_SUBMITTER}] \label{param:NegotiatorMaxTimePerSubmitter} This macro limits the amount of time the negotiator will 
spend with a submitter. It defaults to one year.

\item[\Macro{NEGOTIATOR\_MATCH\_EXPRS}]
\label{param:NegotiatorMatchExprs} This macro specifies a list of macro names
that are inserted as ClassAd attributes into matched job ClassAds.
The attribute name in the ClassAd will be given the prefix
NegotiatorMatchExpr if the macro name doesn't already begin with that.
Example:

\begin{verbatim}
NegotiatorName = "My Negotiator"
NEGOTIATOR_MATCH_EXPRS = NegotiatorName
\end{verbatim}

As a result of the above configuration, jobs that are matched by this
negotiator will contain the following attribute when they are sent to
the startd:

\begin{verbatim}
NegotiatorMatchExprNegotiatorName = "My Negotiator"
\end{verbatim}

The expressions inserted by the negotiator may be useful in startd
policy expressions when the startd belongs to multiple condor pools.

\end{description}
The following configuration macros affect negotiation for group users.
\begin{description}

\item[\Macro{GROUP\_NAMES}] \label{param:GroupNames}
  A comma-separated list of the recognized group names, case insensitive.
  If undefined (the default), group support is disabled.
  Group names must not conflict with any user names.
  That is, if there is a \verb@physics@ group, there may not be
  a \verb@physics@ user.
  Any group that is defined here must also have a quota,
  or the group will be ignored. Example: 
  \begin{verbatim}
    GROUP_NAMES = group_physics, group_chemistry 
  \end{verbatim}

\item[\Macro{GROUP\_QUOTA\_<groupname>}] \label{param:GroupQuotaGroupname}
  A positive integer  to represent a static quota specifying
  the exact number of machines owned by this group.
  Note that Condor does not verify or check consistency of quota values.
  % When both are defined, static quotas supersede dynamic quotas. 
  Example:
  \begin{verbatim}
    GROUP_QUOTA_group_physics = 20
    GROUP_QUOTA_group_chemistry = 10
  \end{verbatim}

%\item[\Macro{GROUP\_DYNAMIC\_MACH\_CONSTRAINT}] \label{param:GroupDynamicMachConstraint}
%  Filter the number of machines for use with dynamic groups accounting
%  quotas, below.  Without this filter, dynamic quotas are calculated
%  from the total of all machines, in all states, that have reported to
%  the collector, which is probably not what you want.  Use this filter to
%  restrict the number of machines reserved for calculating dynamic
%  quotas.  The value is a ClassAd expression, and is evaluated every
%  negotiation cycle.  For example, the following expression will only
%  consider unclaimed Intel/Linux machines for dynamic quotas.
%
%  \begin{verbatim}
%    GROUP_DYNAMIC_MACH_CONSTRAINT = State == "Unclaimed" && \
%    Arch == "INTEL" && OpSys == "LINUX"
%  \end{verbatim}
%  will only considered. If all computers should be considered for
%  dynamic quotas, then the following expression could be used:
%  \begin{verbatim}
%      GROUP_DYNAMIC_MACH_CONSTRAINT = TRUE
%  \end{verbatim}


%\item[\Macro{GROUP\_QUOTA\_DYNAMIC\_<groupname>}] \label{param:GroupQuotaDynamicGroupname}
%  Specify a dynamic group accounting quota.  For example, the following
%  specifies that a quota of 25\% of the total dynamic quota machines are
%  reserved for members of the group\_biology group.
%  \begin{verbatim}
%  	GROUP_QUOTA_DYNAMIC_group_biology = 0.25
%  \end{verbatim}
%  The total dynamic quota machine count is determined using
%  \Macro{GROUP\_DYNAMIC\_MACH\_CONSTRAINT above}. The group name must
%  also be specified in the \Macro{GROUP\_NAMES} list. The value must be
%  positive float between 0.0 and 1.0. Condor does not verify that the
%  quota value is reasonable, nor does Condor verify that all specified
%  quotas are consistent.  This parameter is evaluated whenever Condor
%  negotiates for the group.  When both are defined, static quotas
%  supersede dynamic quotas.

\item[\Macro{GROUP\_PRIO\_FACTOR\_<groupname>}] \label{param:GroupPrioFactorGroupname}
  A floating point value greater than or equal to 1.0 to specify the
  default user priority factor for \verb@<groupname>@. 
  The group name must also be specified in the \MacroNI{GROUP\_NAMES} list.
  \MacroNI{GROUP\_PRIO\_FACTOR\_<groupname>} is evaluated when
  the negotiator first negotiates for the user as a member of the group.
  All members of the group inherit the default priority factor
  when no other value is present.
  For example, the following setting
  specifies that all members of the group named \verb@group_physics@
  inherit a default user priority factor of 2.0:
  \begin{verbatim}
    GROUP_PRIO_FACTOR_group_physics = 2.0
  \end{verbatim}

\item[\Macro{GROUP\_AUTOREGROUP}] \label{param:GroupAutoregroup}
  A boolean value (defaults to \Expr{False}) that when \Expr{True},
  causes users who submitted to a specific group to
  also negotiate a second time with the \verb@none@ group,
  to be considered with the independent job submitters. 
  This allows group submitted jobs to be matched with idle machines
  even if the group is over its quota.

\item[\Macro{GROUP\_AUTOREGROUP\_<groupname>}]
  This is the same as \MacroNI{GROUP\_AUTOREGROUP}, but it is settable
  on a per-group basis.  If no value is specified for a given group,
  the default behavior is determined by \MacroNI{GROUP\_AUTOREGROUP},
  which in turn defaults to \Expr{False}.

\item[\Macro{NEGOTIATOR\_CONSIDER\_PREEMPTION}] \label{param:NegotiatorConsiderPreemption}
  For expert users only. A boolean value (defaults to \Expr{True}),
  that when \Expr{False},
  can cause the negotiator to run
  faster and also have better spinning pie accuracy.
  \emph{Only set this to \Expr{False} if \Macro{PREEMPTION\_REQUIREMENTS}
  is \Expr{False},
  and if all \condor{startd} rank expressions are \Expr{False}.}

\end{description}

% entire section commented out Nov 2005
%%%%%%%%%%%%%%%%%%%%%%%%%%%%%%%%%%%%%%%%%%%%%%%%%%%%%%%%%%%%%%%%%%%%%%%%%%%
% \subsection{\label{sec:Eventd-Config-File-Entries}
% \condor{eventd} Configuration File Entries}
%%%%%%%%%%%%%%%%%%%%%%%%%%%%%%%%%%%%%%%%%%%%%%%%%%%%%%%%%%%%%%%%%%%%%%%%%%%

% \index{configuration!condor\_eventd configuration variables}
% These macros affect the Condor Event daemon.  See
% section~\ref{sec:EventD} on page~\pageref{sec:EventD} for an
% introduction.  The eventd is not included in the main Condor binary
% distribution or installation procedure.  It can be installed as a
% contrib module.
% 
% \begin{description}
  
% \item[\Macro{EVENT\_LIST}] \label{param:EventList} List of macros
% which define events to be managed by the event daemon.

% \item[\Macro{EVENTD\_CAPACITY\_INFO}] \label{param:EventdCapInfo}
% Configures the bandwidth limits used when scheduling job checkpoint
% transfers before \MacroNI{SHUTDOWN} events.
% The \MacroNI{EVENTD\_CAPACITY\_INFO} file has the same
% format as the \MacroNI{NETWORK\_CAPACITY\_INFO} file, described in
% section~\ref{sec:Bandwidth-Alloc-Capinfo}.

% \item[\Macro{EVENTD\_ROUTING\_INFO}] \label{param:EventdRouteInfo}
% Configures the network routing information used when scheduling job
% checkpoint transfers before \MacroNI{SHUTDOWN} events.
% The \MacroNI{EVENTD\_ROUTING\_INFO} file has the same
% format as the \MacroNI{NETWORK\_ROUTING\_INFO} file, described in
% section~\ref{sec:Bandwidth-Alloc-Routes}.

% \item[\Macro{EVENTD\_INTERVAL}] \label{param:EventdInterval} The number
% of seconds between collector queries to determine pool
% state.  The default is 15 minutes (300 seconds).

% \item[\Macro{EVENTD\_MAX\_PREPARATION}]
% \label{param:EventdMaxPreparation}  The number of minutes before a
% scheduled event when the eventd should start periodically querying the
% collector.  If 0 (default), the eventd always polls.

% \item[\Macro{EVENTD\_SHUTDOWN\_SLOW\_START\_INTERVAL}]
% \label{param:EventdShutdownSlowStartInterval} The number of seconds
% between each machine startup after a shutdown event.  The default is 0.

% \item[\Macro{EVENTD\_SHUTDOWN\_CLEANUP\_INTERVAL}]
% \label{param:EventdShutdownCleanupInterval} The number of seconds
% between each check for old shutdown configurations in the pool.  The default
% is one hour (3600 seconds).

% \end{description}

%%%%%%%%%%%%%%%%%%%%%%%%%%%%%%%%%%%%%%%%%%%%%%%%%%%%%%%%%%%%%%%%%%%%%%
\subsection{\label{sec:Procd-Config-File-Entries}\condor{procd}
Configuration File Macros}
%%%%%%%%%%%%%%%%%%%%%%%%%%%%%%%%%%%%%%%%%%%%%%%%%%%%%%%%%%%%%%%%%%%%%%

\begin{description}

\item[\Macro{USE\_PROCD}]\label{param:UseProcd} This boolean parameter
is used to determine whether the \Condor{procd} will be used for
managing process families. If the \Condor{procd} is not used, each
daemon will run the process family tracking logic on its own. Use of
the \Condor{procd} results in improved scalability because only one
instance of this logic is required. The \Condor{procd} is required
when using privilege separation (see Section~\ref{sec:PrivSep}) or
group ID-based process tracking (see
Section~\ref{sec:GroupTracking}). In either of these cases, the
\MacroNI{USE\_PROCD} setting will be ignored and a \Condor{procd} will
always be used. By default, the \Condor{master} will not use a
\Condor{procd} but all other daemons that need process family tracking
will.
A daemon that uses the \Condor{procd} will start a \Condor{procd} for
use by itself and all of its child daemons.

\item[\Macro{PROCD\_MAX\_SNAPSHOT\_INTERVAL}]\label{param:ProcdMaxSnapshotInterval}
This setting determines the maximum time that the \Condor{procd} will
wait between probes of the system for information about the process
families it is tracking.

\item[\Macro{PROCD\_LOG}]\label{param:ProcdLog} Specifies a log file
for the ProcD to use. Note that by design, the \Condor{procd} does not
include most of the other logic that is shared amongst the various
Condor daemons. This is because the \Condor{procd} is a component of
the PrivSep Kernel (see Section~\ref{sec:PrivSep} for more information
regarding privilege separation). This means that the \Condor{procd}
does not include the normal Condor logging subsystem, and thus things
like multiple debug levels and log rotation are not
supported. Therefore, \MacroNI{PROCD\_LOG} is not set by default and
is only intended to debug problems should they arise. Note, however,
that enabling \Dflag{PROCFAMILY} in the debug level for any other
daemon will cause it to log all interactions with the \Condor{procd}.

\item[\Macro{PROCD\_ADDRESS}]\label{param:ProcdAddress} This specifies
the ``address'' that the \Condor{procd} will use to receive requests
from other Condor daemons. On UNIX, this should point to a file system
location that can be used for a named pipe. On Windows, named pipes
are also used but they do not exist in the file system. The default
setting therefore depends on the platform: \Code{\$(LOCK)/procd\_pipe}
on UNIX and
\Code{$\backslash$$\backslash$.$\backslash$pipe$\backslash$procd\_pipe}
on Windows.

\end{description}

%%%%%%%%%%%%%%%%%%%%%%%%%%%%%%%%%%%%%%%%%%%%%%%%%%%%%%%%%%%%%%%%%%%%%%
\subsection{\label{sec:Credd-Config-File-Entries}\condor{credd}
Configuration File Macros}
%%%%%%%%%%%%%%%%%%%%%%%%%%%%%%%%%%%%%%%%%%%%%%%%%%%%%%%%%%%%%%%%%%%%%%
 
\begin{description}

\item[\Macro{CREDD\_HOST}]\label{param:CreddHost}
The host name of the machine running the \Condor{credd} daemon.

\item[\Macro{CREDD\_CACHE\_LOCALLY}]\label{param:CreddCacheLocally}
A boolean value that defaults to \Expr{False}.
When \Expr{True}, the first successful password fetch operation to the
\Condor{credd} daemon causes the password to be stashed in a local, 
secure password store.
Subsequent uses of that password do not require
communication with the \Condor{credd} daemon.

\end{description}


%%%%%%%%%%%%%%%%%%%%%%%%%%%%%%%%%%%%%%%%%%%%%%%%%%%%%%%%%%%%%%%%%%%%%%%%%%%
\subsection{\label{sec:Gridmanager-Config-File-Entries}\condor{gridmanager}
Configuration File Entries}
%%%%%%%%%%%%%%%%%%%%%%%%%%%%%%%%%%%%%%%%%%%%%%%%%%%%%%%%%%%%%%%%%%%%%%%%%%%

\index{configuration!condor\_gridmanager configuration variables}
These macros affect the \Condor{gridmanager}.
\begin{description}

\item[\Macro{GRIDMANAGER\_LOG}]
\label{param:GridmanagerLog} Defines the path and file name
  for the log of the \Condor{gridmanager}. 
  The owner of the file is the \Login{condor} user.

\item[\Macro{GRIDMANAGER\_CHECKPROXY\_INTERVAL}]
\label{param:GridmanagerCheckproxyInterval} The number of seconds
between checks for an updated X509 proxy credential. The default
is 10 minutes (600 seconds).

\item[\Macro{GRIDMANAGER\_MINIMUM\_PROXY\_TIME}]
\label{param:GridmanagerMinimumProxyTime} The minimum number of
seconds before expiration of the X509 proxy credential for the
gridmanager to continue operation. If seconds until expiration is
less than this number, the gridmanager will shutdown and wait for
a refreshed proxy credential. The default is 3 minutes (180 seconds).

\item[\Macro{HOLD\_JOB\_IF\_CREDENTIAL\_EXPIRES}]
\label{param:HoldJobIfCredentialExpires} True or False.
Defaults to True.
If True, and for grid universe jobs only,
Condor-G will place a job on hold
\MacroNI{GRIDMANAGER\_MINIMUM\_PROXY\_TIME} seconds
before the proxy expires.
If False,
the job will stay in the last known state,
and Condor-G will periodically check to see if the job's proxy has been
refreshed, at which point management of the job will resume.

\item[\Macro{GRIDMANAGER\_CONTACT\_SCHEDD\_DELAY}]
\label{param:GridmanagerContactScheddDelay} The minimum number of
seconds between connections to the \Condor{schedd}. The default is 5 seconds.

\item[\Macro{GRIDMANAGER\_JOB\_PROBE\_INTERVAL}]
\label{param:GridmanagerJobProbeInterval}
The number of seconds between
active probes of the status of a submitted job. The default is 5
minutes (300 seconds).

\item[\Macro{CONDOR\_JOB\_POLL\_INTERVAL}]
\label{param:CondorJobPollInterval}
After a condor grid type job is submitted, how often (in seconds) the \Condor{gridmanager}
should probe the remote \Condor{schedd} to check the jobs status.  
This defaults to 300 seconds (5 minutes).
Setting this to a lower number will decrease latency (Condor will discover
that a job has finished more quickly), but will increase network traffic.


\item[\Macro{GRIDMANAGER\_RESOURCE\_PROBE\_INTERVAL}]
\label{param:GridmanagerResourceProbeInterval}
When a resource appears to be down, how often (in seconds) the
\Condor{gridmanager}
should ping it to test if it is up again.

\item[\Macro{GRIDMANAGER\_RESOURCE\_PROBE\_DELAY}]
\label{param:GridmanagerResourceProbeDelay} The number of seconds
between pings of a remote resource that is currently down. The default
is 5 minutes (300 seconds).

\item[\Macro{GRIDMANAGER\_EMPTY\_RESOURCE\_DELAY}]
  \label{param:GridmanagerEmptyResourceDelay} The number of seconds
  that the \Condor{gridmanager} retains information about a grid
  resource, once the \Condor{gridmanager} has no active jobs
  on that resource.
  An active job is a grid universe job that is in the queue,
  but is not in the HELD state.
  Defaults to 300 seconds.

\item[\Macro{GRIDMANAGER\_MAX\_SUBMITTED\_JOBS\_PER\_RESOURCE}]
\label{param:GridmanagerMaxSubmittedJobsPerResource}
Limits the number of jobs
that a \Condor{gridmanager} daemon will submit to a resource.
It is useful for controlling the number of \Prog{jobmanager}
processes running on the front-end node of a cluster.
This number may be exceeded if it is reduced through the use
of \Condor{reconfig} while the \Condor{gridmanager} is running
or if the \Condor{gridmanager} receives new
jobs from the \Condor{schedd} that were already submitted
(that is, their \MacroNI{GridJobId} is not undefined).
In these cases, submitted jobs will not be killed,
but no new jobs can be submitted until the number of submitted
jobs falls below the current limit.
Defaults to 100.

\item[\Macro{GRIDMANAGER\_MAX\_PENDING\_SUBMITS\_PER\_RESOURCE}]
\label{param:GridmanagerMaxPendingSubmitsPerResource} The maximum
number of jobs
that can be in the process of being submitted at any time (that is,
how many \Procedure{globus\_gram\_client\_job\_request} calls are pending).
It is useful for controlling the number of new
connections/processes created at a given time.
The default value is 5.
This variable allows
you to set different limits for each resource.
After the first integer in the value
comes a list of resourcename/number pairs,
where each number is the limit for that resource.
If a resource is not in the list,
Condor uses the first integer.
An example usage:
\begin{verbatim}
GRIDMANAGER_MAX_PENDING_SUBMITS_PER_RESOURCE=20,nostos,5,beak,50
\end{verbatim}

\item[\Macro{GRIDMANAGER\_MAX\_PENDING\_SUBMITS}]
\label{param:GridmanagerMaxPendingSubmits} Configuration variable
still recognized, but the name has changed to be
\MacroNI{GRIDMANAGER\_MAX\_PENDING\_SUBMITS\_PER\_RESOURCE}.

\item[\Macro{GRIDMANAGER\_MAX\_JOBMANAGERS\_PER\_RESOURCE}]
\label{param:GridmanagerMaxJobmanagersPerResource}
For grid jobs of type \SubmitCmd{gt2}, limits the number of globus-job-manager
processes that the \Condor{gridmanager} lets run at a time on
the remote head node. Allowing too many globus-job-managers to run
causes severe load on the headnote, possibly making it
non-functional.
This number may be exceeded if it is reduced through the use
of \Condor{reconfig} while the \Condor{gridmanager} is running
or if some globus-job-managers take a few extra seconds to exit.
The value 0 means there is no limit. The default value is 10.

\item[\Macro{GAHP}]
\label{param:Gahp} The full path to the binary of the GAHP server.
This configuration variable is no longer used.
Use \MacroNI{GT2\_GAHP} at section~\ref{param:GT2GAHP} instead.

\item[\Macro{GAHP\_ARGS}]
\label{param:GahpArgs} Arguments to be passed to the GAHP server.
This configuration variable is no longer used.

\item[\Macro{GRIDMANAGER\_GAHP\_CALL\_TIMEOUT}]
\label{param:GridmanagerGahpCallTimeout} The number of seconds after
which a pending GAHP command should time out. The default is 5 minutes
(300 seconds).

\item[\Macro{GRIDMANAGER\_MAX\_PENDING\_REQUESTS}]
\label{param:GridmanagerMaxPendingRequests} The maximum number of GAHP
commands that can be pending at any time. The default is 50.

\item[\Macro{GRIDMANAGER\_CONNECT\_FAILURE\_RETRY\_COUNT}]
\label{param:GridmanagerConnectFailureRetryCount} The number of times
to retry a command that failed due to a timeout or a failed connection.
The default is 3.

\item[\Macro{GRIDMANAGER\_GLOBUS\_COMMIT\_TIMEOUT}]
\label{param:GridmanagerGlobusCommitTimeout}  The duration, in seconds, of the
two phase commit timeout to Globus for gt2 jobs only.  This maps directly to the \texttt{two\_phase} setting in the Globus RSL.

% configuration variable introduced into Condor due to Globus a GAHP
% bug.  Users with Globus where this bug has been fixed should never
% use nor even know about this configuration variable.
% \item[\Macro{GRIDMANAGER\_RESTART\_ON\_ANY\_DOWN\_RESOURCES}]
% \label{param:GridmanagerRestartOnAnyDownResources} 
% The current GAHP server can transition into a state where it
% cannot contact remote machines,
% making the machines appear to be down.
% Restarting the GAHP server fixes the problem.
% This parameter controls whether one machine or
% all machine need to appear to be down to trigger a restart of the GAHP
% server.

% configuration variable introduced into Condor due to Globus a GAHP
% bug.  Users with Globus where this bug has been fixed should never
% use nor even know about this configuration variable.
%\item[\Macro{GRIDMANAGER\_MAX\_TIME\_DOWN\_RESOURCES}]
%\label{param:GridmanagerMaxTimeDownResources} 
%Related to GRIDMANAGER\_RESTART\_ON\_ANY\_DOWN\_RESOURCES,
%this configuration variable defines how long (in seconds) one or
%more machines need to be down (or just appear to be) to trigger a restart
%of the GAHP server.

\item[\Macro{GLOBUS\_GATEKEEPER\_TIMEOUT}]
\label{param:GlobusGatekeeperTimeout} The number of seconds after
which if a gt2 grid
universe job fails to ping the gatekeeper,
the job will be put on hold.
Defaults to 5 days (in seconds).

\item[\Macro{C\_GAHP\_LOG}]
\label{param:CGAHPLog} The complete path and file name of the
Condor GAHP server's log.
There is no default value. The expected location as defined
in the example configuration is \File{/temp/CGAHPLog.\MacroUNI{USERNAME}}.

\item[\Macro{MAX\_C\_GAHP\_LOG}]
\label{param:MaxCGAHPLog} The maximum size of the \MacroNI{C\_GAHP\_LOG}.

\item[\Macro{C\_GAHP\_WORKER\_THREAD\_LOG}]
\label{param:CGAHPWorkerThreadLog} The complete path and file name of the
Condor GAHP worker process' log.
There is no default value. The expected location as defined
in the example configuration is \File{/temp/CGAHPWorkerLog.\MacroUNI{USERNAME}}.

\item[\Macro{GLITE\_LOCATION}]
\label{param:GLITELocation} The complete path to the directory
containing the Glite software.
There is no default value. The expected location as given
in the example configuration is \File{\MacroUNI{LIB}/glite}.
The necessary Glite software is included with Condor,
and is required for pbs and lsf jobs.

\item[\Macro{CONDOR\_GAHP}]
\label{param:CondorGAHP} The complete path and file name of the
Condor GAHP executable.
There is no default value. The expected location as given
in the example configuration is \File{\MacroUNI{SBIN}/condor\_c-gahp}.

\item[\Macro{GT2\_GAHP}]
\label{param:GT2GAHP} The complete path and file name of the
GT2 GAHP executable.
There is no default value. The expected location as given
in the example configuration is \File{\MacroUNI{SBIN}/gahp\_server}.

\item[\Macro{GT4\_GAHP}]
\label{param:GT4GAHP} The complete path and file name of the
wrapper script that invokes the GT4 GAHP executable.
There is no default value. The expected location as given
in the example configuration is \File{\MacroUNI{SBIN}/gt4\_gahp}.

\item[\Macro{PBS\_GAHP}]
\label{param:PBSGAHP} The complete path and file name of the
PBS GAHP executable.
There is no default value. The expected location as given
in the example configuration is \File{\MacroUNI{GLITE\_LOCATION}/bin/batch\_gahp}.

\item[\Macro{LSF\_GAHP}]
\label{param:LSFGAHP} The complete path and file name of the
LSF GAHP executable.
There is no default value. The expected location as given
in the example configuration is \File{\MacroUNI{GLITE\_LOCATION}/bin/batch\_gahp}.

\item[\Macro{UNICORE\_GAHP}]
\label{param:UnicoreGAHP} The complete path and file name of the
wrapper script that invokes the Unicore GAHP executable.
There is no default value. The expected location as given
in the example configuration is \File{\MacroUNI{SBIN}/unicore\_gahp}.

\item[\Macro{NORDUGRID\_GAHP}]
\label{param:NorduGridGAHP} The complete path and file name of the
wrapper script that invokes the NorduGrid GAHP executable.
There is no default value. The expected location as given
in the example configuration is \File{\MacroUNI{SBIN}/nordugrid\_gahp}.

\end{description}

%%%%%%%%%%%%%%%%%%%%%%%%%%%%%%%%%%%%%%%%%%%%%%%%%%%%%%%%%%%%%%%%%%%%%%%%%%%
\subsection{\label{sec:JobRouter-Config-File-Entries}\condor{job\_router}
Configuration File Entries}
%%%%%%%%%%%%%%%%%%%%%%%%%%%%%%%%%%%%%%%%%%%%%%%%%%%%%%%%%%%%%%%%%%%%%%%%%%%

\index{configuration!condor\_job\_router configuration variables}
These macros affect the \Condor{job\_router}.

\Todo
%\begin{description}
%\end{description}


%%%%%%%%%%%%%%%%%%%%%%%%%%%%%%%%%%%%%%%%%%%%%%%%%%%%%%%%%%%%%%%%%%%%%%%%%%%
\subsection{\label{sec:GridMonitor-Config-File-Entries}grid\_monitor
Configuration File Entries}
%%%%%%%%%%%%%%%%%%%%%%%%%%%%%%%%%%%%%%%%%%%%%%%%%%%%%%%%%%%%%%%%%%%%%%%%%%%

\index{configuration!grid\_monitor configuration variables}
These macros affect the \Prog{grid\_monitor}.
\begin{description}

\item[\Macro{ENABLE\_GRID\_MONITOR}] \label{param:EnableGridMonitor}
  When set to \Expr{True} enables the \Prog{grid\_monitor} tool.  The
  \Prog{grid\_monitor} tool is used to reduce load on Globus gatekeepers.
  This parameter only affects 
  grid jobs of type \SubmitCmd{gt2}.
  \MacroNI{GRID\_MONITOR} must also be correctly configured.
  Defaults to \Expr{False}.
  See section~\ref{sec:Condor-G-GridMonitor} on
  page~\pageref{sec:Condor-G-GridMonitor}
  for more information.

\item[\Macro{GRID\_MONITOR}] \label{param:GridMonitor}
  The complete pathname of the \Prog{grid\_monitor} tool used to reduce load on
  Globus gatekeepers.  This parameter only affects 
  grid jobs of type \SubmitCmd{gt2}.
  This parameter is not referenced unless
  \MacroNI{ENABLE\_GRID\_MONITOR} is set to \Expr{True}. 
  See section~\ref{sec:Condor-G-GridMonitor} on
  page~\pageref{sec:Condor-G-GridMonitor}
  for more information.

\item[\Macro{GRID\_MONITOR\_HEARTBEAT\_TIMEOUT}] \label{param:GridMonitorHeartbeatTimeout}
  If this many seconds pass without hearing from a \Prog{grid\_monitor}, it is
  assumed to be dead.  Defaults to 300 (5 minutes).  Increasing this number
  will improve the ability of the \Prog{grid\_monitor} to survive in the face of
  transient problems but will also increase the time before Condor notices a
  problem.

\item[\Macro{GRID\_MONITOR\_RETRY\_DURATION}] \label{param:GridMonitorRetryDuration}
  If something goes wrong with the \Prog{grid\_monitor} at a particular site
  (like \MacroNI{GRID\_MONITOR\_HEARTBEAT\_TIMEOUT} expiring), Condor-G will
  attempt to restart the \Prog{grid\_monitor} for this many seconds.  Defaults
  to 900 (15 minutes).  If this duration passes without success the
  \Prog{grid\_monitor} will be disabled for the site in question until 60
  minutes have passed.

\item[\Macro{GRID\_MONITOR\_NO\_STATUS\_TIMEOUT}] \label{param:GridMonitorNoStatusTimeout}
  Jobs can disappear from the \Prog{grid\_monitor}'s status reports for
  short periods of time under normal circumstances, but a prolonged
  absence is often a sign of problems on the remote machine. This parameter
  sets the amount of time (in seconds) that a job can be absent before the
  \Condor{gridmanager} reacts by restarting the GRAM \Prog{jobmanager}.
  The default if 15 minutes.

\end{description}


%%%%%%%%%%%%%%%%%%%%%%%%%%%%%%%%%%%%%%%%%%%%%%%%%%%%%%%%%%%%%%%%%%%%%%%%%%%
\subsection{\label{sec:Grid-Config-File-Entries}Configuration File
Entries Relating to Grid Usage and Glidein}
%%%%%%%%%%%%%%%%%%%%%%%%%%%%%%%%%%%%%%%%%%%%%%%%%%%%%%%%%%%%%%%%%%%%%%%%%%%

\index{configuration!grid and glidein configuration variables}
These macros affect the Condor's usage of grid resources
and glidein.
\begin{description}
\item[\Macro{GLIDEIN\_SERVER\_URLS}]
  \label{param:GlideinServerURLS}
  A comma or space-separated list of URLs that contain the binaries
  that must be copied by \Condor{glidein}.
  There are no default values, but working URLs that copy from the UW site
  are provided in the distributed sample configuration files.

\end{description}

%%%%%%%%%%%%%%%%%%%%%%%%%%%%%%%%%%%%%%%%%%%%%%%%%%%%%%%%%%%%%%%%%%%%%%%%%%%
\subsection{\label{sec:DAGMan-Config-File-Entries}Configuration File 
Entries for DAGMan}
%%%%%%%%%%%%%%%%%%%%%%%%%%%%%%%%%%%%%%%%%%%%%%%%%%%%%%%%%%%%%%%%%%%%%%%%%%%

\index{configuration!DAGMan configuration variables}
These macros affect the operation of DAGMan and DAGMan
jobs within Condor.

\begin{description}
\item[\Macro{DAGMAN\_MAX\_SUBMITS\_PER\_INTERVAL}]
\label{param:DAGManMaxSubmitsPerInterval}
An integer that controls how many individual jobs
\Condor{dagman} will submit in a row
before servicing other requests (such as a \Condor{rm}).
The legal range of values is 1 to 1000.
If defined with a value less than 1, the  value 1 will be used.
If defined with a value greater than 1000, the value 1000 will be used.
If not defined, it defaults to 5.

\item[\Macro{DAGMAN\_MAX\_SUBMIT\_ATTEMPTS}]
\label{param:DAGManMaxSubmitAttempts}
An integer that controls how
many times in a row \Condor{dagman} will attempt to execute
\Condor{submit} for a given job before giving up.
Note that consecutive attempts use an exponential backoff,
starting with 1 second.
The legal range of values is 1 to 16.
If defined with a value less than 1, the  value 1 will be used.
If defined with a value greater than 16, the value 16 will be used.
Note that a value of 16 would result in \Condor{dagman} trying for
approximately 36 hours before giving up.
If not defined,
it defaults to 6 (approximately two minutes before giving up).

\item[\Macro{DAGMAN\_SUBMIT\_DELAY}]
\label{param:DAGManSubmitDelay}
An integer that controls the number of seconds
that \Condor{dagman} will sleep before submitting consecutive jobs.
It can be increased to help reduce the load on the \Condor{schedd} daemon.
The legal range of values is 0 to 60.
If defined with a value less than 0, the  value 0 will be used.
If defined with a value greater than 60, the value 60 will be used.
The default value is 0.

\item[\Macro{DAGMAN\_STARTUP\_CYCLE\_DETECT}]
\label{param:DAGManStartupCycleDetect}
A boolean value that when \Expr{True}
causes \Condor{dagman} to check for cycles in the DAG before
submitting DAG node jobs,
in addition to its run time cycle detection.
If not defined, it defaults to \Expr{False}.

\item[\Macro{DAGMAN\_RETRY\_SUBMIT\_FIRST}]
\label{param:DAGManRetrySubmitFirst}
A boolean value that controls whether a failed submit is retried first
(before any other submits) or last (after all other ready jobs are
submitted).  If this value is set to \Expr{True}, when a job submit
fails, the job is placed at the head of the queue of ready jobs, so
that it will be submitted again before any other jobs are submitted
(this has been the behavior of \Condor{dagman} up to this point).
If this value is set to \Expr{False}, when a job submit fails, the job
is placed at the tail of the queue of ready jobs.
If not defined, it defaults to \Expr{True}.

\item[\Macro{DAGMAN\_RETRY\_NODE\_FIRST}]
\label{param:DAGManRetryNodeFirst}
A boolean value that controls whether a failed node (with retries)
is retried first (before any other ready nodes) or last (after all
other ready nodes).  If this value is set to \Expr{True}, when a
node with retries fails (after the submit succeeded), the node is
placed at the head of the queue of ready nodes, so that it will be
tried again before any other jobs are submitted.  If this value is
set to \Expr{False}, when a node with retries fails, the node
is placed at the tail of the queue of ready nodes (this has been the
behavior of \Condor{dagman} up to this point).  If not defined, it
defaults to \Expr{False}.

\item[\Macro{DAGMAN\_MAX\_JOBS\_IDLE}]
\label{param:DAGManMaxJobsIdle}
An integer value that controls the maximum number of idle node jobs
allowed within the DAG before \Condor{dagman} temporarily stops
submitting jobs.  Once idle jobs start to run, \Condor{dagman} will
resume submitting jobs.  If both the command-line flag and the
configuration parameter are specified, the command-line flag overrides
the configuration parameter.  Unfortunately,
\MacroNI{DAGMAN\_MAX\_JOBS\_IDLE} currently counts each individual
process within a cluster as a job, which is inconsistent with
\MacroNI{DAGMAN\_MAX\_JOBS\_SUBMITTED}.  The default is that there is
no limit on the maximum number of idle jobs.

\item[\Macro{DAGMAN\_MAX\_JOBS\_SUBMITTED}]
\label{param:DAGManMaxJobsSubmitted}
An integer value that controls the maximum number of node jobs within the
DAG that will  be submitted to Condor at one time.  Note that this
parameter is the same as the \OptArg{-maxjobs} command-line flag
to condor\_submit\_dag.  If both the command-line flag and the
configuration parameter are specified, the command-line flag overrides
the configuration parameter.  A single invocation of \Condor{submit}
counts as one job, even if the submit file produces a multi-job cluster.
The default is that there is no limit on the maximum number of jobs
run at one time.

\item[\Macro{DAGMAN\_MUNGE\_NODE\_NAMES}]
\label{param:DAGManMungeNodeNames}
A boolean value that controls whether \Condor{dagman} automatically
renames nodes when running multiple DAGs (the renaming is done to
avoid possible name conflicts).  If this value is set to \Expr{True},
all node names have the "DAG number" prepended to them.  For example,
the first DAG specified on the \Condor{submit\_dag} command line is
considered DAG number 0, the second is DAG number 1, etc.  So if
DAG number 2 has a node B, that node will internally be renamed
to "2.B".
If not defined, \MacroNI{DAGMAN\_MUNGE\_NODE\_NAMES} defaults to \Expr{True}.

% \item[\Macro{STORK\_SERVER}]
% \label{param:StorkServer}

\item[\Macro{DAGMAN\_IGNORE\_DUPLICATE\_JOB\_EXECUTION}]
\label{param:DAGManIgnoreDuplicateJobExecution}
This macro is no longer used. The improved functionality
of the \MacroNI{DAGMAN\_ALLOW\_EVENTS} macro eliminates the
need for this variable.

A boolean value that controls
whether \Condor{dagman} aborts or continues with a DAG
in the rare case that Condor erroneously executes
the job within a DAG node more than once.
A bug in Condor very occasionally causes a job to run twice.
Running a job twice is contrary to the semantics of a DAG.
The configuration macro \MacroNI{DAGMAN\_IGNORE\_DUPLICATE\_JOB\_EXECUTION}
determines whether  \Condor{dagman} considers this a fatal error or not.
The default value is \Expr{False}; \Condor{dagman} considers
running the job more than once a fatal error, 
logs this fact,
and aborts the DAG.
When set to \Expr{True}, \Condor{dagman} still
logs this fact,
but continues with the DAG. 

This configuration macro is to remain at its default value 
except in the case
where a site encounters the Condor bug in which DAG job nodes
are executed twice,
and where it is certain
that having a DAG job node run twice will not corrupt the DAG.
The logged messages within \File{*.dagman.out} files
in the case of that a node job runs twice
contain the string
"EVENT ERROR."

\item[\Macro{DAGMAN\_ALLOW\_EVENTS}]
\label{param:DAGManAllowEvents}
An integer that controls which "bad" events are considered
fatal errors by \Condor{dagman}.  This macro replaces and expands
upon the functionality of the
\MacroNI{DAGMAN\_IGNORE\_DUPLICATE\_JOB\_EXECUTION} macro.
If \MacroNI{DAGMAN\_ALLOW\_EVENTS} is set, it overrides the
setting of \MacroNI{DAGMAN\_IGNORE\_DUPLICATE\_JOB\_EXECUTION}.

The \MacroNI{DAGMAN\_ALLOW\_EVENTS} value is a bitwise-OR of the
following values:
\begin{description}
\item 0 = allow no "bad" events
\item 1 = allow almost all "bad" events (all except "job re-run after
terminated event")
\item 2 = allow terminated/aborted event combination
\item 4 = allow "job re-run after terminated event" bug
\item 8 = allow garbage/orphan events
\item 16 = allow execute or terminate event before job's submit event
\item 32 = allow two terminated events per job (sometimes seen
with grid jobs)
\item 64 = allow duplicated events in general
\end{description}

The default value is 114 (allow terminated/aborted event combination,
allow execute and/or terminated event before job's submit event, allow
double terminated events, and allow general duplicate events).

For example, a value of 6 instructs \Condor{dagman} to allow both
the terminated/aborted event combination and the "job re-run
after terminated event" bug.  A value of 0 means that any "bad"
event will be considered a fatal error.

A value of 5 (1 + 4) will never abort the DAG because of a "bad"
event -- but you should almost never use this setting, because
the "job re-run after terminated event" bug breaks the semantics of
the DAG.

This macro should almost always remain set to the default value!

\item[\Macro{DAGMAN\_DEBUG}] \label{param:DAGManDebug} This macro
is described in section~\ref{param:SubsysDebug} as
\MacroNI{<SUBSYS>\_DEBUG}.

\item[\Macro{MAX\_DAGMAN\_LOG}] \label{Param:MaxDAGManLog} This macro
is described in section~\ref{param:MaxSubsysLog} as
\MacroNI{MAX\_<SUBSYS>\_LOG}.

\item[\Macro{DAGMAN\_CONDOR\_SUBMIT\_EXE}]
\label{param:DAGManCondorSubmitExe}
The executable that \Condor{dagman} will use to submit Condor jobs.
If not defined, \Condor{dagman} looks for \Condor{submit} in the PATH.

\item[\Macro{DAGMAN\_STORK\_SUBMIT\_EXE}]
\label{param:DAGManStorkSubmitExe}
The executable that \Condor{dagman} will use to submit Stork jobs.
If not defined, \Condor{dagman} looks for \Stork{submit} in the PATH.

\item[\Macro{DAGMAN\_CONDOR\_RM\_EXE}]
\label{param:DAGManCondorRmExe}
The executable that \Condor{dagman} will use to remove Condor jobs.
If not defined, \Condor{dagman} looks for \Condor{rm} in the PATH.

\item[\Macro{DAGMAN\_STORK\_RM\_EXE}]
\label{param:DAGManStorkRmExe}
The executable that \Condor{dagman} will use to remove Stork jobs.
If not defined, \Condor{dagman} looks for \Stork{rm} in the PATH.

\item[\Macro{DAGMAN\_PROHIBIT\_MULTI\_JOBS}]
\label{param:DAGManProhibitMultiJobs}
A boolean value that controls whether \Condor{dagman} prohibits
node job submit files that queue multiple job procs (other than 
parallel universe).  If a DAG references such a submit file, the
DAG will abort during the initialization process.  If not defined,
\MacroNI{DAGMAN\_PROHIBIT\_MULTI\_JOBS} defaults to \Expr{False}.

\item[\Macro{DAGMAN\_LOG\_ON\_NFS\_IS\_ERROR}]
\label{param:DAGManLogOnNfsIsError}
A boolean value that controls whether \Condor{dagman} prohibits
node job submit files with user log files on NFS.  If a DAG
references such a submit file and \MacroNI{DAGMAN\_LOG\_ON\_NFS\_IS\_ERROR}
is \Expr{True}, the DAG will abort during the initialization process. 
If \MacroNI{DAGMAN\_LOG\_ON\_NFS\_IS\_ERROR} is \Expr{False}, a warning
will be issued but the DAG will still be submitted.  It is \emph{strongly}
recommended that \MacroNI{DAGMAN\_LOG\_ON\_NFS\_IS\_ERROR}
remain set to the default value, because running a DAG with node job
log files on NFS will often cause errors.
If not defined, \MacroNI{DAGMAN\_LOG\_ON\_NFS\_IS\_ERROR} defaults to
\Expr{True}.

\item[\Macro{DAGMAN\_ABORT\_DUPLICATES}]
\label{param:DAGManAbortDuplicates}
A boolean value that controls whether to attempt to abort duplicate
instances of \Condor{dagman} running the same DAG on the same
machine.  When \Condor{dagman} starts up, if no DAG lock file exists,
\Condor{dagman} creates the lock file and writes its PID into it.  If
the lock file does exist, and \MacroNI{DAGMAN\_ABORT\_DUPLICATES} is
set to \Expr{True}, \Condor{dagman} checks whether a process with the
given PID exists, and if so, it assumes that there is already another
instance of \Condor{dagman} running on the same DAG.  Note that this
test is not foolproof: it is possible that, if \Condor{dagman} crashes,
the same PID gets reused by another process before \Condor{dagman}
gets rerun on that DAG.  This should be quite rare, however.
If not defined, \MacroNI{DAGMAN\_ABORT\_DUPLICATES} defaults to
\Expr{True}.

\item[\Macro{DAGMAN\_SUBMIT\_DEPTH\_FIRST}]
\label{param:DAGManSubmitDepthFirst}
A boolean value that controls whether to submit ready DAG node jobs
in (more-or-less) depth first order, as opposed to breadth-first order.
Setting \MacroNI{DAGMAN\_SUBMIT\_DEPTH\_FIRST} to \Expr{True} does
\emph{not} override dependencies defined in the DAG.  Rather, it
causes newly-ready nodes to be added to the head, rather than the tail,
of the ready node list.  If there are no PRE scripts in the DAG, this
will cause the ready nodes to be submitted depth-first.  If there
are PRE scripts, the order will not be strictly depth-first, but it
will tend to favor depth rather than breadth in executing the DAG.
If you set \MacroNI{DAGMAN\_SUBMIT\_DEPTH\_FIRST} to \Expr{True},
you may also want to set \MacroNI{DAGMAN\_RETRY\_SUBMIT\_FIRST} and
\Macro{DAGMAN\_RETRY\_NODE\_FIRST} to \Expr{True}.
If not defined, \MacroNI{DAGMAN\_SUBMIT\_DEPTH\_FIRST} defaults to
\Expr{false}.

\item[\Macro{DAGMAN\_ON\_EXIT\_REMOVE}]
\label{param:DAGManOnExitRemove}
The \Attr{OnExitRemove} expression put into the \Condor{dagman} submit
file by \Condor{submit\_dag}.
The default expression is designed to ensure that \Condor{dagman} is
automatically re-queued by the schedd if it exits abnormally or is
killed (e.g., during a reboot).  If this results in \Condor{dagman}
staying in the queue when it should exit, you may want to change
to a less restrictive expression, for example:
\begin{verbatim}
(ExitBySignal == false || ExitSignal =!= 9)
\end{verbatim}
If not defined, \MacroNI{DAGMAN\_ON\_EXIT\_REMOVE} defaults to
\begin{verbatim}
( ExitSignal =?= 11 || (ExitCode =!= UNDEFINED && ExitCode >=0 && ExitCode <= 2))
\end{verbatim}

\item[\Macro{DAGMAN\_ABORT\_ON\_SCARY\_SUBMIT}]
\label{param:DAGManAbortOnScarySubmit}
A boolean value that controls whether to abort a DAG upon detection of
a ``scary'' submit event (one in which the Condor ID does not match
the expected value).  Note that in all versions prior to 6.9.3,
\Condor{dagman} has \emph{not} aborted a DAG upon detection of
a ``scary'' submit event (this behavior is what now happens if
\MacroNI{DAGMAN\_ABORT\_ON\_SCARY\_SUBMIT} is set to \Expr{false}).
If not defined, \MacroNI{DAGMAN\_ABORT\_ON\_SCARY\_SUBMIT} defaults to
\Expr{true}.

\item[\Macro{DAGMAN\_PENDING\_REPORT\_INTERVAL}]
\label{param:DAGManPendingReportInterval}
An integer value (in seconds) that controls how often \Condor{dagman}
will print a report of pending nodes to the \File{dagman.out} file.
Note that the report will only be printed if \Condor{dagman} has
been waiting at least \MacroNI{DAGMAN\_PENDING\_REPORT\_INTERVAL}
seconds without seeing any node job user log events, in order to
avoid cluttering the \File{dagman.out} file.  (This feature is mainly
intended to help diagnose "stuck" \Condor{dagman} processes that
are waiting indefinitely for a job to finish.) If not defined,
\MacroNI{DAGMAN\_PENDING\_REPORT\_INTERVAL} defaults to 600 seconds
(10 minutes).

\item[\Macro{DAGMAN\_INSERT\_SUB\_FILE}]
\label{param:DAGManInsertSubFile}
A file name of a file containing submit file commands to be inserted
into the \File{.condor.sub} file created by \Condor{submit\_dag}.
The specified file is inserted into the \File{.condor.sub} file before
the \SubmitCmd{queue} command and before any commands specified with the
\Opt{-append} \Condor{submit\_dag} command-line option.
Note that the \MacroNI{DAGMAN\_INSERT\_SUB\_FILE} value can be overridden
by the \Opt{-insert\_sub\_file} \Condor{submit\_dag} command-line option.

\item[\Macro{DAGMAN\_OLD\_RESCUE}]
\label{param:DAGManOldRescue}
A boolean value that controls whether \Condor{dagman} uses "old-style"
rescue DAG naming when creating a rescue DAG.  (With "old-style" rescue
DAG naming, if your DAG file is \File{my.dag}, the rescue DAG file will
be \File{my.dag.rescue}, and that file will be overwritten if you re-run
\File{my.dag} and it fails again.  With "new-style" rescue DAG naming,
the first time a rescue DAG is created for \File{my.dag}, it will be
named \File{my.dag.rescue001},and subsequent failures of
\File{my.dag} will produce rescue DAGs named \File{my.dag.rescue002},
\File{my.dag.rescue003}, etc.)
If not defined, \MacroNI{DAGMAN\_OLD\_RESCUE} defaults to
\Expr{false}.

\item[\Macro{DAGMAN\_AUTO\_RESCUE}]
\label{param:DAGManAutoRescue}
A boolean value that controls whether \Condor{dagman} automatically
runs rescue DAGs.  If \MacroNI{DAGMAN\_AUTO\_RESCUE} is true and you
run the DAG file \File{my.dag}, if a rescue dag such as
\File{my.dag.rescue001}, \File{my.dag.rescue002}, etc., exists, the newest
(highest-numbered) such rescue DAG will be run.
If not defined, \MacroNI{DAGMAN\_AUTO\_RESCUE} defaults to
\Expr{true}.

Note: having \MacroNI{DAGMAN\_OLD\_RESCUE} and
\MacroNI{DAGMAN\_AUTO\_RESCUE} both set to \Expr{true} is a
fatal error.

\item[\Macro{DAGMAN\_MAX\_RESCUE\_NUM}]
\label{param:DAGManMaxRescueNum}
An integer value that controls the maximum "new-style" rescue DAG
number that will be written (if \MacroNI{DAGMAN\_OLD\_RESCUE} is
\Expr{false}) or run (if \MacroNI{DAGMAN\_AUTO\_RESCUE} is
\Expr{true}).  The maximum legal value is 999; the minimum value
is 0 (which will prevent a rescue DAG from being written at all,
or automatically run).  If not defined,
\MacroNI{DAGMAN\_MAX\_RESCUE\_NUM} defaults to 100.

\end{description}

%%%%%%%%%%%%%%%%%%%%%%%%%%%%%%%%%%%%%%%%%%%%%%%%%%%%%%%%%%%%%%%%%%%%%%%%%%%
\subsection{\label{sec:Config-Security}Configuration File Entries
Relating to Security}
%%%%%%%%%%%%%%%%%%%%%%%%%%%%%%%%%%%%%%%%%%%%%%%%%%%%%%%%%%%%%%%%%%%%%%%%%%%

\index{configuration!security configuration variables}
These macros affect the secure operation of Condor.
Many of these macros are described in
section~\ref{sec:Security} on Security.

\begin{description}
\item[\Macro{SEC\_*\_AUTHENTICATION}]
\label{param:SecAuthentication} \Todo

\item[\Macro{SEC\_*\_ENCRYPTION}]
\label{param:SecEncryption} \Todo

\item[\Macro{SEC\_*\_INTEGRITY}]
\label{param:SecIntegrity} \Todo

\item[\Macro{SEC\_*\_NEGOTIATION}]
\label{param:SecNegotiation} \Todo

\item[\Macro{SEC\_*\_AUTHENTICATION\_METHODS}]
\label{param:SecAuthenticationMethods} \Todo

\item[\Macro{SEC\_*\_CRYPTO\_METHODS}]
\label{param:SecCryptoMethods} \Todo

\item[\Macro{GSI\_DAEMON\_NAME}]
\label{param:GSIDaemonName} A comma separated list of the subject
name(s) of the certificate(s) that the daemons use.

\item[\Macro{GSI\_DAEMON\_DIRECTORY}]
\label{param:GSIDaemonDirectory} A directory name used in the
construction of complete paths for the configuration variables
\MacroNI{GSI\_DAEMON\_CERT},
\MacroNI{GSI\_DAEMON\_KEY}, and
\MacroNI{GSI\_DAEMON\_TRUSTED\_CA\_DIR},
for any of these configuration variables are not explicitly set.

\item[\Macro{GSI\_DAEMON\_CERT}]
\label{param:GSIDaemonCert} A complete path and file name to the
X.509 certificate to be used in GSI authentication.
If this configuration variable is not defined, and
\MacroNI{GSI\_DAEMON\_DIRECTORY} is defined, then Condor uses
\MacroNI{GSI\_DAEMON\_DIRECTORY} to construct the path and file name as
\begin{verbatim}
GSI_DAEMON_CERT  = $(GSI_DAEMON_DIRECTORY)/hostcert.pem
\end{verbatim}

\item[\Macro{GSI\_DAEMON\_KEY}]
\label{param:GSIDaemonKey}  A complete path and file name to the
X.509 private key to be used in GSI authentication.
If this configuration variable is not defined, and
\MacroNI{GSI\_DAEMON\_DIRECTORY} is defined, then Condor uses
\MacroNI{GSI\_DAEMON\_DIRECTORY} to construct the path and file name as
\begin{verbatim}
GSI_DAEMON_KEY  = $(GSI_DAEMON_DIRECTORY)/hostkey.pem
\end{verbatim}

\item[\Macro{GSI\_DAEMON\_TRUSTED\_CA\_DIR}]
\label{param:GSIDaemonTrustedCADir} The directory that contains the
list of trusted certification authorities to be used in GSI authentication.
The files in this directory are the public keys and signing policies
of the trusted certification authorities.
If this configuration variable is not defined, and
\MacroNI{GSI\_DAEMON\_DIRECTORY} is defined, then Condor uses
\MacroNI{GSI\_DAEMON\_DIRECTORY} to construct the directory path as
\begin{verbatim}
GSI_DAEMON_TRUSTED_CA_DIR  = $(GSI_DAEMON_DIRECTORY)/certificates
\end{verbatim}

\item[\Macro{GSI\_DAEMON\_PROXY}]
\label{param:GSIDaemonProxy} A complete path and file name to the
X.509 proxy to be used in GSI authentication.
When this configuration variable is defined, use of this proxy
takes precedence over use of a certificate and key.

\item[\Macro{DELEGATE\_JOB\_GSI\_CREDENTIALS}]
\label{param:DelegateJobGSICredentials} 
A boolean value that defaults to \Expr{True} for Condor version 6.7.19
and more recent versions.
When \Expr{True}, a job's GSI X.509 credentials are delegated,
instead of being copied.
This results in a more secure communication when not encrypted.

\item[\Macro{GRIDMAP}]
\label{param:GridMap}
The complete path and file name of the Globus Gridmap file.
The Gridmap file is used to map
X.509 distinguished names to Condor user ids.

\item[\Macro{SEC\_DEFAULT\_SESSION\_DURATION}]
\label{param:SessionDuration} The amount of time in seconds before
a communication session expires.
%Defaults to 3600 seconds (1 hour).
Defaults to 8640000 seconds (100 days) to avoid a bug in session
renegotiation for Condor Version 6.6.0.
A session is a record of necessary information to do communication
between a client and daemon, and is protected by a shared secret key.
The session expires to reduce the window of opportunity where
the key may be compromised by attack.

\item[\Macro{FS\_REMOTE\_DIR}]
\label{param:FSRemoteDir}
The location of a file visible to both server and client in
Remote File System authentication.
The default when not defined is the directory 
\File{/shared/scratch/tmp}.

\item[\Macro{ENCRYPT\_EXECUTE\_DIRECTORY}]
\label{param:EncryptExecuteDirectory}
The execute directory for jobs on Windows platforms may be
encrypted by setting this configuration variable to \Expr{True}.
Defaults to \Expr{False}.
The method of encryption uses the EFS (Encrypted File System)
feature of Windows NTFS v5.

\item[\Macro{SEC\_TCP\_SESSION\_TIMEOUT}]
\label{param:SecTCPSessionTimeout}
The length of time in seconds until the timeout
when establishing a UDP security session via TCP.
The default value is 20 seconds.
Scalability issues with a large pool would be the only basis
for a change from the default value.

\item[\Macro{SEC\_PASSWORD\_FILE}]
\label{param:SecPasswordFile} For Unix machines, the path and file name
of  the file containing the pool password for password authentication.


\item[\Macro{AUTH\_SSL\_SERVER\_CAFILE}]
\label{param:AuthSSLServerCAFile}  The path and file name of
a file containing one or more trusted CA's certificates
for the server side of a communication authenticating 
with SSL.

\item[\Macro{AUTH\_SSL\_CLIENT\_CAFILE}]
\label{param:AuthSSLClientCAFile} The path and file name of
a file containing one or more trusted CA's certificates
for the client side of a communication authenticating 
with SSL.


\item[\Macro{AUTH\_SSL\_SERVER\_CADIR}]
\label{param:AuthSSLServerCADir}  
The path to a directory that may contain the 
certificates (each in its own file) for multiple trusted CAs 
for the server side of a communication authenticating 
with SSL.
When defined, the authenticating entity's certificate 
is utilized to identify the trusted CA's certificate
within the directory.

\item[\Macro{AUTH\_SSL\_CLIENT\_CADIR}]
\label{param:AuthSSLClientCADir} 
The path to a directory that may contain the 
certificates (each in its own file) for multiple trusted CAs 
for the client side of a communication authenticating 
with SSL.
When defined, the authenticating entity's certificate 
is utilized to identify the trusted CA's certificate
within the directory.


\item[\Macro{AUTH\_SSL\_SERVER\_CERTFILE}]
\label{param:AuthSSLServerCertfile}  
The path and file name of
the file containing the public certificate
for the server side of a communication authenticating
with SSL.

\item[\Macro{AUTH\_SSL\_CLIENT\_CERTFILE}]
\label{param:AuthSSLClientCertfile}
The path and file name of
the file containing the public certificate
for the client side of a communication authenticating
with SSL.


\item[\Macro{AUTH\_SSL\_SERVER\_KEYFILE}]
\label{param:AuthSSLServerKeyfile}
The path and file name of
the file containing the private key
for the server side of a communication authenticating
with SSL.

\item[\Macro{AUTH\_SSL\_CLIENT\_KEYFILE}]
\label{param:AuthSSLClientKeyfile}
The path and file name of
the file containing the private key
for the client side of a communication authenticating
with SSL.


\item[\Macro{CERTIFICATE\_MAPFILE}]
\label{param:CertificateMapfile}
A path and file name of the unified map file.

\end{description}

%%%%%%%%%%%%%%%%%%%%%%%%%%%%%%%%%%%%%%%%%%%%%%%%%%%%%%%%%%%%%%%%%%%%%%%%%%%
\subsection{\label{sec:Config-PrivSep}Configuration File Entries
Relating to PrivSep}
%%%%%%%%%%%%%%%%%%%%%%%%%%%%%%%%%%%%%%%%%%%%%%%%%%%%%%%%%%%%%%%%%%%%%%%%%%%
\index{configuration!PrivSep configuration variables}
\begin{description}
\item[\Macro{PRIVSEP\_ENABLED}]
  \label{param:PrivSepEnabled}
  A boolean variable that, when \Expr{True}, enables PrivSep.
  When \Expr{True}, the \Condor{procd} is used,
  ignoring the definition of the configuration variable \Macro{USE\_PROCD}.
  The default value when this configuration variable is not defined
  is \Expr{False}.

\item[\Macro{PRIVSEP\_SWITCHBOARD}]
  \label{param:PrivSepSwitchboard}
  The full (trusted) path and file name of the \Condor{root\_switchboard}
  executable.
\end{description}

%%%%%%%%%%%%%%%%%%%%%%%%%%%%%%%%%%%%%%%%%%%%%%%%%%%%%%%%%%%%%%%%%%%%%%%%%%%
\subsection{\label{sec:Config-VMs}Configuration File Entries
Relating to Virtual Machines}
%%%%%%%%%%%%%%%%%%%%%%%%%%%%%%%%%%%%%%%%%%%%%%%%%%%%%%%%%%%%%%%%%%%%%%%%%%%

\index{configuration!virtual machine configuration variables}
These macros affect how Condor runs \SubmitCmd{vm} universe jobs on
a matched machine within the pool.
They specify items related to the \Condor{vm-gahp}.

\begin{description}
\item[\Macro{VM\_GAHP\_SERVER}]
  \label{param:VMGAHPServer}
  The complete path and file name of the \Condor{vm-gahp}.
  There is no default value for this required configuration variable.

\item[\Macro{VM\_GAHP\_CONFIG}]
  \label{param:VMGAHPConfig}
  The complete path and file name of a separate and required
  Condor configuration file containing settings specific
  to the execution of either a VMware or Xen virtual machine.
  There is no default value for this required configuration variable.

\item[\Macro{VM\_GAHP\_LOG}]
  \label{param:VMGAHPLog}
  The complete path and file name of the \Condor{vm-gahp} log.
  If not specified on a Unix platform, the \Condor{starter}
  log will be used for \Condor{vm-gahp} log items. 
  There is no default value for this required configuration variable
  on Windows platforms.

\item[\Macro{MAX\_VM\_GAHP\_LOG}]
  \label{param:MaxVMGAHPLog}
  Controls the maximum length (in bytes) to which the \Condor{vm-gahp} log
  will be allowed to grow.

\item[\Macro{VM\_TYPE}]
  \label{param:VMType}
  Specifies the type of supported virtual machine software.
  It will be the value \verb@xen@ or \verb@vmware@.
  There is no default value for this required configuration variable.

\item[\Macro{VM\_MEMORY}]
  \label{param:VMMaxMemory}
  An integer to specify the maximum amount of memory in Mbytes
  that will be allowed to the virtual machine program.
  The amount of memory allowed will be the smaller of this
  variable and the value set by \MacroNI{VM\_MAX\_MEMORY}, as defined within
  the separate configuration file used by the \Condor{vm-gahp}.

\item[\Macro{VM\_MAX\_NUMBER}]
  \label{param:VMMaxNumber}
  An integer limit on the number of executing virtual machines.
  When not defined, the default value is the same \MacroNI{NUM\_CPUS}.

\item[\Macro{VM\_STATUS\_INTERVAL}]
  \label{param:VMStatusInterval}
  An integer number of seconds that defaults to 60,
  representing the interval between job status checks by the
  \Condor{starter} to see if the job has finished.
  A minimum value of 30 seconds is enforced.

\item[\Macro{VM\_GAHP\_REQ\_TIMEOUT}]
  \label{param:VMGAHPReqTimeout}
  An integer number of seconds that defaults to 300 (five minutes),
  representing the amount of time Condor will wait for a command issued
  from the \Condor{starter} to the \Condor{vm-gahp} to be completed.
  When a command times out, an error is reported to the \Condor{startd}.

\item[\Macro{VM\_RECHECK\_INTERVAL}]
  \label{param:VMRecheckInterval}
  An integer number of seconds that defaults to 600 (ten minutes),
  representing the amount of time the \Condor{startd} waits after a
  virtual machine error as reported by the \Condor{starter},
  and before checking a final time on the status of the virtual machine.
  If the check fails, Condor disables starting any new vm universe jobs
  by removing the \Attr{VM\_Type} attribute from the machine ClassAd.

\item[\Macro{VM\_SOFT\_SUSPEND}]
  \label{param:VMSoftSuspend}
  A boolean value that defaults to \Expr{False},
  causing Condor to free the memory of a vm universe job when
  the job is suspended.
  When \Expr{True}, the memory is not freed.

\item[\Macro{VM\_UNIV\_NOBODY\_USER}]
  \label{param:VMUnivNobodyUser}
  Identifies a login name of a user with a home directory that
  may be used for job owner of a vm universe job.
  The \Login{nobody} user normally utilized when the job arrives
  from a different UID domain will not be allowed to invoke a VMware
  virtual machine.

\item[\Macro{ALWAYS\_VM\_UNIV\_USE\_NOBODY}]
  \label{param:AlwaysVMUnivUseNobody}
  A boolean value that defaults to \Expr{False}.
  When \Expr{True}, all vm universe jobs (independent of their
  UID domain) will run as the user defined in \MacroNI{VM\_UNIV\_NOBODY\_USER}.
\end{description}

The following configuration variable may be specified in both
the Condor configuration file and in the separate
virtual machine-specific configuration file used by the \Condor{vm-gahp}.
Please note that this machine-specific configuration file 
used by the \Condor{vm-gahp} does not inherit configuration
from Condor.  Therefore, definitions may not utilize the values
of configuration variables set outside this file.
For example,
a setting of \File{\$(BIN)/condor\_vm\_vmware.pl} for
\Macro{VMWARE\_PERL} cannot work,
as \MacroUNI{BIN} is defined within Condor's configuration,
and not in the machine-specific configuration file
used by the \Condor{vm-gahp}.
Instead, \MacroNI{BIN} must be redefined,
or a full path must be used in every case.

\begin{description}
\item[\Macro{VM\_NETWORKING}]
  \label{param:VMNetworking}
  A boolean variable describing if networking is supported.
  When not defined, the default value is \Expr{False}.
  When defined in both configuration files,
  the value used is a logical AND of both values,
  implying that the value will only be \Expr{True} when
  both files define \MacroNI{VM\_NETWORKING} to be \Expr{True}.
\end{description}

The following configuration variables are not specific to either the VMware
or the Xen virtual machine software, but will appear in the
virtual machine-specific configuration file used by the \Condor{vm-gahp}.
Please note that this machine-specific configuration file 
used by the \Condor{vm-gahp} does not inherit configuration
from Condor.  Therefore, definitions may not utilize the values
of configuration variables set outside this file.
For example,
a setting of \File{\$(BIN)/condor\_vm\_vmware.pl} for
\Macro{VMWARE\_PERL} cannot work,
as \MacroUNI{BIN} is defined within Condor's configuration,
and not in the machine-specific configuration file
used by the \Condor{vm-gahp}.
Instead, \MacroNI{BIN} must be redefined,
or a full path must be used in every case.

\begin{description}

\item[\Macro{VM\_TYPE}]
  \label{param:VMType}
  Specifies the type of supported virtual machine software.
  It will be the value \verb@xen@ or \verb@vmware@.
  There is no default value for this required configuration variable.

\item[\Macro{VM\_VERSION}]
  \label{param:VMVersion}
  Specifies the version of supported virtual machine software
  defined by \MacroNI{VM\_TYPE}.  
  There is no default value for this required configuration variable.
  This configuration variable does not currently alter the behavior of
  the \Condor{vm-gahp}; instead, it is used in \Condor{status} when
  printing VM-capable hosts and slots.

\item[\Macro{VM\_MAX\_MEMORY}]
  \label{param:VMMaxMemory}
  An integer to specify the maximum amount of memory in Mbytes
  that may be used by the supported virtual machine program.
  There is no default value for this required configuration variable.

\item[\Macro{VM\_NETWORKING\_TYPE}]
  \label{param:VMNetworkingType}
  A string describing the type of networking,
  required and relevant only when \MacroNI{VM\_NETWORKING} is \Expr{True}.
  Defined strings are
  \begin{verbatim}
    bridge
    nat
    nat, bridge
  \end{verbatim}

\item[\Macro{VM\_NETWORKING\_DEFAULT\_TYPE}]
  \label{param:VMNetworkingDefaultType}
  Where multiple networking types are given in \MacroNI{VM\_NETWORKING\_TYPE},
  this optional configuration variable identifies which to use.
  Therefore, for 
  \begin{verbatim}
  VM_NETWORKING_TYPE = nat, bridge
  \end{verbatim}
  this variable may be defined as either \Expr{nat} or \Expr{bridge}.
  Where multiple networking types are given in \MacroNI{VM\_NETWORKING\_TYPE},
  and this variable is \emph{not} defined, a default of \Expr{nat}
  is used.

\end{description}

The following configuration variables are specific to the VMware
virtual machine software.  They are specified within the separate configuration
file read by the \Condor{vm-gahp} and allow the \Condor{vm-gahp} to
correctly interface with the VMware software.
Please note that this machine-specific configuration file 
used by the \Condor{vm-gahp} does not inherit configuration
from Condor.  Therefore, definitions may not utilize the values
of configuration variables set outside this file.
For example,
a setting of \File{\$(BIN)/condor\_vm\_vmware.pl} for
\Macro{VMWARE\_PERL} cannot work,
as \MacroUNI{BIN} is defined within Condor's configuration,
and not in the machine-specific configuration file
used by the \Condor{vm-gahp}.
Instead, \MacroNI{BIN} must be redefined,
or a full path must be used in every case.

\begin{description}
\item[\Macro{VMWARE\_PERL}]
  \label{param:VMwarePerl}
  The complete path and file name to \Prog{Perl}.
  There is no default value for this required variable.

\item[\Macro{VMWARE\_SCRIPT}]
  \label{param:VMwareScript}
  The complete path and file name of the script that controls VMware.
  There is no default value for this required variable.

\item[\Macro{VMWARE\_NETWORKING\_TYPE}]
  \label{param:VMwareNetworkingType}
  An optional string used in networking that the \Condor{vm-gahp}
  inserts into the VMware configuration file to define a networking type.
  Defined types are \Expr{nat} or \Expr{bridged}.
  If a default value is needed, the inserted string will be \Expr{nat}.

\item[\Macro{VMWARE\_NAT\_NETWORKING\_TYPE}]
  \label{param:VMwareNatNetworkingType}
  An optional string used in networking that the \Condor{vm-gahp}
  inserts into the VMware configuration file to define a networking type.
  If nat networking is used, this variable's definition takes
  precedence over one defined by \MacroNI{VMWARE\_NETWORKING\_TYPE}.

\item[\Macro{VMWARE\_BRIDGE\_NETWORKING\_TYPE}]
  \label{param:VMwareBridgeNetworkingType}
  An optional string used in networking that the \Condor{vm-gahp}
  inserts into the VMware configuration file to define a networking type.
  If bridge networking is used, this variable's definition takes
  precedence over one defined by \MacroNI{VMWARE\_NETWORKING\_TYPE}.

\end{description}

The following configuration variables are specific to the Xen
virtual machine software.  They are specified within the separate configuration
file read by the \Condor{vm-gahp} and allow the \Condor{vm-gahp} to
correctly interface with the Xen software.
Please note that this machine-specific configuration file 
used by the \Condor{vm-gahp} does not inherit configuration
from Condor.  Therefore, definitions may not utilize the values
of configuration variables set outside this file.
For example,
a setting of \File{\$(BIN)/condor\_vm\_vmware.pl} for
\Macro{VMWARE\_PERL} cannot work,
as \MacroUNI{BIN} is defined within Condor's configuration,
and not in the machine-specific configuration file
used by the \Condor{vm-gahp}.
Instead, \MacroNI{BIN} must be redefined,
or a full path must be used in every case.

\begin{description}

% \item[\Macro{VM\_HARDWARE\_VT}]
% \label{param:VMHardwareVT}
% A boolean variable describing whether a machine supports
% hardware virtualization such as Intel VT and AMD-V.
% When not defined, the default value is \Expr{False}.

\item[\Macro{XEN\_SCRIPT}]
  \label{param:XenScript}
  The complete path and file name of the script that controls Xen.
  There is no default value for this required variable.

\item[\Macro{XEN\_DEFAULT\_KERNEL}]
  \label{param:XenDefaultKernel}
  The complete path and executable name of a Xen kernel to be utilized
  if the job's submission does not specify its own kernel image.

\item[\Macro{XEN\_DEFAULT\_INITRD}]
  \label{param:XenDefaultInitrd}
  The complete path and image file name for the initrd image,
  if used with the default kernel image.

\item[\Macro{XEN\_BOOTLOADER}]
  \label{param:XenBootloader}
  A required full path and executable for the Xen bootloader,
  if the kernel image includes a disk image.

\item[\Macro{XEN\_CONTROLLER}]
  \label{param:XenController}
  A required variable that will be set to either
  \Expr{xm} or \Expr{virsh},
  specifying whether the Xen hypervisor is 
  controlled by the \Prog{xm} or \Prog{virsh} program.
  \Prog{xm} is part of the Xen distribution,
  while \Prog{virsh} is part of the libvirt library.
  These controllers both provide the same functionality 
  with respect to the Xen hypervisor, but with different configurations.
  The differences are: \Prog{xm}
  requires the \MacroNI{XEN\_IMAGE\_IO\_TYPE} configuration variable
  to be defined; 
  while, \Prog{virsh} uses the \MacroNI{XEN\_BRIDGE\_SCRIPT} 
  configuration variable to set up a bridged network,
  does not need the \MacroNI{XEN\_IMAGE\_IO\_TYPE} configuration variable,
  and the configuration variable
  \MacroNI{XEN\_VIF\_PARAMETER} is irrelevant.

\item[\Macro{XEN\_VIF\_PARAMETER}]
  \label{param:XenVIFParameter}
  An optional string used in networking that the \Condor{vm-gahp}
  inserts into the Xen configuration file for a vif parameter.
  If a default value is needed, the inserted string will be
  \begin{verbatim}
  vif = ['']
  \end{verbatim}

\item[\Macro{XEN\_NAT\_VIF\_PARAMETER}]
  \label{param:XenNatVIFParameter}
  An optional string used in networking that the \Condor{vm-gahp}
  inserts into the Xen configuration file for a vif parameter.
  If nat networking is used, this variable's definition takes
  precedence over one defined by \MacroNI{XEN\_VIF\_PARAMETER}.

\item[\Macro{XEN\_BRIDGE\_VIF\_PARAMETER}]
  \label{param:XenBridgeVIFParameter}
  An optional string used in networking that the \Condor{vm-gahp}
  inserts into the Xen configuration file for a vif parameter.
  If bridge networking is used, this variable's definition takes
  precedence over one defined by \MacroNI{XEN\_VIF\_PARAMETER}.

\item[\Macro{XEN\_IMAGE\_IO\_TYPE}]
  \label{param:XenImageIOType}
  An optional string that defines a file I/O device.
  For Xen checkpoints, all machines with a shared file system must
  use the same file I/O type.
  When not defined, the default value uses a loopback device:
  \begin{verbatim}
  XEN_IMAGE_IO_TYPE = file:
  \end{verbatim}

\item[\Macro{XEN\_BRIDGE\_SCRIPT}]
  \label{param:XenBridgeScript}
  A path, file name, and command-line arguments to specify a script
  that will be run to set up a bridging network interface for guests.
  The interface should provide direct access to the host system's LAN,
  that is, not be NAT'd on the host.
  An example:
  \begin{verbatim}
  XEN_BRIDGE_SCRIPT = vif-bridge bridge=xenbr0
  \end{verbatim}

% \item[\Macro{XEN\_VT\_KERNEL}]
%   \label{param:XenVTKernel}
%  An optional string that the \Condor{vm-gahp}
%  inserts into the Xen configuration file, relating to hardware virtualization.
%  As an example, the string inserted is
%  \begin{verbatim}
%  kernel="/usr/lib/xen/boot/hvmloader"
%  \end{verbatim}
%  for a configuration variable definition of
%  \begin{verbatim}
%  XEN_VT_KERNEL = "/usr/lib/xen/boot/hvmloader"
%  \end{verbatim}

%\item[\Macro{XEN\_VT\_BUILDER}]
%  \label{param:XenVTBuilder}
%  An optional string that the \Condor{vm-gahp}
%  inserts into the Xen configuration file, relating to hardware virtualization.
%  As an example, the string inserted is
%  \begin{verbatim}
%  builder='hvm'
%  \end{verbatim}
%  for a configuration variable definition of
%  \begin{verbatim}
%  XEN_VT_BUILDER = 'hvm'
%  \end{verbatim}

%\item[\Macro{XEN\_VT\_SHADOW\_MEMORY}]
%  \label{param:XenVTShadowMemory}
%  An optional string that the \Condor{vm-gahp}
%  inserts into the Xen configuration file, relating to hardware virtualization.
%  As an example, the string inserted is
%  \begin{verbatim}
%  shadow_memory=8
%  \end{verbatim}
%  for a configuration variable definition of
%  \begin{verbatim}
%  XEN_VT_SHADOW_MEMORY = 8
%  \end{verbatim}

%\item[\Macro{XEN\_VT\_DEVICE\_MODEL}]
%  \label{param:XenVTDeviceModel}
%  An optional string that the \Condor{vm-gahp}
%  inserts into the Xen configuration file, relating to hardware virtualization.
%  As an example, the string inserted is
%  \begin{verbatim}
%  device_model=/usr/lib/xen/bin/qemu-dm
%  \end{verbatim}
%  for a configuration variable definition of
%  \begin{verbatim}
%  XEN_VT_DEVICE_MODEL = '/usr/lib/xen/bin/qemu-dm'
%  \end{verbatim}

%\item[\Macro{XEN\_VT\_SDL}]
%  \label{param:XenVTSDL}
%  An optional string that the \Condor{vm-gahp}
%  inserts into the Xen configuration file, relating to hardware virtualization.
%  As an example, the string inserted is
%  \begin{verbatim}
%  sdl=0
%  \end{verbatim}
%  for a configuration variable definition of
%  \begin{verbatim}
%  XEN_VT_SDL = 0
%  \end{verbatim}

%\item[\Macro{XEN\_VT\_VNC}]
%  \label{param:XenVTVNC}
%  An optional string that the \Condor{vm-gahp}
%  inserts into the Xen configuration file, relating to hardware virtualization.
%  As an example, the string inserted is
%  \begin{verbatim}
%  vnc=1
%  \end{verbatim}
%  for a configuration variable definition of
%  \begin{verbatim}
%  XEN_VT_VNC = 1
%  \end{verbatim}

%\item[\Macro{XEN\_VT\_STDVGA}]
%  \label{param:XenVTSTDVGA}
%  An optional string that the \Condor{vm-gahp}
%  inserts into the Xen configuration file, relating to hardware virtualization.
%  As an example, the string inserted is
%  \begin{verbatim}
%  stdvga=0
%  \end{verbatim}
%  for a configuration variable definition of
%  \begin{verbatim}
%  XEN_VT_STDVGA = 0
%  \end{verbatim}

%\item[\Macro{XEN\_VT\_SERIAL}]
%  \label{param:XenVTSerial}
%  An optional string that the \Condor{vm-gahp}
%  inserts into the Xen configuration file, relating to hardware virtualization.
%  As an example, the string inserted is
%  \begin{verbatim}
%  serial='pty'
%  \end{verbatim}
%  for a configuration variable definition of
%  \begin{verbatim}
%  XEN_VT_SERIAL = 'pty'
%  \end{verbatim}

%\item[\Macro{XEN\_DEVICE\_TYPE\_FOR\_VT}]
%  \label{param:XenDeviceTypeForVT}
%  Defaults to the string \verb@vbd:@ for paravirtualized guests.
%  Use the string \verb@ioemu:@ for unmodified guest domains. 

%\item[\Macro{XEN\_ALLOW\_HARDWARE\_VT\_SUSPEND}]
%  \label{param:XenAllowHardwareVTSuspend}
%  A boolean value that defaults to \Expr{False}.
%  In the future, as Xen hardware virtualization guests may be saved
%  and restored, this configuration variable may be utilized.

\end{description}

The following two macros affect the configuration of Condor where Condor is
running on a host machine, the host machine is running an
inner virtual machine,
and Condor is also running on that inner virtual machine.
These two variables have nothing to do with the \SubmitCmd{vm}
universe.

\begin{description}
\item[\Macro{VMP\_HOST\_MACHINE}]
\label{param:VMPHostMachine}
A configuration variable for the inner virtual machine,
which specifies the host name.

\item[\Macro{VMP\_VM\_LIST}]
\label{param:VMPVMList}
For the host, 
a comma separated list of the host names or IP addresses
for machines running inner virtual machines on a host.
\end{description}

%%%%%%%%%%%%%%%%%%%%%%%%%%%%%%%%%%%%%%%%%%%%%%%%%%%%%%%%%%%%%%%%%%%%%%%%%%%
\subsection{\label{sec:HA-Config-File-Entries}Configuration File Entries
Relating to High Availability}
%%%%%%%%%%%%%%%%%%%%%%%%%%%%%%%%%%%%%%%%%%%%%%%%%%%%%%%%%%%%%%%%%%%%%%%%%%%

\index{configuration!high availability configuration variables}
These macros affect the high availability operation of Condor.

\begin{description}
\item[\Macro{MASTER\_HA\_LIST}] \label{param:MasterHAList} Similar to
  \MacroNI{DAEMON\_LIST}, this macro defines a list of daemons that
  the \Condor{master} starts and keeps its watchful eyes on.
  However, the \MacroNI{MASTER\_HA\_LIST} daemons are run in a
  \emph{High Availability} mode.
  The list is a comma or space separated list of subsystem names
  (as listed in section~\ref{sec:Condor-Subsystem-Names}).
  For example,
  \begin{verbatim}
        MASTER_HA_LIST = SCHEDD
  \end{verbatim}

  The \emph{High Availability} feature allows for several \Condor{master}
  daemons (most likely on separate machines) to work together to
  insure that a particular service stays available.  These
  \Condor{master} daemons ensure that one and only one of them will
  have the listed daemons running.

  To use this feature, the lock URL must be set with
  \MacroNI{HA\_LOCK\_URL}.

  Currently, only file URLs are supported 
  (those with \File{file:\Dots}).
  The default value for \MacroNI{MASTER\_HA\_LIST} is 
  the empty string, which disables the feature.
  
\item[\Macro{HA\_LOCK\_URL}] \label{param:HALockURL} This macro
  specifies the URL that the \Condor{master} processes use to
  synchronize for the \emph{High Availability} service.
  Currently, only file URLs are supported; for example,
  \File{file:/share/spool}.  Note that this URL must be identical
  for all \Condor{master} processes sharing this resource.  For
  \Condor{schedd} sharing, we recommend setting up \MacroNI{SPOOL}
  on an NFS share and having all \emph{High Availability}
  \Condor{schedd} processes sharing it,
  and setting the \MacroNI{HA\_LOCK\_URL} to point at this directory
  as well.  For example:
\begin{verbatim}
        MASTER_HA_LIST = SCHEDD
        SPOOL = /share/spool
        HA_LOCK_URL = file:/share/spool
        VALID_SPOOL_FILES = SCHEDD.lock
\end{verbatim}

  A separate lock is created for each \emph{High Availability} daemon.

  There is no default value for \MacroNI{HA\_LOCK\_URL}.

  Lock files are in the form \verb@<@SUBSYS\verb@>@.lock.
  \Condor{preen} is not currently aware of the lock files and will
  delete them if they are placed in the \MacroNI{SPOOL} directory, so be
  sure to add \verb@<@SUBSYS\verb@>@.lock to \Macro{VALID\_SPOOL\_FILES} for each
  \emph{High Availability} daemon.

\item[\Macro{HA\_<SUBSYS>\_LOCK\_URL}]
  \label{param:HASubsysLockURL} This macro controls the 
  \emph{High Availability} lock URL for a specific subsystem
  as specified in the configuration variable name,
  and it overrides the system-wide lock URL specified by
  \MacroNI{HA\_LOCK\_URL}.  If not defined for each subsystem,
  \MacroNI{HA\_<SUBSYS>\_LOCK\_URL} is ignored, and the value of
  \MacroNI{HA\_LOCK\_URL} is used.

\item[\Macro{HA\_LOCK\_HOLD\_TIME}] \label{param:HALockHoldTime}
  This macro
  specifies the number of seconds that the \Condor{master} will hold the
  lock for each \emph{High Availability} daemon.
  Upon gaining the shared lock,
  the \Condor{master} will hold the lock for this number of seconds.
  Additionally, the \Condor{master} will periodically renew
  each lock as long as the \Condor{master} and the daemon are running.
  When the daemon dies, or the \Condor{master} exists, the
  \Condor{master} will immediately release the lock(s) it holds.

  \MacroNI{HA\_LOCK\_HOLD\_TIME} defaults to 3600 seconds (one hour).

\item[\Macro{HA\_<SUBSYS>\_LOCK\_HOLD\_TIME}]
  \label{param:HASubsysLockHoldTime}
  This macro controls the \emph{High Availability} lock
  hold time for a specific subsystem
  as specified in the configuration variable name,
  and it overrides the system wide poll period specified by
  \MacroNI{HA\_LOCK\_HOLD\_TIME}.
  If not defined for each subsystem,
  \MacroNI{HA\_<SUBSYS>\_LOCK\_HOLD\_TIME} is ignored,
  and the value of \MacroNI{HA\_LOCK\_HOLD\_TIME} is used.

\item[\Macro{HA\_POLL\_PERIOD}] \label{param:HALockPollPeriod} 
  This macro specifies how often the \Condor{master} polls the
  \emph{High Availability} locks to see if any locks are either stale
  (meaning not updated for \MacroNI{HA\_LOCK\_HOLD\_TIME} seconds),
  or have been released by the owning \Condor{master}.
  Additionally, the \Condor{master} renews any locks that it
  holds during these polls.

  \MacroNI{HA\_POLL\_PERIOD} defaults to 300 seconds (five minutes).

\item[\Macro{HA\_<SUBSYS>\_POLL\_PERIOD}]
  \label{param:HALockPollSubsysPeriod}
  This macro controls the \emph{High Availability} poll period
  for a specific subsystem
  as specified in the configuration variable name,
  and it overrides the system wide poll period specified by
  \MacroNI{HA\_POLL\_PERIOD}.
  If not defined for each subsystem,
  \MacroNI{HA\_<SUBSYS>\_POLL\_PERIOD} is ignored,
  and the value of \MacroNI{HA\_POLL\_PERIOD} is used.

\item[\Macro{MASTER\_<SUBSYS>\_CONTROLLER}]
  \label{param:MasterSubsysController} Used only in HA configurations
  involving the \Condor{had}.

  The \Condor{master} has the concept of a controlling and controlled
  daemon, typically
  with the \Condor{had} daemon serving as the controlling process.
  In this case, all \Condor{on} and \Condor{off} commands directed
  at controlled daemons are given to the controlling daemon, which
  then handles the command, and, when required, sends appropriate
  commands to the \Condor{master} to do the actual work.  This allows
  the controlling daemon to know the state of the controlled daemon.

  As of 6.7.14, this configuration variable must be specified for all
  configurations using \Condor{had}.
  To configure the \Condor{negotiator} controlled by \Condor{had}:

\begin{verbatim}
MASTER_NEGOTIATOR_CONTROLLER = HAD
\end{verbatim}

  The macro is named by substituting \MacroNI{<SUBSYS>}
  with the appropriate subsystem string as defined in
  section~\ref{sec:Condor-Subsystem-Names}.


\item[\Macro{HAD\_LIST}]
  \label{param:HADList}
  A comma-separated list of all \Condor{had} daemons
  in the form \Expr{IP:port} or \Expr{hostname:port}.
  Each central manager machine that runs the \Condor{had} daemon
  should appear in this list.
  If \MacroNI{HAD\_USE\_PRIMARY} is set to \Expr{True},
  then the first machine in this list is the primary central
  manager, and all others in the list are backups.

  All central manager machines must be configured with 
  an identical \MacroNI{HAD\_LIST}.
  The machine addresses are identical to the addresses defined
  in \MacroNI{COLLECTOR\_HOST}.

%The following examples are all valid HAD\_LIST declarations: 
%
%HAD\_LIST =<132.68.37.104:10001>,<132.68.37.105:10002>,<132.68.37.106:1045>
%
%HAD\_LIST =132.68.37.104:10001,132.68.37.105:10002,132.68.37.106:1045
%
%HAD\_LIST=ds-r3.cs.technion.ac.il:10001,ds-r3.cs.technion.ac.il:10002,ds-r3.cs.technion.ac.il:1045
%

\item[\Macro{HAD\_USE\_PRIMARY}]
  \label{param:HADUsePrimary}
  Boolean value to determine if the first machine in the 
  \MacroNI{HAD\_LIST} configuration variable is
  a primary central manager.
  Defaults to \Expr{False}.


\item[\Macro{HAD\_CONNECTION\_TIMEOUT}]
  \label{param:HADConnectionTimeout}
  The time (in seconds) that the \Condor{had} daemon waits before giving
  up on the establishment of a TCP connection.
  The failure of the communication connection
  is the detection mechanism for the failure of a central
  manager machine.
  For a LAN, a recommended value is 2 seconds.
  The use of authentication (by Condor) increases the connection
  time.
  The default value is 5 seconds.
  If this value is set too low,
  \Condor{had} daemons will incorrectly assume
  the failure of other machines.

\item[\Macro{HAD\_ARGS}]
  \label{param:HADArgs}
  Command line arguments passed by the \Condor{master} daemon
  as it invokes the \Condor{had} daemon.
  To make high availability work, the \Condor{had} daemon
  requires the port number it is to use.
  This argument is of the form
  \begin{verbatim}
   -p $(HAD_PORT_NUMBER)
  \end{verbatim}
  where \MacroNI{HAD\_PORT\_NUMBER} is a helper configuration variable
  defined with the desired port number.
  Note that this port number must be the same value here as
  used in \MacroNI{HAD\_LIST}.
  There is no default value.


\item[\Macro{HAD}]
  \label{param:HAD}
  The path to the \Condor{had} executable. Normally it is defined
  relative to \MacroUNI{SBIN}.
  This configuration variable has no default value.

\item[\Macro{MAX\_HAD\_LOG}]
  \label{param:MaxHADLog}
  Controls the maximum length in bytes to which the \Condor{had}
  daemon log will be allowed to grow. It will grow to the specified length,
  then be saved to a file with the suffix \File{.old}. 
  The \File{.old}  file is overwritten each time the log is saved,
  thus the maximum space devoted to logging is twice the maximum length
  of this log file.
  A value of 0 specifies that this file may grow without bounds.
  The default is 1 Mbyte.

\item[\Macro{HAD\_DEBUG}]
  \label{param:HADDebug}
  Logging level for the \Condor{had} daemon.
  See \MacroNI{<SUBSYS>\_DEBUG} for values.

\item[\Macro{HAD\_LOG}]
  \label{param:HADLog}
  Full path and file name of the log file.
  There is no default value.

\item[\Macro{REPLICATION\_LIST}]
  \label{param:ReplicationList}
  A comma-separated list of all \Condor{replication} daemons
  in the form \Expr{IP:port} or \Expr{hostname:port}.
  Each central manager machine that runs the \Condor{had} daemon
  should appear in this list.
  All potential central manager machines must be configured with
  an identical \MacroNI{REPLICATION\_LIST}.

\item[\Macro{STATE\_FILE}]
  \label{param:StateFile}
  A full path and file name of the file protected by the replication
  mechanism.
  When not defined, the default path and file used is
  \begin{verbatim}
  $(SPOOL)/Accountantnew.log
  \end{verbatim}

\item[\Macro{REPLICATION\_INTERVAL}]
  \label{param:ReplicationInterval}
  Sets how often the \Condor{replication} daemon initiates its tasks of
  replicating the \MacroUNI{STATE\_FILE}.
  It is defined in seconds and defaults to 300 (5 minutes).
  This is the same as the default \MacroNI{NEGOTIATOR\_INTERVAL}.

\item[\Macro{MAX\_TRANSFER\_LIFETIME}]
  \label{param:MaxTransferLifetime}
  A timeout period within which the process that
  transfers the state file must complete its transfer.
  The recommended value is
  \Expr{2 * average size of state file / network rate}.
  It is defined in seconds and defaults to 300 (5 minutes).

\item[\Macro{HAD\_UPDATE\_INTERVAL}]
  \label{param:HADUpdateInterval}
  Like \MacroNI{UPDATE\_INTERVAL},
  determines how often the \Condor{had} is to send a ClassAd update
  to the \Condor{collector}.
  Updates are also sent at each and every change in state.
  It is defined in seconds and defaults to 300 (5 minutes).

\item[\Macro{HAD\_USE\_REPLICATION}]
  \label{param:HADUseReplication}
  A boolean value that defaults to \Expr{False}.
  When \Expr{True}, the use of \Condor{replication} daemons is enabled.

\item[\Macro{REPLICATION\_ARGS}]
  \label{param:ReplicationArgs}
  Command line arguments passed by the \Condor{master} daemon
  as it invokes the \Condor{replication} daemon.
  To make high availability work, the \Condor{replication} daemon
  requires the port number it is to use.
  This argument is of the form
  \begin{verbatim}
  -p $(REPLICATION_PORT_NUMBER)
  \end{verbatim}
  where \MacroNI{REPLICATION\_PORT\_NUMBER} is a helper configuration
  variable defined with the desired port number.
  Note that this port number must be the same value as
  used in \MacroNI{REPLICATION\_LIST}.
  There is no default value.

\item[\Macro{REPLICATION}]
  \label{param:Replication}
  The full path and file name of the \Condor{replication} executable.
  It is normally defined relative to \MacroUNI{SBIN}.
  There is no default value.

\item[\Macro{MAX\_REPLICATION\_LOG}]
  \label{param:MaxReplicationLog}
  Controls the maximum length in bytes to which the \Condor{replication}
  daemon log will be allowed to grow. It will grow to the specified length,
  then be saved to a file with the suffix \File{.old}.
  The \File{.old}  file is overwritten each time the log is saved,
  thus the maximum space devoted to logging is twice the maximum length
  of this log file.
  A value of 0 specifies that this file may grow without bounds.
  The default is 1 Mbyte.

\item[\Macro{REPLICATION\_DEBUG}]
  \label{param:ReplicationDebug}
  Logging level for the \Condor{replication} daemon.
  See \MacroNI{<SUBSYS>\_DEBUG} for values.


\item[\Macro{REPLICATION\_LOG}]
  \label{param:ReplicationLog}
  Full path and file name to the log file.
  There is no default value.

\end{description}


%%%%%%%%%%%%%%%%%%%%%%%%%%%%%%%%%%%%%%%%%%%%%%%%%%%%%%%%%%%%%%%%%%%%%%%%%%%
\subsection{\label{sec:Quill-Config-File-Entries}Configuration File
Entries Relating to Quill}
%%%%%%%%%%%%%%%%%%%%%%%%%%%%%%%%%%%%%%%%%%%%%%%%%%%%%%%%%%%%%%%%%%%%%%%%%%%

\index{configuration!Quill configuration variables}
These macros affect the Quill database
management and interface to its representation of the job queue.

\begin{description}
\item[\Macro{QUILL}] \label{param:Quill}  The full path name to the
  \Condor{quill} daemon.

\item[\Macro{QUILL\_ARGS}] \label{param:QuillArgs} Arguments
  to be passed to the \Condor{quill} daemon upon its invocation.

\item[\Macro{QUILL\_LOG}] \label{param:QuillLog}
  Path to the Quill daemon's log file.

\item[\Macro{QUILL\_ENABLED}] \label{param:QuillEnabled}
  A boolean variable that defaults to \Expr{False}.
  When \Expr{True}, Quill functionality is enabled.
  When \Expr{False}, the Quill daemon writes a message to its log and exits.
  The \Condor{q} and \Condor{history} tools then do not use Quill.

\item[\Macro{QUILL\_NAME}] \label{param:QuillName}
  A string that uniquely identifies an instance of the \Condor{quill}
  daemon, as there be more than \Condor{quill} daemon per pool.
  The string must not be the same as for any \Condor{schedd} daemon.
  A convenient definition to choose is of the form
\footnotesize
\begin{verbatim}
quill-for-schedd_name@machinename.fully.qualified.address
\end{verbatim}
\normalsize

\item[\Macro{QUILL\_USE\_SQL\_LOG}] \label{param:QuillUseSQLLog}
  In order for Quill to store historical job information or resource
  information, the Condor daemons must write information to the SQL logfile.
  By default, this is set to \Expr{False}, and the only information Quill
  stores in the database is the current job queue.
  This can be set on a per daemon basis. For example, to store information
  about historical jobs, but not store execute resource information, set
  \MacroNI{QUILL\_USE\_SQL\_LOG} to \Expr{False} and set
  \MacroNI{SCHEDD.\_QUILL\_USE\_SQL\_LOG} to \Expr{True}.

\item[\Macro{QUILL\_DB\_NAME}] \label{param:QuillDBName}
  A string that identifies a database within a database server.

\item[\Macro{QUILL\_DB\_USER}]
	A string that identifies the \Prog{PostgreSQL} user that Quill will
    connect to the database as.
	We recommend \Username{quillwriter} for this setting. 

\item[\Macro{QUILL\_DB\_TYPE}] \label{param:QuillDBType}
  A string that distinguishes between database system types.
  Defaults to the only database system currently defined,
  \verb@"PGSQL"@.

\item[\Macro{QUILL\_DB\_IP\_ADDR}] \label{param:QuillDBIPAddr}
  The host address of the database server. It can be either an IP address
  or an IP address.
  It must match exactly what is used in the \File{.pgpass} file.

\item[\Macro{QUILL\_POLLING\_PERIOD}] \label{param:QuillPollingPeriod}
  The frequency, in number of seconds, at which the Quill daemon
  polls the file \File{job\_queue.log} for updates.
  New information in the log file is sent to the database.
  The default value is 10.

\item[\Macro{QUILL\_NOT\_RESPONDING\_TIMEOUT}]
  \label{param:QuillNotRespondingTimeout}
  The length of time, in seconds, before the \Condor{master}
  may decide that the \Condor{quill} daemon is hung due to 
  a lack of communication,
  potentially causing  the \Condor{master} to kill and
  restart the \Condor{quill} daemon.
  When the \Condor{quill} daemon is processing a very long log file, it 
  may not be able to communicate with the master. 
  The default is 3600 seconds, or one hour. It may be
  advisable to increase this to several hours. 

\item[\Macro{QUILL\_MAINTAIN\_DB\_CONN}] \label{param:QuillMaintainDBConn}
  A boolean variable that defaults to \Expr{True}.
  When \Expr{True}, the \Condor{quill} daemon
  maintains an open connection the database server,
  which speeds up updates to the database.
  As each open connection consumes resources at the database server,
  we recommend a setting of \Expr{False} for large pools.

\item[\Macro{DATABASE\_PURGE\_INTERVAL}] 
  \label{param:QuillDatabasePurgeInterval}
  The interval, in seconds, between scans of the database to identify and
  delete records that are beyond their history durations. 
  The default value is 86400, or one day.

\item[\Macro{DATABASE\_REINDEX\_INTERVAL}] 
  \label{param:QuillDatabaseReindexInterval}
  The interval, in seconds, between reindex commands on the database.
  The default value is 86400, or one day.
  This is only used when the \MacroNI{QUILL\_DB\_TYPE} is set to
  \verb@"PGSQL"@.

\item[\Macro{QUILL\_JOB\_HISTORY\_DURATION}]
  \label{param:QuillJobHistoryDuration}
  The number of days after entry into the database that a job will
  remain in the database.
  After \MacroNI{QUILL\_JOB\_HISTORY\_DURATION} days, the job is deleted.
  The job history is the final ClassAd, and contains all information 
  necessary for \Condor{history} to succeed.
  The default is 3650, or about 10 years. 

\item[\Macro{QUILL\_RUN\_HISTORY\_DURATION}]
  \label{param:QuillRunHistoryDuration}
  The number of days after entry into the database that extra information 
  about the job will remain in the database.
  After \MacroNI{QUILL\_RUN\_HISTORY\_DURATION} days, the records are deleted.
  This data includes matches made for the job, file transfers the job 
  performed, and user log events.
  The default is 7 days, or one week. 

\item[\Macro{QUILL\_RESOURCE\_HISTORY\_DURATION}]
  \label{param:QuillResourceHistoryDuration}
  The number of days after entry into the database that a resource record will
  remain in the database.
  After \MacroNI{QUILL\_RESOURCE\_HISTORY\_DURATION} days, the record is 
  deleted.
  The resource history data includes the ClassAd of a compute slot,
  submitter ClassAds, and daemon ClassAds.
  The default is 7 days, or one week. 

\item[\Macro{QUILL\_DBSIZE\_LIMIT}] \label{param:QuillDBSizeLimit}
  After each purge, the \Condor{quill} daemon estimates 
  the size of the database. 
  If the size of the database exceeds this limit, 
  the \Condor{quill} daemon will e-mail the administrator a warning. 
  This size is given in gigabytes, and defaults to 20. 

\item[\Macro{QUILL\_MANAGE\_VACUUM}]
  \label{param:QuillManageVacuum}
  A boolean value that defaults to \Expr{False}.
  When \Expr{True}, the \Condor{quill} daemon takes on 
  the maintenance task of vacuuming the database.
  As of \Prog{PostgreSQL} version 8.1, the database
  can perform this task automatically; 
  therefore, having the \Condor{quill} daemon vacuum is not necessary.
  A value of \Expr{True} causes warnings to be written to the log file.

\item[\Macro{QUILL\_SHOULD\_REINDEX}]
  \label{param:QuillShouldReindex}
  A boolean value that defaults to \Expr{True}.
  When \Expr{True}, the \Condor{quill} daemon will re-index the database
  tables when the history file is purged of old data. So, if Quill is
  configured to never delete history data, the tables are never re-indexed.

\item[\Macro{QUILL\_IS\_REMOTELY\_QUERYABLE}]
  \label{param:QuillIsRemotelyQueryable}
  A boolean value that defaults to \Expr{True}.
  When \Expr{False}, the remote database tables may not be remotely
  queryable.

\item[\Macro{QUILL\_DB\_QUERY\_PASSWORD}] \label{param:QuillDBQueryPassword}
  Defines the password string needed by \Condor{q} to gain read
  access for remotely querying the Quill database.

\item[\Macro{QUILL\_ADDRESS\_FILE}] \label{param:QuillAddressFile}
  When defined, it specifies the path and file name of a local file
  containing the IP address and port number of the Quill daemon.
  By using the file, tools executed on the local machine do not need
  to query the central manager in order to find the \Condor{quill} daemon.

\item[\Macro{DBMSD}] \label{param:DBMSD} 
  The full path name to the \Condor{dbmsd} daemon.
  The default location is \File{\$(SBIN)/condor\_dbmsd}.

\item[\Macro{DBMSD\_ARGS}] \label{param:DBMSDArgs} Arguments
  to be passed to the \Condor{dbmsd} daemon upon its invocation.
  The default arguments are \verb@-f@.

\item[\Macro{DBMSD\_LOG}] \label{param:DBMSDLog}
  Path to the \Condor{dbmsd} daemon's log file.
  The default log location is \File{\$(LOG)/DbmsdLog}.

\item[\Macro{DBMSD\_NOT\_RESPONDING\_TIMEOUT}]
  The length of time, in seconds, before the \Condor{master}
  may decide that the \Condor{dbmsd} is hung due to a lack of communication,
  potentially causing  the \Condor{master} to kill and
  restart the \Condor{dbmsd} daemon.
  When the \Condor{dbmsd} is purging or reindexing a very large database, it 
  may not be able to communicate with the master. 
  The default is 3600 seconds, or one hour. It may be
  advisable to increase this to several hours. 

\end{description}



%%%%%%%%%%%%%%%%%%%%%%%%%%%%%%%%%%%%%%%%%%%%%%%%%%%%%%%%%%%%%%%%%%%%%%
\subsection{\label{sec:MyProxy-Config-File-Entries}MyProxy
Configuration File Macros}
%%%%%%%%%%%%%%%%%%%%%%%%%%%%%%%%%%%%%%%%%%%%%%%%%%%%%%%%%%%%%%%%%%%%%%
 
In some cases, Condor can autonomously refresh GSI certificate proxies
via \Prog{MyProxy}, available from
\URL{http://myproxy.ncsa.uiuc.edu/}.

\begin{description}

\item[\Macro{MYPROXY\_GET\_DELEGATION}]
\label{param:MyProxyGetDelegation}  The full path name to the
\Prog{myproxy-get-delegation} executable, installed as part of the
\Prog{MyProxy} software.  Often, it is necessary to wrap the actual
executable with a script that sets the environment, such as the
\MacroNI{LD\_LIBRARY\_PATH}, correctly.  If this macro is defined,
Condor-G and \Condor{credd} will have the capability to autonomously
refresh proxy certificates.  By default, this macro is undefined.

\end{description}

%%%%%%%%%%%%%%%%%%%%%%%%%%%%%%%%%%%%%%%%%%%%%%%%%%%%%%%%%%%%%%%%%%%%%%
\subsection{\label{sec:API-Config-File-Entries}
Configuration File Macros Affecting APIs}
%%%%%%%%%%%%%%%%%%%%%%%%%%%%%%%%%%%%%%%%%%%%%%%%%%%%%%%%%%%%%%%%%%%%%%

\begin{description}

\item[\Macro{ENABLE\_SOAP}]
  \label{param:EnableSoap}
  A boolean value that defaults to \Expr{False}.
  When \Expr{True}, Condor daemons will respond to HTTP PUT commands
  as if they were SOAP calls. When \Expr{False},
  all HTTP PUT commands are denied.

\item[\Macro{ENABLE\_WEB\_SERVER}]
  \label{param:EnableWebServer}
  A boolean value that defaults to \Expr{False}.
  When \Expr{True}, Condor daemons will respond to HTTP GET commands,
  and send the static files sitting in the subdirectory defined
  by the configuration variable \MacroNI{WEB\_ROOT\_DIR}.
  In addition, web commands are considered a READ command,
  so the client will be checked by host-based security.

\item[\Macro{SOAP\_LEAVE\_IN\_QUEUE}]
  \label{param:SoapLeaveInQueue}
  A boolean value that when \Expr{True},
  causes a job in the completed state to remain in the queue,
  instead of being removed based on the completion of file transfer.
  There is no default value.

\item[\Macro{WEB\_ROOT\_DIR}]
  \label{param:WebRootDir}
  A complete path to the directory containing all the files served
  by the web server.

\item[\MacroB{<SUBSYS>\_ENABLE\_SOAP\_SSL}]
  \label{param:SubsysEnableSoapSSL}
  \index{SUBSYS\_ENABLE\_SOAP\_SSL macro@\texttt{<SUBSYS>\_ENABLE\_SOAP\_SSL} macro}
  A boolean value that defaults to \Expr{False}.
  When \Expr{True}, enables SOAP over SSL for the specified
  \MacroNI{<SUBSYS>}.
  Any specific \MacroNI{<SUBSYS>\_ENABLE\_SOAP\_SSL} setting overrides
  the value of \MacroNI{ENABLE\_SOAP\_SSL}.

\item[\Macro{ENABLE\_SOAP\_SSL}]
  \label{param:EnableSoapSSL}
  A boolean value that defaults to \Expr{False}.
  When \Expr{True}, enables SOAP over SSL for all daemons.

\item[\MacroB{<SUBSYS>\_SOAP\_SSL\_PORT}]
  \label{param:SubsysSoapSSLPort}
  \index{SUBSYS\_SOAP\_SSL\_PORT macro@\texttt{<SUBSYS>\_SOAP\_SSL\_PORT} macro}
  A required port number on which SOAP over SSL messages are
  accepted, when SOAP over SSL is enabled.
  The \MacroNI{<SUBSYS>} must be specified, because multiple daemons
  running on a single machine may not share a port.
  There is no default value.

  The macro is named by substituting \MacroNI{<SUBSYS>}
  with the appropriate subsystem string as defined in
  section~\ref{sec:Condor-Subsystem-Names}.

\item[\Macro{SOAP\_SSL\_SERVER\_KEYFILE}]
  \label{param:SoapSSLServerKeyfile}
  A required complete path and file name to specify the daemon's
  identity, as used in authentication when SOAP over SSL is enabled.
  The file is to be  an OpenSSL PEM file containing a certificate
  and private key.
  There is no default value.

\item[\Macro{SOAP\_SSL\_SERVER\_KEYFILE\_PASSWORD}]
  \label{param:SoapSSLServerKeyfilePassword}
  An optional complete path and file name to specify
  a password for unlocking the daemon's private key.
  There is no default value.

\item[\Macro{SOAP\_SSL\_CA\_FILE}]
  \label{param:SoapSSLCaFile}
  A required complete path and file name to specify 
  a file containing certificates of trusted Certificate Authorities (CAs).
  Only clients who present a certificate signed by a trusted
  CA will be authenticated.
  There is no default value.

\item[\Macro{SOAP\_SSL\_CA\_DIR}]
  \label{param:SoapSSLCaDir}
  A required complete path to a directory
  containing certificates of trusted Certificate Authorities (CAs).
  Only clients who present a certificate signed by a trusted
  CA will be authenticated.
  There is no default value.

\item[\Macro{SOAP\_SSL\_DH\_FILE}]
  \label{param:SoapSSLDhFile}
  An optional complete path and file name to a DH file
  containing keys for a DH key exchange.
  There is no default value.

\end{description}



%%%%%%%%%%%%%%%%%%%%%%%%%%%%%%%%%%%%%%%%%%%%%%%%%%%%%%%%%%%%%%%%%%%%%%
\subsection{\label{sec:Stork-Config-File-Entries}Stork Configuration
File Macros}
%%%%%%%%%%%%%%%%%%%%%%%%%%%%%%%%%%%%%%%%%%%%%%%%%%%%%%%%%%%%%%%%%%%%%%
 
\begin{description}

\item[\Macro{STORK\_MAX\_NUM\_JOBS}]
\label{param:StorkMaxNumJobs}
An integer limit on the number of concurrent data placement jobs
handled by Stork.  The default value when not defined is 10.

\item[\Macro{STORK\_MAX\_RETRY}]
\label{param:StorkMaxRetry}
An integer limit on the
number of attempts for a single data placement job.  For data transfers,
this includes transfer attempts on the primary protocol, all
alternate protocols, and all retries.
The default value when not defined is 10.

\item[\Macro{STORK\_MAXDELAY\_INMINUTES}]
\label{param:StorkMaxDelayInMinutes}
An integer limit (in minutes) on the run time for a data placement job,
after which the job is considered failed.
The default value when not defined is 10,
and the minimum legal value is 1.

\item[\Macro{STORK\_TMP\_CRED\_DIR}]
\label{param:StorkTmpCredDir}
The full path to the temporary credential storage directory used by Stork.
The default value is \File{/tmp} when not defined. 

\item[\Macro{STORK\_MODULE\_DIR}]
\label{param:StorkModuleDir}
The full path to the directory containing Stork modules.
The default value when not defined is 
as defined by \MacroUNI{LIBEXEC}.  It is a fatal error for
both \MacroNI{STORK\_MODULE\_DIR} and \MacroNI{LIBEXEC} to be undefined.

\item[\Macro{CRED\_SUPER\_USERS}]
\label{param:CreddSuperUsers}   Access to a stored credential is
restricted to the user who submitted the credential, and any user
names specified in this macro.  The format is a space or comma
separated list of user names which are valid on the \Stork{credd}
host.
The default value of this macro is \Expr{root} on Unix systems, and
\Expr{Administrator} on Windows systems.

\item[\Macro{CRED\_STORE\_DIR}]
\label{param:CredStoreDir}   Directory for storing credentials.  This
directory must exist prior to starting \Stork{credd}.  It is highly
recommended to restrict access permissions to \emph{only} the
directory owner.
The default value is
\Expr{\$(SPOOL\_DIR)/cred}.

\item[\Macro{CRED\_INDEX\_FILE}]
\label{param:CredIndexFile}   Index file path of saved credentials.
This file will be automatically created if it does not exist.
The default value is
\Expr{\$(CRED\_STORE\_DIR)/cred-index}.

\item[\Macro{DEFAULT\_CRED\_EXPIRE\_THRESHOLD}]
\label{param:DefaultCredExpireThreshold}  \Stork{credd} will attempt
to refresh credentials when their remaining lifespan is less than this
value.
Units = seconds.  Default value = 3600 seconds (1 hour).

\item[\Macro{CRED\_CHECK\_INTERVAL}]
\label{param:CredCheckInterval} \Stork{credd} periodically checks
remaining lifespan of stored credentials, at this interval.
Units = seconds.  Default value = 60 seconds (1 minute).

\end{description}

