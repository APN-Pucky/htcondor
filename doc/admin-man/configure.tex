%%%%%%%%%%%%%%%%%%%%%%%%%%%%%%%%%%%%%%%%%%%%%%%%%%%%%%%%%%%%%%%%%%%%%%
\section{\label{sec:Configuring-HTCondor}Configuration}
%%%%%%%%%%%%%%%%%%%%%%%%%%%%%%%%%%%%%%%%%%%%%%%%%%%%%%%%%%%%%%%%%%%%%%

\index{HTCondor!configuration}
\index{configuration}

This section describes how to configure all parts of the HTCondor
system.  General information about the configuration
files and their syntax is followed by a description of
settings that affect all
HTCondor daemons and tools.
The 
settings that control the policy under which HTCondor will start,
suspend, resume, vacate or kill jobs
are described in 
section~\ref{sec:Configuring-Policy} on Policy Configuration for the
\Condor{startd}. 

%%%%%%%%%%%%%%%%%%%%%%%%%%%%%%%%%%%%%%%%%%%%%%%%%%%%%%%%%%%%%%%%%%%%%%
\subsection{\label{sec:Intro-to-Config-Files}Introduction to
Configuration Files} 
%%%%%%%%%%%%%%%%%%%%%%%%%%%%%%%%%%%%%%%%%%%%%%%%%%%%%%%%%%%%%%%%%%%%%%

The HTCondor configuration files are used to customize how HTCondor
operates at a given site.  The basic configuration as shipped with
HTCondor works well for most sites.

Each HTCondor program will, as part of its initialization process,
configure itself by calling a library routine which parses the
various configuration files that might be used, including pool-wide,
platform-specific, and machine-specific configuration files.
Environment variables may also contribute to the configuration.

The result of configuration is a list of key/value pairs.
Each key is a configuration variable name,
and each value is a string literal
that may utilize macro substitution (as defined below).
Some configuration variables are evaluated by HTCondor as ClassAd
expressions; some are not.  Consult the documentation for each specific
case.  Unless otherwise noted, configuration values that are expected
to be numeric or boolean constants may be any valid ClassAd expression
of operators on constants.  Example:

\begin{verbatim}
MINUTE          = 60
HOUR            = (60 * $(MINUTE))
SHUTDOWN_GRACEFUL_TIMEOUT = ($(HOUR)*24)
\end{verbatim}

%%%%%%%%%%%%%%%%%%%%%%%%%%%%%%%%%%%%%%%%%%%%%%%%%%%%%%%%%%%%%%%%%%%%%%
\subsubsection{\label{sec:Ordering-Config-File}Ordered Evaluation to
Set the Configuration} 
%%%%%%%%%%%%%%%%%%%%%%%%%%%%%%%%%%%%%%%%%%%%%%%%%%%%%%%%%%%%%%%%%%%%%%
\index{configuration file!evaluation order}

Multiple files, as well as a program's environment variables
determine the configuration.
The order in which attributes are defined is important, as later
definitions override existing definitions.
The order in which the (multiple) configuration files are parsed 
is designed to ensure the security of the system.
Attributes which must be set a specific way 
must appear in the last file to be parsed.
This prevents both the naive and the malicious HTCondor user 
from subverting the system through its configuration.
The order in which items are parsed is
\begin{enumerate}
\item global configuration file
\item local configuration file
\item specific environment variables prefixed with \MacroNI{\_CONDOR\_}
\end{enumerate}

The locations for these files are as given in
section~\ref{sec:Config-File-Locations} on
page~\pageref{sec:Config-File-Locations}.

Some HTCondor tools utilize environment variables to set their
configuration.
These tools search for specifically-named environment variables.
The variables are prefixed by the string \MacroNI{\_CONDOR\_}
or \MacroNI{\_condor\_}.
The tools strip off the prefix, and utilize what remains
as configuration.
As the use of environment variables is the last within
the ordered evaluation, 
the environment variable definition is used.
The security of the system is not compromised,
as only specific variables are considered for definition
in this manner, not any environment variables with
the \MacroNI{\_CONDOR\_} prefix.


%%%%%%%%%%%%%%%%%%%%%%%%%%%%%%%%%%%%%%%%%%%%%%%%%%%%%%%%%%%%%%%%%%%%%%
\subsubsection{\label{sec:Config-File-Macros}Configuration File Macros} 
%%%%%%%%%%%%%%%%%%%%%%%%%%%%%%%%%%%%%%%%%%%%%%%%%%%%%%%%%%%%%%%%%%%%%%

\index{macro!in configuration file}
\index{configuration file!macro definitions}

Macro definitions are of the form:
\begin{verbatim}
<macro_name> = <macro_definition>
\end{verbatim}

The macro name given on the left hand side of the definition is
a case sensitive identifier.
There must be white space between the macro name, the
equals sign (\verb@=@), and the macro definition.
The macro definition is a string literal that may utilize macro substitution.

Macro invocations are of the form: 
\begin{verbatim}
$(macro_name)
\end{verbatim}

Macro definitions may contain references to other macros, even ones
that are not yet defined, as long as they are eventually defined in
the configuration files.
All macro expansion is done after all configuration files have been parsed,
with the exception of macros that reference themselves.

\begin{verbatim}
A = xxx
C = $(A) 
\end{verbatim}
is a legal set of macro definitions, and the resulting value of 
\MacroNI{C} is
\Expr{xxx}.
Note that
\MacroNI{C} is actually bound to 
\MacroUNI{A}, not its value.

As a further example,
\begin{verbatim}
A = xxx
C = $(A)
A = yyy
\end{verbatim}
is also a legal set of macro definitions, and the resulting value of
\MacroNI{C} is \Expr{yyy}.  

A macro may be incrementally defined by invoking itself in its
definition.  For example,
\begin{verbatim}
A = xxx
B = $(A)
A = $(A)yyy
A = $(A)zzz
\end{verbatim}
is a legal set of macro definitions, and the resulting value of 
\MacroNI{A}
is \Expr{xxxyyyzzz}.
Note that invocations of a macro in
its own definition are immediately
expanded.
\MacroUNI{A} is immediately expanded in line 3 of the example.
If it were not, then the definition would be impossible to
evaluate.

Recursively defined macros such as
\begin{verbatim}
A = $(B)
B = $(A)
\end{verbatim}
are \emph{not} allowed.
They create definitions that HTCondor refuses to parse. 

All entries in a configuration file must have an operator,
which will be an equals sign (\verb@=@).
Identifiers are alphanumerics combined with the underscore character,
optionally with a subsystem name and a period as a prefix.
As a special case,
a line without an operator that begins with a left square bracket
will be ignored.
The following two-line example treats the first line as a comment,
and correctly handles the second line.
\begin{verbatim}
[HTCondor Settings]
my_classad = [ foo=bar ]
\end{verbatim}

% functionality added to version 6.7.13
To simplify pool administration,
any configuration variable name may be prefixed by
a subsystem 
(see the \MacroUNI{SUBSYSTEM} macro in 
section~\ref{sec:Pre-Defined-Macros}
for the list of subsystems)
and the period (\verb@.@) character.
For configuration variables defined this way,
the value is applied to the specific subsystem.
For example,
the ports that HTCondor may use can be restricted to a range 
using the \MacroNI{HIGHPORT} and \MacroNI{LOWPORT} configuration
variables.

\begin{verbatim}
  MASTER.LOWPORT   = 20000
  MASTER.HIGHPORT  = 20100
\end{verbatim}

Note that all configuration variables may utilize this syntax,
but nonsense configuration variables may result.
For example, it makes no sense to define
\begin{verbatim}
  NEGOTIATOR.MASTER_UPDATE_INTERVAL = 60
\end{verbatim}
since the \Condor{negotiator} daemon does not use the
\MacroNI{MASTER\_UPDATE\_INTERVAL} variable.

It makes little sense to do so, but HTCondor will configure
correctly with a definition such as
\begin{verbatim}
  MASTER.MASTER_UPDATE_INTERVAL = 60
\end{verbatim}
The \Condor{master} uses this configuration variable,
and the prefix of \MacroNI{MASTER.} causes this configuration
to be specific to the \Condor{master} daemon.

% the local functionality added in 7.1.4
This syntax has been further expanded to allow for the
specification of a local name on the command line 
using the command line option
\begin{verbatim}
  -local-name <local-name>
\end{verbatim}
This allows multiple instances of a daemon to be run 
by the same \Condor{master} daemon,
each instance with its own local configuration variable.

The ordering used to look up a variable, called \verb@<parameter name>@:

\begin{enumerate}
\item \verb@<subsystem name>.<local name>.<parameter name>@

\item \verb@<local name>.<parameter name>@

\item \verb@<subsystem name>.<parameter name>@

\item \verb@<parameter name>@
\end{enumerate}

If this local name is not specified on the command line, 
numbers 1 and 2 are skipped.
As soon as the first match is found, the search is completed,
and the corresponding value is used.

This example configures a \Condor{master} to run 2 \Condor{schedd}
daemons.  The \Condor{master} daemon needs the configuration:
\begin{verbatim}
  XYZZY           = $(SCHEDD)
  XYZZY_ARGS      = -local-name xyzzy
  DAEMON_LIST     = $(DAEMON_LIST) XYZZY
  DC_DAEMON_LIST  = + XYZZY
  XYZZY_LOG       = $(LOG)/SchedLog.xyzzy
\end{verbatim}

Using this example configuration, the \Condor{master} starts up a
second \Condor{schedd} daemon, 
where this second \Condor{schedd} daemon is passed 
\OptArg{-local-name}{xyzzy}
on the command line.

Continuing the example,
configure the \Condor{schedd} daemon named \Attr{xyzzy}.
This \Condor{schedd} daemon will share all configuration variable
definitions with the other \Condor{schedd} daemon,
except for those specified separately.

\begin{verbatim}
  SCHEDD.XYZZY.SCHEDD_NAME = XYZZY
  SCHEDD.XYZZY.SCHEDD_LOG  = $(XYZZY_LOG)
  SCHEDD.XYZZY.SPOOL       = $(SPOOL).XYZZY
\end{verbatim}

Note that the example \MacroNI{SCHEDD\_NAME} and \MacroNI{SPOOL} are
specific to the \Condor{schedd} daemon, as opposed to a different daemon
such as the \Condor{startd}.
Other HTCondor daemons using this feature will
have different requirements for which parameters need to be
specified individually.  This example works for the \Condor{schedd},
and more local configuration can, and likely would be specified.

Also note that each daemon's log file must be specified individually,
and in two places: one specification is for use by the \Condor{master},
and the other is for use by the daemon itself.
In the example,
the \Attr{XYZZY} \Condor{schedd} configuration variable
\MacroNI{SCHEDD.XYZZY.SCHEDD\_LOG} definition references the
\Condor{master} daemon's \MacroNI{XYZZY\_LOG}.


%%%%%%%%%%%%%%%%%%%%%%%%%%%%%%%%%%%%%%%%%%%%%%%%%%%%%%%%%%%%%%%%%%%%%%
\subsubsection{\label{sec:Other-Syntax}Comments and Line Continuations}
%%%%%%%%%%%%%%%%%%%%%%%%%%%%%%%%%%%%%%%%%%%%%%%%%%%%%%%%%%%%%%%%%%%%%%

An HTCondor configuration file may contain comments and
line continuations.
A comment is any line beginning with a pound character (\verb@#@).
A continuation is any entry that continues across multiples lines.
Line continuation is accomplished by placing the backslash
character (\Bs) at the end of any line to be continued onto another.
Valid examples of line continuation are
\begin{verbatim}
  START = (KeyboardIdle > 15 * $(MINUTE)) && \
  ((LoadAvg - CondorLoadAvg) <= 0.3)
\end{verbatim}
and
\begin{verbatim}
  ADMIN_MACHINES = condor.cs.wisc.edu, raven.cs.wisc.edu, \
  stork.cs.wisc.edu, ostrich.cs.wisc.edu, \
  bigbird.cs.wisc.edu
  HOSTALLOW_ADMINISTRATOR = $(ADMIN_MACHINES)
\end{verbatim}

Note that a line continuation character may currently be used within
a comment, so the following example does \emph{not} set the
configuration variable \MacroNI{FOO}:
\begin{verbatim}
  # This comment includes the following line, so FOO is NOT set \
  FOO = BAR
\end{verbatim}
It is a poor idea to use this functionality.

%%%%%%%%%%%%%%%%%%%%%%%%%%%%%%%%%%%%%%%%%%%%%%%%%%%%%%%%%%%%%%%%%%%%%%
\subsubsection{\label{sec:Program-Defined-Macros}Executing a Program to Produce Configuration Macros}
%%%%%%%%%%%%%%%%%%%%%%%%%%%%%%%%%%%%%%%%%%%%%%%%%%%%%%%%%%%%%%%%%%%%%%

Instead of reading from a file,
HTCondor may run a program to obtain configuration macros.
The vertical bar character (\Bar) as the last character defining
a file name provides the syntax necessary to tell 
HTCondor to run a program.
This syntax may only be used in the definition of
the \Env{CONDOR\_CONFIG} environment variable,
or the \Macro{LOCAL\_CONFIG\_FILE} configuration variable.

The command line for the program 
is formed by the characters preceding the vertical bar character.
The standard output of the program is parsed as a configuration 
file would be.

An example:
\begin{verbatim}
LOCAL_CONFIG_FILE = /bin/make_the_config|
\end{verbatim}

Program \Prog{/bin/make\_the\_config} is executed, and its output
is the set of configuration macros.

Note that either a program is executed to generate the
configuration macros or the configuration is read from 
one or more files.
The syntax uses space characters to separate command line elements,
if an executed program produces the configuration macros.
Space characters would otherwise separate the list of files.
This syntax does not permit distinguishing one from the other,
so only one may be specified.

%%%%%%%%%%%%%%%%%%%%%%%%%%%%%%%%%%%%%%%%%%%%%%%%%%%%%%%%%%%%%%%%%%%%%%
\subsubsection{\label{sec:Macros-Requiring-Restart}Macros That Will Require a Restart When Changed}
%%%%%%%%%%%%%%%%%%%%%%%%%%%%%%%%%%%%%%%%%%%%%%%%%%%%%%%%%%%%%%%%%%%%%%
\index{configuration change requiring a restart of HTCondor}
When any of the following listed configuration variables are changed,
HTCondor must be restarted.
Reconfiguration using \Condor{reconfig} will not be enough.

\begin{itemize}
  \item \verb@BIND_ALL_INTERFACES@
  \item \verb@FetchWorkDelay@
  \item \verb@MAX_NUM_CPUS@
  \item \verb@MAX_TRACKING_GID@
  \item \verb@MIN_TRACKING_GID@
  \item \verb@NETWORK_INTERFACE@
  \item \verb@NUM_CPUS@
  \item \verb@PREEMPTION_REQUIREMENTS_STABLE@
  \item \verb@PRIVSEP_ENABLED@
  \item \verb@PROCD_ADDRESS@
  \item \verb@SLOT_TYPE_<N>@
\end{itemize}

%%%%%%%%%%%%%%%%%%%%%%%%%%%%%%%%%%%%%%%%%%%%%%%%%%%%%%%%%%%%%%%%%%%%%%
\subsubsection{\label{sec:Pre-Defined-Macros}Pre-Defined Macros}
%%%%%%%%%%%%%%%%%%%%%%%%%%%%%%%%%%%%%%%%%%%%%%%%%%%%%%%%%%%%%%%%%%%%%%

\index{configuration!pre-defined macros}
\index{configuration file!pre-defined macros}
HTCondor provides pre-defined macros that help configure HTCondor.
Pre-defined macros are listed as \MacroUNI{macro\_name}.

This first set are entries whose values are determined at
run time and cannot be overwritten.  These are inserted automatically by
the library routine which parses the configuration files.
This implies that a change to the underlying value of any of these
variables will require a full restart of HTCondor in order to use
the changed value.
\begin{description}
  
\label{param:FullHostname}
\item[\MacroU{FULL\_HOSTNAME}]
  The fully qualified host name of the local machine, 
  which is host name plus domain name.
  
\label{param:Hostname}
\item[\MacroU{HOSTNAME}]
  The host name of the local machine, \emph{without} a domain name.
  
\label{param:IpAddress}
\item[\MacroU{IP\_ADDRESS}]
  The ASCII string version of the local machine's IP address.

\label{param:Tilde}
\item[\MacroU{TILDE}]
  The full path to the
  home directory of the Unix user \Login{condor}, if such a user exists on the
  local machine.

  \label{sec:HTCondor-Subsystem-Names}
  \index{configuration file!subsystem names}
\label{param:Subsystem}
\item[\MacroU{SUBSYSTEM}]
  The subsystem
  name of the daemon or tool that is evaluating the macro.
  This is a unique string which identifies a given daemon within the
  HTCondor system.  The possible subsystem names are:

  \index{subsystem names}
  \index{macro!subsystem names}
  \begin{itemize}
  \label{list:subsystem names}
  \item \verb@C_GAHP@
  \item \verb@CKPT_SERVER@
  \item \verb@COLLECTOR@
  \item \verb@DBMSD@
  \item \verb@DEFRAG@
  \item \verb@EC2_GAHP@
  \item \verb@GRIDMANAGER@
  \item \verb@HAD@
  \item \verb@HDFS@
  \item \verb@JOB_ROUTER@
  \item \verb@KBDD@ 
  \item \verb@LEASEMANAGER@
  \item \verb@MASTER@
  \item \verb@NEGOTIATOR@
  \item \verb@QUILL@
  \item \verb@REPLICATION@
  \item \verb@ROOSTER@
  \item \verb@SCHEDD@
  \item \verb@SHADOW@
  \item \verb@STARTD@
  \item \verb@STARTER@
  %\item \verb@STORK@
  \item \verb@SUBMIT@
  \item \verb@TOOL@
  \item \verb@TRANSFERER@
  \end{itemize}

\end{description}

This second set of macros are entries whose default values are
determined automatically at run time but which can be overwritten.  
\index{configuration file!macros}
\begin{description}

\label{param:Arch}
\item[\MacroU{ARCH}]
  Defines the string
  used to identify the architecture of the local machine to HTCondor.
  The \Condor{startd} will advertise itself with this attribute so
  that users can submit binaries compiled for a given platform and
  force them to run on the correct machines.  \Condor{submit} will
  append a requirement to the job ClassAd that it must
  run on the same \MacroNI{ARCH} and \MacroNI{OPSYS} of the machine where
  it was submitted, unless the user specifies \MacroNI{ARCH} and/or
  \MacroNI{OPSYS} explicitly in their submit file.  See the
  the \Condor{submit} manual page
  on page~\pageref{man-condor-submit} for details.

\label{param:OpSys}
\item[\MacroU{OPSYS}]
  Defines the string used to identify the operating system
  of the local machine to HTCondor.
  If it is not defined in the configuration file, HTCondor will
  automatically insert the operating system of this machine as
  determined by \Prog{uname}.

\label{param:OpSysVer}
\item[\MacroU{OPSYS\_VER}]
  Defines the integer used to identify the operating system version number.

\label{param:OpSysAndVer}
\item[\MacroU{OPSYS\_AND\_VER}]
  Defines the string used prior to HTCondor version 7.7.2 as \MacroUNI{OPSYS}.

\label{param:UnameArch}
\item[\MacroU{UNAME\_ARCH}]
  The architecture as reported by \Prog{uname}(2)'s \Code{machine} field.
  Always the same as \MacroNI{ARCH} on Windows.

\label{param:UnameOpsys}
\item[\MacroU{UNAME\_OPSYS}]
  The operating system as reported by \Prog{uname}(2)'s \Code{sysname} field.
  Always the same as \MacroNI{OPSYS} on Windows.

\label{param:DetectedMemory}
\item[\MacroU{DETECTED\_MEMORY}]
  The amount of detected physical memory (RAM) in Mbytes.

\label{param:DetectedCores}
\item[\MacroU{DETECTED\_CORES}]
  The number of detected CPU cores.  
  This includes hyper threaded cores, if there are any.

\label{param:Pid}
\item[\MacroU{PID}]
  The process ID for the daemon or tool.

\label{param:Ppid}
\item[\MacroU{PPID}]
  The process ID of the parent process for the daemon or tool.

\label{param:Username}
\item[\MacroU{USERNAME}]
  The user name of the UID of the daemon or tool.
  For daemons started as root, but running under another UID
  (typically the user \Login{condor}), this will be the other UID.

\label{param:FilesystemDomain}
\item[\MacroU{FILESYSTEM\_DOMAIN}]
  Defaults to the fully
  qualified host name of the machine it is evaluated on.  See
  section~\ref{sec:Shared-Filesystem-Config-File-Entries}, Shared
  File System Configuration File Entries for the full description of
  its use and under what conditions it could be desirable to change it.

\label{param:UIDDomain}
\item[\MacroU{UID\_DOMAIN}]
  Defaults to the fully
  qualified host name of the machine it is evaluated on.  See
  section~\ref{sec:Shared-Filesystem-Config-File-Entries} 
  for the full description of this configuration variable.

\end{description}

Since \MacroUNI{ARCH} and \MacroUNI{OPSYS} will automatically be set to the
correct values, we recommend that you do not overwrite them.


%%%%%%%%%%%%%%%%%%%%%%%%%%%%%%%%%%%%%%%%%%%%%%%%%%%%%%%%%%%%%%%%%%%%%%
\subsection{\label{sec:Config-File-Special}Special Macros}
%%%%%%%%%%%%%%%%%%%%%%%%%%%%%%%%%%%%%%%%%%%%%%%%%%%%%%%%%%%%%%%%%%%%%%

\index{configuration file!\$ENV definition}
\index{\$ENV!in configuration file}
References to the HTCondor process's environment are allowed in the
configuration files.
Environment references use the \Macro{ENV} macro and are of the form:
\begin{verbatim}
  $ENV(environment_variable_name)
\end{verbatim}
For example, 
\begin{verbatim}
  A = $ENV(HOME)
\end{verbatim}
binds \MacroNI{A} to the value of the \Env{HOME} environment variable.
Environment references are not currently used in standard HTCondor
configurations.
However, they can sometimes be useful in custom configurations.

\index{\$RANDOM\_CHOICE()!in configuration}
This same syntax is used in the \Macro{RANDOM\_CHOICE()} macro to
allow a random choice of a parameter
within a configuration file.
These references are of the form:
\begin{verbatim}
  $RANDOM_CHOICE(list of parameters)
\end{verbatim}
This allows a random choice within the parameter list to be made
at configuration time.  Of the list of parameters, one is
chosen when encountered during configuration.  For example,
if one of the integers 0-8 (inclusive) should be randomly
chosen, the macro usage is
\begin{verbatim}
  $RANDOM_CHOICE(0,1,2,3,4,5,6,7,8)
\end{verbatim}

\index{\$RANDOM\_INTEGER()!in configuration}
The \Macro{RANDOM\_INTEGER()} macro is similar to the \MacroNI{RANDOM\_CHOICE()}
macro, and is used to select a random integer within a configuration file.
References are of the form:
\begin{verbatim}
  $RANDOM_INTEGER(min, max [, step])
\end{verbatim}
A random integer within the range \verb@min@ and \verb@max@, inclusive,
is selected at configuration time.
The optional \verb@step@ parameter
controls the stride within the range, and it defaults to the value 1.
For example, to randomly chose an even integer in the range 0-8 (inclusive),
the macro usage is
\begin{verbatim}
  $RANDOM_INTEGER(0, 8, 2)
\end{verbatim}

See section~\ref{sec:randomintegerusage} on
page~\pageref{sec:randomintegerusage}
for an actual use of this specialized macro.
%%%%%%%%%%%%%%%%%%%%%%%%%%%%%%%%%%%%%%%%%%%%%%%%%%%%%%%%%%%%%%%%%%%%%%
\subsection{\label{sec:HTCondor-wide-Config-File-Entries}HTCondor-wide Configuration File Entries} 
%%%%%%%%%%%%%%%%%%%%%%%%%%%%%%%%%%%%%%%%%%%%%%%%%%%%%%%%%%%%%%%%%%%%%%

\index{configuration!HTCondor-wide configuration variables}

This section describes settings which affect all parts of the HTCondor
system. 
Other system-wide settings can be found in
section~\ref{sec:Network-Related-Config-File-Entries} on
``Network-Related Configuration File Entries'', and
section~\ref{sec:Shared-Filesystem-Config-File-Entries} on ``Shared
File System Configuration File Entries''. 

\begin{description}
  
\label{param:CondorHost}
\item[\Macro{CONDOR\_HOST}]
  This macro is used to define the
  \MacroUNI{COLLECTOR\_HOST} macro.  Normally the \Condor{collector}
  and \Condor{negotiator} would run on the same machine.  If for some
  reason they were not run on the same machine,
  \MacroUNI{CONDOR\_HOST} would not be needed.  Some
  of the host-based security macros use \MacroUNI{CONDOR\_HOST} by
  default.  See section~\ref{sec:Host-Security}, on Setting up
  IP/host-based security in HTCondor for details.
  
\label{param:CollectorHost}
\item[\Macro{COLLECTOR\_HOST}]
  The host name of the machine where the \Condor{collector} is running for
  your pool.  Normally, it is defined relative to
  the \MacroUNI{CONDOR\_HOST}
  macro.  There is no default value for this macro;
  \MacroNI{COLLECTOR\_HOST} must be defined for the pool to work
  properly.

  In addition to defining the host name, this setting can optionally be
  used to specify the network port of the \Condor{collector}.
  The port is separated from the host name by a colon ('\verb@:@').
  For example,
  \begin{verbatim}
    COLLECTOR_HOST = $(CONDOR_HOST):1234
  \end{verbatim}
  If no port is specified, the default port of 9618 is used.
  Using the default port is recommended for most sites.
  It is only changed if there is a conflict with another
  service listening on the same network port.
  For more information about specifying a non-standard port for the
  \Condor{collector} daemon,
  see section~\ref{sec:Ports-NonStandard} on
  page~\pageref{sec:Ports-NonStandard}.


\label{param:NegotiatorHost} 
\item[\Macro{NEGOTIATOR\_HOST}]
  This configuration variable is no longer used.
  It previously defined the host name of the machine where 
  the \Condor{negotiator} is running.
  At present, the port where the \Condor{negotiator} is listening 
  is dynamically allocated. 

  % commented out by Karen in 2008, as this 6.7ism is too old
  %For pools running 6.7.3 and older versions: The
  %host name of the machine where the \Condor{negotiator} is running for
  %the pool.
  %Normally, it is defined relative to the \MacroUNI{CONDOR\_HOST}
  %macro.  There is no default value for this macro;
  %\MacroNI{NEGOTIATOR\_HOST} must be defined for the pool to work
  %properly.
  %This variable may also be used to optionally define a network port for
  %the \Condor{negotiator} daemon, as explained for the
  %\MacroNI{COLLECTOR\_HOST} variable.

\label{param:CondorViewHost}
\item[\Macro{CONDOR\_VIEW\_HOST}]
  A list of HTCondorView servers, separated by commas and/or spaces.
  Each HTCondorView server is denoted by the host name of the machine
  it is running on, optionally appended by a colon and the port number.
  This service is optional, and requires additional configuration 
  to enable it.  There is no default value for
  \MacroNI{CONDOR\_VIEW\_HOST}.  If \MacroNI{CONDOR\_VIEW\_HOST} is not
  defined, no HTCondorView server is used.
  See section~\ref{sec:Contrib-HTCondorView-Install} on
  page~\pageref{sec:Contrib-HTCondorView-Install} for more details.

\label{param:ScheddHost}
\item[\Macro{SCHEDD\_HOST}]
  The host name of the machine where the \Condor{schedd} is running for
  your pool.  This is the host that queues submitted jobs.
  If the host specifies \Macro{SCHEDD\_NAME} or \Macro{MASTER\_NAME}, that
  name must be included in the form name\verb$@$hostname.
  In most condor installations, there is a \Condor{schedd} running on
  each host from which jobs are submitted.  The default value of
  \Macro{SCHEDD\_HOST} is the current host with the optional name included.  For most pools, this
  macro is not defined, nor does it need to be defined..

\label{param:ReleaseDir}
\item[\Macro{RELEASE\_DIR}]
  The full path to
  the HTCondor release directory, which holds the \File{bin},
  \File{etc}, \File{lib}, and \File{sbin} directories.  Other macros
  are defined relative to this one.  There is no default value for
  \Macro{RELEASE\_DIR}.

\label{param:Bin}
\item[\Macro{BIN}]
  This directory points to the
  HTCondor directory where user-level programs are installed.  It is
  usually defined relative to the \MacroUNI{RELEASE\_DIR} macro.
  There is no default value for \Macro{BIN}.
  
\label{param:Lib}
\item[\Macro{LIB}]
  This directory points to the
  HTCondor directory where libraries used to link jobs for HTCondor's
  standard universe are stored.  The \Condor{compile} program uses
  this macro to find these libraries, so it must be defined for
  \Condor{compile} to function.  \MacroUNI{LIB} is usually defined
  relative to the \MacroUNI{RELEASE\_DIR} macro, and has no default
  value.

\label{param:LibExec}
\item[\Macro{LIBEXEC}]
  This directory points
  to the HTCondor directory where support commands that HTCondor
  needs will be placed.
  Do not add this directory to a user or system-wide path.

\label{param:Include}
\item[\Macro{INCLUDE}]
  This directory points to the HTCondor directory where header files reside.
  \MacroUNI{INCLUDE} would usually be defined relative to
  the \MacroUNI{RELEASE\_DIR} configuration macro.
  There is no default value, but
  if defined, it can make inclusion of necessary header files
  for compilation of programs (such as those programs
  that use \File{libcondorapi.a})
  easier through the use of \Condor{config\_val}.

\label{param:Sbin}
\item[\Macro{SBIN}]
  This directory points to the
  HTCondor directory where HTCondor's system binaries (such as the
  binaries for the HTCondor daemons) and administrative tools are
  installed.  Whatever directory \MacroU{SBIN} points to ought
  to be in the \Env{PATH} of users acting as HTCondor
  administrators.  \Macro{SBIN} has no default value.

\label{param:LocalDir}
\item[\Macro{LOCAL\_DIR}]
  The location of the
  local HTCondor directory on each machine in your pool.  One common
  option is to use the condor user's home directory which may be
  specified with \MacroUNI{TILDE}.  There is no default value for
  \Macro{LOCAL\_DIR}.  For example:
  \begin{verbatim}
    LOCAL_DIR = $(tilde)
  \end{verbatim}
  
  On machines with a shared file system, where either the
  \MacroUNI{TILDE} directory or another directory you want to use is
  shared among all machines in your pool, you might use the
  \MacroUNI{HOSTNAME} macro and have a directory with many
  subdirectories, one for each machine in your pool, each named by
  host names.  For example:
  \begin{verbatim}
    LOCAL_DIR = $(tilde)/hosts/$(hostname)      
  \end{verbatim}
  or:
  \begin{verbatim}
    LOCAL_DIR = $(release_dir)/hosts/$(hostname)
  \end{verbatim}
  
\label{param:Log}
\item[\Macro{LOG}]
  Used to specify the
  directory where each HTCondor daemon writes its log files.  The names
  of the log files themselves are defined with other macros, which use
  the \MacroUNI{LOG} macro by default.  The log directory also acts as
  the current working directory of the HTCondor daemons as the run, so
  if one of them should produce a core file for any reason, it would
  be placed in the directory defined by this macro.  \MacroNI{LOG} is
  required to be defined.  Normally, \MacroUNI{LOG} is defined in
  terms of \MacroUNI{LOCAL\_DIR}.

\label{param:Run}
\item[\Macro{RUN}]
  A path and directory name to be used by the HTCondor init script to 
  specify the directory where the \Condor{master} should write its process ID
  (PID) file. 
  The default if not defined is \MacroUNI{LOG}.
  
\label{param:Spool}
\item[\Macro{SPOOL}]
  The spool directory is where
  certain files used by the \Condor{schedd} are stored, such as the
  job queue file and the initial executables of any jobs that have
  been submitted.  In addition, for systems not using a checkpoint
  server, all the checkpoint files from jobs that have been submitted
  from a given machine will be store in that machine's spool
  directory.  Therefore, you will want to ensure that the spool
  directory is located on a partition with enough disk space.  If a
  given machine is only set up to execute HTCondor jobs and not submit
  them, it would not need a spool directory (or this macro defined).
  There is no default value for \Macro{SPOOL}, and the \Condor{schedd}
  will not function without it \Macro{SPOOL} defined.  Normally,
  \MacroUNI{SPOOL} is defined in terms of \MacroUNI{LOCAL\_DIR}.
  
\label{param:Execute}
\item[\Macro{EXECUTE}]
  This directory acts as
  a place to create the scratch directory of any HTCondor job that is executing
  on
  the local machine.  The scratch directory is the destination of
  any input files that were specified for transfer.  It also serves
  as the job's working directory if the job is using file transfer
  mode and no other working directory was specified.
  If a given machine is set up to only submit
  jobs and not execute them, it would not need an execute directory,
  and this macro need not be defined.  There is no default value for
  \MacroNI{EXECUTE}, and the \Condor{startd} will not function if
  \MacroNI{EXECUTE} is undefined.  Normally, \MacroUNI{EXECUTE} is
  defined in terms of \MacroUNI{LOCAL\_DIR}.  To customize the execute
  directory independently for each batch slot, use \MacroNI{SLOT<N>\_EXECUTE}.

\label{param:SlotNExecute}
\item[\Macro{SLOT<N>\_EXECUTE}]
  Specifies an
  execute directory for use by a specific batch slot.
  \MacroNI{<N>} represents the number of the batch slot, such as 1, 2, 3, etc.
  This execute directory serves the same purpose as \Macro{EXECUTE}, but it
  allows the configuration of the directory independently for each batch
  slot.  Having slots each using a different partition would be
  useful, for example, in preventing one job from filling up the same
  disk that other jobs are trying to write to.  If this parameter is
  undefined for a given batch slot, it will use \MacroNI{EXECUTE} as
  the default.  Note that each slot will advertise \AdAttr{TotalDisk}
  and \AdAttr{Disk} for the partition containing its execute
  directory.

\label{param:LocalConfigFile}
\item[\Macro{LOCAL\_CONFIG\_FILE}]
  Identifies the
  location of the local, machine-specific configuration
  file for each machine
  in the pool.  The two most common choices would be putting this
  file in the \MacroUNI{LOCAL\_DIR}, or putting all
  local configuration files for the pool in a shared directory, each one
  named by host name.  For example,
  \begin{verbatim}
    LOCAL_CONFIG_FILE = $(LOCAL_DIR)/condor_config.local
  \end{verbatim}
  or,
  \begin{verbatim}
    LOCAL_CONFIG_FILE = $(release_dir)/etc/$(hostname).local
  \end{verbatim}
  or, not using the release directory
  \begin{verbatim}
    LOCAL_CONFIG_FILE = /full/path/to/configs/$(hostname).local
  \end{verbatim}
  
  The value of \MacroNI{LOCAL\_CONFIG\_FILE} is treated as a list of files,
  not a
  single file.  The items in the list are delimited by either commas
  or space characters.
  This allows the specification of multiple files as
  the local configuration file, each one processed in the
  order given (with parameters set in later files overriding values
  from previous files).  This allows the use of one global
  configuration file for multiple platforms in the pool, defines a
  platform-specific configuration file for each platform, and uses a
  local configuration file for each machine. 
  If the list of files is changed in one of the later read files, the new list
  replaces the old list, but any files that have already been processed
  remain processed, and are removed from the new list if they are present
  to prevent cycles.
  See section~\ref{sec:Program-Defined-Macros} on 
  page~\pageref{sec:Program-Defined-Macros} for directions on
  using a program to generate the configuration macros that would
  otherwise reside in one or more files as described here.
  If \MacroNI{LOCAL\_CONFIG\_FILE} is not defined, no local configuration
  files are processed.  For more information on this, see
  section~\ref{sec:Multiple-Platforms} about Configuring HTCondor for
  Multiple Platforms on page~\pageref{sec:Multiple-Platforms}.

  If all files in a directory are local configuration files to be processed,
  then consider using \MacroNI{LOCAL\_CONFIG\_DIR}, defined at
  section~\ref{param:LocalConfigDir}.

\label{param:RequireLocalConfigFile}
\item[\Macro{REQUIRE\_LOCAL\_CONFIG\_FILE}]
  A boolean value that defaults to \Expr{True}.
  When \Expr{True}, HTCondor exits with an error,
  if any file listed in \MacroNI{LOCAL\_CONFIG\_FILE} cannot be read.
  A value of \Expr{False} allows local configuration files to be missing.
  This is most useful for sites that have 
  both large numbers of machines in the pool and a local configuration file
  that uses the \MacroUNI{HOSTNAME} macro in its definition.
  Instead of having an empty file for every host
  in the pool, files can simply be omitted.

\label{param:LocalConfigDir} 
\item[\Macro{LOCAL\_CONFIG\_DIR}]
  A directory may be used as a container for local configuration files. 
  The files found in the directory are sorted into lexicographical order 
  by file name, and 
  then each file is treated as though it was listed in 
  \MacroNI{LOCAL\_CONFIG\_FILE}. 
  \MacroNI{LOCAL\_CONFIG\_DIR} is processed before any files listed in 
  \MacroNI{LOCAL\_CONFIG\_FILE}, and is checked again after processing
  the \MacroNI{LOCAL\_CONFIG\_FILE} list. 
  It is a list of directories, and each directory is processed in the order
  it appears in the list. 
  The process is not recursive, so any directories found inside the directory
  being processed are ignored. 
  See also \MacroNI{LOCAL\_CONFIG\_DIR\_EXCLUDE\_REGEXP}.

\label{param:LocalConfigDirExcludeRegexp}
\item[\Macro{LOCAL\_CONFIG\_DIR\_EXCLUDE\_REGEXP}]
  A regular expression that specifies file names to be ignored when
  looking for configuration files within the directories specified via
  \MacroNI{LOCAL\_CONFIG\_DIR}.  The default expression ignores files
  with names beginning with a `.' or a `\verb|#|', as well as files with names
  ending in `\~{}'.  This avoids accidents that can be caused by
  treating temporary files created by text editors as configuration
  files.

\label{param:CondorIds}
\item[\Macro{CONDOR\_IDS}]
  The User ID (UID) and Group ID (GID) pair that the HTCondor daemons
  should run as, if the daemons are spawned as root.
  This value can also be specified in the \Env{CONDOR\_IDS}
  environment variable.
  If the HTCondor daemons are not started as root, then neither this
  \MacroNI{CONDOR\_IDS} configuration macro nor the \Env{CONDOR\_IDS}
  environment variable are used.
  The value is given by two integers, separated by a period.  For
  example, \verb@CONDOR_IDS = 1234.1234@.
  If this pair is not specified in either the configuration file or in the
  environment, and the HTCondor daemons are spawned as root,
  then HTCondor will
  search for a \verb@condor@ user on the system, and run as that user's
  UID and GID.
  See section~\ref{sec:uids} on UIDs in HTCondor for more details.

\label{param:CondorAdmin}
\item[\Macro{CONDOR\_ADMIN}]
  The email address that HTCondor will send mail to if something goes wrong in
  the pool.  For example, if a daemon crashes, the \Condor{master}
  can send an \Term{obituary} to this address with the last few lines
  of that daemon's log file and a brief message that describes what
  signal or exit status that daemon exited with.  There is no default
  value for \MacroNI{CONDOR\_ADMIN}.

\label{param:SubsysAdminEmail}
\item[\MacroB{<SUBSYS>\_ADMIN\_EMAIL}]
\index{SUBSYS\_ADMIN\_EMAIL macro@\texttt{<SUBSYS>\_ADMIN\_EMAIL} macro}
  The email address that HTCondor will send mail to if something goes wrong
  with the named \MacroNI{<SUBSYS>}.  Identical to \MacroNI{CONDOR\_ADMIN},
  but done on a per subsystem basis. There is no default value.
  
\label{param:CondorSupportEmail}
\item[\Macro{CONDOR\_SUPPORT\_EMAIL}]
  The email address to be included at the bottom of all email HTCondor
  sends out under the label ``Email address of the local HTCondor
  administrator:''.  
  This is the address where HTCondor users at your site should send
  their questions about HTCondor and get technical support.
  If this setting is not defined, HTCondor will use the address
  specified in \MacroNI{CONDOR\_ADMIN} (described above).

\label{param:EmailSignature}
\item[\Macro{EMAIL\_SIGNATURE}]
  Every e-mail sent by HTCondor includes a short signature line appended
  to the body.  By default, this signature includes the URL to the
  global HTCondor project website.  
  When set, this variable defines an alternative signature line to be
  used instead of the default. 
  Note that the value can only be one line in length.
  This variable could be used to direct users
  to look at local web site with information specific to the installation
  of HTCondor.

\label{param:Mail}
\item[\Macro{MAIL}]
  The full path to a mail
  sending program that uses \Opt{-s} to specify a subject for the
  message.  On all platforms, the default shipped with HTCondor should
  work.  Only if you installed things in a non-standard location on
  your system would you need to change this setting.  There is no
  default value for \MacroNI{MAIL}, and the \Condor{schedd} will not
  function unless \MacroNI{MAIL} is defined.

\label{param:MailFrom}
\item[\Macro{MAIL\_FROM}]
  The e-mail address that notification e-mails appear to come from.
  Contents is that of the \Expr{From} header.
  There is no default value; if undefined, the \Expr{From} header may
  be nonsensical.

\label{param:SMTPServer}
\item[\Macro{SMTP\_SERVER}]
  For Windows platforms only, the host name of the server through which
  to route notification e-mail.
  There is no default value; if undefined and the debug level is
  at  \Expr{FULLDEBUG}, an error message will be generated.

\label{param:ReservedSwap}
\item[\Macro{RESERVED\_SWAP}]
  The amount of swap space in Mbytes to reserve for this machine.
  HTCondor will not start up more \Condor{shadow} processes if the
  amount of free swap space on this machine falls below this level.
  The default value is 0, which disables this check.
  It is anticipated that this configuration variable will no longer
  be used in the near future.
  If \MacroNI{RESERVED\_SWAP} is \emph{not} set to 0,
  the value of \Macro{SHADOW\_SIZE\_ESTIMATE} is used.

\label{param:ReservedDisk}
\item[\Macro{RESERVED\_DISK}]
  Determines how much disk space you want to reserve for your own machine.
  When HTCondor is reporting the amount of free disk space in a given
  partition on your machine, it will always subtract this amount.  An
  example is the \Condor{startd}, which advertises the amount of free
  space in the \MacroUNI{EXECUTE} directory.  The default value of
  \Macro{RESERVED\_DISK} is zero.
  
\label{param:Lock}
\item[\Macro{LOCK}]
  HTCondor needs to create
  lock files to synchronize access to various log files.  Because of
  problems with network file systems and file locking over
  the years, we \emph{highly} recommend that you put these lock
  files on a local partition on each machine.  If you do not have your
  \MacroUNI{LOCAL\_DIR} on a local partition, be sure to change this
  entry.

  Whatever user or group HTCondor is running as needs to have
  write access to this directory.  If you are not running as root, this
  is whatever user you started up the \Condor{master} as.  If you are
  running as root, and there is a condor account, it is most
  likely condor.
  Otherwise, it is whatever you set in the \Env{CONDOR\_IDS}
  \index{environment variables!CONDOR\_IDS@\texttt{CONDOR\_IDS}}
  \index{CONDOR\_IDS@\texttt{CONDOR\_IDS}!environment variable}
  environment variable, or whatever you define in the
  \MacroNI{CONDOR\_IDS} setting in the HTCondor config files.
  See section~\ref{sec:uids} on UIDs in HTCondor for details.

  If no value for \MacroNI{LOCK} is provided, the value of \MacroNI{LOG}
  is used.


\label{param:History}
\item[\Macro{HISTORY}]
  Defines the
  location of the HTCondor history file, which stores information about
  all HTCondor jobs that have completed on a given machine.  This macro
  is used by both the \Condor{schedd} which appends the information
  and \Condor{history}, the user-level program used to view
  the history file.
  This configuration macro is given the default value of
  \File{\$(SPOOL)/history} in the default configuration.
  If not defined,
  no history file is kept.
  % PKK
  % Described in default config file: YES
  % Defined in the default config file: YES 
  % Default definition in config file: $(SPOOL)/history
  % Result if not defined or RHS is empty: no history file is kept.

\label{param:EnableHistoryRotation} 
\item[\Macro{ENABLE\_HISTORY\_ROTATION}]
  If this is defined to be true, then the
  history file will be rotated. If it is false, then it will not be
  rotated, and it will grow indefinitely, to the limits allowed by the
  operating system. If this is not defined, it is assumed to be
  true. The rotated files will be stored in the same directory as the
  history file. 

\label{param:MaxHistoryLog}
\item[\Macro{MAX\_HISTORY\_LOG}]
  Defines the maximum size for the history file, in bytes. It defaults
  to 20MB. This parameter is only used if history file rotation is
  enabled. 

\label{param:MaxHistoryRotations}
\item[\Macro{MAX\_HISTORY\_ROTATIONS}]
  When history file rotation is turned on, this controls how many
  backup files there are. It default to 2, which means that there may
  be up to three history files (two backups, plus the history file
  that is being currently written to). When the history file is
  rotated, and this rotation would cause the number of backups to be
  too large, the oldest file is removed. 

\label{param:MaxJobQueueLogRotations}
\item[\Macro{MAX\_JOB\_QUEUE\_LOG\_ROTATIONS}]
  The schedd periodically rotates the job queue database file in order
  to save disk space.  This option controls how many rotated files are
  saved.  It defaults to 1, which means there may be up to two history
  files (the previous one, which was rotated out of use, and the current one
  that is being written to).  When the job queue file is rotated,
  and this rotation would cause the number of backups to be larger
  the the maximum specified, the oldest file is removed.

\label{param:ClassadLogStrictParsing}
\item[\Macro{CLASSAD\_LOG\_STRICT\_PARSING}]
  A boolean value that defaults to \Expr{True}. 
  When \Expr{True}, ClassAd log files will be read using 
  a strict syntax checking for ClassAd expressions.  
  ClassAd log files include the job queue log and the accountant log.
  When \Expr{False}, 
  ClassAd log files are read without strict expression syntax checking, 
  which allows some legacy ClassAd log data to be read in a backward
  compatible manner.  
  This configuration variable may no longer be supported in future releases, 
  eventually requiring all ClassAd log files to pass strict 
  ClassAd syntax checking. 

\label{param:DefaultDomainName}
\item[\Macro{DEFAULT\_DOMAIN\_NAME}]
  The value to be appended to a machine's host name,
  representing a domain name, which HTCondor then uses
  to form a fully qualified host name.
  This is required if there is no fully qualified host name 
  in file \File{/etc/hosts} or in NIS.
  Set the value in the global configuration file,
  as HTCondor may depend on knowing this value in order to locate
  the local configuration file(s).
  The default value as given in the sample configuration file of
  the HTCondor download is bogus, and must be changed.
  If this variable is removed from the global configuration file,
  or if the definition is empty, then HTCondor attempts to discover
  the value.

\label{param:NoDNS}
\item[\Macro{NO\_DNS}]
  A boolean value that defaults to \Expr{False}.
  When \Expr{True}, HTCondor constructs host names using the host's IP address
  together with the value defined for \MacroNI{DEFAULT\_DOMAIN\_NAME}. 

\label{param:CMIPAddr}
\item[\Macro{CM\_IP\_ADDR}]
  If neither \MacroNI{COLLECTOR\_HOST} nor 
  \MacroNI{COLLECTOR\_IP\_ADDR} macros are defined, then this
  macro will be used to determine the IP address of the central
  manager (collector daemon).
  This macro is defined by an IP address.
  % PKK
  % Described in default config file: NO
  % Defined in the default config file: NO
  % Default definition in config file: N/A
  % Result if not defined or RHS is empty: HTCondor performs above algorithm

\label{param:EmailDomain}
\item[\Macro{EMAIL\_DOMAIN}]
  By default, if a user does not specify \AdAttr{notify\_user} in the
  submit description file, any email HTCondor sends about that job will
  go to "username@UID\_DOMAIN".
  If your machines all share a common UID domain (so that you would
  set \MacroNI{UID\_DOMAIN} to be the same across all machines in your
  pool), but email to user@UID\_DOMAIN is not the right place for
  HTCondor to send email for your site, you can define the default
  domain to use for email.
  A common example would be to set \MacroNI{EMAIL\_DOMAIN} to the fully
  qualified host name of each machine in your pool, so users submitting
  jobs from a specific machine would get email sent to
  user@machine.your.domain, instead of user@your.domain.  
  You would do this by setting \MacroNI{EMAIL\_DOMAIN} to
  \MacroUNI{FULL\_HOSTNAME}. 
  In general, you should leave this setting commented out unless two
  things are true: 1) \MacroNI{UID\_DOMAIN} is set to your domain, not
  \MacroUNI{FULL\_HOSTNAME}, and 2) email to user@UID\_DOMAIN will not 
  work. 
  % PKK
  % Described in default config file: YES
  % Defined in the default config file: NO
  % Default definition in config file: bogus
  % Result if not defined or RHS is empty: 
  %	HTCondor will try to use the notify_user attribute email in the job ad.
  %	If that is not present, then it will use the UID_DOMAIN embedded in
  %	the job ad.
  %	If that is not present, then it will use the UID_DOMAIN found in the
  %	config file.
  %	If that is not present, then I suspect there is a bug and the code will
  %	segfault!!! (This needs fixing...)
  
\label{param:CreateCoreFiles}
\item[\Macro{CREATE\_CORE\_FILES}]
  Defines whether or not HTCondor daemons are to
  create a core file in the \Macro{LOG} directory
  if something really bad happens.  It is
  used to set
  the resource limit for the size of a core file.  If not defined,
  it leaves in place whatever limit was in effect
  when the HTCondor daemons (normally the \Condor{master}) were started.
  This allows HTCondor to inherit the default system core file generation
  behavior at start up.  For Unix operating systems, this behavior can
  be inherited from the parent shell, or specified in a shell script
  that starts HTCondor.
  If this parameter is set and \Expr{True}, the limit is increased to
  the maximum.  If it is set to \Expr{False}, the limit is set at 0
  (which means that no core files are created).  Core files
  greatly help the HTCondor developers debug any problems you might be
  having.  By using the parameter, you do not have to worry about
  tracking down where in your boot scripts you need to set the core
  limit before starting HTCondor. You set the parameter
  to whatever behavior you want HTCondor to enforce.  This parameter
  defaults to undefined to allow the initial operating system default
  value to take precedence, 
  and is commented out in the default configuration file. 
  % PKK
  % Described in default config file: YES
  % Defined in the default config file: NO
  % Default definition in config file: bogus
  % Result if not defined or RHS is empty: shell's default corelimit size applies

\label{param:CkptProbe}
\item[\Macro{CKPT\_PROBE}]
  Defines the path and executable name of the helper process HTCondor will use to
  determine information for the \Attr{CheckpointPlatform} attribute
  in the machine's ClassAd. 
  The default value is \File{\$(LIBEXEC)/condor\_ckpt\_probe}.

\label{param:AbortOnException}
\item[\Macro{ABORT\_ON\_EXCEPTION}]
  When HTCondor programs detect a fatal internal exception, they
  normally log an error message and exit.  If you have turned on
  \Macro{CREATE\_CORE\_FILES}, in some cases you may also want to turn
  on \Macro{ABORT\_ON\_EXCEPTION} so that core files are generated
  when an exception occurs.  Set the following to True if that is what
  you want.

\label{param:QQueryTimeout}
\item[\Macro{Q\_QUERY\_TIMEOUT}]
  Defines the timeout (in seconds) that \Condor{q} uses when trying to
  connect to the \Condor{schedd}.  Defaults to 20 seconds.
  % PKK
  % Described in default config file: NO
  % Defined in the default config file: NO
  % Default definition in config file: N/A
  % Result if not defined or RHS is empty: defaults to 20 seconds.

\label{param:DeadCollectorMaxAvoidanceTime}
\item[\Macro{DEAD\_COLLECTOR\_MAX\_AVOIDANCE\_TIME}]
  Defines the interval of time
  (in seconds) between checks for a failed primary \Condor{collector} daemon.
  If connections to the dead primary \Condor{collector} take very
  little time to fail, new attempts to query the primary \Condor{collector} may
  be more frequent than the specified maximum avoidance time.
  The default value equals one hour.
  This variable has relevance to flocked jobs, as it defines 
  the maximum time they may be reporting to the primary \Condor{collector}
  without the \Condor{negotiator} noticing.

\label{param:PasswdCacheRefresh}
\item[\Macro{PASSWD\_CACHE\_REFRESH}]
  HTCondor can cause NIS servers to become overwhelmed by queries for uid
  and group information in large pools. In order to avoid this problem,
  HTCondor caches UID and group information internally. This integer value allows
  pool administrators to specify (in seconds) how long HTCondor should wait
  until refreshes a cache entry. The default is set to 300 seconds, or
  5 minutes, plus a random number of seconds between 0 and 60 to avoid
  having lots of processes refreshing at the same time.
  This means that if a pool administrator updates the user
  or group database (for example, \File{/etc/passwd} or \File{/etc/group}),
  it can take up
  to 6 minutes before HTCondor will have the updated information. This
  caching feature can be disabled by setting the refresh interval to
  0. In addition, the cache can also be flushed explicitly by running
  the command \Condor{reconfig}.
  This configuration variable has no effect on Windows.
  % PKK
  % Described in default config file: NO
  % Defined in the default config file: NO
  % Default definition in config file: N/A
  % Result if not defined or RHS is empty: 300 seconds

\label{param:SysapiGetLoadavg}
\item[\Macro{SYSAPI\_GET\_LOADAVG}]
  If set to False, then HTCondor will not attempt to compute the load average
  on the system, and instead will always report the system load average
  to be 0.0.  Defaults to True.

\label{param:NetworkMaxPendingConnects}
\item[\Macro{NETWORK\_MAX\_PENDING\_CONNECTS}]
  This specifies a limit to the maximum number of simultaneous network
  connection attempts.  This is primarily relevant to \Condor{schedd},
  which may try to connect to large numbers of startds when claiming
  them.  The negotiator may also connect to large numbers of startds
  when initiating security sessions used for sending MATCH messages.  On
  Unix, the default for this parameter is eighty percent of the process file
  descriptor limit.  On windows, the default is 1600.

\label{param:WantUDPCommandSocket}
\item[\Macro{WANT\_UDP\_COMMAND\_SOCKET}]
  This setting, added in version 6.9.5, controls if HTCondor daemons
  should create a UDP command socket in addition to the TCP command
  socket (which is required).
  The default is \Expr{True}, and modifying it requires restarting all
  HTCondor daemons, not just a \Condor{reconfig} or SIGHUP.

  Normally, updates sent to the \Condor{collector} use UDP, in
  addition to certain keep alive messages and other non-essential
  communication.
  However, in certain situations, it might be desirable to disable the
  UDP command port.

  Unfortunately, due to a limitation in how these command sockets are
  created, it is not possible to define this setting on a per-daemon
  basis, for example, by trying to set
  \MacroNI{STARTD.WANT\_UDP\_COMMAND\_SOCKET}.
  At least for now, this setting must be defined machine wide to
  function correctly.

  If this setting is set to true on a machine running a
  \Condor{collector}, the pool should be configured to use TCP updates
  to that collector (see section~\ref{sec:tcp-collector-update} on
  page~\pageref{sec:tcp-collector-update} for more information).

\label{param:AllowScriptsToRunAsExecutables}
\item[\Macro{ALLOW\_SCRIPTS\_TO\_RUN\_AS\_EXECUTABLES}]
  A boolean value that, when \Expr{True}, permits scripts on Windows
  platforms to be used in place of the \SubmitCmd{executable} in a job
  submit description file, in place of a \Condor{dagman} pre or post script,
  or in producing the configuration, for example. 
  Allows a script to be used in any circumstance previously
  limited to a Windows executable or a batch file.
  The default value is \Expr{True}.
  See section~\ref{sec:windows-scripts-as-executables} on
  page~\pageref{sec:windows-scripts-as-executables} for further description.

\label{param:OpenVerbForExtFiles}
\item[\Macro{OPEN\_VERB\_FOR\_<EXT>\_FILES}]
  A string that defines a Windows \Term{verb} for use in a root hive
  registry look up.
  \verb@<EXT>@ defines the file name extension, which represents a
  scripting language, also needed for the look up.
  See section~\ref{sec:windows-scripts-as-executables} on
  page~\pageref{sec:windows-scripts-as-executables} for a more complete
  description.

\label{param:EnableClassadCaching}
\item[\Macro{ENABLE\_CLASSAD\_CACHING}]
  A boolean value that controls the caching of ClassAds.
  Caching saves memory when an HTCondor process contains
  many ClassAds with the same expressions.
  The default value is \Expr{False}, which disables caching.

\label{param:StrictClassadEvaluation}
\item[\Macro{STRICT\_CLASSAD\_EVALUATION}]
  A boolean value that controls how ClassAd expressions are evaluated. 
  If set to \Expr{True}, then New ClassAd evaluation semantics are used.
  This means that attribute references without a \Attr{MY.} or
  \Attr{TARGET.} prefix are only looked up in the local ClassAd.
  If set to the default value of \Expr{False}, 
  Old ClassAd evaluation semantics are used.
  See section~\ref{sec:classad-newandold}  on
  page~\pageref{sec:classad-newandold} for details.

\label{param:ClassadUserLibs}
\item[\Macro{CLASSAD\_USER\_LIBS}]
  A comma separated list of paths to shared libraries that contain
  additional ClassAd functions to be used during ClassAd evaluation.
  
\label{param:CondorFsync}
\item[\Macro{CONDOR\_FSYNC}]
  A boolean value that controls whether HTCondor calls \Procedure{fsync} when
  writing the user job and transaction logs.  
  Setting this value to \Expr{False}
  will disable calls to \Procedure{fsync}, 
  which can help performance for \Condor{schedd}
  log writes at the cost of some durability of the log contents,
  should there be a power or hardware failure.  
  The default value is \Expr{True}.

\label{param:StatisticsToPublish}
\item[\Macro{STATISTICS\_TO\_PUBLISH}]
  A comma and/or space separated list that identifies which daemons are to
  publish Statistics attributes in their ClassAds, as well as a level of
  verbosity to identify which attributes to include and which to omit from
  the ClassAd. 
  The syntax defines the two aspects by separating them with a colon;
  the first aspect defines which daemon is to publish the statistics,
  and the second aspect defines the verbosity.
  This first aspect may be \Expr{SCHEDD} or \Expr{SCHEDULER} to publish
  Statistics attributes in the ClassAd of the \Condor{schedd}.
  Or, it may be \Expr{DC} or \Expr{DAEMONCORE} to publish 
  DaemonCore statistics.  It may also be \Expr{TRANSFER} to publish
  file transfer statistics.
  After the colon may be the value \Expr{0}, \Expr{1}, \Expr{2}, or \Expr{3}.
  A value of  \Expr{0} turns off the publishing of any Statistics attributes.
  A value of  \Expr{1} is the default level, where some Statistics attributes
  are published and others are omitted.
  A value of  \Expr{2} is the verbose level, where all Statistics attributes
  are published.
  A value of  \Expr{3} is the super verbose level, which is currently unused,
  but intended to be all Statistics attributes published at the verbose
  level plus extra information.
  As an example, to cause a verbose setting of the publication of Statistics
  attributes for the \Condor{schedd}:
\begin{verbatim}
  STATISTICS_TO_PUBLISH = SCHEDD:2
\end{verbatim}
  
\label{param:StatisticsWindowSeconds}
\item[\Macro{STATISTICS\_WINDOW\_SECONDS}]
  An integer value that controls the time window size, in seconds, for
  collecting windowed daemon statistics.
  These statistics are, by convention, those attributes with names that 
  are of the form \Attr{Recent<attrname>}.  Any data contributing to a
  windowed statistic that is older than this number of seconds is dropped
  from the statistic.  
  For example, if \Expr{STATISTICS\_WINDOW\_SECONDS = 300},
  then any jobs submitted more than 300 seconds ago are not counted 
  in the windowed statistic \Attr{RecentJobsSubmitted}.  
  Defaults to 1200 seconds, which is 20 minutes.

  The window is broken into smaller time pieces called quantum.
  The window advances one quantum at a time.

\label{param:StatisticsWindowSecondsCollection}
\item[\Macro{STATISTICS\_WINDOW\_SECONDS\_<collection>}]
  The same as \MacroNI{STATISTICS\_WINDOW\_SECONDS}, 
  but used to 
  override the global setting for a particular statistic collection.  
  Collection names currently implemented are \Expr{DC} or \Expr{DAEMONCORE}
  and \Expr{SCHEDD} or \Expr{SCHEDULER}.

\label{param:StatisticsWindowQuantum}
\item[\Macro{STATISTICS\_WINDOW\_QUANTUM}]
  For experts only,
  an integer value that controls the time quantization that form a
  time window, 
  in seconds, for the data structures that maintain windowed statistics.
  Defaults to 240 seconds, which is 6 minutes.
  This default is purposely set to be slightly smaller than the update
  rate to the \Condor{collector}. 
  Setting a smaller value than the default increases the memory requirement
  for the statistics.
  Graphing of statistics at the level of the quantum
  expects to see counts that appear like a saw tooth. 

\label{param:StatisticsWindowQuantumCollection}
\item[\Macro{STATISTICS\_WINDOW\_QUANTUM\_<collection>}]
  The same as \MacroNI{STATISTICS\_WINDOW\_QUANTUM}, 
  but used to 
  override the global setting for a particular statistic collection.  
  Collection names currently implemented are \Expr{DC} or \Expr{DAEMONCORE}
  and \Expr{SCHEDD} or \Expr{SCHEDULER}.

\end{description}


%%%%%%%%%%%%%%%%%%%%%%%%%%%%%%%%%%%%%%%%%%%%%%%%%%%%%%%%%%%%%%%%%%%%%%%%%%%
\subsection{\label{sec:Daemon-Logging-Config-File-Entries}Daemon Logging Configuration File Entries} 
%%%%%%%%%%%%%%%%%%%%%%%%%%%%%%%%%%%%%%%%%%%%%%%%%%%%%%%%%%%%%%%%%%%%%%%%%%%

\index{configuration!daemon logging configuration variables}
These entries control how and where the HTCondor daemons write to log
files.  Many of the entries in this section represents multiple
macros. There is one for each subsystem (listed
in section~\ref{sec:HTCondor-Subsystem-Names}).
The macro name for each substitutes \MacroNI{<SUBSYS>} with the name
of the subsystem corresponding to the daemon.
\begin{description}
  
\label{param:SubsysLog}
\item[\MacroB{<SUBSYS>\_LOG}]
\index{SUBSYS\_LOG macro@\texttt{<SUBSYS>\_LOG} macro}
  The name of
  the log file for a given subsystem.  For example,
  \MacroUNI{STARTD\_LOG} gives the location of the log file for
  \Condor{startd}. The default is \File{\$(LOG)/<SUBSYS>LOG}.
  If the log file cannot be written to,
  then the daemon will attempt to log this into a new file of the name
  \File{\$(LOG)/dprintf\_failure.<SUBSYS>} before the daemon exits.

\label{param:MaxSubsysLog}
\item[\Macro{MAX\_<SUBSYS>\_LOG}]
  Controls the maximum size in bytes or amount of time that a
  log will be allowed to grow.  When it is time to rotate a log file,
  it will be saved to a file with an ISO timestamp 
  suffix. The oldest rotated file receives the ending \File{.old}. 
  The \File{.old} files are overwritten each time the maximum 
  number of rotated files (determined by the value of
  \MacroNI{MAX\_NUM\_<SUBSYS>\_LOG}) is exceeded.
  A value of 0 specifies that the file may grow without bounds. 
  If not specified, the default is 1 Mbyte in size.
  A single integer value is specified; without a suffix, defaults to
  specifying a size in bytes.  The following suffixes may be used to
  qualify the integer:
\begin{description} 
  \item{\Expr{Bytes} for bytes}
  \item{\Expr{Kb} for kilobytes}
  \item{\Expr{Mb} for megabytes}
  \item{\Expr{Gb} for gigabytes}
  \item{\Expr{Tb} for terabytes}
  \item{\Expr{Sec} for seconds}
  \item{\Expr{Min} for minutes}
  \item{\Expr{Hr} for hours}
  \item{\Expr{Day} for days}
  \item{\Expr{Wk} for weeks}
\end{description} 
 
\label{param:MaxNumSubsysLog}
\item[\Macro{MAX\_NUM\_<SUBSYS>\_LOG}]
  An integer that controls the maximum number of rotations a log file 
  is allowed to perform before the oldest one will be 
  rotated away. Thus, at most \Expr{MAX\_NUM\_<SUBSYS>\_LOG + 1}
  log files of the same program coexist at a given time.
  The default value is 1.

\label{param:TruncSubsysLogOnOpen}
\item[\Macro{TRUNC\_<SUBSYS>\_LOG\_ON\_OPEN}]
  If this macro is defined and set
  to \Expr{True}, the affected log will be truncated and started from an
  empty file with each invocation of the program.  Otherwise, new
  invocations of the program will append to the previous log
  file.  By default this setting is \Expr{False} for all daemons.
  
\label{param:SubsysLogKeepOpen}
\item[\MacroB{<SUBSYS>\_LOG\_KEEP\_OPEN}]
\index{SUBSYS\_LOG\_KEEP\_OPEN macro@\texttt{<SUBSYS>\_LOG\_KEEP\_OPEN} macro}
  A boolean value that controls whether or not the log file is kept open 
  between writes.
  When \Expr{True}, the daemon will not open and close the log file
  between writes.  Instead the daemon will hold the log file open until the log
  needs to be rotated. 
  When \Expr{False}, the daemon reverts to the previous behavior
  of opening and closing the log file between writes.  
  When the \MacroUNI{<SUBSYS>\_LOCK} macro is defined,
  setting \MacroUNI{<SUBSYS>\_LOG\_KEEP\_OPEN} has no effect,
  as the daemon
  will unconditionally revert back to the open/close between writes behavior.
  On Windows platforms,
  the value defaults to \Expr{True} for all daemons. 
  On Linux platforms,
  the value defaults to \Expr{True} for all daemons,
  except the \Condor{shadow},
  due to a global file descriptor limit.

\label{param:SubsysLock} 
\item[\MacroB{<SUBSYS>\_LOCK}]
\index{SUBSYS\_LOCK macro@\texttt{<SUBSYS>\_LOCK} macro}
  This macro
  specifies the lock file used to synchronize append operations to the
  log file for this subsystem.  It must be a separate file from the
  \MacroUNI{<SUBSYS>\_LOG} file, since the \MacroUNI{<SUBSYS>\_LOG} file may be
  rotated and you want to be able to synchronize access across log
  file rotations.  A lock file is only required for log files which
  are accessed by more than one process.  Currently, this includes
  only the \MacroNI{SHADOW} subsystem.  This macro is defined relative
  to the \MacroUNI{LOCK} macro.

\label{param:JobQueueLog} 
\item[\Macro{JOB\_QUEUE\_LOG}]
  A full path and file name, specifying the job queue log.  
  The default value, when not defined is \File{\MacroUNI{SPOOL}/job\_queue.log}.
  This specification can be useful,
  if there is a solid state drive which is big enough to hold the
  frequently written to \File{job\_queue.log},
  but not big enough to hold the whole contents of the spool directory.

\label{param:FileLockViaMutex} 
\item[\Macro{FILE\_LOCK\_VIA\_MUTEX}]
  This macro setting only works on Win32 -- it is ignored on Unix.  If set
  to be \Expr{True}, then log locking is implemented via a kernel mutex
  instead of via file locking.  On Win32, mutex access is FIFO, while
  obtaining a file lock is non-deterministic.  Thus setting to \Expr{True}
  fixes problems on Win32 where processes (usually shadows) could starve
  waiting for a lock on a log file.  Defaults to \Expr{True} on Win32, and is
  always \Expr{False} on Unix.

\label{param:LockDebugLogToAppend}
\item[\Macro{LOCK\_DEBUG\_LOG\_TO\_APPEND}]
  A boolean value that defaults to \Expr{False}.
  This variable controls whether a daemon's debug lock is used when
  appending to the log.  
  When \Expr{False}, the debug lock is only used when rotating the log file.
  This is more efficient, 
  especially when many processes share the same log file.
  When \Expr{True}, the debug lock is used when writing to the log,
  as well as when rotating the log file.  
  This setting is ignored under Windows,
  and the behavior of Windows platforms is as though 
  this variable were \Expr{True}.
  Under Unix, the default value of \Expr{False} is appropriate when
  logging to file systems that support the POSIX semantics of \Expr{O\_APPEND}.
  On non-POSIX-compliant file systems, 
  it is possible for the characters in log messages from multiple processes
  sharing the same log to be interleaved, unless locking is used.
  Since HTCondor does not support sharing of debug logs between
  processes running on different machines, many non-POSIX-compliant
  file systems will still avoid interleaved messages without requiring
  HTCondor to use a lock.  Tests of AFS and NFS have
  not revealed any problems when appending to the log without locking.

\label{param:EnableUserlogLocking}
\item[\Macro{ENABLE\_USERLOG\_LOCKING}]
  When \Expr{True} (the default value),
  a user's job log (as specified in a submit description file)
  will be locked before being written to.
  If \Expr{False}, HTCondor will not lock the file before writing.

\label{param:NewLocking}
\item[\Macro{CREATE\_LOCKS\_ON\_LOCAL\_DISK}]
  A boolean value utilized only for Unix operating systems, 
  that defaults to \Expr{True}. 
  This variable is only relevant if \MacroNI{ENABLE\_USERLOG\_LOCKING}
  is \Expr{True}.
  When \Expr{True}, lock files are written to a directory named \File{condorLocks},
  thereby using a local drive to avoid known problems with locking on NFS.
  The location of the \File{condorLocks} directory is determined by
  \begin{enumerate}
  \item The value of \MacroNI{TEMP\_DIR}, if defined.
  \item The value of \MacroNI{TMP\_DIR}, if defined and \MacroNI{TEMP\_DIR}
  is not defined.
  \item The default value of \File{/tmp}, if neither \MacroNI{TEMP\_DIR}
  nor \MacroNI{TMP\_DIR} is defined.
  \end{enumerate}

\label{param:TouchLogInterval}
\item[\Macro{TOUCH\_LOG\_INTERVAL}]
  The time interval in seconds between when daemons touch
  their log files.  The change in last modification time for the
  log file is useful when a daemon restarts after failure or shut down.
  The last modification date is printed, and it provides an upper bound
  on the length of time that the daemon was not running.
  Defaults to 60 seconds.

\label{param:LogsUseTimestamp}
\item[\Macro{LOGS\_USE\_TIMESTAMP}]
  This macro controls how the current time is formatted at the start of
  each line in the daemon log files. When \Expr{True}, the Unix time is
  printed (number of seconds since 00:00:00 UTC, January 1, 1970).
  When \Expr{False} (the default value), the time is printed like so:
  \Expr{<Month>/<Day> <Hour>:<Minute>:<Second>} in the local timezone.

\label{param:DebugTimeFormat}
\item[\Macro{DEBUG\_TIME\_FORMAT}]
  This string defines how to format the current time printed at the
  start of each line in the daemon log files.  The value is a format 
  string is passed to the C \Procedure{strftime} function,
  so see that manual page for platform-specific details.
  If not defined, the default value is 
\begin{verbatim}
   "%m/%d %H:%M:%S "  
\end{verbatim}

\label{param:SubsysDebug}
\item[\MacroB{<SUBSYS>\_DEBUG}]
\index{SUBSYS\_DEBUG macro@\texttt{<SUBSYS>\_DEBUG} macro}
  All of the
  HTCondor daemons can produce different levels of output depending on
  how much information is desired.  The various levels of
  verbosity for a given daemon are determined by this macro.  All
  daemons have the default level \Dflag{ALWAYS}, and log messages for
  that level will be printed to the daemon's log, regardless of this
  macro's setting.  Settings are a comma- or space-separated list
  of the following values:

  \begin{description}
    \label{list:debug-level-description}

  \label{dflag:all}
  \item[\Dflag{ALL}]
    \index{SUBSYS\_DEBUG macro levels@\texttt{<SUBSYS>\_DEBUG} macro levels!D\_ALL@\texttt{D\_ALL}}
    This flag turns on \emph{all} debugging output by enabling all of the debug
    levels at once.  There is no need to list any other debug levels in addition
    to \Dflag{ALL}; doing so would be redundant.  Be warned: this will
    generate
    about a \emph{HUGE} amount of output.
    To obtain a higher
    level of output than the default, consider using \Dflag{FULLDEBUG} before
    using this option.

  \label{dflag:fulldebug}
  \item[\Dflag{FULLDEBUG}]
    \index{SUBSYS\_DEBUG macro levels@\texttt{<SUBSYS>\_DEBUG} macro levels!D\_FULLDEBUG@\texttt{D\_FULLDEBUG}}
    This level
    provides verbose output of a general nature into the log files.  
    Frequent log messages for very specific debugging
    purposes would be excluded. In those cases, the messages would
    be viewed by having that another flag and \Dflag{FULLDEBUG} both
    listed in the configuration file.

  \label{dflag:daemoncore} 
  \item[\Dflag{DAEMONCORE}]
    \index{SUBSYS\_DEBUG macro levels@\texttt{<SUBSYS>\_DEBUG} macro levels!D\_DAEMONCORE@\texttt{D\_DAEMONCORE}}
    Provides log
    file entries specific to DaemonCore, such as
    timers the daemons have set and the commands that are registered.
    If both \Dflag{FULLDEBUG} and \Dflag{DAEMONCORE} are set,
    expect \emph{very} verbose output.

  \label{dflag:priv}
  \item[\Dflag{PRIV}]
    \index{SUBSYS\_DEBUG macro levels@\texttt{<SUBSYS>\_DEBUG} macro levels!D\_PRIV@\texttt{D\_PRIV}}
    This flag provides log
    messages about the \Term{privilege state} switching that the daemons
    do.  See section~\ref{sec:uids} on UIDs in HTCondor for details.

  \label{dflag:command}
  \item[\Dflag{COMMAND}]
    \index{SUBSYS\_DEBUG macro levels@\texttt{<SUBSYS>\_DEBUG} macro levels!D\_COMMAND@\texttt{D\_COMMAND}}
    With this flag set, any
    daemon that uses DaemonCore will print out a log message
    whenever a command comes in.  The name and integer of the command,
    whether the command was sent via UDP or TCP, and where
    the command was sent from are all logged.  
    Because the messages about the command used by \Condor{kbdd} to
    communicate with the \Condor{startd} whenever there is activity on
    the X server, and the command used for keep-alives are both only
    printed with \Dflag{FULLDEBUG} enabled, it is best if this setting
    is used for all daemons.

  \label{dflag:load}
  \item[\Dflag{LOAD}]
    \index{SUBSYS\_DEBUG macro levels@\texttt{<SUBSYS>\_DEBUG} macro levels!D\_LOAD@\texttt{D\_LOAD}}
    The \Condor{startd} keeps track
    of the load average on the machine where it is running.  Both the
    general system load average, and the load average being generated by
    HTCondor's activity there are determined.
    With this flag set, the \Condor{startd}
    will log a message with the current state of both of these
    load averages whenever it computes them.  This flag only affects the
    \Condor{startd}.

  \label{dflag:keyboard} 
  \item[\Dflag{KEYBOARD}]
    \index{SUBSYS\_DEBUG macro levels@\texttt{<SUBSYS>\_DEBUG} macro levels!D\_KEYBOARD@\texttt{D\_KEYBOARD}}
    With this flag set, the \Condor{startd} will print out a log message
    with the current values for remote and local keyboard idle time.
    This flag affects only the \Condor{startd}.

  \label{dflag:job}
  \item[\Dflag{JOB}]
    \index{SUBSYS\_DEBUG macro levels@\texttt{<SUBSYS>\_DEBUG} macro levels!D\_JOB@\texttt{D\_JOB}}
    When this flag is set, the
    \Condor{startd} will send to its log file the contents of any
    job ClassAd that the \Condor{schedd} sends to claim the
    \Condor{startd} for its use.  This flag affects only the
    \Condor{startd}.
    
  \label{dflag:machine}
  \item[\Dflag{MACHINE}]
    \index{SUBSYS\_DEBUG macro levels@\texttt{<SUBSYS>\_DEBUG} macro levels!D\_MACHINE@\texttt{D\_MACHINE}}
    When this flag is set,
    the \Condor{startd} will send to its log file the contents of
    its resource ClassAd when the \Condor{schedd} tries to claim the
    \Condor{startd} for its use.  This flag affects only the
    \Condor{startd}.

  \label{dflag:syscalls}
  \item[\Dflag{SYSCALLS}]
    \index{SUBSYS\_DEBUG macro levels@\texttt{<SUBSYS>\_DEBUG} macro levels!D\_SYSCALLS@\texttt{D\_SYSCALLS}}
    This flag is used to
    make the \Condor{shadow} log remote syscall requests and return
    values.  This can help track down problems a user is having with a
    particular job by providing the system calls the job is
    performing. If any are failing, the reason for the
    failure is given.  The \Condor{schedd} also uses this flag for the server
    portion of the queue management code.  With \Dflag{SYSCALLS}
    defined in \MacroNI{SCHEDD\_DEBUG} there will be verbose logging of all
    queue management operations the \Condor{schedd} performs.  

  \label{dflag:match}
  \item[\Dflag{MATCH}]
    \index{SUBSYS\_DEBUG macro levels@\texttt{<SUBSYS>\_DEBUG} macro levels!D\_MATCH@\texttt{D\_MATCH}}
    When this flag is
    set, the \Condor{negotiator} logs a message for every match.

  \label{dflag:network}
  \item[\Dflag{NETWORK}]
    \index{SUBSYS\_DEBUG macro levels@\texttt{<SUBSYS>\_DEBUG} macro levels!D\_NETWORK@\texttt{D\_NETWORK}}
    When this flag is set,
    all HTCondor daemons will log a message on every TCP accept, connect,
    and close, and on every UDP send and receive.  This flag is not
    yet fully supported in the \Condor{shadow}.

  \label{dflag:hostname}
  \item[\Dflag{HOSTNAME}]
    \index{SUBSYS\_DEBUG macro levels@\texttt{<SUBSYS>\_DEBUG} macro levels!D\_HOSTNAME@\texttt{D\_HOSTNAME}}
    When this flag is set, the HTCondor daemons and/or tools will print
    verbose messages explaining how they resolve host names, domain
    names, and IP addresses.
    This is useful for sites that are having trouble getting HTCondor to
    work because of problems with DNS, NIS or other host name resolving
    systems in use.

  \label{dflag:ckpt}
  \item[\Dflag{CKPT}]
    \index{SUBSYS\_DEBUG macro levels@\texttt{<SUBSYS>\_DEBUG} macro levels!D\_CKPT@\texttt{D\_CKPT}}
    When this flag is set,
    the HTCondor process checkpoint support code, which is linked into a STANDARD 
    universe user job, will output some low-level details about the checkpoint
    procedure into the \MacroUNI{SHADOW\_LOG}.

  \label{dflag:security}
  \item[\Dflag{SECURITY}]
    \index{SUBSYS\_DEBUG macro levels@\texttt{<SUBSYS>\_DEBUG} macro levels!D\_SECURITY@\texttt{D\_SECURITY}}
    This flag will enable debug messages pertaining to the setup of 
    secure network communication, 
    including messages for the negotiation of a socket 
    authentication mechanism, the management of a session key cache.
    and messages about the authentication process itself.  See
    section~\ref{sec:Config-Security} for more information about
    secure communication configuration.

  \label{dflag:procfamily}
  \item[\Dflag{PROCFAMILY}]
    \index{SUBSYS\_DEBUG macro levels@\texttt{<SUBSYS>\_DEBUG} macro levels!D\_PROCFAMILY@\texttt{D\_PROCFAMILY}}
    HTCondor often times needs to manage an entire family of processes, (that
    is, a 
    process and all descendants of that process).  This debug flag will 
    turn on debugging output for the management of families of processes.

  \label{dflag:accountant}
  \item[\Dflag{ACCOUNTANT}]
    \index{SUBSYS\_DEBUG macro levels@\texttt{<SUBSYS>\_DEBUG} macro levels!D\_ACCOUNTANT@\texttt{D\_ACCOUNTANT}}
    When this flag is set,
    the \Condor{negotiator} will output debug messages relating to the computation
    of user priorities (see section~\ref{sec:UserPrio}).

  \label{dflag:protocol}
  \item[\Dflag{PROTOCOL}]
    \index{SUBSYS\_DEBUG macro levels@\texttt{<SUBSYS>\_DEBUG} macro levels!D\_PROTOCOL@\texttt{D\_PROTOCOL}}
    Enable debug messages relating to the protocol for HTCondor's matchmaking and
    resource claiming framework.
    
  \label{dflag:pid}
  \item[\Dflag{PID}]
    \index{SUBSYS\_DEBUG macro levels@\texttt{<SUBSYS>\_DEBUG} macro levels!D\_PID@\texttt{D\_PID}}
    This flag is different from the other flags, because it is
    used to change the formatting of all log messages that are printed,
    as opposed to specifying what kinds of messages should be printed.
    If \Dflag{PID} is set, HTCondor will always print out the process
    identifier (PID) of the process writing each line to the log file.
    This is especially helpful for HTCondor daemons that can fork
    multiple helper-processes (such as the \Condor{schedd} or
    \Condor{collector}) so the log file will clearly show which thread
    of execution is generating each log message.
    
  \label{dflag:fds}
  \item[\Dflag{FDS}]
    \index{SUBSYS\_DEBUG macro levels@\texttt{<SUBSYS>\_DEBUG} macro levels!D\_FDS@\texttt{D\_FDS}}
    This flag is different from the other flags, because it is
    used to change the formatting of all log messages that are printed,
    as opposed to specifying what kinds of messages should be printed.
    If \Dflag{FDS} is set, HTCondor will always print out the file descriptor
    that the open of the log file was allocated by the operating system.
    This can be helpful in debugging HTCondor's use of system file
    descriptors as it will generally track the number of file descriptors
    that HTCondor has open.

  \label{dflag:category}
  \item[\Dflag{CATEGORY}]
    \index{SUBSYS\_DEBUG macro levels@\texttt{<SUBSYS>\_DEBUG} macro levels!D\_CATEGORY@\texttt{D\_CATEGORY}}
    This flag is different from the other flags, because it is
    used to change the formatting of all log messages that are printed,
    as opposed to specifying what kinds of messages should be printed.
    If \Dflag{CATEGORY} is set, Condor will include the debugging level flags 
    that were in effect for each line of output.  
    This may be used to filter log output by the level or 
    tag it, for example, identifying all
    logging output at level \Dflag{SECURITY}, or \Dflag{ACCOUNTANT}.

  \end{description}

\label{param:AllDebug}
\item[\Macro{ALL\_DEBUG}]
  Used to make all subsystems
  share a debug flag. Set the parameter \MacroNI{ALL\_DEBUG}
  instead of changing all of the individual parameters.  For example,
  to turn on all debugging in all subsystems, set
  \verb$ALL_DEBUG = D_ALL$.

\label{param:ToolDebug}
\item[\Macro{TOOL\_DEBUG}]
  Uses the same values (debugging levels) as \MacroNI{<SUBSYS>\_DEBUG} to
  describe the amount of debugging information sent to \File{stderr} 
  for HTCondor tools.

\end{description}

Log files may optionally be specified per debug level as follows:
\begin{description}

\label{param:SubsysLevelLog}
\item[\MacroB{<SUBSYS>\_<LEVEL>\_LOG}]
\index{SUBSYS\_LEVEL\_LOG macro@\texttt{<SUBSYS>\_<LEVEL>\_LOG} macro}
  The name of a log file for messages at a specific debug level for a
  specific subsystem.  
  \verb@<LEVEL>@ is defined by any debug level,
  but without the \verb@D_@ prefix.
  See section~\ref{list:debug-level-description} for the list of debug levels.
  If the debug level is included in
  \MacroUNI{<SUBSYS>\_DEBUG}, then all messages of this debug level will be
  written both to the log file defined by \MacroNI{<SUBSYS>\_LOG} and the
  the log file defined by \MacroNI{<SUBSYS>\_<LEVEL>\_LOG}.  As examples,
  \MacroNI{SHADOW\_SYSCALLS\_LOG} specifies a log file for all remote
  system call debug messages,
  and \Macro{NEGOTIATOR\_MATCH\_LOG} specifies a log file that only captures
  \Condor{negotiator} debug events occurring with matches.

\label{param:MaxSubsysLevelLog}
\item[\Macro{MAX\_<SUBSYS>\_<LEVEL>\_LOG}]
  See section~\ref{param:MaxSubsysLog}, the definition of
  \MacroNI{MAX\_<SUBSYS>\_LOG}.

\label{param:TruncSubsysLevelLogOnOpen}
\item[\Macro{TRUNC\_<SUBSYS>\_<LEVEL>\_LOG\_ON\_OPEN}]
  Similar to \Macro{TRUNC\_<SUBSYS>\_LOG\_ON\_OPEN}.

\end{description}

The following macros control where and what is written to the 
event log,
a file that receives job user log events, 
but across all users and user's jobs.

\begin{description}

\label{param:EventLog}
\item[\Macro{EVENT\_LOG}]
  The full path and file name of the event log.
  There is no default value for this variable,
  so no event log will be written, if not defined.

\label{param:EventLogMaxSize}
\item[\Macro{EVENT\_LOG\_MAX\_SIZE}]
  Controls the maximum length in bytes to which the event log
  will be allowed to grow. The log file will grow to the specified length,
  then be saved to a file with the suffix .old.
  The .old  files are overwritten each time the log is saved.
  A value of 0 specifies that the file may grow without bounds (and
  disables rotation).   The default is 1 Mbyte.
  For backwards compatibility, \MacroNI{MAX\_EVENT\_LOG} will be used if
  \MacroNI{EVENT\_LOG\_MAX\_SIZE} is not defined.
  If \MacroNI{EVENT\_LOG} is not defined, this parameter has no effect.

\label{param:MaxEventLog}
\item[\Macro{MAX\_EVENT\_LOG}]
  See \MacroNI{EVENT\_LOG\_MAX\_SIZE}.

\label{param:EventLogMaxRotations}
\item[\Macro{EVENT\_LOG\_MAX\_ROTATIONS}]
  Controls the maximum number of rotations of the event log that
  will be stored.  If this value is 1 (the default), the event log
  will be rotated to a ``.old'' file as described above.  However, if
  this is greater than 1, then multiple rotation files will be stores,
  up to \MacroNI{EVENT\_LOG\_MAX\_ROTATIONS} of them.  These files
  will be named, instead of the ``.old'' suffix, ``.1'', ``.2'', with
  the ``.1'' being the most recent rotation.  This is an integer
  parameter with a default value of 1.
  If \MacroNI{EVENT\_LOG} is not defined, or if
  \MacroNI{EVENT\_LOG\_MAX\_SIZE} has a value of 0 (which disables
  event log rotation), this parameter has no effect.

\label{param:EventLogRotationLock}
\item[\Macro{EVENT\_LOG\_ROTATION\_LOCK}]
  Controls the lock file that will be used to ensure that, when
  rotating files, the rotation is done by a single process.  This is a
  string parameter; it's default value is the file path of the
  event log itself, with a ``.lock'' appended.
  If \MacroNI{EVENT\_LOG} is not defined, or if
  \MacroNI{EVENT\_LOG\_MAX\_SIZE} has a value of 0 (which disables
  event log rotation), this parameter has no effect.

\label{param:EventLogFsync}
\item[\Macro{EVENT\_LOG\_FSYNC}]
  A boolean value that controls whether HTCondor will perform an
  \Procedure{fsync} after writing each event to the event log.
  When \Expr{True},
  an \Procedure{fsync} operation is performed after each event.
  This \Procedure{fsync} operation forces the operating system to
  synchronize the updates to the event log to the disk, but can
  negatively affect the performance of the system.  
  Defaults to \Expr{False}.

\label{param:EventLogLocking}
\item[\Macro{EVENT\_LOG\_LOCKING}]
  A boolean value that defaults to \Expr{True}.
  When \Expr{True},
  the event log (as specified by \MacroNI{EVENT\_LOG})
  will be locked before being written to.
  When \Expr{False}, HTCondor does not lock the file before writing.

\label{param:EventLogUseXML}
\item[\Macro{EVENT\_LOG\_USE\_XML}]
  A boolean value that defaults to \Expr{False}.
  When \Expr{True}, events are logged in XML format.
  If \MacroNI{EVENT\_LOG} is not defined, this parameter has no effect.

\label{param:EventLogJobAdInformationAttrs}
\item[\Macro{EVENT\_LOG\_JOB\_AD\_INFORMATION\_ATTRS}]
  A comma separated list of job ClassAd attributes,
  whose evaluated values form a new event, the \Expr{JobAdInformationEvent},
  given Event Number 028.
  This new event is placed in the event log in addition to each logged event.
  If \MacroNI{EVENT\_LOG} is not defined, this configuration variable
  has no effect.
  This configuration variable is the same as the job ClassAd attribute
  \AdAttr{JobAdInformationAttrs} (see
  page~\pageref{JobAdInformationAttrs-job-attribute}),
  but it applies to the system Event Log rather than the user job log.

\end{description}

%%%%%%%%%%%%%%%%%%%%%%%%%%%%%%%%%%%%%%%%%%%%%%%%%%%%%%%%%%%%%%%%%%%%%%%%%%%
\subsection{\label{sec:DaemonCore-Config-File-Entries}DaemonCore Configuration File Entries} 
%%%%%%%%%%%%%%%%%%%%%%%%%%%%%%%%%%%%%%%%%%%%%%%%%%%%%%%%%%%%%%%%%%%%%%%%%%%

\index{configuration!DaemonCore configuration variables}
Please read section~\ref{sec:DaemonCore} for details
on DaemonCore.  There are certain configuration file settings that
DaemonCore uses which affect all HTCondor daemons (except the checkpoint
server, standard universe shadow, and standard universe starter, none of
which use DaemonCore).
\begin{description}

\label{param:HostAllow}
\item[\Macro{HOSTALLOW\Dots}]
  All macros that begin with either \Macro{HOSTALLOW} or
  \Macro{HOSTDENY} are settings for HTCondor's host-based security.
  See section~\ref{sec:Host-Security} on Setting up
  IP/host-based security in HTCondor for details on these
  macros and how to configure them.

\label{param:EnableRuntimeConfig}
\item[\Macro{ENABLE\_RUNTIME\_CONFIG}]
  The \Condor{config\_val} tool has an option \Opt{-rset} for
  dynamically setting run time configuration values, and which only affect
  the in-memory configuration variables.
  Because of the potential security implications of this feature, by
  default, HTCondor daemons will not honor these requests.
  To use this functionality, HTCondor administrators must specifically
  enable it by setting \MacroNI{ENABLE\_RUNTIME\_CONFIG} to \Expr{True}, and
  specify what configuration variables can be changed using the
  \MacroNI{SETTABLE\_ATTRS\Dots} family of configuration options.
  Defaults to \Expr{False}.

\label{param:EnablePersistentConfig}
\item[\Macro{ENABLE\_PERSISTENT\_CONFIG}]
  The \Condor{config\_val} tool has a \Opt{-set} option for
  dynamically setting persistent configuration values.
  These values override options in the normal HTCondor configuration
  files.
  Because of the potential security implications of this feature, by
  default, HTCondor daemons will not honor these requests.
  To use this functionality, HTCondor administrators must specifically
  enable it by setting \MacroNI{ENABLE\_PERSISTENT\_CONFIG} to \Expr{True},
  creating a directory where the HTCondor daemons will hold these
  dynamically-generated persistent configuration files (declared using
  \MacroNI{PERSISTENT\_CONFIG\_DIR}, described below) and specify what
  configuration variables can be changed using the
  \MacroNI{SETTABLE\_ATTRS\Dots} family of configuration options.
  Defaults to \Expr{False}.

\label{param:PersistentConfigDir}
\item[\Macro{PERSISTENT\_CONFIG\_DIR}]
  Directory where daemons should store dynamically-generated
  persistent configuration files (used to support
  \Condor{config\_val} \Opt{-set})
  This directory should \Bold{only} be writable by root, or the user
  the HTCondor daemons are running as (if non-root).
  There is no default, administrators that wish to use this
  functionality must create this directory and define this setting.
  This directory must not be shared by multiple HTCondor installations,
  though it can be shared by all HTCondor daemons on the same host.
  Keep in mind that this directory should not be placed on an NFS
  mount where ``root-squashing'' is in effect, or else HTCondor daemons
  running as root will not be able to write to them.
  A directory (only writable by root) on the local file system is
  usually the best location for this directory.

\label{param:SettableAttrs}
\item[\Macro{SETTABLE\_ATTRS\Dots}]
  All macros that begin with \Macro{SETTABLE\_ATTRS} or
  \MacroNI{<SUBSYS>.SETTABLE\_ATTRS} are settings used to restrict the 
  configuration values that can be changed using the \Condor{config\_val} 
  command.
  Section~\ref{sec:Host-Security} on Setting up
  IP/Host-Based Security in HTCondor for details on these
  macros and how to configure them.  
  In particular, section~\ref{sec:Host-Security}
  on page~\pageref{sec:Host-Security} contains details specific to
  these macros.

\label{param:ShutdownGracefulTimeout}
\item[\Macro{SHUTDOWN\_GRACEFUL\_TIMEOUT}]
  Determines how long
  HTCondor will allow daemons try their graceful shutdown methods
  before they do a hard shutdown.  It is defined in terms of seconds.
  The default is 1800 (30 minutes).

\label{param:SubsysAddressFile}
\item[\MacroB{<SUBSYS>\_ADDRESS\_FILE}]
  \index{SUBSYS\_ADDRESS\_FILE macro@\texttt{<SUBSYS>\_ADDRESS\_FILE} macro}
  \index{NEGOTIATOR\_ADDRESS\_FILE macro@\texttt{NEGOTIATOR\_ADDRESS\_FILE} macro}
  \index{configuration macro!\texttt{NEGOTIATOR\_ADDRESS\_FILE}}
  \index{COLLECTOR\_ADDRESS\_FILE macro@\texttt{COLLECTOR\_ADDRESS\_FILE} macro}
  \index{configuration macro!\texttt{COLLECTOR\_ADDRESS\_FILE}}
  A complete path to a file that is to contain an
  IP address and port number for a daemon. 
  Every HTCondor daemon that uses
  DaemonCore has a command port where commands are sent.
  The IP/port of the daemon is put in that daemon's ClassAd,
  so that other machines in the pool can query the
  \Condor{collector} (which listens on a well-known port)
  to find the address of a given daemon on a given machine.
  When tools and daemons are all executing on the same
  single machine, communications do not require a query of the
  \Condor{collector} daemon.
  Instead, they look in a file on the local disk
  to find the IP/port.
  This macro causes daemons to write the
  IP/port of their command socket to a specified file.
  In this way,
  local tools will continue to operate,
  even if the machine running the \Condor{collector} crashes.
  Using this file will also generate
  slightly less network traffic in the pool,
  since tools including \Condor{q} and
  \Condor{rm} do not need to send any messages over the network to
  locate the \Condor{schedd} daemon.
  This macro is not necessary for the \Condor{collector} 
  daemon, since its command socket is at a well-known port.  
  
  The macro is named by substituting \MacroNI{<SUBSYS>}
  with the appropriate subsystem string as defined in
  section~\ref{sec:HTCondor-Subsystem-Names}.
  
\label{param:SubsysDaemonAdFile}
\item[\MacroB{<SUBSYS>\_DAEMON\_AD\_FILE}]
  \index{SUBSYS\_DAEMON\_AD\_FILE macro@\texttt{<SUBSYS>\_DAEMON\_AD\_FILE} macro}
  A complete path to a file that is to contain the ClassAd for a daemon.
  When the daemon sends a ClassAd describing itself to the
  \Condor{collector}, it will also place a copy of the ClassAd in this
  file. Currently, this setting only works for the \Condor{schedd}.

\index{SUBSYS\_ATTRS macro@\texttt{<SUBSYS>\_ATTRS} macro}
\index{SUBSYS\_EXPRS macro@\texttt{<SUBSYS>\_EXPRS} macro}
\MacroIndex{STARTD\_ATTRS}
\MacroIndex{STARTD\_EXPRS}
\label{param:SubsysExprs}
\item[\MacroB{<SUBSYS>\_ATTRS} or
\label{param:SubsysAttrs}
\MacroB{<SUBSYS>\_EXPRS}]
  Allows any DaemonCore daemon to advertise arbitrary
  expressions from the configuration file in its ClassAd.  Give the
  comma-separated list of entries from the configuration file you want in the
  given daemon's ClassAd.
  Frequently used to add attributes to machines so that the
  machines can discriminate between other machines in a job's 
  \Opt{rank} and \Opt{requirements}.

  The macro is named by substituting \MacroNI{<SUBSYS>}
  with the appropriate subsystem string as defined in
  section~\ref{sec:HTCondor-Subsystem-Names}.

  \MacroNI{<SUBSYS>\_EXPRS} is a historic setting that functions identically to
  \MacroNI{<SUBSYS>\_ATTRS}. Use \MacroNI{<SUBSYS>\_ATTRS}.

  \Note The \Condor{kbdd} does not send
  ClassAds now, so this entry does not affect it.  The
  \Condor{startd}, \Condor{schedd}, \Condor{master}, and
  \Condor{collector} do send ClassAds, so those would be valid
  subsystems to set this entry for.
  
  \MacroNI{SUBMIT\_EXPRS} not part of the \MacroNI{<SUBSYS>\_EXPRS}, it is
  documented in section~\ref{sec:Submit-Config-File-Entries}

  Because of the different syntax of the configuration
  file and ClassAds, a little extra work is required to get a
  given entry into a ClassAd.  In particular, ClassAds require quote
  marks (") around strings.  Numeric values and boolean expressions
  can go in directly.  
  For example, if the \Condor{startd} is to advertise a string macro, a numeric
  macro, and a boolean expression, do something similar to:

  \begin{verbatim}
    STRING = This is a string 
    NUMBER = 666
    BOOL1 = True
    BOOL2 = CurrentTime >= $(NUMBER) || $(BOOL1)
    MY_STRING = "$(STRING)"
    STARTD_ATTRS = MY_STRING, NUMBER, BOOL1, BOOL2
  \end{verbatim}

\label{param:DaemonShutdown}
\item[\Macro{DAEMON\_SHUTDOWN}]
  Starting with HTCondor version 6.9.3, whenever a daemon is about to
  publish a ClassAd update to the \Condor{collector}, it will evaluate
  this expression.
  If it evaluates to \Expr{True}, the daemon will gracefully shut itself down,
  exit with the exit code 99,
  and will not be restarted by the \Condor{master} (as if it sent
  itself a \Condor{off} command).
  The expression is evaluated in the context of the ClassAd that is
  being sent to the \Condor{collector}, so it can reference any
  attributes that can be seen with
  \verb@condor_status -long [-daemon_type]@ (for example,
  \verb@condor_status -long [-master]@ for the \Condor{master}).
  Since each daemon's ClassAd will contain different attributes,
  administrators should define these shutdown expressions specific to
  each daemon, for example:
  \begin{verbatim}
    STARTD.DAEMON_SHUTDOWN = when to shutdown the startd
    MASTER.DAEMON_SHUTDOWN = when to shutdown the master
  \end{verbatim}
  Normally, these expressions would not be necessary, so if not
  defined, they default to FALSE.

  \Note This functionality does not work in conjunction with HTCondor's
  high-availability support (see section~\ref{sec:high-availability}
  on page~\pageref{sec:high-availability} for more information).
  If you enable high-availability for a particular daemon, you should
  not define this expression.

\label{param:DaemonShutdownFast}
\item[\Macro{DAEMON\_SHUTDOWN\_FAST}]
  Identical to \MacroNI{DAEMON\_SHUTDOWN} (defined above), except the
  daemon will use the fast shutdown mode (as if it sent itself a
  \Condor{off} command using the \Opt{-fast} option).

\label{param:UseCloneToCreateProcesses}
\item[\Macro{USE\_CLONE\_TO\_CREATE\_PROCESSES}]
  A boolean value that controls how an HTCondor daemon creates a new process on
  Linux platforms. If set to the default value of \Expr{True},
  the \Expr{clone} system call is used. Otherwise, the \Expr{fork} system
  call is used. \Expr{clone} provides scalability improvements for daemons
  using a large amount of memory, for example, a \Condor{schedd} with a lot of
  jobs in the queue. Currently, the use of \Expr{clone} is available on
  Linux systems.  If
  HTCondor detects that it is running under the \Prog{valgrind} analysis tools,
  this setting is ignored and treated as \Expr{False}, to work around
  incompatibilities.

\label{param:NotRespondingTimeout}
\item[\Macro{NOT\_RESPONDING\_TIMEOUT}]
  When an HTCondor daemon's parent process is another HTCondor daemon, 
  the child daemon will
  periodically send a short message to its parent stating that it is alive
  and well. If the parent does not hear from the child for a while,
  the parent assumes that the child is hung,
  kills the child, and restarts the child. This parameter
  controls how long the parent waits before killing the child. It is defined
  in terms of seconds and defaults to 3600 (1 hour). The child sends its
  alive and well messages at an interval of one third of this value.

\label{param:SubsysNotRespondingTimeout}
\item[\MacroB{<SUBSYS>\_NOT\_RESPONDING\_TIMEOUT}]
  Identical to \MacroNI{NOT\_RESPONDING\_TIMEOUT}, but controls the timeout
  for a specific type of daemon. For example,
  \MacroNI{SCHEDD\_NOT\_RESPONDING\_TIMEOUT} controls how long the
  \Condor{schedd}'s parent daemon will wait without receiving an 
  alive and well
  message from the \Condor{schedd} before killing it.

\label{param:NotRespondingWantCore}
\item[\Macro{NOT\_RESPONDING\_WANT\_CORE}]
  A boolean value with a default value of \Expr{False}.
  This parameter is for debugging purposes on Unix systems, and it
  controls the behavior of the parent process when the parent process
  determines that a child process is not responding.
  If \MacroNI{NOT\_RESPONDING\_WANT\_CORE} is \Expr{True}, the parent 
  will send a SIGABRT instead of SIGKILL to the child process.
  If the child process is configured with the configuration variable
  \MacroNI{CREATE\_CORE\_FILES} enabled, the child process will then
  generate a core dump.
  See \MacroNI{NOT\_RESPONDING\_TIMEOUT} on page
  \pageref{param:NotRespondingTimeout}, and 
  \MacroNI{CREATE\_CORE\_FILES} on page
  \pageref{param:CreateCoreFiles} for related details.

\label{param:LockFileUpdateInterval}
\item[\Macro{LOCK\_FILE\_UPDATE\_INTERVAL}]
  An integer value representing seconds,
  controlling how often valid lock files should have their on disk
  timestamps updated. Updating the timestamps prevents administrative programs,
  such as \Prog{tmpwatch}, from deleting long lived lock files.
  If set to a value less than 60, the update time will be 60 seconds.
  The default value is 28800, which is 8 hours. 
  This variable only takes effect at the start or restart of a daemon.

\label{param:MaxAcceptsPerCycle}
\item[\Macro{MAX\_ACCEPTS\_PER\_CYCLE}]
  An integer value that defaults to 4.
  It is a limit on the number of accepts of new, incoming, 
  socket connect requests per DaemonCore event cycle.
  It has the most noticeable effect on the \Condor{schedd},
  and would be given a higher integer value for tuning purposes when 
  there is a high number of jobs starting and exiting per second.

\end{description}

%%%%%%%%%%%%%%%%%%%%%%%%%%%%%%%%%%%%%%%%%%%%%%%%%%%%%%%%%%%%%%%%%%%%%%%%%%%
\subsection{\label{sec:Network-Related-Config-File-Entries}Network-Related Configuration File Entries}
%%%%%%%%%%%%%%%%%%%%%%%%%%%%%%%%%%%%%%%%%%%%%%%%%%%%%%%%%%%%%%%%%%%%%%%%%%%
\index{configuration!network-related configuration variables}

More information about networking in HTCondor can be found in
section~\ref{sec:Networking} on page~\pageref{sec:Networking}.

\begin{description}

\label{param:BindAllInterfaces}
\item[\Macro{BIND\_ALL\_INTERFACES}]
  For systems with multiple network interfaces, if this configuration
  setting is \Expr{False}, HTCondor will only bind network sockets to 
  the IP address specified with
  \MacroNI{NETWORK\_INTERFACE} (described below).  If set to \Expr{True},
  the default value, HTCondor will listen on all interfaces.
  However, currently HTCondor is still only able to advertise a single
  IP address, even if it is listening on multiple interfaces.  By
  default, it will advertise the IP address of the network interface
  used to contact the collector, since this is the most likely to be
  accessible to other processes which query information from the same
  collector.
  More information about using this setting can be found in
  section~\ref{sec:Using-BindAllInterfaces} on
  page~\pageref{sec:Using-BindAllInterfaces}. 

\label{param:CcbAddress}
\item[\Macro{CCB\_ADDRESS}] This is the address of a
  \Condor{collector} that will serve as this daemon's HTCondor
  Connection Broker (CCB).  Multiple addresses may be listed
  (separated by commas and/or spaces) for redundancy.  The CCB server
  must authorize this daemon at DAEMON level for this configuration to
  succeed.  It is highly recommended to also configure
  \Macro{PRIVATE\_NETWORK\_NAME} if you configure \Macro{CCB\_ADDRESS}
  so communications originating within the same private network do not
  need to go through CCB.  For more information about CCB,
  see page~\pageref{sec:CCB}.

\label{param:CcbHeartbeatInterval}
\item[\Macro{CCB\_HEARTBEAT\_INTERVAL}] This is the maximum
  number of seconds of silence on a daemon's connection to the CCB server
  after which it will ping the server to verify that the connection still
  works.  The default is 20 minutes.  This feature serves to both speed
  up detection of dead connections and to generate a guaranteed minimum
  frequency of activity to attempt to prevent the connection from being
  dropped.  The special value 0 disables the heartbeat.  The heartbeat
  is automatically disabled if the CCB server is older than 7.5.0.

\label{param:UseSharedPort}
\item[\Macro{USE\_SHARED\_PORT}] A boolean value that
  specifies whether an HTCondor process should rely on
  \Condor{shared\_port} for receiving incoming connections.  
  Under Unix, write access to the location defined by 
  \Macro{DAEMON\_SOCKET\_DIR} is required for this to take
  effect.  The default is \Expr{False}.  If set to \Expr{True},
  \MacroNI{SHARED\_PORT} should be added to \MacroNI{DAEMON\_LIST}.
  For more information about using
  a shared port, see page~\pageref{sec:Config-shared-port}.

\label{param:SubsysMaxFileDescriptors}
\item[\MacroB{<SUBSYS>\_MAX\_FILE\_DESCRIPTORS}]
\index{SUBSYS\_MAX\_FILE\_DESCRIPTORS macro@\texttt{<SUBSYS>\_MAX\_FILE\_DESCRIPTORS}}
This setting is identical to \MacroNI{MAX\_FILE\_DESCRIPTORS}, but it
only applies to a specific condor subsystem.  If the
subsystem-specific setting is unspecified, \MacroNI{MAX\_FILE\_DESCRIPTORS}
is used.

\label{param:MaxFileDescriptors}
\item[\Macro{MAX\_FILE\_DESCRIPTORS}] Under Unix, this specifies the
maximum number of file descriptors to allow the HTCondor daemon to use.
File descriptors are a system resource used for open files and for
network connections.  HTCondor daemons that make many simultaneous
network connections may require an increased number of file
descriptors.  For example, see page~\pageref{sec:CCB} for information
on file descriptor requirements of CCB.  Changes to this configuration
variable require a restart of HTCondor in order to take effect.  Also note
that only if HTCondor is running as root will it be able to increase the
limit above the hard limit (on maximum open files) that it inherits.

\label{param:NetworkInterface}
\item[\Macro{NETWORK\_INTERFACE}]
  An IP address of the form \Expr{123.123.123.123} or the name
  of a network device, as in the example \Expr{eth0}.
  The wild card character (\Expr{*}) may be used within either.
  For example, \Expr{123.123.*} would match a network interface
  with an IP address of \Expr{123.123.123.123} or \Expr{123.123.100.100}.
  The default value is \Expr{*}, which matches all network interfaces.

  The effect of this variable depends on the value of
  \MacroNI{BIND\_ALL\_INTERFACES}.  There are two cases:

  If \MacroNI{BIND\_ALL\_INTERFACES} is \Expr{True} (the default),
  \MacroNI{NETWORK\_INTERFACE} controls what IP address
  will be advertised as the public address of the daemon.  
  If multiple network interfaces match the value and
  \MacroNI{ENABLE\_ADDRESS\_REWRITING} is \Expr{True} (the default),
  the IP address that is chosen to be advertised will be the one that
  is used to communicate with the \Condor{collector}.
  If \MacroNI{ENABLE\_ADDRESS\_REWRITING} is \Expr{False}, the IP address
  that is chosen to be advertised will be the one associated with the
  first device (in system-defined order) that is in a public address
  space, or a private address space, or a loopback address, in that
  order of preference.  
  If it is desired to advertise an IP address
  that is not associated with any local network interface, 
  for example, when TCP forwarding is being used,
  then \Macro{TCP\_FORWARDING\_HOST} should be used instead of
  \MacroNI{NETWORK\_INTERFACE}.

  If \MacroNI{BIND\_ALL\_INTERFACES} is \Expr{False},
  then \MacroNI{NETWORK\_INTERFACE} specifies which IP address HTCondor
  should use for all incoming and outgoing communication.
  If more than one IP address matches the value, 
  then the IP address that is
  chosen will be the one associated with the first device (in
  system-defined order) that is in a public address space, or a
  private address space, or a loopback address, in that order of
  preference.

  More information about configuring HTCondor on machines with multiple
  network interfaces can be found in
  section~\ref{sec:Multiple-Interfaces} on
  page~\pageref{sec:Multiple-Interfaces}.

\label{param:PrivateNetworkName}
\item[\Macro{PRIVATE\_NETWORK\_NAME}]
  If two HTCondor daemons are trying to communicate with each other, and
  they both belong to the same private network, this setting will
  allow them to communicate directly using the private network
  interface, instead of having to use CCB or to go through a public IP address.
  Each private network should be assigned a unique network name.
  This string can have any form, but it must be unique for a
  particular private network.
  If another HTCondor daemon or tool is configured with the same
  \MacroNI{PRIVATE\_NETWORK\_NAME}, it will attempt to contact this
  daemon using its private network address.
  Even for sites using CCB, this is an important optimization, since
  it means that two daemons on the same network can communicate
  directly, without having to go through the broker.
  If CCB is enabled, and the \MacroNI{PRIVATE\_NETWORK\_NAME} is
  defined, the daemon's private address will be defined automatically.
  Otherwise, you can specify a particular private IP address to use by
  defining the \MacroNI{PRIVATE\_NETWORK\_INTERFACE} setting
  (described below).
  There is no default for this setting.
  After changing this setting and running \Condor{reconfig}, it may
  take up to one \Condor{collector} update interval before the change becomes visible.
  % DWW
  % Described in default config file: NO
  % Defined in the default config file: NO
  % Default definition in config file: N/A
  % Result if not defined or RHS is empty: No change in behavior.

\label{param:PrivateNetworkInterface}
\item[\Macro{PRIVATE\_NETWORK\_INTERFACE}]
  For systems with multiple network interfaces, if this configuration
  setting and \MacroNI{PRIVATE\_NETWORK\_NAME} are both defined,
  HTCondor daemons will advertise some additional attributes in their
  ClassAds to help other HTCondor daemons and tools in the same private
  network to communicate directly.

  \MacroNI{PRIVATE\_NETWORK\_INTERFACE} defines what IP address
  of the form \Expr{123.123.123.123} or name
  of a network device (as in the example \Expr{eth0})
  a given multi-homed machine should use for the private network.
  The asterisk (\verb|*|) may be used as a wild card character within either
  the IP address or the device name.
  If another HTCondor daemon or tool is configured with the same
  \MacroNI{PRIVATE\_NETWORK\_NAME}, it will attempt to contact this
  daemon using the IP address specified here.
  The syntax for specifying an IP address is identical to 
  \MacroNI{NETWORK\_INTERFACE}.
  Sites using CCB only need to define
  the \MacroNI{PRIVATE\_NETWORK\_NAME},
  and the \MacroNI{PRIVATE\_NETWORK\_INTERFACE} will be defined automatically.
  Unless CCB is enabled, there is no default value for this variable.
  After changing this variable and running \Condor{reconfig}, 
  it may take up to one \Condor{collector} update interval 
  before the change becomes visible.

\label{param:TcpForwardingHost}
\item[\Macro{TCP\_FORWARDING\_HOST}]
  This specifies the host or IP address that should be used as the
  public address of this daemon.  If a host name is specified, be aware
  that it will be resolved to an IP address by this daemon, not by the clients
  wishing to connect to it.  It is the IP address that is advertised, not
  the host name.  This setting is useful if HTCondor on this
  host may be reached through a NAT or firewall by connecting to an
  IP address that forwards connections to this host.  It
  is assumed that the port number on the \MacroNI{TCP\_FORWARDING\_HOST}
  that forwards to this host is the same port number assigned to
  HTCondor on this host.  This option could also be used when ssh port
  forwarding is being used.  In this case, the incoming addresses
  of connections to this daemon will appear as though they are coming
  from the forwarding host rather than from the real remote host, so any
  authorization settings that rely on host addresses should be
  considered accordingly.

\label{param:EnableAddressRewriting}
\item[\Macro{ENABLE\_ADDRESS\_REWRITING}]
  A boolean value that defaults to \Expr{True}.  When
  \MacroNI{NETWORK\_INTERFACE} matches only one IP address or
  \MacroNI{TCP\_FORWARDING\_HOST} is defined or
  \MacroNI{NET\_REMAP\_ENABLE} is \Expr{True}, this setting has no effect and
  the behavior is as though it had been set to \Expr{False}.  When \Expr{True},
  IP addresses published by HTCondor daemons are automatically rewritten to
  match the IP address of the network interface used to make the
  publication.  For example, if the \Condor{schedd} advertises itself to
  two pools via flocking, and the \Condor{collector} for one pool is reached
  by the \Condor{schedd} through a private network interface, while
  the \Condor{collector} for the other pool is reached through a different
  network interface, the IP address published by the \Condor{schedd}
  daemon will match the address of the respective network interfaces
  used in the two cases.  The intention is to make it easier for
  HTCondor daemons to operate in a multi-homed environment.

\label{param:HighPort}
\item[\Macro{HIGHPORT}]
  Specifies an upper limit of given port numbers for HTCondor to use,
  such that HTCondor is restricted to a range of port numbers.
  If this macro is not explicitly specified, then HTCondor will
  not restrict the port numbers that it uses. HTCondor will use
  system-assigned port numbers.
  For this macro to work, both \MacroNI{HIGHPORT} and
  \MacroNI{LOWPORT} (given below) must be defined.
  % PKK
  % Described in default config file: YES
  % Defined in the default config file: NO
  % Default definition in config file: bogus
  % Result if not defined or RHS is empty: HTCondor uses any ports available,
  % regardless of the LOWPORT setting

\label{param:LowPort}
\item[\Macro{LOWPORT}]
  Specifies a lower limit of given port numbers for HTCondor to use,
  such that HTCondor is restricted to a range of port numbers.
  If this macro is not explicitly specified, then HTCondor will
  not restrict the port numbers that it uses. HTCondor will use
  system-assigned port numbers.
  For this macro to work, both \MacroNI{HIGHPORT} (given above) and
  \MacroNI{LOWPORT} must be defined.
  % PKK
  % Described in default config file: YES
  % Defined in the default config file: NO
  % Default definition in config file: bogus
  % Result if not defined or RHS is empty: HTCondor uses any ports available,
  % regardless of the HIGHPORT setting

\label{param:InLowPort}
\item[\Macro{IN\_LOWPORT}]
  An integer value that specifies a lower limit of given port numbers
  for HTCondor to use on incoming connections (ports for listening),
  such that HTCondor is restricted to a range of port numbers.
  This range implies the use of both \MacroNI{IN\_LOWPORT} and
  \MacroNI{IN\_HIGHPORT}.
  A range of port numbers less than 1024 may be used for daemons 
  running as root.
  Do not specify \MacroNI{IN\_LOWPORT} in combination with 
  \MacroNI{IN\_HIGHPORT} such that the range crosses the port 1024
  boundary.
  Applies only to Unix machine configuration.
  Use of \MacroNI{IN\_LOWPORT} and \MacroNI{IN\_HIGHPORT} overrides
  any definition of \MacroNI{LOWPORT} and \MacroNI{HIGHPORT}.

\label{param:InHighPort}
\item[\Macro{IN\_HIGHPORT}]
  An integer value that specifies an upper limit of given port numbers
  for HTCondor to use on incoming connections (ports for listening),
  such that HTCondor is restricted to a range of port numbers.
  This range implies the use of both \MacroNI{IN\_LOWPORT} and
  \MacroNI{IN\_HIGHPORT}.
  A range of port numbers less than 1024 may be used for daemons 
  running as root.
  Do not specify \MacroNI{IN\_LOWPORT} in combination with 
  \MacroNI{IN\_HIGHPORT} such that the range crosses the port 1024
  boundary.
  Applies only to Unix machine configuration.
  Use of \MacroNI{IN\_LOWPORT} and \MacroNI{IN\_HIGHPORT} overrides
  any definition of \MacroNI{LOWPORT} and \MacroNI{HIGHPORT}.

\label{param:OutLowPort}
\item[\Macro{OUT\_LOWPORT}]
  An integer value that specifies a lower limit of given port numbers
  for HTCondor to use on outgoing connections,
  such that HTCondor is restricted to a range of port numbers.
  This range implies the use of both \MacroNI{OUT\_LOWPORT} and
  \MacroNI{OUT\_HIGHPORT}.
  A range of port numbers less than 1024 is inappropriate, as
  not all daemons and tools will be run as root.
  Applies only to Unix machine configuration.
  Use of \MacroNI{OUT\_LOWPORT} and \MacroNI{OUT\_HIGHPORT} overrides
  any definition of \MacroNI{LOWPORT} and \MacroNI{HIGHPORT}.

\label{param:OutHighPort}
\item[\Macro{OUT\_HIGHPORT}]
  An integer value that specifies an upper limit of given port numbers
  for HTCondor to use on outgoing connections,
  such that HTCondor is restricted to a range of port numbers.
  This range implies the use of both \MacroNI{OUT\_LOWPORT} and
  \MacroNI{OUT\_HIGHPORT}.
  A range of port numbers less than 1024 is inappropriate, as
  not all daemons and tools will be run as root.
  Applies only to Unix machine configuration.
  Use of \MacroNI{OUT\_LOWPORT} and \MacroNI{OUT\_HIGHPORT} overrides
  any definition of \MacroNI{LOWPORT} and \MacroNI{HIGHPORT}.

\label{param:UpdateCollectorWithTcp}
\item[\Macro{UPDATE\_COLLECTOR\_WITH\_TCP}]
  If your site needs to use TCP connections to send ClassAd updates to
  your collector, set to \Expr{True}
  to enable this feature.
  Please read section~\ref{sec:tcp-collector-update} on ``Using TCP to
  Send Collector Updates'' on page~\pageref{sec:tcp-collector-update}
  for more details and a discussion of when this
  functionality is needed. 
  At this time, this setting only affects the main \Condor{collector}
  for the site, not any sites that a \Condor{schedd} might flock to. 
  For large pools, it is also necessary to
  ensure that the collector has a high enough file descriptor limit
  (e.g. using \Macro{MAX\_FILE\_DESCRIPTORS}.
  Defaults to \Expr{False}.
  % PKK
  % Described in default config file: YES
  % Defined in the default config file: NO
  % Default definition in config file: bogus
  % Result if not defined or RHS is empty: disable the feature of TCP updates to
  % collectors
  % WARNING: The code also looks for UPDATE_COLLECTORS_WITH_TCP, which
  % means the SAME THING as UPDATE_COLLECTOR_WITH_TCP

\label{param:TcpUpdateCollectors}
\item[\Macro{TCP\_UPDATE\_COLLECTORS}]
  The list of collectors which will be updated with TCP instead of UDP.
  Please read section~\ref{sec:tcp-collector-update} on ``Using TCP to
  Send Collector Updates'' on page~\pageref{sec:tcp-collector-update}
  for more details and a discussion of when a site needs this
  functionality. 
  If not defined, no collectors use TCP instead of UDP.
  % PKK
  % Described in default config file: NO
  % Defined in the default config file: NO
  % Default definition in config file: N/A
  % Result if not defined or RHS is empty: no collectors are updated with TCP
  % WARNING: Look in Version 6.5.2 version history for more information about
  % this particular entry.

\label{param:SubsysTimeoutMultiplier}
\item[\MacroB{<SUBSYS>\_TIMEOUT\_MULTIPLIER}]
  \index{SUBSYS\_TIMEOUT\_MULTIPLIER macro@\texttt{<SUBSYS>\_TIMEOUT\_MULTIPLIER} macro}
  An integer value that defaults to 1.
  This value multiplies configured timeout values
  for all targeted subsystem communications,
  thereby increasing the time until a timeout occurs.
  This configuration variable is intended for use by developers for
  debugging purposes, where communication timeouts interfere.

\label{param:NonblockingCollectorUpdate}
\item[\Macro{NONBLOCKING\_COLLECTOR\_UPDATE}]
  A boolean value that defaults to \Expr{True}.
  When \Expr{True}, the establishment of TCP connections
  to the \Condor{collector} daemon
  for a security-enabled pool are done in a nonblocking manner.

\label{param:NegotiatorUseNonblockingStartdContact}
\item[\Macro{NEGOTIATOR\_USE\_NONBLOCKING\_STARTD\_CONTACT}]
  A boolean value that defaults to \Expr{True}.
  When \Expr{True}, the establishment of TCP connections
  from the \Condor{negotiator} daemon to the \Condor{startd} daemon
  for a security-enabled pool are done in a nonblocking manner.

\end{description}

%%%%%%%%%%%%%%%%%%%%%%%%%%%%%%%%%%%%%%%%%%%%%%%%%%%%%%%%%%%%%%%%%%%%%%%%%%%
\subsection{\label{sec:Shared-Filesystem-Config-File-Entries}Shared File System Configuration File Macros} 
%%%%%%%%%%%%%%%%%%%%%%%%%%%%%%%%%%%%%%%%%%%%%%%%%%%%%%%%%%%%%%%%%%%%%%%%%%%
\index{configuration!shared file system configuration variables}

These macros control how HTCondor interacts with various shared and
network file systems.  If you are using AFS as your shared file system,
be sure to read section~\ref{sec:HTCondor-AFS} on Using HTCondor with
AFS.
For information on submitting jobs under shared file systems,
see
section~\ref{sec:shared-fs}.
\begin{description}

\label{param:UidDomain}
\item[\Macro{UID\_DOMAIN}]
  The \MacroNI{UID\_DOMAIN} macro
  is used to decide under which user to run jobs.
  If the \MacroUNI{UID\_DOMAIN}
  on the submitting machine is different than
  the \MacroUNI{UID\_DOMAIN}
  on the machine that runs a job, then HTCondor runs
  the job as the user \Login{nobody}.
  For example, if the submit machine has
  a \MacroUNI{UID\_DOMAIN} of
  flippy.cs.wisc.edu, and the machine where the job will execute
  has a \MacroUNI{UID\_DOMAIN} of
  cs.wisc.edu, the job will run as user \Login{nobody}, because
  the two \MacroUNI{UID\_DOMAIN}s are not the same.
  If the \MacroUNI{UID\_DOMAIN}
  is the same on both the submit and execute machines,
  then HTCondor will run the job as the user that submitted the job.

  A further check attempts to assure that the submitting
  machine can not lie about its \MacroNI{UID\_DOMAIN}.
  HTCondor compares the 
  submit machine's claimed value for \MacroNI{UID\_DOMAIN}
  to its fully qualified name.
  If the two do not end the same, then the submit machine
  is presumed to be lying about its \MacroNI{UID\_DOMAIN}.
  In this case, HTCondor will run the job as user \Login{nobody}.
  For example, a job submission to the HTCondor pool at the UW Madison
  from flippy.example.com, claiming a \MacroNI{UID\_DOMAIN} of
  of cs.wisc.edu,
  will run the job as the user \Login{nobody}.

  Because of this verification,
  \MacroUNI{UID\_DOMAIN} must be a real domain name.
  At the Computer Sciences department
  at the UW Madison, we set the \MacroUNI{UID\_DOMAIN}
  to be cs.wisc.edu to
  indicate that whenever someone submits from a department machine, we
  will run the job as the user who submits it.

  Also see \MacroNI{SOFT\_UID\_DOMAIN}
  below for information about one more check
  that HTCondor performs before running a job as a given user.

  A few details:

  An administrator could set \MacroNI{UID\_DOMAIN}
  to *. This will match all domains,
  but it is a gaping security hole. It is not recommended.

  An administrator can also leave \MacroNI{UID\_DOMAIN} undefined.
  This will force HTCondor to always run jobs as user \Login{nobody}.
  Running standard universe jobs as user \Login{nobody} enhances
  security and should cause no problems, because the jobs use remote
  I/O to access all of their files.
  However, if vanilla jobs are run as
  user \Login{nobody}, then files that need to be accessed by the job will need
  to be marked as world readable/writable so the user \Login{nobody} can access
  them.

  When HTCondor sends e-mail about a job, HTCondor sends the e-mail to
  %% This is the wrong LaTeX  macro to use, but we don't have a correct one.
  \File{user@\$(UID\_DOMAIN)}.
  If \MacroNI{UID\_DOMAIN}
  is undefined, the e-mail is sent to \File{user@submitmachinename}.


\label{param:TrustUidDomain}
\item[\Macro{TRUST\_UID\_DOMAIN}]
  As an added security precaution when HTCondor is about to spawn a job,
  it ensures that the \MacroNI{UID\_DOMAIN} of a given
  submit machine is a substring of that machine's fully-qualified
  host name.
  However, at some sites, there may be multiple UID spaces that do
  not clearly correspond to Internet domain names.
  In these cases, administrators may wish to use names to describe the
  UID domains which are not substrings of the host names of the
  machines.
  For this to work, HTCondor must not do this regular security check.
  If the \MacroNI{TRUST\_UID\_DOMAIN} setting is defined to \Expr{True},
  HTCondor will not perform this test, and will trust whatever
  \MacroNI{UID\_DOMAIN} is presented by the submit machine when trying
  to spawn a job, instead of making sure the submit machine's host name
  matches the \MacroNI{UID\_DOMAIN}.
  When not defined, the default is \Expr{False},
  since it is more secure to perform this test. 
  % PKK
  % Described in default config file: YES
  % Defined in the default config file: NO
  % Default definition in config file: bogus
  % Result if not defined or RHS is empty: this entry is considered False

\label{param:SoftUidDomain}
\item[\Macro{SOFT\_UID\_DOMAIN}]
  A boolean variable that defaults to \Expr{False} when not defined.
  When HTCondor is about to run a job as a particular user 
  (instead of as user \Login{nobody}),
  it verifies that the UID given for the user is in the
  password file and actually matches the given user name.
  However, under installations that do not have every user
  in every machine's password file,
  this check will fail and the execution attempt will be aborted.
  To cause HTCondor not to do
  this check, set this configuration variable to \Expr{True}.
  HTCondor will then run the job under the user's UID.

\label{param:SlotNUser}
\item[\Macro{SLOT<N>\_USER}]
  The name of a user for HTCondor to use instead of
  user nobody,
  as part of a solution that plugs a security hole whereby
  a lurker process can prey on a subsequent job run as user name nobody. 
  \MacroNI{<N>} is an integer associated with slots.
  On Windows, \MacroNI{SLOT<N>\_USER}
  will only work if the credential of the specified
  user is stored on the execute machine using \Condor{store\_cred}.
  See Section~\ref{sec:RunAsNobody} for more information.

\label{param:StarterAllowRunAsOwner}
\item[\Macro{STARTER\_ALLOW\_RUNAS\_OWNER}]
  A boolean expression evaluated with the job ad as the
  target, that determines whether the job may run under the job owner's
  account (\Expr{True}) or whether it will run as \MacroNI{SLOT<N>\_USER} or
  nobody (\Expr{False}).  On Unix, this defaults to \Expr{True}.
  On Windows, it defaults to \Expr{False}.
  The job ClassAd may also contain the attribute
  \Attr{RunAsOwner} which is logically ANDed with the \Condor{starter} daemon's
  boolean value.  Under Unix, if the job does not specify it, this
  attribute defaults to \Expr{True}.
  Under Windows, the attribute defaults to \Expr{False}.
  In Unix, if the \Attr{UidDomain} of the machine and job do not
  match, then there is no possibility to run the job as the owner
  anyway, so, in that case, this setting has no effect.  See
  Section~\ref{sec:RunAsNobody} for more information.

\label{param:DedicatedExecuteAccountRegexp}
\item[\Macro{DEDICATED\_EXECUTE\_ACCOUNT\_REGEXP}]
  This is a regular expression (i.e. a string matching pattern) that
  matches the account name(s) that are dedicated to running condor
  jobs on the execute machine and which will never be used for more
  than one job at a time.  The default matches no account name.  If
  you have configured \MacroNI{SLOT<N>\_USER} to be a \emph{different}
  account for each HTCondor slot, and no non-condor processes will ever be
  run by these accounts, then this pattern should match the names of
  all \MacroNI{SLOT<N>\_USER} accounts.  Jobs run under a dedicated
  execute account are reliably tracked by HTCondor, whereas other jobs,
  may spawn processes that HTCondor fails to detect.  Therefore, a
  dedicated execution account provides more reliable tracking of CPU
  usage by the job and it also guarantees that when the job exits, no
  ``lurker'' processes are left behind.  When the job exits, condor
  will attempt to kill all processes owned by the dedicated execution
  account.  Example:

\begin{verbatim}
SLOT1_USER = cndrusr1
SLOT2_USER = cndrusr2
STARTER_ALLOW_RUNAS_OWNER = False
DEDICATED_EXECUTE_ACCOUNT_REGEXP = cndrusr[0-9]+
\end{verbatim}

  You can tell if the starter is in fact treating the account as a
  dedicated account, because it will print a line such as the following
  in its log file:

\begin{verbatim}
Tracking process family by login "cndrusr1"
\end{verbatim}


\label{param:ExecuteLoginIsDedicated}
\item[\Macro{EXECUTE\_LOGIN\_IS\_DEDICATED}]
  This configuration setting is deprecated because it cannot handle the
  case where some jobs run as dedicated accounts and some do not.  Use
  \MacroNI{DEDICATED\_EXECUTE\_ACCOUNT\_REGEXP} instead.

  A boolean value that defaults to \Expr{False}.  When \Expr{True},
  HTCondor knows that all jobs are being run by dedicated execution
  accounts (whether they are running as the job owner or as nobody or as
  \MacroNI{SLOT<N>\_USER}).  Therefore, when the job exits, all processes
  running under the same account will be killed.

\label{param:FilesystemDomain}
\item[\Macro{FILESYSTEM\_DOMAIN}]
  The \MacroNI{FILESYSTEM\_DOMAIN}
  macro is an arbitrary string that is used to decide if
  two machines (a submitting machine and an execute machine) share a
  file system.
  Although the macro name contains the word ``DOMAIN'',
  the macro is not required to be a domain name. 
  It often is a domain name.

  % NO LONGER TRUE
  % Vanilla Unix jobs currently require a shared file system in order to
  % share any data files or see the output of the program.
  % HTCondor decides if there is a shared filesystem by comparing the values
  % of 
  % \MacroUNI{FILESYSTEM\_DOMAIN}
  % of both the submitting and execute machines.
  % If the values are the same,
  % HTCondor assume there is a shared file system.
  % HTCondor implements the check
  % by extending the Requirements for your job.
  % You can see these requirements by using the \oArg{-v} argument
  % to \Condor{submit}.

  Note that this implementation is not ideal: machines may share some
  file systems but not others. HTCondor currently has no way to express
  this automatically. You can express the need to use a
  particular file system by adding additional attributes to your machines
  and submit files, similar to the example given in 
  Frequently Asked Questions, 
  section~\ref{sec:FAQ} on
  how to run jobs only on machines that have 
  certain software packages.

  Note that if you do not set 
  \MacroUNI{FILESYSTEM\_DOMAIN}, HTCondor defaults
  to setting the macro's value to be the fully qualified host name
  of the local machine.
  Since each machine will have a different
  \MacroUNI{FILESYSTEM\_DOMAIN},
  they will not be considered to have shared file systems.

  
  % no longer used, and gone from the sample config file as of 5/30/03.
  %\item[\Macro{HAS\_AFS}] \label{param:HasAfs} Set this macro to \Expr{True} if
  %  all the machines you plan on adding in your pool can all access a
  %  common set of AFS fileservers.  Otherwise, set it to \Expr{False}.
  
\label{param:ReserveAfsCache}
\item[\Macro{RESERVE\_AFS\_CACHE}]
  If your machine is running AFS and the AFS cache lives on the same
  partition as the other HTCondor directories, and you want HTCondor to
  reserve the space that your AFS cache is configured to use, set this
  macro to \Expr{True}.  It defaults to \Expr{False}.
  
\label{param:UseNfs}
\item[\Macro{USE\_NFS}]
  This macro influences
  how HTCondor jobs running in the standard universe access their
  files.  By default, HTCondor will redirect the file I/O requests
  of standard universe jobs from the executing machine to the submitting
  machine.  So, as an HTCondor job migrates around
  the network, the file system always appears to be identical to the
  file system where the job was submitted.  However, consider the case
  where a user's data files are sitting on an NFS server. The machine
  running the user's program will send all I/O over the network to the
  submitting machine, which in turn sends all the I/O back
  over the network to the NFS file server. Thus,
  all of the program's I/O is being sent over the network twice.
  
  If this configuration variable is \Expr{True}, 
  then HTCondor will attempt to read/write files directly on the 
  executing machine without redirecting I/O back to the submitting
  machine, if both the submitting machine and the machine running the job
  are both accessing the same NFS servers (\emph{if} they are both in the
  same \MacroUNI{FILESYSTEM\_DOMAIN} and in the same \MacroUNI{UID\_DOMAIN},
  as described above).  The result is I/O performed by HTCondor standard
  universe jobs is only sent over the network once.  
  While sending all file operations over the network twice might sound
  really bad, unless you are operating over networks where bandwidth
  as at a very high premium, practical experience reveals that this
  scheme offers very little real performance gain.  There are also
  some (fairly rare) situations where this scheme can break down.
  
  Setting \MacroUNI{USE\_NFS} to \Expr{False} is always safe.  It may result
  in slightly more network traffic, but HTCondor jobs are most often heavy
  on CPU and light on I/O.  It also ensures that a remote
  standard universe HTCondor job will always use HTCondor's remote system
  calls mechanism to reroute I/O and therefore see the exact same
  file system that the user sees on the machine where she/he submitted
  the job.
  
  Some gritty details for folks who want to know: If the you set
  \MacroUNI{USE\_NFS} to \Expr{True}, and the \MacroUNI{FILESYSTEM\_DOMAIN} of
  both the submitting machine and the remote machine about to execute
  the job match, and the \MacroUNI{FILESYSTEM\_DOMAIN} claimed by the
  submit machine is indeed found to be a subset of what an inverse
  look up to a DNS (domain name server) reports as the fully qualified
  domain name for the submit machine's IP address (this security
  measure safeguards against the submit machine from lying),
  \emph{then} the job will access files using a local system call,
  without redirecting them to the submitting machine (with
  NFS).  Otherwise, the system call will get routed back to the
  submitting machine using HTCondor's remote system call mechanism.
  \Note When submitting a vanilla job, \Condor{submit} will, by default,
  append requirements to the Job ClassAd that specify the machine to run
  the job must be in the same \MacroUNI{FILESYSTEM\_DOMAIN} and the same
  \MacroUNI{UID\_DOMAIN}.

  This configuration variable similarly changes the semantics of Chirp file I/O
  when running in the vanilla, java or parallel universe.  
  If this variable is set in those universes, Chirp will not send I/O requests 
  over the network as requested, 
  but perform them directly to the locally mounted file system.  
  Other than Chirp file access, this variable is unused
  outside of the standard universe.

\label{param:IgnoreNFSLockErrors}
\item[\Macro{IGNORE\_NFS\_LOCK\_ERRORS}]
  When set to \Expr{True}, all errors related to file locking errors from
  NFS are ignored.
  Defaults to \Expr{False}, not ignoring errors.
  
\label{param:UseAfs}
\item[\Macro{USE\_AFS}]
  If your machines have AFS,
  this macro determines whether HTCondor will use remote system calls for
  standard universe jobs to send I/O requests to the submit machine,
  or if it should use local file access on the execute machine (which
  will then use AFS to get to the submitter's files).  Read the
  setting above on \MacroUNI{USE\_NFS} for a discussion of why you might
  want to use AFS access instead of remote system calls.
  
  One important difference between \MacroUNI{USE\_NFS} and
  \MacroUNI{USE\_AFS} is the AFS cache.  With \MacroUNI{USE\_AFS} set to
  \Expr{True}, the remote HTCondor job executing on some machine will start
  modifying the AFS cache, possibly evicting the machine owner's
  files from the cache to make room for its own.  Generally speaking,
  since we try to minimize the impact of having an HTCondor job run on a
  given machine, we do not recommend using this setting.

  While sending all file operations over the network twice might sound
  really bad, unless you are operating over networks where bandwidth
  as at a very high premium, practical experience reveals that this
  scheme offers very little real performance gain.  There are also
  some (fairly rare) situations where this scheme can break down.
  
  Setting \MacroUNI{USE\_AFS} to \Expr{False} is always safe.  It may result
  in slightly more network traffic, but HTCondor jobs are usually heavy
  on CPU and light on I/O.  \Expr{False} ensures that a remote
  standard universe HTCondor job will always see the exact same
  file system that the user on sees on the machine where he/she
  submitted the job.  Plus, it will ensure that the machine where the
  job executes does not have its AFS cache modified as a result of
  the HTCondor job being there.  
  
  However, things may be different at your site, which is why the
  setting is there.

\end{description}

%%%%%%%%%%%%%%%%%%%%%%%%%%%%%%%%%%%%%%%%%%%%%%%%%%%%%%%%%%%%%%%%%%%%%%%%%%%
\subsection{\label{sec:Checkpoint-Server-Config-File-Entries}Checkpoint Server Configuration File Macros} 
%%%%%%%%%%%%%%%%%%%%%%%%%%%%%%%%%%%%%%%%%%%%%%%%%%%%%%%%%%%%%%%%%%%%%%%%%%%

\index{configuration!checkpoint server configuration variables}
These macros control whether or not HTCondor uses a checkpoint server.
This section
describes the settings that the checkpoint server itself needs defined. 
See section~\ref{sec:Ckpt-Server} on Installing a Checkpoint Server
for details on installing and running a checkpoint server.

\begin{description}
  
\label{param:CkptServerHost}
\item[\Macro{CKPT\_SERVER\_HOST}]
  The host name of a checkpoint server.

\label{param:StarterChoosesCkptServer}
\item[\Macro{STARTER\_CHOOSES\_CKPT\_SERVER}]
  If this parameter is \Expr{True} or undefined on
  the submit machine, the checkpoint server specified by
  \MacroUNI{CKPT\_SERVER\_HOST} on the execute machine is used.  If it is
  \Expr{False} on the submit machine, the checkpoint server
  specified by \MacroUNI{CKPT\_SERVER\_HOST} on the submit machine is
  used.
  
\label{param:CkptServerDir}
\item[\Macro{CKPT\_SERVER\_DIR}]
  The full path of the
  directory the checkpoint server should use to store checkpoint files.
  Depending on the size of the pool and the size of the jobs submitted,
  this directory and its subdirectories might
  need to store many Mbytes of data.

\label{param:UseCkptServer}
\item[\Macro{USE\_CKPT\_SERVER}]
  A boolean which determines if a given submit machine is to use a
  checkpoint server if one is available.  If a
  checkpoint server is not available or the variable \MacroNI{USE\_CKPT\_SERVER}
  is set to \Expr{False},
  checkpoints will be written to the local \MacroUNI{SPOOL} directory on
  the submission machine.

\label{param:MaxDiscardedRunTime}
\item[\Macro{MAX\_DISCARDED\_RUN\_TIME}]
  If the \Condor{shadow} daemon is unable to read a
  checkpoint file from the checkpoint server, it keeps trying only if
  the job has accumulated more than this many seconds of CPU usage.
  Otherwise, the job is started from scratch.  
  Defaults to 3600 (1 hour). 
  This variable is only used if \MacroUNI{USE\_CKPT\_SERVER} is \Expr{True}.

\label{param:CkptServerCheckParentInterval}
\item[\Macro{CKPT\_SERVER\_CHECK\_PARENT\_INTERVAL}]
  This is the number of seconds between checks to see whether the parent
  of the checkpoint server (usually the \Condor{master}) has died.  If the
  parent has died, the checkpoint server shuts itself down.
  The default is 120 seconds.
  A setting of 0 disables this check.

\label{param:CkptServerInterval}
\item[\Macro{CKPT\_SERVER\_INTERVAL}]
  The maximum number of seconds the checkpoint server
  waits for activity on network sockets before performing other
  tasks. The default value is 300 seconds.

\label{param:CkptServerClassadFile}
\item[\Macro{CKPT\_SERVER\_CLASSAD\_FILE}]
  A string that represents a file in the file system to which
  ClassAds will be written. The ClassAds denote information about stored
  checkpoint files, such as owner, shadow IP address, name of the
  file, and size of the file. This information is also independently
  recorded in the \File{TransferLog}. The default setting is undefined,
  which means a checkpoint server ClassAd file will not be kept.

\label{param:CkptServerCleanInterval}
\item[\Macro{CKPT\_SERVER\_CLEAN\_INTERVAL}]
  The number of seconds that must pass until the ClassAd log file
  as described by the \MacroNI{CKPT\_SERVER\_CLASSAD\_FILE} variable gets
  truncated. The default is 86400 seconds, which is one day.

\label{param:CkptServerRemoveStaleCkptInterval}
\item[\Macro{CKPT\_SERVER\_REMOVE\_STALE\_CKPT\_INTERVAL}]
  The number of seconds between attempts to discover and remove
  stale checkpoint files. It defaults to 86400 seconds, which is one day.

\label{param:CkptServerSocketBufsize}
\item[\Macro{CKPT\_SERVER\_SOCKET\_BUFSIZE}]
  The number of bytes representing the size of the TCP
  send/recv buffer on the socket file descriptor related to moving
  the checkpoint file to and from the checkpoint server. 
  The default value is 0, which allows the operating system to decide the size.

\label{param:CkptServerMaxProcesses}
\item[\Macro{CKPT\_SERVER\_MAX\_PROCESSES}]
  The maximum number of child processes that could be working on
  behalf of the checkpoint server. This includes store processes and
  restore processes. The default value is 50.

\label{param:CkptServerMaxStoreProcesses}
\item[\Macro{CKPT\_SERVER\_MAX\_STORE\_PROCESSES}]
  The maximum number of child process strictly devoted
  to the storage of checkpoints. 
  The default is the value of \MacroNI{CKPT\_SERVER\_MAX\_PROCESSES}.

\label{param:CkptServerMaxRestoreProcesses}
\item[\Macro{CKPT\_SERVER\_MAX\_RESTORE\_PROCESSES}]
  The maximum number of child process strictly devoted
  to the restoring of checkpoints. 
  The default is the value of \MacroNI{CKPT\_SERVER\_MAX\_PROCESSES}.

\label{param:CkptServerStaleCkptAgeCutoff}
\item[\Macro{CKPT\_SERVER\_STALE\_CKPT\_AGE\_CUTOFF}]
  The number of seconds after which if a checkpoint file has not
  been accessed, it is considered stale.
  The default value is 5184000 seconds, which is sixty days.

\end{description}


%%%%%%%%%%%%%%%%%%%%%%%%%%%%%%%%%%%%%%%%%%%%%%%%%%%%%%%%%%%%%%%%%%%%%%%%%%%
\subsection{\label{sec:Master-Config-File-Entries}\condor{master} Configuration File Macros} 
%%%%%%%%%%%%%%%%%%%%%%%%%%%%%%%%%%%%%%%%%%%%%%%%%%%%%%%%%%%%%%%%%%%%%%%%%%%

\index{configuration!condor\_master configuration variables}
These macros control the \Condor{master}.
\begin{description}
  
\label{param:DaemonList}
\item[\Macro{DAEMON\_LIST}]
  This macro
  determines what daemons the \Condor{master} will start and keep its
  watchful eyes on.  The list is a comma or space separated list of
  subsystem names (listed in
  section~\ref{sec:HTCondor-Subsystem-Names}).  For example,
\begin{verbatim}
  DAEMON_LIST = MASTER, STARTD, SCHEDD
\end{verbatim}

  \Note This configuration variable cannot be changed 
  by using \Condor{reconfig} or 
  by sending a SIGHUP.
  To change this configuration variable, restart the
  \Condor{master} daemon
  by using \Condor{restart}.
  Only then will the change take effect.

  \Note On your central manager, your \MacroUNI{DAEMON\_LIST}
  will be different from your regular pool, since it will include
  entries for the \Condor{collector} and \Condor{negotiator}.  
  
\label{param:DCDaemonList}
\item[\Macro{DC\_DAEMON\_LIST}]
  A list delimited by commas and/or spaces that
  lists the daemons in \MacroNI{DAEMON\_LIST} which use the HTCondor
  DaemonCore library.  The \Condor{master} must differentiate between
  daemons that use DaemonCore and those that do not,
  so it uses the appropriate inter-process communication mechanisms.
  This list currently includes all HTCondor daemons except the checkpoint server
  by default.

  As of HTCondor version 7.2.1, a daemon may be appended to the default
  \MacroNI{DC\_DAEMON\_LIST} value by placing the plus character
  (\verb@+@) before the first entry in the \MacroNI{DC\_DAEMON\_LIST} 
  definition.
  For example:
\begin{verbatim}
  DC_DAEMON_LIST = +NEW_DAEMON
\end{verbatim}

\label{param:SUBSYS}
\item[\MacroB{<SUBSYS>}]
  \index{SUBSYS macro@\texttt{<SUBSYS>} macro}
  Once you have defined which
  subsystems you want the \Condor{master} to start, you must provide
  it with the full path to each of these binaries.  For example:
  \begin{verbatim}
    MASTER          = $(SBIN)/condor_master
    STARTD          = $(SBIN)/condor_startd
    SCHEDD          = $(SBIN)/condor_schedd
  \end{verbatim}
  These are most often defined relative to the \MacroUNI{SBIN} macro.

  The macro is named by substituting \MacroNI{<SUBSYS>}
  with the appropriate subsystem string as defined in
  section~\ref{sec:HTCondor-Subsystem-Names}.

\label{param:DaemonNameEnvironment}
\item[\Macro{<DaemonName>\_ENVIRONMENT}]
  \MacroNI{<DaemonName>} is the name of a daemon listed in
  \MacroNI{DAEMON\_LIST}.
  Defines changes to the environment that the daemon is invoked with.
  It should use the same syntax
  for specifying the environment as the environment specification in
  a submit description file.
  For example, to redefine the
  \Env{TMP} and \Env{CONDOR\_CONFIG} environment variables seen by the
  \Condor{schedd}, place the following in the configuration:
  \footnotesize
\begin{verbatim}
  SCHEDD_ENVIRONMENT = "TMP=/new/value CONDOR_CONFIG=/special/config"
\end{verbatim}
  \normalsize
  When the \Condor{schedd} daemon is started by the \Condor{master}, it would
  see the specified values of \Env{TMP} and \Env{CONDOR\_CONFIG}.

\label{param:SubsysArgs}
\item[\MacroB{<SUBSYS>\_ARGS}]
  \index{SUBSYS\_ARGS macro@\texttt{<SUBSYS>\_ARGS} macro}
  This macro allows
  the specification of additional command line arguments for any
  process spawned by the \Condor{master}.  List the desired arguments
  using the same syntax as the arguments specification in a
  \Condor{submit} submit file (see
  page~\pageref{man-condor-submit-arguments}), with one exception: do
  not escape double-quotes when using the old-style syntax (this is
  for backward compatibility).  Set the arguments for a specific
  daemon with this macro, and the macro will affect only that
  daemon. Define one of these for each daemon the \Condor{master} is
  controlling.  For example, set \MacroUNI{STARTD\_ARGS} to specify
  any extra command line arguments to the \Condor{startd}.

  The macro is named by substituting \MacroNI{<SUBSYS>}
  with the appropriate subsystem string as defined in
  section~\ref{sec:HTCondor-Subsystem-Names}.

\label{param:SubsysUserid}
\item[\MacroB{<SUBSYS>\_USERID}]
  \index{SUBSYS\_USERID macro@\texttt{<SUBSYS>\_USERID} macro}
  The account name that should be used to run the \MacroNI{SUBSYS} process
  spawned by the \Condor{master}.  When not defined, the process is
  spawned as the same user that is running \Condor{master}.  When
  defined, the real user id of the spawned process will be set to the
  specified account, so if this account is not \Login{root}, the process will
  not have \Login{root} privileges.  The \Condor{master} must be running as
  root in order to start processes as other users.  Example configuration:

\begin{verbatim}
COLLECTOR_USERID = condor
NEGOTIATOR_USERID = condor
\end{verbatim}

  The above example runs the \Condor{collector} and \Condor{negotiator}
  as the \Login{condor} user with no \Login{root} privileges.
  If we specified some account other than the \Login{condor} user,
  as set by the (\MacroNI{CONDOR\_IDS}) configuration variable, then we
  would need to configure the log files for these daemons to be in a
  directory that they can write to.  When using GSI security or any
  other security method in which the daemon credential is owned by \Login{root},
  it is also necessary to make a copy of the credential, make it be
  owned by the account the daemons are using, and configure the daemons
  to use that copy.

\label{param:Preen}
\item[\Macro{PREEN}]
  In addition to the daemons
  defined in \MacroUNI{DAEMON\_LIST}, the \Condor{master} also starts up
  a special process, \Condor{preen} to clean out junk files that have
  been left laying around by HTCondor.  This macro determines where the
  \Condor{master} finds the \Condor{preen} binary.
  If this macro is set to nothing, \Condor{preen} will not run.

\label{param:PreenArgs}
\item[\Macro{PREEN\_ARGS}]
  Controls how \Condor{preen} behaves by allowing the specification
  of command-line arguments.
  This macro works as \MacroUNI{<SUBSYS>\_ARGS} does.
  The difference is that you must specify this macro for
  \Condor{preen} if you want it to do anything.
  \Condor{preen} takes action only
  because of command line arguments.
  \Opt{-m} means you want e-mail about files \Condor{preen} finds that it
  thinks it should remove.
  \Opt{-r} means you want \Condor{preen} to actually remove these files.

\item[\Macro{PREEN\_INTERVAL}]
\label{param:PreenInterval}
  This macro determines how often \Condor{preen} should be started.
  It is defined in terms of seconds and defaults to 86400 (once a day).

\label{param:PublishObituaries}
\item[\Macro{PUBLISH\_OBITUARIES}]
  When a daemon crashes, the \Condor{master} can send e-mail to the
  address specified by \MacroUNI{CONDOR\_ADMIN} with an obituary letting
  the administrator know that the daemon died, the cause of
  death (which signal or exit status it exited with), and
  (optionally) the last few entries from that daemon's log file.  If
  you want obituaries, set this macro to \Expr{True}.

\label{param:ObituaryLogLength}
\item[\Macro{OBITUARY\_LOG\_LENGTH}]
  This macro controls how many lines
  of the log file are part of obituaries.  This macro has a default
  value of 20 lines.

\label{param:StartMaster}
\item[\Macro{START\_MASTER}]
  If this setting is defined and set to \Expr{False}
  when the \Condor{master} starts up, the first
  thing it will do is exit.  This appears strange, but perhaps you
  do not want HTCondor to run on certain machines in your pool, yet
  the boot scripts for your entire pool are handled by a centralized
  This is
  an entry you would most likely find in a local configuration file,
  not a global configuration file.

\label{param:StartDaemons}
\item[\Macro{START\_DAEMONS}]
  This macro
  is similar to the \MacroUNI{START\_MASTER} macro described above.
  However, the \Condor{master} does not exit; it does not start any
  of the daemons listed in the \MacroUNI{DAEMON\_LIST}.
  The daemons may be started at a later time with a \Condor{on}
  command.

\label{param:MasterUpdateInterval}
\item[\Macro{MASTER\_UPDATE\_INTERVAL}]
  This macro determines how often
  the \Condor{master} sends a ClassAd update to the
  \Condor{collector}.  It is defined in seconds and defaults to 300
  (every 5 minutes).
  
\label{param:MasterCheckNewExecInterval}
\item[\Macro{MASTER\_CHECK\_NEW\_EXEC\_INTERVAL}]
  This macro controls how often the \Condor{master} checks the timestamps
  of the running daemons.  If any daemons have been modified, the
  master restarts them.  It is defined in seconds and defaults to 300
  (every 5 minutes).

\label{param:MasterNewBinaryRestart}
\item[\Macro{MASTER\_NEW\_BINARY\_RESTART}]
  Defines a mode of operation for the restart of the \Condor{master},
  when it notices that the \Condor{master} binary has changed.
  Valid values are \Expr{GRACEFUL}, \Expr{PEACEFUL}, and \Expr{NEVER},
  with a default value of \Expr{GRACEFUL}.
  On a \Expr{GRACEFUL} restart of the master, child processes are told to exit,
  but if they do not before a timer expires, then they are killed.
  On a \Expr{PEACEFUL} restart, child processes are told to exit,
  after which the \Condor{master} waits until they do so.

\label{param:MasterNewBinaryDelay}
\item[\Macro{MASTER\_NEW\_BINARY\_DELAY}]
  Once the \Condor{master} has
  discovered a new binary, this macro controls how long it waits
  before attempting to execute the new binary.  This delay exists
  because the \Condor{master} might notice a new binary while it
  is in the process of being copied,
  in which case trying to execute it yields
  unpredictable results.  The entry is defined in seconds and
  defaults to 120 (2 minutes).

\label{param:ShutdownFastTimeout}
\item[\Macro{SHUTDOWN\_FAST\_TIMEOUT}]
  This macro determines the maximum
  amount of time daemons are given to perform their
  fast shutdown procedure before the \Condor{master} kills them
  outright.  It is defined in seconds and defaults to 300 (5 minutes).

\label{param:MasterShutdownProgram}
\item[\Macro{MASTER\_SHUTDOWN\_<Name>}]
  A full path and file name of
  a program that the \Condor{master} is to execute
  via the Unix \Procedure{execl} call, 
  or the similar Win32 \Procedure{\_execl} call,
  instead of the normal call to \Procedure{exit}.
  Multiple programs to execute may be defined with multiple entries,
  each with a unique \MacroNI{Name}.
  These macros have no effect on a
  \Condor{master} unless \Condor{set\_shutdown} is run.  
  The \MacroNI{Name} specified as an argument to the \Condor{set\_shutdown}
  program must match the \MacroNI{Name} portion of one of these
  \MacroNI{MASTER\_SHUTDOWN\_<Name>} macros; if not, the
  \Condor{master} will log an error and ignore the command.  If a
  match is found, the \Condor{master} will attempt to verify the
  program, and it will store the path and program name.  When the
  \Condor{master} shuts down (that is, just before it exits),
  the program is then executed as described above.
  The manual page for \Condor{set\_shutdown} on
  page~\pageref{man-condor-set-shutdown} contains details on the use
  of this program.

  \Note This program will be run with root privileges under Unix 
  or administrator privileges under Windows.  
  The administrator must ensure that this cannot
  be used in such a way as to violate system integrity.

\label{param:MasterBackoffConstant}
\item[\Macro{MASTER\_BACKOFF\_CONSTANT} and
  \Macro{MASTER\_<name>\_BACKOFF\_CONSTANT}]
  When a daemon crashes, \Condor{master} uses an exponential back off
  delay before restarting it; see the discussion at the end of this
  section for a detailed discussion on how these parameters work together.
  These settings define the constant value of the expression used to
  determine how long to wait before starting the daemon again (and,
  effectively becomes the initial backoff time).  It is an integer in
  units of seconds, and defaults to 9 seconds.

  \MacroUNI{MASTER\_<name>\_BACKOFF\_CONSTANT} is the daemon-specific
  form of \MacroNI{MASTER\_BACKOFF\_CONSTANT}; if this daemon-specific
  macro is not defined for a specific daemon, the non-daemon-specific
  value will used.

\label{param:MasterBackoffFactor}
\item[\Macro{MASTER\_BACKOFF\_FACTOR} and
      \Macro{MASTER\_<name>\_BACKOFF\_FACTOR}]
  When a daemon crashes, \Condor{master} uses an exponential back off
  delay before restarting it; see the discussion at the end of this
  section for a detailed discussion on how these parameters work together.
  This setting is the base of the
  exponent used to determine how long to wait before starting the
  daemon again.  It defaults to 2 seconds.

  \MacroUNI{MASTER\_<name>\_BACKOFF\_FACTOR} is the daemon-specific
  form of \MacroNI{MASTER\_BACKOFF\_FACTOR}; if this daemon-specific
  macro is not defined for a specific daemon, the non-daemon-specific
  value will used.

\label{param:MasterBackoffCeiling}
\item[\Macro{MASTER\_BACKOFF\_CEILING} and
      \Macro{MASTER\_<name>\_BACKOFF\_CEILING}]
  When a daemon crashes, \Condor{master} uses an exponential back off
  delay before restarting it; see the discussion at the end of this
  section for a detailed discussion on how these parameters work together.
  This entry determines the maximum amount of time you want the master
  to wait between attempts to start a given daemon.
  (With 2.0 as the \MacroUNI{MASTER\_BACKOFF\_FACTOR},
  1 hour is obtained in 12 restarts).  It is defined in terms of
  seconds and defaults to 3600 (1 hour).

  \MacroUNI{MASTER\_<name>\_BACKOFF\_CEILING} is the daemon-specific
  form of \MacroNI{MASTER\_BACKOFF\_CEILING}; if this daemon-specific
  macro is not defined for a specific daemon, the non-daemon-specific
  value will used.

\label{param:MasterRecoverFactor}
\item[\Macro{MASTER\_RECOVER\_FACTOR} and
      \Macro{MASTER\_<name>\_RECOVER\_FACTOR}]
  A macro to set how long a daemon 
  needs to run without crashing before it is considered \emph{recovered}.
  Once a
  daemon has recovered, the number of restarts is reset, so the
  exponential back off returns to its initial state.  
  The macro is defined in
  terms of seconds and defaults to 300 (5 minutes).

  \MacroUNI{MASTER\_<name>\_RECOVER\_FACTOR} is the daemon-specific
  form of \MacroNI{MASTER\_RECOVER\_FACTOR}; if this daemon-specific
  macro is not defined for a specific daemon, the non-daemon-specific
  value will used.

\end{description}

When a daemon crashes, \Condor{master} will restart the daemon after a
delay (a back off).
The length of this delay is based on how many times it has been
restarted, and gets larger after each crashes. 
The equation for calculating this backoff time is
given by: $$t = c + k^n$$ where $t$ is the calculated time, $c$ is
the constant defined by \MacroUNI{MASTER\_BACKOFF\_CONSTANT}, $k$ is
the ``factor'' defined by \MacroUNI{MASTER\_BACKOFF\_FACTOR}, and $n$
is the number of restarts already attempted (0 for the first restart,
1 for the next, etc.).

With default values, after the first crash, the delay would be $t = 9
+ 2.0^0$, giving 10 seconds (remember, $n = 0$).  If the daemon keeps
crashing, the delay increases.

For example, take the \MacroUNI{MASTER\_BACKOFF\_FACTOR} (which defaults
to 2.0) to the power the number of times the daemon has restarted, and add
\MacroUNI{MASTER\_BACKOFF\_CONSTANT} (which defaults to 9).
Thus:

 $1^{st}$ crash:  $n = 0$, so: $t = 9 + 2^0 = 9 + 1 = 10\ seconds$

 $2^{nd}$ crash:  $n = 1$, so: $t = 9 + 2^1 = 9 + 2 = 11\ seconds$

 $3^{rd}$ crash:  $n = 2$, so: $t = 9 + 2^2 = 9 + 4 = 13\ seconds$

    ...

 $6^{th}$ crash:  $n = 5$, so: $t = 9 + 2^5 = 9 + 32 = 41\ seconds$

    ...

 $9^{th}$ crash:  $n = 8$, so: $t = 9 + 2^8 = 9 + 256 = 265\ seconds$

And, after the 13 crashes, it would be:

 $13^{th}$ crash:  $n = 12$, so: $t = 9 + 2^{12} = 9 + 4096 = 4105\ seconds$

This is bigger than the \MacroUNI{MASTER\_BACKOFF\_CEILING}, which
defaults to 3600, so the daemon would really be restarted after only
3600 seconds, not 4105.
The \Condor{master} tries again every hour (since the numbers would
get larger and would always be capped by the ceiling).
Eventually, imagine that daemon finally started and did not crash.
This might happen if, for example, an administrator reinstalled
an accidentally deleted binary after receiving e-mail about
the daemon crashing.
If it stayed alive for
\MacroUNI{MASTER\_RECOVER\_FACTOR} seconds (defaults to 5 minutes),
the count of how many restarts this daemon has performed is reset to
0.

The moral of the example is that 
the defaults work quite well, and you probably 
will not want to change them for any reason.
\begin{description}

\label{param:MasterName}
\item[\Macro{MASTER\_NAME}]
  Defines a unique name given for a \Condor{master} daemon on a machine.
  For a \Condor{master} running as \Login{root},
  it defaults to the fully qualified host name.
  When \emph{not} running as \Login{root},
  it defaults to the user that instantiates the
  \Condor{master}, concatenated with an at symbol (\verb$@$),
  concatenated with the fully qualified host name.
  If more than one \Condor{master} is running on the same host, 
  then the \MacroNI{MASTER\_NAME} for each
  \Condor{master} must be defined to uniquely identify the separate
  daemons. 

  A defined \MacroNI{MASTER\_NAME} is presumed to be of the form
  \verb$identifying-string@full.host.name$.
  If the string does not include an \verb$@$ sign,
  HTCondor appends one, followed by the fully qualified host name
  of the local machine.
  The \verb$identifying-string$ portion may contain any
  alphanumeric ASCII characters or punctuation marks, except the \verb$@$ sign.
  We recommend that the string does not contain the \verb$:$ (colon)
  character, since that might cause problems with certain tools.
  Previous to HTCondor 7.1.1, when the string included
  an \verb$@$ sign, HTCondor replaced whatever followed the \verb$@$
  sign with the fully qualified host name of the local machine.
  HTCondor does not modify any portion of the string, if it
  contains an \verb$@$ sign.
  This is useful for remote job submissions under the high availability
  of the job queue.

  If the \MacroNI{MASTER\_NAME} setting is used, and the
  \Condor{master} is configured to spawn a \Condor{schedd},
  the name
  defined with \MacroNI{MASTER\_NAME} takes precedence over the
  \Macro{SCHEDD\_NAME} setting (see section~\ref{param:ScheddName} on
  page~\pageref{param:ScheddName}). 
  Since HTCondor makes the assumption that there is only one
  instance of the \Condor{startd} running on a machine,
  the \MacroNI{MASTER\_NAME} is not automatically propagated to the
  \Condor{startd}.
  However, in situations where multiple \Condor{startd} daemons are
  running on the same host,
  the \MacroNI{STARTD\_NAME} should be set to uniquely identify 
  the \Condor{startd} daemons.

  If an HTCondor daemon (master, schedd or startd) has been given a
  unique name, all HTCondor tools that need to contact that daemon can
  be told what name to use via the \Opt{-name} command-line option.


\label{param:MasterExprs}
\item[\Macro{MASTER\_ATTRS}]
  This macro is described in section~\ref{param:SubsysExprs} as
  \MacroNI{<SUBSYS>\_ATTRS}.

\label{param:MasterDebug}
\item[\Macro{MASTER\_DEBUG}]
  This macro is described in section~\ref{param:SubsysDebug} as
  \MacroNI{<SUBSYS>\_DEBUG}.

\label{param:MasterAddressFile}
\item[\Macro{MASTER\_ADDRESS\_FILE}]
  This macro is described in
  section~\ref{param:SubsysAddressFile} as
  \MacroNI{<SUBSYS>\_ADDRESS\_FILE}. 

\label{param:SecondaryCollectorList}
\item[\Macro{SECONDARY\_COLLECTOR\_LIST}]
  This macro has been removed as of HTCondor version 6.9.3.
  Use the \Macro{COLLECTOR\_HOST} configuration variable, which may define a
  list of \Condor{collector} daemons.

\label{param:AllowAdminCommands}
\item[\Macro{ALLOW\_ADMIN\_COMMANDS}]
  If set to NO for a given host, this
  macro disables administrative commands, such as 
  \Condor{restart}, \Condor{on}, and \Condor{off}, to that host.

\label{param:MasterInstanceLock}
\item[\Macro{MASTER\_INSTANCE\_LOCK}]
  Defines the name of a file for the \Condor{master} daemon
  to lock in order to prevent multiple \Condor{master}s
  from starting.
  This is useful when using shared file systems like NFS which do
  not technically support locking in the case where the lock files
  reside on a local disk.
  If this macro is not defined, the default file name will be
  \File{\$(LOCK)/InstanceLock}.
  \File{\$(LOCK)} can instead be defined to
  specify the location of all lock files, not just the 
  \Condor{master}'s \File{InstanceLock}.
  If \File{\$(LOCK)} is undefined, then the master log itself is locked.

\label{param:AddWindowsFirewallException}
\item[\Macro{ADD\_WINDOWS\_FIREWALL\_EXCEPTION}]
  When set to \Expr{False}, the
  \Condor{master} will not automatically add HTCondor to the Windows
  Firewall list of trusted applications. Such trusted applications can
  accept incoming connections without interference from the firewall. This
  only affects machines running Windows XP SP2 or higher. The default
  is \Expr{True}.

\label{param:WindowsFirewallFailureRetry} 
\item[\Macro{WINDOWS\_FIREWALL\_FAILURE\_RETRY}]
  An integer value (default value is 60) that represents
  the number of times the \Condor{master} will retry to add
  firewall exceptions.
  When a Windows machine boots
  up, HTCondor starts up by default as well. Under certain conditions, the
  \Condor{master} may have difficulty adding exceptions to the Windows
  Firewall because of a delay in other services starting up.
  Examples of services that may possibly be slow are the 
  SharedAccess service, the Netman service, or the Workstation service.
  This configuration variable allows administrators to set the number of
  times (once every 10 seconds) that the \Condor{master} will retry
  to add firewall exceptions. A value of 0 means that HTCondor will
  retry indefinitely.

\label{param:UseProcessGroups} 
\item[\Macro{USE\_PROCESS\_GROUPS}]
  A boolean value that defaults to \Expr{True}.  When \Expr{False},
  HTCondor daemons on Unix machines will \emph{not} create new sessions
  or process groups. HTCondor uses processes groups to help it track the
  descendants of processes it creates. This can cause problems when
  HTCondor is run under another job execution system.

\end{description}

%%%%%%%%%%%%%%%%%%%%%%%%%%%%%%%%%%%%%%%%%%%%%%%%%%%%%%%%%%%%%%%%%%%%%%%%%%%
\subsection{\label{sec:Startd-Config-File-Entries}\condor{startd}
Configuration File Macros}
%%%%%%%%%%%%%%%%%%%%%%%%%%%%%%%%%%%%%%%%%%%%%%%%%%%%%%%%%%%%%%%%%%%%%%%%%%%

\index{configuration!condor\_startd configuration variables}
\Note If you are running HTCondor on a multi-CPU machine, be sure
to also read section~\ref{sec:Configuring-SMP} on
page~\pageref{sec:Configuring-SMP} which describes how to set up and
configure HTCondor on multi-core machines.

These settings control general operation of the \Condor{startd}.
Examples using these configuration macros,
as well as further explanation is found in
section~\ref{sec:Configuring-Policy} on
Configuring The Startd Policy.

\begin{description}

\label{param:Start}
\item[\Macro{START}]
  A boolean expression
  that, when \Expr{True}, indicates that the machine is willing
  to start running an HTCondor job.
  \MacroNI{START} is considered when the \Condor{negotiator} daemon
  is considering evicting the job to replace it with one that will
  generate a better rank for the \Condor{startd} daemon,
  or a user with a higher priority.

\label{param:Suspend}
\item[\Macro{SUSPEND}]
  A boolean expression that, when \Expr{True},
  causes HTCondor to suspend running an HTCondor job.
  The machine may still be claimed, but the job makes no further
  progress, and HTCondor does not generate a load on the machine.

\label{param:Preempt}
\item[\Macro{PREEMPT}]
  A boolean expression that, when \Expr{True},
  causes HTCondor to stop a currently running job once
  \Macro{MAXJOBRETIREMENTTIME} has expired.  This expression is not
  evaluated if \MacroNI{WANT\_SUSPEND} is \Expr{True}.

\label{param:WantHold}
\item[\Macro{WANT\_HOLD}]
  A boolean expression that defaults to \Expr{False}.
  When \Expr{True} and the value of \MacroNI{PREEMPT} becomes \Expr{True}
  and \MacroNI{WANT\_SUSPEND} is \Expr{False} and \MacroNI{MAXJOBRETIREMENTTIME}
  has expired,
  the job is put on hold for the reason
  (optionally) specified by the variables \MacroNI{WANT\_HOLD\_REASON} and
  \MacroNI{WANT\_HOLD\_SUBCODE}.
  As usual, the job owner may specify
  \SubmitCmd{periodic\_release} and/or \SubmitCmd{periodic\_remove}
  expressions to react to specific hold states automatically.
  The attribute \AdAttr{HoldReasonCode} in the job ClassAd is set to 
  the value 21 when
  \MacroNI{WANT\_HOLD} is responsible for putting the job on hold.

  Here is an example policy that puts jobs on hold
  that use too much virtual memory:

\footnotesize
\begin{verbatim}
VIRTUAL_MEMORY_AVAILABLE_MB = (VirtualMemory*0.9)
MEMORY_EXCEEDED = ImageSize/1024 > $(VIRTUAL_MEMORY_AVAILABLE_MB)
PREEMPT = ($(PREEMPT)) || ($(MEMORY_EXCEEDED))
WANT_SUSPEND = ($(WANT_SUSPEND)) && ($(MEMORY_EXCEEDED)) =!= TRUE
WANT_HOLD = ($(MEMORY_EXCEEDED))
WANT_HOLD_REASON = \
   ifThenElse( $(MEMORY_EXCEEDED), \
               "Your job used too much virtual memory.", \
               undefined )
\end{verbatim}
\normalsize

\label{param:WantHoldReason}
\item[\Macro{WANT\_HOLD\_REASON}]
  An expression that defines a string utilized to set the job ClassAd
  attribute \AdAttr{HoldReason} when a job is put on hold due to
  \MacroNI{WANT\_HOLD}.
  If not defined or if the expression evaluates to \Expr{Undefined},
  a default hold reason is provided.

\label{param:WantHoldSubcode}
\item[\Macro{WANT\_HOLD\_SUBCODE}]
  An expression that defines an integer value utilized to set the job ClassAd
  attribute \AdAttr{HoldReasonSubCode}  when a job is put on hold due to
  \MacroNI{WANT\_HOLD}.
  If not defined or if the expression evaluates to \Expr{Undefined},
  the value is set to 0.
  Note that \AdAttr{HoldReasonCode} is always set to 21.

\label{param:Continue}
\item[\Macro{CONTINUE}]
  A boolean expression that, when \Expr{True},
  causes HTCondor to continue the execution of a suspended job.

\label{param:Kill}
\item[\Macro{KILL}]
  A boolean expression that, when \Expr{True},
  causes HTCondor to immediately stop the
  execution of a vacating job, without delay.
  The job is hard-killed, so any attempt by the job to checkpoint or
  clean up will be aborted.  This expression should normally be
  \Expr{False}.  When desired, it may be used to abort the graceful
  shutdown of a job earlier than the limit imposed by
  \Macro{MachineMaxVacateTime}.

\label{param:PeriodicCheckpoint}
\item[\Macro{PERIODIC\_CHECKPOINT}]
  A boolean expression that, when \Expr{True}, causes HTCondor to
  initiate a checkpoint of the currently running job. This setting
  applies to all standard universe jobs and to vm universe jobs
  that have set \SubmitCmd{vm\_checkpoint} to \Expr{True}
  in the submit description file.

\label{param:Rank}
\item[\Macro{RANK}]
  A floating point value that HTCondor uses to compare potential jobs.
  A larger value for a specific job ranks that job above
  others with lower values for \MacroNI{RANK}.

\label{param:IsValidCheckpointPlatform}
\item[\Macro{IS\_VALID\_CHECKPOINT\_PLATFORM}]
  A boolean expression that is logically ANDed with the
  with the \MacroNI{START} expression to limit which machines a
  standard universe job may continue execution on once they have
  produced a checkpoint.
  The default expression is

   \footnotesize
   \begin{verbatim}
   IS_VALID_CHECKPOINT_PLATFORM =
   (
     ( (TARGET.JobUniverse == 1) == FALSE) ||
   
     (
       (MY.CheckpointPlatform =!= UNDEFINED) &&
       (
         (TARGET.LastCheckpointPlatform =?= MY.CheckpointPlatform) ||
         (TARGET.NumCkpts == 0)
       )
     )
   )
   \end{verbatim}
   \normalsize

\label{param:CheckpointPlatform}
\item[\Macro{CHECKPOINT\_PLATFORM}]
  A string used to
  override the automatically-generated machine ClassAd attribute 
  \AdAttr{CheckpointPlatform} (see section
  \ref{CheckpointPlatform-machine-attribute}),
  which is used to identify the platform upon which a job previously generated
  a checkpoint under the standard universe.
  This restricts the machine matches that may be considered for a job
  and where the job may resume.
  Overriding the value may be necessary for architectures that are
  the same in name, but actually have differences in instruction sets,
  such as the AVX extensions to the Intel processor.

\label{param:WantSuspend}
\item[\Macro{WANT\_SUSPEND}]
  A boolean expression that, when \Expr{True},
  tells HTCondor to evaluate the \MacroNI{SUSPEND} expression to decide
  whether to suspend a running job.  When \Expr{True}, the \MacroNI{PREEMPT}
  expression is not evaluated.
  When not explicitly set, the \Condor{startd} exits with an error.
  When explicitly set, but the evaluated value is anything other than
  \Expr{True}, the value is utilized as if it were \Expr{False}.

\label{param:WantVacate}
\item[\Macro{WANT\_VACATE}]
  A boolean expression that, when \Expr{True}, defines that a preempted
  HTCondor job is to be vacated, instead of killed.  This means the
  job will be soft-killed and given time to checkpoint or clean up.
  The amount of time given depends on \Macro{MachineMaxVacateTime}
  and \Macro{KILL}.

\label{param:EnableVersionedOpsys}
\item[\Macro{ENABLE\_VERSIONED\_OPSYS}]
  A boolean expression that determines whether pre-7.7.2 strings used for
  the machine ClassAd attribute \Attr{OpSys} are used or not.
  Defaults to \Expr{False} on Windows platforms, meaning that the newer
  behavior of setting \Expr{OpSys = "WINDOWS"} and \Expr{OpSysVer = 601}
  (for example), while \Expr{OpSysAndVer = "WINNT61"}.
  On platforms \emph{other} than Windows, the default value is \Expr{True},
  meaning that the values for \Attr{OpSys} and \Attr{OpSysAndVer} are the same,
  implementing the pre-7.7.2 behavior.

\label{param:IsOwner}
\item[\Macro{IS\_OWNER}]
  A boolean expression that defaults to being defined as
\begin{verbatim}
        IS_OWNER = (START =?= FALSE)
\end{verbatim}
  Used to describe the state of the machine with respect to its use
  by its owner.
  Job ClassAd attributes are not used in defining \MacroNI{IS\_OWNER},
  as they would be \Expr{Undefined}.

\label{param:StartdHistory}
\item[\Macro{STARTD\_HISTORY}]
  A file name where the \Condor{startd} daemon will
  maintain a job history file in an analogous way to that of the 
  history file defined by the configuration variable \MacroNI{HISTORY}.
  It will be rotated in the same way,
  and the same parameters that apply to the \MacroNI{HISTORY} file
  rotation apply to the \Condor{startd} daemon history as well.

\end{description}




\begin{description}

\label{param:Starter}
\item[\Macro{STARTER}]
  This macro holds the
  full path to the \Condor{starter} binary that the \Condor{startd} should 
  spawn.
  It is normally defined relative to \MacroUNI{SBIN}.
  
\label{param:KillingTimeout}
\item[\Macro{KILLING\_TIMEOUT}]
  The amount of time in seconds that the \Condor{startd} should wait after 
  sending a fast shutdown request to \Condor{starter} before forcibly killing the job and \Condor{starter}.  
  The default value is 30 seconds. 

\label{param:PollingInterval}
\item[\Macro{POLLING\_INTERVAL}]
  When a
  \Condor{startd} enters the claimed state, this macro determines how often
  the state of the machine is polled to check the need to suspend, resume,
  vacate or kill the job.  It is defined in terms of seconds and defaults to
  5.
  
\label{param:UpdateInterval}
\item[\Macro{UPDATE\_INTERVAL}]
  Determines how often the \Condor{startd} should send a ClassAd update
  to the \Condor{collector}.  The \Condor{startd} also sends update on any
  state or activity change, or if the value of its \Expr{START} expression
  changes.  See section~\ref{sec:States} on \Condor{startd}
  states, section~\ref{sec:Activities} on \Condor{startd}
  Activities, and section~\ref{sec:Start-Expr} on \Condor{startd}
  \Expr{START} expression for details on states, activities, and the
  \Expr{START} expression.  This macro is defined in
  terms of seconds and defaults to 300 (5 minutes).
  
\label{param:UpdateOffset}
\item[\Macro{UPDATE\_OFFSET}]
  An integer value representing the number of seconds of delay 
  that the \Condor{startd} should wait
  before sending its initial update, and the first update after a
  \Condor{reconfig} command is sent to the \Condor{collector}.
  The time of all other updates sent after this initial update
  is determined by \MacroUNI{UPDATE\_INTERVAL}.
  Thus, the first update will be sent after
  \MacroUNI{UPDATE\_OFFSET} seconds, and the second update will be sent after
  \MacroUNI{UPDATE\_OFFSET} + \MacroUNI{UPDATE\_INTERVAL}.
  This is useful when used in conjunction
  with the \MacroNI{\$RANDOM\_INTEGER()} macro for large pools,
  to spread out the updates
  sent by a large number of \Condor{startd} daemons.
  Defaults to zero.
  The example configuration
  \footnotesize
  \begin{verbatim}
  startd.UPDATE_INTERVAL = 300
  startd.UPDATE_OFFSET   = $RANDOM_INTEGER(0,300)
  \end{verbatim}
  \normalsize
  causes the initial update to occur at a random number of seconds
  falling between 0 and 300,
  with all further updates occurring at fixed 300
  second intervals following the initial update.

\label{param:MachineMaxVacateTime}
\item[\Macro{MachineMaxVacateTime}] An integer expression representing
  the number of seconds the machine is willing to wait for a job that
  has been soft-killed to gracefully shut down.  The default value is
  600 seconds (10 minutes).  This expression is evaluated when the job
  starts running.  The job may adjust the wait time by setting
  \Attr{JobMaxVacateTime}.  If the job's setting is less than the
  machine's, the job's specification is used.  
  If the job's setting is larger than the machine's,
  the result depends on whether the job has any excess
  retirement time.  If the job has more retirement time left than the
  machine's maximum vacate time setting, then retirement time will be
  converted into vacating time, up to the amount of
  \Attr{JobMaxVacateTime}.  The \Macro{KILL} expression may be used
  to abort the graceful shutdown of the job at any time.  At the time
  when the job is preempted, the \Macro{WANT\_VACATE} expression may
  be used to skip the graceful shutdown of the job.

\label{param:MaxJobRetirementTime}
\item[\Macro{MAXJOBRETIREMENTTIME}]
  An integer value representing the number of seconds a preempted job
  will be allowed to run before being evicted. The default value of 0
  (when the configuration variable is not present) means that the job
  gets no retirement time.  If the job vacating policy grants the job
  X seconds of vacating time, a preempted job will be soft-killed X seconds
  before the end of its retirement time, so that hard-killing of the job
  will not happen until the end of the retirement time if the job does
  not finish shutting down before then.
  Note that in peaceful shutdown mode of the
  \Condor{startd}, retirement time is treated as though infinite.
  In graceful shutdown mode, the job will not be preempted until the
  configured retirement time expires or \Macro{SHUTDOWN\_GRACEFUL\_TIMEOUT}
  expires.  In fast shutdown mode, retirement time is ignored.  See
  \MacroNI{MAXJOBRETIREMENTTIME} in
  section~\ref{sec:State-Expression-Summary} for further explanation.

  By default the \Condor{negotiator} will not match jobs to a slot with
  retirement time remaining.  This behavior is controlled by
  \Macro{NEGOTIATOR\_CONSIDER\_EARLY\_PREEMPTION}.

\label{param:ClaimWorklife}
\item[\Macro{CLAIM\_WORKLIFE}]
  This expression specifies the number of seconds after which a claim
  will stop accepting additional jobs.  The default is 3600.  Once the
  \Condor{negotiator} gives a \Condor{schedd} a claim to a slot, 
  the \Condor{schedd} will keep
  running jobs on that slot as long as it has more jobs with matching
  requirements, and \MacroNI{CLAIM\_WORKLIFE} has not expired, and it is
  not preempted.  Once \MacroNI{CLAIM\_WORKLIFE} expires, any existing
  job may continue to run as usual, but once it finishes or is
  preempted, the claim is closed.  When \MacroNI{CLAIM\_WORKLIFE} is -1,
  this is treated as an infinite claim worklife, so claims may be held
  indefinitely (as long as they are not preempted and the user does
  not run out of jobs, of course).  A value of 0 has the effect of not
  allowing more than one job to run per claim, since it immediately
  expires after the first job starts running.

\label{param:MaxClaimAlivesMissed}
\item[\Macro{MAX\_CLAIM\_ALIVES\_MISSED}]
  The \Condor{schedd} sends periodic updates
  to each \Condor{startd} as a keep alive (see the description of
  \Macro{ALIVE\_INTERVAL} on page~\pageref{param:AliveInterval}).  
  If the \Condor{startd} does not receive any keep alive messages, it assumes
  that something has gone wrong with the \Condor{schedd} and that the resource
  is not being effectively used.
  Once this happens, the \Condor{startd} considers the claim to have timed out,
  it releases the claim, and starts advertising itself as available
  for other jobs.
  Because these keep alive messages are sent via UDP, they are
  sometimes dropped by the network.
  Therefore, the \Condor{startd} has some tolerance for missed keep alive
  messages, so that in case a few keep alives are lost, the \Condor{startd}
  will not immediately release the claim.
  This setting controls how many keep alive messages can be missed
  before the \Condor{startd} considers the claim no longer valid.
  The default is 6.

\label{param:StartdHasBadUtmp}
\item[\Macro{STARTD\_HAS\_BAD\_UTMP}]
  When the \Condor{startd} is computing the idle time of all the
  users of the machine (both local and remote), it checks the
  \File{utmp} file to find all the currently active ttys, and only
  checks access time of the devices associated with active logins.
  Unfortunately, on some systems, \File{utmp} is unreliable, and the
  \Condor{startd} might miss keyboard activity by doing this.  So, if your
  \File{utmp} is unreliable, set this macro to \Expr{True} and the
  \Condor{startd} will check the access time on all tty and pty devices.
  
\label{param:ConsoleDevices}
\item[\Macro{CONSOLE\_DEVICES}]
  This macro allows the \Condor{startd} to monitor console (keyboard and mouse)
  activity by checking the access times on special files in
  \File{/dev}.  Activity on these files shows up as 
  \AdAttr{ConsoleIdle}
  time in the \Condor{startd}'s ClassAd.  Give a comma-separated list of
  the names of devices considered the console, without the
  \File{/dev/} portion of the path name.  The defaults vary from
  platform to platform, and are usually correct.  

  One possible exception to this is on Linux, where
  we use ``mouse'' as
  one of the entries.  Most Linux installations put in a
  soft link from \File{/dev/mouse} that points to the appropriate
  device (for example, \File{/dev/psaux} for a PS/2 bus mouse, or
  \File{/dev/tty00} for a serial mouse connected to com1).  However,
  if your installation does not have this soft link, you will either
  need to put it in (you will be glad you did), or change this
  macro to point to the right device. 
  
  Unfortunately, modern versions of Linux do not update the access time of
  device files for USB devices. Thus, these files cannot be be used to
  determine when the console is in use. Instead, use the \Condor{kbdd} daemon,
  which gets this information by connecting to the X server.
  
\label{param:StartdJobExprs}
\item[\Macro{STARTD\_JOB\_EXPRS}]
  When the machine is claimed by a remote user,
  the \Condor{startd} can also advertise
  arbitrary attributes from the job ClassAd in the machine ClassAd.
  List the attribute names to be advertised.  \Note Since
  these are already ClassAd expressions, do not do anything
  unusual with strings.   
  This setting defaults to ``JobUniverse''.

\label{param:StartdAttrs}
\item[\Macro{STARTD\_ATTRS}]
  This macro is described in section~\ref{param:SubsysAttrs} as
  \MacroNI{<SUBSYS>\_ATTRS}.

\label{param:StartdDebug}
\item[\Macro{STARTD\_DEBUG}]
  This macro
  (and other settings related to debug logging in the \Condor{startd}) is
  described in section~\ref{param:SubsysDebug} as
  \MacroNI{<SUBSYS>\_DEBUG}.

\label{param:StartdAddressFile}
\item[\Macro{STARTD\_ADDRESS\_FILE}]
  This macro is described in section~\ref{param:SubsysAddressFile} as
  \MacroNI{<SUBSYS>\_ADDRESS\_FILE} 

\label{param:StartdShouldWriteClaimIdFile}
\item[\Macro{STARTD\_SHOULD\_WRITE\_CLAIM\_ID\_FILE}]
  The \Condor{startd} can be configured
  to write out the \Attr{ClaimId} for the next available claim on all
  slots to separate files.
  This boolean attribute controls whether the \Condor{startd} should
  write these files.
  The default value is \Expr{True}.
  
\label{param:StartdClaimIdFile}
\item[\Macro{STARTD\_CLAIM\_ID\_FILE}]
  This macro controls what file names are used if the above
  \MacroNI{STARTD\_SHOULD\_WRITE\_CLAIM\_ID\_FILE} is true.  By
  default, HTCondor will write the ClaimId into a file in the
  \MacroU{LOG} directory called \File{.startd\_claim\_id.slotX}, where
  \verb@X@ is the value of \Attr{SlotID}, the integer that
  identifies a given slot on the system, or \verb@1@ on a
  single-slot machine.
  If you define your own value for this setting, you should provide a
  full path, and HTCondor will automatically append the \verb@.slotX@
  portion of the file name.

\label{param:SlotWeight}
\item[\Macro{SlotWeight}]
  This may be used to give a slot greater weight when
  calculating usage, computing fair shares, and enforcing group
  quotas.  For example, claiming a slot with \Expr{SlotWeight = 2} is
  equivalent to claiming two \Expr{SlotWeight = 1} slots.  The default
  value is \AdAttr{Cpus}, the number of CPUs associated with the slot,
  which is 1 unless specially configured.  Any expression referring to
  attributes of the slot ClassAd and evaluating to a positive floating
  point number is valid.

\label{param:NumCpus}
\item[\Macro{NUM\_CPUS}]
  An integer value, which can be used to lie to the \Condor{startd} daemon
  about how many CPUs a machine has.
  When set, it overrides the value determined with HTCondor's 
  automatic computation of the number of CPUs in the machine.
  Lying in this way can allow multiple HTCondor jobs to run on a
  single-CPU machine, by having that machine treated like a multi-core 
  machine with multiple CPUs, which could have different HTCondor jobs
  running on each one.
  Or, a multi-core machine may advertise more slots than it has CPUs.
  However, lying in this manner will hurt the performance of the jobs,
  since now multiple jobs will run on the same CPU,
  and the jobs will compete with each other.
  The option is only meant for people who specifically want this
  behavior and know what they are doing.  
  It is disabled by default.

  The default value is equal to \Macro{DETECTED\_CORES} minus
  hyperthreaded cores if \Macro{COUNT\_HYPERTHREAD\_CPUS} is false.  If
  that value exceeds \Macro{MAX\_NUM\_CPUS}, then the latter is used
  instead.

  Note that this setting cannot be changed with a simple reconfigure, 
  either by sending a SIGHUP or by using the \Condor{reconfig} command.
  To change this, restart the \Condor{startd} daemon for the
  change to take effect. The command will be
\begin{verbatim}
  condor_restart -startd
\end{verbatim}

  If lying about a given machine, 
  this fact should probably be advertised in the machine's ClassAd
  by using the \MacroNI{STARTD\_ATTRS} setting.
  This way, jobs submitted in the pool could specify that they did or
  did not want to be matched with machines that were only really
  offering these fractional CPUs.

\label{param:MaxNumCpus}
\item[\Macro{MAX\_NUM\_CPUS}]
  An integer value used as a ceiling for the number of CPUs detected
  by HTCondor on a machine.
  This value is ignored if \MacroNI{NUM\_CPUS} is set.
  If set to zero, there is no ceiling. 
  If not defined, the default value is zero, and thus there is no ceiling. 

  Note that this setting cannot be changed with a simple reconfigure, 
  either by sending a SIGHUP or by using the \Condor{reconfig} command.
  To change this, restart the \Condor{startd} daemon for the
  change to take effect. The command will be
\begin{verbatim}
  condor_restart -startd
\end{verbatim}

\label{param:CountHyperthreadCpus}
\item[\Macro{COUNT\_HYPERTHREAD\_CPUS}]
  This macro controls how HTCondor sees hyper threaded
  processors. When set to \Expr{True} (the default), it includes virtual CPUs in
  the default value of \MacroNI{NUM\_CPUS}. On dedicated cluster nodes, 
  counting virtual CPUs can sometimes improve total throughput at the expense 
  of individual job speed. However, counting them on desktop workstations can
  interfere with interactive job performance.

\label{param:Memory}
\item[\Macro{MEMORY}]
  Normally, HTCondor will automatically detect the amount of physical
  memory available on your machine.  Define \MacroNI{MEMORY} to tell
  HTCondor how much physical memory (in MB) your machine has, overriding
  the value HTCondor computes automatically.  The actual amount of memory
  detected by HTCondor is always available in the pre-defined configuration
  macro \Macro{DETECTED\_MEMORY}.

\label{param:ReservedMemory}
\item[\Macro{RESERVED\_MEMORY}]
  How much memory would you like reserved from HTCondor?  By default,
  HTCondor considers all the physical memory of your machine as
  available to be used by HTCondor jobs.  If \MacroNI{RESERVED\_MEMORY} is
  defined, HTCondor subtracts it from the amount of memory it advertises
  as available.

\label{param:StartdName}
\item[\Macro{STARTD\_NAME}]
  Used to give an alternative value to the \Attr{Name} attribute
  in the \Condor{startd}'s ClassAd.
  This esoteric configuration macro might be used in the situation
  where there are two \Condor{startd} daemons running on one machine,
  and each reports to the same \Condor{collector}.
  Different names will distinguish the two daemons.
  See the description of \MacroNI{MASTER\_NAME} in
  section~\ref{param:MasterName} on page~\pageref{param:MasterName}
  for defaults and composition of valid HTCondor daemon names.

\label{param:RunBenchmarks}
\item[\Macro{RUNBENCHMARKS}]
  Specifies when to run benchmarks.
  When the machine is in the Unclaimed state and this expression
  evaluates to \Expr{True}, benchmarks will be run.
  If RunBenchmarks is specified and set to anything other than \Expr{False},
  additional benchmarks will be run when the \Condor{startd} initially starts.
  To disable start up benchmarks, set \MacroNI{RunBenchmarks} to \Expr{False},
  or comment it out of the configuration file.

\label{param:DedicatedScheduler}
\item[\Macro{DedicatedScheduler}]
  A string that identifies the dedicated scheduler this machine is managed by.
  Section~\ref{sec:Configure-Dedicated-Resource}
  on page~\pageref{sec:Configure-Dedicated-Resource} details the use of
  a dedicated scheduler.

\label{param:StartdNoclaimShutdown}
\item[\Macro{STARTD\_NOCLAIM\_SHUTDOWN}]
  The number of seconds to run without receiving a claim before
  shutting HTCondor down on this machine.  Defaults to unset, which
  means to never shut down.  
  This is primarily intended to facilitate glidein;
  use in other situations is not recommended.

\label{param:StartdPublishWinreg}
\item[\Macro{STARTD\_PUBLISH\_WINREG}]
  A string containing a semicolon-separated list of Windows registry key names.
  For each registry key, the contents of the registry key are published in
  the machine ClassAd.
  All attribute names are prefixed with \Expr{WINREG\_}. 
  The remainder of the attribute name is formed in one of two ways.
  The first way explicitly specifies the name within the list with the
  syntax
\begin{verbatim}
  STARTD_PUBLISH_WINREG = AttrName1 = KeyName1; AttrName2 = KeyName2
\end{verbatim}

  The second way of forming the attribute name derives the attribute names
  from the key names in the list.
  The derivation uses the last three path elements in the key name and changes
  each illegal character to an underscore character.
  Illegal characters are essentially any non-alphanumeric character. 
  In addition, the percent character (\verb@%@) is replaced by the
  string \Expr{Percent},
  and the string \Expr{/sec} is replaced by the string \Expr{\_Per\_Sec}.

  HTCondor expects that the hive identifier, 
  which is the first element in the full path given by a key name,
  will be the valid abbreviation.
  Here is a list of abbreviations:
  \begin{description}
    \item \Expr{HKLM} is the abbreviation for  \Expr{HKEY\_LOCAL\_MACHINE}
    \item \Expr{HKCR} is the abbreviation for  \Expr{HKEY\_CLASSES\_ROOT}
    \item \Expr{HKCU} is the abbreviation for  \Expr{HKEY\_CURRENT\_USER}
    \item \Expr{HKPD} is the abbreviation for  \Expr{HKEY\_PERFORMANCE\_DATA}
    \item \Expr{HKCC} is the abbreviation for  \Expr{HKEY\_CURRENT\_CONFIG}
    \item \Expr{HKU} is the abbreviation for  \Expr{HKEY\_USERS}
  \end{description}
  The \Expr{HKPD} key names are unusual, 
  as they are not shown in \Prog{regedit}.
  Their values are periodically updated at the interval defined by 
  \MacroNI{UPDATE\_INTERVAL}. 
  The others are not updated until \Condor{reconfig} is issued.

  Here is a complete example of the configuration variable definition,
\begin{verbatim}
    STARTD_PUBLISH_WINREG = HKLM\Software\Perl\BinDir; \
     BATFile_RunAs_Command = HKCR\batFile\shell\RunAs\command; \
     HKPD\Memory\Available MBytes; \
     BytesAvail = HKPD\Memory\Available Bytes; \
     HKPD\Terminal Services\Total Sessions; \
     HKPD\Processor\% Idle Time; \
     HKPD\System\Processes
\end{verbatim}
  which generates the following portion of a machine ClassAd: 
\begin{verbatim}
  WINREG_Software_Perl_BinDir = "C:\Perl\bin\perl.exe"
  WINREG_BATFile_RunAs_Command = "%SystemRoot%\System32\cmd.exe /C \"%1\" %*"
  WINREG_Memory_Available_MBytes = 5331
  WINREG_BytesAvail = 5590536192.000000
  WINREG_Terminal_Services_Total_Sessions = 2
  WINREG_Processor_Percent_Idle_Time = 72.350384
  WINREG_System_Processes = 166
\end{verbatim}

\label{param:MountUnderScratch}
\item[\Macro{MOUNT\_UNDER\_SCRATCH}]
  A comma separated list of directories. 
  For each directory in the list, 
  HTCondor creates a directory in the job's temporary scratch directory 
  with that name,
  and makes it available at the given name using bind mounts.
  This is available on Linux systems which provide bind mounts 
  and per-process tree mount tables,
  such as Red Hat Enterprise Linux 5.  
  A bind mount is like a symbolic link,
  but is not globally visible to all processes.
  It is only visible to the job and the job's child processes.
  As an example:
\begin{verbatim}
  MOUNT_UNDER_SCRATCH = /tmp,/var/tmp
\end{verbatim}
  The job will see the usual \File{/tmp} and \File{/var/tmp}
  directories, but when accessing files via these paths, the
  system will redirect the access. 
  The resultant files will actually end up in directories named 
  \File{tmp} or \File{var/tmp}
  under the the job's temporary scratch directory. 
  This is useful, because the job's scratch directory will be cleaned up 
  after the job completes, 
  two concurrent jobs will not interfere with each other,
  and because jobs will not be able to fill up the real \File{/tmp} directory. 
  Another use case might be for home directories, which some jobs might want
  to write to, but that should be cleaned up after each job run.
  The default value if not defined will be that no directories are mounted in
  the job's temporary scratch directory.

\end{description}

These macros control if the \Condor{startd} daemon should perform
backfill computations whenever resources would otherwise be idle.  
See section~\ref{sec:Backfill} on page~\pageref{sec:Backfill} on
Configuring HTCondor for Running Backfill Jobs for details.

\begin{description}

\label{param:EnableBackfill}
\item[\Macro{ENABLE\_BACKFILL}]
  A boolean value that, when \Expr{True}, indicates that the machine is willing
  to perform backfill computations when it would otherwise be idle.
  This is not a policy expression that is evaluated, it is a simple
  \Expr{True} or \Expr{False}.
  This setting controls if any of the other backfill-related
  expressions should be evaluated.
  The default is \Expr{False}.

\label{param:BackfillSystem}
\item[\Macro{BACKFILL\_SYSTEM}]
  A string that defines what backfill system to use for spawning and managing
  backfill computations.
  Currently, the only supported value for this is \AdStr{BOINC}, which
  stands for the \Term{Berkeley Open Infrastructure for Network
  Computing}.
  See \URL{http://boinc.berkeley.edu} for more information about
  BOINC.
  There is no default value, administrators must define this.
  
\label{param:StartBackfill}
\item[\Macro{START\_BACKFILL}]
  A boolean expression that is evaluated whenever an HTCondor resource is in the
  Unclaimed/Idle state and the \MacroNI{ENABLE\_BACKFILL} expression
  is \Expr{True}.  
  If \MacroNI{START\_BACKFILL} evaluates to \Expr{True}, the machine
  will enter the Backfill state and attempt to spawn a backfill
  computation. 
  This expression is analogous to the \Macro{START} expression that
  controls when an HTCondor resource is available to run normal HTCondor
  jobs.
  The default value is \Expr{False} (which means do not spawn a
  backfill job even if the machine is idle and
  \MacroNI{ENABLE\_BACKFILL} expression is \Expr{True}).
  For more information about policy expressions and the Backfill
  state, see section~\ref{sec:Configuring-Policy} beginning on
  page~\pageref{sec:Configuring-Policy}, especially
  sections~\ref{sec:States}, \ref{sec:Activities}, and
  \ref{sec:State-and-Activity-Transitions}.

\label{param:EvictBackfill}
\item[\Macro{EVICT\_BACKFILL}]
  A boolean expression that is evaluated whenever an HTCondor resource is in the
  Backfill state which, when \Expr{True}, indicates the machine should
  immediately kill the currently running backfill computation and
  return to the Owner state.
  This expression is a way for administrators to define a policy where
  interactive users on a machine will cause backfill jobs to be
  removed.
  The default value is \Expr{False}.
  For more information about policy expressions and the Backfill
  state, see section~\ref{sec:Configuring-Policy} beginning on
  page~\pageref{sec:Configuring-Policy}, especially
  sections~\ref{sec:States}, \ref{sec:Activities}, and
  \ref{sec:State-and-Activity-Transitions}.

\end{description}


These macros only apply to the \Condor{startd} daemon when it is running on 
a multi-core machine. 
See section~\ref{sec:Configuring-SMP} on
page~\pageref{sec:Configuring-SMP} for details.

\begin{description}

\label{param:StartdResourcePrefix}
\item[\Macro{STARTD\_RESOURCE\_PREFIX}] 
  A string which specifies what prefix to give the unique HTCondor
  resources that are advertised on multi-core machines.
  Previously, HTCondor used the term \Term{virtual machine} to describe
  these resources, so the default value for this setting was \Expr{vm}.
  However, to avoid confusion with other kinds of virtual machines,
  such as the ones created using tools like VMware or Xen, the old
  \Term{virtual machine} terminology has been changed, and has become
  the term \Term{slot}.
  Therefore, the default value of this prefix is now \Expr{slot}.
  If sites want to continue using \Expr{vm}, or prefer something other
  \Expr{slot}, this setting enables sites to define what string the
  \Condor{startd} will use to name the individual resources on a multi-core
  machine.

\label{param:SlotsConnectedToConsole}
\item[\Macro{SLOTS\_CONNECTED\_TO\_CONSOLE}] 
  An integer which indicates how many of the machine slots
  the \Condor{startd} is representing should be "connected" to the
  console (in other words, notice when there's console activity).
  This defaults to all slots (N in a machine with N CPUs).

\label{param:SlotsConnectedToKeyboard}
\item[\Macro{SLOTS\_CONNECTED\_TO\_KEYBOARD}]
  An integer which indicates how many of the machine slots
  the \Condor{startd} is representing should be "connected" to the
  keyboard (for remote tty activity, as well as console activity).
  Defaults to 1.

\label{param:DisconnectedKeyboardIdleBoost}
\item[\Macro{DISCONNECTED\_KEYBOARD\_IDLE\_BOOST}]
  If there are slots not connected to either the keyboard
  or the console, the corresponding idle time reported will be the
  time since the \Condor{startd} was spawned, plus the value of this macro.
  It defaults to 1200 seconds (20 minutes). 
  We do this because if the slot is configured not to care
  about keyboard activity, we want it to be available to HTCondor jobs
  as soon as the \Condor{startd} starts up, instead of having to wait for 15
  minutes or more (which is the default time a machine must be idle
  before HTCondor will start a job).
  If you do not want this boost, set the value to 0.  
  If you change your START expression to require more than 15 minutes
  before a job starts, but you still want jobs to start right away on
  some of your multi-core nodes, increase this macro's value.

\label{param:StartdSlotAttrs}
\item[\Macro{STARTD\_SLOT\_ATTRS}]
  The list of ClassAd attribute names that should be shared across all
  slots on the same machine.
  This setting was formerly know as \Macro{STARTD\_VM\_ATTRS} or
  \Macro{STARTD\_VM\_EXPRS} (before version 6.9.3).
  For each attribute in the list, the attribute's value is taken from
  each slot's machine ClassAd and placed into the machine
  ClassAd of all the other slots within the machine.
  For example, if the configuration file for a 2-slot machine
  contains
\begin{verbatim}
        STARTD_SLOT_ATTRS = State, Activity, EnteredCurrentActivity
\end{verbatim}
  then the machine ClassAd for both slots will contain
  attributes that will be of the form:
\begin{verbatim}
     slot1_State = "Claimed"
     slot1_Activity = "Busy"
     slot1_EnteredCurrentActivity = 1075249233
     slot2_State = "Unclaimed"
     slot2_Activity = "Idle"
     slot2_EnteredCurrentActivity = 1075240035
\end{verbatim}

\end{description}

The following settings control the number of slots reported
for a given multi-core host, and what attributes each one has.  
They are only needed if you do not want to have a multi-core machine report
to HTCondor with a separate slot for each CPU, with all
shared system resources evenly divided among them.
Please read section~\ref{sec:SMP-Divide} on
page~\pageref{sec:SMP-Divide} for details on how to properly configure
these settings to suit your needs.

\Note You can only change the number of each type of slot
the \Condor{startd} is reporting with a simple reconfig (such as
sending a SIGHUP signal, or using the \Condor{reconfig} command).
You cannot change the definition of the different slot
types with a reconfig.  
If you change them, you must restart the \Condor{startd} for the
change to take effect (for example, using 
\Code{condor\_restart -startd}).

\Note Prior to version 6.9.3, any settings that included the term
\Expr{slot} used to use virtual machine or \Expr{vm}.
If searching for information about one of these older settings,
search for the corresponding attribute names using \Expr{slot}, instead.

\begin{description}

\label{param:MaxSlotTypes}
\item[\Macro{MAX\_SLOT\_TYPES}]
  The maximum number of different slot types.  
  Note: this is the maximum number of different \emph{types}, not of
  actual slots.
  Defaults to 10.  
  (You should only need to change this setting if you define more than
  10 separate slot types, which would be pretty rare.)

\label{param:SlotTypeN}
\item[\Macro{SLOT\_TYPE\_<N>}]
  This setting defines a given slot type, by specifying
  what part of each shared system resource (like RAM, swap space, etc)
  this kind of slot gets.  This setting has \emph{no} effect unless you also
  define \MacroNI{NUM\_SLOTS\_TYPE\_<N>}.
  N can be any integer from 1 to the value of
  \MacroUNI{MAX\_SLOT\_TYPES}, such as
  \MacroNI{SLOT\_TYPE\_1}. 
  The format of this entry can be somewhat complex, so please refer to
  section~\ref{sec:SMP-Divide} on page~\pageref{sec:SMP-Divide} for
  details on the different possibilities.

\label{param:SlotTypeNPartitionable}
\item[\Macro{SLOT\_TYPE\_<N>\_PARTITIONABLE}]
  A boolean variable that defaults to \Expr{False}.
  When \Expr{True}, this slot permits dynamic provisioning, as specified in
  section~ \ref{sec:SMP-dynamicprovisioning}.

\label{param:ClaimPartitionableLeftovers}
\item[\Macro{CLAIM\_PARTITIONABLE\_LEFTOVERS}]
  A boolean variable that defaults to \Expr{True}.
  When \Expr{True} within the configuration for both the \Condor{schedd}
  and the \Condor{startd}, 
  and the \Condor{schedd} claims a partitionable slot,
  the \Condor{startd} returns the slot's ClassAd and a claim id for
  leftover resources.
  In doing so, the \Condor{schedd} can claim multiple dynamic slots 
  without waiting for a negotiation cycle.

\label{param:MachineResourceNames}
\item[\Macro{MACHINE\_RESOURCE\_NAMES}]
  A comma and/or space separated list of resource names that represent
  custom resources specific to a machine.
  These resources are further intended to be statically divided or
  partitioned, and these resource names identify the configuration variables
  that define the partitioning.

\label{param:MachineResourceResourcename}
\item[\Macro{MACHINE\_RESOURCE\_<name>}]
  An integer that specifies the quantity of the customized local machine
  resource available for an SMP machine.
  The portion of this configuration variable's name identified with
  \Expr{<name>} is as defined in \Macro{MACHINE\_RESOURCE\_NAMES}.

\label{param:MustModifyRequestExprs}
\item[\Macro{MUST\_MODIFY\_REQUEST\_EXPRS}]
  A boolean value that defaults to \Expr{False}.
  When \Expr{False},
  configuration variables whose names begin with 
  \MacroNI{MODIFY\_REQUEST\_EXPR} are only applied if the job claim 
  still matches the partitionable slot after modification.
  If \Expr{True},
  the modifications always take place, 
  and if the modifications cause the claim to no longer match, 
  then the \Condor{startd} will simply refuse the claim.

\label{param:ModifyRequestExprRequestmemory}
\item[\Macro{MODIFY\_REQUEST\_EXPR\_REQUESTMEMORY}]
  A boolean expression used by the \Condor{startd} daemon to modify the
  evaluated value of the \Attr{RequestMemory} job ClassAd attribute,
  before it used to provision a dynamic slot. 
  The default value is given by
  \footnotesize
  \begin{verbatim}
  quantize(RequestMemory,{128})
  \end{verbatim}
  \normalsize

\label{param:ModifyRequestExprRequestdisk}
\item[\Macro{MODIFY\_REQUEST\_EXPR\_REQUESTDISK}]
  A boolean expression used by the \Condor{startd} daemon to modify the
  evaluated value of the \Attr{RequestDisk} job ClassAd attribute,
  before it used to provision a dynamic slot. 
  The default value is given by
  \Expr{quantize(RequestDisk,{1024})}.

\label{param:ModifyRequestExprRequestcpus}
\item[\Macro{MODIFY\_REQUEST\_EXPR\_REQUESTCPUS}]
  A boolean expression used by the \Condor{startd} daemon to modify the
  evaluated value of the \Attr{RequestCpus} job ClassAd attribute,
  before it used to provision a dynamic slot. 
  The default value is given by
  \Expr{quantize(RequestCpus,{1})}

\label{param:NumSlotsTypeN}
\item[\Macro{NUM\_SLOTS\_TYPE\_<N>}]
  This macro controls how many of a given slot type
  are actually reported to HTCondor.
  There is no default.

\label{param:NumSlots}
\item[\Macro{NUM\_SLOTS}]
  An integer value representing the number of slots reported when
  the multi-core machine is being evenly divided, and the slot
  type settings described above are not being used.
  The default is one slot for each CPU.
  This setting can be used to reserve some CPUs on a multi-core machine,
  which would
  not be reported to the HTCondor pool.
  This value cannot be used to
  make HTCondor advertise more slots than there are CPUs on the machine.
  To do that, use \Macro{NUM\_CPUS}.

\label{param:AllowVMCruft}
\item[\Macro{ALLOW\_VM\_CRUFT}]
  A boolean value that HTCondor sets and uses internally, currently
  defaulting to \Expr{True}.  When \Expr{True},
  HTCondor looks for configuration variables named with the
  previously used string \MacroNI{VM} after searching unsuccessfully
  for variables named with the currently used string \MacroNI{SLOT}.
  When \Expr{False}, HTCondor does \emph{not} look for variables named
  with the previously used string \MacroNI{VM} after searching
  unsuccessfully for the string \MacroNI{SLOT}. 

\end{description}

The following configuration variables support java universe jobs.

\begin{description}
\label{param:Java}
\item[\Macro{JAVA}]
  The full path to the Java interpreter (the Java Virtual Machine).

\label{param:JavaClasspathArgument}
\item[\Macro{JAVA\_CLASSPATH\_ARGUMENT}]
  The command line argument to the Java interpreter (the Java Virtual Machine)
  that specifies the Java Classpath.
  Classpath is a Java-specific term that denotes the list of
  locations (\File{.jar} files and/or directories)
  where the Java interpreter can
  look for the Java class files that a Java program requires.

\label{param:JavaClasspathSeparator}
\item[\Macro{JAVA\_CLASSPATH\_SEPARATOR}]
  The single character used to delimit constructed entries in the
  Classpath for the given operating system and Java Virtual Machine.
  If not defined, the operating system is queried for its default
  Classpath separator.

\label{param:JavaClasspathDefault}
\item[\Macro{JAVA\_CLASSPATH\_DEFAULT}]
  A list of path names to \File{.jar} files to be added to the Java Classpath 
  by default.
  The comma and/or space character delimits list entries.

\label{param:JavaExtraArguments}
\item[\Macro{JAVA\_EXTRA\_ARGUMENTS}]
  A list of additional arguments to be passed to the Java executable.
\end{description}

The following configuration variables control .NET version advertisement.

\begin{description}
\label{param:StartdPublishDotnet}
\item[\Macro{STARTD\_PUBLISH\_DOTNET}]
  A boolean value that controls the advertising of the .NET framework 
  on Windows platforms.
  When \Expr{True}, the \Condor{startd} will advertise all installed versions 
  of the .NET framework within the \Attr{DotNetVersions} attribute 
  in the \Condor{startd} machine ClassAd.
  The default value is \Expr{True}.
  Set the value to \Expr{false} to turn off .NET version advertising.
  
\label{param:DotNetVersions}
\item[\Macro{DOT\_NET\_VERSIONS}]
  A string expression that administrators can use to override the way that
  .NET versions are advertised.  
  If the administrator wishes to advertise .NET installations,
  but wishes to do so in a format different than what the \Condor{startd}
  publishes in its ClassAds,
  setting a string in this expression will result in the \Condor{startd}
  publishing the string when \Macro{STARTD\_PUBLISH\_DOTNET} is \Expr{True}.
  No value is set by default.
\end{description}

These macros control the power management capabilities of the 
\Condor{startd} to optionally put the machine in to a low power state
and wake it up later.
See section~\ref{sec:power-man} on page~\pageref{sec:power-man} on
Power Management for more details.

\begin{description}

\label{param:HibernateCheckInterval}
\item[\Macro{HIBERNATE\_CHECK\_INTERVAL}]
  An integer number of seconds that
  determines how often the \Condor{startd} checks to see if the 
  machine is ready to enter a low power state.
  The default value is 0,
  which disables the check.
  If not 0, the \MacroNI{HIBERNATE} expression is
  evaluated within the context of each slot at the given interval.  
  If used, a value 300 (5 minutes) is recommended.

  As a special case, the interval is ignored when the 
  machine has just returned from a low power state, 
  excluding \Expr{"SHUTDOWN"}.
  In order to avoid machines from volleying between 
  a running state and a low power state, an hour of uptime is enforced
  after a machine has been woken.  After the hour has passed,
  regular checks resume.

\label{param:Hibernate}
\item[\Macro{HIBERNATE}]
  A string expression that represents lower power state.  When this
  state name evaluates to a valid state other than \Expr{"NONE"},
  causes HTCondor to put the machine into the specified low power state.
  The following names are supported
  (and are not case sensitive):

  \begin{itemize}
  \item[] \Expr{"NONE"}, \Expr{"0"}: No-op; do not enter a low power state
  \item[] \Expr{"S1"},   \Expr{"1"}, \Expr{"STANDBY"}, \Expr{"SLEEP"}: On Windows, this is Sleep (standby)
  \item[] \Expr{"S2"},   \Expr{"2"}: On Windows, this is Sleep (standby)
  \item[] \Expr{"S3"},   \Expr{"3"}, \Expr{"RAM"}, \Expr{"MEM"}, \Expr{"SUSPEND"}: On Windows, this is Sleep (standby)
  \item[] \Expr{"S4"},   \Expr{"4"}, \Expr{"DISK"}, \Expr{"HIBERNATE"}: Hibernate
  \item[] \Expr{"S5"},   \Expr{"5"}, \Expr{"SHUTDOWN"}, \Expr{"OFF"}: Shutdown (soft-off)
  \end{itemize}
  
  The \MacroNI{HIBERNATE} expression is written in terms of the S-states
  as defined in the Advanced Configuration and Power Interface 
  (ACPI) specification.  The S-states take the form S<n>, where <n> is 
  an integer in the range 0 to 5, inclusive.  The number that results 
  from evaluating the expression determines which S-state to enter. The 
  notation was adopted because
  it appears to be the standard naming scheme for power states on several
  popular operating systems, including various flavors of Windows and Linux
  distributions.  The other strings, such as \Expr{"RAM"} and \Expr{"DISK"},
  are provided for ease of configuration.

  Since this expression is evaluated in the context of each slot on the
  machine, any one slot has veto power over the other slots.  If the 
  evaluation of \MacroNI{HIBERNATE} in one slot evaluates to \Expr{"NONE"}
  or \Expr{"0"}, then the machine will not be placed into a low power
  state.  On the other 
  hand, if all slots evaluate to a non-zero value, but differ in value, 
  then the largest value is used as the representative power state.

  Strings that do not match any in the table above are treated as
  \Expr{"NONE"}.

\label{param:Unhibernate}
\item[\Macro{UNHIBERNATE}]
  A boolean expression that specifies when an offline machine should be
  woken up.
  The default value is \Expr{MachineLastMatchTime =!= UNDEFINED}.
  This expression does not do anything,
  unless there is an instance of \Condor{rooster} running,
  or another program that evaluates the
  \Attr{Unhibernate} expression of offline machine ClassAds.
  In addition, the collecting of offline machine ClassAds must be enabled
  for this expression to work.  The variable 
  \Macro{COLLECTOR\_PERSISTENT\_AD\_LOG} on page~\pageref{param:CollectorPersistentAdLog}
  detailed on page~\pageref{param:OfflineLog} explains this.
  The special attribute
  \Attr{MachineLastMatchTime} is updated in the ClassAds of offline machines
  when a job would have been matched to the machine if it had been online.
  For multi-slot machines, the offline ClassAd for slot1 will also contain
  the attributes \Attr{slot<X>\_MachineLastMatchTime},
  where \MacroNI{X} is replaced by the
  slot id of the other slots that would have been matched while offline.
  This allows the slot1 \Macro{UNHIBERNATE} expression to refer to
  all of the slots on the machine, in case that is necessary.
  By default,
  \Condor{rooster} will wake up a machine if any slot on the machine has
  its \Macro{UNHIBERNATE} expression evaluate to \Expr{True}.

\label{param:HibernationPlugin}
\item[\Macro{HIBERNATION\_PLUGIN}]
  A string which specifies the path and executable name of 
  the hibernation plug-in that the \Condor{startd} should use 
  in the detection of low power states and switching to the low power states.
  The default value is \File{\$(LIBEXEC)/power\_state}.  
  A default executable in that location which meets these specifications is
  shipped with HTCondor. 

  The \Condor{startd} initially invokes this plug-in with both the
  value defined for \MacroNI{HIBERNATION\_PLUGIN\_ARGS}
  and the argument \Arg{ad}, 
  and expects the plug-in to output a ClassAd to its standard output stream.
  The \Condor{startd} will use this ClassAd to determine what low power
  setting to use on further invocations of the plug-in.
  To that end, the ClassAd must contain the attribute
  \Attr{HibernationSupportedStates}, a comma separated list of
  low power modes that are available.  
  The recognized mode strings are the same as those in the table for
  the configuration variable \MacroNI{HIBERNATE}.
  The optional attribute \Attr{HibernationMethod} specifies a string 
  which describes the mechanism used by the plug-in.
  The default Linux plug-in shipped with HTCondor will produce 
  one of the strings
  \verb@NONE@, \verb@/sys@, \verb@/proc@, or \verb@pm-utils@.
  The optional attribute \Attr{HibernationRawMask}
  is an integer which represents the bit mask of the modes detected.

  Subsequent \Condor{startd} invocations of the plug-in have command
  line arguments defined by \MacroNI{HIBERNATION\_PLUGIN\_ARGS} plus the
  argument \OptArg{set}{<power-mode>}, where \Arg{<power-mode>}
  is one of the supported states as given in the attribute
  \Attr{HibernationSupportedStates}.

\label{param:HibernationPluginArgs}
\item[\Macro{HIBERNATION\_PLUGIN\_ARGS}]
  Command line arguments appended to the command that invokes the plug-in.
  The additional argument \Arg{ad} is appended  
  when the \Condor{startd} initially invokes the plug-in.

\label{param:HibernationOverrideWOL}
\item[\Macro{HIBERNATION\_OVERRIDE\_WOL}]
  A boolean value that defaults to \Expr{False}.
  When \Expr{True}, it causes the \Condor{startd} daemon's detection of
  the whether or not the network interface handles WOL packets to be ignored.
  When \Expr{False}, hibernation is disabled if the network interface
  does not use WOL packets to wake from hibernation.
  Therefore, when \Expr{True} hibernation can be enabled despite
  the fact that WOL packets are not used to wake machines.

\label{param:LinuxHibernationMethod}
\item[\Macro{LINUX\_HIBERNATION\_METHOD}]
  A string that can be used to override the default search used by
  HTCondor on Linux platforms to detect the hibernation method to use.
  This is used by the default hibernation plug-in executable that is
  shipped with HTCondor.  The default behavior orders its search with:
  \begin{enumerate}
  \item Detect and use the \Prog{pm-utils} command line tools.
    The corresponding string is defined with \verb@"pm-utils"@. 
  \item Detect and use the directory in the virtual file system
    \File{/sys/power}.
    The corresponding string is defined with \verb@"/sys"@.
  \item Detect and use the directory in the virtual file system
    \File{/proc/ACPI}.
    The corresponding string is defined with \verb@"/proc"@.
  \end{enumerate}
  To override this ordered search behavior,
  and force the use of one particular method,
  set \MacroNI{LINUX\_HIBERNATION\_METHOD} to one of the defined strings.

\label{param:OfflineLog}
\item[\Macro{OFFLINE\_LOG}]
  This configuration variable is no longer used.
  It has been replaced by \MacroNI{COLLECTOR\_PERSISTENT\_AD\_LOG}.

\label{param:OfflineExpireAdsAfter}
\item[\Macro{OFFLINE\_EXPIRE\_ADS\_AFTER}]
  An integer number of seconds specifying the lifetime of the
  persistent machine ClassAd representing a hibernating machine.
  Defaults to the largest 32-bit integer.

\end{description}

The following macros control the optional computation of resource
availability statistics in the \Condor{startd}.

\begin{description}

\label{param:StartdComputeAvailStats}
\item[\Macro{STARTD\_COMPUTE\_AVAIL\_STATS}]
  A boolean value that determines if the \Condor{startd} computes resource
  availability statistics.  The default is \Expr{False}.

  If \Macro{STARTD\_COMPUTE\_AVAIL\_STATS} is \Expr{True}, 
  the \Condor{startd} will
  define the following ClassAd attributes for resources:

  \begin{description}
  \item[\AdAttr{AvailTime}]
  \index{ClassAd machine attribute!AvailTime}
    The proportion of the time (between 0.0 and 1.0)
    that this resource has been in a state other than Owner.
  \item[\AdAttr{LastAvailInterval}]
  \index{ClassAd machine attribute!LastAvailInterval}
    The duration in seconds of the last period between Owner states.
  \end{description}

  The following attributes will also be included if the resource is
  not in the Owner state:

  \begin{description}
  \item[\AdAttr{AvailSince}]
  \index{ClassAd machine attribute!AvailSince}
    The time at which the resource last left the
    Owner state.  Measured in the number of seconds since the
    epoch (00:00:00 UTC, Jan 1, 1970).
  \item[\AdAttr{AvailTimeEstimate}]
  \index{ClassAd machine attribute!AvailTimeEstimate}
    Based on past history, an estimate
    of how long the current period between Owner states will last.
  \end{description}

\label{param:StartdAvailConfidence}
\item[\Macro{STARTD\_AVAIL\_CONFIDENCE}]
  A floating point number representing the confidence level of the
  \Condor{startd} daemon's \Attr{AvailTime} estimate.
  By default, the estimate is based on
  the 80th percentile of past values, so the value is initially set to 0.8.

\label{param:StartdMaxAvailPeriodSamples}
\item[\Macro{STARTD\_MAX\_AVAIL\_PERIOD\_SAMPLES}]
  An integer that limits the number of samples of past available
  intervals stored by the \Condor{startd} to limit memory and disk consumption.
  Each sample requires 4 bytes of memory and approximately 10 bytes of
  disk space.

\end{description}

%%%%%%%%%%%%%%%%%%%%%%%%%%%%%%%%%%%%%%%%%%%%%%%%%%%%%%%%%%%%%%%%%%%%%%%%%%%
\subsection{\label{sec:Schedd-Config-File-Entries}\condor{schedd}
Configuration File Entries}
%%%%%%%%%%%%%%%%%%%%%%%%%%%%%%%%%%%%%%%%%%%%%%%%%%%%%%%%%%%%%%%%%%%%%%%%%%%

\index{configuration!condor\_schedd configuration variables}
These macros control the \Condor{schedd}.
\begin{description}

\label{param:Shadow}
\item[\Macro{SHADOW}]
  This macro determines the
  full path of the \Condor{shadow} binary that the \Condor{schedd}
  spawns.  It is normally defined in terms of \MacroUNI{SBIN}. 
  
\label{param:StartLocalUniverse}
\item[\Macro{START\_LOCAL\_UNIVERSE}]
  A boolean value that defaults to \Expr{TotalLocalJobsRunning < 200}.
  The \Condor{schedd} uses this macro to determine whether to start
  a \SubmitCmd{local} universe job. 
  At intervals determined by \MacroNI{SCHEDD\_INTERVAL}, 
  the \Condor{schedd} daemon evaluates this macro
  for each idle \SubmitCmd{local} universe job that it has.
  For each job, if the \MacroNI{START\_LOCAL\_UNIVERSE} 
  macro is \Expr{True}, then the job's \Macro{Requirements} expression
  is evaluated. If both conditions are met, then the job is allowed
  to begin execution. 
  
  The following example only allows 10 \SubmitCmd{local} universe jobs to
  execute concurrently. The attribute \Attr{TotalLocalJobsRunning}
  is supplied by \Condor{schedd}'s ClassAd:
  
  \footnotesize
  \begin{verbatim}
    START_LOCAL_UNIVERSE = TotalLocalJobsRunning < 10
  \end{verbatim}
  \normalsize
  
\label{param:StarterLocal}
\item[\Macro{STARTER\_LOCAL}]
  The complete path and executable name of the \Condor{starter} to
  run for \SubmitCmd{local} universe jobs.  This variable's value
  is defined in the initial configuration provided with HTCondor as
  \footnotesize
  \begin{verbatim}
  STARTER_LOCAL = $(SBIN)/condor_starter
  \end{verbatim}
  \normalsize
  This variable would only be modified or hand added into 
  the configuration for a pool to be upgraded from one
  running a version of HTCondor that existed before the
  \SubmitCmd{local} universe to one that includes the
  \SubmitCmd{local} universe, but without utilizing the newer, provided
  configuration files.

\label{param:LocalUnivExecute}
\item[\Macro{LOCAL\_UNIV\_EXECUTE}]
  A string value specifying the execute location for local
  universe jobs.  Each running local universe job will receive a
  uniquely named subdirectory within this directory.
  If not specified, it defaults to \File{\$(SPOOL)/local\_univ\_execute}.

\label{param:StartSchedulerUniverse}
\item[\Macro{START\_SCHEDULER\_UNIVERSE}]
  A boolean value that defaults to \Expr{TotalSchedulerJobsRunning < 200}.
  The \Condor{schedd} uses this macro to determine whether to start
  a \SubmitCmd{scheduler} universe job. 
  At intervals determined by \MacroNI{SCHEDD\_INTERVAL}, 
  the \Condor{schedd} daemon evaluates this macro
  for each idle \SubmitCmd{scheduler} universe job that it has.
  For each job, if the \MacroNI{START\_SCHEDULER\_UNIVERSE} 
  macro is \Expr{True}, then the job's \Macro{Requirements} expression
  is evaluated. If both conditions are met, then the job is allowed
  to begin execution. 
  
  The following example only allows 10 \SubmitCmd{scheduler} universe jobs to
  execute concurrently. The attribute \Attr{TotalSchedulerJobsRunning}
  is supplied by \Condor{schedd}'s ClassAd:
  
  \footnotesize
  \begin{verbatim}
    START_SCHEDULER_UNIVERSE = TotalSchedulerJobsRunning < 10
  \end{verbatim}
  \normalsize
  
  
\label{param:MaxJobsRunning}
\item[\Macro{MAX\_JOBS\_RUNNING}]
  An integer representing a limit on the number of processes
  spawned by a given \Condor{schedd} daemon,
  for all job universes except the grid universe. 
  The number of processes limit includes \Condor{shadow} processes,
  scheduler universe processes, including \Condor{dagman}, and
  local universe \Condor{starter} processes.
  Limiting the number of running scheduler and local universe
  jobs below the upper limit set by \MacroNI{MAX\_JOBS\_RUNNING} is best
  done using \MacroNI{START\_LOCAL\_UNIVERSE} and
  \MacroNI{START\_SCHEDULER\_UNIVERSE}.
  The actual number of allowed \Condor{shadow} daemons may be reduced,
  if the amount of memory defined by \MacroNI{RESERVED\_SWAP} limits the
  number of \Condor{shadow} daemons.
  A value for \MacroNI{MAX\_JOBS\_RUNNING} that is less than or equal to 0
  prevents any new job from starting.  Changing this setting to be below
  the current number of jobs that are running will cause running jobs to
  be aborted until the number running is within the limit.

  Like all integer configuration variables, \MacroNI{MAX\_JOBS\_RUNNING}
  may be a ClassAd expression that evaluates to an integer, and which
  refers to constants either directly or via macro substitution.
  The default value is an expression that depends on the total amount
  of memory and the operating system.  The default
  expression requires 1MByte of RAM per running job on the submit machine.
  In some environments and configurations, this is overly
  generous and can be cut by as much as 50\%.
  On Windows platforms, the number of running jobs is still capped at 200.
  A 64-bit version of Windows  is recommended in order to raise the value
  above the default.
  Under Unix, the maximum default is now 10,000.  To scale higher, we
  recommend that the system ephemeral port range is extended
  such that there are at least 2.1 ports per running job.

  Here are example configurations:

\footnotesize
\begin{verbatim}
## Example 1:
MAX_JOBS_RUNNING	= 10000

## Example 2:
## This is more complicated, but it produces the same limit as the default.
## First define some expressions to use in our calculation.
## Assume we can use up to 80% of memory and estimate shadow private data
## size of 800k.
MAX_SHADOWS_MEM	= ceiling($(DETECTED_MEMORY)*0.8*1024/800)
## Assume we can use ~21,000 ephemeral ports (avg ~2.1 per shadow).
## Under Linux, the range is set in /proc/sys/net/ipv4/ip_local_port_range.
MAX_SHADOWS_PORTS	= 10000
## Under windows, things are much less scalable, currently.
## Note that this can probably be safely increased a bit under 64-bit windows.
MAX_SHADOWS_OPSYS	= ifThenElse(regexp("WIN.*","$(OPSYS)"),200,100000)
## Now build up the expression for MAX_JOBS_RUNNING.  This is complicated
## due to lack of a min() function.
MAX_JOBS_RUNNING	= $(MAX_SHADOWS_MEM)
MAX_JOBS_RUNNING	= \
  ifThenElse( $(MAX_SHADOWS_PORTS) < $(MAX_JOBS_RUNNING), \
              $(MAX_SHADOWS_PORTS), \
              $(MAX_JOBS_RUNNING) )
MAX_JOBS_RUNNING	= \
  ifThenElse( $(MAX_SHADOWS_OPSYS) < $(MAX_JOBS_RUNNING), \
              $(MAX_SHADOWS_OPSYS), \
              $(MAX_JOBS_RUNNING) )
\end{verbatim}
\normalsize

\label{param:MaxJobsSubmitted}
\item[\Macro{MAX\_JOBS\_SUBMITTED}]
  This integer value limits the number of jobs permitted in 
  a \Condor{schedd} daemon's queue. Submission of a new cluster
  of jobs fails, if the total number of jobs would exceed this limit. 
  The default value for this variable is the largest positive
  integer value.

\label{param:MaxShadowExceptions}
\item[\Macro{MAX\_SHADOW\_EXCEPTIONS}]
  This macro controls the maximum
  number of times that \Condor{shadow} processes can have a fatal
  error (exception) before the \Condor{schedd} will relinquish
  the match associated with the dying shadow.  Defaults to 5.

\label{param:MaxPendingStartdContacts}
\item[\Macro{MAX\_PENDING\_STARTD\_CONTACTS}]
  An integer value that limits
  the number of simultaneous connection attempts by the \Condor{schedd}
  when it is requesting claims from one or more \Condor{startd} daemons.
  The intention is to protect the \Condor{schedd} from being overloaded
  by authentication operations.  The default value is 0.
  The special value 0 indicates no limit.

\label{param:MaxConcurrentDownloads}
\item[\Macro{MAX\_CONCURRENT\_DOWNLOADS}]
  This specifies the maximum
  number of simultaneous transfers of output files from execute
  machines to the submit machine.  The limit applies to all jobs
  submitted from the same \Condor{schedd}.  The default is 10.  A
  setting of 0 means unlimited transfers.  This limit currently does
  not apply to grid universe jobs or standard universe jobs, and it
  also does not apply to streaming output files.  When the limit is
  reached, additional transfers will queue up and wait before
  proceeding.

\label{param:MaxConcurrentUploads}
\item[\Macro{MAX\_CONCURRENT\_UPLOADS}]
  This specifies the maximum
  number of simultaneous transfers of input files from the submit
  machine to execute machines.  The limit applies to all jobs
  submitted from the same \Condor{schedd}.  The default is 10.  A
  setting of 0 means unlimited transfers.  This limit currently does
  not apply to grid universe jobs or standard universe jobs.  When the
  limit is reached, additional transfers will queue up and wait before
  proceeding.

\label{param:TransferQueueNameExpr}
\item[\Macro{TRANSFER\_QUEUE\_USER\_EXPR}]
  This rarely configured expression specifies the username to be used
  for scheduling purposes in the file transfer queue.  The scheduler
  attempts to give equal weight to each user when there are multiple
  jobs waiting to transfer files within the limits set by
  \Macro{MAX\_CONCURRENT\_UPLOADS} and/or
  \Macro{MAX\_CONCURRENT\_DOWNLOADS}.  When choosing a new job to
  allow to transfer, the first job belonging to the transfer queue
  user who has least number of active transfers will be selected.
  In case of a tie, the user who has least recently been given an
  opportunity to start a transfer will be selected.  By default, a
  transfer queue user is identified as the job owner.  A different
  user name may be specified by configuring
  \MacroNI{TRANSFER\_QUEUE\_USER\_EXPR} to a string expression that is
  evaluated in the context of the job ad.  For example, if this
  expression were set to a name that is the same for all jobs, file
  transfers would be scheduled in first-in-first-out order rather than
  equal share order.  Note that the string produced by this expression
  is used as a prefix in the ClassAd attributes for per-user file
  transfer I/O statistics that are published in the \Condor{schedd}
  ClassAd.

\label{param:MaxTransferInputMB}
\item[\Macro{MAX\_TRANSFER\_INPUT\_MB}]
  This integer expression specifies the maximum allowed total size in
  Mbytes of the input files that are transferred for a job.  This
  expression does \emph{not} apply to grid universe, standard universe, or
  files transferred via file transfer plug-ins.  The expression may
  refer to attributes of the job.  
  The special value \Expr{-1} indicates no limit.
  The default value is -1.
  The job may override the system setting
  by specifying its own limit using the \Attr{MaxTransferInputMB}
  attribute.  
  If the observed size of all input files at submit time
  is larger than the limit, the job will be immediately placed on hold
  with a \Attr{HoldReasonCode} value of 32.
  If the job passes this initial test, but the
  size of the input files increases or the limit decreases so that the
  limit is violated, the job will be placed on hold at the time when
  the file transfer is attempted.

\label{param:MaxTransferOutputMB}
\item[\Macro{MAX\_TRANSFER\_OUTPUT\_MB}]
  This integer expression specifies the maximum allowed total size in
  Mbytes of the output files that are transferred for a job.  This
  expression does \emph{not} apply to grid universe, standard universe, or
  files transferred via file transfer plug-ins.  
  The expression may refer to attributes of the job.
  The special value \Expr{-1} indicates no limit.  
  The default value is -1.
  The job may override the system setting
  by specifying its own limit using the \Attr{MaxTransferOutputMB}
  attribute.  If the total size of the job's output files to be
  transferred is larger than the limit, the job will be placed on hold
  with a \Attr{HoldReasonCode} value of 33.
  The output will be transferred up to the point
  when the limit is hit, so some files may be fully transferred, some
  partially, and some not at all.

\label{param:TransferIoReportInterval}
\item[\Macro{TRANSFER\_IO\_REPORT\_INTERVAL}]
  The sampling interval in seconds for collecting I/O statistics for
  file transfer.  The default is 10 seconds.  To provide sufficient
  resolution, the sampling interval should be small compared to the
  smallest time span that is configured in
  \Macro{TRANSFER\_IO\_REPORT\_TIMESPANS}.  The shorter the sampling
  interval, the more overhead of data collection, which may slow down
  the \Condor{schedd}.  See
  page~\pageref{sec:FT-Scheduler-ClassAd-Attributes} for a description
  of the published attributes.

\label{param:TransferIoReportTimespans}
\item[\Macro{TRANSFER\_IO\_REPORT\_TIMESPANS}]
  A string that specifies a list of time spans over which I/O
  statistics are reported, using exponential moving averages (like the
  1m, 5m, and 15m load averages in Unix).  Each entry in the list
  consists of a label followed by a colon followed by the number of
  seconds over which the named time span should extend.  The default is
  \verb|1m:60 5m:300 1h:3600 1d:86400|.  To provide sufficient
  resolution, the smallest reported time span should be large compared
  to the sampling interval, which is configured by
  \Macro{TRANSFER\_IO\_REPORT\_INTERVAL}.  See
  page~\pageref{sec:FT-Scheduler-ClassAd-Attributes} for a description
  of the published attributes.

\label{param:ScheddQueryWorkers}
\item[\Macro{SCHEDD\_QUERY\_WORKERS}]
  This specifies the maximum number of concurrent sub-processes that
  the \Condor{schedd} will spawn to handle queries.  The setting is
  ignored in Windows.  In Unix, the default is 3.  If the limit is
  reached, the next query will be handled in the \Condor{schedd}'s main
  process.

\label{param:ScheddInterval}
\item[\Macro{SCHEDD\_INTERVAL}]
  This macro determines the maximum interval for both how often the
  \Condor{schedd} sends a ClassAd update to the \Condor{collector} and
  how often the \Condor{schedd} daemon evaluates jobs.  It is defined
  in terms of seconds and defaults to 300 (every 5 minutes).

\label{param:WindowedStatWidth}
\item[\Macro{WINDOWED\_STAT\_WIDTH}]
  The number of seconds that forms a time window within which performance
  statistics of the \Condor{schedd} daemon are calculated.
  Defaults to 300 seconds.

\label{param:ScheddIntervalTimeslice}
\item[\Macro{SCHEDD\_INTERVAL\_TIMESLICE}]
  The bookkeeping done by the
  \Condor{schedd} takes more time when there are large numbers of jobs
  in the job queue.  However, when it is not too expensive to do this
  bookkeeping, it is best to keep the collector up to date with the
  latest state of the job queue.  Therefore, this macro is used to
  adjust the bookkeeping interval so that it is done more frequently
  when the cost of doing so is relatively small, and less frequently
  when the cost is high.  The default is 0.05, which means the schedd
  will adapt its bookkeeping interval to consume no more than 5\% of the
  total time available to the schedd.  The lower bound is configured by
  \Macro{SCHEDD\_MIN\_INTERVAL} (default 5 seconds), and the upper bound
  is configured by \Macro{SCHEDD\_INTERVAL} (default 300 seconds).


% \label{param:RealTimeJobSuspendUpdates}
% \item[\Macro{REAL\_TIME\_JOB\_SUSPEND\_UPDATES}] 
%  If set to \Expr{True},
%  then the \Condor{shadow} will immediately update the
%  \Condor{schedd} upon the suspension or resumption of a job.
%  This allows \Condor{q} to show a
%  job in a suspended state in its default output.
%  In the \Expr{ST} column, there will be an \Expr{S} instead of an
%  \Expr{R} when the job is running, but suspended.

%  This attribute's real time connotation is currently applied only
%  to jobs in the vanilla, standard, and java universes.
%  Other universes may display a suspension state where applicable,
%  but the information may be stale.

%  The default value and when not present in the configuration
%  file is \Expr{False}.
  
\label{param:JobStartCount}
\item[\Macro{JOB\_START\_COUNT}]
  This macro works together with the \Macro{JOB\_START\_DELAY} macro to
  throttle job starts.  The default and minimum values for this
  integer configuration variable are both 1.

\label{param:JobStartDelay}
\item[\Macro{JOB\_START\_DELAY}]
  This integer-valued macro works together with the
  \Macro{JOB\_START\_COUNT} macro
  to throttle job starts.  The  \Condor{schedd} daemon starts
  \MacroUNI{JOB\_START\_COUNT} jobs at a time, then delays for
  \MacroUNI{JOB\_START\_DELAY} seconds before starting the next set of jobs.
  This delay prevents a sudden, large load on resources required by
  the jobs during their start up phase.
  The resulting job start rate
  averages as fast as
  (\MacroUNI{JOB\_START\_COUNT}/\MacroUNI{JOB\_START\_DELAY}) jobs/second.
  This setting is defined in terms of seconds and defaults to 0, which means
  jobs will be started as fast as possible.  If you wish to throttle
  the rate of specific types of jobs, you can use the job attribute
  \AdAttr{NextJobStartDelay}.

\label{param:MaxNextJobStartDelay}
\item[\Macro{MAX\_NEXT\_JOB\_START\_DELAY}]
  An integer number of seconds representing the maximum allowed value
  of the job ClassAd attribute \AdAttr{NextJobStartDelay}.  It defaults to 600,
  which is 10 minutes.

\label{param:JobStopCount}
\item[\Macro{JOB\_STOP\_COUNT}]
  An integer value representing the number of jobs operated on at one time
  by the \Condor{schedd} daemon, when throttling the rate at which jobs
  are stopped via \Condor{rm}, \Condor{hold}, or \Condor{vacate\_job}.  
  The default and minimum values are both 1.
  This variable is ignored for grid and scheduler universe jobs.

\label{param:JobStopDelay}
\item[\Macro{JOB\_STOP\_DELAY}]
  An integer value representing the number of seconds delay utilized by
  the \Condor{schedd} daemon, when throttling the rate at which jobs
  are stopped via \Condor{rm}, \Condor{hold}, or \Condor{vacate\_job}.  
  The  \Condor{schedd} daemon stops
  \MacroUNI{JOB\_STOP\_COUNT} jobs at a time, then delays for
  \MacroUNI{JOB\_STOP\_DELAY} seconds before stopping the next set of jobs.
  This delay prevents a sudden, large load on resources required by
  the jobs when they are terminating.
  The resulting job stop rate averages as fast as
  \Expr{JOB\_STOP\_COUNT/JOB\_STOP\_DELAY} jobs per second.
  This configuration variable is also used during the graceful shutdown of the
  \Condor{schedd} daemon.
  During graceful shutdown, this macro determines the wait time in
  between requesting each \Condor{shadow} daemon to gracefully shut down.  
  The default value is 0, which means jobs will be stopped as fast as possible.
  This variable is ignored for grid and scheduler universe jobs.

\label{param:JobIsFinishedInterval}
\item[\Macro{JOB\_IS\_FINISHED\_INTERVAL}]
  The \Condor{schedd} maintains a list of jobs that are ready to permanently
  leave the job queue, e.g. they have completed or been removed.  This
  integer-valued macro specifies a delay in seconds to place between the
  taking jobs permanently out of the queue.  The default value is 0, which
  tells the \Condor{schedd} to not impose any delay.  
  
\label{param:AliveInterval}
\item[\Macro{ALIVE\_INTERVAL}]
  An initial value for an integer number of seconds defining
  how often the \Condor{schedd} sends a UDP keep
  alive message to any \Condor{startd} it has claimed.
  When the \Condor{schedd} claims a \Condor{startd}, 
  the \Condor{schedd} tells the \Condor{startd} how often it is
  going to send these messages.
  The utilized interval for sending keep alive messages is the smallest of
  the two values \MacroNI{ALIVE\_INTERVAL} and the expression
  \Expr{JobLeaseDuration/3}, formed with the job ClassAd attribute
  \Attr{JobLeaseDuration}.
  The value of the interval is further constrained by the floor value 
  of 10 seconds.
  If the \Condor{startd} does not receive any of these keep alive messages
  during a certain period of time (defined via
  \Macro{MAX\_CLAIM\_ALIVES\_MISSED}, described on
  page~\pageref{param:MaxClaimAlivesMissed})
  the \Condor{startd} releases the claim, and the \Condor{schedd} no longer pays for
  the resource (in terms of user priority in the system).
  The macro is defined in terms of seconds and defaults to 300, which is
  5 minutes.

\label{param:StartdSendsAlives}
\item[\Macro{STARTD\_SENDS\_ALIVES}]
  A boolean value that defaults to \Expr{True},
  causing keep alive messages to be sent from the \Condor{startd} to the
  \Condor{schedd} by TCP during a claim.
  When \Expr{False}, the \Condor{schedd} daemon sends keep alive signals
  to the \Condor{startd}, reversing the direction.
  If both \Condor{startd} and \Condor{schedd} daemons are HTCondor version 7.5.4
  or more recent, this variable is only used by the \Condor{schedd} daemon.
  For earlier HTCondor versions, the variable must be set to the same value,
  and it must be set for both daemons.

\label{param:RequestClaimTimeout}
\item[\Macro{REQUEST\_CLAIM\_TIMEOUT}]
  This macro sets the time (in
  seconds) that the \Condor{schedd} will wait for a claim to be granted by the
  \Condor{startd}.  The default is 30 minutes.  This is only likely to matter
  if \Macro{NEGOTIATOR\_CONSIDER\_EARLY\_PREEMPTION} is \Expr{True}, and the
  \Condor{startd} has an existing claim, and it takes a long time for the
  existing claim to be preempted due to \Expr{MaxJobRetirementTime}.
  Once a request times out, the \Condor{schedd} will simply begin the process
  of finding a machine for the job all over again.

  Normally, it is not a good idea to set this to be very small (e.g. a
  few minutes).  Doing so can lead to failure to preempt, because the
  preempting job will spend a significant fraction of its time waiting
  to be re-matched.  During that time, it would miss out on any
  opportunity to run if the job it is trying to preempt gets out of
  the way.

\label{param:ShadowSizeEstimate}
\item[\Macro{SHADOW\_SIZE\_ESTIMATE}]
  The estimated private virtual memory size of each
  \Condor{shadow} process in Kbytes.
  This value is only used if \MacroNI{RESERVED\_SWAP} is non-zero.
  The default value is 800.

\label{param:ShadowReniceIncrement}
\item[\Macro{SHADOW\_RENICE\_INCREMENT}]
  When the \Condor{schedd} spawns a new
  \Condor{shadow}, it can do so with a \Term{nice-level}.  A
  nice-level is a Unix mechanism that allows users to assign their own
  processes a lower priority so that the processes run with less
  priority than other tasks on the machine.  The value can be any
  integer between 0 and 19, with a value of 19 being the lowest
  priority.  It defaults to 0.

\label{param:SchedUnivReniceIncrement}
\item[\Macro{SCHED\_UNIV\_RENICE\_INCREMENT}]
  Analogous to \MacroNI{JOB\_RENICE\_INCREMENT} and
  \MacroNI{SHADOW\_RENICE\_INCREMENT}, scheduler universe jobs can be
  given a nice-level.  The value can be any integer between 0 and 19,
  with a value of 19 being the lowest priority.  It defaults to 0.

\label{param:QueueCleanInterval}
\item[\Macro{QUEUE\_CLEAN\_INTERVAL}]
  The \Condor{schedd} maintains the job queue on a given machine.  It does so
  in a persistent way such that if the \Condor{schedd} crashes, it can recover
  a valid state of the job queue.  The mechanism it uses is a
  transaction-based log file (the \File{job\_queue.log} file,
  not the \File{SchedLog} file).  This file contains an initial
  state of the job queue, and a series of transactions that were
  performed on the queue (such as new jobs submitted, jobs completing,
  and checkpointing).  Periodically, the \Condor{schedd} will go through
  this log, truncate all the transactions and create a new file with
  containing only the new initial state of the log.
  This is a somewhat expensive operation,
  but it speeds up when the \Condor{schedd} restarts since there are
  fewer transactions it has to play to figure out what state the job
  queue is really in.  This macro determines how often the \Condor{schedd}
  should rework this queue to cleaning it up.  It is defined in terms of
  seconds and defaults to 86400 (once a day). 
  
\label{param:WallClockCkptInterval}
\item[\Macro{WALL\_CLOCK\_CKPT\_INTERVAL}]
  The job queue contains a counter for each job's ``wall clock'' run
  time, i.e., how long each job has executed so far.  This counter is
  displayed by \Condor{q}.  The counter is updated when the job is
  evicted or when the job completes.  When the \Condor{schedd} crashes, the run
  time for jobs that are currently running will not be added to the
  counter (and so, the run time counter may become smaller than the
  CPU time counter).  The \Condor{schedd} saves run time ``checkpoints''
  periodically for running jobs so if the \Condor{schedd} crashes, only run
  time since the last checkpoint is lost.  This macro controls how
  often the \Condor{schedd} saves run time checkpoints.  It is defined in terms
  of seconds and defaults to 3600 (one hour).  A value of 0 will
  disable wall clock checkpoints.

\label{param:QueueAllUsersTrusted}
\item[\Macro{QUEUE\_ALL\_USERS\_TRUSTED}]
  Defaults to False. If set to True, then unauthenticated users are allowed
  to write to the queue, and also we always trust whatever the \Attr{Owner}
  value is set to be by the client in the job ad. This was added so users
  can continue to use the SOAP web-services interface over HTTP (w/o
  authenticating) to submit jobs in a secure, controlled environment -- for
  instance, in a portal setting.
     
\label{param:QueueSuperUsers}
\item[\Macro{QUEUE\_SUPER\_USERS}]
  A comma and/or space separated list of user names on a given machine that
  are given \Term{super-user access} to the job queue, meaning that they can
  modify or delete the job ClassAds of other users.  When not on this list,
  users can only modify or delete their own ClassAds from the job queue.
  Whatever user name corresponds with the UID that HTCondor is running as --
  usually user \Login{condor} --
  will automatically be included in this list,
  because that is needed for HTCondor's proper functioning.
  See section~\ref{sec:uids} on UIDs in HTCondor for more details on this.
  By default, the Unix user \Login{root} and the Windows user 
  \Login{administrator} are given the ability to remove other user's jobs,
  in addition to user \Login{condor}.

\label{param:QueueSuperUserMayImpersonate}
\item[\Macro{QUEUE\_SUPER\_USER\_MAY\_IMPERSONATE}]
  A regular expression that matches the user names 
  (that is, job owner names) 
  that the queue super user may impersonate when managing jobs.
  When not set, the default behavior is to allow impersonation of any
  user who has had a job in the queue during the life of the \Condor{schedd}.
  For proper functioning of the \Condor{shadow}, the \Condor{gridmanager}, and
  the \Condor{job\_router}, this expression, if set, must match the owner
  names of all jobs that these daemons will manage.  
  Note that a regular expression that matches only part of the user name 
  is still considered a match.  
  If acceptance of partial matches is not desired, 
  the regular expression should begin with \verb|^| and end with \verb|$|.

\label{param:SystemJobMachineAttrs}
\item[\Macro{SYSTEM\_JOB\_MACHINE\_ATTRS}]
  This macro specifies a space and/or comma separated list of
  machine attributes that should be recorded in the job ClassAd.  The
  default attributes are \Attr{Cpus} and \Attr{SlotWeight}.  When
  there are multiple run attempts, history of machine attributes from
  previous run attempts may be kept.  The number of run attempts to
  store is specified by the configuration variable
  \Macro{SYSTEM\_JOB\_MACHINE\_ATTRS\_HISTORY\_LENGTH}.  A machine
  attribute named \Attr{X} will be inserted into the job ClassAd as an
  attribute named \Attr{MachineAttrX0}.  The previous value of this
  attribute will be named \Attr{MachineAttrX1}, the previous to that
  will be named \Attr{MachineAttrX2}, and so on, up to the specified
  history length.  A history of length 1 means that only \Attr{MachineAttrX0}
  will be recorded.  Additional attributes to record may be specified on
  a per-job basis by using the \SubmitCmd{job\_machine\_attrs} submit
  file command.  The value recorded in the job ClassAd is the evaluation of
  the machine attribute in the context of the job ClassAd when 
  the \Condor{schedd}
  daemon initiates the start up of the job.  If the evaluation results in
  an \Expr{Undefined} or \Expr{Error} result, 
  the value recorded in the job ClassAd will be
  \Expr{Undefined} or \Expr{Error} respectively.

\label{param:SystemJobMachineAttrsHistoryLength}
\item[\Macro{SYSTEM\_JOB\_MACHINE\_ATTRS\_HISTORY\_LENGTH}]
  The integer number of run attempts to store in
  the job ClassAd when recording the values of machine attributes listed
  in \Macro{SYSTEM\_JOB\_MACHINE\_ATTRS}.  The default is 1.
  The history length may also be extended on a per-job
  basis by using the submit file command
  \SubmitCmd{job\_machine\_attrs\_history\_length}.  The larger of the
  system and per-job history lengths will be used.  A history length of 0
  disables recording of machine attributes.

\label{param:ScheddLock}
\item[\Macro{SCHEDD\_LOCK}]
  This macro specifies what lock file should be used for access to the
  \File{SchedLog} file.  It must be a separate file from the
  \File{SchedLog}, since the \File{SchedLog} may be rotated and
  synchronization across log file rotations
  is desired.
  This macro is defined relative to the \MacroUNI{LOCK} macro.

\label{param:ScheddName}
\item[\Macro{SCHEDD\_NAME}]
  Used to give an alternative value to the \Attr{Name} attribute 
  in the \Condor{schedd}'s ClassAd.

  See the description of \MacroNI{MASTER\_NAME} in
  section~\ref{param:MasterName} on page~\pageref{param:MasterName}
  for defaults and composition of valid HTCondor daemon names.
  Also, note that if the \MacroNI{MASTER\_NAME} setting is defined for
  the \Condor{master} that spawned a given \Condor{schedd}, that name
  will take precedence over whatever is defined in
  \MacroNI{SCHEDD\_NAME}. 

\label{param:ScheddAttrs}
\item[\Macro{SCHEDD\_ATTRS}]
  This macro is described in section~\ref{param:SubsysExprs} as
  \MacroNI{<SUBSYS>\_ATTRS}.

\label{param:ScheddDebug}
\item[\Macro{SCHEDD\_DEBUG}]
  This macro
  (and other settings related to debug logging in the \Condor{schedd}) is
  described in section~\ref{param:SubsysDebug} as
  \MacroNI{<SUBSYS>\_DEBUG}.

\label{param:ScheddAddressFile}
\item[\Macro{SCHEDD\_ADDRESS\_FILE}]
  This macro is described in
  section~\ref{param:SubsysAddressFile} as
  \MacroNI{<SUBSYS>\_ADDRESS\_FILE}. 

\label{param:ScheddExecute}
\item[\Macro{SCHEDD\_EXECUTE}]
  A directory to use as a temporary sandbox for local universe jobs.
  Defaults to \File{\MacroUNI{SPOOL}/execute}.

\label{param:FlockNegotiatorHosts} 
\item[\Macro{FLOCK\_NEGOTIATOR\_HOSTS}]
  Defines a comma and/or space separated list of \Condor{negotiator} host
  names for pools in which the \Condor{schedd} should attempt to run jobs.
  If not set,
  the \Condor{schedd} will query the \Condor{collector} daemons for 
  the addresses of the \Condor{negotiator} daemons.
  If set, then the \Condor{negotiator} daemons must be specified in order,
  corresponding to the list set by \MacroNI{FLOCK\_COLLECTOR\_HOSTS}.
  In the typical case, where each pool
  has the \Condor{collector} and \Condor{negotiator} running on the 
  same machine,
  \MacroUNI{FLOCK\_NEGOTIATOR\_HOSTS} should have the same definition as
  \MacroUNI{FLOCK\_COLLECTOR\_HOSTS}.  This configuration value is also
  typically used as a macro for adding the \Condor{negotiator} to the relevant
  authorization lists.

\label{param:FlockCollectorHosts}
\item[\Macro{FLOCK\_COLLECTOR\_HOSTS}]
  This macro defines a list of collector host names (not including the
  local \MacroUNI{COLLECTOR\_HOST} machine) for pools in which the
  \Condor{schedd} should attempt to run jobs.  Hosts in the list
  should be in order of preference.  The \Condor{schedd} will only
  send a request to a central manager in the list if the local pool
  and pools earlier in the list are not satisfying all the job
  requests.  \MacroUNI{HOSTALLOW\_NEGOTIATOR\_SCHEDD} (see
  section~\ref{param:HostAllow}) must also be configured to allow
  negotiators from all of the pools to contact the \Condor{schedd} at
  the \DCPerm{NEGOTIATOR} authorization level.  Similarly, the central
  managers of the remote pools must be configured to allow this
  \Condor{schedd} to join the pool (this requires \DCPerm{ADVERTISE\_SCHEDD}
  authorization level, which defaults to \DCPerm{WRITE}).

\label{param:FlockIncrement}
\item[\Macro{FLOCK\_INCREMENT}]
  This integer value controls how quickly flocking to various
  pools will occur.  It defaults to 1, meaning that pools will
  be considered for flocking slowly. 
  The first \Condor{collector} daemon listed in
  \Macro{FLOCK\_COLLECTOR\_HOSTS} will be considered for flocking, and then
  the second, and so on.  A larger value increases the number of 
  \Condor{collector} daemons to be considered for flocking.
  For example, a value of 2 will partition the
  \Macro{FLOCK\_COLLECTOR\_HOSTS} into sets of 2 \Condor{collector} daemons,
  and each set will be considered for flocking.

  \label{param:NegotiateAllJobsInCluster}
\item[\Macro{NEGOTIATE\_ALL\_JOBS\_IN\_CLUSTER}]
  If this macro is set to False (the default), when the \Condor{schedd} fails
  to start an idle job, it will not try to start any other
  idle jobs in the same cluster during that negotiation cycle.  This
  makes negotiation much more efficient for large job clusters.
  However, in some cases other jobs in the cluster can be started even
  though an earlier job can't.  For example, the jobs' requirements
  may differ, because of different disk space, memory, or
  operating system requirements.  Or, machines may be willing to run
  only some jobs in the cluster, because their requirements reference
  the jobs' virtual memory size or other attribute.  Setting this
  macro to True will force the \Condor{schedd} to try to start all idle jobs in
  each negotiation cycle.  This will make negotiation cycles last
  longer, but it will ensure that all jobs that can be started will be
  started.

\label{param:PeriodicExprInterval}
\item[\Macro{PERIODIC\_EXPR\_INTERVAL}]
  This macro determines the minimum period,
  in seconds, between evaluation of periodic job control expressions,
  such as periodic\_hold, periodic\_release, and periodic\_remove,
  given by the user in an HTCondor submit file. By default, this value is
  60 seconds.  A value of 0 prevents the \Condor{schedd} from
  performing the periodic evaluations.

\label{param:MaxPeriodicExprInterval}
\item[\Macro{MAX\_PERIODIC\_EXPR\_INTERVAL}]
  This macro determines the maximum period,
  in seconds, between evaluation of periodic job control expressions,
  such as periodic\_hold, periodic\_release, and periodic\_remove,
  given by the user in an HTCondor submit file. By default, this value is
  1200 seconds.  If HTCondor is behind on processing events, the actual
  period between evaluations may be higher than specified.

\label{param:PeriodicExprTimeslice}
\item[\Macro{PERIODIC\_EXPR\_TIMESLICE}]
  This macro is used to adapt the
  frequency with which the \Condor{schedd} evaluates periodic job
  control expressions.  When the job queue is very large, the cost of
  evaluating all of the ClassAds is high, so in order for the
  \Condor{schedd} to continue to perform well, it makes sense to
  evaluate these expressions less frequently.  The default time slice
  is 0.01, so the \Condor{schedd} will set the interval between
  evaluations so that it spends only 1\% of its time in this activity.
  The lower bound for the interval is configured by
  \Macro{PERIODIC\_EXPR\_INTERVAL} (default 60 seconds) and the
  upper bound is configured with \Macro{MAX\_PERIODIC\_EXPR\_INTERVAL}
  (default 1200 seconds).

\label{param:SystemPeriodicHold}
\item[\Macro{SYSTEM\_PERIODIC\_HOLD}]
  This expression behaves identically
  to the job expression \AdAttr{periodic\_hold}, but it is evaluated by
  the \Condor{schedd} daemon individually for each job in the queue.
  It defaults to \Expr{False}.
  When \Expr{True}, it causes the job to stop running and go on hold.
  Here is an
  example that puts jobs on hold if they have been restarted too many
  times, have an unreasonably large virtual memory \Attr{ImageSize}, or have
  unreasonably large disk usage for an invented environment.

\footnotesize
\begin{verbatim}
SYSTEM_PERIODIC_HOLD = \
  (JobStatus == 1 || JobStatus == 2) && \
  (JobRunCount > 10 || ImageSize > 3000000 || DiskUsage > 10000000)
\end{verbatim}
\normalsize

\label{param:SystemPeriodicHoldReason}
\item[\Macro{SYSTEM\_PERIODIC\_HOLD\_REASON}]
  This string expression is evaluated when the job is placed on hold
  due to \MacroNI{SYSTEM\_PERIODIC\_HOLD} evaluating to \Expr{True}.
  If it evaluates to a non-empty string, this value is used to set the
  job attribute \AdAttr{HoldReason}.
  Otherwise, a default description is used.

\label{param:SystemPeriodicHoldSubCode}
\item[\Macro{SYSTEM\_PERIODIC\_HOLD\_SUBCODE}]
  This integer expression is evaluated when the job is placed on hold
  due to \MacroNI{SYSTEM\_PERIODIC\_HOLD} evaluating to \Expr{True}.
  If it evaluates to a valid integer, this value is used to set the job
  attribute \Attr{HoldReasonSubCode}.  
  Otherwise, a default of 0 is used.
  The attribute \AdAttr{HoldReasonCode} is set to 26, which
  indicates that the job went on hold due to a system job policy expression.

\label{param:SystemPeriodicRelease}
\item[\Macro{SYSTEM\_PERIODIC\_RELEASE}]
  This expression behaves identically
  to a job's definition of a \SubmitCmd{periodic\_release} expression
  in a submit description file,
  but it is evaluated by
  the \Condor{schedd} daemon individually for each job in the queue.
  It defaults to \Expr{False}.
  When \Expr{True}, it causes a Held job to return to the Idle state.
  Here is an example
  that releases jobs from hold if they have tried to run less than 20
  times, have most recently been on hold for over 20 minutes, and have
  gone on hold due to \Expr{Connection timed out} when trying to execute
  the job, because the file system containing the job's executable is
  temporarily unavailable.

\footnotesize
\begin{verbatim}
SYSTEM_PERIODIC_RELEASE = \
  (JobRunCount < 20 && (CurrentTime - EnteredCurrentStatus) > 1200 ) &&  \
    (HoldReasonCode == 6 && HoldReasonSubCode == 110) 
\end{verbatim} 
\normalsize


  \label{param:SystemPeriodicRemove}
\item[\Macro{SYSTEM\_PERIODIC\_REMOVE}]
  This expression behaves identically
  to the job expression \AdAttr{periodic\_remove}, but it is evaluated for
  every job in the queue.  As it is in the configuration file, it is
  easy for an administrator to set a remove policy that applies to all jobs.
  It defaults to \Expr{False}.
  When \Expr{True}, it causes the job to be removed from the queue.
  Here is an example
  that removes jobs which have been on hold for 30 days:

\footnotesize
\begin{verbatim}
SYSTEM_PERIODIC_REMOVE = \
  (JobStatus == 5 && CurrentTime - EnteredCurrentStatus > 3600*24*30)
\end{verbatim}
\normalsize

\label{param:ScheddAssumeNegotiatorGone}
\item[\Macro{SCHEDD\_ASSUME\_NEGOTIATOR\_GONE}]
  This macro determines the period,
  in seconds, that the \Condor{schedd} will wait for the \Condor{negotiator} to
  initiate a negotiation cycle before the schedd will simply try to claim
  any local \Condor{startd}.  This allows for a machine that is acting as
  both a submit and execute node to run jobs locally if it cannot
  communicate with the central manager. The default value, if not
  specified, is 1200 (20 minutes).

\label{param:GracefullyRemoveJobs}
\item[\Macro{GRACEFULLY\_REMOVE\_JOBS}]
  A boolean value that causes jobs to be gracefully removed when the
  default value of \Expr{True}.
  A submit description file command \SubmitCmd{want\_graceful\_removal}
  overrides the value set for this configuration variable.

\label{param:ScheddRoundAttr}
\item[\Macro{SCHEDD\_ROUND\_ATTR\_<xxxx>}]
  This is used to round off attributes in
  the job ClassAd so that similar jobs may be grouped together for
  negotiation purposes.  There are two cases.  One is that a
  percentage such as 25\% is specified.  In this case, the value of
  the attribute named \verb@<xxxx>\@ in the job ClassAd will be
  rounded up to the next multiple of the specified percentage of the
  values order of magnitude.  For example, a setting of 25\% will
  cause a value near 100 to be rounded up to the next multiple of 25
  and a value near 1000 will be rounded up to the next multiple of
  250.  The other case is that an integer, such as 4, is specified
  instead of a percentage.  In this case, the job attribute is rounded
  up to the specified number of decimal places.
  Replace \verb@<xxxx>@ with the name of the attribute to round, and set this
  macro equal to the number of decimal places to round up.  For example, to
  round the value of job ClassAd attribute \Attr{foo}  up to the nearest
  100, set 
\begin{verbatim}
        SCHEDD_ROUND_ATTR_foo = 2
\end{verbatim}
  When the schedd rounds up an attribute value, it will save the raw 
  (un-rounded) actual value in an attribute with the same name appended
  with ``\_RAW".  So in the above example, the raw value will be stored
  in attribute \Attr{foo\_RAW} in the job ClassAd.
  The following are set by default:
\begin{verbatim}
        SCHEDD_ROUND_ATTR_ImageSize = 25%
        SCHEDD_ROUND_ATTR_ResidentSetSize = 25%
        SCHEDD_ROUND_ATTR_ProportionalSetSizeKb = 25%
        SCHEDD_ROUND_ATTR_ImageSize = 25%
        SCHEDD_ROUND_ATTR_ExecutableSize = 25%
        SCHEDD_ROUND_ATTR_DiskUsage = 25%
        SCHEDD_ROUND_ATTR_NumCkpts = 4
\end{verbatim}
  Thus, an ImageSize near 100MB will be rounded up to the next
  multiple of 25MB.  If your batch slots have less
  memory or disk than the rounded values, it may be necessary to
  reduce the amount of rounding, because the job requirements
  will not be met.

\label{param:ScheddBackupSpool}
\item[\Macro{SCHEDD\_BACKUP\_SPOOL}]
  A boolean value that, when \Expr{True}, causes the
  \Condor{schedd} to make a backup of the job queue as it starts.
  When \Expr{True}, the \Condor{schedd} creates a host-specific
  backup of the current spool file to the spool directory.  This
  backup file will be overwritten each time the \Condor{schedd} starts.
  Defaults to \Expr{False}.

\label{param:ScheddPreemptionRequirements}
\item[\Macro{SCHEDD\_PREEMPTION\_REQUIREMENTS}]
  This boolean expression is
  utilized only for machines allocated by a dedicated scheduler.
  When \Expr{True}, a machine becomes a candidate for job preemption.
  This configuration variable has no default;
  when not defined, preemption will never be considered.

\label{param:ScheddPreemptionRank}
\item[\Macro{SCHEDD\_PREEMPTION\_RANK}]
  This floating point value is
  utilized only for machines allocated by a dedicated scheduler.
  It is evaluated in context of a job ClassAd,
  and it represents a machine's preference for running a job.
  This configuration variable has no default;
  when not defined, preemption will never be considered.

\label{param:ParallelSchedulingGroup}
\item[\Macro{ParallelSchedulingGroup}]
  For parallel jobs which must be assigned within a group
  of machines (and not cross group boundaries),
  this configuration variable is a string which 
  identifies a group of which this machine is a member. 
  Each machine within a group sets this configuration variable with 
  a string that identifies the group.

\label{param:PerJobHistoryDir}
\item[\Macro{PER\_JOB\_HISTORY\_DIR}]
  If set to a directory writable by the HTCondor user, when a job
  leaves the \Condor{schedd}'s queue, a copy of the job's ClassAd will
  be written in that directory.  The files are named \File{history},
  with the job's cluster and process number appended.  
  For example, job 35.2 will result in a file named \File{history.35.2}.
  HTCondor does not rotate or delete the files, so without an
  external entity to clean the directory, it can grow very large.
  This option defaults to being unset.  When not set, no
  files are written.

\label{param:DedicatedSchedulerUseFifo}
\item[\Macro{DEDICATED\_SCHEDULER\_USE\_FIFO}]
  When this parameter is set to true (the default), parallel 
  universe jobs will be scheduled in a first-in, first-out manner.
  When set to false, parallel jobs are scheduled using a
  best-fit algorithm. Using the best-fit algorithm is not recommended,
  as it can cause starvation.

\label{param:DedicatedSchedulerWaitForSpooler}
\item[\Macro{DEDICATED\_SCHEDULER\_WAIT\_FOR\_SPOOLER}]
  When this parameter is set to true, the dedicated scheduler will
schedule parallel jobs in very strict first-in, first-out manner.
When set to false, which is the default, parallel jobs that are being remotely
submitted to a scheduler, and are on hold, waiting for spooled input
files to arrive at the scheduler, will not block jobs that arrived later,
but whose inputs files have finished spooling.  When set to true,
jobs with larger cluster ids, but that are in the Idle state will not
be scheduled to run until all earlier jobs have finished spooling in
their input files and have been scheduled.

\label{param:ScheddSendVacateViaTcp}
\item[\Macro{SCHEDD\_SEND\_VACATE\_VIA\_TCP}]
  A boolean value that defaults to \Expr{False}.
  When \Expr{True}, the \Condor{schedd} daemon sends vacate signals via TCP,
  instead of the default UDP.

\label{param:ScheddClusterInitialValue}
\item[\Macro{SCHEDD\_CLUSTER\_INITIAL\_VALUE}]
  An integer that specifies the initial cluster number value to use within a
  job id when a job is first submitted.
  If the job cluster number reaches the value set by 
  \MacroNI{SCHEDD\_CLUSTER\_MAXIMUM\_VALUE} and wraps,
  it will be re-set to the value given by this variable.
  The default value is 1.

\label{param:ScheddClusterIncrementValue}
\item[\Macro{SCHEDD\_CLUSTER\_INCREMENT\_VALUE}]
  A positive integer that defaults to 1, representing a stride used
  for the assignment of cluster numbers within a job id.
  When a job is submitted, the job will be assigned a job id.  The cluster
  number of the job id will be equal to the previous cluster number used
  plus the value of this variable.

\label{param:ScheddClusterMaximumValue}
\item[\Macro{SCHEDD\_CLUSTER\_MAXIMUM\_VALUE}]
  An integer that specifies an upper bound on assigned job cluster id values.
  For value $M$, the maximum job cluster id assigned to 
  any job will be $M-1$. When the maximum id is reached, cluster ids will 
  continue assignment using \MacroNI{SCHEDD\_CLUSTER\_INITIAL\_VALUE}. The 
  default value of this variable is zero,
  which represents the behavior of having no maximum cluster id value. 

  Note that HTCondor does not check for nor take responsibility for duplicate
  cluster ids for queued jobs. 
  If \MacroNI{SCHEDD\_CLUSTER\_MAXIMUM\_VALUE} is set to a non-zero value,
  the system administrator is
  responsible for ensuring that older jobs do not stay in the queue long
  enough for cluster ids of new jobs to wrap around and reuse the same id.
  With a low enough value, it is possible for jobs to be erroneously assigned 
  duplicate cluster ids, which will result in a corrupt job queue.

\label{param:GridManagerSelectionExpr}
\item[\Macro{GRIDMANAGER\_SELECTION\_EXPR}]
  By default, the \Condor{schedd} daemon will start a new 
  \Condor{gridmanager} process for each
  discrete user that submits a grid universe job,
  that is, for each discrete value of job attribute \Attr{Owner} across
  all grid universe job ClassAds.
  For additional isolation and/or scalability of grid job management,
  additional \Condor{gridmanager} processes can be spawned to share the load;
  to do so, set this variable to be a ClassAd expression.
  The result of the evaluation of this expression in the
  context of a grid universe job ClassAd will be treated as a hash value.
  All jobs that hash to the same value via this expression will go to the
  same \Condor{gridmanager}.
  For instance, to spawn a separate \Condor{gridmanager} process to
  manage each unique remote site, the following expression works:
\begin{verbatim}
  GRIDMANAGER_SELECTION_EXPR = GridResource 
\end{verbatim}

\label{param:CkptServerClientTimeout}
\item[\Macro{CKPT\_SERVER\_CLIENT\_TIMEOUT}]
  An integer which specifies how long in seconds the \Condor{schedd} is
  willing to wait for a response from a checkpoint server before declaring
  the checkpoint server down. The value of 0 makes the schedd block for
  the operating system configured time (which could be a very long time)
  before the \Syscall{connect} returns on its own with a connection timeout.
  The default value is 20.

\label{param:CkptServerClientTimeoutRetry}
\item[\Macro{CKPT\_SERVER\_CLIENT\_TIMEOUT\_RETRY}]
  An integer which specifies how long in seconds the \Condor{schedd} will
  ignore a checkpoint server that is deemed to be down. After this time
  elapses, the \Condor{schedd} will try again in talking to the checkpoint
  server.
  The default is 1200.

\label{param:ScheddJobQueueLogFlushDelay}
\item[\Macro{SCHEDD\_JOB\_QUEUE\_LOG\_FLUSH\_DELAY}]
  An integer which specifies an upper bound in seconds on how long it
  takes for changes to the job ClassAd to be visible to the HTCondor Job Router.
  The default is 5 seconds.

\label{param:RotateHistoryDaily}
\item[\Macro{ROTATE\_HISTORY\_DAILY}]
  A boolean value that defaults to \Expr{False}.
  When \Expr{True}, the history file will be rotated daily,
  in addition to the rotations that occur due to the definition of  
  \MacroNI{MAX\_HISTORY\_LOG} that rotate due to size.

\label{param:RotateHistoryMonthly}
\item[\Macro{ROTATE\_HISTORY\_MONTHLY}]
  A boolean value that defaults to \Expr{False}.
  When \Expr{True}, the history file will be rotated monthly,
  in addition to the rotations that occur due to the definition of  
  \MacroNI{MAX\_HISTORY\_LOG} that rotate due to size.


\label{param:ScheddCollectStatsForName}
\item[\Macro{SCHEDD\_COLLECT\_STATS\_FOR\_<Name>}]
  A boolean expression that when \Expr{True} creates a set of \Condor{schedd}
  ClassAd attributes of statistics collected for a particular set.
  These attributes are named using the prefix of \Expr{<Name>}.
  The set includes each entity for which this expression is \Expr{True}.
  As an example, assume that \Condor{schedd} statistics attributes are to
  be created for only user Einstein's jobs.
  Defining
\begin{verbatim}
  SCHEDD_COLLECT_STATS_FOR_Einstein = (Owner=="einstein")
\end{verbatim}
  causes the creation of the set of statistics attributes with names such as
  \Attr{EinsteinJobsCompleted} and \Attr{EinsteinJobsCoredumped}.

\label{param:ScheddCollectStatsByName}
\item[\Macro{SCHEDD\_COLLECT\_STATS\_BY\_<Name>}]
  Defines a string expression.  The evaluated string is used in the naming of 
  a set of \Condor{schedd} statistics ClassAd attributes.
  The naming begins with \Expr{<Name>}, an underscore character, 
  and the evaluated string.
  Each character not permitted in an attribute name will be converted
  to the underscore character.
  For example,
\footnotesize
\begin{verbatim}
  SCHEDD_COLLECT_STATS_BY_Host = splitSlotName(RemoteHost)[1]
\end{verbatim}
\normalsize
  a set of statistics attributes will be created and kept. 
  If the string expression were to evaluate to \AdStr{storm.04.cs.wisc.edu},
  the names of two of these attributes will be
  \Attr{Host\_storm\_04\_cs\_wisc\_edu\_JobsCompleted} and 
  \Attr{Host\_storm\_04\_cs\_wisc\_edu\_JobsCoredumped}.
  
\label{param:ScheddExpireStatsByName}
\item[\Macro{SCHEDD\_EXPIRE\_STATS\_BY\_<Name>}]
  The number of seconds after which the \Condor{schedd} daemon will stop
  collecting and discard the statistics for a subset 
  identified by \Expr{<Name>},
  if no event has occurred to cause any counter or statistic for the subset
  to be updated. 
  If this variable is not defined for a particular \Expr{<Name>},
  then the default value will be \Expr{60*60*24*7}, which is one week's time.

\end{description}

%%%%%%%%%%%%%%%%%%%%%%%%%%%%%%%%%%%%%%%%%%%%%%%%%%%%%%%%%%%%%%%%%%%%%%%%%%%
\subsection{\label{sec:Shadow-Config-File-Entries}\condor{shadow}
Configuration File Entries}
%%%%%%%%%%%%%%%%%%%%%%%%%%%%%%%%%%%%%%%%%%%%%%%%%%%%%%%%%%%%%%%%%%%%%%%%%%%

\index{configuration!condor\_shadow configuration variables}
These settings affect the \Condor{shadow}.
\begin{description}

\label{param:ShadowLock}
\item[\Macro{SHADOW\_LOCK}]
  This macro specifies the lock file to be used for access to the
  \File{ShadowLog} file.  It must be a separate file from the
  \File{ShadowLog}, since the \File{ShadowLog} may be rotated 
  and you want to synchronize access across log file rotations.
  This macro is defined relative to the \MacroUNI{LOCK} macro.

\label{param:ShadowDebug}
\item[\Macro{SHADOW\_DEBUG}]
  This macro (and other settings related to debug logging in the shadow) is
  described in section~\ref{param:SubsysDebug} as
  \MacroNI{<SUBSYS>\_DEBUG}.

\label{param:ShadowQueueUpdateInterval}
\item[\Macro{SHADOW\_QUEUE\_UPDATE\_INTERVAL}]
  The amount of time (in seconds) between ClassAd updates that the
  \Condor{shadow} daemon sends to the \Condor{schedd} daemon.
  Defaults to 900 (15 minutes).

\label{param:ShadowLazyQueueUpdate}
\item[\Macro{SHADOW\_LAZY\_QUEUE\_UPDATE}]
  This boolean macro specifies if the \Condor{shadow} should
  immediately update the job queue for certain attributes (at this
  time, it only effects the \AdAttr{NumJobStarts} and
  \AdAttr{NumJobReconnects} counters) or if it should wait and only
  update the job queue on the next periodic update.
  There is a trade-off between performance and the semantics of these
  attributes, which is why the behavior is controlled by a
  configuration macro.
  If the \Condor{shadow} do not use a lazy update, and immediately
  ensures the changes to the job attributes are written to the job
  queue on disk, the semantics for the attributes are very solid
  (there's only a tiny chance that the counters will be out of sync
  with reality), but this introduces a potentially large performance
  and scalability problem for a busy \Condor{schedd}.
  If the \Condor{shadow} uses a lazy update, there is no additional cost
  to the \Condor{schedd}, but it means that \Condor{q} will not
  immediately see the changes to the job attributes, and if the
  \Condor{shadow} happens to crash or be killed during that time, the
  attributes are never incremented.
  Given that the most obvious usage of these counter attributes is for
  the periodic user policy expressions (which are evaluated directly
  by the \Condor{shadow} using its own copy of the job's ClassAd,
  which is immediately updated in either case), and since the
  additional cost for aggressive updates to a busy \Condor{schedd}
  could potentially cause major problems, the default is \Expr{True}
  to do lazy, periodic updates.

\label{param:ShadowWorklife}
\item[\Macro{SHADOW\_WORKLIFE}]
  The integer number of seconds after which the \Condor{shadow} will exit
  when the current job finishes, instead of fetching a new job to
  manage.  Having the \Condor{shadow} continue managing jobs helps
  reduce overhead and can allow the \Condor{schedd} to achieve higher
  job completion rates.  The default is 3600, one hour.  The value 0
  causes \Condor{shadow} to exit after running a single job.

\label{param:CompressPeriodicCkpt}
\item[\Macro{COMPRESS\_PERIODIC\_CKPT}]
  A boolean value that when \Expr{True}, directs the \Condor{shadow}
  to instruct applications to compress periodic checkpoints when possible.
  The default is \Expr{False}.

\label{param:CompressVacateCkpt}
\item[\Macro{COMPRESS\_VACATE\_CKPT}]
  A boolean value that when \Expr{True}, directs the \Condor{shadow}
  to instruct applications to compress vacate checkpoints when possible.
  The default is \Expr{False}.

\label{param:PeriodicMemorySync}
\item[\Macro{PERIODIC\_MEMORY\_SYNC}]
  This boolean value specifies whether the \Condor{shadow} should instruct
  applications to commit dirty memory pages to swap space during a
  periodic checkpoint.  The default is \Expr{False}.  This potentially
  reduces the number of dirty memory pages at vacate time, thereby
  reducing swapping activity on the remote machine.

\label{param:SlowCkptSpeed}
\item[\Macro{SLOW\_CKPT\_SPEED}]
  This macro specifies the speed at which vacate checkpoints should be
  written, in kilobytes per second.  If zero (the default), vacate
  checkpoints are written as fast as possible.  Writing vacate
  checkpoints slowly can avoid overwhelming the remote machine with
  swapping activity.

\label{param:ShadowJobCleanupRetryDelay}
\item[\Macro{SHADOW\_JOB\_CLEANUP\_RETRY\_DELAY}]
  This integer specifies the number of seconds to wait between tries
  to commit the final update to the job ClassAd in the \Condor{schedd}'s
  job queue.  The default is 30.

\label{param:ShadowMaxJobCleanupRetries}
\item[\Macro{SHADOW\_MAX\_JOB\_CLEANUP\_RETRIES}]
  This integer specifies the number of times to try committing
  the final update to the job ClassAd in the \Condor{schedd}'s
  job queue.  The default is 5.

\label{param:ShadowCheckproxyInterval}
\item[\Macro{SHADOW\_CHECKPROXY\_INTERVAL}]
  The number of seconds between tests to see if the job proxy has been
  updated or should be refreshed.  The default is 600 seconds (10 minutes).
  This variable's value should be small in comparison to the refresh interval
  required to keep delegated credentials from expiring 
  (configured via
  \Macro{DELEGATE\_JOB\_GSI\_CREDENTIALS\_REFRESH} and
  \Macro{DELEGATE\_JOB\_GSI\_CREDENTIALS\_LIFETIME}).  
  If this variable's value is too small, 
  proxy updates could happen very frequently, 
  potentially creating a lot of load on the submit machine.

\label{param:ShadowRunUnknownUserJobs}
\item[\Macro{SHADOW\_RUN\_UNKNOWN\_USER\_JOBS}]
  A boolean that defaults to \Expr{False}.
  When \Expr{True}, it allows the \Condor{shadow} daemon to run jobs 
  as user \Login{nobody} when remotely submitted and from
  users not in the local password file.
\end{description}

%%%%%%%%%%%%%%%%%%%%%%%%%%%%%%%%%%%%%%%%%%%%%%%%%%%%%%%%%%%%%%%%%%%%%%%%%%%
\subsection{\label{sec:Starter-Config-File-Entries}\condor{starter}
Configuration File Entries}
%%%%%%%%%%%%%%%%%%%%%%%%%%%%%%%%%%%%%%%%%%%%%%%%%%%%%%%%%%%%%%%%%%%%%%%%%%%

\index{configuration!condor\_starter configuration variables}
These settings affect the \Condor{starter}.
\begin{description}

\label{param:ExecTransferAttempts}
\item[\Macro{EXEC\_TRANSFER\_ATTEMPTS}]
  Sometimes due to a router misconfiguration, kernel bug, or other
  network problem, the transfer of the initial checkpoint from
  the submit machine to the execute machine will fail midway through.
  This parameter allows a retry of the transfer a certain number of times
  that must be equal to or greater than 1. If this parameter is not
  specified, or specified incorrectly, then it will default to three.
  If the transfer of the initial executable fails every attempt, then
  the job goes back into the idle state until the next renegotiation
  cycle.

  \Note: This parameter does not exist in the NT starter.

\label{param:JobReniceIncrement}
\item[\Macro{JOB\_RENICE\_INCREMENT}]
  When the \Condor{starter} spawns an HTCondor job, it can do so with a
  \Term{nice-level}.
  A nice-level is a
  Unix mechanism that allows users to assign their own processes a lower 
  priority, such that these processes do not interfere with interactive
  use of the machine.
  For machines with lots
  of real memory and swap space, such that the only scarce resource is CPU time,
  use this macro in conjunction with a policy that
  allows HTCondor to always start jobs on the machines. 
  HTCondor jobs would always run,
  but interactive response on the machines would never suffer.
  A user most likely will not notice HTCondor is
  running jobs.  See section~\ref{sec:Configuring-Policy} on
  Startd Policy Configuration for more details on setting up a
  policy for starting and stopping jobs on a given machine.

  The ClassAd expression is evaluated in the context of the job ad
  to an integer value, which is
  set by the \Condor{starter} daemon for each job just before the
  job runs.
  The range of allowable values are integers in the range of 0 to 19
  (inclusive),
  with a value of 19 being the lowest priority.  
  If the integer value is outside this range,
  then on a Unix machine, a value greater than 19 is auto-decreased to 19;
  a value less than 0 is treated as 0.
  For values outside this range, a Windows machine ignores the value
  and uses the default instead.
  The default value is 10, which maps to the idle priority class on
  a Windows machine.

\label{param:StarterLocalLogging}
\item[\Macro{STARTER\_LOCAL\_LOGGING}]
  This macro determines whether the
  starter should do local logging to its own log file, or send debug
  information back to the \Condor{shadow} where it will end up in the
  ShadowLog.  It defaults to \Expr{True}.

\label{param:StarterDebug}
\item[\Macro{STARTER\_DEBUG}]
  This setting (and other settings related to debug logging in the starter) is
  described above in section~\ref{param:SubsysDebug} as
  \MacroUNI{<SUBSYS>\_DEBUG}.

\label{param:StarterUpdateInterval}
\item[\Macro{STARTER\_UPDATE\_INTERVAL}]
  An integer value representing the number of seconds between 
  ClassAd updates that the \Condor{starter} daemon sends to the 
  \Condor{shadow} and \Condor{startd} daemons. 
  Defaults to 300 (5 minutes).

\label{param:StarterUpdateIntervalTimeslice}
\item[\Macro{STARTER\_UPDATE\_INTERVAL\_TIMESLICE}]
  A floating point value, specifying the highest fraction of time that the 
  \Condor{starter} daemon should spend collecting
  monitoring information about the job, such as disk usage.
  The default  value is 0.1.  
  If monitoring, such as checking disk usage takes a long time,
  the \Condor{starter} will monitor less frequently than specified by
  \MacroNI{STARTER\_UPDATE\_INTERVAL}.

\label{param:UserJobWrapper} 
\item[\Macro{USER\_JOB\_WRAPPER}]
  The full path and file name of an executable or script.
  If specified, HTCondor never directly executes a job, but instead
  invokes this executable,
  allowing an administrator to specify the executable (wrapper script) 
  that will handle the execution of all user jobs.  
  The command-line arguments passed to this program will include the
  full path to the actual user job which should be executed, followed by all
  the command-line parameters to pass to the user job.
  This wrapper script must ultimately replace its image with the user job;
  thus, it must \Procedure{exec} the user job, not \Procedure{fork} it.

  For Bourne type shells (\Prog{sh}, \Prog{bash}, \Prog{ksh}), 
  the last line should be:
\begin{verbatim}
        exec "$@"
\end{verbatim}
  For the C type shells (\Prog{csh}, \Prog{tcsh}), the last line should be:
\begin{verbatim}
        exec $*:q
\end{verbatim}

  On Windows, the end should look like:
\begin{verbatim}
REM set some environment variables
set LICENSE_SERVER=192.168.1.202:5012
set MY_PARAMS=2

REM Run the actual job now
%*

\end{verbatim}

  This syntax is precise, to correctly handle program arguments
  which contain white space characters.

  For Windows machines, the wrapper will either be
  a batch script with a file extension of \File{.bat} or \File{.cmd},
  or an executable with a file extension of \File{.exe} or \File{.com}.

  If the wrapper script encounters an error as it runs,
  and it is unable to run the user job, 
  it is important that the wrapper script indicate this to the HTCondor system
  so that HTCondor does not assign the exit code of the wrapper script to the
  job.  
  To do this, the wrapper script should write a useful error message
  to the file named in the environment variable 
  \Env{\_CONDOR\_WRAPPER\_ERROR\_FILE},
  and then the wrapper script should exit with a non-zero value.
  If this file is created by the wrapper script, 
  HTCondor assumes that the wrapper script has failed, 
  and HTCondor will place the job back in the queue marking it as Idle, 
  such that the job will again be run.
  The \Condor{starter} will also copy the contents of this error file to
  the \Condor{starter} log, so the administrator can debug the problem.

  When a wrapper script is in use, the executable of a job submission may be
  specified by a relative path, as long as the submit description file
  also contains:
\begin{verbatim}
        +PreserveRelativeExecutable = True
\end{verbatim}
  For example,
\begin{verbatim}
        # Let this executable be resolved by user's path in the wrapper
        cmd = sleep
        +PreserveRelativeExecutable = True
\end{verbatim}
  Without this extra attribute:
\begin{verbatim}
        # A typical fully-qualified executable path
        cmd = /bin/sleep
\end{verbatim}

\label{param:CgroupMemoryLimitPolicy} 
\item[\Macro{CGROUP\_MEMORY\_LIMIT\_POLICY}]
  A string with values of \Expr{hard}, \Expr{soft} and the default value
  of \Expr{none}.
  If set to \Expr{hard}, the cgroup-based limit on the total amount 
  of physical memory used by the sum of all processes in the job 
  will not be allowed to exceed the limit given by the cgroup memory
  controller attribute memory.limit\_in\_bytes.
  If the processes try to allocate more memory, 
  the allocation will succeed, and virtual memory will be allocated, 
  but no additional physical memory will be allocated.
  If set to \Expr{soft}, the cgroup-based limit on the total amount 
  of physical memory used by the sum of all processes in the job 
  will be allowed to go over the limit, 
  if there is free memory available on the system.

\label{param:UseVisibleDesktop} 
\item[\Macro{USE\_VISIBLE\_DESKTOP}]
  This boolean variable is only meaningful on Windows machines.  
  If \Expr{True}, HTCondor will
  allow the job to create windows on the desktop of the execute machine and
  interact with the job.  This is particularly useful for debugging why an
  application will not run under HTCondor.  
  If \Expr{False}, HTCondor uses the default
  behavior of creating a new, non-visible desktop to run the job on.
  See section~\ref{sec:platform-windows} for details on how HTCondor 
  interacts with the desktop.

\label{param:StarterJobEnvironment}
\item[\Macro{STARTER\_JOB\_ENVIRONMENT}]
  This macro sets the default environment inherited by jobs.  The syntax is
  the same as the syntax for environment settings in the job submit file
  (see page~\pageref{man-condor-submit-environment}).
  If the same environment variable is assigned by this macro and by the user
  in the submit file, the user's setting takes precedence.

\label{param:JobInheritsStarterEnvironment} 
\item[\Macro{JOB\_INHERITS\_STARTER\_ENVIRONMENT}]
  A boolean value that defaults to \Expr{False}.
  When \Expr{True},
  it causes jobs to inherit all environment variables from 
  the \Condor{starter}.
  When the user job and/or \MacroNI{STARTER\_JOB\_ENVIRONMENT} define
  an environment variable that is in the \Condor{starter}'s
  environment, the setting from the \Condor{starter}'s environment
  is overridden.
  This variable does not apply to standard universe jobs.

\label{param:NamedChroot} 
\item[\Macro{NAMED\_CHROOT}]
  A comma and/or space separated list of full paths to one or more directories,
  under which the \Condor{starter} may run a chroot-ed job.
  This allows HTCondor to invoke \Procedure{chroot} before launching a job,
  if the job requests such by defining the job ClassAd attribute
  \Attr{RequestedChroot} with a directory that matches one in this list.
  There is no default value for this variable.

\label{param:StarterUploadTimeout} 
\item[\Macro{STARTER\_UPLOAD\_TIMEOUT}]
  An integer value that specifies the network communication timeout to use
  when transferring files back to the submit machine.  The default value is
  set by the \Condor{shadow} daemon to 300.
  Increase this value if the disk on the submit machine
  cannot keep up with large bursts of activity, such as many jobs all
  completing at the same time.

\label{param:EnforceCpuAffinity} 
\item[\Macro{ENFORCE\_CPU\_AFFINITY}]
  A boolean value that defaults to \Expr{False}.  When \Expr{False},
  the affinity of jobs and their descendants to a CPU is not enforced.
  When \Expr{True}, HTCondor jobs and their descendants maintain their
  affinity to a CPU.
  When \Expr{True}, more fine grained affinities may be specified with
  \MacroNI{SLOT<N>\_CPU\_AFFINITY}.

\label{param:SlotNCpuAffinity} 
\item[\Macro{SLOT<N>\_CPU\_AFFINITY}]
  A comma separated list of cores to which an HTCondor job running on
  a specific slot given by the value of \MacroNI{<N>} show affinity.
  Note that slots are numbered beginning with the value 1,
  while CPU cores are numbered beginning with the value 0.
  This affinity list only takes effect if
  \Expr{ENFORCE\_CPU\_AFFINITY = True}.

\label{param:AssignCpuAffinity} 
\item[\Macro{ASSIGN\_CPU\_AFFINITY}]
  A boolean expression that defaults to \Expr{False}.
  When \Expr{True},
  CPU affinity is automatically set and enforced to be one slot per core.
  This permits affinity to work well with dynamic slots. 
  Also when \Expr{True}, overrides any settings specified by
  \MacroNI{ENFORCE\_CPU\_AFFINITY}.

\label{param:EnableURLTransfers} 
\item[\Macro{ENABLE\_URL\_TRANSFERS}]
  A boolean value that when \Expr{True} causes the \Condor{starter} for
  a job to invoke all plug-ins defined by \MacroNI{FILETRANSFER\_PLUGINS}
  to determine their capabilities for handling protocols to be
  used in file transfer specified with a URL.
  When \Expr{False}, a URL transfer specified in a job's submit description
  file will cause an error issued by \Condor{submit}.
  The default value is \Expr{True}.

\label{param:FiletransferPlugins} 
\item[\Macro{FILETRANSFER\_PLUGINS}]
  A comma separated list of full and absolute path and executable names
  for plug-ins that will accomplish the task of doing file transfer
  when a job requests the transfer of an input file by specifying a URL. 
  See section~\ref{sec:URL-transfer} for a description of the functionality
  required of a plug-in.

\label{param:EnableChirp} 
\item[\Macro{ENABLE\_CHIRP}]
  A boolean value that defaults to \Expr{True}. An administrator
  would set the value to \Expr{False} to disable Chirp remote file access 
  from execute machines. 

\label{param:UsePSS} 
\item[\Macro{USE\_PSS}]
  A boolean value, that when \Expr{True} causes the \Condor{starter} to
  measure the PSS (Proportional Set Size) of each HTCondor job.
  The default value is \Expr{True}.
  When running many short lived jobs, performance problems in 
  the \Condor{procd} have been observed, and a setting of \Expr{False}
  may relieve these problems.

\label{param:MemoryUsageMetric} 
\item[\Macro{MEMORY\_USAGE\_METRIC}]
  A ClassAd expression that produces an initial value for the job ClassAd
  attribute \Attr{MemoryUsage} in jobs that are \emph{not} standard 
  universe and \emph{not} vm universe.

\label{param:MemoryUsageMetricVM} 
\item[\Macro{MEMORY\_USAGE\_METRIC\_VM}]
  A ClassAd expression that produces an initial value for the job ClassAd
  attribute \Attr{MemoryUsage} in vm universe jobs.

\label{param:StarterRlimitAs} 
\item[ \Macro{STARTER\_RLIMIT\_AS}]
  An integer ClassAd expression, 
  expressed in Mbytes,
  evaluated by the \Condor{starter} to set the \Code{RLIMIT\_AS} parameter
  of the \Procedure{setrlimit} system call.
  This limits the virtual memory size of each process in the user job.  
  The expression is
  evaluated in the context of both the machine and job ClassAds,
  where the machine ClassAd is the \Expr{MY.} ClassAd,
  and the job ClassAd is the \Expr{TARGET.} ClassAd.
  There is no default value for this variable.
  Since values larger than 2047 have no real meaning on 32-bit platforms,
  values larger than 2047 result in no limit set on 32-bit platforms.

\label{param:UsePidNamespaces} 
\item[ \Macro{USE\_PID\_NAMESPACES}]
  A boolean value that, when \Expr{True}, enables the use of per job PID
  namespaces for HTCondor jobs run on Linux kernels.
  Defaults to \Expr{False}.

\label{param:PerJobNamespaces} 
\item[ \Macro{PER\_JOB\_NAMESPACES}]
  A boolean value that defaults to \Expr{False}.
  Relevant only for Linux platforms using file system namespaces.
  The default value of \Expr{False} ensures that there will be no
  private mount points, because auto mounts done by \Prog{autofs} 
  would use the wrong name for private file system mounts. 
  A \Expr{True} value is useful when private file system mounts are
  permitted and \Prog{autofs} (for NFS) is not used.

\end{description}

%%%%%%%%%%%%%%%%%%%%%%%%%%%%%%%%%%%%%%%%%%%%%%%%%%%%%%%%%%%%%%%%%%%%%%%%%%%
\subsection{\label{sec:Submit-Config-File-Entries}\condor{submit}
Configuration File Entries}
%%%%%%%%%%%%%%%%%%%%%%%%%%%%%%%%%%%%%%%%%%%%%%%%%%%%%%%%%%%%%%%%%%%%%%%%%%%
\index{configuration!condor\_submit configuration variables}

\begin{description}
\label{param:DefaultUniverse}
\item[\Macro{DEFAULT\_UNIVERSE}]
  The universe under which a job is executed may be specified in the submit
  description file.
  If it is not specified in the submit description file, then
  this variable specifies the universe (when defined).
  If the universe is not specified in the submit description
  file, and if this variable is not defined, then
  the default universe for a job will be the vanilla universe.

\label{param:JobDefaultNotification}
\item[\Macro{JOB\_DEFAULT\_NOTIFICATION}]
  The default that sets email notification for jobs. 
  This variable defaults to \Expr{NEVER},
  such that HTCondor will not send email about events for jobs. 
  Possible values are
  \Expr{NEVER}, \Expr{ERROR}, \Expr{ALWAYS}, or \Expr{COMPLETE}. 
  If \Expr{ALWAYS}, the owner will be notified whenever the job produces a
  checkpoint, as well as when the job completes. 
  If \Expr{COMPLETE}, the owner will be notified when the job terminates.
  If \Expr{ERROR}, the owner
  will only be notified if the job terminates abnormally, 
  or if the job is placed on hold because of a failure, 
  and not by user request. 
  If \Expr{NEVER}, the owner will not receive email. 

\label{param:JobDefaultRequestMemory}
\item[\Macro{JOB\_DEFAULT\_REQUESTMEMORY}]
  The amount of memory in Mbytes to acquire for a job, 
  if the job does not specify how much it needs using the 
  \SubmitCmd{request\_memory} submit command.
  If this variable is not defined, then the default is defined by
  the expression
\begin{verbatim}
  ifThenElse(MemoryUsage =!= UNDEFINED,MemoryUsage,1)
\end{verbatim}

\label{param:JobDefaultRequestDisk}
\item[\Macro{JOB\_DEFAULT\_REQUESTDISK}]
  The amount of disk in Kbytes to acquire for a job,
  if the job does not specify how much it needs using the 
  \SubmitCmd{request\_disk} submit command.
  If the job defines the value, then that value takes precedence. 
  If not set, then then the default is defined as \Expr{DiskUsage}.

\label{param:JobDefaultRequestCpus}
\item[\Macro{JOB\_DEFAULT\_REQUESTCPUS}]
  The number of CPUs to acquire for a job,
  if the job does not specify how many it needs using the 
  \SubmitCmd{request\_cpus} submit command.
  If the job defines the value, then that value takes precedence. 
  If not set, then then the default is 1.

\end{description}

If you want \Condor{submit} to automatically append an expression to
the \AdAttr{Requirements} expression or \AdAttr{Rank} expression of 
jobs at your site use the following macros:
\begin{description}
  
\label{param:AppendReqVanilla}
\item[\Macro{APPEND\_REQ\_VANILLA}]
  Expression to be appended to vanilla job requirements.
  
\label{param:AppendReqStandard}
\item[\Macro{APPEND\_REQ\_STANDARD}]
  Expression to be appended to standard job requirements.

\label{param:AppendReq}
\item[\Macro{APPEND\_REQUIREMENTS}]
  Expression to be appended to any type of universe jobs. 
  However, if \MacroNI{APPEND\_REQ\_VANILLA} or \MacroNI{APPEND\_REQ\_STANDARD}
  is defined, then ignore the \MacroNI{APPEND\_REQUIREMENTS} for those
  universes.

\label{param:AppendRank}
\item[\Macro{APPEND\_RANK}]
  Expression to be appended to job rank.  \MacroNI{APPEND\_RANK\_STANDARD} or
  \MacroNI{APPEND\_RANK\_VANILLA} will override this setting if defined.

\label{param:AppendRankStandard}
\item[\Macro{APPEND\_RANK\_STANDARD}]
  Expression to be appended to standard job rank.

\label{param:AppendRankVanilla}
\item[\Macro{APPEND\_RANK\_VANILLA}]
  Expression to append to vanilla job rank.

\end{description}

\Note The \Macro{APPEND\_RANK\_STANDARD} and 
\Macro{APPEND\_RANK\_VANILLA} macros were called
\Macro{APPEND\_PREF\_STANDARD} and
\Macro{APPEND\_PREF\_VANILLA} in previous versions of HTCondor.

In addition, you may provide default \AdAttr{Rank} expressions if your users
do not specify their own with:

\begin{description}

\label{param:DefaultRank}
\item[\Macro{DEFAULT\_RANK}]
  Default rank expression for any job that does not specify
  its own rank expression in the submit description file.  
  There is no default value, such that when undefined,
  the value used will be 0.0.

\label{param:DefaultRankVanilla}
\item[\Macro{DEFAULT\_RANK\_VANILLA}]
  Default rank for vanilla universe jobs.  
  There is no default value, such that when undefined,
  the value used will be 0.0.
  When both \MacroNI{DEFAULT\_RANK} and \MacroNI{DEFAULT\_RANK\_VANILLA}
  are defined, the value for \MacroNI{DEFAULT\_RANK\_VANILLA} is
  used for vanilla universe jobs.

\label{param:DefaultRankStandard}
\item[\Macro{DEFAULT\_RANK\_STANDARD}]
  Default rank for standard universe jobs.
  There is no default value, such that when undefined,
  the value used will be 0.0.
  When both \MacroNI{DEFAULT\_RANK} and \MacroNI{DEFAULT\_RANK\_STANDARD}
  are defined, the value for \MacroNI{DEFAULT\_RANK\_STANDARD} is
  used for standard universe jobs.

\label{param:DefaultBufferSize}
\item[\Macro{DEFAULT\_IO\_BUFFER\_SIZE}]
  HTCondor keeps a buffer of recently-used data for each file an
  application opens.  This macro specifies the default maximum number
  of bytes to be buffered for each open file at the executing machine.
  The \Condor{status} \MacroNI{buffer\_size} command will override this
  default.  If this macro is undefined, a default size of 512 KB will
  be used.

\label{param:DefaultBufferBlockSize}
\item[\Macro{DEFAULT\_IO\_BUFFER\_BLOCK\_SIZE}] 
  When buffering is enabled,
  HTCondor will attempt to consolidate small read and write operations
  into large blocks.  This macro specifies the default block size
  HTCondor will use.  The \Condor{status} \MacroNI{buffer\_block\_size}
  command will override this default.  If this macro is undefined, a
  default size of 32 KB will be used.

\label{param:SubmitSkipFilechecks}
\item[\Macro{SUBMIT\_SKIP\_FILECHECKS}]
  If \Expr{True}, \Condor{submit} behaves as if the \Opt{-disable} 
  command-line option is used.
  This tells \Condor{submit} to disable file permission checks 
  when submitting a job
  for read permissions on all input files, such as those defined by
  commands \SubmitCmd{input} and \SubmitCmd{transfer\_input\_files},
  as well as write permission to output files, such as a
  log file defined by \SubmitCmd{log} and output files defined with 
  \SubmitCmd{output} or \SubmitCmd{transfer\_output\_files}.
  This can significantly decrease the amount of time required to submit
  a large group of jobs.
  The default value is \Expr{False}.

\label{param:WarnOnUnusedSubmitFileMacros}
\item[\Macro{WARN\_ON\_UNUSED\_SUBMIT\_FILE\_MACROS}]
  A boolean variable that defaults to \Expr{True}.
  When \Expr{True}, \Condor{submit}
  performs checks on the job's submit description file contents
  for commands that define a macro, but do not use the macro within
  the file.
  A warning is issued, but job submission continues.
  A definition of a new macro occurs when the lhs of a command is not
  a known submit command.  This check may help spot spelling errors
  of known submit commands.

\label{param:SubmitSendReschedule}
\item[\Macro{SUBMIT\_SEND\_RESCHEDULE}]
  A boolean expression that when False, prevents \Condor{submit} from
  automatically sending a \Condor{reschedule} command as it completes.
  The \Condor{reschedule} command causes the \Condor{schedd} daemon
  to start searching for machines with which to match the submitted
  jobs.  When True, this step always occurs.
  In the case that the machine where the job(s) are submitted is
  managing a huge number of jobs (thousands or tens of thousands),
  this step would hurt performance in such a way that it became
  an obstacle to scalability.
  The default value is True.

\label{param:SubmitExprs}
\item[\Macro{SUBMIT\_EXPRS}]
  A comma-separated and/or space-separated 
  list of ClassAd attribute names for which the attribute and value will
  be inserted into all the job ClassAds that \Condor{submit} creates.  
  In this way,
  it is like the \verb@"+"@ syntax in a submit description file.
  Attributes defined in the submit description file with \verb@"+"@ will
  override attributes defined in the configuration file with
  \MacroNI{SUBMIT\_EXPRS}. 
  Note that adding an attribute to a job's ClassAd will \emph{not} function
  as a method for specifying default values of submit description file commands
  forgotten in a job's submit description file.
  The command in the submit description file results in actions by
  \Condor{submit},
  while the use of \MacroNI{SUBMIT\_EXPRS} adds a job ClassAd attribute
  at a later point in time.

\label{param:LogOnNfsIsError}
\item[\Macro{LOG\_ON\_NFS\_IS\_ERROR}]
  A boolean value that controls whether \Condor{submit} prohibits
  job submit files with user log files on NFS.  If
  \MacroNI{LOG\_ON\_NFS\_IS\_ERROR} is set to \Expr{True}, such
  submit files will be rejected.  If \MacroNI{LOG\_ON\_NFS\_IS\_ERROR}
  is set to \Expr{False},
  the job will be submitted.  If not defined,
  \MacroNI{LOG\_ON\_NFS\_IS\_ERROR} defaults to \Expr{False}.

\label{param:SubmitMaxProcsInCluster}
\item[\Macro{SUBMIT\_MAX\_PROCS\_IN\_CLUSTER}]
  An integer value that limits the maximum number of jobs that would
  be assigned within a single cluster.  Job submissions that would exceed
  the defined value fail, issuing an error message, and with no jobs
  submitted.
  The default value is 0, which does not limit the number of jobs
  assigned a single cluster number.

\label{param:EnableDeprecationWarnings}
\item[\Macro{ENABLE\_DEPRECATION\_WARNINGS}]
  A boolean value that defaults to \Expr{False}.
  When \Expr{True}, \Condor{submit} issues warnings when a job requests
  features that are no longer supported. 

\label{param:InteractiveSubmitFile}
\item[\Macro{INTERACTIVE\_SUBMIT\_FILE}]
  The path and file name of a submit description file that \Condor{submit}
  will use in the specification of an interactive job.
  The default is \File{\MacroUNI{RELEASE\_DIR}/libexec/interactive.sub}
  when not defined.

\end{description}

%%%%%%%%%%%%%%%%%%%%%%%%%%%%%%%%%%%%%%%%%%%%%%%%%%%%%%%%%%%%%%%%%%%%%%%%%%%
\subsection{\label{sec:Preen-Config-File-Entries}\condor{preen}
Configuration File Entries}
%%%%%%%%%%%%%%%%%%%%%%%%%%%%%%%%%%%%%%%%%%%%%%%%%%%%%%%%%%%%%%%%%%%%%%%%%%%

\index{configuration!condor\_preen configuration variables}
These macros affect \Condor{preen}.

\begin{description}

\item[\Macro{PREEN\_ADMIN}]
\label{param:PreenAdmin}
  This macro sets the e-mail address where \Condor{preen} will send e-mail
  (if it is configured to send email at all; see the entry for \MacroNI{PREEN}).
  Defaults to \MacroUNI{CONDOR\_ADMIN}.

\label{param:ValidSpoolFiles}
\item[\Macro{VALID\_SPOOL\_FILES}]
  This macro contains a (comma or space separated) list of files that
  \Condor{preen} considers valid files to find in the \MacroUNI{SPOOL}
  directory. There is no default value. \Condor{preen} will add to the
  list files and directories that are normally present in the
  \MacroUNI{SPOOL} directory.
  
\label{param:InvalidLogFiles}
\item[\Macro{INVALID\_LOG\_FILES}]
  This macro contains a (comma or space separated) list of files that
  \Condor{preen} considers invalid files to find in the \MacroUNI{LOG}
  directory.  There is no default value.

\end{description}


%%%%%%%%%%%%%%%%%%%%%%%%%%%%%%%%%%%%%%%%%%%%%%%%%%%%%%%%%%%%%%%%%%%%%%%%%%%
\subsection{\label{sec:Collector-Config-File-Entries}\condor{collector}
Configuration File Entries}
%%%%%%%%%%%%%%%%%%%%%%%%%%%%%%%%%%%%%%%%%%%%%%%%%%%%%%%%%%%%%%%%%%%%%%%%%%%

\index{configuration!condor\_collector configuration variables}
These macros affect the \Condor{collector}.
\begin{description}
  
\label{param:ClassadLifetime}
\item[\Macro{CLASSAD\_LIFETIME}]
  The default maximum age in seconds for ClassAds collected by the
  \Condor{collector}.  ClassAds older than the maximum age are
  discarded by the \Condor{collector} as stale.

  If present, the ClassAd attribute \Attr{ClassAdLifetime} specifies the
  ClassAd's lifetime in seconds.  
  If \Attr{ClassAdLifetime} is not present in the ClassAd, 
  the \Condor{collector} will use the value of
  \MacroUNI{CLASSAD\_LIFETIME}.  
  This variable is defined in terms of seconds, 
  and it defaults to 900 seconds (15 minutes).
  
\label{param:MasterCheckInterval}
\item[\Macro{MASTER\_CHECK\_INTERVAL}]
  This macro defines how often the
  collector should check for machines that have ClassAds from some
  daemons, but not from the \Condor{master} (\Term{orphaned daemons})
  and send e-mail about it.  It is defined in seconds and 
  defaults to 10800 (3 hours).

\label{param:CollectorRequirements}
\item[\Macro{COLLECTOR\_REQUIREMENTS}]
  A boolean expression that filters out unwanted ClassAd updates.  The
  expression is evaluated for ClassAd updates that have 
  passed through enabled security authorization checks.
  The default behavior when this expression is not
  defined is to allow all ClassAd updates to take place.
  If \Expr{False}, a ClassAd update will be rejected.

  Stronger security mechanisms are the better way to
  authorize or deny updates to the \Condor{collector}.
  This configuration variable exists to help those that
  use host-based security, and
  do not trust all processes that run on the hosts in the pool.
  This configuration variable may be used to throw out ClassAds that
  should not be allowed.  For example, for
  \Condor{startd} daemons that run on a fixed port,
  configure this expression to ensure that 
  only machine ClassAds advertising the expected
  fixed port are accepted.  As a convenience, before evaluating the
  expression, some basic sanity checks are performed on the ClassAd to
  ensure that all of the ClassAd attributes used by HTCondor to contain
  IP:port information are consistent.  To validate this
  information, the attribute to check is \AdAttr{TARGET.MyAddress}.
 

\label{param:ClientTimeout}
\item[\Macro{CLIENT\_TIMEOUT}]
  Network timeout that the \Condor{collector} uses when talking to any daemons
  or tools that are sending it a ClassAd update.
  It is defined in seconds and defaults to 30.
  
\label{param:QueryTimeout}
\item[\Macro{QUERY\_TIMEOUT}]
  Network timeout when talking to anyone doing a query.
  It is defined in seconds and defaults to 60.
  
\label{param:CondorDevelopers}
\item[\Macro{CONDOR\_DEVELOPERS}]
  By default,
  HTCondor will send e-mail once per week to this address with the output
  of the \Condor{status} command, which lists how many machines
  are in the pool and how many are running jobs.  The default
  value of \Email{condor-admin@cs.wisc.edu} will send this report to
  the Center for High Throughput Computing at 
  the University of Wisconsin-Madison.
  The Center for High Throughput Computing uses
  these weekly status messages in order to have some idea as to how
  many HTCondor pools exist in the world.  We appreciate
  getting the reports, as this is one way we can convince funding
  agencies that HTCondor is being used in the real world.  
  If you do not wish this information to be sent to 
  the Center for High Throughput Computing,
  explicitly set the value to \Expr{NONE} to disable this feature,
  or replace the
  address with a desired location.  
  If undefined (commented out) in the configuration file, HTCondor follows
  its default behavior.

\label{param:CollectorName}
\item[\Macro{COLLECTOR\_NAME}]
  This macro is used to specify a short description of your pool.
  It should be about 20 characters long.  For example, the name of the
  UW-Madison Computer Science HTCondor Pool is \AdStr{UW-Madison CS}.  
  While this macro might seem similar to \MacroNI{MASTER\_NAME} or
  \MacroNI{SCHEDD\_NAME}, it is unrelated.
  Those settings are used to uniquely identify (and locate) a specific
  set of HTCondor daemons, if there are more than one running on the same
  machine.
  The \MacroNI{COLLECTOR\_NAME} setting is just used as a
  human-readable string to describe the pool, which is included in the
  updates sent to the \MacroNI{CONDOR\_DEVELOPERS\_COLLECTOR}.

\label{param:CondorDevelopersCollector}
\item[\Macro{CONDOR\_DEVELOPERS\_COLLECTOR}]
  By default, every pool sends
  periodic updates to a central \Condor{collector} at UW-Madison with
  basic information about the status of the pool.  Updates include only
  the number of total machines, the number of jobs submitted, the
  number of machines running jobs, the host name of the central
  manager, and the \MacroUNI{COLLECTOR\_NAME}.  These
  updates help the Center for High Throughput Computing see how HTCondor 
  is being used around the world.
  By default, they will be sent to \File{condor.cs.wisc.edu}.
  To discontinue sending updates,
  explicitly set this macro to \Expr{NONE}. 
  If undefined or commented out in the configuration file, HTCondor follows
  its default behavior.

\label{param:CollectorUpdateInterval}
\item[\Macro{COLLECTOR\_UPDATE\_INTERVAL}]
  This variable is defined in seconds and defaults to 900 (every 15 minutes).
  It controls the frequency of the periodic
  updates sent to a central \Condor{collector} at UW-Madison as
  defined by \MacroNI{CONDOR\_DEVELOPERS\_COLLECTOR}.

\label{param:CollectorSocketBufsize}
\item[\Macro{COLLECTOR\_SOCKET\_BUFSIZE}] 
  This specifies the buffer size, in
  bytes, reserved for \Condor{collector} network UDP sockets.  The default is
  10240000, or a ten megabyte buffer.  This is a healthy size, even for a large
  pool.  The larger this value, the less likely the \Condor{collector} will
  have stale information about the pool due to dropping update packets.  If
  your pool is small or your central manager has very little RAM, considering
  setting this parameter to a lower value (perhaps 256000 or 128000).

  \Note For some Linux distributions, it may be necessary to raise the
  OS's system-wide limit for network buffer sizes. The parameter that
  controls this limit is /proc/sys/net/core/rmem\_max. You can see the
  values that the \Condor{collector} actually uses by enabling D\_FULLDEBUG
  for the collector and looking at the log line that looks like this:

  Reset OS socket buffer size to 2048k (UDP), 255k (TCP).

  For changes to this parameter to take effect, \Condor{collector} must
  be restarted.

\label{param:CollectorTcpSocketBufsize}
\item[\Macro{COLLECTOR\_TCP\_SOCKET\_BUFSIZE}]
  This specifies the TCP buffer
  size, in  bytes, reserved for \Condor{collector} network sockets.  The
  default is 131072, or a 128 kilobyte buffer.  This is a healthy size, even
  for a large pool.  The larger this value, the less likely the
  \Condor{collector} will have stale information about the pool due to
  dropping update packets.  If your pool is small or your central
  manager has very little RAM, considering setting this parameter to a
  lower value (perhaps 65536 or 32768).

  \Note See the note for \Macro{COLLECTOR\_SOCKET\_BUFSIZE}.

\label{param:KeepPoolHistory}
\item[\Macro{KEEP\_POOL\_HISTORY}]
  This boolean macro is used to decide if the collector will write
  out statistical information about the pool to history files.
  The default is \Expr{False}.
  The location, size, and frequency of history logging is controlled
  by the other macros.

\label{param:PoolHistoryDir}
\item[\Macro{POOL\_HISTORY\_DIR}]
  This macro sets the name of the directory where the history
  files reside (if history logging is enabled).
  The default is the \File{SPOOL} directory.

\label{param:PoolHistoryMaxStorage} 
\item[\Macro{POOL\_HISTORY\_MAX\_STORAGE}]
  This macro sets the maximum combined size of the history files.
  When the size of the history files is close to this limit, the oldest
  information will be discarded.
  Thus, the larger this parameter's value is, the larger the time
  range for which history will be available.  The default value is
  10000000 (10 Mbytes).

\label{param:PoolHistorySamplingInterval}
\item[\Macro{POOL\_HISTORY\_SAMPLING\_INTERVAL}]
  This macro sets the interval, in seconds, between samples for
  history logging purposes. 
  When a sample is taken, the collector goes through the information
  it holds, and summarizes it.
  The information is written to the history file once for each 4
  samples.
  The default (and recommended) value is 60 seconds. Setting this
  macro's value too low will increase the load on the collector,
  while setting it to high will produce less precise statistical
  information.

\label{param:CollectorDaemonStats}
\item[\Macro{COLLECTOR\_DAEMON\_STATS}]
  A boolean value that controls whether or not the \Condor{collector} daemon
  keeps update statistics on incoming updates.  
  The default value is \Expr{True}.
  If enabled, the \Condor{collector} will insert several attributes
  into the ClassAds that it stores and sends.  ClassAds without the
  \Attr{UpdateSequenceNumber} and \Attr{DaemonStartTime} attributes will not
  be counted, and will not have attributes inserted (all modern HTCondor
  daemons which publish ClassAds publish these attributes).

  \index{ClassAd attribute added by the condor\_collector!UpdatesTotal}
  \index{ClassAd attribute added by the condor\_collector!UpdatesSequenced}
  \index{ClassAd attribute added by the condor\_collector!UpdatesLost}
  The attributes inserted are \Attr{UpdatesTotal}, \Attr{UpdatesSequenced},
  and \Attr{UpdatesLost}.  \Attr{UpdatesTotal} is the total number of
  updates (of this ClassAd type) the \Condor{collector} has received 
  from this host.
  \Attr{UpdatesSequenced} is the number of updates that the \Condor{collector}
  could have as lost.  In particular, for the first update from a
  daemon, it is impossible to tell if any previous ones have been lost or not.
  \Attr{UpdatesLost} is the number of updates that the \Condor{collector}
  has detected as being lost.
  See page~\pageref{sec:Collector-Added-Attributes} for more information on the
  added attributes.

\label{param:CollectorStatsSweep}
\item[\Macro{COLLECTOR\_STATS\_SWEEP}]
  This value specifies the number of
  seconds between sweeps of the \Condor{collector}'s per-daemon update
  statistics.  Records for daemons which have not reported in this amount
  of time are purged in order to save memory.  The default is two days.
  It is unlikely that you would ever need to adjust this.

\index{ClassAd attribute added by the condor\_collector!UpdatesHistory}
\label{param:CollectorDaemonHistorySize}
\item[\Macro{COLLECTOR\_DAEMON\_HISTORY\_SIZE}]
  This variable controls the
  size of the published update history that the \Condor{collector} inserts into
  the ClassAds it stores and sends.  The default value is 128, which
  means that history is stored and published for the latest 128
  updates.  This variable's value is ignored,
  if \Macro{COLLECTOR\_DAEMON\_STATS} is not enabled.

  If the value is a non-zero one, the \Condor{collector} will insert
  attribute
  \Attr{UpdatesHistory} into the ClassAd (similar to \Attr{UpdatesTotal}).
  Attr{UpdatesHistory} is a hexadecimal string which represents
  a bitmap of the last \Macro{COLLECTOR\_DAEMON\_HISTORY\_SIZE} updates.
  The most significant bit (MSB) of the bitmap represents the
  most recent update, and the least significant bit (LSB) represents
  the least recent.  A value of zero means that the update was not lost,
  and a value of 1 indicates that the update was detected as lost.

  For example, if the last update was not lost, the previous was lost, and
  the previous two not, the bitmap would be 0100, and the matching hex
  digit would be \AdStr{4}.  Note that the MSB can never be marked as lost
  because its loss can only be detected by a non-lost update 
  (a gap is found in the sequence numbers).  
  Thus, \Expr{UpdatesHistory = "0x40"} 
  would be the history for the last 8 updates.  
  If the next updates are all successful, the values published,
  after each update,
  would be: 0x20, 0x10, 0x08, 0x04, 0x02, 0x01, 0x00.

  See page~\pageref{sec:Collector-Added-Attributes} for more information on the
  added attribute.


\label{param:CollectorClassHistorySize}
\item[\Macro{COLLECTOR\_CLASS\_HISTORY\_SIZE}]
  This variable controls the
  size of the published update history that the \Condor{collector} inserts into
  the \Condor{collector} ClassAds it produces.  
  The default value is zero.

  If this variable has a non-zero value, the \Condor{collector} will insert
  \Attr{UpdatesClassHistory} into the \Condor{collector} ClassAd 
  (similar to \Attr{UpdatesHistory}).  
  These are added per class of ClassAd, however.  
  The classes refer to the type of ClassAds.
  Additionally, there is a Total class created,
  which represents the history of all ClassAds that this \Condor{collector}
  receives.

  Note that the \Condor{collector} always publishes Lost, Total and Sequenced
  counts for all ClassAd classes.  This is similar to the
  statistics gathered if \Macro{COLLECTOR\_DAEMON\_STATS} is enabled.

\label{param:CollectorQueryWorkers}
\item[\Macro{COLLECTOR\_QUERY\_WORKERS}]
  This variable sets the maximum
  number of worker processes that the \Condor{collector} can have.  
  When receiving a query request, 
  the Unix \Condor{collector} will \Procedure{fork} a new
  process to handle the query, freeing the main process to handle
  other requests.  When the number of outstanding worker processes
  reaches this maximum, the request is handled by the main process.
  This variable is ignored on Windows, and its default value is zero.
  The default configuration, however, has a value of 2.

\label{param:CollectorDebug}
\item[\Macro{COLLECTOR\_DEBUG}]
  This macro (and other macros related to debug logging in the 
  \Condor{collector}
  is described in section~\ref{param:SubsysDebug} as
  \MacroNI{<SUBSYS>\_DEBUG}.

\label{param:CondorViewClassadTypes}
\item[\Macro{CONDOR\_VIEW\_CLASSAD\_TYPES}]
  Provides the ClassAd types that will be forwarded to the
  \MacroNI{CONDOR\_VIEW\_HOST}. The ClassAd types can be found with 
  \Condor{status} \Opt{-any}. The default forwarding behavior of the 
  \Condor{collector} is equivalent to 
\begin{verbatim}
  CONDOR_VIEW_CLASSAD_TYPES=Machine,Submitter
\end{verbatim} 
  There is no default value for this variable.

\end{description}

The following macros control where, when, and for how long HTCondor 
persistently stores absent ClassAds.
See section~\ref{sec:Absent-Ads} on 
page~\pageref{sec:Absent-Ads} for more details.

\begin{description}

\label{param:AbsentRequirements}
\item[\Macro{ABSENT\_REQUIREMENTS}]
  A boolean expression evaluated by the \Condor{collector} when a 
  machine ClassAd would otherwise expire.
  If \Expr{True}, the ClassAd instead becomes absent.
  If not defined, the implementation will behave as if \Expr{False},
  and no absent ClassAds will be stored.

\label{param:AbsentExpireAdsAfter}
\item[\Macro{ABSENT\_EXPIRE\_ADS\_AFTER}]
  The integer number of seconds after which the \Condor{collector} 
  forgets about an absent ClassAd.
  If 0, the ClassAds persist forever.  
  Defaults to 30 days.

\label{param:CollectorPersisentAdLog}
\item[\Macro{COLLECTOR\_PERSISTENT\_AD\_LOG}]
  The full path and file name of a file that stores machine ClassAds 
  for every hibernating or absent machine.  This forms a persistent storage
  of these ClassAds, in case the \Condor{collector} daemon crashes.

  To avoid \Condor{preen} removing this log, place it in a directory
  other than the directory defined by \MacroUNI{SPOOL}.  
  Alternatively, if this log file is to go in the 
  directory defined by \MacroUNI{SPOOL}, add the file to the list
  given by \MacroNI{VALID\_SPOOL\_FILES}.

  This configuration variable replaces \MacroNI{OFFLINE\_LOG},
  which is no longer used.

\label{param:ExpireInvalidatedAds}
\item[\Macro{EXPIRE\_INVALIDATED\_ADS}]
  A boolean value that defaults to \Expr{False}.
  When \Expr{True}, causes all invalidated ClassAds to be treated 
  as if they expired.
  This permits invalidated ClassAds to be marked absent,
  as defined in section~\ref{sec:Absent-Ads}.
\end{description}

%%%%%%%%%%%%%%%%%%%%%%%%%%%%%%%%%%%%%%%%%%%%%%%%%%%%%%%%%%%%%%%%%%%%%%%%%%%
\subsection{\label{sec:Negotiator-Config-File-Entries}\condor{negotiator}
Configuration File Entries}
%%%%%%%%%%%%%%%%%%%%%%%%%%%%%%%%%%%%%%%%%%%%%%%%%%%%%%%%%%%%%%%%%%%%%%%%%%%
\index{configuration!condor\_negotiator configuration variables}

These macros affect the \Condor{negotiator}.
\begin{description}
  
\label{param:NegotiatorInterval}
\item[\Macro{NEGOTIATOR\_INTERVAL}]
  Sets how often the \Condor{negotiator} starts a negotiation cycle.
  It is defined in seconds and defaults to 60 (1 minute).
  
\label{param:NegotiatorCycleDelay}
\item[\Macro{NEGOTIATOR\_CYCLE\_DELAY}]
  An integer value that represents the minimum number of seconds
  that must pass before a new negotiation cycle may start.
  The default value is 20.
  \MacroNI{NEGOTIATOR\_CYCLE\_DELAY} is intended only for use by
  HTCondor experts.

\label{param:NegotiatorTimeout}
\item[\Macro{NEGOTIATOR\_TIMEOUT}]
  Sets the timeout that the negotiator uses on its network connections
  to the \Condor{schedd} and \Condor{startd}s.
  It is defined in seconds and defaults to 30.

\label{param:NegotiationCycleStatsLength}
\item[\Macro{NEGOTIATION\_CYCLE\_STATS\_LENGTH}] Specifies how many
  recent negotiation cycles should be included in the history that is
  published in the \Condor{negotiator}'s ad.  The default is 3 and the
  maximum allowed value is 100.  Setting this value to 0 disables
  publication of negotiation cycle statistics.  The
  statistics about recent cycles are stored in several attributes per
  cycle.  Each of these attribute names will have a number appended to
  it to indicate how long ago the cycle happened, for example:
  \AdAttr{LastNegotiationCycleDuration0},
  \AdAttr{LastNegotiationCycleDuration1},
  \AdAttr{LastNegotiationCycleDuration2}, \Dots.  The attribute
  numbered 0 applies to the most recent negotiation cycle.  The
  attribute numbered 1 applies to the next most recent negotiation
  cycle, and so on.  See
  page~\pageref{attr:LastNegotiationCycleActiveSubmitterCount<X>} for a
  list of attributes that are published.

\label{param:PriorityHalfLife}
\item[\Macro{PRIORITY\_HALFLIFE}]
  This macro defines the half-life of the user priorities.  See
  section~\ref{sec:user-priority-explained}
  on User Priorities for details.  It is defined in seconds and defaults
  to 86400 (1 day).

\label{param:DefaultPrioFactor} 
\item[\Macro{DEFAULT\_PRIO\_FACTOR}]
  Sets the priority factor for local users as they first submit jobs,
  as described in section~\ref{sec:UserPrio}.
  Defaults to 1.0.

\label{param:NiceUserPrioFactor} 
\item[\Macro{NICE\_USER\_PRIO\_FACTOR}]
  Sets the priority factor for nice users, as described in
  section~\ref{sec:UserPrio}.
  Defaults to 10000000.

\label{param:RemotePrioFactor} 
\item[\Macro{REMOTE\_PRIO\_FACTOR}]
  Defines the priority factor for remote users,
  which are those users who who do not belong to the local domain.
  See section~\ref{sec:UserPrio} for details.  
  Defaults to 10000.

\label{param:AccountantLocalDomain} 
\item[\Macro{ACCOUNTANT\_LOCAL\_DOMAIN}]
  Describes the local UID domain.
  This variable is used to decide if a user is local or remote. 
  A user is considered to be in the local domain if their UID domain matches
  the value of this variable. Usually, this variable is set
  to the local UID domain. 
  If not defined, all users are considered local.

\label{param:MaxAccountantDatabaseSize}
\item[\Macro{MAX\_ACCOUNTANT\_DATABASE\_SIZE}] 
  This macro defines the maximum size (in bytes) that the accountant
  database log file can reach before it is truncated (which re-writes
  the file in a more compact format).
  If, after truncating, the file is larger than one half the maximum
  size specified with this macro, the maximum size will be
  automatically expanded.
  The default is 1 megabyte (1000000).

\label{param:NegotiatorDiscountSuspendedResources} 
\item[\Macro{NEGOTIATOR\_DISCOUNT\_SUSPENDED\_RESOURCES}]
   This macro tells the negotiator to not count resources that are suspended
   when calculating the number of resources a user is using. 
   Defaults to false, that is, a user is still charged for a resource even
   when that resource has suspended the job.

\label{param:NegotiatorSocketCacheSize}
\item[\Macro{NEGOTIATOR\_SOCKET\_CACHE\_SIZE}]
  This macro defines the maximum number of sockets that the \Condor{negotiator}
  keeps in its open socket cache.
  Caching open sockets makes the negotiation
  protocol more efficient by eliminating the need for socket
  connection establishment for each negotiation cycle.  The default is
  currently 16.  To be effective, this parameter should be set to a
  value greater than the number of \Condor{schedd}s submitting jobs to the
  negotiator at any time.  If you lower this number, you must run
  \Condor{restart} and not just \Condor{reconfig} for the change to
  take effect.

\label{param:NegotiatorInformStartd}
\item[\Macro{NEGOTIATOR\_INFORM\_STARTD}]
  Boolean setting that controls if the \Condor{negotiator} should
  inform the \Condor{startd} when it has been matched with a job.
  The default is \Expr{True}.
  When this is set to \Expr{False}, the \Condor{startd} will never
  enter the Matched state, and will go directly from Unclaimed to
  Claimed.
  Because this notification is done via UDP, if a pool is configured
  so that the execute hosts do not create UDP command sockets (see the
  \Macro{WANT\_UDP\_COMMAND\_SOCKET} setting described in
  section~\ref{param:WantUDPCommandSocket} on
  page~\pageref{param:WantUDPCommandSocket} for details), the
  \Condor{negotiator} should be configured not to attempt to contact
  these \Condor{startds} by configuring this setting to \Expr{False}.

\label{param:NegotiatorPreJobRank}
\item[\Macro{NEGOTIATOR\_PRE\_JOB\_RANK}]
  Resources that match a request
  are first sorted by this expression.  If there are any ties in the
  rank of the top choice, the top resources are sorted by the
  user-supplied rank in the job ClassAd, then by
  \MacroNI{NEGOTIATOR\_POST\_JOB\_RANK}, then by
  \MacroNI{PREEMPTION\_RANK} (if the match would cause preemption and
  there are still any ties in the top choice).  \verb@MY@ refers to
  attributes of the machine ClassAd and \verb@TARGET@ refers to the
  job ClassAd.  The purpose of the pre job rank is to allow the pool
  administrator to override any other rankings, in order to optimize
  overall throughput.  For example, it is commonly used to minimize
  preemption, even if the job rank prefers a machine that is busy.  If
  undefined, this expression has no effect on the ranking of matches.
  The standard configuration file shipped with HTCondor specifies an
  expression to steer jobs away from busy resources:

\begin{verbatim}
  NEGOTIATOR_PRE_JOB_RANK = RemoteOwner =?= UNDEFINED
\end{verbatim}

\label{param:NegotiatorPostJobRank}
\item[\Macro{NEGOTIATOR\_POST\_JOB\_RANK}]
  Resources that match a request are first sorted by
  \MacroNI{NEGOTIATOR\_PRE\_JOB\_RANK}.  If there are any ties in the
  rank of the top choice, the top resources are sorted by the
  user-supplied rank in the job ClassAd, then by
  \MacroNI{NEGOTIATOR\_POST\_JOB\_RANK}, then by
  \MacroNI{PREEMPTION\_RANK} (if the match would cause preemption and
  there are still any ties in the top choice).  \Expr{MY.} refers to
  attributes of the machine ClassAd and \Expr{TARGET.} refers to the
  job ClassAd.  The purpose of the post job rank is to allow the pool
  administrator to choose between machines that the job ranks equally.
  The default value is undefined, which causes this rank to have no
  effect on the ranking of matches.  The following example expression
  steers jobs toward faster machines and tends to fill a cluster of
  multi-processors by spreading across all machines before filling up
  individual machines.  In this example, the expression is chosen to
  have no effect when preemption would take place, allowing control to
  pass on to \MacroNI{PREEMPTION\_RANK}.

\begin{verbatim}
  UWCS_NEGOTIATOR_POST_JOB_RANK = \
   (RemoteOwner =?= UNDEFINED) * (KFlops - SlotID)
\end{verbatim}


\label{param:PreemptionRequirements}
\item[\Macro{PREEMPTION\_REQUIREMENTS}]
  When considering user priorities, the negotiator will not preempt
  a job running on a given machine unless the
  \MacroNI{PREEMPTION\_REQUIREMENTS} expression evaluates to \Expr{True} and the
  owner of the idle job has a better priority than the owner of the
  running job. 
  The \MacroNI{PREEMPTION\_REQUIREMENTS} expression is evaluated within the
  context of the candidate machine ClassAd and the candidate idle job
  ClassAd; thus the \verb@MY@ scope prefix refers to the machine ClassAd,
  and the \verb@TARGET@ scope prefix refers to the ClassAd of the idle
  (candidate) job.  There is no direct access to the currently running job,
  but attributes of the currently running job that need to be accessed
  in \MacroNI{PREEMPTION\_REQUIREMENTS} can be placed in the machine ClassAd
  using \Macro{STARTD\_JOB\_EXPRS}.
  If not explicitly set in the HTCondor configuration file, the default value
  for this expression is \Expr{True}.
  \MacroNI{PREEMPTION\_REQUIREMENTS} should include the term 
  \Expr{(SubmitterGroup =?= RemoteGroup)} if a preemption policy that respects
  \Term{group quotas} is desired.
  Note that this setting does not
  influence other potential causes of preemption, such as startd
  \MacroNI{RANK}, or \MacroNI{PREEMPT} expressions.  See
  section \ref{sec:Disabling Preemption} for a general discussion of
  limiting preemption.

\label{param:PreemptionRequirementsStable} 
\item[\Macro{PREEMPTION\_REQUIREMENTS\_STABLE}]
  A boolean value that defaults to \Expr{True}, implying that all attributes
  utilized to define the \MacroNI{PREEMPTION\_REQUIREMENTS} variable will not
  change within a negotiation period time interval.
  If utilized attributes will change during the 
  negotiation period time interval, then set this variable to \Expr{False}. 

\label{param:PreemptionRank}
\item[\Macro{PREEMPTION\_RANK}]
  Resources that match a request are first sorted by
  \MacroNI{NEGOTIATOR\_PRE\_JOB\_RANK}.  If there are any ties in the
  rank of the top choice, the top resources are sorted by the
  user-supplied rank in the job ClassAd, then by
  \MacroNI{NEGOTIATOR\_POST\_JOB\_RANK}, then by
  \MacroNI{PREEMPTION\_RANK} (if the match would cause preemption and
  there are still any ties in the top choice).  \verb@MY@ refers to
  attributes of the machine ClassAd and \verb@TARGET@ refers to the
  job ClassAd.  This expression is used to rank machines that the job
  and the other negotiation expressions rank the same.  For example,
  if the job has no preference, it is usually preferable to preempt a
  job with a small \AdAttr{ImageSize} instead of a job with a large
  \AdAttr{ImageSize}.  The default is to rank all preemptable matches
  the same.  However, the negotiator will always prefer to match the
  job with an idle machine over a preemptable machine, if none of the
  other ranks express a preference between them.

\label{param:PreemptionRankStable}
\item[\Macro{PREEMPTION\_RANK\_STABLE}]
  A boolean value that defaults to \Expr{True}, implying that all attributes
  utilized to define the \MacroNI{PREEMPTION\_RANK} variable will not
  change within a negotiation period time interval.
  If utilized attributes will change during the 
  negotiation period time interval, then set this variable to \Expr{False}.

\label{param:NegotiatorSlotConstraint}
\item[\Macro{NEGOTIATOR\_SLOT\_CONSTRAINT}]
  An expression which constrains which machine ClassAds are fetched from the
  \Condor{collector} by the \Condor{negotiator} during a negotiation
  cycle.

\label{param:NegotiatorSlotPoolsizeConstraint}
\label{param:GroupDynamicMachConstraint}
\item[\Macro{NEGOTIATOR\_SLOT\_POOLSIZE\_CONSTRAINT} or
  \Macro{GROUP\_DYNAMIC\_MACH\_CONSTRAINT}]
  This optional expression specifies which machine ClassAds should be counted
  when computing the size of the pool.
  It applies both for group quota allocation and when there are no groups.
  The default is to count all machine ClassAds.
  When extra slots exist for special purposes,
  as, for example, suspension slots or file transfer slots,
  this expression can be used to inform the \Condor{negotiator} that 
  only normal slots should be counted when computing how big each group's 
  share of the pool should be.
 
  The name \MacroNI{NEGOTIATOR\_SLOT\_POOLSIZE\_CONSTRAINT} replaces
  \MacroNI{GROUP\_DYNAMIC\_MACH\_CONSTRAINT} as of HTCondor version 7.7.3.
  Using the older name causes a warning to be logged, although the
  behavior is unchanged.

\label{param:NegotiatorDebug}
\item[\Macro{NEGOTIATOR\_DEBUG}]
  This macro (and other settings related to debug logging in the negotiator) is
  described in section~\ref{param:SubsysDebug} as \MacroNI{<SUBSYS>\_DEBUG}.

\label{param:NegotiatorMaxTimePerSubmitter}
\item[\Macro{NEGOTIATOR\_MAX\_TIME\_PER\_SUBMITTER}]
  The maximum number of seconds
  the \Condor{negotiator} will spend with a submitter during one
  negotiation cycle.  Once this time limit has been reached, the
  \Condor{negotiator} will still finish its current pie spin, but it will skip
  over the submitter if subsequent pie spins are needed to dish out all
  of the available machines.  It defaults to one year.  See
  \MacroNI{NEGOTIATOR\_MAX\_TIME\_PER\_PIESPIN} for more information.

\label{param:NegotiatorMaxTimePerPieSpin}
\item[\Macro{NEGOTIATOR\_MAX\_TIME\_PER\_PIESPIN}]
  The maximum number of seconds the
  \Condor{negotiator} will spend with a submitter in one pie spin.
  A negotiation cycle is composed of at least one pie spin, possibly more,
  depending on whether there are still machines left over after
  computing fair shares and negotiating with each submitter.  By
  limiting the maximum length of a pie spin or the maximum time per
  submitter per negotiation cycle, the \Condor{negotiator} is protected
  against spending a long time talking to one submitter, for example someone
  with a very slow \Condor{schedd} daemon.
  But, this can result in unfair allocation of
  machines or some machines not being allocated at all.
  See section~\ref{sec:PieSlice} on page~\pageref{sec:PieSlice}
  for a description of a pie slice.

\label{param:NegotiatorMatchExprs}
\item[\Macro{NEGOTIATOR\_MATCH\_EXPRS}]
  A comma-separated list of macro names that are inserted as
  ClassAd attributes into matched job ClassAds.
  The attribute name in the ClassAd will be given the prefix
  \Attr{NegotiatorMatchExpr}, 
  if the macro name does not already begin with that.
  Example:

\footnotesize
\begin{verbatim}
  NegotiatorName = "My Negotiator"
  NEGOTIATOR_MATCH_EXPRS = NegotiatorName
\end{verbatim}
\normalsize

  As a result of the above configuration, jobs that are matched by this
  \Condor{negotiator} will contain the following attribute when they are 
  sent to the \Condor{startd}:

\footnotesize
\begin{verbatim}
  NegotiatorMatchExprNegotiatorName = "My Negotiator"
\end{verbatim}
\normalsize

  The expressions inserted by the \Condor{negotiator} may be useful in 
  \Condor{startd} policy expressions,
  when the \Condor{startd} belongs to multiple HTCondor pools.

\label{param:NegotiatorMatchlistCaching}
\item[\Macro{NEGOTIATOR\_MATCHLIST\_CACHING}]
  A boolean value that defaults to \Expr{True}.
  When \Expr{True}, it enables an optimization in the \Condor{negotiator}
  that works with auto clustering.
  In determining the sorted list of machines that a job might use,
  the job goes to the first machine off the top of the list. 
  If \MacroNI{NEGOTIATOR\_MATCHLIST\_CACHING} is \Expr{True},
  and if the next job is part of the same auto cluster,
  meaning that it is a very similar job,
  the \Condor{negotiator} will reuse the previous list of machines,
  instead of recreating the list from scratch.

  If matching grid resources, and the desire is for a
  given resource to potentially match multiple times per \Condor{negotiator}
  pass, \MacroNI{NEGOTIATOR\_MATCHLIST\_CACHING} should be \Expr{False}.
  See section~\ref{sec:Grid-Matchmaking} on page~\pageref{sec:Grid-Matchmaking}
  in the subsection on Advertising Grid Resources to HTCondor for an example.

\label{param:NegotiatorConsiderPreemption}
\item[\Macro{NEGOTIATOR\_CONSIDER\_PREEMPTION}]
  For expert users only. A boolean value (defaults to \Expr{True}),
  that when \Expr{False},
  can cause the \Condor{negotiator} to run
  faster and also have better spinning pie accuracy.
  \emph{Only set this to \Expr{False} if \Macro{PREEMPTION\_REQUIREMENTS}
  is \Expr{False},
  and if all \Condor{startd} rank expressions are \Expr{False}.}

\label{param:NegotiatorConsiderEarlyPreemption}
\item[\Macro{NEGOTIATOR\_CONSIDER\_EARLY\_PREEMPTION}]
  A boolean value that when \Expr{False} (the default),
  prevents the \Condor{negotiator} from matching jobs
  to claimed slots that cannot immediately be preempted
  due to \Macro{MAXJOBRETIREMENTTIME}.

\label{param:StartdAdReevalExpr}
\item[\Macro{STARTD\_AD\_REEVAL\_EXPR}]
  A boolean value evaluated in the context of each machine ClassAd within
  a negotiation cycle that determines whether the ClassAd from the
  \Condor{collector} is to replace the stashed ClassAd utilized during
  the previous negotiation cycle.
  When \Expr{True},
  the ClassAd from the \Condor{collector} does replace the stashed one.
  When not defined, the default value is to replace the stashed ClassAd
  if the stashed ClassAd's sequence number is older than its potential
  replacement.

\label{param:NegotiatorUpdateAfterCycle}
\item[\Macro{NEGOTIATOR\_UPDATE\_AFTER\_CYCLE}]
  A boolean value that defaults to \Expr{False}.
  When \Expr{True}, it will force the \Condor{negotiator} daemon to publish 
  an update to the \Condor{collector} at the end of every negotiation cycle.
  This is useful if monitoring statistics for the previous negotiation cycle. 

\label{param:NegotiatorReadConfigBeforeCycle}
\item[\Macro{NEGOTIATOR\_READ\_CONFIG\_BEFORE\_CYCLE}]
  A boolean value that defaults to \Expr{False}.
  When \Expr{True}, the \Condor{negotiator} will re-read the configuration
  prior to beginning each negotiation cycle.  
  Note that this operation will update configured behaviors such as 
  concurrency limits, but not data structures
  constructed during a full reconfiguration,
  such as the group quota hierarchy.
  A full reconfiguration, for example as accomplished with \Condor{reconfig},
  remains the best way to 
  guarantee that all \Condor{negotiator} configuration is completely updated.

\label{param:NameLimit}
\item[\Macro{<NAME>\_LIMIT}]
  An integer value that defines the amount of resources available for
  jobs which declare that they use some consumable resource 
  as described in section~\ref{sec:Concurrency-Limits}. 
  \MacroNI{<Name>} is a string invented to uniquely describe the resource.

\label{param:ConcurrencyLimitDefault}
\item[\Macro{CONCURRENCY\_LIMIT\_DEFAULT}]
  An integer value that describes the number of resources available for
  any resources that are not explicitly named defined with the
  configuration variable \MacroNI{<NAME>\_LIMIT}.
  If not defined, no limits are set for resources not explicitly identified
  using \MacroNI{<NAME>\_LIMIT}.

\label{param:ConcurrencyLimitDefaultName}
\item[\Macro{CONCURRENCY\_LIMIT\_DEFAULT\_<NAME>}]
	If set, this defines a default concurrency limit for all resources
that start with \MacroNI{<NAME>.}

\end{description}

The following configuration macros affect negotiation for group users.
\begin{description}

\label{param:GroupNames}
\item[\Macro{GROUP\_NAMES}]
  A comma-separated list of the recognized group names, case insensitive.
  If undefined (the default), group support is disabled.
  Group names must not conflict with any user names.
  That is, if there is a \verb@physics@ group, there may not be
  a \verb@physics@ user.
  Any group that is defined here must also have a quota,
  or the group will be ignored. Example: 
  \begin{verbatim}
    GROUP_NAMES = group_physics, group_chemistry 
  \end{verbatim}

\label{param:GroupQuotaGroupname}
\item[\Macro{GROUP\_QUOTA\_<groupname>}]
  A floating point value to represent a static quota specifying
  an integral number of machines for the hierarchical group
  identified by \Expr{<groupname>}.
  It is meaningless to specify a non integer value, 
  since only integral numbers of machines can be allocated.
  Example:
  \begin{verbatim}
    GROUP_QUOTA_group_physics = 20
    GROUP_QUOTA_group_chemistry = 10
  \end{verbatim}
  When both static and dynamic quotas are defined for a specific group,
  the static quota is used and the dynamic quota is ignored. 

\label{param:GroupQuotaDynamicGroupname}
\item[\Macro{GROUP\_QUOTA\_DYNAMIC\_<groupname>}]
  A floating point value in the range 0.0 to 1.0, inclusive,
  representing a fraction of a pool's machines (slots) set as
  a dynamic quota for the hierarchical group identified by \Expr{<groupname>}.
  For example, the following
  specifies that a quota of 25\% of the total machines are
  reserved for members of the group\_biology group.
  \begin{verbatim}
  	GROUP_QUOTA_DYNAMIC_group_biology = 0.25
  \end{verbatim}
  The group name must be specified in the \Macro{GROUP\_NAMES} list.
  \MoreTodo
  %HTCondor does not verify that the
  %quota value is reasonable, nor does HTCondor verify that all specified
  %quotas are consistent.  This parameter is evaluated whenever HTCondor
  %negotiates for the group.  When both are defined, static quotas
  %supersede dynamic quotas.

\label{param:GroupPrioFactorGroupname}
\item[\Macro{GROUP\_PRIO\_FACTOR\_<groupname>}]
  A floating point value greater than or equal to 1.0 to specify the
  default user priority factor for \verb@<groupname>@. 
  The group name must also be specified in the \MacroNI{GROUP\_NAMES} list.
  \MacroNI{GROUP\_PRIO\_FACTOR\_<groupname>} is evaluated when
  the negotiator first negotiates for the user as a member of the group.
  All members of the group inherit the default priority factor
  when no other value is present.
  For example, the following setting
  specifies that all members of the group named \verb@group_physics@
  inherit a default user priority factor of 2.0:
  \begin{verbatim}
    GROUP_PRIO_FACTOR_group_physics = 2.0
  \end{verbatim}

\label{param:GroupAutoregroup}
\item[\Macro{GROUP\_AUTOREGROUP}]
  A boolean value (defaults to \Expr{False}) that when \Expr{True},
  causes users who submitted to a specific group to
  also negotiate a second time with the \Expr{<none>} group,
  to be considered with the independent job submitters. 
  This allows group submitted jobs to be matched with idle machines
  even if the group is over its quota.  The user name that is
  used for accounting and prioritization purposes is still
  the group user as specified by \AdAttr{AccountingGroup}
  in the job ClassAd.

\label{param:GroupAutoregroupGroupname}
\item[\Macro{GROUP\_AUTOREGROUP\_<groupname>}]
  This is the same as \MacroNI{GROUP\_AUTOREGROUP}, but it is settable
  on a per-group basis.  If no value is specified for a given group,
  the default behavior is determined by \MacroNI{GROUP\_AUTOREGROUP},
  which in turn defaults to \Expr{False}.

\label{param:GroupAcceptSurplus}
\item[\Macro{GROUP\_ACCEPT\_SURPLUS}]
  A boolean value that, when \Expr{True}, specifies that groups should be 
  allowed to use more than their configured quota when there is not enough 
  demand from other groups to use all of the available machines.
  The default value is \Expr{False}. 

\label{param:GroupAcceptSurplusGroupname}
\item[\Macro{GROUP\_ACCEPT\_SURPLUS\_<groupname>}]
  A boolean value applied as a group-specific version of 
  \MacroNI{GROUP\_ACCEPT\_SURPLUS}.
  When not specified, the value of \MacroNI{GROUP\_ACCEPT\_SURPLUS} applies
  to the named group.

\label{param:GroupQuotaRoundRobinRate}
\item[\Macro{GROUP\_QUOTA\_ROUND\_ROBIN\_RATE}]
  The maximum sum of weighted slots that should be handed out to an individual 
  submitter in each iteration within a negotiation cycle. 
  If slot weights are not being used by the \Condor{negotiator},
  as specified by \Expr{NEGOTIATOR\_USE\_SLOT\_WEIGHTS = False},
  then this value is just the (unweighted) number of slots. 
  The default value is a very big number, effectively infinite. 
  Setting the value to a number smaller than the size of the pool 
  can help avoid starvation. 
  An example of the starvation problem is when there are a subset of machines 
  in a pool with large memory,
  and there are multiple job submitters who desire all of these machines.
  Normally, HTCondor will decide how much of the full pool each person should get,
  and then attempt to hand out that number of resources to each person. 
  Since the big memory machines are only a subset of pool, 
  it may happen that they are all given to the first person contacted, 
  and the remainder requiring large memory machines get nothing. 
  Setting \MacroNI{GROUP\_QUOTA\_ROUND\_ROBIN\_RATE} to a value that is small 
  compared to the size of subsets of machines will reduce starvation at the 
  cost of possibly slowing down the rate at which resources are allocated.

\label{param:GroupQuotaMaxAllocationRounds}
\item[\Macro{GROUP\_QUOTA\_MAX\_ALLOCATION\_ROUNDS}]
  An integer that specifies the maximum number of times within 
  one negotiation cycle the \Condor{negotiator} will calculate how many 
  slots each group deserves and attempt to allocate them. 
  The default value is 3. 
  The reason it may take more than one round is that some groups may not 
  have jobs that match some of the available machines, 
  so some of the slots that were withheld for those groups 
  may not get allocated in any given round.

\label{param:NegotiatorUseSlotWeights}
\item[\Macro{NEGOTIATOR\_USE\_SLOT\_WEIGHTS}]
  A boolean value with a default of \Expr{True}. 
  When \Expr{True}, the \Condor{negotiator} pays attention to the
  machine ClassAd attribute \Attr{SlotWeight}. 
  When \Expr{False}, each slot effectively has a weight of 1.

\label{param:NegotiatorUseWeightedDemand}
\item[\Macro{NEGOTIATOR\_USE\_WEIGHTED\_DEMAND}]
  A boolean value with a default of \Expr{False}. 
  When \Expr{False}, the behavior is the same as for HTCondor versions
  prior to 7.9.6.
  If \Expr{True}, when the \Condor{schedd} advertises \Attr{IdleJobs}
  in the submitter ClassAd,
  the number of idle jobs in the queue for that submitter,
  it will also advertise the total number of requested cores across
  all idle jobs from that submitter, \Attr{WeightedIdleJobs}.
  If partitionable slots are being used,
  and if hierarchical group quotas are used,
  and if any hierarchical group quotas set \Macro{GROUP\_ACCEPT\_SURPLUS}
  to \Expr{True},
  and if configuration variable \Macro{SlotWeight} is set to 
  the number of cores,
  then setting this configuration variable to \Expr{True} allows
  the amount of surplus allocated to each group to be calculated correctly. 
   

\label{param:GroupSortExpr}
\item[\Macro{GROUP\_SORT\_EXPR}]
  \fbox{This definition is not complete.}
  A ClassAd expression that controls the order in which the 
  \Condor{negotiator} considers groups when allocating resources.
  The default value is set such that group \Expr{<none>} always
  goes last when considering hierarchical group quotas.
%  The default value uses the expression
%\footnotesize
%\begin{verbatim}
%\end{verbatim}
%\normalsize

\label{param:NegotiatorAllowQuotaOversubscription}
\item[\Macro{NEGOTIATOR\_ALLOW\_QUOTA\_OVERSUBSCRIPTION}]
  A boolean value that defaults to \Expr{True}.
  When \Expr{True}, 
  the behavior of resource allocation when considering groups is
  more like it was in the 7.4 stable series of HTCondor.
  In implementation, when \Expr{True}, the static quotas of subgroups
  will \emph{not} be scaled when the sum of these static quotas of subgroups
  sums to more than the group's static quota.
  This behavior is desirable when using static quotas,
  unless the sum of subgroup quotas is considerably less than the group's
  quota, as scaling is currently based on the number of machines available,
  not assigned quotas (for static quotas).

\end{description}

% entire section commented out Nov 2005
%%%%%%%%%%%%%%%%%%%%%%%%%%%%%%%%%%%%%%%%%%%%%%%%%%%%%%%%%%%%%%%%%%%%%%%%%%%
% \subsection{\label{sec:Eventd-Config-File-Entries}
% \condor{eventd} Configuration File Entries}
%%%%%%%%%%%%%%%%%%%%%%%%%%%%%%%%%%%%%%%%%%%%%%%%%%%%%%%%%%%%%%%%%%%%%%%%%%%

% \index{configuration!condor\_eventd configuration variables}
% These macros affect the HTCondor Event daemon.  See
% section~\ref{sec:EventD} on page~\pageref{sec:EventD} for an
% introduction.  The eventd is not included in the main HTCondor binary
% distribution or installation procedure.  It can be installed as a
% contrib module.
% 
% \begin{description}
  
% \label{param:EventList}
% \item[\Macro{EVENT\_LIST}]
% List of macros
% which define events to be managed by the event daemon.

% \label{param:EventdCapInfo}
% \item[\Macro{EVENTD\_CAPACITY\_INFO}]
% Configures the bandwidth limits used when scheduling job checkpoint
% transfers before \MacroNI{SHUTDOWN} events.
% The \MacroNI{EVENTD\_CAPACITY\_INFO} file has the same
% format as the \MacroNI{NETWORK\_CAPACITY\_INFO} file, described in
% section~\ref{sec:Bandwidth-Alloc-Capinfo}.

% \label{param:EventdRouteInfo}
% \item[\Macro{EVENTD\_ROUTING\_INFO}]
% Configures the network routing information used when scheduling job
% checkpoint transfers before \MacroNI{SHUTDOWN} events.
% The \MacroNI{EVENTD\_ROUTING\_INFO} file has the same
% format as the \MacroNI{NETWORK\_ROUTING\_INFO} file, described in
% section~\ref{sec:Bandwidth-Alloc-Routes}.

% \label{param:EventdInterval}
% \item[\Macro{EVENTD\_INTERVAL}]
% The number
% of seconds between collector queries to determine pool
% state.  The default is 15 minutes (300 seconds).

% \label{param:EventdMaxPreparation}
% \item[\Macro{EVENTD\_MAX\_PREPARATION}]
%  The number of minutes before a
% scheduled event when the eventd should start periodically querying the
% collector.  If 0 (default), the eventd always polls.

% \label{param:EventdShutdownSlowStartInterval}
% \item[\Macro{EVENTD\_SHUTDOWN\_SLOW\_START\_INTERVAL}]
% The number of seconds
% between each machine startup after a shutdown event.  The default is 0.

% \label{param:EventdShutdownCleanupInterval}
% \item[\Macro{EVENTD\_SHUTDOWN\_CLEANUP\_INTERVAL}]
% The number of seconds
% between each check for old shutdown configurations in the pool.  The default
% is one hour (3600 seconds).

% \end{description}

%%%%%%%%%%%%%%%%%%%%%%%%%%%%%%%%%%%%%%%%%%%%%%%%%%%%%%%%%%%%%%%%%%%%%%
\subsection{\label{sec:Procd-Config-File-Entries}\condor{procd}
Configuration File Macros}
%%%%%%%%%%%%%%%%%%%%%%%%%%%%%%%%%%%%%%%%%%%%%%%%%%%%%%%%%%%%%%%%%%%%%%

\begin{description}

\label{param:UseProcd}
\item[\Macro{USE\_PROCD}]
  This boolean variable
  determines whether the \Condor{procd} will be used for
  managing process families. If the \Condor{procd} is not used, each
  daemon will run the process family tracking logic on its own. Use of
  the \Condor{procd} results in improved scalability because only one
  instance of this logic is required. The \Condor{procd} is required
  when using privilege separation (see Section~\ref{sec:PrivSep}) or
  group ID-based process tracking (see
  Section~\ref{sec:GroupTracking}). In either of these cases, the
  \MacroNI{USE\_PROCD} setting will be ignored and a \Condor{procd} will
  always be used. 
  By default, the \Condor{master} will start a
  \Condor{procd} that all other daemons that need process family tracking 
  will use.
  A daemon that uses the \Condor{procd} will start a \Condor{procd} for
  use by itself and all of its child daemons.

\label{param:ProcdMaxSnapshotInterval}
\item[\Macro{PROCD\_MAX\_SNAPSHOT\_INTERVAL}]
  This setting determines the maximum time that the \Condor{procd} will
  wait between probes of the system for information about the process
  families it is tracking.

\label{param:ProcdLog}
\item[\Macro{PROCD\_LOG}]
  Specifies a log file for the \Condor{procd} to use.
  Note that by design, the \Condor{procd} does not
  include most of the other logic that is shared amongst the various
  HTCondor daemons. This is because the \Condor{procd} is a component of
  the PrivSep Kernel (see Section~\ref{sec:PrivSep} for more information
  regarding privilege separation). This means that the \Condor{procd}
  does not include the normal HTCondor logging subsystem, and thus 
  multiple debug levels are not supported.
  \MacroNI{PROCD\_LOG} defaults to \File{\$(LOG)/ProcLog}.
  Note that enabling \Dflag{PROCFAMILY} in the debug level for any
  other daemon will cause it to log all interactions with the
  \Condor{procd}.


\label{param:MaxProcdLog}
\item[\Macro{MAX\_PROCD\_LOG}]
  Controls the maximum length in bytes to which the \Condor{procd}
  log will be allowed to grow.  The log file will grow to the
  specified length, then be saved to a file with the suffix
  \File{.old}.  The \File{.old}
  file is overwritten each time the log is saved, thus the maximum
  space devoted to logging will be twice the
  maximum length of this log file.  A value of 0 specifies that the
  file may grow without bounds.  The default is 10 Mbyte.

\label{param:ProcdAddress}
\item[\Macro{PROCD\_ADDRESS}]
  This specifies
  the address that the \Condor{procd} will use to receive requests
  from other HTCondor daemons. On Unix, this should point to a file system
  location that can be used for a named pipe. On Windows, named pipes
  are also used but they do not exist in the file system. The default
  setting therefore depends on the platform: 
  \verb@$(LOCK)/procd_pipe@ on Unix and
%  \Code{$\backslash$$\backslash$.$\backslash$pipe$\backslash$procd\_pipe}
  \verb@\\.\pipe\procd_pipe@ on Windows.

\label{param:UseGIDProcessTracking}
\item[\Macro{USE\_GID\_PROCESS\_TRACKING}]
  A boolean value that defaults to \Expr{False}.
  When \Expr{True}, a job's initial process is assigned a dedicated GID
  which is further used by the \Condor{procd} to reliably track all
  processes associated with a job.
  When \Expr{True}, values for \MacroNI{MIN\_TRACKING\_GID} and 
  \MacroNI{MAX\_TRACKING\_GID} must also be set, or HTCondor will abort,
  logging an error message.
  See section~\ref{sec:GroupTracking} on page~\pageref{sec:GroupTracking} for 
  a detailed description.

\label{param:MinTrackingGID}
\item[\Macro{MIN\_TRACKING\_GID}]
  An integer value, that together with \MacroNI{MAX\_TRACKING\_GID}
  specify a range of GIDs to be assigned on a per slot basis for
  use by the \Condor{procd} in tracking processes associated with a job.
  See section~\ref{sec:GroupTracking} on page~\pageref{sec:GroupTracking} for 
  a detailed description.

\label{param:MaxTrackingGID}
\item[\Macro{MAX\_TRACKING\_GID}]
  An integer value, that together with \MacroNI{MIN\_TRACKING\_GID}
  specify a range of GIDs to be assigned on a per slot basis for
  use by the \Condor{procd} in tracking processes associated with a job.
  See section~\ref{sec:GroupTracking} on page~\pageref{sec:GroupTracking} for 
  a detailed description.

\label{param:BaseCgroup}
\item[\Macro{BASE\_CGROUP}]
  The path to the directory used as the virtual file system for the
  implementation of Linux kernel cgroups.
  This variable has no default value,
  and if not defined, cgroup tracking will not be used.
  See section~\ref{sec:CGroupTracking} on page~\pageref{sec:CGroupTracking} for 
  a description of cgroup-based process tracking.

\end{description}

%%%%%%%%%%%%%%%%%%%%%%%%%%%%%%%%%%%%%%%%%%%%%%%%%%%%%%%%%%%%%%%%%%%%%%
\subsection{\label{sec:Credd-Config-File-Entries}\condor{credd}
Configuration File Macros}
%%%%%%%%%%%%%%%%%%%%%%%%%%%%%%%%%%%%%%%%%%%%%%%%%%%%%%%%%%%%%%%%%%%%%%
 
\index{HTCondor daemon!condor\_credd@\Condor{credd}}
\index{daemon!condor\_credd@\Condor{credd}}
\index{condor\_credd daemon}
\index{configuration!condor\_credd configuration variables}
These macros affect the \Condor{credd}.

\begin{description}

\label{param:CreddHost}
\item[\Macro{CREDD\_HOST}]
  The host name of the machine running the \Condor{credd} daemon.

\label{param:CreddCacheLocally}
\item[\Macro{CREDD\_CACHE\_LOCALLY}]
  A boolean value that defaults to \Expr{False}.
  When \Expr{True}, the first successful password fetch operation to the
  \Condor{credd} daemon causes the password to be stashed in a local, 
  secure password store.
  Subsequent uses of that password do not require
  communication with the \Condor{credd} daemon.
  
\label{param:SkipWindowsLogonNetwork}
\item[\Macro{SKIP\_WINDOWS\_LOGON\_NETWORK}]
  A boolean value that defaults to \Expr{False}.
  When \Expr{True}, Windows authentication skips trying authentication
  with the \Expr{LOGON\_NETWORK} method first, 
  and attempts authentication  with \Expr{LOGON\_INTERACTIVE} method. 
  This can be useful if many authentication failures are noticed, 
  potentially leading to users getting locked out.

\end{description}


%%%%%%%%%%%%%%%%%%%%%%%%%%%%%%%%%%%%%%%%%%%%%%%%%%%%%%%%%%%%%%%%%%%%%%%%%%%
\subsection{\label{sec:Gridmanager-Config-File-Entries}\condor{gridmanager}
Configuration File Entries}
%%%%%%%%%%%%%%%%%%%%%%%%%%%%%%%%%%%%%%%%%%%%%%%%%%%%%%%%%%%%%%%%%%%%%%%%%%%

\index{configuration!condor\_gridmanager configuration variables}
These macros affect the \Condor{gridmanager}.
\begin{description}

\label{param:GridmanagerLog}
\item[\Macro{GRIDMANAGER\_LOG}]
  Defines the path and file name for the log of the \Condor{gridmanager}. 
  The owner of the file is the \Login{condor} user.

\label{param:GridmanagerCheckproxyInterval}
\item[\Macro{GRIDMANAGER\_CHECKPROXY\_INTERVAL}]
  The number of seconds
  between checks for an updated X509 proxy credential. The default
  is 10 minutes (600 seconds).

\label{param:GridmanagerProxyRefreshTime}
\item[\Macro{GRIDMANAGER\_PROXY\_REFRESH\_TIME}]
  For GRAM jobs, the \Condor{gridmanager} will not forward a refreshed
  proxy until the lifetime left for the proxy on the remote machine
  falls below this value.
  The value is in seconds and the default is 21600 (6 hours).

\label{param:GridmanagerMinimumProxyTime}
\item[\Macro{GRIDMANAGER\_MINIMUM\_PROXY\_TIME}]
  The minimum number of
  seconds before expiration of the X509 proxy credential for the
  gridmanager to continue operation. If seconds until expiration is
  less than this number, the gridmanager will shutdown and wait for
  a refreshed proxy credential. The default is 3 minutes (180 seconds).

\label{param:HoldJobIfCredentialExpires}
\item[\Macro{HOLD\_JOB\_IF\_CREDENTIAL\_EXPIRES}]
  True or False.  Defaults to True.
  If True, and for grid universe jobs only,
  HTCondor-G will place a job on hold
  \MacroNI{GRIDMANAGER\_MINIMUM\_PROXY\_TIME} seconds
  before the proxy expires.
  If False,
  the job will stay in the last known state,
  and HTCondor-G will periodically check to see if the job's proxy has been
  refreshed, at which point management of the job will resume.

\label{param:GridmanagerContactScheddDelay}
\item[\Macro{GRIDMANAGER\_CONTACT\_SCHEDD\_DELAY}]
  The minimum number of
  seconds between connections to the \Condor{schedd}. The default is 5 seconds.

\label{param:GridmanagerJobProbeInterval}
\item[\Macro{GRIDMANAGER\_JOB\_PROBE\_INTERVAL}]
  The number of seconds between
  active probes for the status of a submitted job.
  The default is 1 minute (60 seconds).
  Intervals specific to grid types can be set by appending the
  name of the grid type to the configuration variable name, as the example
  \begin{verbatim}
  GRIDMANAGER_JOB_PROBE_INTERVAL_GT5 = 300
  \end{verbatim}

\label{param:GridmanagerJobProbeRate}
\item[\Macro{GRIDMANAGER\_JOB\_PROBE\_RATE}]
  The maximum number of job status probes per second that will be
  issued to a given remote resource.
  The time between status probes for individual jobs may be lengthened
  beyond \Macro{GRIDMANAGER\_JOB\_PROBE\_INTERVAL} to enforce this rate.
  The default is 5 probes per second.
  Rates specific to grid types can be set by appending the
  name of the grid type to the configuration variable name, as the example
  \begin{verbatim}
  GRIDMANAGER_JOB_PROBE_RATE_GT5 = 15
  \end{verbatim}

\label{param:GridmanagerResourceProbeInterval}
\item[\Macro{GRIDMANAGER\_RESOURCE\_PROBE\_INTERVAL}]
  When a resource appears to be down, how often (in seconds) the
  \Condor{gridmanager}
  should ping it to test if it is up again.

\label{param:GridmanagerResourceProbeDelay}
\item[\Macro{GRIDMANAGER\_RESOURCE\_PROBE\_DELAY}]
  The number of seconds
  between pings of a remote resource that is currently down.
  The default is 5 minutes (300 seconds).

\label{param:GridmanagerEmptyResourceDelay}
\item[\Macro{GRIDMANAGER\_EMPTY\_RESOURCE\_DELAY}]
  The number of seconds
  that the \Condor{gridmanager} retains information about a grid
  resource, once the \Condor{gridmanager} has no active jobs
  on that resource.
  An active job is a grid universe job that is in the queue,
  for which \Attr{JobStatus} is anything other than Held. 
  Defaults to 300 seconds.

\label{param:GridmanagerMaxSubmittedJobsPerResource}
\item[\Macro{GRIDMANAGER\_MAX\_SUBMITTED\_JOBS\_PER\_RESOURCE}]
  An integer value that limits the number of jobs
  that a \Condor{gridmanager} daemon will submit to a resource.
  A comma-separated list of pairs that follows this integer limit
  will specify limits for specific remote resources.
  Each pair is a host name and the job limit for that host.
  Consider the example:
  \footnotesize
  \begin{verbatim}
  GRIDMANAGER_MAX_SUBMITTED_JOBS_PER_RESOURCE = 200, foo.edu, 50, bar.com, 100
  \end{verbatim}
  \normalsize
  In this example, all resources have a job limit of 200, except foo.edu,
  which has a limit of 50, and bar.com, which has a limit of 100.

  Limits specific to grid types can be set by appending the
  name of the grid type to the configuration variable name, as the example
  \begin{verbatim}
  GRIDMANAGER_MAX_SUBMITTED_JOBS_PER_RESOURCE_CREAM = 300
  \end{verbatim}
  In this example, the job limit for all CREAM resources is 300.
  Defaults to 1000.

\label{param:GridmanagerMaxJobmanagersPerResource}
\item[\Macro{GRIDMANAGER\_MAX\_JOBMANAGERS\_PER\_RESOURCE}]
  For grid jobs of type \SubmitCmd{gt2}, limits the number of globus-job-manager
  processes that the \Condor{gridmanager} lets run at a time on
  the remote head node. Allowing too many globus-job-managers to run
  causes severe load on the head note, possibly making it
  non-functional.
  This number may be exceeded if it is reduced through the use
  of \Condor{reconfig} while the \Condor{gridmanager} is running,
  or if some globus-job-managers take a few extra seconds to exit.
  The value 0 means there is no limit. The default value is 10.

\label{param:Gahp}
\item[\Macro{GAHP}]
  The full path to the binary of the GAHP server.
  This configuration variable is no longer used.
  Use \MacroNI{GT2\_GAHP} at section~\ref{param:GT2GAHP} instead.

\label{param:GahpArgs}
\item[\Macro{GAHP\_ARGS}]
  Arguments to be passed to the GAHP server.
  This configuration variable is no longer used.

\label{param:GridmanagerGahpCallTimeout}
\item[\Macro{GRIDMANAGER\_GAHP\_CALL\_TIMEOUT}]
  The number of seconds after
  which a pending GAHP command should time out. 
  The default is 5 minutes (300 seconds).

\label{param:GridmanagerMaxPendingRequests}
\item[\Macro{GRIDMANAGER\_MAX\_PENDING\_REQUESTS}]
  The maximum number of GAHP
  commands that can be pending at any time. The default is 50.

\label{param:GridmanagerConnectFailureRetryCount}
\item[\Macro{GRIDMANAGER\_CONNECT\_FAILURE\_RETRY\_COUNT}]
  The number of times
  to retry a command that failed due to a timeout or a failed connection.
  The default is 3.

\label{param:GridmanagerGlobusCommitTimeout}
\item[\Macro{GRIDMANAGER\_GLOBUS\_COMMIT\_TIMEOUT}]
  The duration, in seconds, of the
  two phase commit timeout to Globus for gt2 jobs only.
  This maps directly to the \texttt{two\_phase} setting in the Globus RSL.

% configuration variable introduced into HTCondor due to Globus a GAHP
% bug.  Users with Globus where this bug has been fixed should never
% use nor even know about this configuration variable.
% \label{param:GridmanagerRestartOnAnyDownResources} 
% \item[\Macro{GRIDMANAGER\_RESTART\_ON\_ANY\_DOWN\_RESOURCES}]
% The current GAHP server can transition into a state where it
% cannot contact remote machines,
% making the machines appear to be down.
% Restarting the GAHP server fixes the problem.
% This parameter controls whether one machine or
% all machine need to appear to be down to trigger a restart of the GAHP
% server.

% configuration variable introduced into HTCondor due to Globus a GAHP
% bug.  Users with Globus where this bug has been fixed should never
% use nor even know about this configuration variable.
%\label{param:GridmanagerMaxTimeDownResources} 
%\item[\Macro{GRIDMANAGER\_MAX\_TIME\_DOWN\_RESOURCES}]
%Related to GRIDMANAGER\_RESTART\_ON\_ANY\_DOWN\_RESOURCES,
%this configuration variable defines how long (in seconds) one or
%more machines need to be down (or just appear to be) to trigger a restart
%of the GAHP server.

\label{param:GlobusGatekeeperTimeout}
\item[\Macro{GLOBUS\_GATEKEEPER\_TIMEOUT}]
  The number of seconds after which if a gt2 grid
  universe job fails to ping the gatekeeper, the job will be put on hold.
  Defaults to 5 days (in seconds).

\label{param:GramVersionDetection}
\item[\Macro{GRAM\_VERSION\_DETECTION}]
  A boolean value that defaults to \Expr{True}.
  When \Expr{True}, the \Condor{gridmanager} treats grid types
  \Expr{gt2} and \Expr{gt5} identically, and queries each server to
  determine which protocol it is using.
  When \Expr{False}, the \Condor{gridmanager} trusts the grid type
  provided in job attribute \Attr{GridResource}, and treats the server
  accordingly.
  Beware that identifying a \Expr{gt2} server as \Expr{gt5} can result in
  overloading the server, if a large number of jobs are submitted.

\label{param:BatchGahpCheckStatusAttempts}
\item[\Macro{BATCH\_GAHP\_CHECK\_STATUS\_ATTEMPTS}]
  The number of times a failed status command issued to the
  \Prog{batch\_gahp} should be retried. These retries allow the
  \Condor{gridmanager} to tolerate short-lived failures of the underlying
  batch system. The default value is 5.

\label{param:CGAHPLog}
\item[\Macro{C\_GAHP\_LOG}]
  The complete path and file name of the HTCondor GAHP server's log.
  There is no default value. The expected location as defined
  in the example configuration is \File{/temp/CGAHPLog.\MacroUNI{USERNAME}}.

\label{param:MaxCGAHPLog}
\item[\Macro{MAX\_C\_GAHP\_LOG}]
  The maximum size of the \MacroNI{C\_GAHP\_LOG}.

\label{param:CGAHPWorkerThreadLog}
\item[\Macro{C\_GAHP\_WORKER\_THREAD\_LOG}]
  The complete path and file name of the HTCondor GAHP worker process' log.
  There is no default value.
  The expected location as defined in the example configuration is
  \File{/temp/CGAHPWorkerLog.\MacroUNI{USERNAME}}.

\label{param:CGAHPContactScheddDelay}
\item[\Macro{C\_GAHP\_CONTACT\_SCHEDD\_DELAY}]
  The number of seconds that the \Condor{C-gahp} daemon waits between
  consecutive connections to the remote \Condor{schedd} in order to
  send batched sets of commands to be executed on that remote \Condor{schedd}
  daemon.
  The default value is 5.

\label{param:GLITELocation}
\item[\Macro{GLITE\_LOCATION}]
  The complete path to the directory containing the Glite software.
  There is no default value. The expected location as given
  in the example configuration is \File{\MacroUNI{LIB}/glite}.
  The necessary Glite software is included with HTCondor,
  and is required for pbs and lsf jobs.

\label{param:CondorGAHP}
\item[\Macro{CONDOR\_GAHP}]
  The complete path and file name of the HTCondor GAHP executable.
  There is no default value. The expected location as given
  in the example configuration is \File{\MacroUNI{SBIN}/condor\_c-gahp}.

\label{param:EC2GAHP}
\item[\Macro{EC2\_GAHP}]
  The complete path and file name of the EC2 GAHP executable.
  There is no default value. The expected location as given
  in the example configuration is \File{\MacroUNI{SBIN}/ec2\_gahp}.

\label{param:GT2GAHP}
\item[\Macro{GT2\_GAHP}]
  The complete path and file name of the GT2 GAHP executable.
  There is no default value. The expected location as given
  in the example configuration is \File{\MacroUNI{SBIN}/gahp\_server}.

\label{param:BatchGAHP}
\item[\Macro{BATCH\_GAHP}]
  The complete path and file name of the batch GAHP executable,
  to be used for PBS, LSF, SGE, and similar batch systems.
  The default location is
  \File{\MacroUNI{GLITE\_LOCATION}/bin/batch\_gahp}.

\label{param:PBSGAHP}
\item[\Macro{PBS\_GAHP}]
  The complete path and file name of the PBS GAHP executable.
  The use of the configuration variable \MacroNI{BATCH\_GAHP}
  is preferred and encouraged,
  as this variable may no longer be supported in a future
  version of HTCondor.
  A value given with this configuration variable will override
  a value specified by \MacroNI{BATCH\_GAHP},
  and the value specified by \MacroNI{BATCH\_GAHP} is the default
  if this variable is not defined.

\label{param:LSFGAHP}
\item[\Macro{LSF\_GAHP}]
  The complete path and file name of the LSF GAHP executable.
  The use of the configuration variable \MacroNI{BATCH\_GAHP}
  is preferred and encouraged,
  as this variable may no longer be supported in a future
  version of HTCondor.
  A value given with this configuration variable will override
  a value specified by \MacroNI{BATCH\_GAHP},
  and the value specified by \MacroNI{BATCH\_GAHP} is the default
  if this variable is not defined.

\label{param:UnicoreGAHP}
\item[\Macro{UNICORE\_GAHP}]
  The complete path and file name of the
  wrapper script that invokes the Unicore GAHP executable.
  There is no default value. The expected location as given
  in the example configuration is \File{\MacroUNI{SBIN}/unicore\_gahp}.

\label{param:NorduGridGAHP}
\item[\Macro{NORDUGRID\_GAHP}]
  The complete path and file name of the
  wrapper script that invokes the NorduGrid GAHP executable.
  There is no default value. The expected location as given
  in the example configuration is \File{\MacroUNI{SBIN}/nordugrid\_gahp}.

\label{param:CREAMGAHP}
\item[\Macro{CREAM\_GAHP}]
  The complete path and file name of the CREAM GAHP executable.
  There is no default value.
  The expected location as given in the example configuration is
  \File{\MacroUNI{SBIN}/cream\_gahp}.

\label{param:DeltacloudGAHP}
\item[\Macro{DELTACLOUD\_GAHP}]
  The complete path and file name of the Deltacloud GAHP executable.
  There is no default value.
  The expected location as given in the example configuration is
  \File{\MacroUNI{SBIN}/deltacloud\_gahp}.

\label{param:SGEGAHP}
\item[\Macro{SGE\_GAHP}]
  The complete path and file name of the SGE GAHP executable.
  The use of the configuration variable \MacroNI{BATCH\_GAHP}
  is preferred and encouraged,
  as this variable may no longer be supported in a future
  version of HTCondor.
  A value given with this configuration variable will override
  a value specified by \MacroNI{BATCH\_GAHP},
  and the value specified by \MacroNI{BATCH\_GAHP} is the default
  if this variable is not defined.

\end{description}

%%%%%%%%%%%%%%%%%%%%%%%%%%%%%%%%%%%%%%%%%%%%%%%%%%%%%%%%%%%%%%%%%%%%%%%%%%
\subsection{\label{sec:JobRouter-Config-File-Entries}\condor{job\_router}
Configuration File Entries}
%%%%%%%%%%%%%%%%%%%%%%%%%%%%%%%%%%%%%%%%%%%%%%%%%%%%%%%%%%%%%%%%%%%%%%%%%%%

\index{configuration!condor\_job\_router configuration variables}
These macros affect the \Condor{job\_router} daemon.

\begin{description}

\label{param:JobRouterDefaults}
\item[\Macro{JOB\_ROUTER\_DEFAULTS}]
  Defined by a single ClassAd in New ClassAd syntax, 
  used to provide default values for all routes in the \Condor{job\_router}
  daemon's routing table.
  Where an attribute is set outside of these defaults,
  that attribute value takes precedence.

\label{param:JobRouterEntries}
\item[\Macro{JOB\_ROUTER\_ENTRIES}]
  Specification of the job routing table.  It is a list of ClassAds,
  in New ClassAd syntax,
  where each individual ClassAd is surrounded by square brackets,
  and the ClassAds are separated from each other by spaces.
  Each ClassAd describes one entry in the routing table,
  and each describes a site that jobs may be routed to.

  A \Condor{reconfig} command causes the \Condor{job\_router} daemon
  to rebuild the routing table.
  Routes are distinguished by a routing table entry's ClassAd attribute
  \Attr{Name}.
  Therefore, a \Attr{Name} change in an existing route has the potential to
  cause the inaccurate reporting of routes.

  Instead of setting job routes using this configuration variable,
  they may be read from an
  external source using the \MacroNI{JOB\_ROUTER\_ENTRIES\_FILE} or
  be dynamically generated by an external program via the
  \MacroNI{JOB\_ROUTER\_ENTRIES\_CMD} configuration variable.


\label{param:JobRouterEntriesFile}
\item[\Macro{JOB\_ROUTER\_ENTRIES\_FILE}]
  A path and file name of a file that contains the ClassAds,
  in New ClassAd syntax, describing the routing table.
  The specified file is periodically reread to check for new information.
  This occurs every \MacroUNI{JOB\_ROUTER\_ENTRIES\_REFRESH} seconds.

\label{param:JobRouterEntriesCmd}
\item[\Macro{JOB\_ROUTER\_ENTRIES\_CMD}]
  Specifies the command line of an external program
  to run.  The output of the program defines or updates the routing table,
  and the output must be given in New ClassAd syntax.
  The specified command is periodically rerun to regenerate or update
  the routing table.
  This occurs every \MacroUNI{JOB\_ROUTER\_ENTRIES\_REFRESH} seconds.
  Specify the full path and file name of the executable within this
  command line, as no assumptions may be made about the current working
  directory upon command invocation.
  To enter spaces in any command-line arguments or in the command name itself,
  surround the right hand side of this definition with double quotes,
  and use single quotes around individual arguments that contain spaces.
  This is the same as when dealing with spaces within job arguments
  in an HTCondor submit description file. 

\label{param:JobRouterEntriesRefresh}
\item[\Macro{JOB\_ROUTER\_ENTRIES\_REFRESH}]
  The number of seconds between updates to the routing table described by
  \MacroNI{JOB\_ROUTER\_ENTRIES\_FILE} or
  \MacroNI{JOB\_ROUTER\_ENTRIES\_CMD}.
  The default value is 0, meaning no periodic updates occur.
  With the default value of 0, the routing table can be modified
  when a \Condor{reconfig} command is invoked 
  or when the \Condor{job\_router} daemon restarts.

\label{param:JobRouterLock}
\item[\Macro{JOB\_ROUTER\_LOCK}] This specifies the name of a lock
 file that is used to ensure that multiple instances of
 \condor{job\_router} never run with the same
 \MacroNI{JOB\_ROUTER\_NAME}.  Multiple instances running with the
 same name could lead to mismanagement of routed jobs.  The default
 value is \verb@$(LOCK)/$(JOB_ROUTER_NAME)Lock@.

\label{param:JobRouterSourceJobConstraint}
\item[\Macro{JOB\_ROUTER\_SOURCE\_JOB\_CONSTRAINT}]
  Specifies a global \Attr{Requirements} expression that must be true
  for all newly routed jobs,
  in addition to any \Attr{Requirements} specified within a routing table entry.
  In addition to the configurable constraints, the
 \Condor{job\_router} also has some hard-coded constraints.  It avoids
 recursively routing jobs by requiring that the job's attribute \Attr{RoutedBy}
 does not match \Macro{JOB\_ROUTER\_NAME}.  When not running as root,
 it also avoids routing jobs belonging to other users.

\label{param:JobRouterMaxJobs}
\item[\Macro{JOB\_ROUTER\_MAX\_JOBS}]
  An integer value representing the maximum number of jobs that may be routed,
  summed over all routes.
  The default value is -1, which means an unlimited number of jobs
  may be routed.

\label{param:MaxJobMirrorUpdateLag}
\item[\Macro{MAX\_JOB\_MIRROR\_UPDATE\_LAG}]
  An integer value that administrators will rarely consider changing,
  representing the maximum number of
  seconds the \Condor{job\_router} daemon waits,
  before it decides that routed copies have gone awry,
  due to the failure of events to appear
  in the \Condor{schedd}'s job queue log file.
  The default value is 600.
  As the \Condor{job\_router} daemon uses the \Condor{schedd}'s
  job queue log file entries for synchronization of routed copies,
  when an expected log file event fails to appear after this wait period,
  the \Condor{job\_router} daemon acts presuming the expected event
  will never occur.

\label{param:JobRouterPollingPeriod}
\item[\Macro{JOB\_ROUTER\_POLLING\_PERIOD}]
  An integer value representing the number of seconds
  between cycles in the \Condor{job\_router} daemon's task loop.
  The default is 10 seconds.
  A small value makes the \Condor{job\_router} daemon 
  quick to see new candidate jobs for routing.
  A large value makes the \Condor{job\_router} daemon generate less
  overhead at the cost of being slower to see new candidates for routing.
  For very large job queues where a few minutes of
  routing latency is no problem, increasing this value to a few
  hundred seconds would be reasonable.

\label{param:JobRouterName}
\item[\Macro{JOB\_ROUTER\_NAME}]
  A unique identifier utilized to name multiple instances of
  the \Condor{job\_router} daemon on the same machine.
  Each instance must have a different name,
  or all but the first to start up will refuse to run.
  The default is \AdStr{jobrouter}.

  Changing this value when routed jobs already exist is not currently
  gracefully handled.  However, it can be done if one also uses
  \Condor{qedit} to change the value of \Attr{ManagedManager} and
  \Attr{RoutedBy} from the old name to the new name.  The following commands
  may be helpful:

\begin{verbatim}
condor_qedit -constraint 'RoutedToJobId =!= undefined && \
  ManagedManager == "insert_old_name"' \
  ManagedManager '"insert_new_name"'
condor_qedit -constraint 'RoutedBy == "insert_old_name"' \
  RoutedBy '"insert_new_name"'
\end{verbatim}

\label{param:JobRouterReleaseOnHold}
\item[\Macro{JOB\_ROUTER\_RELEASE\_ON\_HOLD}]
  A boolean value that defaults to \Expr{True}.
  It controls how the \Condor{job\_router} handles the routed copy when it
  goes on hold.
  When \Expr{True}, the \Condor{job\_router} leaves the original job 
  ClassAd in the same state as when claimed.  When \Expr{False},
  the \Condor{job\_router} does not attempt to reset the original job
  ClassAd to a pre-claimed state upon yielding control of the job.

\label{param:JobRouterSchedd1Spool}
\item[\Macro{JOB\_ROUTER\_SCHEDD1\_SPOOL}]
  The path to the spool directory for the \Condor{schedd} serving as the
  source of jobs for routing.  If not specified, this defaults to
  \MacroUNI{SPOOL}.  
  If specified, this parameter must point to the spool directory of 
  the \Condor{schedd} identified by \MacroNI{JOB\_ROUTER\_SCHEDD1\_NAME}.

\label{param:JobRouterSchedd2Spool}
\item[\Macro{JOB\_ROUTER\_SCHEDD2\_SPOOL}]
  The path to the spool directory for the \Condor{schedd} to which the
  routed copy of the jobs are submitted.  If not specified, this defaults to
  \MacroUNI{SPOOL}.  
  If specified, this parameter must point to the spool directory of 
  the \Condor{schedd} identified by \MacroNI{JOB\_ROUTER\_SCHEDD2\_NAME}.
  Note that when \Condor{job\_router} is running as \Login{root} and is
  submitting routed jobs to a different \Condor{schedd} than the
  source \Condor{schedd}, it is required that \Condor{job\_router}
  have permission to impersonate the job owners of the routed jobs.
  It is therefore usually necessary to configure
  \Macro{QUEUE\_SUPER\_USER\_MAY\_IMPERSONATE} in the configuration
  of the target \Condor{schedd}.

\label{param:JobRouterSchedd1Name}
\item[\Macro{JOB\_ROUTER\_SCHEDD1\_NAME}]
  The advertised daemon name of the \Condor{schedd} serving as the
  source of jobs for routing.  If not specified, this defaults to the
  local \Condor{schedd}.  If specified, this parameter must name the
  same \Condor{schedd} whose spool is configured in
  \MacroNI{JOB\_ROUTER\_SCHEDD1\_SPOOL}.  If the named \Condor{schedd} is
  not advertised in the local pool, \Macro{JOB\_ROUTER\_SCHEDD1\_POOL}
  will also need to be set.

\label{param:JobRouterSchedd2Name}
\item[\Macro{JOB\_ROUTER\_SCHEDD2\_NAME}]
  The advertised daemon name of the \Condor{schedd} to which the
  routed copy of the jobs are submitted.  If not specified, this defaults to
  the local \Condor{schedd}.  If specified, this parameter must name the
  same \Condor{schedd} whose spool is configured in
  \MacroNI{JOB\_ROUTER\_SCHEDD2\_SPOOL}.  If the named \Condor{schedd} is
  not advertised in the local pool, \Macro{JOB\_ROUTER\_SCHEDD2\_POOL}
  will also need to be set.
  Note that when \Condor{job\_router} is running as \Login{root} and is
  submitting routed jobs to a different \Condor{schedd} than the
  source \Condor{schedd}, it is required that \Condor{job\_router}
  have permission to impersonate the job owners of the routed jobs.
  It is therefore usually necessary to configure
  \Macro{QUEUE\_SUPER\_USER\_MAY\_IMPERSONATE} in the configuration
  of the target \Condor{schedd}.

\label{param:JobRouterSchedd1Pool}
\item[\Macro{JOB\_ROUTER\_SCHEDD1\_POOL}]
  The Condor pool (\Condor{collector} address) of the \Condor{schedd} 
  serving as the source of jobs for routing.
  If not specified, defaults to the local pool.

\label{param:JobRouterSchedd2Pool}
\item[\Macro{JOB\_ROUTER\_SCHEDD2\_POOL}]
  The Condor pool (\Condor{collector} address) of the \Condor{schedd}
  to which the routed copy of the jobs are submitted.
  If not specified, defaults to the local pool.

\end{description}



%%%%%%%%%%%%%%%%%%%%%%%%%%%%%%%%%%%%%%%%%%%%%%%%%%%%%%%%%%%%%%%%%%%%%%%%%%%
\subsection{\label{sec:LeaseManager-Config-File-Entries}\condor{lease\_manager}
Configuration File Entries}
%%%%%%%%%%%%%%%%%%%%%%%%%%%%%%%%%%%%%%%%%%%%%%%%%%%%%%%%%%%%%%%%%%%%%%%%%%%$

\index{configuration!condor\_lease\_manager configuration variables}
These macros affect the \Condor{lease\_manager}.

The \Condor{lease\_manager} expects to use the syntax
\begin{verbatim}
 <subsystem name>.<parameter name>
\end{verbatim}
in configuration.
This allows multiple instances of the
\Condor{lease\_manager} to be easily configured using the syntax
\begin{verbatim}
 <subsystem name>.<local name>.<parameter name>
\end{verbatim}

\begin{description}

\label{param:LeaseManager.GetAdsInterval}
\item[\Macro{LeaseManager.GETADS\_INTERVAL}]
  An integer value, given in seconds, that controls the frequency
  with which the \Condor{lease\_manager}
  pulls relevant resource ClassAds from the \Condor{collector}.
  The default value is 60 seconds, with a minimum value of 2 seconds.


\label{param:LeaseManager.UpdateInterval}
\item[\Macro{LeaseManager.UPDATE\_INTERVAL}]
  An integer value, given in seconds, that controls the frequency
  with which the \Condor{lease\_manager}
  sends its ClassAds to the \Condor{collector}.
  The default value is 60 seconds, with a minimum value of 5 seconds.

\label{param:LeaseManager.PruneInterval}
\item[\Macro{LeaseManager.PRUNE\_INTERVAL}]
  An integer value, given in seconds, that controls the frequency
  with which the \Condor{lease\_manager} \Term{prunes} its leases.
  This involves checking all leases to see if they have expired.
  The default value is 60 seconds, with no minimum value.

\label{param:LeaseManager.DebugAds}
\item[\Macro{LeaseManager.DEBUG\_ADS}]
  A boolean value that defaults to \Expr{False}.
  When \Expr{True}, it enables extra
  debugging information about the resource ClassAds that it retrieves
  from the \Condor{collector} and about the search ClassAds that it sends
  to the \Condor{collector}.

\label{param:LeaseManager.MaxLeaseDuration}
\item[\Macro{LeaseManager.MAX\_LEASE\_DURATION}]
  An integer value representing seconds which determines 
  the maximum duration of a lease.  This can
  be used to provide a hard limit on lease durations.  Normally, the
  \Condor{lease\_manager} honors the \Attr{MaxLeaseDuration} attribute
  from the resource ClassAd.  If this configuration variable is defined,
  it limits the effective maximum duration for all resources to this value.
  The default value is 1800 seconds.

  Note that leases can be renewed, and thus can be extended beyond this
  limit.  To provide a limit on the total duration of a lease, use 
  \MacroNI{LeaseManager.MAX\_TOTAL\_LEASE\_DURATION}.

\label{param:LeaseManager.MaxTotalLeaseDuration}
\item[\Macro{LeaseManager.MAX\_TOTAL\_LEASE\_DURATION}]
  An integer value representing seconds used to limit
  the \emph{total} duration of leases, over
  all its renewals.  
  The default value is 3600 seconds.

\label{param:LeaseManager.DefaultMaxLeaseDuration}
\item[\Macro{LeaseManager.DEFAULT\_MAX\_LEASE\_DURATION}]
  The \Condor{lease\_manager} uses the
  \Attr{MaxLeaseDuration} attribute from the resource ClassAd to limit the
  lease duration.  If this attribute is not present in a resource
  ClassAd, then this configuration variable is used instead.
  This integer value is given in units of seconds,
  with a default value of 60 seconds.

\label{param:LeaseManager.ClassadLog}
\item[\Macro{LeaseManager.CLASSAD\_LOG}]
  This variable defines a full path and file name to the location
  where the \Condor{lease\_manager} keeps persistent state information.
  This variable has no default value.

\label{param:LeaseManager.QueryAdtype}
\item[\Macro{LeaseManager.QUERY\_ADTYPE}]
  This parameter controls the type of the query in the ClassAd sent to
  the \Condor{collector}, which will control the types of ClassAds
  returned by the \Condor{collector}.  This parameter must be a valid
  ClassAd type name, with a default value of \AdStr{Any}.

\label{param:LeaseManager.QueryConstraints}
\item[\Macro{LeaseManager.QUERY\_CONSTRAINTS}]
  A ClassAd expression that controls the constraint in the query sent to the
  \Condor{collector}.
  It is used to further constrain the types
  of ClassAds from the \Condor{collector}.
  There is no default value, resulting in no constraints being placed on query.

\end{description}

%%%%%%%%%%%%%%%%%%%%%%%%%%%%%%%%%%%%%%%%%%%%%%%%%%%%%%%%%%%%%%%%%%%%%%%%%%%
\subsection{\label{sec:GridMonitor-Config-File-Entries}Grid Monitor
Configuration File Entries}
%%%%%%%%%%%%%%%%%%%%%%%%%%%%%%%%%%%%%%%%%%%%%%%%%%%%%%%%%%%%%%%%%%%%%%%%%%%

\index{configuration!Grid Monitor configuration variables}
These macros affect the Grid Monitor.
\begin{description}

\label{param:EnableGridMonitor}
\item[\Macro{ENABLE\_GRID\_MONITOR}]
  A boolean value that when \Expr{True} enables the Grid Monitor.
  The Grid Monitor is used to reduce load on Globus gatekeepers.
  This parameter only affects grid jobs of type \SubmitCmd{gt2}.
  The variable \MacroNI{GRID\_MONITOR} must also be correctly configured.
  Defaults to \Expr{True}.
  See section~\ref{sec:HTCondor-G-GridMonitor} on
  page~\pageref{sec:HTCondor-G-GridMonitor}
  for more information.

\label{param:GridMonitor}
\item[\Macro{GRID\_MONITOR}]
  The complete path name of the \Prog{grid\_monitor.sh} tool used to reduce
  the load on Globus gatekeepers.
  This parameter only affects grid jobs of type \SubmitCmd{gt2}.
  This parameter is not referenced unless
  \MacroNI{ENABLE\_GRID\_MONITOR} is set to \Expr{True} (the default value). 

\label{param:GridMonitorHeartbeatTimeout}
\item[\Macro{GRID\_MONITOR\_HEARTBEAT\_TIMEOUT}]
  The integer number of seconds that may pass without hearing from a 
  working Grid Monitor before it is assumed to be dead.
  Defaults to 300 (5 minutes).  Increasing this number
  will improve the ability of the Grid Monitor to survive in the face of
  transient problems,
  but will also increase the time before HTCondor notices a problem.

\label{param:GridMonitorRetryDuration}
\item[\Macro{GRID\_MONITOR\_RETRY\_DURATION}]
  When HTCondor-G attempts to start the Grid Monitor at a particular
  site, it will wait this many seconds to start hearing from the
  Grid Monitor. Defaults to 900 (15 minutes).  If this duration
  passes without success, the Grid Monitor will be disabled for the
  site in question for the period of time set by
  \Macro{GRID\_MONITOR\_DISABLE\_TIME}.

\label{param:GridMonitorNoStatusTimeout}
\item[\Macro{GRID\_MONITOR\_NO\_STATUS\_TIMEOUT}]
  Jobs can disappear from the Grid Monitor's status reports for
  short periods of time under normal circumstances, but a prolonged
  absence is often a sign of problems on the remote machine. This variable
  sets the amount of time (in seconds) that a job can be absent before the
  \Condor{gridmanager} reacts by restarting the GRAM \Prog{jobmanager}.
  The default is 900, which is 15 minutes.

\label{param:GridMonitorDisableTime}
\item[\Macro{GRID\_MONITOR\_DISABLE\_TIME}]
  When an error occurs with a Grid Monitor job, this parameter controls
  how long the \Condor{gridmanager} will wait before attempting to
  start a new Grid Monitor job. The value is in seconds and the default
  is 3600 (1 hour).

\end{description}


%%%%%%%%%%%%%%%%%%%%%%%%%%%%%%%%%%%%%%%%%%%%%%%%%%%%%%%%%%%%%%%%%%%%%%%%%%%
\subsection{\label{sec:Grid-Config-File-Entries}Configuration File
Entries Relating to Grid Usage}
%%%%%%%%%%%%%%%%%%%%%%%%%%%%%%%%%%%%%%%%%%%%%%%%%%%%%%%%%%%%%%%%%%%%%%%%%%%

\index{configuration!grid configuration variables}
These macros affect the HTCondor's usage of grid resources.
\begin{description}

\label{param:GlexecJob}
\item[\Macro{GLEXEC\_JOB}]
  A boolean value that defaults to \Expr{False}.
  When \Expr{True}, it enables the use of \Prog{glexec} on the machine.

\label{param:Glexec}
\item[\Macro{GLEXEC}]
  The full path and file name of the \Prog{glexec} executable.

\label{param:GlexecRetries}
\item[\Macro{GLEXEC\_RETRIES}]
An integer value that specifies the maximum number of times to retry a
call to \Prog{glexec} when \Prog{glexec} exits with status 202 or 203,
error codes that indicate a possible transient error condition.  The
default number of retries is 3.

\label{param:GlexecRetryDelay}
\item[\Macro{GLEXEC\_RETRY\_DELAY}]
An integer value that specifies the minimum number of seconds to wait
between retries of a failed call to \Prog{glexec}.
The default is 5 seconds.
The actual delay to be used is determined by a random exponential backoff
algorithm that chooses a delay with a minimum of
the value of \MacroNI{GLEXEC\_RETRY\_DELAY} 
and a maximum of 100 times that value.

\label{param:GlexecHoldOnInitialFailure}
\item[\Macro{GLEXEC\_HOLD\_ON\_INITIAL\_FAILURE}]
A boolean value that when \Expr{False} prevents a job from being put
on hold when a failure is encountered during the glexec setup phase
of managing a job.  The default is \Expr{True}.
\Prog{glexec} is invoked multiple times during each attempt to run a job.
This configuration setting only disables putting the job on hold for the
initial invocation.  Subsequent failures during that run attempt always
put the job on hold.

\end{description}

%%%%%%%%%%%%%%%%%%%%%%%%%%%%%%%%%%%%%%%%%%%%%%%%%%%%%%%%%%%%%%%%%%%%%%%%%%%
\subsection{\label{sec:DAGMan-Config-File-Entries}Configuration File 
Entries for DAGMan}
%%%%%%%%%%%%%%%%%%%%%%%%%%%%%%%%%%%%%%%%%%%%%%%%%%%%%%%%%%%%%%%%%%%%%%%%%%%

\index{configuration!DAGMan configuration variables}
These macros affect the operation of DAGMan and DAGMan
jobs within HTCondor.

\Bold{Note}: Many, if not all, of these configuration variables will
be most appropriately set on a per DAG basis, rather than in the
global HTCondor configuration files.  Per DAG configuration is explained
in section~\ref{sec:DAG-configuration}.

\begin{description}

\label{param:DAGManUserLogScanInterval}
\item[\Macro{DAGMAN\_USER\_LOG\_SCAN\_INTERVAL}]
  An integer value representing the number of seconds that 
  \Condor{dagman} waits between checking job log files for status updates.
  Setting this value lower than the default increases the CPU
  time \Condor{dagman} spends checking files, perhaps fruitlessly, but
  increases responsiveness to nodes completing or failing.
  The legal range of values is 1 to INT\_MAX.
  If not defined, it defaults to 5 seconds.

\label{param:DAGManDebugCacheEnable}
\item[\Macro{DAGMAN\_DEBUG\_CACHE\_ENABLE}]
  A boolean value that determines if log line caching for the \File{dagman.out}
  file should be enabled in the \Condor{dagman} process to increase
  performance (potentially by orders of magnitude) when writing the
  \File{dagman.out} file to an NFS server. 
  Currently, this cache is only utilized in Recovery Mode.  
  If not defined, it defaults to \Expr{False}.

\label{param:DAGManDebugCacheSize}
\item[\Macro{DAGMAN\_DEBUG\_CACHE\_SIZE}]
  An integer value representing the number of bytes of log lines to
  be stored in the log line cache. When the cache surpasses this number,
  the entries are written out in one call to the logging subsystem. A value of
  zero is not recommended since each log line would surpass the cache size 
  and be emitted in addition to bracketing log lines explaining that the 
  flushing was happening.
  The legal range of values is 0 to INT\_MAX.
  If defined with a value less than 0, the  value 0 will be used.
  If not defined, it defaults to 5 Megabytes.

\label{param:DAGManMaxSubmitsPerInterval}
\item[\Macro{DAGMAN\_MAX\_SUBMITS\_PER\_INTERVAL}]
  An integer that controls how many individual jobs
  \Condor{dagman} will submit in a row
  before servicing other requests (such as a \Condor{rm}).
  The legal range of values is 1 to 1000.
  If defined with a value less than 1, the  value 1 will be used.
  If defined with a value greater than 1000, the value 1000 will be used.
  If not defined, it defaults to 5.

\label{param:DAGManMaxSubmitAttempts}
\item[\Macro{DAGMAN\_MAX\_SUBMIT\_ATTEMPTS}]
  An integer that controls how
  many times in a row \Condor{dagman} will attempt to execute
  \Condor{submit} for a given job before giving up.
  Note that consecutive attempts use an exponential backoff,
  starting with 1 second.
  The legal range of values is 1 to 16.
  If defined with a value less than 1, the  value 1 will be used.
  If defined with a value greater than 16, the value 16 will be used.
  Note that a value of 16 would result in \Condor{dagman} trying for
  approximately 36 hours before giving up.
  If not defined,
  it defaults to 6 (approximately two minutes before giving up).

\label{param:DAGManSubmitDelay}
\item[\Macro{DAGMAN\_SUBMIT\_DELAY}]
  An integer that controls the number of seconds that \Condor{dagman}
  will sleep before submitting consecutive jobs.  It can be increased to
  help reduce the load on the \Condor{schedd} daemon.  The legal range
  of values is any non negative integer.  If defined with a value less
  than 0, the value 0 will be used.

\label{param:DAGManStartupCycleDetect}
\item[\Macro{DAGMAN\_STARTUP\_CYCLE\_DETECT}]
  A boolean value that defaults to \Expr{False}.
  When \Expr{True},
  causes \Condor{dagman} to check for cycles in the DAG before
  submitting DAG node jobs,
  in addition to its run time cycle detection.

\label{param:DAGManRetrySubmitFirst}
\item[\Macro{DAGMAN\_RETRY\_SUBMIT\_FIRST}]
  A boolean value that controls whether a failed submit is retried first
  (before any other submits) or last (after all other ready jobs are
  submitted).  If this value is set to \Expr{True}, when a job submit
  fails, the job is placed at the head of the queue of ready jobs, so
  that it will be submitted again before any other jobs are submitted.
  This had been the behavior of \Condor{dagman}.
  If this value is set to \Expr{False}, when a job submit fails, the job
  is placed at the tail of the queue of ready jobs.
  If not defined, it defaults to \Expr{True}.

\label{param:DAGManRetryNodeFirst}
\item[\Macro{DAGMAN\_RETRY\_NODE\_FIRST}]
  A boolean value that controls whether a failed node with retries
  is retried first (before any other ready nodes) or last (after all
  other ready nodes).  If this value is set to \Expr{True}, when a
  node with retries fails after the submit succeeded, the node is
  placed at the head of the queue of ready nodes, so that it will be
  tried again before any other jobs are submitted.  If this value is
  set to \Expr{False}, when a node with retries fails, the node
  is placed at the tail of the queue of ready nodes.
  This had been the behavior of \Condor{dagman}.
  If not defined, it defaults to \Expr{False}.

\label{param:DAGManMaxJobsIdle}
\item[\Macro{DAGMAN\_MAX\_JOBS\_IDLE}]
  An integer value that controls the maximum number of idle node jobs
  allowed within the DAG before \Condor{dagman} temporarily stops
  submitting jobs.  Once idle jobs start to run, \Condor{dagman} will
  resume submitting jobs.  If both the command line option and the
  configuration variable are specified, the command line option overrides
  the configuration variable.  Unfortunately,
  \MacroNI{DAGMAN\_MAX\_JOBS\_IDLE} currently counts each individual
  process within a cluster as a job, which is inconsistent with
  \MacroNI{DAGMAN\_MAX\_JOBS\_SUBMITTED}.  The default is that there is
  no limit on the maximum number of idle jobs. 
  Note that nothing special is 
  done to the submit description file.
  If a submit description file contains \Expr{queue 5000}
  and \Macro{DAGMAN\_MAX\_JOBS\_IDLE} is set to 250, 
  this will result in 5000 jobs being submitted to the \Condor{schedd},
  not 250; in this case, no
  further jobs will then be submitted by \Condor{dagman} until the number of
  idle jobs falls below 250.

\label{param:DAGManMaxJobsSubmitted}
\item[\Macro{DAGMAN\_MAX\_JOBS\_SUBMITTED}]
  An integer value that controls the maximum number of node jobs within the
  DAG that will  be submitted to HTCondor at one time.  Note that this
  variable has the same functionality as the \OptArg{-maxjobs} 
  command line option to \Condor{submit\_dag}.
  If both the command line option and the
  configuration parameter are specified, the command line option overrides
  the configuration variable.  A single invocation of \Condor{submit}
  counts as one job, even if the submit file produces a multi-job cluster.
  The default is that there is no limit on the maximum number of jobs
  run at one time.

\label{param:DAGManMungeNodeNames}
\item[\Macro{DAGMAN\_MUNGE\_NODE\_NAMES}]
  A boolean value that controls whether \Condor{dagman} automatically
  renames nodes when running multiple DAGs.
  The renaming is done to avoid possible name conflicts.
  If this value is set to \Expr{True},
  all node names have the DAG number followed by the period character
  (\verb@.@) prepended to them.
  For example, the first DAG specified on the \Condor{submit\_dag}
  command line is considered DAG number 0, the second is DAG number 1, etc.
  So if DAG number 2 has a node named B,
  that node will internally be renamed to 2.B.
  If not defined, \MacroNI{DAGMAN\_MUNGE\_NODE\_NAMES} defaults to \Expr{True}.

\label{param:DAGManIgnoreDuplicateJobExecution}
\item[\Macro{DAGMAN\_IGNORE\_DUPLICATE\_JOB\_EXECUTION}]
  This configuration variable is no longer used. The improved functionality
  of the \MacroNI{DAGMAN\_ALLOW\_EVENTS} macro eliminates the
  need for this variable.

  For completeness, here is the definition for historical purposes: 
  A boolean value that controls
  whether \Condor{dagman} aborts or continues with a DAG
  in the rare case that HTCondor erroneously executes
  the job within a DAG node more than once.
  A bug in HTCondor very occasionally causes a job to run twice.
  Running a job twice is contrary to the semantics of a DAG.
  The configuration macro \MacroNI{DAGMAN\_IGNORE\_DUPLICATE\_JOB\_EXECUTION}
  determines whether  \Condor{dagman} considers this a fatal error or not.
  The default value is \Expr{False}; \Condor{dagman} considers
  running the job more than once a fatal error, 
  logs this fact,
  and aborts the DAG.
  When set to \Expr{True}, \Condor{dagman} still
  logs this fact,
  but continues with the DAG. 

  This configuration macro is to remain at its default value 
  except in the case
  where a site encounters the HTCondor bug in which DAG job nodes
  are executed twice,
  and where it is certain
  that having a DAG job node run twice will not corrupt the DAG.
  The logged messages within \File{*.dagman.out} files
  in the case of that a node job runs twice
  contain the string
  "EVENT ERROR."

\label{param:DAGManAllowEvents}
\item[\Macro{DAGMAN\_ALLOW\_EVENTS}]
  An integer that controls which bad events are considered
  fatal errors by \Condor{dagman}.  This macro replaces and expands
  upon the functionality of the
  \MacroNI{DAGMAN\_IGNORE\_DUPLICATE\_JOB\_EXECUTION} macro.
  If \MacroNI{DAGMAN\_ALLOW\_EVENTS} is set, it overrides the
  setting of \MacroNI{DAGMAN\_IGNORE\_DUPLICATE\_JOB\_EXECUTION}.

  The \MacroNI{DAGMAN\_ALLOW\_EVENTS} value is a logical bitwise OR of the
  following values:
  \begin{description}
  \item 0 = allow no bad events
  \item 1 = allow all bad events, \emph{except} the event
    \AdStr{job re-run after terminated event}
  \item 2 = allow terminated/aborted event combination
  \item 4 = allow a \AdStr{job re-run after terminated event} bug
  \item 8 = allow garbage or orphan events
  \item 16 = allow an execute or terminate event before job's submit event
  \item 32 = allow two terminated events per job, as sometimes seen
    with grid jobs
  \item 64 = allow duplicated events in general
  \end{description}

  The default value is 114, which allows terminated/aborted event combination,
  allows an execute and/or terminated event before job's submit event,
  allows double terminated events, and allows general duplicate events.

  As examples, a value of 6 instructs \Condor{dagman} to allow both
  the terminated/aborted event combination and the 
  \AdStr{job re-run after terminated event} bug.
  A value of 0 means that any bad event will be considered a fatal error.

  A value of 5 will never abort the DAG because of a bad event.
  But this value should almost never be used,
  because the \AdStr{job re-run after terminated event} 
  bug breaks the semantics of the DAG.

\label{param:DAGManDebug}
\item[\Macro{DAGMAN\_DEBUG}]
  This variable is described in section~\ref{param:SubsysDebug} as
  \MacroNI{<SUBSYS>\_DEBUG}.

\label{Param:MaxDAGManLog}
\item[\Macro{MAX\_DAGMAN\_LOG}]
  This variable is described in section~\ref{param:MaxSubsysLog} as
  \MacroNI{MAX\_<SUBSYS>\_LOG}.

\label{param:DAGManCondorSubmitExe}
\item[\Macro{DAGMAN\_CONDOR\_SUBMIT\_EXE}]
  The executable that \Condor{dagman} will use to submit HTCondor jobs.
  If not defined, \Condor{dagman} looks for \Condor{submit} in the path.

\label{param:DAGManStorkSubmitExe}
\item[\Macro{DAGMAN\_STORK\_SUBMIT\_EXE}]
  The executable that \Condor{dagman} will use to submit Stork jobs.
  If not defined, \Condor{dagman} looks for \Prog{stork\_submit} in the path.

\label{param:DAGManCondorRmExe}
\item[\Macro{DAGMAN\_CONDOR\_RM\_EXE}]
  The executable that \Condor{dagman} will use to remove HTCondor jobs.
  If not defined, \Condor{dagman} looks for \Condor{rm} in the path.

\label{param:DAGManStorkRmExe}
\item[\Macro{DAGMAN\_STORK\_RM\_EXE}]
  The executable that \Condor{dagman} will use to remove Stork jobs.
  If not defined, \Condor{dagman} looks for \Prog{stork\_rm} in the path.

\label{param:DAGManProhibitMultiJobs}
\item[\Macro{DAGMAN\_PROHIBIT\_MULTI\_JOBS}]
  A boolean value that controls whether \Condor{dagman} prohibits
  node job submit description files that queue multiple job procs other than 
  parallel universe.  If a DAG references such a submit file, the
  DAG will abort during the initialization process.  If not defined,
  \MacroNI{DAGMAN\_PROHIBIT\_MULTI\_JOBS} defaults to \Expr{False}.

\label{param:DAGManLogOnNfsIsError}
\item[\Macro{DAGMAN\_LOG\_ON\_NFS\_IS\_ERROR}]
  A boolean value that controls whether \Condor{dagman} prohibits
  node job submit description files with user log files on NFS.
  This value is ignored if \MacroNI{CREATE\_LOCKS\_ON\_LOCAL\_DISK} is 
  \Expr{True} and
  \MacroNI{ENABLE\_USERLOG\_LOCKING} is \Expr{True}.
  If a DAG references such a submit description file and
  \MacroNI{DAGMAN\_LOG\_ON\_NFS\_IS\_ERROR} is \Expr{True},
  the DAG will abort during the initialization process. 
  If \MacroNI{DAGMAN\_LOG\_ON\_NFS\_IS\_ERROR} is \Expr{False}, a warning
  will be issued, but the DAG will still be submitted.
  It is \emph{strongly}
  recommended that \MacroNI{DAGMAN\_LOG\_ON\_NFS\_IS\_ERROR}
  remain set to the default value, because running a DAG with node job
  log files on NFS will often cause errors.
  If not defined, \MacroNI{DAGMAN\_LOG\_ON\_NFS\_IS\_ERROR} defaults to
  \Expr{True}.

\label{param:DAGManAbortDuplicates}
\item[\Macro{DAGMAN\_ABORT\_DUPLICATES}]
  A boolean value that controls whether to attempt to abort duplicate
  instances of \Condor{dagman} running the same DAG on the same
  machine.  When \Condor{dagman} starts up, if no DAG lock file exists,
  \Condor{dagman} creates the lock file and writes its PID into it.  If
  the lock file does exist, and \MacroNI{DAGMAN\_ABORT\_DUPLICATES} is
  set to \Expr{True}, \Condor{dagman} checks whether a process with the
  given PID exists, and if so, it assumes that there is already another
  instance of \Condor{dagman} running the same DAG.  Note that this
  test is not foolproof: it is possible that, if \Condor{dagman} crashes,
  the same PID gets reused by another process before \Condor{dagman}
  gets rerun on that DAG.  This should be quite rare, however.
  If not defined, \MacroNI{DAGMAN\_ABORT\_DUPLICATES} defaults to
  \Expr{True}.

\label{param:DAGManSubmitDepthFirst}
\item[\Macro{DAGMAN\_SUBMIT\_DEPTH\_FIRST}]
  A boolean value that controls whether to submit ready DAG node jobs
  in (more-or-less) depth first order, as opposed to breadth-first order.
  Setting \MacroNI{DAGMAN\_SUBMIT\_DEPTH\_FIRST} to \Expr{True} does
  \emph{not} override dependencies defined in the DAG.  Rather, it
  causes newly ready nodes to be added to the head, rather than the tail,
  of the ready node list.  If there are no PRE scripts in the DAG, this
  will cause the ready nodes to be submitted depth-first.  If there
  are PRE scripts, the order will not be strictly depth-first, but it
  will tend to favor depth rather than breadth in executing the DAG.
  If \MacroNI{DAGMAN\_SUBMIT\_DEPTH\_FIRST} is set to \Expr{True},
  consider also setting \MacroNI{DAGMAN\_RETRY\_SUBMIT\_FIRST} and
  \Macro{DAGMAN\_RETRY\_NODE\_FIRST} to \Expr{True}.
  If not defined, \MacroNI{DAGMAN\_SUBMIT\_DEPTH\_FIRST} defaults to
  \Expr{False}.

\label{param:DAGManOnExitRemove}
\item[\Macro{DAGMAN\_ON\_EXIT\_REMOVE}]
  Defines the \Attr{OnExitRemove} ClassAd expression placed
  into the \Condor{dagman} submit description file by \Condor{submit\_dag}.
  The default expression is designed to ensure that \Condor{dagman} is
  automatically re-queued by the \Condor{schedd} daemon if it exits abnormally
  or is killed (for example, during a reboot).
  If this results in \Condor{dagman}
  staying in the queue when it should exit, consider changing 
  to a less restrictive expression, as in the example
\footnotesize
\begin{verbatim}
  (ExitBySignal == false || ExitSignal =!= 9)
\end{verbatim}
\normalsize
  If not defined, \MacroNI{DAGMAN\_ON\_EXIT\_REMOVE} defaults to
  the expression
\footnotesize
\begin{verbatim}
  ( ExitSignal =?= 11 || (ExitCode =!= UNDEFINED && ExitCode >=0 && ExitCode <= 2))
\end{verbatim}
\normalsize

\item[\Macro{DAGMAN\_ABORT\_ON\_SCARY\_SUBMIT}]
\label{param:DAGManAbortOnScarySubmit}
  A boolean value that controls whether to abort a DAG upon detection of
  a scary submit event.
  An example of a scary submit event is one in which the HTCondor ID
  does not match the expected value.
  Note that in all HTCondor versions prior to 6.9.3,
  \Condor{dagman} did \emph{not} abort a DAG upon detection of
  a scary submit event.
  This behavior is what now happens if
  \MacroNI{DAGMAN\_ABORT\_ON\_SCARY\_SUBMIT} is set to \Expr{False}.
  If not defined, \MacroNI{DAGMAN\_ABORT\_ON\_SCARY\_SUBMIT} defaults to
  \Expr{True}.

\label{param:DAGManPendingReportInterval}
\item[\Macro{DAGMAN\_PENDING\_REPORT\_INTERVAL}]
  An integer value representing the number of seconds that controls
  how often \Condor{dagman}
  will print a report of pending nodes to the \File{dagman.out} file.
  The report will only be printed if \Condor{dagman} has
  been waiting at least \MacroNI{DAGMAN\_PENDING\_REPORT\_INTERVAL}
  seconds without seeing any node job user log events, in order to
  avoid cluttering the \File{dagman.out} file.
  This feature is mainly intended to help diagnose \Condor{dagman} processes 
  that are stuck waiting indefinitely for a job to finish.
  If not defined,
  \MacroNI{DAGMAN\_PENDING\_REPORT\_INTERVAL} defaults to 600 seconds
  (10 minutes).

\label{param:DAGManInsertSubFile}
\item[\Macro{DAGMAN\_INSERT\_SUB\_FILE}]
  A file name of a file containing submit description file commands to be
  inserted into the \File{.condor.sub} file created by \Condor{submit\_dag}.
  The specified file is inserted into the \File{.condor.sub} file before
  the \SubmitCmd{queue} command and before any commands specified with the
  \Opt{-append} \Condor{submit\_dag} command line option.
  Note that the \MacroNI{DAGMAN\_INSERT\_SUB\_FILE} value can be overridden
  by the \Condor{submit\_dag} \Opt{-insert\_sub\_file} command line option.

\label{param:DAGManAutoRescue}
\item[\Macro{DAGMAN\_AUTO\_RESCUE}]
  A boolean value that controls whether \Condor{dagman} automatically
  runs Rescue DAGs.  If \MacroNI{DAGMAN\_AUTO\_RESCUE} is \Expr{True}
  and the DAG input file \File{my.dag} is submitted,
  and if a Rescue DAG such as the examples \File{my.dag.rescue001} or
  \File{my.dag.rescue002} exists, 
  then the largest magnitude Rescue DAG will be run.
  If not defined, \MacroNI{DAGMAN\_AUTO\_RESCUE} defaults to \Expr{True}.

\label{param:DAGManMaxRescueNum}
\item[\Macro{DAGMAN\_MAX\_RESCUE\_NUM}]
  An integer value that controls the maximum rescue DAG
  number that will be written, 
  in the case that \MacroNI{DAGMAN\_OLD\_RESCUE} is \Expr{False},
  or run if \MacroNI{DAGMAN\_AUTO\_RESCUE} is \Expr{True}.
  The maximum legal value is 999; the minimum value is 0,
  which prevents a rescue DAG from being written at all,
  or automatically run.
  If not defined, \MacroNI{DAGMAN\_MAX\_RESCUE\_NUM} defaults to 100.

\label{param:DAGManWritePartialRescue}
\item[\Macro{DAGMAN\_WRITE\_PARTIAL\_RESCUE}]
  A boolean value that controls whether \Condor{dagman} writes a partial
  or a full DAG file as a Rescue DAG.  
  As of HTCondor version 7.2.2, writing a partial DAG is preferred.
  If not defined, \MacroNI{DAGMAN\_WRITE\_PARTIAL\_RESCUE} defaults to
  \Expr{True}.

\label{param:DAGManResetRetriesUponRescue}
\item[\Macro{DAGMAN\_RESET\_RETRIES\_UPON\_RESCUE}]
  A boolean value that controls whether node retries are reset in a Rescue
  DAG.  If this value is \Expr{False}, the number of node retries written
  in a Rescue DAG is decreased,
  if any retries were used in the original run of the DAG; 
  otherwise, the original number of retries is allowed
  when running the Rescue DAG.
  If not defined, \MacroNI{DAGMAN\_RESET\_RETRIES\_UPON\_RESCUE} defaults to
  \Expr{True}.

\label{param:DAGManCopyToSpool}
\item[\Macro{DAGMAN\_COPY\_TO\_SPOOL}]
  A boolean value that when \Expr{True} copies the \Condor{dagman} binary
  to the spool directory when a DAG is submitted.
  Setting this variable to \Expr{True} allows
  long-running DAGs to survive a DAGMan version upgrade.
  For running large numbers of small DAGs, leave this
  variable unset or set it to \Expr{False}.
  The default value if not defined is \Expr{False}.

\label{param:DAGManDefaultNodeLog}
\item[\Macro{DAGMAN\_DEFAULT\_NODE\_LOG}]
  The name of a file to be used as a user log by any node jobs that
  do not define their own log files.
  The default value if not defined is \File{<DagFile>.nodes.log},
  where \verb@<DagFile>@ is replaced by the command line argument
  to \Condor{submit\_dag} that specifies the DAG input file.

\label{param:DAGManGenerateSubDagSubmits}
\item[\Macro{DAGMAN\_GENERATE\_SUBDAG\_SUBMITS}]
  A boolean value specifying whether \Condor{dagman} itself should
  create the \File{.condor.sub} files for nested DAGs.  
  If set to \Expr{False}, nested DAGs will fail unless
  the \File{.condor.sub} files are generated manually by running
  \Condor{submit\_dag} \Arg{-no\_submit} on each nested DAG, or the
  \Arg{-do\_recurse} flag is passed to \Condor{submit\_dag} for the
  top-level DAG.
  DAG nodes specified with the
  \MacroNI{SUBDAG EXTERNAL} keyword or with submit description file names ending
  in \File{.condor.sub} are considered nested DAGs.
  The default value if not defined is \Expr{True}.

\label{param:DAGManMaxJobHolds}
\item[\Macro{DAGMAN\_MAX\_JOB\_HOLDS}]
  An integer value defining the maximum number of times a node job is
  allowed to go on hold. As a job goes on hold this number of
  times, it is removed from the queue.  For example, if the value
  is 2, as the job goes on hold for the second time,
  it will be removed.
  At this time, this feature is not fully compatible with node jobs
  that have more than one \Attr{ProcID}.
  The number of holds of each process in the cluster count towards the
  total, rather than counting individually.
  So, this setting should take that possibility into account,
  possibly using a larger value.
  A value of 0 allows a job to go on hold any number of times.
  The default value if not defined is 100.

\label{param:DAGManVerbosity}
\item[\Macro{DAGMAN\_VERBOSITY}]
  An integer value defining the verbosity of output to the
  \File{dagman.out} file, as follows (each level includes all output
  from lower debug levels):
  \begin{itemize}
    \item level = 0; never produce output,
          except for usage info
    \item level = 1; very quiet, output severe errors
    \item level = 2; output errors and warnings
    \item level = 3; normal output
    \item level = 4; internal debugging output
    \item level = 5; internal debugging output; outer loop debugging
    \item level = 6; internal debugging output; inner loop debugging
    \item level = 7; internal debugging output; rarely used
  \end{itemize}
  The default value if not defined is 3.

\label{param:DAGManMaxPreScripts}
\item[\Macro{DAGMAN\_MAX\_PRE\_SCRIPTS}]
  An integer defining the maximum number of PRE scripts that any given
  \Condor{dagman} will run at the same time.  The default value if not
  defined is 0, which means to allow any number of PRE scripts to run.

\label{param:DAGManMaxPostScripts}
\item[\Macro{DAGMAN\_MAX\_POST\_SCRIPTS}]
  An integer defining the maximum number of POST scripts that any given
  \Condor{dagman} will run at the same time.  The default value if not
  defined is 0, which means to allow any number of POST scripts to run.

\label{param:DAGManAllowLogError}
\item[\Macro{DAGMAN\_ALLOW\_LOG\_ERROR}]
  A boolean value defining whether \Condor{dagman} will still attempt
  to run a node job, even if errors are detected in the user log
  specification.  This setting has an effect only on nodes that are
  Stork jobs (not HTCondor jobs).  The default value if not defined is
  \Expr{False}.

\label{param:DAGManUseStrict}
\item[\Macro{DAGMAN\_USE\_STRICT}]
  An integer defining the level of strictness \Condor{dagman} will apply
  when turning warnings into fatal errors, as follows:
  \begin{itemize}
    \item 0: no warnings become errors
    \item 1: severe warnings become errors
    \item 2: medium-severity warnings become errors
    \item 3: almost all warnings become errors
  \end{itemize}
  Using a strictness value greater than 0 may help find problems with
  a DAG that may otherwise escape notice.
  The default value if not defined is 1.

\label{param:DAGmanAlwaysRunPost}
\item[\Macro{DAGMAN\_ALWAYS\_RUN\_POST}]
  A boolean value defining whether \Condor{dagman} will ignore the return value
  of a PRE script when deciding to run a POST script.  The default is
  \Expr{True}, which says that the POST script will run regardless of the return
  value of the PRE script. Changing this to \Expr{False} will restore old
  behavior of \Condor{dagman}, which is that the failure of a PRE script causes
  the POST script to not be executed.

\label{param:DAGmanHoldClaimTime}
\item[\Macro{DAGMAN\_HOLD\_CLAIM\_TIME}]
  An integer defining the number of seconds that \Condor{dagman} will cause a
  hold on a claim after a job is finished, 
  using the job ClassAd attribute \Attr{KeepClaimIdle}.
  The default value is 20. 
  A value of 0 causes \Condor{dagman} not to set the job ClassAd attribute.

\label{param:DAGmanAlwaysUseNodeLog}
\item[\Macro{DAGMAN\_ALWAYS\_USE\_NODE\_LOG}]
  A boolean value defining whether \Condor{dagman} watches for events in its
  default node log.
  The default log is defined by the value of \Macro{DAGMAN\_DEFAULT\_NODE\_LOG}.
  The default value is \Expr{True}.
  If \Expr{True},
  then \Condor{dagman} writes POST script terminate events to the default log,
  and not to the user log specified in the submit description file.
  \Condor{dagman} will also only read events from the default log file.
  This variable must be \Expr{False} if \Condor{dagman} from Condor
  version 7.9.0 or later is submitting jobs to a \Condor{schedd} daemon
  or using a \Condor{submit} executable that is older than Condor version 7.9.0.

\label{param:DAGmanSuppressNotification}
\item[\Macro{DAGMAN\_SUPPRESS\_NOTIFICATION}]
  A boolean value defining whether jobs submitted by \Condor{dagman} will use
  email notification when certain events occur.  
  If \Expr{True}, 
  all jobs submitted by \Condor{dagman} will have the equivalent of the
  submit command \Expr{notification = never} set. 
  This does not affect the notification for events relating to 
  the \Condor{dagman} job itself. 
  Defaults to \Expr{True}, implementing behavior in which the user
  receives one notification email when the \Condor{dagman} job completes,
  rather than thousands of notifications from each of the jobs submitted by
  \Condor{dagman}. 

\end{description}

%%%%%%%%%%%%%%%%%%%%%%%%%%%%%%%%%%%%%%%%%%%%%%%%%%%%%%%%%%%%%%%%%%%%%%%%%%%
\subsection{\label{sec:Config-Security}Configuration File Entries
Relating to Security}
%%%%%%%%%%%%%%%%%%%%%%%%%%%%%%%%%%%%%%%%%%%%%%%%%%%%%%%%%%%%%%%%%%%%%%%%%%%

\index{configuration!security configuration variables}
These macros affect the secure operation of HTCondor.
Many of these macros are described in
section~\ref{sec:Security} on Security.

\begin{description}
\label{param:SecAuthentication}
\item[\Macro{SEC\_*\_AUTHENTICATION}]
\Todo

\label{param:SecEncryption}
\item[\Macro{SEC\_*\_ENCRYPTION}]
\Todo

\label{param:SecIntegrity}
\item[\Macro{SEC\_*\_INTEGRITY}]
\Todo

\label{param:SecNegotiation}
\item[\Macro{SEC\_*\_NEGOTIATION}]
\Todo

\label{param:SecAuthenticationMethods}
\item[\Macro{SEC\_*\_AUTHENTICATION\_METHODS}]
\Todo

\label{param:SecCryptoMethods}
\item[\Macro{SEC\_*\_CRYPTO\_METHODS}]
\Todo

\label{param:GSIDaemonName}
\item[\Macro{GSI\_DAEMON\_NAME}]
  This configuration variable is retired.
  Instead use \Macro{ALLOW\_CLIENT} or \Macro{DENY\_CLIENT} as
  appropriate. When used, this variable defined
  a comma separated list of the subject
  name(s) of the certificate(s) used by Condor daemons to
  which this configuration of Condor will connect.
  The \verb|*| character may be used as a wild card character.
  When \MacroNI{GSI\_DAEMON\_NAME} is defined, 
  only certificates matching
  \MacroNI{GSI\_DAEMON\_NAME} pass the authentication step, and no
  check is performed to require that the host name of the daemon
  matches the host name in the daemon's certificate.  
  When \MacroNI{GSI\_DAEMON\_NAME} is not defined, 
  the host name of the daemon and certificate must match unless
  exempted by the use of \MacroNI{GSI\_SKIP\_HOST\_CHECK} and/or
  \MacroNI{GSI\_SKIP\_HOST\_CHECK\_CERT\_REGEX}.

\label{param:GSISkipHostCheck}
\item[\Macro{GSI\_SKIP\_HOST\_CHECK}]
  A boolean variable that controls whether a check is performed during
  GSI authentication of a Condor daemon.  
  When the default value of \Expr{False},
  the check is not skipped, so the daemon host name must match the
  host name in the daemon's certificate, 
  unless otherwise exempted by the use of \Macro{GSI\_DAEMON\_NAME} or
  \MacroNI{GSI\_SKIP\_HOST\_CHECK\_CERT\_REGEX}.
  When \Expr{True}, this check is skipped, and hosts will not be rejected
  due to a mismatch of certificate and host name.

\label{param:GSISkipHostCheckCertRegex}
\item[\Macro{GSI\_SKIP\_HOST\_CHECK\_CERT\_REGEX}]
  This may be set to a regular expression.  GSI certificates of Condor
  daemons with a subject name that are matched in full by this regular
  expression are not required to have a matching daemon host name and
  certificate host name.  
  The default is an empty regular expression,
  which will not match any certificates,
  even if they have an empty subject name.

\label{param:HostAlias}
\item[\Macro{HOST\_ALIAS}] Specifies the fully qualified
  host name that clients authenticating this daemon with GSI should
  expect the daemon's certificate to match.  The alias is advertised
  to the \Condor{collector} as part of the address of the daemon.
  When this is not set, clients validate the daemon's certificate
  host name by matching it against DNS A records for the host they
  are connected to.  See \Macro{GSI\_SKIP\_HOST\_CHECK} for ways
  to disable this validation step.

\label{param:GSIDaemonDirectory}
\item[\Macro{GSI\_DAEMON\_DIRECTORY}]
  A directory name used in the
  construction of complete paths for the configuration variables
  \MacroNI{GSI\_DAEMON\_CERT},
  \MacroNI{GSI\_DAEMON\_KEY}, and
  \MacroNI{GSI\_DAEMON\_TRUSTED\_CA\_DIR},
  for any of these configuration variables are not explicitly set.

\label{param:GSIDaemonCert}
\item[\Macro{GSI\_DAEMON\_CERT}]
  A complete path and file name to the
  X.509 certificate to be used in GSI authentication.
  If this configuration variable is not defined, and
  \MacroNI{GSI\_DAEMON\_DIRECTORY} is defined, then HTCondor uses
  \MacroNI{GSI\_DAEMON\_DIRECTORY} to construct the path and file name as
  \begin{verbatim}
  GSI_DAEMON_CERT  = $(GSI_DAEMON_DIRECTORY)/hostcert.pem
  \end{verbatim}

\label{param:GSIDaemonKey}
\item[\Macro{GSI\_DAEMON\_KEY}]
  A complete path and file name to the
  X.509 private key to be used in GSI authentication.
  If this configuration variable is not defined, and
  \MacroNI{GSI\_DAEMON\_DIRECTORY} is defined, then HTCondor uses
  \MacroNI{GSI\_DAEMON\_DIRECTORY} to construct the path and file name as
  \begin{verbatim}
  GSI_DAEMON_KEY  = $(GSI_DAEMON_DIRECTORY)/hostkey.pem
  \end{verbatim}

\label{param:GSIDaemonTrustedCADir}
\item[\Macro{GSI\_DAEMON\_TRUSTED\_CA\_DIR}]
  The directory that contains the
  list of trusted certification authorities to be used in GSI authentication.
  The files in this directory are the public keys and signing policies
  of the trusted certification authorities.
  If this configuration variable is not defined, and
  \MacroNI{GSI\_DAEMON\_DIRECTORY} is defined, then HTCondor uses
  \MacroNI{GSI\_DAEMON\_DIRECTORY} to construct the directory path as
  \begin{verbatim}
  GSI_DAEMON_TRUSTED_CA_DIR  = $(GSI_DAEMON_DIRECTORY)/certificates
  \end{verbatim}

\label{param:GSIDaemonProxy}
\item[\Macro{GSI\_DAEMON\_PROXY}]
  A complete path and file name to the
  X.509 proxy to be used in GSI authentication.
  When this configuration variable is defined, use of this proxy
  takes precedence over use of a certificate and key.

\label{param:GSIAuthzConf}
\item[\Macro{GSI\_AUTHZ\_CONF}]
  A complete path and file name of the
  Globus mapping library that looks for the mapping call out configuration.
  There is no default value; as such, HTCondor uses the environment
  variable \Env{GSI\_AUTHZ\_CONF} when this variable is not defined.
  Setting this variable to \Expr{/dev/null} disables callouts.

\label{param:DelegateJobGSICredentials} 
\item[\Macro{DELEGATE\_JOB\_GSI\_CREDENTIALS}]
  A boolean value that defaults to \Expr{True} for HTCondor version 6.7.19
  and more recent versions.
  When \Expr{True}, a job's GSI X.509 credentials are delegated,
  instead of being copied.
  This results in a more secure communication when not encrypted.

\label{param:DelegateFullJobGSICredentials} 
\item[\Macro{DELEGATE\_FULL\_JOB\_GSI\_CREDENTIALS}]
  A boolean value that controls whether HTCondor will delegate a full or limited
  GSI X.509 proxy.  
  The default value of \Expr{False} indicates the limited GSI X.509 proxy.

\label{param:DelegateJobGSICredentialsLifetime}
\item[\Macro{DELEGATE\_JOB\_GSI\_CREDENTIALS\_LIFETIME}]
  An integer value that specifies the maximum number of seconds for
  which delegated proxies should be valid.  
  The default value is one day.
  A value of 0 indicates that the delegated proxy should be valid for as
  long as allowed by the credential used to create the proxy.  
  The job may override this configuration setting by using the
  \SubmitCmd{delegate\_job\_GSI\_credentials\_lifetime} submit file
  command.  This configuration variable currently only applies to
  proxies delegated for non-grid jobs and HTCondor-C jobs.  It does not
  currently apply to globus grid jobs, which always behave as though
  the value is 0.
  This variable has no effect if \Macro{DELEGATE\_JOB\_GSI\_CREDENTIALS}
  is \Expr{False}.

\label{param:DelegateJobGSICredentialsRefresh}
\item[\Macro{DELEGATE\_JOB\_GSI\_CREDENTIALS\_REFRESH}]
  A floating point number between 0 and 1 that indicates the fraction of 
  a proxy's lifetime at which point delegated
  credentials with a limited lifetime should be renewed.  
  The renewal is attempted periodically at or near the specified fraction
  of the lifetime of the delegated credential.  
  The default value is 0.25.
  This setting has no effect if \Macro{DELEGATE\_JOB\_GSI\_CREDENTIALS} is 
  \Expr{False} or if
  \Macro{DELEGATE\_JOB\_GSI\_CREDENTIALS\_LIFETIME} is 0.  
  For non-grid jobs, the precise timing of the proxy refresh depends on
  \Macro{SHADOW\_CHECKPROXY\_INTERVAL}.  
  To ensure that the delegated proxy remains valid, 
  the interval for checking the proxy should be,
  at most, half of the interval for refreshing it.

\label{param:GSIDelegationKeybits}
\item[\Macro{GSI\_DELEGATION\_KEYBITS}]
  The integer number of bits in the GSI key.
  If set to 0, the number of bits will be that preferred by the GSI library.
  If set to less than 512, the value will be ignored, and the key size
  will be 512 bits.
  Setting the value greater than 4096 is likely to cause long compute times.
  
\label{param:GridMap}
\item[\Macro{GRIDMAP}]
  The complete path and file name of the Globus Gridmap file.
  The Gridmap file is used to map
  X.509 distinguished names to HTCondor user ids.

\label{param:SecDefaultSessionDuration}
\MacroIndex{SEC\_DEFAULT\_SESSION\_DURATION}
\item[\Macro{SEC\_<access-level>\_SESSION\_DURATION}]
  The amount of time in seconds before
  a communication session expires.
  A session is a record of necessary information to do communication
  between a client and daemon, and is protected by a shared secret key.
  The session expires to reduce the window of opportunity where
  the key may be compromised by attack.  A short session duration
  increases the frequency with which daemons have to reauthenticate
  with each other, which may impact performance.

  If the client and server are configured with different durations,
  the shorter of the two will be used.  The default for daemons is
  86400 seconds (1 day) and the default for command-line tools is 60
  seconds.  The shorter default for command-line tools is intended to
  prevent daemons from accumulating a large number of communication
  sessions from the short-lived tools that contact them over time.  A
  large number of security sessions consumes a large amount of memory.
  It is therefore important when changing this configuration setting
  to preserve the small session duration for command-line tools.

  One example of how to safely change the session duration is to
  explicitly set a short duration for tools and \Condor{submit}
  and a longer duration for everything else:

\begin{verbatim}
SEC_DEFAULT_SESSION_DURATION = 50000
TOOL.SEC_DEFAULT_SESSION_DURATION = 60
SUBMIT.SEC_DEFAULT_SESSION_DURATION = 60
\end{verbatim}

Another example of how to safely change the session duration is to
explicitly set the session duration for a specific daemon:

\begin{verbatim}
COLLECTOR.SEC_DEFAULT_SESSION_DURATION = 50000
\end{verbatim}

\label{param:SecDefaultSessionLease}
\MacroIndex{SEC\_DEFAULT\_SESSION\_LEASE}
\item[\Macro{SEC\_<access-level>\_SESSION\_LEASE}]
  The maximum number of seconds an unused security session will be
  kept in a daemon's session cache before being removed to save memory.
  The default is 3600.  If the server and client have different
  configurations, the smaller one will be used.

\label{param:SecInvalidateSessionsViaTcp}
\item[\Macro{SEC\_INVALIDATE\_SESSIONS\_VIA\_TCP}]
  Use TCP (if True) or UDP (if False)
  for responding to attempts to use an invalid security session.  This happens,
  for example, if a daemon restarts and receives incoming commands from
  other daemons that are still using a previously established security session.
  The default is True.

\label{param:FSRemoteDir}
\item[\Macro{FS\_REMOTE\_DIR}]
  The location of a file visible to both server and client in
  Remote File System authentication.
  The default when not defined is the directory 
  \File{/shared/scratch/tmp}.

\label{param:EncryptExecuteDirectory}
\item[\Macro{ENCRYPT\_EXECUTE\_DIRECTORY}]
  The execute directory for jobs on Windows platforms may be
  encrypted by setting this configuration variable to \Expr{True}.
  Defaults to \Expr{False}.
  The method of encryption uses the EFS (Encrypted File System)
  feature of Windows NTFS v5.

\label{param:SecTCPSessionTimeout}
\item[\Macro{SEC\_TCP\_SESSION\_TIMEOUT}]
  The length of time in seconds until the timeout
  on individual network operations when establishing a UDP security
  session via TCP.
  The default value is 20 seconds.
  Scalability issues with a large pool would be the only basis
  for a change from the default value.

\label{param:SecTCPSessionDeadline}
\item[\Macro{SEC\_TCP\_SESSION\_DEADLINE}]
  An integer representing the total length of time in seconds until giving up
  when establishing a security session.  Whereas
  \Macro{SEC\_TCP\_SESSION\_TIMEOUT} specifies the timeout
  for individual blocking operations (connect, read, write), this
  setting specifies the total time across all operations, including
  non-blocking operations that have little cost other than holding
  open the socket.
  The default value is 120 seconds.
  The intention of this setting is to avoid waiting for hours
  for a response in the rare event that the other side
  freezes up and the socket remains in a connected state.
  This problem has been observed in some types of operating system
  crashes.

\label{param:SecDefaultAuthenticationTimeout}
\item[\Macro{SEC\_DEFAULT\_AUTHENTICATION\_TIMEOUT}]
  The length of time in seconds that HTCondor should attempt
  authenticating network connections before giving up.
  The default is 20 seconds.
  Like other security settings, the portion of the configuration variable
  name, \MacroNI{DEFAULT}, 
  may be replaced by a different access level to specify the timeout to use for
  different types of commands, for example
  \MacroNI{SEC\_CLIENT\_AUTHENTICATION\_TIMEOUT}.

\label{param:SecPasswordFile}
\item[\Macro{SEC\_PASSWORD\_FILE}]
  For Unix machines, the path and file name
  of  the file containing the pool password for password authentication.


\label{param:AuthSSLServerCAFile}
\item[\Macro{AUTH\_SSL\_SERVER\_CAFILE}]
  The path and file name of
  a file containing one or more trusted CA's certificates
  for the server side of a communication authenticating 
  with SSL.

\label{param:AuthSSLClientCAFile}
\item[\Macro{AUTH\_SSL\_CLIENT\_CAFILE}]
  The path and file name of
  a file containing one or more trusted CA's certificates
  for the client side of a communication authenticating 
  with SSL.


\label{param:AuthSSLServerCADir}  
\item[\Macro{AUTH\_SSL\_SERVER\_CADIR}]
  The path to a directory that may contain the 
  certificates (each in its own file) for multiple trusted CAs 
  for the server side of a communication authenticating 
  with SSL.
  When defined, the authenticating entity's certificate 
  is utilized to identify the trusted CA's certificate
  within the directory.

\label{param:AuthSSLClientCADir} 
\item[\Macro{AUTH\_SSL\_CLIENT\_CADIR}]
  The path to a directory that may contain the 
  certificates (each in its own file) for multiple trusted CAs 
  for the client side of a communication authenticating with SSL.
  When defined, the authenticating entity's certificate 
  is utilized to identify the trusted CA's certificate
  within the directory.


\label{param:AuthSSLServerCertfile}  
\item[\Macro{AUTH\_SSL\_SERVER\_CERTFILE}]
  The path and file name of the file containing the public certificate
  for the server side of a communication authenticating with SSL.

\label{param:AuthSSLClientCertfile}
\item[\Macro{AUTH\_SSL\_CLIENT\_CERTFILE}]
  The path and file name of the file containing the public certificate
  for the client side of a communication authenticating with SSL.


\label{param:AuthSSLServerKeyfile}
\item[\Macro{AUTH\_SSL\_SERVER\_KEYFILE}]
  The path and file name of the file containing the private key
  for the server side of a communication authenticating with SSL.

\label{param:AuthSSLClientKeyfile}
\item[\Macro{AUTH\_SSL\_CLIENT\_KEYFILE}]
  The path and file name of the file containing the private key
  for the client side of a communication authenticating with SSL.


\label{param:CertificateMapfile}
\item[\Macro{CERTIFICATE\_MAPFILE}]
  A path and file name of the unified map file.

\label{param:SecEnableMatchPasswordAuthentication}
\item[\Macro{SEC\_ENABLE\_MATCH\_PASSWORD\_AUTHENTICATION}]
  This is a special authentication mechanism designed to minimize
  overhead in the \Condor{schedd} when communicating with the execute
  machine.  Essentially, matchmaking results in a secret being shared
  between the \Condor{schedd} and \Condor{startd}, and this is used to
  establish a strong security session between the execute and submit
  daemons without going through the usual security negotiation protocol.
  This is especially important when operating at large scale over high
  latency networks (for example, on a pool with one \Condor{schedd} daemon
  and thousands of \Condor{startd} daemons on a network with a 0.1 second 
  round trip time).

  The default value for this configuration option is \Expr{False}.  To
  have any effect, it must be \Expr{True} in the configuration of both
  the execute side (\Condor{startd}) as well as the submit side 
  (\Condor{schedd}).  When
  this authentication method is used, all other security negotiation
  between the submit and execute daemons is bypassed.  All inter-daemon
  communication between the submit and execute side will use the
  \Condor{startd} daemon's settings for \MacroNI{SEC\_DAEMON\_ENCRYPTION} and
  \MacroNI{SEC\_DAEMON\_INTEGRITY}; the configuration of these values in
  the \Condor{schedd}, \Condor{shadow}, and \Condor{starter} are ignored.

  Important: For strong security, at least one of the two, integrity or
  encryption, should be enabled in the startd configuration.  Also, some
  form of strong mutual authentication (e.g. GSI) should be enabled
  between all daemons and the central manager or the shared secret which
  is exchanged in matchmaking cannot be safely encrypted when transmitted
  over the network.

  The \Condor{schedd} and \Condor{shadow} will be authenticated as
  \verb|submit-side@matchsession| when they talk to the \Condor{startd} and
  \Condor{starter}.  The \Condor{startd} and \Condor{starter} will be authenticated as
  \verb|execute-side@matchsession| when they talk to the \Condor{schedd} and
  \Condor{shadow}.  On the submit side, authorization of the execute side happens
  automatically.  On the execute side, it is necessary to explicitly
  authorize the submit side.  Example:

\begin{verbatim}
  ALLOW_DAEMON = submit-side@matchsession/192.168.123.*
\end{verbatim}

  Replace the example netmask with something suitable for your situation.

\label{param:KerberosServerKeytab}
\item[\Macro{KERBEROS\_SERVER\_KEYTAB}]
  The path and file name of the keytab file that holds the necessary Kerberos
  principals.
  If not defined, this variable's value is set by the installed Kerberos;
  it is \File{/etc/v5srvtab} on most systems.

\label{param:KerberosServerPrincipal}
\item[\Macro{KERBEROS\_SERVER\_PRINCIPAL}]
  An exact Kerberos principal to use.
  The default value is \verb$host/<hostname>@<realm>$, as set by the
  installed Kerberos.
  Where both \MacroNI{KERBEROS\_SERVER\_PRINCIPAL} and
  \MacroNI{KERBEROS\_SERVER\_SERVICE} are defined, this value takes
  precedence.

\label{param:KerberosServerUser}
\item[\Macro{KERBEROS\_SERVER\_USER}]
  The user name that the Kerberos server principal will map to after
  authentication.
  The default value is \verb@condor@.

\label{param:KerberosServerService}
\item[\Macro{KERBEROS\_SERVER\_SERVICE}]
  A string representing the Kerberos service name.
  This string is prepended with a slash character (\verb@/@) and the host name
  in order to form the Kerberos server principal.
  This value defaults to \verb@host@, resulting in the same default value
  as specified by using \MacroNI{KERBEROS\_SERVER\_PRINCIPAL}.
  Where both \MacroNI{KERBEROS\_SERVER\_PRINCIPAL} and
  \MacroNI{KERBEROS\_SERVER\_SERVICE} are defined, the value of
  \MacroNI{KERBEROS\_SERVER\_PRINCIPAL} takes precedence.


\label{param:KerberosClientKeytab}
\item[\Macro{KERBEROS\_CLIENT\_KEYTAB}]
  The path and file name of the keytab file for the client
  in Kerberos authentication.
  This variable has no default value.

\end{description}

%%%%%%%%%%%%%%%%%%%%%%%%%%%%%%%%%%%%%%%%%%%%%%%%%%%%%%%%%%%%%%%%%%%%%%%%%%%
\subsection{\label{sec:Config-PrivSep}Configuration File Entries
Relating to PrivSep}
%%%%%%%%%%%%%%%%%%%%%%%%%%%%%%%%%%%%%%%%%%%%%%%%%%%%%%%%%%%%%%%%%%%%%%%%%%%
\index{configuration!PrivSep configuration variables}
\begin{description}
\label{param:PrivSepEnabled}
\item[\Macro{PRIVSEP\_ENABLED}]
  A boolean variable that, when \Expr{True}, enables PrivSep.
  When \Expr{True}, the \Condor{procd} is used,
  ignoring the definition of the configuration variable \Macro{USE\_PROCD}.
  The default value when this configuration variable is not defined
  is \Expr{False}.

\label{param:PrivSepSwitchboard}
\item[\Macro{PRIVSEP\_SWITCHBOARD}]
  The full (trusted) path and file name of the \Condor{root\_switchboard}
  executable.

\end{description}

%%%%%%%%%%%%%%%%%%%%%%%%%%%%%%%%%%%%%%%%%%%%%%%%%%%%%%%%%%%%%%%%%%%%%%%%%%%
\subsection{\label{sec:Config-VMs}Configuration File Entries
Relating to Virtual Machines}
%%%%%%%%%%%%%%%%%%%%%%%%%%%%%%%%%%%%%%%%%%%%%%%%%%%%%%%%%%%%%%%%%%%%%%%%%%%

\index{configuration!virtual machine configuration variables}
These macros affect how HTCondor runs \SubmitCmd{vm} universe jobs on
a matched machine within the pool.
They specify items related to the \Condor{vm-gahp}.

\begin{description}
\label{param:VMGAHPServer}
\item[\Macro{VM\_GAHP\_SERVER}]
  The complete path and file name of the \Condor{vm-gahp}.
  There is no default value for this required configuration variable.

\label{param:VMGAHPLog}
\item[\Macro{VM\_GAHP\_LOG}]
  The complete path and file name of the \Condor{vm-gahp} log.
  If not specified on a Unix platform, the \Condor{starter}
  log will be used for \Condor{vm-gahp} log items. 
  There is no default value for this required configuration variable
  on Windows platforms.

\label{param:MaxVMGAHPLog}
\item[\Macro{MAX\_VM\_GAHP\_LOG}]
  Controls the maximum length (in bytes) to which the \Condor{vm-gahp} log
  will be allowed to grow.

\label{param:VMType}
\item[\Macro{VM\_TYPE}]
  Specifies the type of supported virtual machine software.
  It will be the value \verb@kvm@, \verb@xen@ or \verb@vmware@.
  There is no default value for this required configuration variable.

\label{param:VMMemory}
\item[\Macro{VM\_MEMORY}]
  An integer specifying the maximum amount of memory in Mbytes
  to be shared among the VM universe jobs run on this machine.

\label{param:VMMaxNumber}
\item[\Macro{VM\_MAX\_NUMBER}]
  An integer limit on the number of executing virtual machines.
  When not defined, the default value is the same \MacroNI{NUM\_CPUS}.
  When it evaluates to \Expr{Undefined},
  as is the case when not defined with a numeric value,
  no meaningful limit is imposed.

\label{param:VMStatusInterval}
\item[\Macro{VM\_STATUS\_INTERVAL}]
  An integer number of seconds that defaults to 60,
  representing the interval between job status checks by the
  \Condor{starter} to see if the job has finished.
  A minimum value of 30 seconds is enforced.

\label{param:VMGAHPReqTimeout}
\item[\Macro{VM\_GAHP\_REQ\_TIMEOUT}]
  An integer number of seconds that defaults to 300 (five minutes),
  representing the amount of time HTCondor will wait for a command issued
  from the \Condor{starter} to the \Condor{vm-gahp} to be completed.
  When a command times out, an error is reported to the \Condor{startd}.

\label{param:VMRecheckInterval}
\item[\Macro{VM\_RECHECK\_INTERVAL}]
  An integer number of seconds that defaults to 600 (ten minutes),
  representing the amount of time the \Condor{startd} waits after a
  virtual machine error as reported by the \Condor{starter},
  and before checking a final time on the status of the virtual machine.
  If the check fails, HTCondor disables starting any new vm universe jobs
  by removing the \Attr{VM\_Type} attribute from the machine ClassAd.

\label{param:VMSoftSuspend}
\item[\Macro{VM\_SOFT\_SUSPEND}]
  A boolean value that defaults to \Expr{False},
  causing HTCondor to free the memory of a vm universe job when
  the job is suspended.
  When \Expr{True}, the memory is not freed.

\label{param:VMUnivNobodyUser}
\item[\Macro{VM\_UNIV\_NOBODY\_USER}]
  Identifies a login name of a user with a home directory that
  may be used for job owner of a vm universe job.
  The \Login{nobody} user normally utilized when the job arrives
  from a different UID domain will not be allowed to invoke a VMware
  virtual machine.

\label{param:AlwaysVMUnivUseNobody}
\item[\Macro{ALWAYS\_VM\_UNIV\_USE\_NOBODY}]
  A boolean value that defaults to \Expr{False}.
  When \Expr{True}, all vm universe jobs (independent of their
  UID domain) will run as the user defined in \MacroNI{VM\_UNIV\_NOBODY\_USER}.

\label{param:VMNetworking}
\item[\Macro{VM\_NETWORKING}]
  A boolean variable describing if networking is supported.
  When not defined, the default value is \Expr{False}.

\label{param:VMNetworkingType}
\item[\Macro{VM\_NETWORKING\_TYPE}]
  A string describing the type of networking,
  required and relevant only when \MacroNI{VM\_NETWORKING} is \Expr{True}.
  Defined strings are
  \begin{verbatim}
    bridge
    nat
    nat, bridge
  \end{verbatim}

\label{param:VMNetworkingDefaultType}
\item[\Macro{VM\_NETWORKING\_DEFAULT\_TYPE}]
  Where multiple networking types are given in \MacroNI{VM\_NETWORKING\_TYPE},
  this optional configuration variable identifies which to use.
  Therefore, for 
  \begin{verbatim}
  VM_NETWORKING_TYPE = nat, bridge
  \end{verbatim}
  this variable may be defined as either \Expr{nat} or \Expr{bridge}.
  Where multiple networking types are given in \MacroNI{VM\_NETWORKING\_TYPE},
  and this variable is \emph{not} defined, a default of \Expr{nat}
  is used.

\label{param:VMNetworkingBridgeInterface}
\item[\Macro{VM\_NETWORKING\_BRIDGE\_INTERFACE}]
  For Xen and KVM only, a required string if bridge networking is to be
  enabled.  It specifies the networking interface that vm universe jobs
  will use.

\label{param:LibvirtXmlScript}
\item[\Macro{LIBVIRT\_XML\_SCRIPT}]
  For Xen and KVM only, a path and executable specifying a program.
  When the \Condor{vm-gahp} is ready to start a Xen or KVM 
  \SubmitCmd{vm} universe job, 
  it will invoke this program to generate the XML description of 
  the virtual machine,
  which it then provides to the virtualization software.
  The job ClassAd will be provided to this program via standard input. 
  This program should print the XML to standard output.
  If this configuration variable is not set,
  the \Condor{vm-gahp} will generate the XML itself. 
  The provided script in \File{\MacroUNI{LIBEXEC}/libvirt\_simple\_script.awk} 
  will generate the same XML that the \Condor{vm-gahp} would.

\label{param:LibvirtXmlScriptArgs}
\item[\Macro{LIBVIRT\_XML\_SCRIPT\_ARGS}]
  For Xen and KVM only, the command-line arguments to be given to 
  the program specified by  \MacroNI{LIBVIRT\_XML\_SCRIPT}. 

\end{description}

The following configuration variables are specific to the VMware
virtual machine software.

\begin{description}
\label{param:VMwarePerl}
\item[\Macro{VMWARE\_PERL}]
  The complete path and file name to \Prog{Perl}.
  There is no default value for this required variable.

\label{param:VMwareScript}
\item[\Macro{VMWARE\_SCRIPT}]
  The complete path and file name of the script that controls VMware.
  There is no default value for this required variable.

\label{param:VMwareNetworkingType}
\item[\Macro{VMWARE\_NETWORKING\_TYPE}]
  An optional string used in networking that the \Condor{vm-gahp}
  inserts into the VMware configuration file to define a networking type.
  Defined types are \Expr{nat} or \Expr{bridged}.
  If a default value is needed, the inserted string will be \Expr{nat}.

\label{param:VMwareNatNetworkingType}
\item[\Macro{VMWARE\_NAT\_NETWORKING\_TYPE}]
  An optional string used in networking that the \Condor{vm-gahp}
  inserts into the VMware configuration file to define a networking type.
  If nat networking is used, this variable's definition takes
  precedence over one defined by \MacroNI{VMWARE\_NETWORKING\_TYPE}.

\label{param:VMwareBridgeNetworkingType}
\item[\Macro{VMWARE\_BRIDGE\_NETWORKING\_TYPE}]
  An optional string used in networking that the \Condor{vm-gahp}
  inserts into the VMware configuration file to define a networking type.
  If bridge networking is used, this variable's definition takes
  precedence over one defined by \MacroNI{VMWARE\_NETWORKING\_TYPE}.

\label{param:VMwareLocalSettingsFile}
\item[\Macro{VMWARE\_LOCAL\_SETTINGS\_FILE}]
  The complete path and file name to a file, whose contents will be
  inserted into the VMware description file (i.e., the .vmx file) before
  HTCondor starts the virtual machine. This parameter is optional.

\end{description}

The following configuration variables are specific to the Xen
virtual machine software.

\begin{description}

\label{param:XenBootloader}
\item[\Macro{XEN\_BOOTLOADER}]
  A required full path and executable for the Xen bootloader,
  if the kernel image includes a disk image.

\end{description}

The following two macros affect the configuration of HTCondor where HTCondor is
running on a host machine, the host machine is running an
inner virtual machine,
and HTCondor is also running on that inner virtual machine.
These two variables have nothing to do with the \SubmitCmd{vm}
universe.

\begin{description}
\label{param:VMPHostMachine}
\item[\Macro{VMP\_HOST\_MACHINE}]
  A configuration variable for the inner virtual machine,
  which specifies the host name.

\label{param:VMPVMList}
\item[\Macro{VMP\_VM\_LIST}]
  For the host, 
  a comma separated list of the host names or IP addresses
  for machines running inner virtual machines on a host.
\end{description}

%%%%%%%%%%%%%%%%%%%%%%%%%%%%%%%%%%%%%%%%%%%%%%%%%%%%%%%%%%%%%%%%%%%%%%%%%%%
\subsection{\label{sec:HA-Config-File-Entries}Configuration File Entries
Relating to High Availability}
%%%%%%%%%%%%%%%%%%%%%%%%%%%%%%%%%%%%%%%%%%%%%%%%%%%%%%%%%%%%%%%%%%%%%%%%%%%

\index{configuration!high availability configuration variables}
These macros affect the high availability operation of HTCondor.

\begin{description}
\label{param:MasterHAList}
\item[\Macro{MASTER\_HA\_LIST}]
  Similar to \MacroNI{DAEMON\_LIST}, this macro defines a list of daemons that
  the \Condor{master} starts and keeps its watchful eyes on.
  However, the \MacroNI{MASTER\_HA\_LIST} daemons are run in a
  \emph{High Availability} mode.
  The list is a comma or space separated list of subsystem names
  (as listed in section~\ref{sec:HTCondor-Subsystem-Names}).
  For example,
  \begin{verbatim}
        MASTER_HA_LIST = SCHEDD
  \end{verbatim}

  The \emph{High Availability} feature allows for several \Condor{master}
  daemons (most likely on separate machines) to work together to
  insure that a particular service stays available.  These
  \Condor{master} daemons ensure that one and only one of them will
  have the listed daemons running.

  To use this feature, the lock URL must be set with
  \MacroNI{HA\_LOCK\_URL}.

  Currently, only file URLs are supported 
  (those with \File{file:\Dots}).
  The default value for \MacroNI{MASTER\_HA\_LIST} is 
  the empty string, which disables the feature.
  
\label{param:HALockURL}
\item[\Macro{HA\_LOCK\_URL}]
  This macro specifies the URL that the \Condor{master} processes use to
  synchronize for the \emph{High Availability} service.
  Currently, only file URLs are supported; for example,
  \File{file:/share/spool}.  Note that this URL must be identical
  for all \Condor{master} processes sharing this resource.  For
  \Condor{schedd} sharing, we recommend setting up \MacroNI{SPOOL}
  on an NFS share and having all \emph{High Availability}
  \Condor{schedd} processes sharing it,
  and setting the \MacroNI{HA\_LOCK\_URL} to point at this directory
  as well.  For example:
\begin{verbatim}
        MASTER_HA_LIST = SCHEDD
        SPOOL = /share/spool
        HA_LOCK_URL = file:/share/spool
        VALID_SPOOL_FILES = SCHEDD.lock
\end{verbatim}

  A separate lock is created for each \emph{High Availability} daemon.

  There is no default value for \MacroNI{HA\_LOCK\_URL}.

  Lock files are in the form \verb@<@SUBSYS\verb@>@.lock.
  \Condor{preen} is not currently aware of the lock files and will
  delete them if they are placed in the \MacroNI{SPOOL} directory,
  so be sure to add \verb@<@SUBSYS\verb@>@.lock to 
  \Macro{VALID\_SPOOL\_FILES} for each \emph{High Availability} daemon.

\label{param:HASubsysLockURL}
\item[\Macro{HA\_<SUBSYS>\_LOCK\_URL}]
  This macro controls the 
  \emph{High Availability} lock URL for a specific subsystem
  as specified in the configuration variable name,
  and it overrides the system-wide lock URL specified by
  \MacroNI{HA\_LOCK\_URL}.  If not defined for each subsystem,
  \MacroNI{HA\_<SUBSYS>\_LOCK\_URL} is ignored, and the value of
  \MacroNI{HA\_LOCK\_URL} is used.

\label{param:HALockHoldTime}
\item[\Macro{HA\_LOCK\_HOLD\_TIME}]
  This macro
  specifies the number of seconds that the \Condor{master} will hold the
  lock for each \emph{High Availability} daemon.
  Upon gaining the shared lock,
  the \Condor{master} will hold the lock for this number of seconds.
  Additionally, the \Condor{master} will periodically renew
  each lock as long as the \Condor{master} and the daemon are running.
  When the daemon dies, or the \Condor{master} exists, the
  \Condor{master} will immediately release the lock(s) it holds.

  \MacroNI{HA\_LOCK\_HOLD\_TIME} defaults to 3600 seconds (one hour).

\label{param:HASubsysLockHoldTime}
\item[\Macro{HA\_<SUBSYS>\_LOCK\_HOLD\_TIME}]
  This macro controls the \emph{High Availability} lock
  hold time for a specific subsystem
  as specified in the configuration variable name,
  and it overrides the system wide poll period specified by
  \MacroNI{HA\_LOCK\_HOLD\_TIME}.
  If not defined for each subsystem,
  \MacroNI{HA\_<SUBSYS>\_LOCK\_HOLD\_TIME} is ignored,
  and the value of \MacroNI{HA\_LOCK\_HOLD\_TIME} is used.

\label{param:HALockPollPeriod} 
\item[\Macro{HA\_POLL\_PERIOD}]
  This macro specifies how often the \Condor{master} polls the
  \emph{High Availability} locks to see if any locks are either stale
  (meaning not updated for \MacroNI{HA\_LOCK\_HOLD\_TIME} seconds),
  or have been released by the owning \Condor{master}.
  Additionally, the \Condor{master} renews any locks that it
  holds during these polls.

  \MacroNI{HA\_POLL\_PERIOD} defaults to 300 seconds (five minutes).

\label{param:HALockPollSubsysPeriod}
\item[\Macro{HA\_<SUBSYS>\_POLL\_PERIOD}]
  This macro controls the \emph{High Availability} poll period
  for a specific subsystem
  as specified in the configuration variable name,
  and it overrides the system wide poll period specified by
  \MacroNI{HA\_POLL\_PERIOD}.
  If not defined for each subsystem,
  \MacroNI{HA\_<SUBSYS>\_POLL\_PERIOD} is ignored,
  and the value of \MacroNI{HA\_POLL\_PERIOD} is used.

\label{param:MasterSubsysController}
\item[\Macro{MASTER\_<SUBSYS>\_CONTROLLER}]
  Used only in HA configurations involving the \Condor{had}.

  The \Condor{master} has the concept of a controlling and controlled
  daemon, typically
  with the \Condor{had} daemon serving as the controlling process.
  In this case, all \Condor{on} and \Condor{off} commands directed
  at controlled daemons are given to the controlling daemon, which
  then handles the command, and, when required, sends appropriate
  commands to the \Condor{master} to do the actual work.  This allows
  the controlling daemon to know the state of the controlled daemon.

  As of 6.7.14, this configuration variable must be specified for all
  configurations using \Condor{had}.
  To configure the \Condor{negotiator} controlled by \Condor{had}:

\begin{verbatim}
MASTER_NEGOTIATOR_CONTROLLER = HAD
\end{verbatim}

  The macro is named by substituting \MacroNI{<SUBSYS>}
  with the appropriate subsystem string as defined in
  section~\ref{sec:HTCondor-Subsystem-Names}.


\label{param:HADList}
\item[\Macro{HAD\_LIST}]
  A comma-separated list of all \Condor{had} daemons
  in the form \Expr{IP:port} or \Expr{hostname:port}.
  Each central manager machine that runs the \Condor{had} daemon
  should appear in this list.
  If \MacroNI{HAD\_USE\_PRIMARY} is set to \Expr{True},
  then the first machine in this list is the primary central
  manager, and all others in the list are backups.

  All central manager machines must be configured with 
  an identical \MacroNI{HAD\_LIST}.
  The machine addresses are identical to the addresses defined
  in \MacroNI{COLLECTOR\_HOST}.

%The following examples are all valid HAD\_LIST declarations: 
%
%HAD\_LIST =<132.68.37.104:10001>,<132.68.37.105:10002>,<132.68.37.106:1045>
%
%HAD\_LIST =132.68.37.104:10001,132.68.37.105:10002,132.68.37.106:1045
%
%HAD\_LIST=ds-r3.cs.technion.ac.il:10001,ds-r3.cs.technion.ac.il:10002,ds-r3.cs.technion.ac.il:1045
%

\label{param:HADUsePrimary}
\item[\Macro{HAD\_USE\_PRIMARY}]
  Boolean value to determine if the first machine in the 
  \MacroNI{HAD\_LIST} configuration variable is
  a primary central manager.
  Defaults to \Expr{False}.

\label{param:HADControllee}
\item[\Macro{HAD\_CONTROLLEE}]
  This variable is used to specify the name of the daemon which the
  \Condor{had} daemon controls.   This name should match the daemon
  name in the \Condor{master} daemon's \MacroNI{DAEMON\_LIST} definition.  
  The default value is \Expr{NEGOTIATOR}.

\label{param:HADConnectionTimeout}
\item[\Macro{HAD\_CONNECTION\_TIMEOUT}]
  The time (in seconds) that the \Condor{had} daemon waits before giving
  up on the establishment of a TCP connection.
  The failure of the communication connection
  is the detection mechanism for the failure of a central
  manager machine.
  For a LAN, a recommended value is 2 seconds.
  The use of authentication (by HTCondor) increases the connection
  time.
  The default value is 5 seconds.
  If this value is set too low,
  \Condor{had} daemons will incorrectly assume
  the failure of other machines.

\label{param:HADArgs}
\item[\Macro{HAD\_ARGS}]
  Command line arguments passed by the \Condor{master} daemon
  as it invokes the \Condor{had} daemon.
  To make high availability work, the \Condor{had} daemon
  requires the port number it is to use.
  This argument is of the form
  \begin{verbatim}
   -p $(HAD_PORT_NUMBER)
  \end{verbatim}
  where \MacroNI{HAD\_PORT\_NUMBER} is a helper configuration variable
  defined with the desired port number.
  Note that this port number must be the same value here as
  used in \MacroNI{HAD\_LIST}.
  There is no default value.


\label{param:HAD}
\item[\Macro{HAD}]
  The path to the \Condor{had} executable. Normally it is defined
  relative to \MacroUNI{SBIN}.
  This configuration variable has no default value.

\label{param:MaxHADLog}
\item[\Macro{MAX\_HAD\_LOG}]
  Controls the maximum length in bytes to which the \Condor{had}
  daemon log will be allowed to grow. It will grow to the specified length,
  then be saved to a file with the suffix \File{.old}. 
  The \File{.old}  file is overwritten each time the log is saved,
  thus the maximum space devoted to logging is twice the maximum length
  of this log file.
  A value of 0 specifies that this file may grow without bounds.
  The default is 1 Mbyte.

\label{param:HADDebug}
\item[\Macro{HAD\_DEBUG}]
  Logging level for the \Condor{had} daemon.
  See \MacroNI{<SUBSYS>\_DEBUG} for values.

\label{param:HADLog}
\item[\Macro{HAD\_LOG}]
  Full path and file name of the log file.
  There is no default value.

\label{param:ReplicationList}
\item[\Macro{REPLICATION\_LIST}]
  A comma-separated list of all \Condor{replication} daemons
  in the form \Expr{IP:port} or \Expr{hostname:port}.
  Each central manager machine that runs the \Condor{had} daemon
  should appear in this list.
  All potential central manager machines must be configured with
  an identical \MacroNI{REPLICATION\_LIST}.

\label{param:StateFile}
\item[\Macro{STATE\_FILE}]
  A full path and file name of the file protected by the replication
  mechanism.
  When not defined, the default path and file used is
  \begin{verbatim}
  $(SPOOL)/Accountantnew.log
  \end{verbatim}

\label{param:ReplicationInterval}
\item[\Macro{REPLICATION\_INTERVAL}]
  Sets how often the \Condor{replication} daemon initiates its tasks of
  replicating the \MacroUNI{STATE\_FILE}.
  It is defined in seconds and defaults to 300 (5 minutes).

\label{param:MaxTransferLifetime}
\item[\Macro{MAX\_TRANSFERER\_LIFETIME}]
  A timeout period within which the process that
  transfers the state file must complete its transfer.
  The recommended value is
  \Expr{2 * average size of state file / network rate}.
  It is defined in seconds and defaults to 300 (5 minutes).

\label{param:HADUpdateInterval}
\item[\Macro{HAD\_UPDATE\_INTERVAL}]
  Like \MacroNI{UPDATE\_INTERVAL},
  determines how often the \Condor{had} is to send a ClassAd update
  to the \Condor{collector}.
  Updates are also sent at each and every change in state.
  It is defined in seconds and defaults to 300 (5 minutes).

\label{param:HADUseReplication}
\item[\Macro{HAD\_USE\_REPLICATION}]
  A boolean value that defaults to \Expr{False}.
  When \Expr{True}, the use of \Condor{replication} daemons is enabled.

\label{param:ReplicationArgs}
\item[\Macro{REPLICATION\_ARGS}]
  Command line arguments passed by the \Condor{master} daemon
  as it invokes the \Condor{replication} daemon.
  To make high availability work, the \Condor{replication} daemon
  requires the port number it is to use.
  This argument is of the form
  \begin{verbatim}
  -p $(REPLICATION_PORT_NUMBER)
  \end{verbatim}
  where \MacroNI{REPLICATION\_PORT\_NUMBER} is a helper configuration
  variable defined with the desired port number.
  Note that this port number must be the same value as
  used in \MacroNI{REPLICATION\_LIST}.
  There is no default value.

\label{param:Replication}
\item[\Macro{REPLICATION}]
  The full path and file name of the \Condor{replication} executable.
  It is normally defined relative to \MacroUNI{SBIN}.
  There is no default value.

\label{param:MaxReplicationLog}
\item[\Macro{MAX\_REPLICATION\_LOG}]
  Controls the maximum length in bytes to which the \Condor{replication}
  daemon log will be allowed to grow. It will grow to the specified length,
  then be saved to a file with the suffix \File{.old}.
  The \File{.old}  file is overwritten each time the log is saved,
  thus the maximum space devoted to logging is twice the maximum length
  of this log file.
  A value of 0 specifies that this file may grow without bounds.
  The default is 1 Mbyte.

\label{param:ReplicationDebug}
\item[\Macro{REPLICATION\_DEBUG}]
  Logging level for the \Condor{replication} daemon.
  See \MacroNI{<SUBSYS>\_DEBUG} for values.

\label{param:ReplicationLog}
\item[\Macro{REPLICATION\_LOG}]
  Full path and file name to the log file.
  There is no default value.

\label{param:Transferer}
\item[\Macro{TRANSFERER}]
  The full path and file name of the \Condor{transferer} executable.
  Versions of HTCondor previous to 7.2.2 hard coded the location
  as \File{\MacroUNI{RELEASE\_DIR}/sbin/condor\_transferer}.
  This is now the default value.
  The future default value is likely to change, 
  and be defined relative to \MacroUNI{SBIN}.

\label{param:TransfererLog}
\item[\Macro{TRANSFERER\_LOG}]
  Full path and file name to the log file.
  There is no default value for this variable; a definition is required
  if the \Condor{replication} daemon does a file transfer.

\label{param:TransfererDebug}
\item[\Macro{TRANSFERER\_DEBUG}]
  Logging level for the \Condor{transferer} daemon.
  See \MacroNI{<SUBSYS>\_DEBUG} for values.

\label{param:MaxTransfererLog}
\item[\Macro{MAX\_TRANSFERER\_LOG}]
  Controls the maximum length in bytes to which the \Condor{transferer}
  daemon log will be allowed to grow.
  A value of 0 specifies that this file may grow without bounds.
  The default is 1 Mbyte.

\end{description}

%%%%%%%%%%%%%%%%%%%%%%%%%%%%%%%%%%%%%%%%%%%%%%%%%%%%%%%%%%%%%%%%%%%%%%
\subsection{\label{sec:MyProxy-Config-File-Entries}MyProxy
Configuration File Macros}
%%%%%%%%%%%%%%%%%%%%%%%%%%%%%%%%%%%%%%%%%%%%%%%%%%%%%%%%%%%%%%%%%%%%%%
 
In some cases, HTCondor can autonomously refresh GSI certificate proxies
via \Prog{MyProxy}, available from
\URL{http://myproxy.ncsa.uiuc.edu/}.

\begin{description}

\label{param:MyProxyGetDelegation}
\item[\Macro{MYPROXY\_GET\_DELEGATION}]
  The full path name to the
  \Prog{myproxy-get-delegation} executable, installed as part of the
  \Prog{MyProxy} software.  Often, it is necessary to wrap the actual
  executable with a script that sets the environment, such as the
  \MacroNI{LD\_LIBRARY\_PATH}, correctly.  If this macro is defined,
  HTCondor-G and \Condor{credd} will have the capability to autonomously
  refresh proxy certificates.  By default, this macro is undefined.

\end{description}

%%%%%%%%%%%%%%%%%%%%%%%%%%%%%%%%%%%%%%%%%%%%%%%%%%%%%%%%%%%%%%%%%%%%%%
\subsection{\label{sec:API-Config-File-Entries}
Configuration File Macros Affecting APIs}
%%%%%%%%%%%%%%%%%%%%%%%%%%%%%%%%%%%%%%%%%%%%%%%%%%%%%%%%%%%%%%%%%%%%%%

\begin{description}

\label{param:EnableSoap}
\item[\Macro{ENABLE\_SOAP}]
  A boolean value that defaults to \Expr{False}.
  When \Expr{True}, HTCondor daemons will respond to HTTP PUT commands
  as if they were SOAP calls. When \Expr{False},
  all HTTP PUT commands are denied.

\label{param:EnableWebServer}
\item[\Macro{ENABLE\_WEB\_SERVER}]
  A boolean value that defaults to \Expr{False}.
  When \Expr{True}, HTCondor daemons will respond to HTTP GET commands,
  and send the static files sitting in the subdirectory defined
  by the configuration variable \MacroNI{WEB\_ROOT\_DIR}.
  In addition, web commands are considered a READ command,
  so the client will be checked by host-based security.

\label{param:SoapLeaveInQueue}
\item[\Macro{SOAP\_LEAVE\_IN\_QUEUE}]
  A boolean expression that when \Expr{True},
  causes a job in the completed state to remain in the queue,
  instead of being removed based on the completion of file transfer.
  If provided, this expression will be logically ANDed with the
  default behavior of leaving the job in the queue until \Expr{FilesRetrieved}
  becomes \Expr{True}.

\label{param:WebRootDir}
\item[\Macro{WEB\_ROOT\_DIR}]
  A complete path to the directory containing all the files served
  by the web server.

\label{param:SubsysEnableSoapSSL}
\item[\MacroB{<SUBSYS>\_ENABLE\_SOAP\_SSL}]
  \index{SUBSYS\_ENABLE\_SOAP\_SSL macro@\texttt{<SUBSYS>\_ENABLE\_SOAP\_SSL} macro}
  A boolean value that defaults to \Expr{False}.
  When \Expr{True}, enables SOAP over SSL for the specified
  \MacroNI{<SUBSYS>}.
  Any specific \MacroNI{<SUBSYS>\_ENABLE\_SOAP\_SSL} setting overrides
  the value of \MacroNI{ENABLE\_SOAP\_SSL}.

\label{param:EnableSoapSSL}
\item[\Macro{ENABLE\_SOAP\_SSL}]
  A boolean value that defaults to \Expr{False}.
  When \Expr{True}, enables SOAP over SSL for all daemons.

\label{param:SubsysSoapSSLPort}
\item[\MacroB{<SUBSYS>\_SOAP\_SSL\_PORT}]
  \index{SUBSYS\_SOAP\_SSL\_PORT macro@\texttt{<SUBSYS>\_SOAP\_SSL\_PORT} macro}
  The port number on which SOAP over SSL messages are
  accepted, when SOAP over SSL is enabled.
  The \MacroNI{<SUBSYS>} must be specified, because multiple daemons
  running on a single machine may not share a port.
  This parameter is required when SOAP over SSL is enabled.
  There is no default value.

  The macro is named by substituting \MacroNI{<SUBSYS>}
  with the appropriate subsystem string as defined in
  section~\ref{sec:HTCondor-Subsystem-Names}.

\label{param:SoapSSLServerKeyfile}
\item[\Macro{SOAP\_SSL\_SERVER\_KEYFILE}]
  The complete path and file name to specify the daemon's
  identity, as used in authentication when SOAP over SSL is enabled.
  The file is to be  an OpenSSL PEM file containing a certificate
  and private key.
  This parameter is required when SOAP over SSL is enabled.
  There is no default value.

\label{param:SoapSSLServerKeyfilePassword}
\item[\Macro{SOAP\_SSL\_SERVER\_KEYFILE\_PASSWORD}]
  An optional complete path and file name to specify
  a password for unlocking the daemon's private key.
  There is no default value.

\label{param:SoapSSLCaFile}
\item[\Macro{SOAP\_SSL\_CA\_FILE}]
  The complete path and file name to specify 
  a file containing certificates of trusted Certificate Authorities (CAs).
  Only clients who present a certificate signed by a trusted
  CA will be authenticated.
  When SOAP over SSL is enabled, this parameter or
  \Macro{SOAP\_SSL\_CA\_DIR} must be set.
  There is no default value.

\label{param:SoapSSLCaDir}
\item[\Macro{SOAP\_SSL\_CA\_DIR}]
  The complete path to a directory
  containing certificates of trusted Certificate Authorities (CAs).
  Only clients who present a certificate signed by a trusted
  CA will be authenticated.
  When SOAP over SSL is enabled, this variable or the variable
  \Macro{SOAP\_SSL\_CA\_FILE} must be defined.
  There is no default value.

\label{param:SoapSSLDhFile}
\item[\Macro{SOAP\_SSL\_DH\_FILE}]
  An optional complete path and file name to a DH file
  containing keys for a DH key exchange.
  There is no default value.

\label{param:SoapSslSkipHostCheck}
\item[\Macro{SOAP\_SSL\_SKIP\_HOST\_CHECK}]
  When a SOAP server is authenticated via SSL, the server's host name
  is normally compared with the host name contained in the server's
  X.509 credential. If the two do not match, authentication fails.
  When this boolean variable is set to \Expr{True},
  the host name comparison is disabled.
  The default value is \Expr{False}.

\end{description}



%%%%%%%%%%%%%%%%%%%%%%%%%%%%%%%%%%%%%%%%%%%%%%%%%%%%%%%%%%%%%%%%%%%%%%
%\subsection{\label{sec:Stork-Config-File-Entries}Stork Configuration
%File Macros}
%%%%%%%%%%%%%%%%%%%%%%%%%%%%%%%%%%%%%%%%%%%%%%%%%%%%%%%%%%%%%%%%%%%%%%
%
%\begin{description}
%
%\label{param:StorkMaxNumJobs}
%\item[\Macro{STORK\_MAX\_NUM\_JOBS}]
  %An integer limit on the number of concurrent data placement jobs
  %handled by Stork.  The default value when not defined is 10.
%
%\label{param:StorkMaxRetry}
%\item[\Macro{STORK\_MAX\_RETRY}]
  %An integer limit on the
  %number of attempts for a single data placement job.  For data transfers,
  %this includes transfer attempts on the primary protocol, all
  %alternate protocols, and all retries.
  %The default value when not defined is 10.
%
%\label{param:StorkMaxDelayInMinutes}
%\item[\Macro{STORK\_MAXDELAY\_INMINUTES}]
  %An integer limit (in minutes) on the run time for a data placement job,
  %after which the job is considered failed.
  %The default value when not defined is 10,
  %and the minimum legal value is 1.
%
%\label{param:StorkTmpCredDir}
%\item[\Macro{STORK\_TMP\_CRED\_DIR}]
  %The full path to the temporary credential storage directory used by Stork.
  %The default value is \File{/tmp} when not defined. 
%
%\label{param:StorkModuleDir}
%\item[\Macro{STORK\_MODULE\_DIR}]
  %The full path to the directory containing Stork modules.
  %The default value when not defined is 
  %as defined by \MacroUNI{LIBEXEC}.  It is a fatal error for
  %both \MacroNI{STORK\_MODULE\_DIR} and \MacroNI{LIBEXEC} to be undefined.
%
%\label{param:CreddSuperUsers}
%\item[\Macro{CRED\_SUPER\_USERS}]
  %Access to a stored credential is
  %restricted to the user who submitted the credential, and any user
  %names specified in this macro.  The format is a space or comma
  %separated list of user names which are valid on the \Stork{credd}
  %host.
  %The default value of this macro is \Expr{root} on Unix systems, and
  %\Expr{Administrator} on Windows systems.
%
%\label{param:CredStoreDir}
%\item[\Macro{CRED\_STORE\_DIR}]
  %Directory for storing credentials.  This
  %directory must exist prior to starting \Stork{credd}.  It is highly
  %recommended to restrict access permissions to \emph{only} the
  %directory owner.
  %The default value is \Expr{\$(SPOOL\_DIR)/cred}.
%
%\label{param:CredIndexFile}
%\item[\Macro{CRED\_INDEX\_FILE}]
  %Index file path of saved credentials.
  %This file will be automatically created if it does not exist.
  %The default value is \Expr{\$(CRED\_STORE\_DIR)/cred-index}.
%
%\label{param:DefaultCredExpireThreshold}
%\item[\Macro{DEFAULT\_CRED\_EXPIRE\_THRESHOLD}]
  %\Stork{credd} will attempt
  %to refresh credentials when their remaining lifespan is less than this
  %value.
  %Units = seconds.  Default value = 3600 seconds (1 hour).
%
%\label{param:CredCheckInterval}
%\item[\Macro{CRED\_CHECK\_INTERVAL}]
  %\Stork{credd} periodically checks
  %remaining lifespan of stored credentials, at this interval.
  %Units = seconds.  Default value = 60 seconds (1 minute).
%
%\end{description}

%%%%%%%%%%%%%%%%%%%%%%%%%%%%%%%%%%%%%%%%%%%%%%%%%%%%%%%%%%%%%%%%%%%%%%%%%%%
\subsection{\label{sec:Config-ssh-to-job}Configuration File Entries
Relating to \Condor{ssh\_to\_job}}
%%%%%%%%%%%%%%%%%%%%%%%%%%%%%%%%%%%%%%%%%%%%%%%%%%%%%%%%%%%%%%%%%%%%%%%%%%%

\index{configuration!condor\_ssh\_to\_job configuration variables}
These macros affect how HTCondor deals with \Condor{ssh\_to\_job},
a tool that allows users to interactively debug jobs.  
With these configuration variables, 
the administrator can control who can use the tool,
and how the \Prog{ssh} programs are invoked.
The manual page for \Condor{ssh\_to\_job} is at
section~\ref{man-condor-ssh-to-job}.

\begin{description}
\label{param:EnableSSHToJob}
\item[\Macro{ENABLE\_SSH\_TO\_JOB}]
  A boolean expression read by the \Condor{starter},
  that when \Expr{True} allows
  the owner of the job or a queue super user on the \Condor{schedd}
  where the job was submitted to connect to the job via \Prog{ssh}.
  The expression may refer to attributes of both the job and 
  the machine ClassAds.
  The job ClassAd attributes may be referenced by using the prefix
  \Expr{TARGET.},
  and the machine ClassAd attributes may be referenced by using the prefix
  \Expr{MY.}.
  When \Expr{False},
  it prevents \Condor{ssh\_to\_job} from starting an \Prog{ssh} session.
  The default value is \Expr{True}.

\label{param:ScheddEnableSSHToJob}
\item[\Macro{SCHEDD\_ENABLE\_SSH\_TO\_JOB}]
  A boolean expression read by the \Condor{schedd},
  that when \Expr{True} allows the owner of the job or a queue super user 
  to connect to the job via \Prog{ssh} if the execute machine also
  allows \Condor{ssh\_to\_job} access (see \MacroNI{ENABLE\_SSH\_TO\_JOB}).
  The expression may refer to attributes of only the job ClassAd.
  When \Expr{False},
  it prevents \Condor{ssh\_to\_job} from starting an
  \Prog{ssh} session for all jobs managed by the \Condor{schedd}.
  The default value is \Expr{True}.

\label{param:SSHToJobSSHClientCmd}
\item[\Macro{SSH\_TO\_JOB\_<SSH-CLIENT>\_CMD}]
  A string read by the \Condor{ssh\_to\_job} tool.
  It specifies the command and arguments to use when invoking
  the program specified by \MacroNI{<SSH-CLIENT>}.
  Values substituted for the placeholder \MacroNI{<SSH-CLIENT>} may be
  \verb@SSH@, \verb@SFTP@, \verb@SCP@, or any other \Prog{ssh} client capable
  of using a command as a proxy for the connection to \Prog{sshd}.
  The entire command plus arguments string is enclosed in double quote marks.
  Individual arguments may be quoted with single quotes,
  using the same syntax as for arguments in a \Condor{submit} file.
  The following substitutions are made within the arguments:

\begin{description}
  \item \verb@%h@: is substituted by the remote host
  \item \verb@%i@: is substituted by the ssh key
  \item \verb@%k@: is substituted by the known hosts file
  \item \verb@%u@: is substituted by the remote user
  \item \verb@%x@: is substituted by a proxy command suitable for use with the \Prog{OpenSSH}
  ProxyCommand option
  \item \verb@%%@:  is substituted by the percent mark character
\end{description}

  The default string is:\\
  \Expr{"ssh -oUser=\%u -oIdentityFile=\%i -oStrictHostKeyChecking=yes -oUserKnownHostsFile=\%k -oGlobalKnownHostsFile=\%k -oProxyCommand=\%x \%h"}

  When the \MacroNI{<SSH-CLIENT>} is \Prog{scp}, \%h is omitted.


\label{param:SSHToJobSSHD}
\item[\Macro{SSH\_TO\_JOB\_SSHD}]
  The path and executable name of the \Prog{ssh} daemon.
  The value is read by the \Condor{starter}.
  The default value is \File{/usr/sbin/sshd}.

\label{param:SSHToJobSSHDArgs}
\item[\Macro{SSH\_TO\_JOB\_SSHD\_ARGS}]
  A string, read by the \Condor{starter} that specifies the command-line
  arguments to be passed to the \Prog{sshd} to handle an incoming ssh
  connection on its \File{stdin} or \File{stdout} streams in inetd mode.
  Enclose the entire arguments string in double quote marks.
  Individual arguments may be quoted with single quotes,
  using the same syntax as
  for arguments in an HTCondor submit description file.
  Within the arguments, 
  the characters \verb@%f@ are replaced by the path to the \Prog{sshd}
  configuration file
  the characters \verb@%%@ are replaced by a single percent character.
  The default value is the string \verb@"-i -e -f %f"@.

\label{param:SSHToJobSSHDConfigTemplate}
\item[\Macro{SSH\_TO\_JOB\_SSHD\_CONFIG\_TEMPLATE}]
  A string, read by the \Condor{starter} that specifies 
  the path and file name of an \Prog{sshd} configuration template file.
  The template is turned into an \Prog{sshd}
  configuration file by replacing macros within the template that
  specify such things as the paths to key files.
  The macro replacement
  is done by the script \Expr{\$(LIBEXEC)/condor\_ssh\_to\_job\_sshd\_setup}.
  The default value is 
  \Expr{\$(LIB)/condor\_ssh\_to\_job\_sshd\_config\_template}.

\label{param:SSHToJobSSHKeygen}
\item[\Macro{SSH\_TO\_JOB\_SSH\_KEYGEN}]
  A string, read by the \Condor{starter} that specifies 
  the path to \Prog{ssh\_keygen}, the program used to create ssh keys.

\label{param:SSHToJobSSHKeygenArgs}
\item[\Macro{SSH\_TO\_JOB\_SSH\_KEYGEN\_ARGS}]
  A string, read by the \Condor{starter} that specifies 
  the command-line arguments to be passed to the \Prog{ssh\_keygen}
  to generate an ssh key.
  Enclose the entire arguments string in double quotes.
  Individual arguments may be quoted with single quotes, using the same
  syntax as for arguments in an HTCondor submit description file.
  Within the arguments, 
  the characters \verb@%f@ are replaced by the path to the key file to be
  generated,
  and the characters \verb@%%@ are replaced by a single percent character.
  The default value is the string
  \verb|"-N '' -C '' -q -f %f -t rsa"|.
  If the user specifies additional
  arguments with the command
  \verb@condor_ssh_to_job -keygen-options@,
  then those arguments are placed after the arguments specified by
  the value of \MacroNI{SSH\_TO\_JOB\_SSH\_KEYGEN\_ARGS}.

\end{description}

%%%%%%%%%%%%%%%%%%%%%%%%%%%%%%%%%%%%%%%%%%%%%%%%%%%%%%%%%%%%%%%%%%%%%%%%%%%
\subsection{\label{sec:Config-rooster}\Condor{rooster} Configuration File Macros}
%%%%%%%%%%%%%%%%%%%%%%%%%%%%%%%%%%%%%%%%%%%%%%%%%%%%%%%%%%%%%%%%%%%%%%%%%%%

\index{configuration!condor\_rooster configuration variables}
\Condor{rooster} is an optional daemon that may be added to the
\Condor{master} daemon's \MacroNI{DAEMON\_LIST}.
It is responsible for waking up
hibernating machines when their \Macro{UNHIBERNATE} expression becomes
\Expr{True}.
In the typical case, a pool runs a single instance of
\Condor{rooster} on the central manager.
However, if the network topology requires that 
Wake On LAN packets be sent to specific machines from different locations,
\Condor{rooster} can be run on any
machine(s) that can read from the pool's \Condor{collector} daemon.

For \Condor{rooster} to wake up hibernating machines, the collecting
of offline machine ClassAds must be enabled.  See variable
\Macro{COLLECTOR\_PERSISTENT\_AD\_LOG} on page~\pageref{param:CollectorPersistentAdLog}
for details on how to do this.

\begin{description}

\label{param:RoosterInterval}
\item[\Macro{ROOSTER\_INTERVAL}]
  The integer number of seconds between checks for offline machines that
  should be woken.  The default value is 300.

\label{param:RoosterMaxUnhibernate}
\item[\Macro{ROOSTER\_MAX\_UNHIBERNATE}]
  An integer specifying the maximum number of machines to wake up per
  cycle.  The default value of 0 means no limit.

\label{param:RoosterUnhibernate}
\item[\Macro{ROOSTER\_UNHIBERNATE}]
  A boolean expression that specifies which machines should be woken up.
  The default expression is \Expr{Offline \&\& Unhibernate}.
  If network topology or other considerations demand that some machines
  in a pool be woken up by one instance of \Condor{rooster}, 
  while others be woken up by a different instance,
  \Macro{ROOSTER\_UNHIBERNATE} may be set locally such that it is
  different for the two instances of \Condor{rooster}.
  In this way, the different instances will only
  try to wake up their respective subset of the pool.

\label{param:RoosterUnhibernateRank}
\item[\Macro{ROOSTER\_UNHIBERNATE\_RANK}] A ClassAd expression
 specifying which machines should be woken up first in a given cycle.
  Higher ranked machines are woken first.  If the number of machines
 to be woken up is limited by \Macro{ROOSTER\_MAX\_UNHIBERNATE}, the
 rank may be used for determining which machines are woken before
 reaching the limit.

\label{param:RoosterWakeupCmd}
\item[\Macro{ROOSTER\_WAKEUP\_CMD}]
  A string representing the command line invoked by \Condor{rooster}
  that is to wake up a machine.
  The command and any arguments should be enclosed in double quote marks,
  the same as \SubmitCmd{arguments} syntax in an HTCondor submit description file.
  The default value is \verb@"$(BIN)/condor_power -d -i"@.
  The command is expected to read
  from its standard input a ClassAd representing the offline machine.

\end{description}

%%%%%%%%%%%%%%%%%%%%%%%%%%%%%%%%%%%%%%%%%%%%%%%%%%%%%%%%%%%%%%%%%%%%%%%%%%%
\subsection{\label{sec:Config-shared-port}\Condor{shared\_port} Configuration File Macros}
%%%%%%%%%%%%%%%%%%%%%%%%%%%%%%%%%%%%%%%%%%%%%%%%%%%%%%%%%%%%%%%%%%%%%%%%%%%
\index{configuration!condor\_shared\_port configuration variables}
These configuration variables affect the \Condor{shared\_port} daemon.
For general discussion of \Condor{shared\_port},
see~\pageref{sec:shared-port-daemon}.

\begin{description}

\label{param:SharedPortDaemonAdFile}
\item[\Macro{SHARED\_PORT\_DAEMON\_AD\_FILE}]
  This specifies the full path and name of a file used to publish the
  address of \Condor{shared\_port}.  This file is read by the other
  daemons that have \Expr{USE\_SHARED\_PORT=True} and which are therefore
  sharing the same port.  The default typically does not need to be changed.

\label{param:SharedPortMaxWorkers}
\item[\Macro{SHARED\_PORT\_MAX\_WORKERS}] An integer that specifies
 the maximum number of sub-processes created by \Condor{shared\_port}
 while servicing requests to connect to the daemons that are sharing the port.
 The default is 50.

\label{param:DaemonSocketDir}
\item[\Macro{DAEMON\_SOCKET\_DIR}] This specifies the directory where
 Unix versions of HTCondor daemons will create named sockets so that incoming
 connections can be forwarded to them by \Condor{shared\_port}.  If
 this directory does not exist, it will be created. The maximum length
 of named socket paths plus names is restricted by the operating system,
 so it is important that this path not exceed 90 characters.

Write access to this directory grants permission to receive
 connections through the shared port.  By default, the directory is
 created to be owned by HTCondor and is made to be only writable by
 HTCondor.  One possible reason to broaden access to this directory is
 if execute nodes are accessed via CCB and the submit node is behind a
 firewall with only one open port (the port assigned to
 \Condor{shared\_port}).  In this case, commands that interact with
 the execute node such as \Condor{ssh\_to\_job} will not be able to
 operate unless run by a user with write access to
 \MacroNI{DAEMON\_SOCKET\_DIR}.  In this case, one could grant
 tmp-like permissions to this directory so that all users can receive
 CCB connections back through the firewall.  (But consider the wisdom
 of having a firewall in the first place if you are going to
 circumvent it in this way.)  The default
 \MacroNI{DAEMON\_SOCKET\_DIR} is \verb|$(LOCK)/daemon_sock|.  This
 directory must be on a local file system that supports named sockets.

\label{param:SharedPortArgs}
\item[\Macro{SHARED\_PORT\_ARGS}] Like all daemons started by
 \Condor{master}, \Condor{shared\_port} arguments can be customized.
  One reason to do this is to specify the port number that
 \Condor{shared\_port} should use.  For example, the following line
 configures \Condor{shared\_port} to use port 4080.

\begin{verbatim}
SHARED_PORT_ARGS = -p 4080
\end{verbatim}

If no port is specified, a port will be dynamically chosen; it may be
different each time HTCondor is started.  In order to decrease the
duration of possible communication errors resulting from HTCondor or just
\Condor{shared\_port} itself restarting, it is recommended that a fixed
port be used instead of a dynamic one.
\end{description}

%%%%%%%%%%%%%%%%%%%%%%%%%%%%%%%%%%%%%%%%%%%%%%%%%%%%%%%%%%%%%%%%%%%%%%%%%%%
\subsection{\label{sec:Config-hooks}Configuration File Entries Relating to Hooks}
%%%%%%%%%%%%%%%%%%%%%%%%%%%%%%%%%%%%%%%%%%%%%%%%%%%%%%%%%%%%%%%%%%%%%%%%%%%
\index{configuration!hook configuration variables}
\index{Job Router}

These macros control the various hooks that interact with HTCondor.
Currently, there are two independent sets of hooks.
One is a set of fetch work hooks, some of which are invoked by
the \Condor{startd} to optionally fetch work,
and some are invoked by the \Condor{starter}.
See section~\ref{sec:job-hooks} on page~\pageref{sec:job-hooks} on
Job Hooks for more details.
The other set replace functionality of the \Condor{job\_router} daemon.
Documentation for the \Condor{job\_router} daemon is in
section~\ref{sec:JobRouter} on page~\pageref{sec:JobRouter}.

\begin{description}

\label{param:SlotNJobHookKeyword}
\item[\Macro{SLOT<N>\_JOB\_HOOK\_KEYWORD}]
  For the fetch work hooks,
  the keyword used to define which set of hooks a particular
  compute slot should invoke.
  The value of \verb@<N>@ is replaced by the slot
  identification number. For example, on slot 1, the variable name will be
  called \MacroNI{[SLOT1\_JOB\_HOOK\_KEYWORD}.
  There is no default keyword.
  Sites that wish to use these job hooks must explicitly define the
  keyword and the corresponding hook paths.

\label{param:StartdJobHookKeyword}
\item[\Macro{STARTD\_JOB\_HOOK\_KEYWORD}]
  For the fetch work hooks,
  the keyword used to define which set of hooks a particular
  \Condor{startd} should invoke.
  This setting is only used if a slot-specific keyword is not defined
  for a given compute slot.
  There is no default keyword.
  Sites that wish to use job hooks must explicitly define the
  keyword and the corresponding hook paths.

\label{param:HookFetchWork}
\item[\Macro{<Keyword>\_HOOK\_FETCH\_WORK}]
  For the fetch work hooks,
  the full path to the program to invoke whenever the \Condor{startd}
  wants to fetch work.
  \MacroNI{<Keyword>} is the hook keyword defined to distinguish
  between sets of hooks.
  There is no default.

\label{param:HookReplyFetch}
\item[\Macro{<Keyword>\_HOOK\_REPLY\_FETCH}]
  For the fetch work hooks,
  the full path to the program to invoke when the hook defined by
  \MacroNI{<Keyword>\_HOOK\_FETCH\_WORK}  returns data and the the \Condor{startd}
  decides if it is going to accept the fetched job or not.
  \MacroNI{<Keyword>} is the hook keyword defined to distinguish
  between sets of hooks.

\label{param:HookReplyClaim}
\item[\Macro{<Keyword>\_HOOK\_REPLY\_CLAIM}]
  For the fetch work hooks,
  the full path to the program to invoke whenever the \Condor{startd}
  finishes fetching a job and decides what to do with it.
  \MacroNI{<Keyword>} is the hook keyword defined to distinguish

  between sets of hooks.
  There is no default.

\label{param:HookPrepareJob}
\item[\Macro{<Keyword>\_HOOK\_PREPARE\_JOB}]
  For the fetch work hooks,
  the full path to the program invoked by the \Condor{starter} before it
  runs the job.
  \MacroNI{<Keyword>} is the hook keyword defined to distinguish
  between sets of hooks.

\label{param:HookUpdateJobInfo}
\item[\Macro{<Keyword>\_HOOK\_UPDATE\_JOB\_INFO}]
  This configuration variable is used by both fetch work hooks and
  by \Condor{job\_router} hooks.

  For the fetch work hooks,
  the full path to the program invoked by the \Condor{starter} periodically
  as the job runs, allowing the \Condor{starter} to present an updated
  and augmented job ClassAd to the program.
  See section~\ref{sec:job-hooks-hooks} on page~\pageref{sec:job-hooks-hooks}
  for the list of additional attributes included.
  When the job is first invoked, the \Condor{starter} will invoke the program
  after \MacroUNI{STARTER\_INITIAL\_UPDATE\_INTERVAL} seconds.
  Thereafter, the \Condor{starter} will invoke the program every 
  \MacroUNI{STARTER\_UPDATE\_INTERVAL} seconds.
  \MacroNI{<Keyword>} is the hook keyword defined to distinguish
  between sets of hooks.

  As a Job Router hook,
  the full path to the program invoked when the Job Router polls the status
  of routed jobs at intervals set by \MacroNI{JOB\_ROUTER\_POLLING\_PERIOD}.
  \MacroNI{<Keyword>} is the hook keyword defined by
  \MacroNI{JOB\_ROUTER\_HOOK\_KEYWORD} to identify the hooks.

\label{param:HookEvictClaim}
\item[\Macro{<Keyword>\_HOOK\_EVICT\_CLAIM}]
  For the fetch work hooks,
  the full path to the program to invoke whenever the \Condor{startd}
  needs to evict a fetched claim.
  \MacroNI{<Keyword>} is the hook keyword defined to distinguish
  between sets of hooks.
  There is no default.

\label{param:HookJobExit}
\item[\Macro{<Keyword>\_HOOK\_JOB\_EXIT}]
  For the fetch work hooks,
  the full path to the program invoked by the \Condor{starter}
  whenever a job exits,
  either on its own or when being evicted from an execution slot. 
  \MacroNI{<Keyword>} is the hook keyword defined to distinguish
  between sets of hooks.

\label{param:HookJobExitTimeout}
\item[\Macro{<Keyword>\_HOOK\_JOB\_EXIT\_TIMEOUT}]
  For the fetch work hooks,
  the number of seconds the \Condor{starter} will wait for the hook
  defined by \MacroNI{<Keyword>\_HOOK\_JOB\_EXIT} hook to exit,
  before continuing with job clean up.  Defaults to 30 seconds.
  \MacroNI{<Keyword>} is the hook keyword defined to distinguish
  between sets of hooks.

\label{param:FetchWorkDelay}
\item[\Macro{FetchWorkDelay}]
  An expression that defines the number of seconds that the
  \Condor{startd} should wait after an invocation of
  \Macro{<Keyword>\_HOOK\_FETCH\_WORK} completes before the hook should be
  invoked again.
  The expression is evaluated in the context of the slot ClassAd, and
  the ClassAd of the currently running job (if any).
  The expression must evaluate to an integer.
  If not defined, the \Condor{startd} will wait 300 seconds (five
  minutes) between attempts to fetch work.
  For more information about this expression, see
  section~\ref{sec:job-hooks-fetch-work-delay} on
  page~\pageref{sec:job-hooks-fetch-work-delay}.

\label{param:JobRouterHookKeyword}
\item[\Macro{JOB\_ROUTER\_HOOK\_KEYWORD}]
  For the Job Router hooks,
  the keyword used to define the set of hooks the \Condor{job\_router}
  is to invoke to replace functionality of routing translation.
  There is no default keyword.
  Use of these hooks requires the explicit definition of the
  keyword and the corresponding hook paths.

\label{param:HookTranslateJob}
\item[\Macro{<Keyword>\_HOOK\_TRANSLATE\_JOB}]
  A Job Router hook,
  the full path to the program invoked when the Job Router has determined
  that a job meets the definition for a route.  
  This hook is responsible for doing the transformation of the job.
  \MacroNI{<Keyword>} is the hook keyword defined by
  \MacroNI{JOB\_ROUTER\_HOOK\_KEYWORD} to identify the hooks.

\label{param:HookJobFinalize}
\item[\Macro{<Keyword>\_HOOK\_JOB\_FINALIZE}]
  A Job Router hook,
  the full path to the program invoked when the Job Router has determined
  that the job completed.
  \MacroNI{<Keyword>} is the hook keyword defined by
  \MacroNI{JOB\_ROUTER\_HOOK\_KEYWORD} to identify the hooks.

\label{param:HookJobCleanup}
\item[\Macro{<Keyword>\_HOOK\_JOB\_CLEANUP}]
  A Job Router hook,
  the full path to the program invoked when the Job Router finishes 
  managing the job.
  \MacroNI{<Keyword>} is the hook keyword defined by
  \MacroNI{JOB\_ROUTER\_HOOK\_KEYWORD} to identify the hooks.

\end{description}

The following macros describe the \Term{Daemon ClassAd Hook}
capabilities of HTCondor.  
The Daemon ClassAd Hook mechanism is used to run executables (called jobs) 
directly from the \Condor{startd} and \Condor{schedd} daemons.
The output from the jobs is incorporated into the machine ClassAd
generated by the respective daemon. 
The mechanism is described in section~\ref{sec:daemon-classad-hooks} 
on page~\pageref{sec:daemon-classad-hooks}.

\begin{description}

\label{param:StartdCronName}
\label{param:ScheddCronName}
\item[\Macro{STARTD\_CRON\_NAME} and \Macro{SCHEDD\_CRON\_NAME}]
  These variables will be honored through HTCondor versions 7.6,
  and support will be removed in HTCondor version 7.7.
  They are no longer documented as to their usage.

  Defines a logical name to be used in the formation of related
  configuration macro names.
  This macro made other Daemon ClassAd Hook macros
  more readable and maintainable.  A common example was
\begin{verbatim}
   STARTD_CRON_NAME = HAWKEYE
\end{verbatim}
  This example allowed the naming of other related macros
  to contain the string \verb@HAWKEYE@ in their name, replacing the
  string \verb@STARTD_CRON@.

  The value of these variables may not be \verb@BENCHMARKS@.
  The Daemon ClassAd Hook mechanism is used to implement a set of provided
  hooks that provide benchmark attributes.

\label{param:StartdCronConfigVal}
\label{param:ScheddCronConfigVal}
\label{param:BenchmarkConfigVal}
\item[\Macro{STARTD\_CRON\_CONFIG\_VAL} and \Macro{SCHEDD\_CRON\_CONFIG\_VAL}
  and \Macro{BENCHMARKS\_CONFIG\_VAL}]
  This configuration variable can be used to specify the
  path and executable name of the
  \Condor{config\_val} program which the jobs (hooks) should use to
  get configuration information from the daemon.  If defined,
  an environment variable by the same name with the same value will be
  passed to all jobs.

\label{param:StartdCronAutopublish}
\item[\Macro{STARTD\_CRON\_AUTOPUBLISH}]
  Optional setting that determines if the \Condor{startd} should
  automatically publish a new update to the \Condor{collector} after
  any of the jobs produce output.
  Beware that enabling this setting can greatly increase the network
  traffic in an HTCondor pool, especially when many modules are
  executed, or if the period in which they run is short.
  There are three possible (case insensitive) values for this
  variable: 
  \begin{description}
     \item[\Expr{Never}] This default value causes the
     \Condor{startd} to not automatically publish updates based on
     any jobs. Instead, updates rely on the usual behavior for sending
     updates, which is periodic, based on the \Macro{UPDATE\_INTERVAL}
     configuration variable, or whenever a given slot changes state.
     \item[\Expr{Always}] Causes the \Condor{startd} to always send a new
     update to the \Condor{collector} whenever any job exits.
     \item[\Expr{If\_Changed}] Causes the \Condor{startd} to only send a
     new update to the \Condor{collector} if the output produced by a
     given job is different than the previous output of the
     same job.
     The only exception is the \Attr{LastUpdate} attribute, 
     which is automatically set for all jobs to be the timestamp when
     the job last ran. It is ignored when
     \MacroNI{STARTD\_CRON\_AUTOPUBLISH} is set to \Expr{If\_Changed}.
  \end{description}

\label{param:StartdCronJobList}
\label{param:ScheddCronJobList}
\label{param:BenchmarksJobList}
\item[\Macro{STARTD\_CRON\_JOBLIST} and \Macro{SCHEDD\_CRON\_JOBLIST}
  and \Macro{BENCHMARKS\_JOBLIST}]
  These configuration variables are defined by a comma and/or white space
  separated list of job names to run.  Each is the logical name of a job.
  This name must be unique; no two jobs may have the same name.

\label{param:StartdCronJobPrefix}
\label{param:ScheddCronJobPrefix}
\label{param:BenchmarksJobPrefix}
\item[\Macro{STARTD\_CRON\_<JobName>\_PREFIX} 
       and \Macro{SCHEDD\_CRON\_<JobName>\_PREFIX}
       and \Macro{BENCHMARKS\_<JobName>\_PREFIX}]
  Specifies a string which is prepended by
  HTCondor to all attribute names that the job generates.
  The use of prefixes avoids the conflicts that would be caused by
  attributes of the same name generated and utilized by different jobs.
  For example, if a module prefix is \verb@xyz_@,
  and an individual attribute is named \verb@abc@,
  then the resulting attribute name will be \verb@xyz_abc@.
  Due to restrictions on ClassAd names, a prefix is only permitted to contain
  alpha-numeric characters and the underscore character.

  \Expr{<JobName>} is the logical name assigned for a job as defined by
  configuration variable \MacroNI{STARTD\_CRON\_JOBLIST}, 
  \MacroNI{SCHEDD\_CRON\_JOBLIST}, or \MacroNI{BENCHMARKS\_JOBLIST}.

\label{param:StartdCronJobSlots}
\label{param:BenchmarksJobSlots}
\item[\Macro{STARTD\_CRON\_<JobName>\_SLOTS} 
       and \Macro{BENCHMARKS\_<JobName>\_SLOTS}]
  A comma separated list of slots.
  The output of the job specified by \Expr{<JobName>}
  is incorporated into ClassAds;
  this list specifies which slots are to incorporate the output attributes
  of the job.
  If not specified, the default is to incorporate the output attributes into
  the ClassAd of all slots.

  \Expr{<JobName>} is the logical name assigned for a job as defined by
  configuration variable \MacroNI{STARTD\_CRON\_JOBLIST} 
  or \MacroNI{BENCHMARKS\_JOBLIST}.

\label{param:StartdCronJobExecutable}
\label{param:ScheddCronJobExecutable}
\label{param:BenchmarksJobExecutable}
\item[\Macro{STARTD\_CRON\_<JobName>\_EXECUTABLE} 
       and \Macro{SCHEDD\_CRON\_<JobName>\_EXECUTABLE}
       and \Macro{BENCHMARKS\_<JobName>\_EXECUTABLE}]
  The full path and executable to run for this job.
  Note that multiple jobs may specify the same executable,
  although the jobs need to have different logical names.

  \Expr{<JobName>} is the logical name assigned for a job as defined by
  configuration variable \MacroNI{STARTD\_CRON\_JOBLIST}, 
  \MacroNI{SCHEDD\_CRON\_JOBLIST}, or \MacroNI{BENCHMARKS\_JOBLIST}.

\label{param:StartdCronJobPeriod}
\label{param:ScheddCronJobPeriod}
\label{param:BenchmarksJobPeriod}
\item[\Macro{STARTD\_CRON\_<JobName>\_PERIOD} 
       and \Macro{SCHEDD\_CRON\_<JobName>\_PERIOD}
       and \Macro{BENCHMARKS\_<JobName>\_PERIOD}]
  The period specifies time intervals at which the job should be run.
  For periodic jobs, this
  is the time interval that passes between starting the execution of the job.
  The value may be specified in seconds, minutes,  or hours.
  Specify this time by appending the character \Expr{s}, \Expr{m}, or \Expr{h}
  to the value.
  As an example, 5m starts the execution of the job every five minutes.
  If no character is appended to the value, seconds are used as a default.
  In \Expr{WaitForExit} mode, the value has a different meaning:
  the period specifies the length of time after the job ceases execution and
  before it is restarted.
  The minimum valid value of the period is 1 second.

  \Expr{<JobName>} is the logical name assigned for a job as defined by
  configuration variable \MacroNI{STARTD\_CRON\_JOBLIST}, 
  \MacroNI{SCHEDD\_CRON\_JOBLIST}, or \MacroNI{BENCHMARKS\_JOBLIST}.

\label{param:StartdCronJobMode}
\label{param:ScheddCronJobMode}
\label{param:BenchmarksJobMode}
\item[\Macro{STARTD\_CRON\_<JobName>\_MODE} 
       and \Macro{SCHEDD\_CRON\_<JobName>\_MODE}
       and \Macro{BENCHMARKS\_<JobName>\_MODE}]
  A string that specifies a mode within which the job operates.
  Legal values are 
  \begin{itemize}
  \item \Expr{Periodic}, which is the default.  
  \item \Expr{WaitForExit}
  \item \Expr{OneShot}
  \item \Expr{OnDemand}
  \end{itemize}

  \Expr{<JobName>} is the logical name assigned for a job as defined by
  configuration variable \MacroNI{STARTD\_CRON\_JOBLIST}, 
  \MacroNI{SCHEDD\_CRON\_JOBLIST}, or \MacroNI{BENCHMARKS\_JOBLIST}.

  The default \Expr{Periodic} mode is used for most jobs.
  In this mode, the job is expected to be started by the
  \Condor{startd} daemon, gather and publish its data, and then exit.

  In \Expr{WaitForExit} mode
  the \Condor{startd} daemon interprets the period as defined by 
  \MacroNI{STARTD\_CRON\_<JobName>\_PERIOD} differently.
  In this case, it refers to the amount of time to wait after the job exits
  before restarting it.  With a value of 1, the job is kept
  running nearly continuously.
  In general, \Expr{WaitForExit} mode is for jobs that produce
  a periodic stream of updated data, but it can be used for other
  purposes, as well.

  The \Expr{OneShot} mode is used for jobs that are run once at the
  start of the daemon.  If the \Expr{reconfig\_rerun} option is
  specified, the job will be run again after any reconfiguration.

  The \Expr{OnDemand} mode is used only by the \Expr{BENCHMARKS} mechanism.
  All benchmark jobs must be be \Expr{OnDemand} jobs.  Any other jobs
  specified as \Expr{OnDemand} will never run.  Additional future
  features may allow for other \Expr{OnDemand} job uses.

\label{param:StartdCronJobReconfig}
\label{param:ScheddCronJobReconfig}
\item[\Macro{STARTD\_CRON\_<JobName>\_RECONFIG} 
       and \Macro{SCHEDD\_CRON\_<JobName>\_RECONFIG}]
  A boolean value that when \Expr{True}, causes the 
  daemon to send an HUP signal to the job when the daemon is reconfigured.
  The job is expected to reread its configuration at that time.

  \Expr{<JobName>} is the logical name assigned for a job as defined by
  configuration variable \MacroNI{STARTD\_CRON\_JOBLIST} or
  \MacroNI{SCHEDD\_CRON\_JOBLIST}.

\label{param:StartdCronJobReconfigReRun}
\label{param:ScheddCronJobReconfigReRun}
\item[\Macro{STARTD\_CRON\_<JobName>\_RECONFIG\_RERUN} 
       and \Macro{SCHEDD\_CRON\_<JobName>\_RECONFIG\_RERUN}]
  A boolean value that when \Expr{True}, causes the daemon ClassAd hooks
  mechanism to re-run the specified job when the daemon is
  reconfigured via \Condor{reconfig}.
  The default value is \Expr{False}.

  \Expr{<JobName>} is the logical name assigned for a job as defined by
  configuration variable \MacroNI{STARTD\_CRON\_JOBLIST} or
  \MacroNI{SCHEDD\_CRON\_JOBLIST}.

\label{param:StartdCronJobJobLoad}
\label{param:ScheddCronJobJobLoad}
\label{param:BenchmarksJobJobLoad}
\item[\Macro{STARTD\_CRON\_<JobName>\_JOB\_LOAD} 
       and \Macro{SCHEDD\_CRON\_<JobName>\_JOB\_LOAD}
       and \Macro{BENCHMARKS\_<JobName>\_JOB\_LOAD}]
  A floating point value that represents the assumed and therefore expected
  CPU load that a job induces on the system.
  This job load is then used to limit the total number of jobs that run
  concurrently, by not starting new jobs if the assumed total load from
  all jobs is over a set threshold.
  The default value for each individual 
  \MacroNI{STARTD\_CRON} or a \MacroNI{SCHEDD\_CRON} job is 0.01.
  The default value for each individual 
  \MacroNI{BENCHMARKS} job is 1.0.

  \Expr{<JobName>} is the logical name assigned for a job as defined by
  configuration variable \MacroNI{STARTD\_CRON\_JOBLIST}, 
  \MacroNI{SCHEDD\_CRON\_JOBLIST}, or \MacroNI{BENCHMARKS\_JOBLIST}.

\label{param:StartdCronMaxJobLoad}
\label{param:ScheddCronMaxJobLoad}
\label{param:BenchmarksMaxJobLoad}
\item[\Macro{STARTD\_CRON\_MAX\_JOB\_LOAD} 
       and \Macro{SCHEDD\_CRON\_MAX\_JOB\_LOAD}
       and \Macro{BENCHMARKS\_MAX\_JOB\_LOAD}]
  A floating point value representing a threshold for CPU load,
  such that if starting another job would cause the sum of assumed loads
  for all running jobs to exceed this value,
  no further jobs will be started.
  The default value for \MacroNI{STARTD\_CRON} or a \MacroNI{SCHEDD\_CRON} 
  hook managers is 0.1.
  This implies that a maximum of 10 jobs (using their default, assumed
  load) could be concurrently running.
  The default value for the \MacroNI{BENCHMARKS} hook manager is 1.0.
  This implies that only 1 \MacroNI{BENCHMARKS} job (at the default, assumed
  load) may be running.

\label{param:StartdCronJobKill}
\label{param:ScheddCronJobKill}
\label{param:BenchmarksJobKill}
\item[\Macro{STARTD\_CRON\_<JobName>\_KILL} 
       and \Macro{SCHEDD\_CRON\_<JobName>\_KILL}
       and \Macro{BENCHMARKS\_<JobName>\_KILL}]
  A boolean value applicable only for jobs with a \Expr{MODE} of anything
  other than  \Expr{WaitForExit}.
  The default value is \Expr{False}.

  This variable controls the behavior of the daemon hook manager when it
  detects that an instance of the job's executable is still running
  as it is time to invoke the job again.
  If \Expr{True}, the daemon hook manager will kill the currently running job
  and then invoke an new instance of the job.
  If \Expr{False}, the existing job invocation is allowed to
  continue running. 

  \Expr{<JobName>} is the logical name assigned for a job as defined by
  configuration variable \MacroNI{STARTD\_CRON\_JOBLIST}, 
  \MacroNI{SCHEDD\_CRON\_JOBLIST}, or \MacroNI{BENCHMARKS\_JOBLIST}.

\label{param:StartdCronJobArgs}
\label{param:ScheddCronJobArgs}
\label{param:BenchmarksJobArgs}
\item[\Macro{STARTD\_CRON\_<JobName>\_ARGS} 
       and \Macro{SCHEDD\_CRON\_<JobName>\_ARGS}
       and \Macro{BENCHMARKS\_<JobName>\_ARGS}]
  The command line arguments to pass to the job as it is invoked.
  The first argument will be  \Expr{<JobName>}.

  \Expr{<JobName>} is the logical name assigned for a job as defined by
  configuration variable \MacroNI{STARTD\_CRON\_JOBLIST}, 
  \MacroNI{SCHEDD\_CRON\_JOBLIST}, or \MacroNI{BENCHMARKS\_JOBLIST}.

\label{param:StartdCronJobEnv}
\label{param:ScheddCronJobEnv}
\label{param:BenchmarksJobEnv}
\item[\Macro{STARTD\_CRON\_<JobName>\_ENV} 
       and \Macro{SCHEDD\_CRON\_<JobName>\_ENV}
       and \Macro{BENCHMARKS\_<JobName>\_ENV}]
  The environment string to pass to the job.
  The syntax is the same as that of \MacroNI{<DaemonName>\_ENVIRONMENT}
  as defined at ~\ref{param:DaemonNameEnvironment}.

  \Expr{<JobName>} is the logical name assigned for a job as defined by
  configuration variable \MacroNI{STARTD\_CRON\_JOBLIST}, 
  \MacroNI{SCHEDD\_CRON\_JOBLIST}, or \MacroNI{BENCHMARKS\_JOBLIST}.

\label{param:StartdCronJobCwd}
\label{param:ScheddCronJobCwd}
\label{param:BenchmarksJobCwd}
\item[\Macro{STARTD\_CRON\_<JobName>\_CWD} 
       and \Macro{SCHEDD\_CRON\_<JobName>\_CWD}
       and \Macro{BENCHMARKS\_<JobName>\_CWD}]
  The working directory in which to start the job.

  \Expr{<JobName>} is the logical name assigned for a job as defined by
  configuration variable \MacroNI{STARTD\_CRON\_JOBLIST}, 
  \MacroNI{SCHEDD\_CRON\_JOBLIST}, or \MacroNI{BENCHMARKS\_JOBLIST}.

\end{description}

%%%%%%%%%%%%%%%%%%%%%%%%%%%%%%%%%%%%%%%%%%%%%%%%%%%%%%%%%%%%%%%%%%%%%%%%%%%
\subsection{\label{sec:Config-hooks}Configuration File Entries Only for Windows Platforms}
%%%%%%%%%%%%%%%%%%%%%%%%%%%%%%%%%%%%%%%%%%%%%%%%%%%%%%%%%%%%%%%%%%%%%%%%%%%
\index{configuration!Windows platform configuration variables}
These macros are utilized only on Windows platforms.

\begin{description}

\label{param:WindowsRmdir}
\item[\Macro{WINDOWS\_RMDIR}] The complete path and executable name of the
  HTCondor version of the built-in \Prog{rmdir} program.
  The HTCondor version will not fail when the directory contains files that have
  ACLs that deny the SYSTEM process delete access.
  If not defined, the built-in Windows \Prog{rmdir} program is invoked,
  and a value defined for \Macro{WINDOWS\_RMDIR\_OPTIONS} is ignored.

\label{param:WindowsRmdirOptions}
\item[\Macro{WINDOWS\_RMDIR\_OPTIONS}] Command line options to be specified
  when configuration variable \MacroNI{WINDOWS\_RMDIR} is defined.
  Defaults to \Opt{/S} \Opt{/C} when configuration variable 
  \MacroNI{WINDOWS\_RMDIR} is defined and its definition contains the
  string \AdStr{condor\_rmdir.exe}.

\end{description}

%%%%%%%%%%%%%%%%%%%%%%%%%%%%%%%%%%%%%%%%%%%%%%%%%%%%%%%%%%%%%%%%%%%%%%%%%%%
\subsection{\label{sec:Config-defrag}\Condor{defrag} Configuration File Macros}
%%%%%%%%%%%%%%%%%%%%%%%%%%%%%%%%%%%%%%%%%%%%%%%%%%%%%%%%%%%%%%%%%%%%%%%%%%%
\index{configuration!condor\_defrag configuration variables}
These configuration variables affect the \Condor{defrag} daemon.  A general
discussion of \Condor{defrag} may be found in section~\ref{sec:SMP-defrag}.

\begin{description}

\label{param:DefragName}
\item[\Macro{DEFRAG\_NAME}]
  Used to give an alternative value to the \Attr{Name} attribute
  in the \Condor{defrag}'s ClassAd.
  This esoteric configuration macro might be used in the situation
  where there are two \Condor{defrag} daemons running on one machine,
  and each reports to the same \Condor{collector}.
  Different names will distinguish the two daemons.
  See the description of \MacroNI{MASTER\_NAME} in
  section~\ref{param:MasterName} on page~\pageref{param:MasterName}
  for defaults and composition of valid HTCondor daemon names.

\label{param:DefragDrainingMachinesPerHour}
\item[\Macro{DEFRAG\_DRAINING\_MACHINES\_PER\_HOUR}] A floating point
  number that specifies how many machines should be drained per hour.
  The default is 0, so no draining will happen unless this setting is changed.
  Each \Condor{startd} is considered to be one machine.  
  The actual number of machines drained per hour may be less than this if
  draining is halted by one of the other defragmentation policy controls.
  The granularity in timing of draining initiation is
  controlled by \Macro{DEFRAG\_INTERVAL}.  The lowest rate of draining
  that is supported is one machine per day or one machine per
  \Macro{DEFRAG\_INTERVAL}, whichever is lower.
  A fractional number of machines contributing to the value of 
  \MacroNI{DEFRAG\_DRAINING\_MACHINES\_PER\_HOUR} is rounded to the nearest
  whole number of machines on a per day basis.

\label{param:DefragRequirements}
\item[\Macro{DEFRAG\_REQUIREMENTS}] An expression that specifies which
  machines to drain.  The default is 
\begin{verbatim}
  PartitionableSlot && Offline=!=True  
\end{verbatim}
  A machine, meaning a \Condor{startd},
  is matched if \emph{any} of its slots match this expression. 
  Machines are automatically excluded if they are already draining,
  or if they match \Macro{DEFRAG\_WHOLE\_MACHINE\_EXPR}.

\label{param:DefragCancelRequirements}
\item[\Macro{DEFRAG\_CANCEL\_REQUIREMENTS}] An expression that specifies
  which draining machines should have draining be canceled.  This defaults
  to \MacroU{DEFRAG\_WHOLE\_MACHINE\_EXPR}.  This could be used to drain
  partial rather than whole machines.

\label{param:DefragRank}
\item[\Macro{DEFRAG\_RANK}] An expression that specifies which machines
  are more desirable to drain.  
  The expression should evaluate to a number for each
  candidate machine to be drained.  If the number of machines to be
  drained is less than the number of candidates, 
  the machines with higher rank will be chosen.  
  The rank of a machine, meaning a \Condor{startd},
  is the rank of its highest ranked slot.
  The default rank is \Expr{-ExpectedMachineGracefulDrainingBadput}.

\label{param:DefragWholeMachineExpr}
\item[\Macro{DEFRAG\_WHOLE\_MACHINE\_EXPR}] An expression that
  specifies which machines are already operating as whole machines.  
  The default is 
\begin{verbatim}
  Cpus == TotalCpus && Offline=!=True
\end{verbatim}
  A machine is
  matched if \emph{any} slot on the machine matches this expression.
  Each \Condor{startd} is considered to be one machine.  Whole machines
  are excluded when selecting machines to drain.  They are also counted
  against \Macro{DEFRAG\_MAX\_WHOLE\_MACHINES}.

\label{param:DefragMaxWholeMachines}
\item[\Macro{DEFRAG\_MAX\_WHOLE\_MACHINES}] An integer that specifies
  the maximum number of whole machines.  When the number of whole
  machines is greater than or equal to this, 
  no new machines will be selected for draining.  
  Each \Condor{startd} is counted as one machine.  
  The special value -1 indicates that there is no limit.
  The default is -1.

\label{param:DefragMaxConcurrentDraining}
\item[\Macro{DEFRAG\_MAX\_CONCURRENT\_DRAINING}] An integer that
  specifies the maximum number of draining machines.  
  When the number of machines that are draining is greater than 
  or equal to this, no new machines will be selected for draining.
  Each draining \Condor{startd} is counted as one machine.  
  The special value -1 indicates that there is no limit.  
  The default is -1.

\label{param:DefragInterval}
\item[\Macro{DEFRAG\_INTERVAL}] An integer that specifies the number
  of seconds between evaluations of the defragmentation policy.  
  In each cycle, the state of the pool is observed and machines are drained, 
  if specified by the policy.  
  The default is 600 seconds.  
  Very small intervals could create excessive load on the \Condor{collector}.

\label{param:DefragSchedule}
\item[\Macro{DEFRAG\_SCHEDULE}] A setting that specifies the draining
  schedule to use when draining machines.  
  Possible values are \Expr{graceful}, \Expr{quick}, and \Expr{fast}.
  The default is \Expr{graceful}.

\begin{description}
  \item[graceful] Initiate a graceful eviction of the job.  This means
  all promises that have been made to the job are honored, including
  \MacroNI{MaxJobRetirementTime}.  The eviction of jobs is coordinated
  to reduce idle time.  This means that if one slot has a job with a long
  retirement time and the other slots have jobs with shorter retirement times,
  the effective retirement time for all of the jobs is the longer one.

  \item[quick] \MacroNI{MaxJobRetirementTime} is not honored.  Eviction
  of jobs is immediately initiated.  Jobs are given time to shut down
  and produce a checkpoint according to the usual policy,
  as given by \MacroNI{MachineMaxVacateTime}.

  \item[fast] Jobs are immediately hard-killed, with no chance to
  gracefully shut down or produce a checkpoint.
\end{description}

\label{param:DefragStateFile}
\item[\Macro{DEFRAG\_STATE\_FILE}] The path to a file used to record
  information used by \Condor{defrag} when it is restarted.  
  This should only need to be modified if there will be multiple instances of
  the \Condor{defrag} daemon running on the same machine.  
  The default is \File{\$(LOCK)/defrag\_state}.

\label{param:DefragLog}
\item[\Macro{DEFRAG\_LOG}] The path to the \Condor{defrag} daemon's log file.
  The default log location is \File{\$(LOG)/DefragLog}.

\end{description}
