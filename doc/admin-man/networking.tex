%%%%%%%%%%%%%%%%%%%%%%%%%%%%%%%%%%%%%%%%%%%%%%%%%%%%%%%%%%%%%%%%%%%%%%
\section{\label{sec:Networking}Networking (includes sections on Port Usage, CCB, and GCB)}
%%%%%%%%%%%%%%%%%%%%%%%%%%%%%%%%%%%%%%%%%%%%%%%%%%%%%%%%%%%%%%%%%%%%%%
\index{network}

This section on
network communication in Condor
discusses which network ports are used,
how Condor behaves on machines with multiple network interfaces
and IP addresses,
and how to facilitate functionality in a pool that spans
firewalls and private networks.

The security section of the manual contains some
information that is relevant to the discussion of network
communication which will not be duplicated here, so please
see section~\ref{sec:Security} as well.

Firewalls, private networks, and network address translation (NAT)
pose special problems for Condor.
There are currently two main mechanisms for dealing with firewalls
within Condor:

\begin{enumerate}

\item Restrict Condor to use a specific range of port numbers, and
  allow connections through the firewall that use any port within the
  range.

\item Use \Term{Condor Connection Brokering} (CCB) or \Term{Generic Connection Brokering} (GCB).

\end{enumerate}

Each method has its own advantages and disadvantages,
as described below.


%%%%%%%%%%%%%%%%%%%%%%%%%%%%%%%%%%%%%%%%%%%%%%%%%%%%%%%%%%%%%%%%%%%%%%
% all of these define their own \subsection, so include them directly
%%%%%%%%%%%%%%%%%%%%%%%%%%%%%%%%%%%%%%%%%%%%%%%%%%%%%%%%%%%%%%%%%%%%%%

%%%%%%%%%%%%%%%%%%%%%%%%%%%%%%%%%%%%%%%%%%%%%%%%%%%%%%%%%%%%%%%%%%%%%%%%%%%
\subsection{\label{sec:Port-Details}Port Usage in Condor}
%%%%%%%%%%%%%%%%%%%%%%%%%%%%%%%%%%%%%%%%%%%%%%%%%%%%%%%%%%%%%%%%%%%%%%%%%%

\index{port usage}


%%%%%%%%%%%%%%%%%%%%%%%%%%%%%%%%%%%%%%%%%%%%%%%%%%%%%%%%%%%%%%%%%%%%%%%%%%%
\subsubsection{\label{sec:Ports-Standard}Default Port Usage}
%%%%%%%%%%%%%%%%%%%%%%%%%%%%%%%%%%%%%%%%%%%%%%%%%%%%%%%%%%%%%%%%%%%%%%%%%%%

Every Condor daemon listens on a network port for incoming commands.
(Using \Condor{shared\_port}, this port may be shared between multiple
daemons.)
Most daemons listen on a dynamically assigned port.
In order to send a message,
Condor daemons and tools locate the correct port to use
by querying the \Condor{collector},
extracting the port number from the ClassAd.
One of the attributes included in every daemon's ClassAd is the full
IP address and port number upon which the daemon is listening.

To access the \Condor{collector} itself,
all Condor daemons and tools
must know the port number  where the \Condor{collector} is listening.
The \Condor{collector} is the only daemon with a well-known,
fixed port.
By default, Condor uses port 9618 for the \Condor{collector} daemon.
However, this port number can be changed (see below).

As an optimization for daemons and tools communicating with another
daemon that is running on the same host,
each Condor daemon can be configured to
write its IP address and port number into a well-known file.
The file names are controlled using the \MacroB{<SUBSYS>\_ADDRESS\_FILE}
configuration variables,
as described in section~\ref{param:SubsysAddressFile} on
page~\pageref{param:SubsysAddressFile}. 

\Note In the 6.6 stable series, and Condor versions earlier than
6.7.5, the \Condor{negotiator} also listened on a fixed, well-known
port (the default was 9614).
However, beginning with version 6.7.5, the \Condor{negotiator} behaves
like all other Condor daemons, and publishes its own ClassAd to the
\Condor{collector} which includes the dynamically assigned port 
the \Condor{negotiator} is listening on.
All Condor tools and daemons that need to communicate with the
\Condor{negotiator} will either use the
\Macro{NEGOTIATOR\_ADDRESS\_FILE} or will query the
\Condor{collector} for the \Condor{negotiator}'s ClassAd.

Sites that configure any checkpoint servers will introduce
other fixed ports into their network.
Each \Condor{ckpt\_server} will listen to 4 fixed ports: 5651, 5652,
5653, and 5654.
There is currently no way to configure alternative values for any of
these ports.


%%%%%%%%%%%%%%%%%%%%%%%%%%%%%%%%%%%%%%%%%%%%%%%%%%%%%%%%%%%%%%%%%%%%%%%%%%%
\subsubsection{\label{sec:Ports-NonStandard}Using 
a Non Standard, Fixed Port for the \Condor{collector}}
%%%%%%%%%%%%%%%%%%%%%%%%%%%%%%%%%%%%%%%%%%%%%%%%%%%%%%%%%%%%%%%%%%%%%%%%%%%
\index{port usage!nonstandard ports for central managers}
By default,
Condor uses port 9618 for the \Condor{collector} daemon.
To use a different port number for this daemon,
the configuration variables that tell Condor these communication
details are modified.
Instead of
\begin{verbatim}
CONDOR_HOST = machX.cs.wisc.edu
COLLECTOR_HOST = $(CONDOR_HOST)
\end{verbatim}
the configuration might be
\begin{verbatim}
CONDOR_HOST = machX.cs.wisc.edu
COLLECTOR_HOST = $(CONDOR_HOST):9650
\end{verbatim}

If a non standard port is defined, the same value of
\MacroNI{COLLECTOR\_HOST} (including the port) must be used for all
machines in the Condor pool.
Therefore, this setting should be modified in the global
configuration file (\File{condor\_config} file),
or the value must be duplicated across
all configuration files in the pool if a single configuration file
is not being shared.

When querying the \Condor{collector} for a remote pool that is running
on a non standard port, any Condor tool that accepts the \Opt{-pool}
argument can optionally be given a port number.  For example:
\footnotesize
\begin{verbatim}
        % condor_status -pool foo.bar.org:1234
\end{verbatim}
\normalsize


%%%%%%%%%%%%%%%%%%%%%%%%%%%%%%%%%%%%%%%%%%%%%%%%%%%%%%%%%%%%%%%%%%%%%%%%%%%
\subsubsection{\label{sec:Ports-Dynamic-Collector}Using 
a Dynamically Assigned Port for the \Condor{collector}}
%%%%%%%%%%%%%%%%%%%%%%%%%%%%%%%%%%%%%%%%%%%%%%%%%%%%%%%%%%%%%%%%%%%%%%%%%%%

On single machine pools, 
it is permitted to configure the
\Condor{collector} daemon
to use a dynamically assigned port,
as given out by the operating system.
This prevents port conflicts with other services on the same machine.
However, a dynamically assigned port is only to be used on
single machine Condor pools,
and only if the
\Macro{COLLECTOR\_ADDRESS\_FILE} 
configuration variable has also been defined.
This mechanism allows all of the Condor daemons and tools running on
the same machine to find the port upon which the \Condor{collector}
daemon is listening,
even when this port is not defined in the
configuration file and is not known in advance.

To enable the \Condor{collector} daemon to use a dynamically assigned port,
the port number is set to 0 in the \Macro{COLLECTOR\_HOST}
variable.
The \MacroNI{COLLECTOR\_ADDRESS\_FILE}
configuration variable must also be defined,
as it provides a known file where the IP address
and port information will be stored.
All Condor clients know to look at the
information stored in this file.
For example:
\footnotesize
\begin{verbatim}
COLLECTOR_HOST = $(CONDOR_HOST):0
COLLECTOR_ADDRESS_FILE = $(LOG)/.collector_address
\end{verbatim}
\normalsize

\Note Using a port of 0 for the \Condor{collector}
and specifying a
\MacroNI{COLLECTOR\_ADDRESS\_FILE}
only works in Condor version 6.6.8 or later in the 6.6 stable series,
and in version 6.7.4 or later in the 6.7 development series.
Do not attempt to do this with older versions of Condor.

Configuration definition of \MacroNI{COLLECTOR\_ADDRESS\_FILE}
is in section~\ref{param:SubsysAddressFile} on
page~\pageref{param:SubsysAddressFile},
and
\MacroNI{COLLECTOR\_HOST}
is in
section~\ref{param:CollectorHost} on
page~\pageref{param:CollectorHost}.


%%%%%%%%%%%%%%%%%%%%%%%%%%%%%%%%%%%%%%%%%%%%%%%%%%%%%%%%%%%%%%%%%%%%%%%%%%%
\subsubsection{\label{sec:Ports-Firewalls}Restricting Port Usage to
 Operate with Firewalls}
%%%%%%%%%%%%%%%%%%%%%%%%%%%%%%%%%%%%%%%%%%%%%%%%%%%%%%%%%%%%%%%%%%%%%%%%%%%

\index{port usage!firewalls}
If a Condor pool is completely behind a firewall,
then no special consideration or port usage is needed.
However, if there is a firewall between the machines within
a Condor pool, then
configuration variables may be set to force the usage of
specific ports, and to utilize a specific range of ports.

By default,
Condor uses port 9618 for the \Condor{collector} daemon,
and dynamic (apparently random) ports for everything else.
See section~\ref{sec:Ports-Dynamic-Collector},
if a dynamically assigned port is desired for the
\Condor{collector} daemon.

On platforms that support it, all of the Condor daemons on a machine
may be configured to share a single
port.  See section~\ref{sec:Config-shared-port} for more information.

The configuration variables
\Macro{HIGHPORT} and \Macro{LOWPORT} facilitate setting a restricted
range of ports that Condor will use.
This may be useful when some machines are behind a firewall.
The configuration macros
\MacroNI{HIGHPORT} and \MacroNI{LOWPORT} 
will restrict dynamic ports to the range specified.
The configuration variables are fully defined
in section~\ref{sec:Network-Related-Config-File-Entries}.
All of these ports must be greater than 0 and less than 65,536.
Note that both \MacroNI{HIGHPORT} and \MacroNI{LOWPORT} must be at 
least 1024 for Condor version 6.6.8.
In general, use ports greater than 1024,
in order
to avoid port conflicts with standard services on the machine.
Another reason for using ports greater than 1024 is that
daemons and tools are often not run as \Login{root},
and only \Login{root} may listen to a port lower than 1024.
Also, the range must include enough ports that are not in use, 
or Condor cannot work.

The range of ports assigned may be restricted based on 
incoming (listening) and outgoing (connect) ports
with the configuration variables
\Macro{IN\_HIGHPORT},
\Macro{IN\_LOWPORT},
\Macro{OUT\_HIGHPORT}, and
\Macro{OUT\_LOWPORT}.
See section~\ref{sec:Network-Related-Config-File-Entries}
for complete definitions of these configuration variables.
A range of ports lower than 1024 for daemons
running as \Login{root} is appropriate for incoming ports,
but not for outgoing ports.
The use of ports below 1024 (versus above 1024)
has security implications; 
therefore, it is inappropriate to assign a range that crosses
the 1024 boundary.


\Note Setting \MacroNI{HIGHPORT} and \MacroNI{LOWPORT} will not
automatically force the \Condor{collector} to bind to a port within
the range.
The only way to control what port the \Condor{collector} uses is by
setting the \MacroNI{COLLECTOR\_HOST} (as described above).

The total number of ports needed depends on the size of the pool,
the usage of the machines within the pool (which machines
run which daemons),
and the number of jobs that may execute at one time.
Here we discuss how many ports are used by each
participant in the system.  This assumes that \Condor{shared\_port}
is not being used.  If it \emph{is} being used, then all daemons
can share a single incoming port.

The central manager of the pool needs
\Expr{5 + \MacroNI{NEGOTIATOR\_SOCKET\_CACHE\_SIZE}}
ports for daemon communication,
where 
\Macro{NEGOTIATOR\_SOCKET\_CACHE\_SIZE}
is specified in the
configuration or defaults to the value 16.

Each execute machine (those machines running a \Condor{startd} daemon)
requires
\Expr{ 5 + (5 * number of slots advertised by that machine)}
ports.
By default, the number of slots advertised
will equal the number of physical CPUs in that machine.

Submit machines (those machines running a \Condor{schedd} daemon)
require
\Expr{ 5 + (5 *  \MacroNI{MAX\_JOBS\_RUNNING})} ports.
The configuration variable \Macro{MAX\_JOBS\_RUNNING}
limits (on a per-machine basis, if desired)
the maximum number of jobs.
Without this configuration macro,
the maximum number of jobs that could be simultaneously
executing at one time
is a function of the number of reachable execute machines. 

Also be aware that \MacroNI{HIGHPORT} and \MacroNI{LOWPORT}
only impact dynamic port selection used by the Condor system,
and they do not impact port selection used by jobs submitted to Condor.
Thus, jobs submitted to Condor that may create
network connections may not work in a port restricted environment.
For this reason, specifying \MacroNI{HIGHPORT} and \MacroNI{LOWPORT}
is not going to produce the
expected results if a user submits MPI applications to be executed under
the parallel universe.

Where desired, a local
configuration for machines \emph{not} behind a firewall
can override the usage of \MacroNI{HIGHPORT} and \MacroNI{LOWPORT},
such that the ports used for these machines are not restricted.
This can be accomplished by adding the following to the
local configuration file of those machines \emph{not}
behind a firewall:
\begin{verbatim}
HIGHPORT = UNDEFINED
LOWPORT  = UNDEFINED
\end{verbatim}


If the maximum number of ports allocated using 
\MacroNI{HIGHPORT} and \MacroNI{LOWPORT}
is too few,
socket binding errors of the form
\footnotesize
\begin{verbatim}
failed to bind any port within <$LOWPORT> - <$HIGHPORT>
\end{verbatim}
\normalsize
are likely to appear repeatedly in log files.


%%%%%%%%%%%%%%%%%%%%%%%%%%%%%%%%%%%%%%%%%%%%%%%%%%%%%%%%%%%%%%%%%%%%%%%%%%%
\subsubsection{\label{sec:Ports-MultipleCollectors}Multiple Collectors}
%%%%%%%%%%%%%%%%%%%%%%%%%%%%%%%%%%%%%%%%%%%%%%%%%%%%%%%%%%%%%%%%%%%%%%%%%%%
\index{port usage!multiple collectors}
\Todo


%%%%%%%%%%%%%%%%%%%%%%%%%%%%%%%%%%%%%%%%%%%%%%%%%%%%%%%%%%%%%%%%%%%%%%%%%%%
\subsubsection{\label{sec:Ports-Conflicts}Port Conflicts}
%%%%%%%%%%%%%%%%%%%%%%%%%%%%%%%%%%%%%%%%%%%%%%%%%%%%%%%%%%%%%%%%%%%%%%%%%%%
\index{port usage!conflicts}
\Todo



%%%%%%%%%%%%%%%%%%%%%%%%%%%%%%%%%%%%%%%%%%%%%%%%%%%%%%%%%%%%%%%%%%%%%%%%%%%
\subsection{\label{sec:shared-port-daemon}Reducing Port Usage with the \Condor{shared\_port} Daemon}
%%%%%%%%%%%%%%%%%%%%%%%%%%%%%%%%%%%%%%%%%%%%%%%%%%%%%%%%%%%%%%%%%%%%%%%%%%%

\index{HTCondor daemon!condor\_shared\_port@\Condor{shared\_port}}
\index{daemon!condor\_shared\_port@\Condor{shared\_port}}
\index{condor\_shared\_port daemon}
The \Condor{shared\_port} is an optional daemon
responsible for creating a TCP listener port shared by all of the
HTCondor daemons for which the configuration variable
\Macro{USE\_SHARED\_PORT} is \Expr{True}.
The \Condor{master} will invoke the \Condor{shared\_port} daemon
if it is listed in \MacroNI{DAEMON\_LIST}.  For further configuration
options, such as specifying the port number to use, see page~\pageref{sec:Config-shared-port}.

The main purpose of the \Condor{shared\_port} daemon is to reduce the
number of ports that must be opened when HTCondor needs to be
accessible through a firewall.
This has a greater security benefit
than simply reducing the number of open ports.
Without the \Condor{shared\_port} daemon,
one can configure HTCondor to use a range of ports,
but since some HTCondor daemons are created dynamically, 
this full range of ports will not be in use by HTCondor at all times.
This implies that other non-HTCondor processes not intended to be exposed to
the outside network could unintentionally bind to ports in the range
intended for HTCondor,
unless additional steps are taken to control access to those ports.  
While the \Condor{shared\_port} daemon is running,
it is exclusively bound to its port, which means that other non-HTCondor
processes cannot accidentally bind to that port.

A secondary benefit of the \Condor{shared\_port} daemon
is that it helps address the scalability issues of a submit machine.
Without the \Condor{shared\_port} daemon,
approximately 2.1 ephemeral ports per running job are required,
and possibly more, depending on the rate of job completion.
There are only 64K ports in total,
and most standard Unix installations only allocate a subset of
these as ephemeral ports.
In practice, with long running jobs,
and with between 11K and 14K simultaneously running jobs,
port exhaustion has been observed in typical Linux installations.
After increasing the ephemeral port range as to as many as possible,
port exhaustion occurred between 20K and 25K running jobs.
Using the \Condor{shared\_port} daemon,
each running job requires fewer, approximately 1.1 ephemeral ports
on the submit node, if HTCondor on the submit node connects directly
to HTCondor on the execute node.
If the submit node connects via CCB to the execute
node, \emph{no} ports are required per running job; only the one port
allocated to the \Condor{shared\_port} daemon is used.

When CCB is utilized via setting the configuration variable
\Macro{CCB\_ADDRESS},
the \Condor{shared\_port} daemon registers with
the CCB server on behalf of all daemons sharing the port.
This means that it is not possible to individually enable or disable
CCB connectivity to daemons that are using the shared port;
they all effectively share the same setting,
and the \Condor{shared\_port} daemon handles all CCB connection
requests on their behalf.

HTCondor's authentication and authorization steps are unchanged by the
use of a shared port.  Each HTCondor daemon continues to operate
according to its configured policy.  Requests for connections to the
shared port are not authenticated or restricted by
the \Condor{shared\_port} daemon.
They are simply passed to the requested daemon,
which is then responsible for enforcing the security policy.

When the \Condor{master} is configured to use the shared port
by setting the configuration variable
\begin{verbatim}
  USE_SHARED_PORT = True
\end{verbatim}
the \Condor{shared\_port} daemon is treated specially. 
A command such as \Condor{off},
which shuts down all daemons except for the \Condor{master},
will also leave the \Condor{shared\_port} running.
This prevents the \Condor{master} from getting into a state
where it can no longer receive commands.

The \Condor{collector} daemon typically has its own port;
it uses 9618 by default.
However, it can be configured to use a shared port.
Since the address of the \Condor{collector} must be set in 
the configuration file,
it is necessary to specify the shared port socket name of 
the \Condor{collector},
so that connections to the shared port that are intended for 
the \Condor{collector} can be forwarded to it.
If the shared port number is 11000, a \Condor{collector} address using this
shared port could be configured:

\footnotesize
\begin{verbatim}
COLLECTOR_HOST = collector.host.name:11000?sock=collector
\end{verbatim}
\normalsize

This configuration assumes that the socket name used by 
the \Condor{collector} is \Expr{collector}.
The \Condor{collector} that runs on \Expr{collector.host.name}
will automatically choose this socket name if \MacroNI{COLLECTOR\_HOST}
is configured as in the example above.
If multiple \Condor{collector} daemons are started on the same
machine, the socket name can be explicitly set in the daemon arguments,
as in the example:

\begin{verbatim}
COLLECTOR_ARGS = -sock collector
\end{verbatim}

When the \Condor{collector} address is a shared port,
TCP updates will be automatically used instead of UDP.
Under Unix, this means that the
\Condor{collector} daemon should be configured to have enough file descriptors.
See section~\ref{sec:tcp-collector-update} for more information on using
TCP within HTCondor.

SOAP commands cannot be sent over a shared port.
However, a daemon may be configured to open a fixed, non-shared port,
in addition to using a shared port.
This is done both by setting
\Expr{USE\_SHARED\_PORT = True} and by specifying a fixed port for the daemon
using \verb|<SUBSYS>_ARGS = -p <portnum>|.

The TCP connections required to manage standard universe jobs do not
make use of shared ports.



%%%%%%%%%%%%%%%%%%%%%%%%%%%%%%%%%%%%%%%%%%%%%%%%%%%%%%%%%%%%%%%%%%%%%%%%%%%
\subsection{\label{sec:Multiple-Interfaces}Configuring Condor for
Machines With Multiple Network Interfaces } 
%%%%%%%%%%%%%%%%%%%%%%%%%%%%%%%%%%%%%%%%%%%%%%%%%%%%%%%%%%%%%%%%%%%%%%%%%%

\index{multiple network interfaces}
\index{network interfaces!multiple}
\index{NICs}

Condor can run on machines with
multiple network interfaces.
Starting with Condor version 6.7.13
(and therefore all Condor 6.8 and more recent versions),
new functionality is
available that allows even better support for multi-homed
machines, using the configuration variable \MacroNI{BIND\_ALL\_INTERFACES}.
A multi-homed machine is one that has more than one
NIC (Network Interface Card).
Further improvements to this new functionality will remove the need
for any special configuration in the common case.
For now, care
must still be given to machines with multiple NICs, even
when using this new configuration variable.


%%%%%%%%%%%%%%%%%%%%%%%%%%%%%%%%%%%%%%%%%%%%%%%%%%%%%%%%%%%%
\subsubsection{\label{sec:Using-BindAllInterfaces}Using 
BIND\_ALL\_INTERFACES}
%%%%%%%%%%%%%%%%%%%%%%%%%%%%%%%%%%%%%%%%%%%%%%%%%%%%%%%%%%%%

Machines can be configured such that
whenever Condor daemons or tools
call \Syscall{bind}, the daemons or tools use all network interfaces on
the machine.
This means that outbound connections will always use the appropriate
network interface to connect to a remote host,
instead of being forced to use
an interface that might not have a route to the given destination.
Furthermore, sockets upon which a daemon listens for incoming connections 
will be bound to all network interfaces on the machine.
This means that so long as remote clients know the right port, they can
use any IP address on the machine and still contact a given Condor daemon.

This functionality is on by default.  To disenable this functionality, 
the boolean configuration
variable
\MacroNI{BIND\_ALL\_INTERFACES}
is defined and set to \Expr{False}:

\begin{verbatim}
BIND_ALL_INTERFACES = FALSE
\end{verbatim}

This functionality has limitations.
Here are descriptions of the limitations.

\begin{description}

\item[Using all network interfaces does not work with Kerberos.] 
  Every Kerberos ticket contains a specific IP address within it.
  Authentication over a socket (using Kerberos) requires
  the socket to also specify that same specific IP address.
  Use of \MacroNI{BIND\_ALL\_INTERFACES} causes outbound
  connections from a multi-homed machine to 
  originate over any of the interfaces.
  Therefore, the IP address of the outbound connection and the IP
  address in the Kerberos ticket will not necessarily match,
  causing the authentication to fail.
  Sites using Kerberos authentication on multi-homed machines are
  strongly encouraged not to enable \MacroNI{BIND\_ALL\_INTERFACES},
  at least until Condor's Kerberos functionality
  supports using multiple Kerberos tickets together with finding the right one
  to match the IP address a given socket is bound to. 

\item[There is a potential security risk.]
  Consider the following example of a security risk.
  A multi-homed machine is at a network boundary.
  One interface is on the public Internet, while the other connects to
  a private network.
  Both the multi-homed machine and the private network machines
  comprise a Condor pool.
  If the multi-homed machine enables \MacroNI{BIND\_ALL\_INTERFACES},
  then it is at risk from hackers trying to compromise the security of the pool.
  Should this multi-homed machine be compromised,
  the entire pool is vulnerable.
  Most sites in this situation would run an \Prog{sshd} on the
  multi-homed machine so that remote users who wanted to access the
  pool could log in securely and use the Condor tools directly.
  In this case, remote clients do not need to use Condor tools running
  on machines in the public network to access the Condor daemons on
  the multi-homed machine.
  Therefore, there is no reason to have Condor daemons listening on
  ports on the public Internet, causing a potential security threat.

\item[Only one IP address will be advertised.]
  At present, even though a given Condor daemon will be listening to
  ports on multiple interfaces, each with their own IP address,
  there is currently no mechanism for that daemon to advertise all of
  the possible IP addresses where it can be contacted.
  Therefore, Condor clients (other Condor daemons or tools) will not
  necessarily able to locate and communicate with a given daemon
  running on a multi-homed machine where
  \MacroNI{BIND\_ALL\_INTERFACES} has been enabled.

  Currently, Condor daemons can only advertise a single IP address in
  the ClassAd they send to their \Condor{collector}.
  Condor tools and other daemons only know how to look up a single IP
  address, and they attempt to use that single IP address
  when connecting to the daemon.
  So, even if the daemon is listening on 2 or more different interfaces,
  each with a separate IP, the daemon must choose what IP address to
  publicly advertise so that other daemons and tools can locate it.

  By default, Condor advertises the IP address of the network interface
  used to contact the collector, since this is the most likely to be
  accessible to other processes that query the same collector.
  The \Macro{NETWORK\_INTERFACE} setting can still be used to
  specify the IP address Condor should advertise, even if
  \MacroNI{BIND\_ALL\_INTERFACES} is set to \Expr{True}.
  Therefore, some of the considerations described below regarding what
  interface should be used in various situations still apply when
  deciding what interface is to be advertised.

\end{description}

Sites that make heavy use of private networks and multi-homed machines
should consider if using Generic Connection Brokering, GCB, is
right for them.
More information about GCB and Condor can be found in
section~\ref{sec:GCB} on page~\pageref{sec:GCB}.


%%%%%%%%%%%%%%%%%%%%%%%%%%%%%%%%%%%%%%%%%%%%%%%%%%%%%%%%%%%%
\subsubsection{Central Manager with Two or More NICs}
%%%%%%%%%%%%%%%%%%%%%%%%%%%%%%%%%%%%%%%%%%%%%%%%%%%%%%%%%%%%

Often users of Condor wish to set up ``compute farms'' where there is one
machine with two network interface cards (one for the public Internet,
and one for the private net). It is convenient to set up the ``head''
node as a central manager in most cases and so here are the instructions
required to do so.

Setting up the central manager on a machine with more than one NIC can
be a little confusing because there are a few external variables
that could make the process difficult. One of the biggest mistakes
in getting this to work is that either one of the separate interfaces is
not active, or the host/domain names associated with the interfaces are
incorrectly configured. 

Given that the interfaces are up and functioning, and they have good
host/domain names associated with them here is how to configure Condor:

In this example, \Bold{farm-server.farm.org} maps to the private interface.

On the central manager's global (to the cluster) configuration file: \\
\Macro{CONDOR\_HOST} = \Bold{farm-server.farm.org}

On your central manager's local configuration file: \\
\MacroNI{NETWORK\_INTERFACE} = ip address of \Bold{farm-server.farm.org} \\
\MacroNI{NEGOTIATOR} = \MacroUNI{SBIN}/condor\_negotiator \\
\MacroNI{COLLECTOR} = \MacroUNI{SBIN}/condor\_collector \\
\MacroNI{DAEMON\_LIST} = \MacroNI{MASTER}, \MacroNI{COLLECTOR}, \MacroNI{NEGOTIATOR}, \MacroNI{SCHEDD}, \MacroNI{STARTD}

If your central manager and farm machines are all NT, then you only have
vanilla universe and it will work now.  However, if you have this setup
for UNIX, then at this point, standard universe jobs should be able to
function in the pool, but if you did not configure the \Macro{UID\_DOMAIN}
macro to be homogeneous across the farm machines, the standard universe
jobs will run as \Bold{nobody} on the farm machines.

In order to get vanilla jobs and file server load balancing for standard
universe jobs working (under Unix), do some more work both in
the cluster you have put together and in Condor to make everything work.
First, you need a file server (which could also be the central manager) to
serve files to all of the farm machines. This could be NFS or AFS, it does
not really matter to Condor. The mount point of the directories you wish
your users to use must be the same across all of the farm machines. Now,
configure \Macro{UID\_DOMAIN} and \Macro{FILESYSTEM\_DOMAIN} to be
homogeneous across the farm machines and the central manager. Now, you
will have to inform Condor that an NFS or AFS filesystem exists and that
is done in this manner. In the global (to the farm) configuration file:

\begin{verbatim}
# If you have NFS
USE_NFS = True
# If you have AFS
HAS_AFS = True
USE_AFS = True
# if you want both NFS and AFS, then enable both sets above
\end{verbatim}

Now, if you've set up your cluster so that it is possible for a machine
name to never have a domain name (for example: there is machine
name but no fully qualified domain name in \File{/etc/hosts}), you must
configure \Macro{DEFAULT\_DOMAIN\_NAME} to be the domain that you wish
to be added on to the end of your host name.


%%%%%%%%%%%%%%%%%%%%%%%%%%%%%%%%%%%%%%%%%%%%%%%%%%%%%%%%%%%%
\subsubsection{A Client Machine with Multiple Interfaces}
%%%%%%%%%%%%%%%%%%%%%%%%%%%%%%%%%%%%%%%%%%%%%%%%%%%%%%%%%%%%

If you have a client machine with two or more NICs, then there might be
a specific network interface with which you desire a client machine to
communicate with the rest of the Condor pool. In this case, in the local
configuration file for that machine, place: \\ 
\Macro{NETWORK\_INTERFACE} = ip address of interface desired \\


%%%%%%%%%%%%%%%%%%%%%%%%%%%%%%%%%%%%%%%%%%%%%%%%%%%%%%%%%%%%
\subsubsection{A Checkpoint Server on a Machine with Multiple NICs}
%%%%%%%%%%%%%%%%%%%%%%%%%%%%%%%%%%%%%%%%%%%%%%%%%%%%%%%%%%%%

If your Checkpoint Server is on a machine with multiple interfaces,
the only way to get things to work is if your different interfaces
have different host names associated with them, and you set
\Macro{CKPT\_SERVER\_HOST} to the host name that corresponds with the
IP address you want to use in the global configuration file for your pool.
You will still need to specify \Macro{NETWORK\_INTERFACE} in the local
config file for your Checkpoint Server.



%%%%%%%%%%%%%%%%%%%%%%%%%%%%%%%%%%%%%%%%%%%%%%%%%%%%%%%%%%%%%%%%%%%%%%
\subsection{\label{sec:CCB}Condor Connection Brokering (CCB)}
%%%%%%%%%%%%%%%%%%%%%%%%%%%%%%%%%%%%%%%%%%%%%%%%%%%%%%%%%%%%%%%%%%%%%%
\index{CCB (Condor Connection Brokering)}

Condor Connection Brokering, or CCB, is a way of allowing Condor
components to communicate with each other when one side is in a
private network or behind a firewall.  Specifically, CCB allows
communication across a private network boundary in the following
scenario: a Condor tool or daemon (process A) needs to connect to a
Condor daemon (process B), but the network does not allow a TCP
connection to be created from A to B; it only allows connections from
B to A.  In this case, B may be configured (with \Macro{CCB\_ADDRESS})
to register itself with a CCB server that both A and B can connect to.
Then when A needs to connect to B, it can send a request to the CCB
server, which will instruct B to connect to A so that the two can
communicate.

Example: a Condor execute node that is in a private network cannot
normally run jobs from a schedd that is outside of that private
network.  Bi-directional connectivity between the submit node and the
execute node is normally required.  However, if the startd is
configured with \MacroNI{CCB\_ADDRESS}, and both parties can connect
to that address and are authorized by the CCB server, and it is
possible for the startd to connect to the schedd (one-directional
connectivity), then it is possible for the schedd to run jobs on the
startd.  If the situation is reversed, the same idea applies: the
schedd in a private network can register itself with a CCB server, so
when a job is running and the execute node needs to connect back to
the submit node (e.g. to transfer output files), it can do so by going
through CCB to request a connection.

Any Condor collector may be used as a CCB server.  There is no
requirement that the collector(s) acting as the CCB server
(\MacroNI{CCB\_ADDRESS}) be the same collector(s) that a daemon
advertises itself to (\MacroNI{COLLECTOR\_HOST}).  However, this is
often a convenient choice.

\subsubsection{Example Configuration}

This example assumes that there is a pool of machines in a private
network that need to be made accessible from the outside and that the
collector used by these machines is accessible from the outside
(e.g. by a special firewall rule for the collector port or by being on
a dual-homed machine in both networks).  In the configuration of
machines in the private network, we add the following and issue
\Condor{reconfig}.

\begin{verbatim}
CCB_ADDRESS = $(COLLECTOR_HOST)
PRIVATE_NETWORK_NAME = cs.wisc.edu
\end{verbatim}

By defining the \MacroNI{PRIVATE\_NETWORK\_NAME}, we ensure that all
communication between nodes in the private network continues to happen
as normal, without going through CCB.  This name should be different
from the private network name chosen for any Condor installations that
will be communicating with this pool.

Under UNIX, with large Condor pools, it is also necessary to give the
collector serving as the CCB server a high enough file descriptor
limit (\Macro{MAX\_FILE\_DESCRIPTORS} or equivalent).  Each Condor
process configured to use CCB with \Macro{CCB\_ADDRESS} requires one
persistent TCP connection to the CCB server.  A typical execute node
requires one for \Condor{master}, one for \Condor{startd}, and one for
each running job (\Condor{starter}).  A typical submit machine
requires one for \Condor{master}, one for \Condor{schedd}, and one for
each running job (\Condor{shadow}).  If you don't require any
administrative commands to be sent to \Condor{master} from outside of
the private network, then you could disable CCB in the master
(e.g. by assigning \MacroNI{MASTER.CCB\_ADDRESS} to nothing).

Continuing the example: suppose the pool consists of 500 8-slot
execute nodes and we do not disable CCB in the configuration of the
\Condor{master} processes.  In this case, we need 500*(1+1+8)=5000
file descriptors plus some extra for other transient connections to
the collector.  We'll be generous and give it twice as many
descriptors as needed by CCB alone.

\begin{verbatim}
MAX_FILE_DESCRIPTORS = 10000
\end{verbatim}

\subsubsection{Security and CCB}

The CCB server authorizes all daemons that register themselves with it
(using \Macro{CCB\_ADDRESS}) at the DAEMON authorization level (these
are playing the role of process A in the above description).  It
authorizes all connection requests (from process B) at the READ
authorization level.  As usual, whether process B authorizes process A
to do whatever it is trying to do is up to the security policy for
process B; from the Condor security model's point of view, it is as if
process A connected to process B, even though at the network layer,
the reverse is true.

\subsubsection{Troubleshooting CCB}

Errors registering with CCB or requesting connections via CCB are
logged at level \Dflag{ALWAYS} in the debugging log.  Search for
``CCB'' in the log message.  Command-line tools require the argument
\Opt{-debug} for this information to be visible.  To see details of
the CCB protocol add \Dflag{FULLDEBUG} to the debugging options for
the Condor subsystem you are interested in (or add it to
\MacroNI{ALL\_DEBUG} to get extra debugging from all Condor
components).

A daemon that has successfully registered itself with CCB will
advertise this fact in its address in its ClassAd.  Use
\Condor{status} \Opt{-long} to view this information.

\subsubsection{Scalability and CCB}

Any number of CCB servers may be used to serve a pool of Condor
daemons.  For example, half of the pool could use one CCB server and
half could use another.  Or for redundancy, all daemons could use both
CCB servers and then CCB connection requests will load-balance
across them.  Typically, the limit of how many daemons may be
registered with a single CCB server depends on the authentication
method used by the collector for DAEMON-level and READ-level access
and on the amount of memory available to the CCB server.  We are not
able to provide specific recommendations at this time, but to give a
very rough idea, a server class machine should be able to handle CCB
service plus normal collector service for a pool of a few thousand
slots without much trouble.

%%%%%%%%%%%%%%%%%%%%%%%%%%%%%%%%%%%%%%%%%%%%%%%%%%%%%%%%%%%%%%%%%%%%%%
\subsection{\label{sec:GCB}Generic Connection Brokering (GCB)}
%%%%%%%%%%%%%%%%%%%%%%%%%%%%%%%%%%%%%%%%%%%%%%%%%%%%%%%%%%%%%%%%%%%%%%
\index{GCB (Generic Connection Brokering)}

At this time, the functionality of GCB is being replaced by CCB.
Therefore, consider using CCB instead if it suits your needs.  The
main reason why CCB may not be sufficient is that GCB supports
communication between two different private networks, whereas CCB only
supports communication between nodes with one-directional connectivity
(i.e. node A cannot connect to node B but B can connect to A).  The main
reasons why CCB is preferable, when it provides the desired functionality,
are: support for all platforms (including Windows), easier configuration
and troubleshooting, and ability to restart/reconfigure on the fly.

Generic Connection Brokering, or GCB, is a system for managing network
connections across private network and firewall boundaries.
Condor's Linux releases are linked with GCB,
and can use GCB functionality to run jobs
(either directly or via flocking)
on pools that span public and private networks.

While GCB provides numerous advantages over restricting Condor to use
a range of ports which are then opened on the firewall (see
section~\ref{sec:Ports-Firewalls} on
page~\pageref{sec:Ports-Firewalls}),
GCB is also a very complicated system, with major
implications for Condor's networking and security functionality.
Therefore, sites must carefully weigh the 
advantages and disadvantages
of attempting
to configure and use GCB before making a decision.

Advantages:
\begin{itemize}

\item Better connectivity. GCB works with pools that have multiple
  private networks (even multiple private networks that use the same
  IP addresses (for example, 192.168.2.*).
  GCB also works with sites that use network address translation
  (NAT). 

\item More secure. Administrators never need to allow inbound
  connections through the firewall.
  With GCB, only outbound connections from behind the firewall must be
  allowed (which is a standard firewall configuration).
  It is possible to trade decreased performance for better security, and
  configure the firewall to only allow outbound connections to a
  single public IP address.

\item Does not require \Login{root} access to any machines.
  All parts of a GCB system can be run as an unprivileged user, and in
  the common case, no changes to the firewall configuration are
  required.

\end{itemize}

Disadvantages:
\begin{itemize}

\item The GCB broker 
  (section~\ref{sec:GCB-Broker-Intro} describes the broker)
  node(s) is a potential failure point to the pool.
  Any private nodes that want to communicate outside their own network
  must be represented by a GCB broker.
  This machine must be highly reliable, since if the broker is ever
  down, all inbound communication with the private nodes is
  impossible.
  Furthermore, no other Condor services should be run on a GCB broker
  (for example, the Condor pool's central manager).
  While it is possible to do so, it is not recommended.
  In general, no other services should be run on the machine at all,
  and the host should be dedicated to the task of serving as a GCB
  broker.

\item All Condor nodes behind a given firewall share a single IP
  address (the public IP address of their GCB broker).
  All Condor daemons using a GCB broker will advertise themselves with
  this single IP address, and in some cases, connections to/from those
  daemons will actually originate at the broker.
  This has implications for Condor's host/IP based security,
  and the general level of confusion for users and
  administrators of the pool.
  Debugging problems will be more difficult, as any log messages which 
  only print the IP address (not the name and/or port) will become ambiguous.
  Even log or error messages that include the port will not necessarily
  be helpful, as it is difficult to correlate ports on the broker
  with the corresponding private nodes.
  
\item Can not function with Kerberos authentication.
  Kerberos tickets include the IP address of the machine where they
  were created.
  However, when Condor daemons are using GCB, they use a different IP
  address, and therefore, any attempt to authenticate
  using Kerberos will fail, as Kerberos will consider this a (poor)
  attempt to fool it into using an invalid host principle.

\item Scalability and performance degradation: 
  \begin{itemize}
  \item Connections are more expensive to establish.
  \item In some cases, connections must be forwarded through a proxy
    server on the GCB broker.
  \item Each network port on each private node must correspond to a
    unique port on the broker host, so there is a fixed limit to how
    many private nodes a given broker can service (which is a function
    of the number of ports each private node requires and the total
    number of available ports on the broker).
  \item Each private node must maintain an open TCP connection to its
    GCB broker.  GCB will attempt to recover in the case of the socket
    being closed, but this means the broker must have at least as many
    sockets open as there are private nodes.
  \end{itemize}

\item It is more complex to configure and debug.

\end{itemize}

Given the increased complexity, use of GCB requires a careful
read of this entire manual section, followed by a thorough
installation.
%% Derek's use of non-subtle advice:
%% Please do not skim the information in this section, attempt to quickly
%% install GCB, and then ask for help when things go wrong.

Details of GCB and how it works can be found at the GCB
homepage:

\URL{http://www.cs.wisc.edu/condor/gcb}

This information is useful for understanding the technical
details of how GCB works, and the various parts of the system.
While some of the information is partly out of date (especially the
discussion of how to configure GCB) most of the sections are perfectly
accurate and worth reading.
Ignore the section on ``GCBnize'', which describes
how to get a given application to use GCB, as 
the Linux port of all Condor daemons and tools have already 
been converted to use GCB.



The rest of this section gives the details for configuring a
Condor pool to use GCB.
It is divided into the following topics:

\begin{itemize}
\item Introduction to the GCB broker
\item Configuring the GCB broker
\item Spawning a GCB broker (with a \Condor{master} or using \Prog{initd})
\item How to configure Condor machines to use GCB
\item Configuring the GCB routing table
\item Implications for Condor's host/IP security settings
\item Implications for other Condor configuration settings
\end{itemize}


%%%%%%%%%%%%%%%%%%%%%%%%%%%%%%%%%%%%%%%%%%%%%%%%%%%%%%%%%%%%%%%%%%%%%%
\subsubsection{\label{sec:GCB-Broker-Intro}Introduction to the GCB Broker}
%%%%%%%%%%%%%%%%%%%%%%%%%%%%%%%%%%%%%%%%%%%%%%%%%%%%%%%%%%%%%%%%%%%%%%

\index{GCB (Generic Connection Brokering)!broker}
\index{GCB (Generic Connection Brokering)!inagent}
At the heart of GCB is a logical entity known as a \Term{broker} or
\Term{inagent}.
In reality, the entity is made up of daemon
processes running on
the same machine comprised of the \Prog{gcb\_broker} and a set of
\Prog{gcb\_relay\_server} processes, each one spawned by the
\Prog{gcb\_broker}.

Every private network using GCB
must have at least one broker to arrange connections.
%Every private network in a given network topology trying to use GCB
%must have at least one corresponding broker to arrange connections for it.
The broker must be installed on a machine that nodes in both the
public and the private (firewalled) network can directly talk to.
The broker need not be able to initiate connections to the
private nodes.  
It can take advantage of the case where it can
initiate connections to the private nodes, and that will improve
performance. 
The broker is generally installed on a 
machine with multiple network interfaces
(on the network boundary) or just outside
of a network that allows outbound connections.
If the private network contains many hosts, sites can configure
multiple GCB brokers, and partition the private nodes so that different
subsets of the nodes use different brokers.

For a more thorough explanation of what a GCB broker is, check out:
\URL{http://www.cs.wisc.edu/\~{}sschang/firewall/gcb/mechanism.htm}

A GCB broker should generally be installed on a dedicated machine.
These are machines that are not running other Condor daemons or services.
If running any other Condor service 
(for example, the central manager of the pool)
on the same machine as the GCB broker,
all other machines attempting
to use this Condor service
(for example, to connect to the \Condor{collector} or \Condor{negotiator})
will incur additional connection costs and latency.
It is possible that future versions of GCB and Condor will be able to
overcome these limitations, but for now, we recommend that a broker
is run on a dedicated machine with no other Condor daemons (except
perhaps a single \Condor{master} used to spawn the \Prog{gcb\_broker}
daemon, as described below).

In principle, a GCB broker is a network element that functions almost
like a router.
It allows certain connections through the firewall by redirecting
connections or forwarding connections.
In general, it is not a good idea to run a lot of other services on
the network elements, especially not services like Condor which can
spawn arbitrary jobs.
Furthermore, the GCB broker relies on listening to many network
ports.
If other applications are running on the same host as the broker,
problems exist
where the broker does not have enough network
ports available to forward all the connections that might be required
of it.
Also, all nodes inside a private network rely on the GCB broker for
all incoming communication.
For performance reasons, avoid forcing the GCB broker to
contend with other processes for system resources, such that it is always
available to handle communication requests.
There is nothing in GCB or Condor requiring
the broker to run on a separate machine, 
but it is the recommended configuration.

\index{GCB broker!ports 65432 and 65430}
The \Prog{gcb\_broker} daemon listens on two hard-coded,
fixed ports (65432 and 65430).
A future version of Condor and GCB will remove this limitation.
However, for now, to run a \Prog{gcb\_broker} on a
given host, ensure that ports 65432 and 65430 are not already
in use. 

If \Login{root} access on a machine where a GCB 
broker is planned, one good option is to have \Prog{initd} configured to
spawn (and re-spawn) the \Prog{gcb\_broker} binary (which is located in
the \Release{libexec} directory).
This way, the \Prog{gcb\_broker} will be automatically restarted on
reboots, or in the event that the broker itself crashes or is killed.
Without \Login{root} access, use a \Condor{master} to
manage the \Prog{gcb\_broker} binary. 

%%%%%%%%%%%%%%%%%%%%%%%%%%%%%%%%%%%%%%%%%%%%%%%%%%
\subsubsection{\label{sec:GCB-Broker-Config}
Configuring the GCB broker}
%%%%%%%%%%%%%%%%%%%%%%%%%%%%%%%%%%%%%%%%%%%%%%%%%%

\index{GCB broker!configuration}
Since the \Prog{gcb\_broker} and \Prog{gcb\_relay\_server} are not
Condor daemons, they do not read the Condor configuration
files.
Therefore, they must be configured by other means, namely the
environment and through the use of command-line arguments.

There is one required command-line argument for the \Prog{gcb\_broker}.
This argument defines the public IP address this broker will use to
represent itself and any private network nodes that are configured to
use this broker.
This information is defined with \Opt{-i xxx.xxx.xxx.xxx} on the
command-line when the \Prog{gcb\_broker} is executed.
If the broker is being setup outside the private network, it is likely
that the machine will only have one IP address, which is clearly the
one to use.
However, if the broker is being run on a machine on the
network boundary (a multi-homed machine with interfaces into both the
private and public networks), be sure to use the IP address of the
interface on the public network.

Additionally, specify environment variables to control
how the \Prog{gcb\_broker} (and the \Prog{gcb\_relay\_server}
processes it spawns) will behave.
Some of these settings can also be specified as command-line
arguments to the \Prog{gcb\_broker}.
All of them have reasonable defaults if not defined.

\begin{itemize}

\item General daemon behavior

\begin{description}

\item The environment variable \Env{GCB\_RELAY\_SERVER} \label{Env:GCB-relay-server}
  defines the full path to the \File{gcb\_relay\_server} binary
  the broker should use.
  The command-line override for this is \Opt{-r /full/path/to/relayserver}.
  If not set either on the command-line or in the environment,
  the \Prog{gcb\_broker} process will search for a program named
  \File{gcb\_relay\_server} in the same directory where the
  \File{gcb\_broker} binary is located, and attempt to use that one.

\item The environment variable \Env{GCB\_ACTIVE\_TO\_CLIENT}
  \label{Env:GCB-active-to-client}
  is a boolean that defines whether the GCB broker can directly talk to servers
  running inside the network that it manages
  The value must be \verb@yes@ or \verb@no@, case sensitive.
  \Env{GCB\_ACTIVE\_TO\_CLIENT} should be set to \verb@yes@ only if
  this GCB broker is running on a network boundary and can connect to
  both the private and public nodes.
  If the broker is running in the public network, it should be left
  undefined or set to \verb@no@.

\end{description}

\item Log file locations

\begin{description}

\item The environment variable \Env{GCB\_LOG\_DIR} \label{Env:GCB-log-dir}
  defines a directory to use for all GCB-related log files.
  If defined, and the per-daemon log file settings (described
  below) are not defined, the broker will write to
  \verb@$GCB_LOG_DIR/BrokerLog@ and the relay server will write to
  \verb@$GCB_LOG_DIR/RelayServerLog.<pid>@

\item The environment variable \Env{GCB\_BROKER\_LOG} \label{Env:GCB-broker-log}
  defines the full path for the GCB broker's log file.
  The command-line override is \Opt{-l /full/path/to/log/file}.
  This definition overrides \Env{GCB\_LOG\_DIR}.

\item The environment variable \Env{GCB\_RELAY\_SERVER\_LOG}
  \label{Env:GCB-relay-server-log}
  defines the full path to the GCB relay server's log file.
  Each relay server writes its own log file, so the actual filename
  will be: \verb@$GCB_RELAY_SERVER_LOG.<pid>@ where \verb@<pid>@ is
  replaced with the process id of the corresponding
  \Prog{gcb\_relay\_server}.
  When defined, this setting overrides \Env{GCB\_LOG\_DIR}.

\end{description}

\item Verbose logging 

\begin{description}

\item The environment variable \Env{GCB\_DEBUG\_LEVEL} 
  \label{Env:GCB-DEBUG-LEVEL}
  controls how verbose all the GCB daemon's log files should be.
  Can be either \verb@fulldebug@ (more verbose) or \verb@basic@.
  This defines logging behavior for all GCB daemons, unless
  the following daemon-specific settings are defined.

\item The environment variable \Env{GCB\_BROKER\_DEBUG}
  \label{Env:GCB-BROKER-DEBUG}
  controls verbose logging specifically for the GCB broker.
  The command-line override for this is \Opt{-d level}.
  Overrides \Env{GCB\_DEBUG\_LEVEL}. 

\item The environment variable \Env{GCB\_RELAY\_SERVER\_DEBUG} 
  \label{Env:GCB-RELAY-SERVER-DEBUG}
  controls verbose logging specifically for the GCB relay server.  
  Overrides \Env{GCB\_DEBUG\_LEVEL}. 

\end{description}

\item Maximum log file size

\begin{description}

\item The environment variable \Env{GCB\_MAX\_LOG} \label{Env:GCB-MAX-LOG}
  defines the maximum size in bytes of all GCB log files.
  When the log file reaches this size, the content of the file will be
  moved to \File{filename.old}, and a new log is started.
  This defines logging behavior for all GCB daemons, unless
  the following daemon-specific settings are used.

\item The environment variable \Env{GCB\_BROKER\_MAX\_LOG} \label{Env:GCB-BROKER-MAX-LOG}
  defines the maximum size in bytes of the GCB broker log file.

\item The environment variable \Env{GCB\_RELAY\_SERVER\_MAX\_LOG} 
  \label{Env:GCB-RELAY-SERVER-MAX-LOG}
  defines the maximum size in bytes of the GCB relay server log file.

\end{description}

\end{itemize}

%%%%%%%%%%%%%%%%%%%%%%%%%%%%%%%%%%%%%%%%%%%%%%%%%%
\subsubsection{\label{sec:GCB-broker-spawn}
Spawning the GCB Broker}
%%%%%%%%%%%%%%%%%%%%%%%%%%%%%%%%%%%%%%%%%%%%%%%%%%

\index{GCB broker!how to spawn the broker}
There are two ways to spawn the GCB broker:

\begin{itemize}
\item Use a \Condor{master}.

To spawn the GCB broker with a \Condor{master}, here are 
the recommended \File{condor\_config} settings that will work:

\footnotesize
\begin{verbatim}
# Specify that you only want the master and the broker running
DAEMON_LIST = MASTER, GCB_BROKER

# Define the path to the broker binary for the master to spawn
GCB_BROKER = $(RELEASE_DIR)/libexec/gcb_broker

# Define the path to the release_server binary for the broker to use 
GCB_RELAY = $(RELEASE_DIR)/libexec/gcb_relay_server

# Setup the gcb_broker's environment.  We use a macro to build up the
# environment we want in pieces, and then finally define
# GCB_BROKER_ENVIRONMENT, the setting that condor_master uses.

# Initialize an empty macro
GCB_BROKER_ENV =

# (recommended) Provide the full path to the gcb_relay_server
GCB_BROKER_ENV = $(GCB_BROKER_ENV);GCB_RELAY_SERVER=$(GCB_RELAY)

# (recommended) Tell GCB to write all log files into the Condor log
# directory (the directory used by the condor_master itself)
GCB_BROKER_ENV = $(GCB_BROKER_ENV);GCB_LOG_DIR=$(LOG)
# Or, you can specify a log file separately for each GCB daemon:
#GCB_BROKER_ENV = $(GCB_BROKER_ENV);GCB_BROKER_LOG=$(LOG)/GCB_Broker_Log
#GCB_BROKER_ENV = $(GCB_BROKER_ENV);GCB_RELAY_SERVER_LOG=$(LOG)/GCB_RS_Log

# (optional -- only set if true) Tell the GCB broker that it can
# directly connect to machines in the private network which it is
# handling communication for.  This should only be enabled if the GCB
# broker is running directly on a network boundary and can open direct
# connections to the private nodes.
#GCB_BROKER_ENV = $(GCB_BROKER_ENV);GCB_ACTIVE_TO_CLIENT=yes

# (optional) turn on verbose logging for all of GCB
#GCB_BROKER_ENV = $(GCB_BROKER_ENV);GCB_DEBUG_LEVEL=fulldebug
# Or, you can turn this on separately for each GCB daemon:
#GCB_BROKER_ENV = $(GCB_BROKER_ENV);GCB_BROKER_DEBUG=fulldebug
#GCB_BROKER_ENV = $(GCB_BROKER_ENV);GCB_RELAY_SERVER_DEBUG=fulldebug

# (optional) specify the maximum log file size (in bytes)
#GCB_BROKER_ENV = $(GCB_BROKER_ENV);GCB_MAX_LOG=640000
# Or, you can define this separately for each GCB daemon:
#GCB_BROKER_ENV = $(GCB_BROKER_ENV);GCB_BROKER_MAX_LOG=640000
#GCB_BROKER_ENV = $(GCB_BROKER_ENV);GCB_RELAY_SERVER_MAX_LOG=640000

# Finally, set the value the condor_master really uses
GCB_BROKER_ENVIRONMENT = $(GCB_BROKER_ENV)

# If your Condor installation on this host already has a public
# interface as the default (either because it is the first interface
# listed in this machine's host entry, or because you've already
# defined NETWORK_INTERFACE), you can just use Condor's special macro
# that holds the IP address for this.
GCB_BROKER_IP = $(ip_address)
# Otherwise, you could define it yourself with your real public IP:
# GCB_BROKER_IP = 123.123.123.123

# (required) define the command-line arguments for the broker 
GCB_BROKER_ARGS = -i $(GCB_BROKER_IP)
\end{verbatim}
\normalsize

Once those settings are in place, either spawn or restart the
\Condor{master} and the \Prog{gcb\_broker} should be started.
Ensure the broker is running by reading the log file
specified with \Env{GCB\_BROKER\_LOG}, or in
\File{\MacroUNI{LOG}/BrokerLog} if using the default.


%%%%%%%%%%%%%%%%%%%%%%%%%%%%%%%%%%%%%%%%%%%%%%%%%%
%%\subsubsection{\label{sec:GCB-initd-spawn}
%%Using \Prog{initd} to spawn the GCB broker}
%%%%%%%%%%%%%%%%%%%%%%%%%%%%%%%%%%%%%%%%%%%%%%%%%%

\item Use \Prog{initd}.

The system's \Prog{initd} may be used to manage the
\Prog{gcb\_broker} without running the \Condor{master} on the broker
node, but this requires \Login{root} access.
Generally, this involves adding a line to the \File{/etc/inittab}
file.
Some sites use other means to manage and generate the
\File{/etc/inittab}, such as \Prog{cfengine} or other system configuration
management tools, so check with the local system administrator
to be sure.
An example line might be something like:

\footnotesize
\begin{verbatim}
GB:23:respawn:/path/to/gcb_broker -i 123.123.123.123 -r /path/to/relay_server
\end{verbatim}
\normalsize

It may be easier to wrap the \Prog{gcb\_broker} binary
in a shell script, in order to change the command-line arguments (and
set environment variables) without having to edit \File{/etc/inittab}
all the time.
This will be similar to:

\footnotesize
\begin{verbatim}
GB:23:respawn:/opt/condor-6.7.13/libexec/gcb_broker.sh
\end{verbatim}
\normalsize

Then, create the wrapper, as similar to: 

\footnotesize
\begin{verbatim}
#!/bin/sh

libexec=/opt/condor-6.7.13/libexec
ip=123.123.123.123
relay=$libexec/gcb_relay_server

exec $libexec/gcb_broker -i $ip -r $relay
\end{verbatim}
\normalsize

You will probably also want to set some environment variables to tell
the GCB daemons where to write their log files (\Env{GCB\_LOG\_DIR}),
and possibly some of the other variables described above.

Either way, after updating the \File{/etc/inittab}, send
the \Prog{initd} process (always PID 1) a \verb@SIGHUP@ signal, and it
will re-read the \File{inittab} and spawn the \Prog{gcb\_broker}.

\end{itemize}

%%%%%%%%%%%%%%%%%%%%%%%%%%%%%%%%%%%%%%%%%%%%%%%%%%%%%%%%%%%%%%%%%%%%%%
\subsubsection{\label{sec:GCB-condor-config}
Configuring Condor nodes to be GCB clients}
%%%%%%%%%%%%%%%%%%%%%%%%%%%%%%%%%%%%%%%%%%%%%%%%%%%%%%%%%%%%%%%%%%%%%%

\index{GCB (Generic Connection Brokering)!Condor client configuration}
In general, before configuring a node in a Condor pool to use GCB,
the GCB broker node(s) for the pool must be set up and running.
Set up, configure, and spawn the broker first.

To enable the use of GCB on a given Condor host, set the following
Condor configuration variables:

\footnotesize
\begin{verbatim}
# Tell Condor to use a network remapping service (currently only GCB
# is supported, but in the future, there might be other options)
NET_REMAP_ENABLE = true
NET_REMAP_SERVICE = GCB
\end{verbatim}
\normalsize

Only GCB clients within a private network need to define the following
variable, which specifies the IP addresses of the brokers serving this
network.
Note that these IP addresses must match the IP address
that was specified on each
broker's command-line with the \Opt{-i} option.

\footnotesize
\begin{verbatim}
# Public IP address (in standard dot notation) of the GCB broker(s)
# serving this private node.
NET_REMAP_INAGENT = xxx.xxx.xxx.xxx, yyy.yyy.yyy.yyy
\end{verbatim}
\normalsize

When more than one IP address is given, the \Condor{master} picks one at
random for it and all of its descendants to use.
Because the \MacroNI{NET\_REMAP\_INAGENT} setting is only
valid on private nodes, it should not be defined in a global
Condor configuration file (\File{condor\_config}) if the pool also
contains nodes on a public network.

Finally, if setting up  the recommended (but optional) GCB routing table, 
tell Condor daemons where to find their table.
Define the following variable:

\footnotesize
\begin{verbatim}
# The full path to the routing table used by GCB
NET_REMAP_ROUTE = /full/path/to/GCB-routing-table
\end{verbatim}
\normalsize

Setting \MacroNI{NET\_REMAP\_ENABLE} causes the
\Macro{BIND\_ALL\_INTERFACES} variable to be automatically set.
More information about this setting can be found in
section~\ref{sec:Using-BindAllInterfaces} on
page~\pageref{sec:Using-BindAllInterfaces}.
It would not hurt to place the following in the
configuration file near the other GCB-related settings,
just to remember it:

\footnotesize
\begin{verbatim}
# Tell Condor to bind to all network interfaces, instead of a single
# interface.
BIND_ALL_INTERFACES = true
\end{verbatim}
\normalsize

Once a GCB broker is set up and running to manage connections for
each private network, and the Condor installation for all the nodes in
either private and public networks are configured to enable GCB,
restart the Condor daemons, and all of the different machines
should be able to communicate with each other.


%%%%%%%%%%%%%%%%%%%%%%%%%%%%%%%%%%%%%%%%%%%%%%%%%%%%%%%%%%%%%%%%%%%%%%
\subsubsection{\label{sec:GCB-routing-table}Configuring the GCB
  routing table} 
%%%%%%%%%%%%%%%%%%%%%%%%%%%%%%%%%%%%%%%%%%%%%%%%%%%%%%%%%%%%%%%%%%%%%%

\index{GCB (Generic Connection Brokering)!GCB routing table configuration}
By default, a GCB-enabled application will always attempt to directly
connect to a given IP/port pair.
In the case of a private nodes being represented by a GCB broker, the
IP/port will be a proxy socket on the broker node, not the real
address at each private node.
When the GCB broker receives a direct connection to one of its proxy 
sockets, it notifies the corresponding private node, which
establishes a new connection to the broker.
The broker then forwards packets between these two sockets,
establishing a communication pathway into the private node.
This allows clients which are not linked with the GCB libraries to communicate
with private nodes using a GCB broker.

This mechanism is expensive
in terms of latency (time between messages) and total bandwidth
(how much data can be moved in a given time period),
as well as expensive in terms of the broker's
system resources such as network I/O, processor time, and memory.
This expensive mechanism is 
unnecessary in the case of
GCB-aware clients trying to connect to private nodes that can directly
communicate with the public host.
The alternative is to contact the GCB broker's command interface (the
fixed port where the broker is listening for GCB management commands),
and use
a GCB-specific protocol to request a connection to the given IP/port.
In this case, the GCB broker will notify the private node to directly
connect to the public client (technically, to a new socket created by
the GCB client library linked in with the client's application), and a
direct socket between the two is established, removing the need for
packet forwarding between the proxy sockets at the GCB broker.

On the other hand, in cases where a direct connection from the client
to a given server is possible
(for example, two GCB-aware clients in the same
public network attempting to communicate with each other),
it is
expensive and unnecessary to attempt to contact a GCB broker, and the
client should connect directly.

To allow a GCB-enabled client to know if it should make a direct
connection (which might involve packet forwarding through proxy
sockets), or if it should use the GCB protocol to communicate with the
broker's command port and arrange a direct socket,
GCB provides a \Term{routing table}.
Using this table, an administrator can define what IP addresses should
be considered private nodes where the GCB connection protocol will be
used, and what nodes are public, where a direct connection (without
incurring the latency of contacting the GCB broker, only to find out
there is no information about the given IP/port) should be made
immediately. 

If the attempt to contact the GCB broker for a given IP/port fails, or
if the desired port is not being managed by the broker, the GCB client
library making the connection will fall back and attempt a direct
connection.
Therefore, configuring a GCB routing table is not required for
communication to work within a GCB-enabled environment.
However, the GCB routing table can significantly improve performance
for communication with private nodes being represented by a GCB
broker. 

One confusing aspect of GCB is that all of the nodes on a private
network believe that their own IP address is the address of their GCB
broker.
Due to this, all the Condor daemons on a private network advertise
themselves with the same IP address (though the broker will map the
different ports to different nodes within the private network).
Therefore, a given node in the public network needs to be told that if
it is contacting this IP address, it should know that the IP address is really
a GCB broker representing a node in the private network, so that 
the public network node
can contact the broker to arrange a single socket from the private
node to the public one, instead of relying on forwarding packets
between proxy sockets at the broker.
Any other addresses, such as other public IP addresses, can be
contacted directly, without going through a GCB broker.
Similarly, other nodes within the same private network will still be
advertising their address with their GCB broker's public IP address.
So, nodes within the same private network also have to know that the
public IP address of the broker is really a GCB broker, yet all other public
IP addresses are valid for direct communication.

In general, all connections can be made directly, except to a host
represented by a GCB broker.
Furthermore, the default behavior of the GCB client library is to make
a direct connection.
The routing table is a (somewhat complicated) way to tell a
given GCB installation what GCB brokers it might have to communicate
with, and that it should directly communicate with anything else.
In practice, the routing table should have a single entry for
each GCB broker in the system.
Future versions of GCB will be able to make use of more complicated
routing behavior, which is why the full routing table infrastructure
described below is implemented, even if the current version of GCB is
not taking advantage of all of it.


%%%%%%%%%%%%%%%%%%%%%%%%%%%%%%%%%%%%%%%%%%%%%%%%%%%%%%%%%%%%%%%%%%%%%%
\Bold{Format of the GCB routing table}
%%%%%%%%%%%%%%%%%%%%%%%%%%%%%%%%%%%%%%%%%%%%%%%%%%%%%%%%%%%%%%%%%%%%%%
\index{GCB (Generic Connection Brokering)!GCB routing table syntax and examples}

The routing table is a plain ASCII text file.
Each line of the file contains one rule.
Each rule consists of a \Term{target} and a \Term{method}.
The target specifies destination IP address(es) to match, and the method
defines what mechanism must be used to connect to the given target.
The target must be a valid IP address string in the standard
dotted notation, followed by a slash character (\verb@/@),
as well as an integer \Term{mask}.
The mask specifies how many bits of the destination IP address
and target IP address must match.
The method must be one of the strings 
\begin{verbatim}
    GCB
    direct
\end{verbatim}
GCB stops searching the table as soon as it finds a matching rule,
therefore place more specific rules
(rules with a larger value for the mask and without wild cards)
before generic rules
(rules with wild cards or smaller mask values).
The default when no rule is matched is to use direct communication.
Some examples and the corresponding routing tables may help clarify
this syntax.


%%%%%%%%%%%%%%%%%%%%%%%%%%%%%%%%%%%%%%%%%%%%%%%%%%%%%%%%%%%%%%%%%%%%%%
\Bold{Simple GCB routing table example (1 private, 1 public)}
%%%%%%%%%%%%%%%%%%%%%%%%%%%%%%%%%%%%%%%%%%%%%%%%%%%%%%%%%%%%%%%%%%%%%%

Consider an example with
a private network that has a set of nodes whose IP
addresses are \verb@192.168.2.*@.
Other nodes are in a public network 
whose IP addresses are \verb@123.123.123.*@.
A GCB broker for the 192
network is running on IP address \verb@123.123.123.123@.
In this case, the routing table for both the public and private nodes
should be:

\begin{verbatim}
123.123.123.123/32 GCB
\end{verbatim}

This rule states that for IP addresses where all 32 bits exactly match
the address \verb@123.123.123.123@, first communicate with the GCB broker.

Since the default is to directly connect when no rule in the routing
table matches a given target IP, this single rule is all that is
required.
However, to illustrate how the routing table syntax works, the
following routing table is equivalent:

\begin{verbatim}
123.123.123.123/32 GCB
*/0 direct
\end{verbatim}

Any attempt to connect to \verb@123.123.123.123@ uses GCB,
as it is the first rule in the file.
All other IP addresses
will connect directly.
This table explicitly defines GCB's default behavior.

%%%%%%%%%%%%%%%%%%%%%%%%%%%%%%%%%%%%%%%%%%%%%%%%%%%%%%%%%%%%%%%%%%%%%%
\Bold{More complex GCB routing table example (2 private, 1 public)}
%%%%%%%%%%%%%%%%%%%%%%%%%%%%%%%%%%%%%%%%%%%%%%%%%%%%%%%%%%%%%%%%%%%%%%

As a more complicated case, consider a single Condor pool that
spans one public network and two private networks.
The two separate private networks each have machines
with private addresses like \verb@192.168.2.*@.
Identify one of these private networks as \verb@A@, and the other one
as \verb@B@. 
The public network has nodes with IP addresses like
\verb@123.123.123.*@.
Assume that the GCB broker for nodes in the \verb@A@ network 
has IP address
\verb@123.123.123.65@,
and the GCB broker for the nodes in the \verb@B@ network
has IP address
\verb@123.123.123.66@.
All of the nodes need to be able to talk to each other.
In this case, nodes in private network \verb@A@ advertise
themselves as \verb@123.123.123.65@, so any node, regardless of being
in A, B, or the public network, must treat that IP address as a GCB broker.
Similarly, nodes in private network \verb@B@ advertise 
themselves as \verb@123.123.123.66@, so any node, regardless of being
in A, B, or the public network, must treat that IP address as a GCB broker.
All other connections from any node can be made directly.
Therefore, here is the appropriate routing table for all nodes:

\begin{verbatim}
123.123.123.65/32 GCB
123.123.123.66/32 GCB
\end{verbatim}


%%%%%%%%%%%%%%%%%%%%%%%%%%%%%%%%%%%%%%%%%%%%%%%%%%%%%%%%%%%%%%%%%%%%%%
\subsubsection{\label{sec:GCB-host-security-implications}Implications
of GCB on Condor's Host/IP-based Security Configuration} 
%%%%%%%%%%%%%%%%%%%%%%%%%%%%%%%%%%%%%%%%%%%%%%%%%%%%%%%%%%%%%%%%%%%%%%

\index{GCB (Generic Connection Brokering)!security implications}
When a message is received at a Condor daemon's command socket,
Condor authenticates based on the IP
address of the incoming socket.
For more information about this host-based security in Condor, see
section~\ref{sec:Host-Security} on page~\pageref{sec:Host-Security}.
Because of the way GCB changes the IP addresses that are used and
advertised by GCB-enabled clients, and since all nodes being
represented by a GCB broker are represented by different ports on the
broker node (a process known as \Term{address leasing}), using GCB has
implications for this process.

Depending on the communication pathway used by a GCB-enabled Condor
client (either a tool or another Condor daemon) to connect to a given
Condor server daemon, and where in the network each side of the
connection resides, the IP address of the resulting socket actually
used will be very different.
In the case of a private client (that is, a client behind a firewall,
which may or may not be using NAT and a fully private, non-routable IP
address) attempting to connect to a server, there are three
possibilities: 

\begin{itemize}

  \item For a direct connection to another node within the private network,
  the server will see the private IP address of the client.

  \item For a direct outbound connection to a public node: if NAT is being
  used, the server will see the IP address of the NAT server for the private
  network.
  If there is no NAT, and the firewall is blocking connections in
  only one direction, but not re-writing IP addresses, the server will see
  the client's real IP address.

  \item For a connection to a host in a different private network that must
  be relayed through the GCB broker, the server will see the IP
  address of the GCB broker representing the server.
  This is an instance of the private server case, as
  described below.

\end{itemize}

Therefore, any public server that wants to allow a command from a
specific client must have any or all of the various IP addresses
mentioned above within the appropriate \MacroNI{HOSTALLOW} settings.  In
practice, that means opening up the \MacroNI{HOSTALLOW} settings to
include not just the actual IP addresses of each node, but also the IP
address of the various GCB brokers in use, and potentially, the public
IP address of the NAT host for each private network.

However, given that all private nodes which are represented by a
given GCB broker could potentially make connections to any other
host using the GCB broker's IP address (whenever proxy socket
forwarding is being used), if a single private node is being granted
a certain level of permission within the Condor pool,
all of the private nodes
using the same GCB broker will have the same level of permission.
This is particularly important in the consideration of granting
\Macro{HOSTALLOW\_ADMINISTRATOR} or \Macro{HOSTALLOW\_CONFIG}
privileges to a private node represented by a GCB broker.

In the case of a public client attempting to connect to a private
server, there are only two possible cases:

\begin{itemize}

  \item the GCB broker can arrange a direct socket from the private server.
  The private server will see the real public IP address of the client.

  \item the GCB broker must forward packets from a proxy socket.
  This may happen because
  of a non-GCB aware public client,
  a misconfigured or missing GCB routing table,
  or a client in a different private network.
  The private server will see the IP address of its own GCB broker.
  In the case where the GCB broker runs on a node on the network
  boundary, the private server will see the GCB broker's private IP
  address (even if the GCB broker is also listening on the public
  interface and the leased addresses it provides use the public IP
  addresses). 
  If the GCB broker is running entirely in the public network and cannot
  directly connect to the private nodes, the private server will see
  the remote connection as coming from the broker's public IP
  address.

\end{itemize}

This second case is particularly troubling.
Since there are legitimate circumstances where a private server would
need to use a forwarded proxy socket from its GCB broker, in general,
the server should allow requests originating from its GCB broker.
But, precisely because of the proxy forwarding, that implies that
\emph{any} client
that can connect to the GCB broker would be allowed into the
private server
(if IP-based authorization was the only defense).

The final host-based security setting that requires special mention is
\Macro{HOSTALLOW\_NEGOTIATOR}.
If the \Condor{negotiator} for the pool is running on a private node
being represented by a GCB broker, there must be
modifications to the default value.
For the purposes of Condor's host-based security, the
\Condor{negotiator} acts as a client when communicating with each 
\Condor{schedd} in the pool which has idle jobs that need to be
matched with available resources.
Therefore, all the possible cases of a private client attempting to
connect to a given server apply to a private \Condor{negotiator}.
In practice, that means adding the public IP address of the broker, the real
private IP address of the negotiator host, and possibly the 
public IP address of
the NAT host for this private network to the
\MacroNI{HOSTALLOW\_NEGOTIATOR} setting.
Unfortunately, this implies that \emph{any} host behind the
same NAT host or using
the same GCB broker will be authorized as if it was the
\Condor{negotiator}. 

Future versions of GCB and Condor will hopefully add some form of
authentication and authorization to the GCB broker itself, to help
alleviate these problems.
Until then, sites using GCB are encouraged to use GSI strong
authentication (since Kerberos also depends on IP addresses and is
therefore incompatible with GCB) to rely on an authorization system
that is not affected by address leasing.
This is especially true for sites that (foolishly) choose to run their
central manager on a private node.


%%%%%%%%%%%%%%%%%%%%%%%%%%%%%%%%%%%%%%%%%%%%%%%%%%%%%%%%%%%%%%%%%%%%%%
\subsubsection{\label{sec:GCB-config-implications}Implications of
GCB for Other Condor Configuration} 
%%%%%%%%%%%%%%%%%%%%%%%%%%%%%%%%%%%%%%%%%%%%%%%%%%%%%%%%%%%%%%%%%%%%%%

Using GCB and address leasing has implications for Condor
configuration settings outside of the Host/IP-based security
settings.
Each is described. 

\begin{description}

% i'm intentionally using \Macro, not \MacroNI for each of these,
% since they should show up in the index...

\item[\Macro{COLLECTOR\_HOST}]
  If the \Condor{collector} for the pool is running on a private node
  being represented by a GCB broker, \MacroNI{COLLECTOR\_HOST}
  must be set to the host name or IP address of the GCB broker machine,
  \emph{not} the real host name/IP address of the private node
  where the daemons are actually running.
  When the \Condor{collector} on the private node attempts to
  \Syscall{bind} to its command port (9618 by default), it will
  request port 9618 on the GCB broker node, instead.
  The port is not a worry, but the host name or IP address 
  is a worry.
  When public nodes want to communicate with the \Condor{collector},
  they must go through the GCB broker.
  In theory, other nodes inside the same private network could be told
  to directly use the private IP address of the \Condor{collector} host,
  but that is
  unnecessary, and would probably lead to other confusion and
  configuration problems.

  However, because the \Condor{collector} is listening on a fixed
  port, and that single port is reserved on the GCB broker node, no
  two private nodes using the same broker can attempt to use the same
  port for their \Condor{collector}.
  Therefore, any site that is attempting to set up multiple pools
  within the same private network is strongly encouraged to set up
  separate GCB brokers for each pool.
  Otherwise, one or both of the pools must use a
  non-standard port for the \Condor{collector}, which adds yet more
  complication to an already complicated situation. 

\item[\Macro{CKPT\_SERVER\_HOST}]
  Much like the case for \MacroNI{COLLECTOR\_HOST} described above,
  a checkpoint server on a private node will have to lease a port on
  the GCB broker node.
  However, the checkpoint server also uses a fixed port, and unlike
  the \Condor{collector}, there is no way to configure an alternate
  value.
  Therefore, only a single checkpoint server can be run behind a given
  GCB broker.
  The same solution works: if multiple checkpoint servers are required,
  multiple
  GCB brokers are deployed and configured.
  Furthermore, the host name of the GCB broker should be used as the
  value for \MacroNI{CKPT\_SERVER\_HOST}, not the real IP address or host name
  of the private node where the \Condor{ckpt\_server} is running.

\item[\Macro{SEC\_DEFAULT\_AUTHENTICATION\_METHODS}]
  \verb@KERBEROS@ may not be used for authentication
  on a GCB-enabled pool.
  The IP addresses used in various
  circumstances will not be the real IP addresses of the machines.
  Since Kerberos stores the IP address of each host as part of the
  Kerberos ticket, authentication will fail on a GCB-enabled
  pool.

\end{description}

Due to the complications and security limitations that arise from
running a central manager on a private node represented by GCB (both
regarding the \MacroNI{COLLECTOR\_HOST} and
\MacroNI{HOSTALLOW\_NEGOTIATOR}), we
recommend that sites avoid locating a central manager on a private
host whenever possible.



%%%%%%%%%%%%%%%%%%%%%%%%%%%%%%%%%%%%%%%%%%%%%%%%%%%%%%%%%%%%%%%%%%%%%%%%%%%
\subsection{\label{sec:tcp-collector-update}
Using TCP to Send Updates to the \Condor{collector}}
%%%%%%%%%%%%%%%%%%%%%%%%%%%%%%%%%%%%%%%%%%%%%%%%%%%%%%%%%%%%%%%%%%%%%%%%%%

\index{TCP}
\index{TCP!sending updates}
\index{UDP}
\index{UDP!lost datagrams}
\index{condor\_collector}

TCP sockets are reliable, connection-based sockets that guarantee
the delivery of any data sent.
However, TCP sockets are fairly expensive to establish, and there is more
network overhead involved in sending and receiving messages.

UDP sockets are datagrams, and are not reliable.
There is very little overhead in establishing or using a UDP socket,
but there is also no guarantee that the data will be delivered.
All previous Condor versions used UDP sockets to send updates to
the \Condor{collector}, and this did not cause problems.

Beginning with version 6.5.0, Condor can be configured to use TCP
sockets to send updates to the \Condor{collector} instead of
UDP datagrams.
It is \emph{not} intended for most sites.
This feature is targeted at sites where UDP updates are
lost because of the underlying network.
Most Condor administrators that believe this is a good idea for
their site are wrong.
Do not enable this feature just because it sounds like a good idea.
The only cases where an administrator would want this feature are if
the ClassAd updates are consistently not getting to the
\Condor{collector}.
An example where this may happen is if the pool is comprised of
machines across a wide area network (WAN) where UDP packets are
frequently dropped.

Configuration variables are set to enable the use of TCP sockets.
There are two variables that an
administrator must define to enable this feature:

\begin{itemize}

\item[\Macro{UPDATE\_COLLECTOR\_WITH\_TCP}]
  When set to TRUE, the Condor daemons to use TCP to
  update the \Condor{collector}, instead of the default UDP.
  Defaults to FALSE.

\item[\Macro{COLLECTOR\_SOCKET\_CACHE\_SIZE}] 
  Specifies the number of TCP sockets cached at the \Condor{collector}.
  The default value for this setting is 0, with no cache enabled.

\end{itemize}

The use of a cache allows Condor to leave established TCP sockets open,
facilitating much better performance.
Subsequent updates can reuse an already open socket.
The work to establish a TCP connection may be lengthy,
including authentication and setting up encryption.
Therefore, Condor requires that
a socket cache be defined if TCP updates are to be used.
TCP updates will be refused by the \Condor{collector} daemon
if a cache is not enabled.

Each Condor daemon will have 1 socket open to the \Condor{collector}.
So, in a pool with N machines, each of them running a \Condor{master},
\Condor{schedd}, and \Condor{startd}, the \Condor{collector} would
need a socket cache that has at least 3*N entries.
Machines running Personal Condor in the pool need
an additional two entries (for the \Condor{master} and
\Condor{schedd}) for each Personal Condor installation.

Every cache entry utilizes a file descriptor within the
\Condor{collector} daemon.
Therefore, be careful not to define a cache that
is larger than the number of file descriptors the underlying operating
system allocates for a single process.

\Note At this time, \MacroNI{UPDATE\_COLLECTOR\_WITH\_TCP}, only
affects the main \Condor{collector} for the site, not any sites that
a \Condor{schedd} might flock to.




% NAT -- Network address translation
% when access to internet uses a single IP addr/port,
%  but there are multiple computers communicating by this single
%  place
% The NAT is an extra layer that translates the multiple addr/port
%  to the single, and visa versa (from the single to one of the
%  multiple).

% Lore from Derek:
% however, "nat" is also one of those strange condor-team terms (like
% "frank") that has it's own, special meaning. :)
%
% a "nat" is a unit for measuring productivity (or lack thereof) in
% condor work over time.  it can be normalized to any time unit you
% want, by dividing the amount of work accomplished into the time scale
% you want.  1 nat is *very* little work over a given time period,
% almost too small to measure unless you use a long time scale.  at the
% time it first came up, we decided that jim basney *sleeps* at about 40
% nats, just for comparison. :)

