%%%%%%%%%%%%%%%%%%%%%%%%%%%%%%%%%%%%%%%%%%%%%%%%%%%%%%%%%%%%%%%%%%%%%%
\section{\label{sec:Networking}Networking (includes sections on Port Usage and CCB)}
%%%%%%%%%%%%%%%%%%%%%%%%%%%%%%%%%%%%%%%%%%%%%%%%%%%%%%%%%%%%%%%%%%%%%%
\index{network}

This section on
network communication in Condor
discusses which network ports are used,
how Condor behaves on machines with multiple network interfaces
and IP addresses,
and how to facilitate functionality in a pool that spans
firewalls and private networks.

The security section of the manual contains some
information that is relevant to the discussion of network
communication which will not be duplicated here, so please
see section~\ref{sec:Security} as well.

Firewalls, private networks, and network address translation (NAT)
pose special problems for Condor.
There are currently two main mechanisms for dealing with firewalls
within Condor:

\begin{enumerate}

\item Restrict Condor to use a specific range of port numbers, and
  allow connections through the firewall that use any port within the
  range.

\item Use \Term{Condor Connection Brokering} (CCB).

\end{enumerate}

Each method has its own advantages and disadvantages,
as described below.


%%%%%%%%%%%%%%%%%%%%%%%%%%%%%%%%%%%%%%%%%%%%%%%%%%%%%%%%%%%%%%%%%%%%%%
% all of these define their own \subsection, so include them directly
%%%%%%%%%%%%%%%%%%%%%%%%%%%%%%%%%%%%%%%%%%%%%%%%%%%%%%%%%%%%%%%%%%%%%%

%%%%%%%%%%%%%%%%%%%%%%%%%%%%%%%%%%%%%%%%%%%%%%%%%%%%%%%%%%%%%%%%%%%%%%%%%%%
\subsection{\label{sec:Port-Details}Port Usage in Condor}
%%%%%%%%%%%%%%%%%%%%%%%%%%%%%%%%%%%%%%%%%%%%%%%%%%%%%%%%%%%%%%%%%%%%%%%%%%

\index{port usage}


%%%%%%%%%%%%%%%%%%%%%%%%%%%%%%%%%%%%%%%%%%%%%%%%%%%%%%%%%%%%%%%%%%%%%%%%%%%
\subsubsection{\label{sec:Ports-Standard}Default Port Usage}
%%%%%%%%%%%%%%%%%%%%%%%%%%%%%%%%%%%%%%%%%%%%%%%%%%%%%%%%%%%%%%%%%%%%%%%%%%%

Every Condor daemon listens on a network port for incoming commands.
(Using \Condor{shared\_port}, this port may be shared between multiple
daemons.)
Most daemons listen on a dynamically assigned port.
In order to send a message,
Condor daemons and tools locate the correct port to use
by querying the \Condor{collector},
extracting the port number from the ClassAd.
One of the attributes included in every daemon's ClassAd is the full
IP address and port number upon which the daemon is listening.

To access the \Condor{collector} itself,
all Condor daemons and tools
must know the port number  where the \Condor{collector} is listening.
The \Condor{collector} is the only daemon with a well-known,
fixed port.
By default, Condor uses port 9618 for the \Condor{collector} daemon.
However, this port number can be changed (see below).

As an optimization for daemons and tools communicating with another
daemon that is running on the same host,
each Condor daemon can be configured to
write its IP address and port number into a well-known file.
The file names are controlled using the \MacroB{<SUBSYS>\_ADDRESS\_FILE}
configuration variables,
as described in section~\ref{param:SubsysAddressFile} on
page~\pageref{param:SubsysAddressFile}. 

\Note In the 6.6 stable series, and Condor versions earlier than
6.7.5, the \Condor{negotiator} also listened on a fixed, well-known
port (the default was 9614).
However, beginning with version 6.7.5, the \Condor{negotiator} behaves
like all other Condor daemons, and publishes its own ClassAd to the
\Condor{collector} which includes the dynamically assigned port 
the \Condor{negotiator} is listening on.
All Condor tools and daemons that need to communicate with the
\Condor{negotiator} will either use the
\Macro{NEGOTIATOR\_ADDRESS\_FILE} or will query the
\Condor{collector} for the \Condor{negotiator}'s ClassAd.

Sites that configure any checkpoint servers will introduce
other fixed ports into their network.
Each \Condor{ckpt\_server} will listen to 4 fixed ports: 5651, 5652,
5653, and 5654.
There is currently no way to configure alternative values for any of
these ports.


%%%%%%%%%%%%%%%%%%%%%%%%%%%%%%%%%%%%%%%%%%%%%%%%%%%%%%%%%%%%%%%%%%%%%%%%%%%
\subsubsection{\label{sec:Ports-NonStandard}Using 
a Non Standard, Fixed Port for the \Condor{collector}}
%%%%%%%%%%%%%%%%%%%%%%%%%%%%%%%%%%%%%%%%%%%%%%%%%%%%%%%%%%%%%%%%%%%%%%%%%%%
\index{port usage!nonstandard ports for central managers}
By default,
Condor uses port 9618 for the \Condor{collector} daemon.
To use a different port number for this daemon,
the configuration variables that tell Condor these communication
details are modified.
Instead of
\begin{verbatim}
CONDOR_HOST = machX.cs.wisc.edu
COLLECTOR_HOST = $(CONDOR_HOST)
\end{verbatim}
the configuration might be
\begin{verbatim}
CONDOR_HOST = machX.cs.wisc.edu
COLLECTOR_HOST = $(CONDOR_HOST):9650
\end{verbatim}

If a non standard port is defined, the same value of
\MacroNI{COLLECTOR\_HOST} (including the port) must be used for all
machines in the Condor pool.
Therefore, this setting should be modified in the global
configuration file (\File{condor\_config} file),
or the value must be duplicated across
all configuration files in the pool if a single configuration file
is not being shared.

When querying the \Condor{collector} for a remote pool that is running
on a non standard port, any Condor tool that accepts the \Opt{-pool}
argument can optionally be given a port number.  For example:
\footnotesize
\begin{verbatim}
        % condor_status -pool foo.bar.org:1234
\end{verbatim}
\normalsize


%%%%%%%%%%%%%%%%%%%%%%%%%%%%%%%%%%%%%%%%%%%%%%%%%%%%%%%%%%%%%%%%%%%%%%%%%%%
\subsubsection{\label{sec:Ports-Dynamic-Collector}Using 
a Dynamically Assigned Port for the \Condor{collector}}
%%%%%%%%%%%%%%%%%%%%%%%%%%%%%%%%%%%%%%%%%%%%%%%%%%%%%%%%%%%%%%%%%%%%%%%%%%%

On single machine pools, 
it is permitted to configure the
\Condor{collector} daemon
to use a dynamically assigned port,
as given out by the operating system.
This prevents port conflicts with other services on the same machine.
However, a dynamically assigned port is only to be used on
single machine Condor pools,
and only if the
\Macro{COLLECTOR\_ADDRESS\_FILE} 
configuration variable has also been defined.
This mechanism allows all of the Condor daemons and tools running on
the same machine to find the port upon which the \Condor{collector}
daemon is listening,
even when this port is not defined in the
configuration file and is not known in advance.

To enable the \Condor{collector} daemon to use a dynamically assigned port,
the port number is set to 0 in the \Macro{COLLECTOR\_HOST}
variable.
The \MacroNI{COLLECTOR\_ADDRESS\_FILE}
configuration variable must also be defined,
as it provides a known file where the IP address
and port information will be stored.
All Condor clients know to look at the
information stored in this file.
For example:
\footnotesize
\begin{verbatim}
COLLECTOR_HOST = $(CONDOR_HOST):0
COLLECTOR_ADDRESS_FILE = $(LOG)/.collector_address
\end{verbatim}
\normalsize

\Note Using a port of 0 for the \Condor{collector}
and specifying a
\MacroNI{COLLECTOR\_ADDRESS\_FILE}
only works in Condor version 6.6.8 or later in the 6.6 stable series,
and in version 6.7.4 or later in the 6.7 development series.
Do not attempt to do this with older versions of Condor.

Configuration definition of \MacroNI{COLLECTOR\_ADDRESS\_FILE}
is in section~\ref{param:SubsysAddressFile} on
page~\pageref{param:SubsysAddressFile},
and
\MacroNI{COLLECTOR\_HOST}
is in
section~\ref{param:CollectorHost} on
page~\pageref{param:CollectorHost}.


%%%%%%%%%%%%%%%%%%%%%%%%%%%%%%%%%%%%%%%%%%%%%%%%%%%%%%%%%%%%%%%%%%%%%%%%%%%
\subsubsection{\label{sec:Ports-Firewalls}Restricting Port Usage to
 Operate with Firewalls}
%%%%%%%%%%%%%%%%%%%%%%%%%%%%%%%%%%%%%%%%%%%%%%%%%%%%%%%%%%%%%%%%%%%%%%%%%%%

\index{port usage!firewalls}
If a Condor pool is completely behind a firewall,
then no special consideration or port usage is needed.
However, if there is a firewall between the machines within
a Condor pool, then
configuration variables may be set to force the usage of
specific ports, and to utilize a specific range of ports.

By default,
Condor uses port 9618 for the \Condor{collector} daemon,
and dynamic (apparently random) ports for everything else.
See section~\ref{sec:Ports-Dynamic-Collector},
if a dynamically assigned port is desired for the
\Condor{collector} daemon.

On platforms that support it, all of the Condor daemons on a machine
may be configured to share a single
port.  See section~\ref{sec:Config-shared-port} for more information.

The configuration variables
\Macro{HIGHPORT} and \Macro{LOWPORT} facilitate setting a restricted
range of ports that Condor will use.
This may be useful when some machines are behind a firewall.
The configuration macros
\MacroNI{HIGHPORT} and \MacroNI{LOWPORT} 
will restrict dynamic ports to the range specified.
The configuration variables are fully defined
in section~\ref{sec:Network-Related-Config-File-Entries}.
All of these ports must be greater than 0 and less than 65,536.
Note that both \MacroNI{HIGHPORT} and \MacroNI{LOWPORT} must be at 
least 1024 for Condor version 6.6.8.
In general, use ports greater than 1024,
in order
to avoid port conflicts with standard services on the machine.
Another reason for using ports greater than 1024 is that
daemons and tools are often not run as \Login{root},
and only \Login{root} may listen to a port lower than 1024.
Also, the range must include enough ports that are not in use, 
or Condor cannot work.

The range of ports assigned may be restricted based on 
incoming (listening) and outgoing (connect) ports
with the configuration variables
\Macro{IN\_HIGHPORT},
\Macro{IN\_LOWPORT},
\Macro{OUT\_HIGHPORT}, and
\Macro{OUT\_LOWPORT}.
See section~\ref{sec:Network-Related-Config-File-Entries}
for complete definitions of these configuration variables.
A range of ports lower than 1024 for daemons
running as \Login{root} is appropriate for incoming ports,
but not for outgoing ports.
The use of ports below 1024 (versus above 1024)
has security implications; 
therefore, it is inappropriate to assign a range that crosses
the 1024 boundary.


\Note Setting \MacroNI{HIGHPORT} and \MacroNI{LOWPORT} will not
automatically force the \Condor{collector} to bind to a port within
the range.
The only way to control what port the \Condor{collector} uses is by
setting the \MacroNI{COLLECTOR\_HOST} (as described above).

The total number of ports needed depends on the size of the pool,
the usage of the machines within the pool (which machines
run which daemons),
and the number of jobs that may execute at one time.
Here we discuss how many ports are used by each
participant in the system.  This assumes that \Condor{shared\_port}
is not being used.  If it \emph{is} being used, then all daemons
can share a single incoming port.

The central manager of the pool needs
\Expr{5 + \MacroNI{NEGOTIATOR\_SOCKET\_CACHE\_SIZE}}
ports for daemon communication,
where 
\Macro{NEGOTIATOR\_SOCKET\_CACHE\_SIZE}
is specified in the
configuration or defaults to the value 16.

Each execute machine (those machines running a \Condor{startd} daemon)
requires
\Expr{ 5 + (5 * number of slots advertised by that machine)}
ports.
By default, the number of slots advertised
will equal the number of physical CPUs in that machine.

Submit machines (those machines running a \Condor{schedd} daemon)
require
\Expr{ 5 + (5 *  \MacroNI{MAX\_JOBS\_RUNNING})} ports.
The configuration variable \Macro{MAX\_JOBS\_RUNNING}
limits (on a per-machine basis, if desired)
the maximum number of jobs.
Without this configuration macro,
the maximum number of jobs that could be simultaneously
executing at one time
is a function of the number of reachable execute machines. 

Also be aware that \MacroNI{HIGHPORT} and \MacroNI{LOWPORT}
only impact dynamic port selection used by the Condor system,
and they do not impact port selection used by jobs submitted to Condor.
Thus, jobs submitted to Condor that may create
network connections may not work in a port restricted environment.
For this reason, specifying \MacroNI{HIGHPORT} and \MacroNI{LOWPORT}
is not going to produce the
expected results if a user submits MPI applications to be executed under
the parallel universe.

Where desired, a local
configuration for machines \emph{not} behind a firewall
can override the usage of \MacroNI{HIGHPORT} and \MacroNI{LOWPORT},
such that the ports used for these machines are not restricted.
This can be accomplished by adding the following to the
local configuration file of those machines \emph{not}
behind a firewall:
\begin{verbatim}
HIGHPORT = UNDEFINED
LOWPORT  = UNDEFINED
\end{verbatim}


If the maximum number of ports allocated using 
\MacroNI{HIGHPORT} and \MacroNI{LOWPORT}
is too few,
socket binding errors of the form
\footnotesize
\begin{verbatim}
failed to bind any port within <$LOWPORT> - <$HIGHPORT>
\end{verbatim}
\normalsize
are likely to appear repeatedly in log files.


%%%%%%%%%%%%%%%%%%%%%%%%%%%%%%%%%%%%%%%%%%%%%%%%%%%%%%%%%%%%%%%%%%%%%%%%%%%
\subsubsection{\label{sec:Ports-MultipleCollectors}Multiple Collectors}
%%%%%%%%%%%%%%%%%%%%%%%%%%%%%%%%%%%%%%%%%%%%%%%%%%%%%%%%%%%%%%%%%%%%%%%%%%%
\index{port usage!multiple collectors}
\Todo


%%%%%%%%%%%%%%%%%%%%%%%%%%%%%%%%%%%%%%%%%%%%%%%%%%%%%%%%%%%%%%%%%%%%%%%%%%%
\subsubsection{\label{sec:Ports-Conflicts}Port Conflicts}
%%%%%%%%%%%%%%%%%%%%%%%%%%%%%%%%%%%%%%%%%%%%%%%%%%%%%%%%%%%%%%%%%%%%%%%%%%%
\index{port usage!conflicts}
\Todo



%%%%%%%%%%%%%%%%%%%%%%%%%%%%%%%%%%%%%%%%%%%%%%%%%%%%%%%%%%%%%%%%%%%%%%%%%%%
\subsection{\label{sec:shared-port-daemon}Reducing Port Usage with the \Condor{shared\_port} Daemon}
%%%%%%%%%%%%%%%%%%%%%%%%%%%%%%%%%%%%%%%%%%%%%%%%%%%%%%%%%%%%%%%%%%%%%%%%%%%

\index{HTCondor daemon!condor\_shared\_port@\Condor{shared\_port}}
\index{daemon!condor\_shared\_port@\Condor{shared\_port}}
\index{condor\_shared\_port daemon}
The \Condor{shared\_port} is an optional daemon
responsible for creating a TCP listener port shared by all of the
HTCondor daemons for which the configuration variable
\Macro{USE\_SHARED\_PORT} is \Expr{True}.
The \Condor{master} will invoke the \Condor{shared\_port} daemon
if it is listed in \MacroNI{DAEMON\_LIST}.  For further configuration
options, such as specifying the port number to use, see page~\pageref{sec:Config-shared-port}.

The main purpose of the \Condor{shared\_port} daemon is to reduce the
number of ports that must be opened when HTCondor needs to be
accessible through a firewall.
This has a greater security benefit
than simply reducing the number of open ports.
Without the \Condor{shared\_port} daemon,
one can configure HTCondor to use a range of ports,
but since some HTCondor daemons are created dynamically, 
this full range of ports will not be in use by HTCondor at all times.
This implies that other non-HTCondor processes not intended to be exposed to
the outside network could unintentionally bind to ports in the range
intended for HTCondor,
unless additional steps are taken to control access to those ports.  
While the \Condor{shared\_port} daemon is running,
it is exclusively bound to its port, which means that other non-HTCondor
processes cannot accidentally bind to that port.

A secondary benefit of the \Condor{shared\_port} daemon
is that it helps address the scalability issues of a submit machine.
Without the \Condor{shared\_port} daemon,
approximately 2.1 ephemeral ports per running job are required,
and possibly more, depending on the rate of job completion.
There are only 64K ports in total,
and most standard Unix installations only allocate a subset of
these as ephemeral ports.
In practice, with long running jobs,
and with between 11K and 14K simultaneously running jobs,
port exhaustion has been observed in typical Linux installations.
After increasing the ephemeral port range as to as many as possible,
port exhaustion occurred between 20K and 25K running jobs.
Using the \Condor{shared\_port} daemon,
each running job requires fewer, approximately 1.1 ephemeral ports
on the submit node, if HTCondor on the submit node connects directly
to HTCondor on the execute node.
If the submit node connects via CCB to the execute
node, \emph{no} ports are required per running job; only the one port
allocated to the \Condor{shared\_port} daemon is used.

When CCB is utilized via setting the configuration variable
\Macro{CCB\_ADDRESS},
the \Condor{shared\_port} daemon registers with
the CCB server on behalf of all daemons sharing the port.
This means that it is not possible to individually enable or disable
CCB connectivity to daemons that are using the shared port;
they all effectively share the same setting,
and the \Condor{shared\_port} daemon handles all CCB connection
requests on their behalf.

HTCondor's authentication and authorization steps are unchanged by the
use of a shared port.  Each HTCondor daemon continues to operate
according to its configured policy.  Requests for connections to the
shared port are not authenticated or restricted by
the \Condor{shared\_port} daemon.
They are simply passed to the requested daemon,
which is then responsible for enforcing the security policy.

When the \Condor{master} is configured to use the shared port
by setting the configuration variable
\begin{verbatim}
  USE_SHARED_PORT = True
\end{verbatim}
the \Condor{shared\_port} daemon is treated specially. 
A command such as \Condor{off},
which shuts down all daemons except for the \Condor{master},
will also leave the \Condor{shared\_port} running.
This prevents the \Condor{master} from getting into a state
where it can no longer receive commands.

The \Condor{collector} daemon typically has its own port;
it uses 9618 by default.
However, it can be configured to use a shared port.
Since the address of the \Condor{collector} must be set in 
the configuration file,
it is necessary to specify the shared port socket name of 
the \Condor{collector},
so that connections to the shared port that are intended for 
the \Condor{collector} can be forwarded to it.
If the shared port number is 11000, a \Condor{collector} address using this
shared port could be configured:

\footnotesize
\begin{verbatim}
COLLECTOR_HOST = collector.host.name:11000?sock=collector
\end{verbatim}
\normalsize

This configuration assumes that the socket name used by 
the \Condor{collector} is \Expr{collector}.
The \Condor{collector} that runs on \Expr{collector.host.name}
will automatically choose this socket name if \MacroNI{COLLECTOR\_HOST}
is configured as in the example above.
If multiple \Condor{collector} daemons are started on the same
machine, the socket name can be explicitly set in the daemon arguments,
as in the example:

\begin{verbatim}
COLLECTOR_ARGS = -sock collector
\end{verbatim}

When the \Condor{collector} address is a shared port,
TCP updates will be automatically used instead of UDP.
Under Unix, this means that the
\Condor{collector} daemon should be configured to have enough file descriptors.
See section~\ref{sec:tcp-collector-update} for more information on using
TCP within HTCondor.

SOAP commands cannot be sent over a shared port.
However, a daemon may be configured to open a fixed, non-shared port,
in addition to using a shared port.
This is done both by setting
\Expr{USE\_SHARED\_PORT = True} and by specifying a fixed port for the daemon
using \verb|<SUBSYS>_ARGS = -p <portnum>|.

The TCP connections required to manage standard universe jobs do not
make use of shared ports.



%%%%%%%%%%%%%%%%%%%%%%%%%%%%%%%%%%%%%%%%%%%%%%%%%%%%%%%%%%%%%%%%%%%%%%%%%%%
\subsection{\label{sec:Multiple-Interfaces}Configuring Condor for
Machines With Multiple Network Interfaces } 
%%%%%%%%%%%%%%%%%%%%%%%%%%%%%%%%%%%%%%%%%%%%%%%%%%%%%%%%%%%%%%%%%%%%%%%%%%

\index{multiple network interfaces}
\index{network interfaces!multiple}
\index{NICs}

Condor can run on machines with
multiple network interfaces.
Starting with Condor version 6.7.13
(and therefore all Condor 6.8 and more recent versions),
new functionality is
available that allows even better support for multi-homed
machines, using the configuration variable \MacroNI{BIND\_ALL\_INTERFACES}.
A multi-homed machine is one that has more than one
NIC (Network Interface Card).
Further improvements to this new functionality will remove the need
for any special configuration in the common case.
For now, care
must still be given to machines with multiple NICs, even
when using this new configuration variable.


%%%%%%%%%%%%%%%%%%%%%%%%%%%%%%%%%%%%%%%%%%%%%%%%%%%%%%%%%%%%
\subsubsection{\label{sec:Using-BindAllInterfaces}Using 
BIND\_ALL\_INTERFACES}
%%%%%%%%%%%%%%%%%%%%%%%%%%%%%%%%%%%%%%%%%%%%%%%%%%%%%%%%%%%%

Machines can be configured such that
whenever Condor daemons or tools
call \Syscall{bind}, the daemons or tools use all network interfaces on
the machine.
This means that outbound connections will always use the appropriate
network interface to connect to a remote host,
instead of being forced to use
an interface that might not have a route to the given destination.
Furthermore, sockets upon which a daemon listens for incoming connections 
will be bound to all network interfaces on the machine.
This means that so long as remote clients know the right port, they can
use any IP address on the machine and still contact a given Condor daemon.

This functionality is on by default.  To disenable this functionality, 
the boolean configuration
variable
\MacroNI{BIND\_ALL\_INTERFACES}
is defined and set to \Expr{False}:

\begin{verbatim}
BIND_ALL_INTERFACES = FALSE
\end{verbatim}

This functionality has limitations.
Here are descriptions of the limitations.

\begin{description}

\item[Using all network interfaces does not work with Kerberos.] 
  Every Kerberos ticket contains a specific IP address within it.
  Authentication over a socket (using Kerberos) requires
  the socket to also specify that same specific IP address.
  Use of \MacroNI{BIND\_ALL\_INTERFACES} causes outbound
  connections from a multi-homed machine to 
  originate over any of the interfaces.
  Therefore, the IP address of the outbound connection and the IP
  address in the Kerberos ticket will not necessarily match,
  causing the authentication to fail.
  Sites using Kerberos authentication on multi-homed machines are
  strongly encouraged not to enable \MacroNI{BIND\_ALL\_INTERFACES},
  at least until Condor's Kerberos functionality
  supports using multiple Kerberos tickets together with finding the right one
  to match the IP address a given socket is bound to. 

\item[There is a potential security risk.]
  Consider the following example of a security risk.
  A multi-homed machine is at a network boundary.
  One interface is on the public Internet, while the other connects to
  a private network.
  Both the multi-homed machine and the private network machines
  comprise a Condor pool.
  If the multi-homed machine enables \MacroNI{BIND\_ALL\_INTERFACES},
  then it is at risk from hackers trying to compromise the security of the pool.
  Should this multi-homed machine be compromised,
  the entire pool is vulnerable.
  Most sites in this situation would run an \Prog{sshd} on the
  multi-homed machine so that remote users who wanted to access the
  pool could log in securely and use the Condor tools directly.
  In this case, remote clients do not need to use Condor tools running
  on machines in the public network to access the Condor daemons on
  the multi-homed machine.
  Therefore, there is no reason to have Condor daemons listening on
  ports on the public Internet, causing a potential security threat.

\item[Only one IP address will be advertised.]
  At present, even though a given Condor daemon will be listening to
  ports on multiple interfaces, each with their own IP address,
  there is currently no mechanism for that daemon to advertise all of
  the possible IP addresses where it can be contacted.
  Therefore, Condor clients (other Condor daemons or tools) will not
  necessarily able to locate and communicate with a given daemon
  running on a multi-homed machine where
  \MacroNI{BIND\_ALL\_INTERFACES} has been enabled.

  Currently, Condor daemons can only advertise a single IP address in
  the ClassAd they send to their \Condor{collector}.
  Condor tools and other daemons only know how to look up a single IP
  address, and they attempt to use that single IP address
  when connecting to the daemon.
  So, even if the daemon is listening on 2 or more different interfaces,
  each with a separate IP, the daemon must choose what IP address to
  publicly advertise so that other daemons and tools can locate it.

  By default, Condor advertises the IP address of the network interface
  used to contact the collector, since this is the most likely to be
  accessible to other processes that query the same collector.
  The \Macro{NETWORK\_INTERFACE} setting can still be used to
  specify the IP address Condor should advertise, even if
  \MacroNI{BIND\_ALL\_INTERFACES} is set to \Expr{True}.
  Therefore, some of the considerations described below regarding what
  interface should be used in various situations still apply when
  deciding what interface is to be advertised.

\end{description}

Sites that make heavy use of private networks and multi-homed machines
should consider if using Generic Connection Brokering, GCB, is
right for them.
More information about GCB and Condor can be found in
section~\ref{sec:GCB} on page~\pageref{sec:GCB}.


%%%%%%%%%%%%%%%%%%%%%%%%%%%%%%%%%%%%%%%%%%%%%%%%%%%%%%%%%%%%
\subsubsection{Central Manager with Two or More NICs}
%%%%%%%%%%%%%%%%%%%%%%%%%%%%%%%%%%%%%%%%%%%%%%%%%%%%%%%%%%%%

Often users of Condor wish to set up ``compute farms'' where there is one
machine with two network interface cards (one for the public Internet,
and one for the private net). It is convenient to set up the ``head''
node as a central manager in most cases and so here are the instructions
required to do so.

Setting up the central manager on a machine with more than one NIC can
be a little confusing because there are a few external variables
that could make the process difficult. One of the biggest mistakes
in getting this to work is that either one of the separate interfaces is
not active, or the host/domain names associated with the interfaces are
incorrectly configured. 

Given that the interfaces are up and functioning, and they have good
host/domain names associated with them here is how to configure Condor:

In this example, \Bold{farm-server.farm.org} maps to the private interface.

On the central manager's global (to the cluster) configuration file: \\
\Macro{CONDOR\_HOST} = \Bold{farm-server.farm.org}

On your central manager's local configuration file: \\
\MacroNI{NETWORK\_INTERFACE} = ip address of \Bold{farm-server.farm.org} \\
\MacroNI{NEGOTIATOR} = \MacroUNI{SBIN}/condor\_negotiator \\
\MacroNI{COLLECTOR} = \MacroUNI{SBIN}/condor\_collector \\
\MacroNI{DAEMON\_LIST} = \MacroNI{MASTER}, \MacroNI{COLLECTOR}, \MacroNI{NEGOTIATOR}, \MacroNI{SCHEDD}, \MacroNI{STARTD}

If your central manager and farm machines are all NT, then you only have
vanilla universe and it will work now.  However, if you have this setup
for UNIX, then at this point, standard universe jobs should be able to
function in the pool, but if you did not configure the \Macro{UID\_DOMAIN}
macro to be homogeneous across the farm machines, the standard universe
jobs will run as \Bold{nobody} on the farm machines.

In order to get vanilla jobs and file server load balancing for standard
universe jobs working (under Unix), do some more work both in
the cluster you have put together and in Condor to make everything work.
First, you need a file server (which could also be the central manager) to
serve files to all of the farm machines. This could be NFS or AFS, it does
not really matter to Condor. The mount point of the directories you wish
your users to use must be the same across all of the farm machines. Now,
configure \Macro{UID\_DOMAIN} and \Macro{FILESYSTEM\_DOMAIN} to be
homogeneous across the farm machines and the central manager. Now, you
will have to inform Condor that an NFS or AFS filesystem exists and that
is done in this manner. In the global (to the farm) configuration file:

\begin{verbatim}
# If you have NFS
USE_NFS = True
# If you have AFS
HAS_AFS = True
USE_AFS = True
# if you want both NFS and AFS, then enable both sets above
\end{verbatim}

Now, if you've set up your cluster so that it is possible for a machine
name to never have a domain name (for example: there is machine
name but no fully qualified domain name in \File{/etc/hosts}), you must
configure \Macro{DEFAULT\_DOMAIN\_NAME} to be the domain that you wish
to be added on to the end of your host name.


%%%%%%%%%%%%%%%%%%%%%%%%%%%%%%%%%%%%%%%%%%%%%%%%%%%%%%%%%%%%
\subsubsection{A Client Machine with Multiple Interfaces}
%%%%%%%%%%%%%%%%%%%%%%%%%%%%%%%%%%%%%%%%%%%%%%%%%%%%%%%%%%%%

If you have a client machine with two or more NICs, then there might be
a specific network interface with which you desire a client machine to
communicate with the rest of the Condor pool. In this case, in the local
configuration file for that machine, place: \\ 
\Macro{NETWORK\_INTERFACE} = ip address of interface desired \\


%%%%%%%%%%%%%%%%%%%%%%%%%%%%%%%%%%%%%%%%%%%%%%%%%%%%%%%%%%%%
\subsubsection{A Checkpoint Server on a Machine with Multiple NICs}
%%%%%%%%%%%%%%%%%%%%%%%%%%%%%%%%%%%%%%%%%%%%%%%%%%%%%%%%%%%%

If your Checkpoint Server is on a machine with multiple interfaces,
the only way to get things to work is if your different interfaces
have different host names associated with them, and you set
\Macro{CKPT\_SERVER\_HOST} to the host name that corresponds with the
IP address you want to use in the global configuration file for your pool.
You will still need to specify \Macro{NETWORK\_INTERFACE} in the local
config file for your Checkpoint Server.



%%%%%%%%%%%%%%%%%%%%%%%%%%%%%%%%%%%%%%%%%%%%%%%%%%%%%%%%%%%%%%%%%%%%%%%%%%%
\subsection{\label{sec:tcp-collector-update}
Using TCP to Send Updates to the \Condor{collector}}
%%%%%%%%%%%%%%%%%%%%%%%%%%%%%%%%%%%%%%%%%%%%%%%%%%%%%%%%%%%%%%%%%%%%%%%%%%

\index{TCP}
\index{TCP!sending updates}
\index{UDP}
\index{UDP!lost datagrams}
\index{condor\_collector}

TCP sockets are reliable, connection-based sockets that guarantee
the delivery of any data sent.
However, TCP sockets are fairly expensive to establish, and there is more
network overhead involved in sending and receiving messages.

UDP sockets are datagrams, and are not reliable.
There is very little overhead in establishing or using a UDP socket,
but there is also no guarantee that the data will be delivered.
All previous Condor versions used UDP sockets to send updates to
the \Condor{collector}, and this did not cause problems.

Beginning with version 6.5.0, Condor can be configured to use TCP
sockets to send updates to the \Condor{collector} instead of
UDP datagrams.
It is \emph{not} intended for most sites.
This feature is targeted at sites where UDP updates are
lost because of the underlying network.
Most Condor administrators that believe this is a good idea for
their site are wrong.
Do not enable this feature just because it sounds like a good idea.
The only cases where an administrator would want this feature are if
the ClassAd updates are consistently not getting to the
\Condor{collector}.
An example where this may happen is if the pool is comprised of
machines across a wide area network (WAN) where UDP packets are
frequently dropped.

Configuration variables are set to enable the use of TCP sockets.
There are two variables that an
administrator must define to enable this feature:

\begin{itemize}

\item[\Macro{UPDATE\_COLLECTOR\_WITH\_TCP}]
  When set to TRUE, the Condor daemons to use TCP to
  update the \Condor{collector}, instead of the default UDP.
  Defaults to FALSE.

\item[\Macro{COLLECTOR\_SOCKET\_CACHE\_SIZE}] 
  Specifies the number of TCP sockets cached at the \Condor{collector}.
  The default value for this setting is 0, with no cache enabled.

\end{itemize}

The use of a cache allows Condor to leave established TCP sockets open,
facilitating much better performance.
Subsequent updates can reuse an already open socket.
The work to establish a TCP connection may be lengthy,
including authentication and setting up encryption.
Therefore, Condor requires that
a socket cache be defined if TCP updates are to be used.
TCP updates will be refused by the \Condor{collector} daemon
if a cache is not enabled.

Each Condor daemon will have 1 socket open to the \Condor{collector}.
So, in a pool with N machines, each of them running a \Condor{master},
\Condor{schedd}, and \Condor{startd}, the \Condor{collector} would
need a socket cache that has at least 3*N entries.
Machines running Personal Condor in the pool need
an additional two entries (for the \Condor{master} and
\Condor{schedd}) for each Personal Condor installation.

Every cache entry utilizes a file descriptor within the
\Condor{collector} daemon.
Therefore, be careful not to define a cache that
is larger than the number of file descriptors the underlying operating
system allocates for a single process.

\Note At this time, \MacroNI{UPDATE\_COLLECTOR\_WITH\_TCP}, only
affects the main \Condor{collector} for the site, not any sites that
a \Condor{schedd} might flock to.




% NAT -- Network address translation
% when access to internet uses a single IP addr/port,
%  but there are multiple computers communicating by this single
%  place
% The NAT is an extra layer that translates the multiple addr/port
%  to the single, and visa versa (from the single to one of the
%  multiple).

% Lore from Derek:
% however, "nat" is also one of those strange condor-team terms (like
% "frank") that has it's own, special meaning. :)
%
% a "nat" is a unit for measuring productivity (or lack thereof) in
% condor work over time.  it can be normalized to any time unit you
% want, by dividing the amount of work accomplished into the time scale
% you want.  1 nat is *very* little work over a given time period,
% almost too small to measure unless you use a long time scale.  at the
% time it first came up, we decided that jim basney *sleeps* at about 40
% nats, just for comparison. :)

