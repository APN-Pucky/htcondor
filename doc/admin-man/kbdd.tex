%%%%%%%%%%%%%%%%%%%%%%%%%%%%%%%%%%%%%%%%%%%%%%%%%%%%%%%%%%%%%%%%%%%%%%
\subsection{\label{sec:kbdd}The \Condor{kbdd}}
%%%%%%%%%%%%%%%%%%%%%%%%%%%%%%%%%%%%%%%%%%%%%%%%%%%%%%%%%%%%%%%%%%%%%%

The Condor keyboard daemon (\Condor{kbdd}) monitors X events on
machines where the operating system does not provide a way of
monitoring the idle time of the keyboard or mouse.
It is not needed for most platforms, as Condor has other ways
of detecting keyboard and mouse activity.

Although great measures have been taken to make this daemon as robust
as possible, the X window system was not designed to facilitate such a
need, and thus is less then optimal on machines where many users log
in and out on the console frequently.

In order to work with X authority, the system by which X authorizes
processes to connect to X servers, the \Condor{kbdd} needs to
run with super user privileges.  Currently, the daemon assumes that X
uses the \Env{HOME} environment variable in order to locate a file
named \File{.Xauthority}, which contains keys necessary to connect to
an X server.  The keyboard daemon attempts to set this environment
variable to various users home directories in order to gain a
connection to the X server and monitor events.  This may fail to work
on your system, if you are using a non-standard approach.  If the
keyboard daemon is not allowed to attach to the X server, the state of
a machine may be incorrectly set to idle when a user is, in fact,
using the machine.

In some environments, the  \Condor{kbdd} will not be able to connect to the X
server because the user currently logged into the system keeps their
authentication token for using the X server in a place that no local user on
the current machine can get to.  
This may be the case for AFS where
the user's \File{.Xauthority} file is in an AFS home directory.
There may also
be cases where the  \Condor{kbdd} may not be run with super user privileges
because of political reasons,
but it is still desired to be able to monitor X activity.
In these cases, change the XDM configuration in order to
start up the \Condor{kbdd} with the permissions of the currently logging in
user.  Although your situation may differ, if you are running X11R6.3, you
will probably want to edit the files in \File{/usr/X11R6/lib/X11/xdm}.
The \File{.xsession}
file should have the keyboard daemon start up at the end,
and the \File{.Xreset} file
should have the keyboard daemon shut down.  
The \Opt{-l} option can be used to write the daemon's log file to a
place where the user running the daemon has permission to write a file.  We
recommend something akin to \File{\$HOME/.kbdd.log},
since this is a place where every
user can write, and it will not get in the way.
The \Opt{-pidfile} and \Opt{-k}
options allow
for easy shut down of the daemon by storing the process id in a file.  
It will be necessary
to add lines to the XDM configuration that look something like:

\footnotesize
\begin{verbatim}
  condor_kbdd -l $HOME/.kbdd.log -pidfile $HOME/.kbdd.pid
\end{verbatim}
\normalsize

This will start the \Condor{kbdd} as the user who is currently logging in
and write the log to a file in the directory 
\File{\$HOME/.kbdd.log/}.  Also, this
will save the process id of the daemon to \File{\Tilde/.kbdd.pid}, so that when the user
logs out, XDM can do:

\footnotesize
\begin{verbatim}
  condor_kbdd -k $HOME/.kbdd.pid
\end{verbatim}
\normalsize

This will shut down the process recorded in \File{\Tilde/.kbdd.pid} and exit.

To see how well the keyboard daemon is working, review
the log for the daemon and look for successful connections to the X
server.  If there are none, the \Condor{kbdd}
is unable to connect to the machine's X server.

