	The condor keyboard daemon monitors X events on machines where the
operating system does not provide a way of monitoring the idle time of the
keyboard or mouse.  In particular, this is necessary on Digial Unix machines
and IRIX machines.  NOTE: If you are running on Solaris, Linux, or HP/UX, you
do not need to use the keyboard daemon. Although great measures have been
taken to make this daemon as robust as possible, the X window system was not
designed to facilitate such a need, and thus is less then optimal on machines
where many users log in and out on the console frequently.

	In order to work with X authority, the system by which X authorizes 
processes to connect to X servers, the condor keyboard daemon needs to run
with super user privaledges.  Currently, the daemon assumes that X uses the
HOME environement variable in order to locate a file named .Xauthority, which
contains keys necessary to connect to an X server.  The keyboard daemon
attempts to set this environment variable to various users home directories in
order to gain a connection to the X server and monitor events.  This may fail
to work on your system, if you are using a non-standard approach.  If the
keyboard daemon is not allowed to attach to the X server, the state of a
machine may be incorrectly set to idle when a user is, in fact, using the
machine.

	To see how well the keyboard daemon is working on your system, review
the log for the daemon and look for succesfull connections to the X
server.  If you see none, you may have a situation where the keyboard daemon
is unable to connect to your machines X server.  If this happends, please
send mail to condor-admin@cs.wisc.edu and let us know about your situation.

