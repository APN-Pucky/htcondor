%%%%%%%%%%%%%%%%%%%%%%%%%%%%%%%%%%%%%%%%%%%%%%%%%%%%%%%%%%%%%%%%%%%%%%
\subsection{Installing the \Condor{kbdd}}
\label{sec:kbdd}
%%%%%%%%%%%%%%%%%%%%%%%%%%%%%%%%%%%%%%%%%%%%%%%%%%%%%%%%%%%%%%%%%%%%%%

The condor keyboard daemon (\Condor{kbdd}) monitors X events on
machines where the operating system does not provide a way of
monitoring the idle time of the keyboard or mouse.  In particular,
this is necessary on Digital Unix machines and IRIX machines.  

\Note If you are running on Solaris, Linux, or HP/UX, you do
not need to use the keyboard daemon.

Although great measures have been taken to make this daemon as robust
as possible, the X window system was not designed to facilitate such a
need, and thus is less then optimal on machines where many users log
in and out on the console frequently.

In order to work with X authority, the system by which X authorizes
processes to connect to X servers, the condor keyboard daemon needs to
run with super user privileges.  Currently, the daemon assumes that X
uses the \Env{HOME} environment variable in order to locate a file
named \File{.Xauthority}, which contains keys necessary to connect to
an X server.  The keyboard daemon attempts to set this environment
variable to various users home directories in order to gain a
connection to the X server and monitor events.  This may fail to work
on your system, if you are using a non-standard approach.  If the
keyboard daemon is not allowed to attach to the X server, the state of
a machine may be incorrectly set to idle when a user is, in fact,
using the machine.

In some environments, the keyboard daemon will not be able to connect to the X
server because the user currently logged into the system keeps there
authentication token for using the X server in a place that no local user on
the current machine can get to.  This may be the case if you are running AFS
and have the user's X authority file in an AFS home directory.  There may also
be cases where you cannot run the daemon with super user privileges because of
political reasons, but you would still like to be able to monitor X activity.
In this case, you will need to change your XDM configuration in order to start
up the keyboard daemon with the permissions of the currently logging in user.
The specifics of this process are beyond the scope of the manual, however your
system administrator should understand how to go about implementing this.  The
important things to remember are to start the daemon with the -t and the -f
options ( -f means run the daemon in the foreground, and -t means output
the logging information to STDERR instead of the condor log files which users
don't have permission to write to. )  You will also want to make sure to
shutdown the program once the user logs out.  This can be done by saving the
process id of the daemon, and sending a
\Env{SIGTERM} to this process id when the user logs out.  

To see how well the keyboard daemon is working on your system, review
the log for the daemon and look for successful connections to the X
server.  If you see none, you may have a situation where the keyboard
daemon is unable to connect to your machines X server.  If this
happens, please send mail to \Email{condor-admin@cs.wisc.edu} and let
us know about your situation.

