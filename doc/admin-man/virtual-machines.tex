%%%%%%%%%%%%%%%%%%%%%%%%%%%%%%%%%%%%%%%%%%%%%%%%%%%%%%%%%%%%%%%%%%%%%%
\subsection{\label{sec:Virtual-Machines}
Running Condor Jobs within a Virtual Machine}
%%%%%%%%%%%%%%%%%%%%%%%%%%%%%%%%%%%%%%%%%%%%%%%%%%%%%%%%%%%%%%%%%%%%%%
\index{virtual machine!running Condor jobs under}

Condor jobs are formed from executables that are compiled to execute
on specific platforms.
This in turn restricts the machines within a Condor pool where
a job may be executed.
A Condor job may now be executed on a 
virtual machine system running VMware, Xen, or KVM.
This allows Windows executables to run on a Linux machine,
and Linux executables to run on a Windows machine.

In older versions of Condor, other parts of the system were also
referred to as \Term{virtual machines}, but in all cases, those are now
known as \Term{slots}.
A virtual machine here describes the environment in which
the outside operating system (called the host) emulates an inner operating
system (called the inner virtual machine),
such that an executable appears to run directly
on the inner virtual machine.
In other parts of Condor, a \Term{slot} (formerly known as
\Term{virtual machine}) refers to the multiple CPUs of an SMP
machine.
Also, be careful not to confuse the virtual machines discussed here
with the Java Virtual Machine (JVM) referenced in other parts of this
manual.

Condor has the flexibility to run a job on either the host
or the inner virtual machine, 
hence two platforms appear to exist on a single machine.
Since two platforms are an illusion, Condor understands the illusion, 
allowing a Condor job to be execute on only
one at a time.

%%%%%%%%%%%%%%%%%%%%%%%%%%%%%%%%%%%%%%%%%%%%%%%%%%%%%%%%%%%%%%%%%%%%%%
\subsubsection{\label{sec:Virtual-Machines-Configuration}
Installation and Configuration}
%%%%%%%%%%%%%%%%%%%%%%%%%%%%%%%%%%%%%%%%%%%%%%%%%%%%%%%%%%%%%%%%%%%%%%

\index{virtual machine!configuration}
Condor must be separately installed, separately configured,
and separately running on both
the host and the inner virtual machine.

The configuration for the host specifies \Macro{VMP\_VM\_LIST}.
This specifies host names or IP addresses of all inner virtual machines
running on this host.
An example configuration on the host machine:

\footnotesize
\begin{verbatim}
VMP_VM_LIST = vmware1.domain.com, vmware2.domain.com
\end{verbatim}
\normalsize


The configuration for each separate inner virtual machine specifies
\Macro{VMP\_HOST\_MACHINE}.
This specifies the host for the inner virtual machine.
An example configuration on an inner virtual machine:

\footnotesize
\begin{verbatim}
VMP_HOST_MACHINE = host.domain.com
\end{verbatim}
\normalsize

Given this configuration, as well as communication between
Condor daemons running on the host and on the inner virtual machine,
the policy for when jobs may execute is set by Condor.
While the host is executing a Condor job,
the \MacroNI{START} policy on the inner virtual machine
is overridden with \Expr{False},
so no Condor jobs will be started on the inner virtual machine.
Conversely, while the inner virtual machine is executing a Condor job,
the \MacroNI{START} policy on the host
is overridden with \Expr{False},
so no Condor jobs will be started on the host.

The inner virtual machine is further provided with a new syntax for
referring to the machine ClassAd attributes of its host.
Any machine ClassAd attribute with a prefix of the string
\MacroNI{HOST\_} explicitly refers to the host's ClassAd attributes.
The \MacroNI{START} policy on the inner virtual machine
ought to use this syntax to avoid starting jobs when its host is
too busy processing other items.
An example configuration for \MacroNI{START} on an inner virtual machine:

\footnotesize
\begin{verbatim}
START = ( (KeyboardIdle > 150 ) && ( HOST_KeyboardIdle > 150 ) \
        && ( LoadAvg <= 0.3 ) && ( HOST_TotalLoadAvg <= 0.3 ) )
\end{verbatim}
\normalsize

