
\section{\label{sec:DaemonCore}DaemonCore}
%%%%%%%%%%%%%%%%%%%%%%%%%%%%%%%%%%%%%%%%%%%%%%%%%%%%%%%%%%%%%%%%%%%%%%%%%%%

\index{daemoncore|(}
\index{Condor!shared functionality in daemons}
This section is a brief description of \Term{DaemonCore}.  DaemonCore
is a library that is shared among most of the Condor daemons which
provides common functionality.  Currently, the following daemons use
DaemonCore:

\begin{itemize}
\item \Condor{master}
\item \Condor{startd}
\item \Condor{schedd}
\item \Condor{collector}
\item \Condor{negotiator}
\item \Condor{kbdd}
\item \Condor{gridmanager}
\item \Condor{credd}
\item \Condor{had}
\item \Condor{replication}
\item \Condor{transferer}
\item \Condor{job\_router}
\item \Condor{lease\_manager}
\item \Condor{rooster}
\item \Condor{shared\_port}
\item \Condor{defrag}
\end{itemize}

Most of DaemonCore's details are not interesting for administrators.
However, DaemonCore does provide a uniform interface for the daemons
to various Unix signals, and provides a common set of command-line
options that can be used to start up each daemon.

%%%%%%%%%%%%%%%%%%%%%%%%%%%%%%%%%%%%%%%%%%%%%%%%%%%%%%%%%%%%%%%%%%%%%%%%%%%
\subsection{\label{sec:DaemonCore-Signals}DaemonCore and Unix signals}
%%%%%%%%%%%%%%%%%%%%%%%%%%%%%%%%%%%%%%%%%%%%%%%%%%%%%%%%%%%%%%%%%%%%%%%%%%%

\index{daemoncore!Unix signals}
One of the most visible features that DaemonCore provides for
administrators is that all daemons which use it behave the same way on
certain Unix signals.  The signals and the behavior DaemonCore
provides are listed below:

\begin{description}
\item[SIGHUP] Causes the daemon to reconfigure itself.
\item[SIGTERM] Causes the daemon to gracefully shutdown.
\item[SIGQUIT] Causes the daemon to quickly shutdown.
\end{description}

Exactly what gracefully and quickly means varies from daemon
to daemon.  For daemons with little or no state 
(the \Condor{kbdd}, \Condor{collector} and \Condor{negotiator})
there is no difference, and both \Code{SIGTERM} and \Code{SIGQUIT} signals
result in the daemon shutting itself down quickly.
For the \Condor{master},
a graceful shutdown causes the \Condor{master} to ask all of 
its children to perform their own graceful shutdown methods.
The quick shutdown causes the \Condor{master} to ask all of 
its children to perform their own quick shutdown methods.
In both cases, the \Condor{master} exits after all its children have exited.
In the \Condor{startd}, if the machine is not claimed and running a job, 
both the \Code{SIGTERM} and \Code{SIGQUIT} signals result in an immediate exit.
However, if the \Condor{startd} is running a job,
a graceful shutdown results in that job writing a checkpoint,
while a fast shutdown does not.
In the \Condor{schedd}, if there are no jobs currently running,
there will be no \Condor{shadow} processes,
and both signals result in an immediate exit.
However, with jobs running, a graceful shutdown causes
the \Condor{schedd} to ask each \Condor{shadow} to gracefully vacate
the job it is serving, 
while a quick shutdown results in a hard kill of every \Condor{shadow},
with no chance to write a checkpoint.  

For all daemons, a reconfigure results in the daemon re-reading
its configuration file(s), causing any settings that have changed
to take effect.
See section~\ref{sec:Configuring-Condor} on
page~\pageref{sec:Configuring-Condor}, Configuring Condor for full
details on what settings are in the configuration files and what they do.

%%%%%%%%%%%%%%%%%%%%%%%%%%%%%%%%%%%%%%%%%%%%%%%%%%%%%%%%%%%%%%%%%%%%%%%%%%%
\subsection{\label{sec:DaemonCore-Arguments}DaemonCore and
Command-line Arguments} 
%%%%%%%%%%%%%%%%%%%%%%%%%%%%%%%%%%%%%%%%%%%%%%%%%%%%%%%%%%%%%%%%%%%%%%%%%%%

\index{daemoncore!command line arguments}
\index{Condor daemon!command line arguments}
The second visible feature that DaemonCore provides to administrators
is a common set of command-line arguments that all daemons understand.
These arguments and what they do are described below:

\begin{description}

\item[-a string] Append a period character (\verb@'.'@) concatenated with
  \Opt{string} to the file name of the log for this daemon,
  as specified in the configuration file.

\item[-b] Causes the daemon to start up in the background.  When a
  DaemonCore process starts up with this option, it disassociates itself
  from the terminal and forks itself, so that it runs in the
  background.  This is the default behavior for Condor daemons.

\item[-c filename] Causes the daemon to use the specified \Opt{filename}
  as a full path and file name as its global configuration file.  This
  overrides the \Env{CONDOR\_CONFIG} environment variable and the
  regular locations that Condor checks for its configuration file.

\item[-d] Use dynamic directories.
 The \MacroUNI{LOG}, \MacroUNI{SPOOL}, and \MacroUNI{EXECUTE} directories
 are all created by the daemon at run time,
 and they are named by appending the
 parent's IP address and PID to the value in the 
 configuration file.
 These values are then inherited by all children of the daemon
 invoked with this \Opt{-d} argument.
 For the \Condor{master},
 all Condor processes will use the new directories.
 If a \Condor{schedd} is invoked with the \Arg{-d} argument,
 then only the \Condor{schedd} daemon and any
 \Condor{shadow} daemons it spawns will use the dynamic
 directories (named with the \Condor{schedd} daemon's PID).

 Note that by using a dynamically-created spool directory
 named by the IP address and PID,
 upon restarting daemons,
 jobs submitted to the original \Condor{schedd} daemon
 that were stored in the old spool directory will not be noticed
 by the new \Condor{schedd} daemon,
 unless you manually specify the old, dynamically-generated 
 \MacroNI{SPOOL} directory path in the configuration of the
 new \Condor{schedd} daemon.

\item[-f] Causes the daemon to start up in the foreground.  Instead of
  forking, the daemon runs in the foreground.  

  \Note When the \Condor{master} starts up daemons, it does
  so with the \Opt{-f} option, as it has already forked a process for the
  new daemon.  There will be a \Opt{-f} in the argument list for all
  Condor daemons that the \Condor{master} spawns.

\item[-k filename] For non-Windows operating systems,
  causes the daemon to read out a PID from the
  specified \Opt{filename}, and send a SIGTERM to that process.
  The daemon started with this optional argument waits until the
  daemon it is attempting to kill has exited.  

\item[-l directory] Overrides the value of \Macro{LOG} as specified in
  the configuration files.  Primarily, this option is used with the
  \Condor{kbdd} when it needs to run as the individual user logged
  into the machine, instead of running as root.  Regular users would
  not normally have permission to write files into Condor's log
  directory.  Using this option, they can override the value of
  \MacroNI{LOG} and have the \Condor{kbdd} write its log file into a
  directory that the user has permission to write to.

\item[-local-name name] Specify a local name for this instance of
  the daemon.  This local name will be used to look up
  configuration parameters. 
  Section~\ref{sec:Config-File-Macros} contains
  details on how this local name will be used in the configuration.

\item[-p port] Causes the daemon to bind to the specified port as its
  command socket.  The \Condor{master} daemon
  uses this option to ensure that the
  \Condor{collector} and \Condor{negotiator} start up using
  well-known ports that the rest of Condor depends upon them using.

\item[-pidfile filename] Causes the daemon to write out its PID
  (process id number) to the specified \Opt{filename}.
  This file can be used to
  help shutdown the daemon without first searching through 
  the output of the Unix \Prog{ps} command.

  Since daemons run with their current working directory set to the
  value of \MacroNI{LOG}, if you don't specify a full path
  (one that begins with a ``/''),
  the file will be placed in the \MacroNI{LOG} directory.

\item[-q] Quiet output; write less verbose error
 messages to \File{stderr} when something goes wrong,
 and before regular logging can be initialized.

\item[-r minutes] Causes the daemon to set a timer, upon expiration
  of which, it sends itself a SIGTERM for graceful shutdown.

\item[-t] Causes the daemon to print out its error message to
  \File{stderr} instead of its specified log file.  This option forces
  the \Opt{-f} option.

\item[-v] Causes the daemon to print out version information and
  exit.

\end{description}

\index{daemoncore|)}
