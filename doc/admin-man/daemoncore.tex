%%%%%%%%%%%%%%%%%%%%%%%%%%%%%%%%%%%%%%%%%%%%%%%%%%%%%%%%%%%%%%%%%%%%%%%%%%%
\section{\label{sec:DaemonCore}DaemonCore}
%%%%%%%%%%%%%%%%%%%%%%%%%%%%%%%%%%%%%%%%%%%%%%%%%%%%%%%%%%%%%%%%%%%%%%%%%%%

This section is a brief description of \Term{DaemonCore}.  DaemonCore
is a library that is shared among most of the Condor daemons which
provides common functionality.  Currently, the following daemons use
DaemonCore:

\begin{itemize}
\item \Condor{master}
\item \Condor{startd}
\item \Condor{schedd}
\item \Condor{collector}
\item \Condor{negotiator}
\item \Condor{kbdd}
\end{itemize}

Most of DaemonCore's details are not interesting for administrators.
However, DaemonCore does provide a uniform interface for the daemons
to various UNIX signals, and provides a common set of command-line
options that can be used to start up each daemon.

%%%%%%%%%%%%%%%%%%%%%%%%%%%%%%%%%%%%%%%%%%%%%%%%%%%%%%%%%%%%%%%%%%%%%%%%%%%
\subsection{\label{sec:DaemonCore-Signals}DaemonCore and UNIX signals}
%%%%%%%%%%%%%%%%%%%%%%%%%%%%%%%%%%%%%%%%%%%%%%%%%%%%%%%%%%%%%%%%%%%%%%%%%%%

One of the most visible features DaemonCore provides for
administrators is that all daemons which use it behave the same way on
certain UNIX signals.  The signals and the behavior DaemonCore
provides are listed below:

\begin{description}
\item[SIGHUP] Causes the daemon to reconfigure itself.
\item[SIGTERM] Causes the daemon to gracefully shutdown.
\item[SIGQUIT] Causes the daemon to quickly shutdown.
\end{description}

Exactly what ``gracefully'' and ``quickly'' means varies from daemon
to daemon.  For daemons with little or no state (the kbdd, collector and
negotiator) there's no difference and both signals result in the
daemon shutting itself down basically right away.  For the master,
graceful shutdown just means it asks all of its children to perform
their own graceful shutdown methods, while fast shutdown means it asks
its children to perform their own fast shutdown methods.  In both
cases, the master only exits once all its children have exited.  In
the startd, if the machine is not claimed and running a job, both
result in an immediate exit.  However, if the startd is running a job,
graceful shutdown results in that job being checkpointed, while fast
shutdown does not.  In the schedd, if there are no jobs currently
running (i.e. no \Condor{shadow} processes), both signals result in an
immediate exit.  With jobs running, however, graceful shutdown means
that the schedd asks each shadow to gracefully vacate whatever job it
is serving, while fast shutdown results in a hard kill of every shadow
with no chance of checkpointing.  

For all daemons, ``reconfigure'' just means that the daemon re-reads
its config file(s) and any settings that have changed take effect.
For example, changing the level of debugging output, the value of
timers that determine how often daemons perform certain actions, the
paths to the binaries you want the \Condor{master} to spawn, etc.
See section~\ref{sec:Configuring-Condor} on
page~\pageref{sec:Configuring-Condor}, ``Configuring Condor'' for full
details on what settings are in the config files and what they do.

%%%%%%%%%%%%%%%%%%%%%%%%%%%%%%%%%%%%%%%%%%%%%%%%%%%%%%%%%%%%%%%%%%%%%%%%%%%
\subsection{\label{sec:DaemonCore-Arguments}DaemonCore and
Command-line Arguments} 
%%%%%%%%%%%%%%%%%%%%%%%%%%%%%%%%%%%%%%%%%%%%%%%%%%%%%%%%%%%%%%%%%%%%%%%%%%%

The other visible feature that DaemonCore provides to administrators
is a common set of command-line arguments that all daemons understand.
The arguments and what they do are described below:

\begin{description}

\item[-b] Causes the daemon to start up in the background.  When a
  DaemonCore process starts up with this option, disassociates itself
  from the terminal and forks itself so that it runs in the
  background.  This is the default behavior for Condor daemons, and
  what you get if you specify no options at all.

\item[-f] Causes the daemon to start up in the foreground.  Instead of
  forking, the daemon just runs in the foreground.  

  \Note when the \Condor{master} starts up daemons, it does
  so with the -f option since it has already forked a process for the
  new daemon.  That is why you will see -f in the argument list of all
  Condor daemons that the master spawns.

\item[-c filename] Causes the daemon to use the specified filename
  (you must use a full path) as its global config file.  This
  overrides the \Env{CONDOR\_CONFIG} environment variable, and the
  regular locations that Condor checks for its config file: the condor
  user's home directory and \File{/etc/condor/\condor{config}}.  

\item[-p port] Causes the daemon to bind to the specified port for its
  \Term{command socket}.  The master uses this option to make sure the
  \Condor{collector} and \Condor{negotiator} start up on the
  well-known ports that the rest of Condor depends on them using.

\item[-t] Causes the daemon to print out its error message to
  \File{stderr} instead of its specified log file.  This option forces
  the -f option described above.

\item[-v] Causes the daemon to print out version information and
  exit.

\item[-l directory] Overrides the value of \Macro{LOG} as specified in
  your config files.  Primarily, this option would be used with the
  \Condor{kbdd} when it needs to run as the individual user logged
  into the machine, instead of running as root.  Regular users would
  not normally have permission to write files into Condor's log
  directory.  Using this option, they can override the value of
  \Macro{LOG} and have the \Condor{kbdd} write its log file into a
  directory that the user has permission to write to.

\item[-a string] Whatever string you specify is automatically
  appended (with a ``.'') to the filename of the log for this daemon,
  as specified in your config file.

\item[-pidfile filename] Causes the daemon to write out its PID, or
  process id number, to the specified file.  This file can be used to
  help shutdown the daemon without searching through the output of the
  ``ps'' command.

  Since daemons run with their current working directory set to the
  value of \Macro{LOG}, if you don't specify a full path (with a ``/''
  to begin), the file will be left in the log directory.  If you leave
  your pidfile in your log directory, you will want to add whatever
  filename you use to the \Macro{VALID\_LOG\_FILES} parameter,
  described in section~\ref{param:ValidLogFiles} on
  page~\pageref{param:ValidLogFiles}, so that \Condor{preen} does not
  remove it.

\item[-k filename] Causes the daemon to read out a pid from the
  specified filename and send a SIGTERM to that process.  The daemon
  that you start up with ``-k'' will wait until the daemon it is
  trying to kill has exited.  

\item[-r minutes] Causes the daemon to set a timer, upon expiration
  of which, sends itself a SIGTERM for graceful shutdown.

\end{description}

