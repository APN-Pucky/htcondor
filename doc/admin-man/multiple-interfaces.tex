%%%%%%%%%%%%%%%%%%%%%%%%%%%%%%%%%%%%%%%%%%%%%%%%%%%%%%%%%%%%%%%%%%%%%%%%%%%
\subsection{\label{sec:Multiple-Interfaces}Configuring Condor for
Machines With Multiple Network Interfaces } 
%%%%%%%%%%%%%%%%%%%%%%%%%%%%%%%%%%%%%%%%%%%%%%%%%%%%%%%%%%%%%%%%%%%%%%%%%%

Beginning with Condor version 6.1.5, Condor can run on machines with
multiple network interfaces.
Basically, you tell each host with multiple interfaces which ip
address you want the host to use for ingoing and outgoing Condor
network communication.
You do this by setting the \Macro{NETWORK\_INTERFACE} parameter in
the local config file for each host you need to.
There are a few other special cases you might have to deal with,
described below.

If your Central Manager is on a machine with multiple interfaces,
instead of defining the \Macro{COLLECTOR\_HOST} or
\Macro{NEGOTIATOR\_HOST} parameters (which are usually both defined in
terms of \Macro{CONDOR\_HOST}, you should set the \Macro{CM_IP_ADDR}. 

If your Checkpoint Server is on a machine with multiple interfaces,
the only way to get things to work is if your different interfaces
have different hostnames associated with them, and you set
\Macro{CKPT\_SERVER\_HOST} to the hostname that corresponds with the
ip address you want to use.  
You will still need to specify \Macro{NETWORK\_INTERFACE} in the local
config file for your Checkpoint Server.
