%%%%%%%%%%%%%%%%%%%%%%%%%%%%%%%%%%%%%%%%%%%%%%%%%%%%%%%%%%%%%%%%%%%%%%%%%%%
\subsection{\label{sec:Multiple-Interfaces}Configuring Condor for
Machines With Multiple Network Interfaces } 
%%%%%%%%%%%%%%%%%%%%%%%%%%%%%%%%%%%%%%%%%%%%%%%%%%%%%%%%%%%%%%%%%%%%%%%%%%

% CMNT I'm going to leave the original documentation here in case someone
% CMNT needs the CM_IP_ADDR stuff better documented. For now, though, I've
% CMNT removed it from the documentation because the code should deal with
% CMNT CONDOR_HOST being a straight ip_addr just fine. The new docs I've
% CMNT written don't even reference CM_IP_ADDR because it seems the code
% CMNT doesn't need it.

%Beginning with Condor version 6.1.5, Condor can run on machines with
%multiple network interfaces.
%Basically, you tell each host with multiple interfaces which IP
%address you want the host to use for ingoing and outgoing Condor
%network communication.
%You do this by setting the \Macro{NETWORK\_INTERFACE} parameter in
%the local config file for each host you need to.
%There are a few other special cases you might have to deal with,
%described below.
%
%If your Central Manager is on a machine with multiple interfaces,
%instead of defining the \Macro{COLLECTOR\_HOST} or
%\Macro{NEGOTIATOR\_HOST} parameters (which are usually both defined in
%terms of \Macro{CONDOR\_HOST}), you should set the
%\Macro{CM\_IP\_ADDR}.
%
%\Warn The default \Macro{HOSTALLOW\_ADMINISTRATOR} setting in the
%config file references \MacroU{CONDOR\_HOST}, and the default
%\Macro{HOSTALLOW\_NEGOTIATOR} setting references
%\MacroU{NEGOTIATOR\_HOST}.
%So you'll need to change both of these settings to reference
%\MacroU{CM\_IP\_ADDR} instead.   
%
%If your Checkpoint Server is on a machine with multiple interfaces,
%the only way to get things to work is if your different interfaces
%have different hostnames associated with them, and you set
%\Macro{CKPT\_SERVER\_HOST} to the hostname that corresponds with the
%IP address you want to use.  
%You will still need to specify \Macro{NETWORK\_INTERFACE} in the local
%config file for your Checkpoint Server.
%
%XXX

Beginning with Condor version 6.1.5, Condor can run on machines with
multiple network interfaces. Here are some common scenarios that you might
encounter and how you go about solving them.

\subsubsection{Central Manager with Two or More NICs}

Often users of Condor wish to set up ``compute farms'' where there is one
machine with two network interface cards(one for the public internet,
and one for the private net). It is convenient to set up the ``head''
node as a central manager in most cases and so here are the instructions
required to do so.

Setting up the central manager on a machine with more than one NIC can
be a little confusing because there are a few external variables
that could make the process difficult. One of the biggest mistakes
in getting this to work is that either one of the separate interfaces is
not active, or the host/domain names associated with the interfaces are
incorrectly configured. 

Given that the interfaces are up and functioning, and they have good
host/domain names associated with them here is how to configure Condor:

In this example, \Bold{farm-server.farm.org} maps to the private interface.

On your central manager's global(to the cluster) config file: \\
\Macro{CONDOR\_HOST} = \Bold{farm-server.farm.org}

On your central manager's local configuration file: \\
\MacroNI{NETWORK\_INTERFACE} = ip address of \Bold{farm-server.farm.org} \\
\MacroNI{NEGOTIATOR} = \MacroUNI{SBIN}/condor\_negotiator \\
\MacroNI{COLLECTOR} = \MacroUNI{SBIN}/condor\_collector \\
\MacroNI{DAEMON\_LIST} = \MacroNI{MASTER}, \MacroNI{COLLECTOR}, \MacroNI{NEGOTIATOR}, \MacroNI{SCHEDD}, \MacroNI{STARTD}

If your central manager and farm machines are all NT, then you only have
vanilla universe and it will work now.  However, if you have this setup
for UNIX, then at this point, standard universe jobs should be able to
function in the pool, but if you did not configure the \Macro{UID\_DOMAIN}
macro to be homogeneous across the farm machines, the standard universe
jobs will run as \Bold{nobody} on the farm machines.

In order to get vanilla jobs and file server load balancing for standard
universe jobs working(under unix), you need to do some more work both in
the cluster you have put together and in Condor to make everything work.
First, you need a file server(which could also be the central manager) to
serve files to all of the farm machines. This could be NFS or AFS, it does
not really matter to Condor. The mount point of the directories you wish
your users to use must be the same across all of the farm machines. Now,
configure \Macro{UID\_DOMAIN} and \Macro{FILESYSTEM\_DOMAIN} to be
homogeneous across the farm machines and the central manager. Now, you
will have to inform Condor that an NFS or AFS filesystem exists and that
is done in this manner. In the global(to the farm) configuration file:

\begin{verbatim}
# If you have NFS
USE_NFS = True
# If you have AFS
HAS_AFS = True
USE_AFS = True
# if you want both NFS and AFS, then enable both sets above
\end{verbatim}

Now, if you've set up your cluster so that it is possible for a machine
name to never have a domain name(for example: you've placed a machine
name but no fully qualified domain name in \File{/etc/hosts}), you must
configure \Macro{DEFAULT\_DOMAIN\_NAME} to be the domain that you wish
to be added on to the end of your hostname.

\subsubsection{A Client Machine with Multiple Interfaces}

If you have a client machine with two or more NICs, then there might be
a specific network interface with which you desire a client machine to
communicate with the rest of the Condor pool. In this case, in the local
configuration file for that machine, place: \\ 
\Macro{NETWORK\_INTERFACE} = ip address of interface desired \\

\subsubsection{A Checkpoint Server on a Machine with Multiple NICs}

If your Checkpoint Server is on a machine with multiple interfaces,
the only way to get things to work is if your different interfaces
have different hostnames associated with them, and you set
\Macro{CKPT\_SERVER\_HOST} to the hostname that corresponds with the
IP address you want to use in the global configuration file for your pool.
You will still need to specify \Macro{NETWORK\_INTERFACE} in the local
config file for your Checkpoint Server.

