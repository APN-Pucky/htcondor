%%%%%%%%%%%%%%%%%%%%%%%%%%%%%%%%%%%%%%%%%%%%%%%%%%%%%%%%%%%%%%%%%%%%%%%%%%%
\subsection{\label{sec:Multiple-Interfaces}Configuring HTCondor for
Machines With Multiple Network Interfaces } 
%%%%%%%%%%%%%%%%%%%%%%%%%%%%%%%%%%%%%%%%%%%%%%%%%%%%%%%%%%%%%%%%%%%%%%%%%%

\index{multiple network interfaces}
\index{network interfaces!multiple}
\index{NICs}

HTCondor can run on machines with
multiple network interfaces.
Starting with HTCondor version 6.7.13
(and therefore all HTCondor 6.8 and more recent versions),
new functionality is
available that allows even better support for multi-homed
machines, using the configuration variable \Macro{BIND\_ALL\_INTERFACES}.
A multi-homed machine is one that has more than one
NIC (Network Interface Card).
Further improvements to this new functionality will remove the need
for any special configuration in the common case.
For now, care
must still be given to machines with multiple NICs, even
when using this new configuration variable.


%%%%%%%%%%%%%%%%%%%%%%%%%%%%%%%%%%%%%%%%%%%%%%%%%%%%%%%%%%%%
\subsubsection{\label{sec:Using-BindAllInterfaces}Using 
BIND\_ALL\_INTERFACES}
%%%%%%%%%%%%%%%%%%%%%%%%%%%%%%%%%%%%%%%%%%%%%%%%%%%%%%%%%%%%

Machines can be configured such that
whenever HTCondor daemons or tools
call \Syscall{bind}, the daemons or tools use all network interfaces on
the machine.
This means that outbound connections will always use the appropriate
network interface to connect to a remote host,
instead of being forced to use
an interface that might not have a route to the given destination.
Furthermore, sockets upon which a daemon listens for incoming connections 
will be bound to all network interfaces on the machine.
This means that so long as remote clients know the right port, they can
use any IP address on the machine and still contact a given HTCondor daemon.

This functionality is on by default.  To disable this functionality, 
the boolean configuration
variable
\MacroNI{BIND\_ALL\_INTERFACES}
is defined and set to \Expr{False}:

\begin{verbatim}
BIND_ALL_INTERFACES = FALSE
\end{verbatim}

This functionality has limitations.
Here are descriptions of the limitations.

\begin{description}

\item[Using all network interfaces does not work with Kerberos.] 
  Every Kerberos ticket contains a specific IP address within it.
  Authentication over a socket (using Kerberos) requires
  the socket to also specify that same specific IP address.
  Use of \MacroNI{BIND\_ALL\_INTERFACES} causes outbound
  connections from a multi-homed machine to 
  originate over any of the interfaces.
  Therefore, the IP address of the outbound connection and the IP
  address in the Kerberos ticket will not necessarily match,
  causing the authentication to fail.
  Sites using Kerberos authentication on multi-homed machines are
  strongly encouraged not to enable \MacroNI{BIND\_ALL\_INTERFACES},
  at least until HTCondor's Kerberos functionality
  supports using multiple Kerberos tickets together with finding the right one
  to match the IP address a given socket is bound to. 

\item[There is a potential security risk.]
  Consider the following example of a security risk.
  A multi-homed machine is at a network boundary.
  One interface is on the public Internet, while the other connects to
  a private network.
  Both the multi-homed machine and the private network machines
  comprise an HTCondor pool.
  If the multi-homed machine enables \MacroNI{BIND\_ALL\_INTERFACES},
  then it is at risk from hackers trying to compromise the security of the pool.
  Should this multi-homed machine be compromised,
  the entire pool is vulnerable.
  Most sites in this situation would run an \Prog{sshd} on the
  multi-homed machine so that remote users who wanted to access the
  pool could log in securely and use the HTCondor tools directly.
  In this case, remote clients do not need to use HTCondor tools running
  on machines in the public network to access the HTCondor daemons on
  the multi-homed machine.
  Therefore, there is no reason to have HTCondor daemons listening on
  ports on the public Internet, causing a potential security threat.

\item[Up to two IP addresses will be advertised.]
  At present, even though a given HTCondor daemon will be listening to
  ports on multiple interfaces, each with their own IP address,
  there is currently no mechanism for that daemon to advertise all of
  the possible IP addresses where it can be contacted.
  Therefore, HTCondor clients (other HTCondor daemons or tools) will not
  necessarily able to locate and communicate with a given daemon
  running on a multi-homed machine where
  \MacroNI{BIND\_ALL\_INTERFACES} has been enabled.

  Currently, HTCondor daemons can only advertise two IP addresses in
  the ClassAd they send to their \Condor{collector}.  One is the
  public IP address and the other is the private IP address.
  HTCondor tools and other daemons that wish to connect to the daemon will
  use the private IP address if they are configured with the same private
  network name, and they will use the public IP address otherwise.
  So, even if the daemon is listening on 3 or more different interfaces,
  each with a separate IP, the daemon must choose which two IP addresses to
  advertise so that other daemons and tools can connect to it.

  By default, HTCondor advertises the most public IP address available
  on the machine.
  The \Macro{NETWORK\_INTERFACE} configuration variable can be used 
  to specify the public IP address HTCondor should advertise, and
  \Macro{PRIVATE\_NETWORK\_INTERFACE}, along with
  \Macro{PRIVATE\_NETWORK\_NAME} can be used to specify the
  private IP address to advertise.

\end{description}

Sites that make heavy use of private networks and multi-homed machines
should consider if using the HTCondor Connection Broker, CCB,
is right for them.
More information about CCB and HTCondor can be found in
section~\ref{sec:CCB} on page~\pageref{sec:CCB}.


%%%%%%%%%%%%%%%%%%%%%%%%%%%%%%%%%%%%%%%%%%%%%%%%%%%%%%%%%%%%
\subsubsection{Central Manager with Two or More NICs}
%%%%%%%%%%%%%%%%%%%%%%%%%%%%%%%%%%%%%%%%%%%%%%%%%%%%%%%%%%%%

Often users of HTCondor wish to set up compute farms where there is one
machine with two network interface cards (one for the public Internet,
and one for the private net). It is convenient to set up the head
node as a central manager in most cases and so here are the instructions
required to do so.

Setting up the central manager on a machine with more than one NIC can
be a little confusing because there are a few external variables
that could make the process difficult. One of the biggest mistakes
in getting this to work is that either one of the separate interfaces is
not active, or the host/domain names associated with the interfaces are
incorrectly configured. 

Given that the interfaces are up and functioning, and they have good
host/domain names associated with them here is how to configure HTCondor:

In this example, \Expr{farm-server.farm.org} maps to the private interface.
In the central manager's global (to the cluster) configuration file:
\begin{verbatim}
CONDOR_HOST = farm-server.farm.org
\end{verbatim}

In the central manager's local configuration file:
\begin{verbatim}
NETWORK_INTERFACE = <IP address of farm-server.farm.org>
NEGOTIATOR = $(SBIN)/condor_negotiator
COLLECTOR = $(SBIN)/condor_collector
DAEMON_LIST = MASTER, COLLECTOR, NEGOTIATOR, SCHEDD, STARTD
\end{verbatim}

If the central manager and farm machines are all NT, then only
vanilla universe will work now.  However, if this is set up
for Unix, then at this point, standard universe jobs should be able to
function in the pool.
But, if \Macro{UID\_DOMAIN} is not configured
to be homogeneous across the farm machines, the standard universe
jobs will run as \Login{nobody} on the farm machines.

In order to get vanilla jobs and file server load balancing for standard
universe jobs working (under Unix), do some more work both in
the cluster you have put together and in HTCondor to make everything work.
First, you need a file server (which could also be the central manager) to
serve files to all of the farm machines. This could be NFS or AFS, 
and it does not really matter to HTCondor. 
The mount point of the directories you wish
your users to use must be the same across all of the farm machines. Now,
configure \Macro{UID\_DOMAIN} and \Macro{FILESYSTEM\_DOMAIN} to be
homogeneous across the farm machines and the central manager. 
Inform HTCondor that an NFS or AFS file system exists and that
is done in this manner. In the global (to the farm) configuration file:

\begin{verbatim}
# If you have NFS
USE_NFS = True
# If you have AFS
HAS_AFS = True
USE_AFS = True
# if you want both NFS and AFS, then enable both sets above
\end{verbatim}

Now, if the cluster is set up so that it is possible for a machine
name to never have a domain name
(for example, there is machine
name but no fully qualified domain name in \File{/etc/hosts}),
configure \Macro{DEFAULT\_DOMAIN\_NAME} to be the domain that is
to be added on to the end of the host name.


%%%%%%%%%%%%%%%%%%%%%%%%%%%%%%%%%%%%%%%%%%%%%%%%%%%%%%%%%%%%
\subsubsection{A Client Machine with Multiple Interfaces}
%%%%%%%%%%%%%%%%%%%%%%%%%%%%%%%%%%%%%%%%%%%%%%%%%%%%%%%%%%%%

If client machine has two or more NICs, then there might be
a specific network interface on which the client machine desires to
communicate with the rest of the HTCondor pool. 
In this case, the local configuration file for the client should have
\begin{verbatim}
  NETWORK_INTERFACE = <IP address of desired interface>
\end{verbatim}

%%%%%%%%%%%%%%%%%%%%%%%%%%%%%%%%%%%%%%%%%%%%%%%%%%%%%%%%%%%%
\subsubsection{A Checkpoint Server on a Machine with Multiple NICs}
%%%%%%%%%%%%%%%%%%%%%%%%%%%%%%%%%%%%%%%%%%%%%%%%%%%%%%%%%%%%

If a checkpoint server is on a machine with multiple interfaces,
then 2 items must be correct to get things to work:
\begin{enumerate}
\item
The different interfaces have different host names associated with them.
\item
In the global configuration file,
set configuration variable \Macro{CKPT\_SERVER\_HOST} to the host name
that corresponds with the IP address desired for the pool.
Configuration variable \Macro{NETWORK\_INTERFACE} must still be specified
in the local configuration file for the checkpoint server.
\end{enumerate}

