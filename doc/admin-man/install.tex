%%%%%%%%%%%%%%%%%%%%%%%%%%%%%%%%%%%%%%%%%%%%%%%%%%%%%%%%%%%%%%%%%%%%%%
\section{\label{sec:install}Installation, Start Up, Shut Down, and Reconfiguration}
%%%%%%%%%%%%%%%%%%%%%%%%%%%%%%%%%%%%%%%%%%%%%%%%%%%%%%%%%%%%%%%%%%%%%%

This section contains the instructions for installing HTCondor.
The installation will have a default configuration that can
be customized.
Sections of the manual below explain customization.

Please read this \emph{entire} section before starting installation.

Please read the copyright and disclaimer information in
section~\ref{sec:license}.
Installation and
use of HTCondor is acknowledgment that you have read and agree to the
terms.

Before installing HTCondor, please consider joining the htcondor-world mailing
list.
Traffic on this list is kept to an absolute minimum;
it is only used to announce new releases of HTCondor.
To subscribe, go to
\URL{https://lists.cs.wisc.edu/mailman/listinfo/htcondor-world},
and fill out the online form.

You might also want to consider joining the htcondor-users mailing list.
This list is meant to be a forum for HTCondor users to learn from each
other and discuss using HTCondor. It is an excellent place to ask
the HTCondor community about using and configuring HTCondor.
To subscribe, go to
\URL{https://lists.cs.wisc.edu/mailman/listinfo/htcondor-users},
and fill out the online form.

\Bold{Note that forward and reverse DNS lookup must be enabled
for HTCondor to work properly.}

%%%%%%%%%%%%%%%%%%%%%%%%%%%%%%%%%%%%%%%%%%%%%%%%%%%%%%%%%%%%%%%%%%%%%%
\subsection{\label{sec:pre-install-procedure}
Obtaining the HTCondor Software}
%%%%%%%%%%%%%%%%%%%%%%%%%%%%%%%%%%%%%%%%%%%%%%%%%%%%%%%%%%%%%%%%%%%%%%
\index{installation!download}
\index{Unix installation!download}
\index{download}
The first step to installing HTCondor is to download it from the HTCondor
web site, \URL{http://htcondor.org/}.
The downloads are available from the downloads page,
at \URL{http://htcondor.org/downloads/}.

%%%%%%%%%%%%%%%%%%%%%%%%%%%%%%%%%%%%%%%%%%%%%%%%%%%%%%%%%%%%%%%%%%%%%%
\subsection{\label{sec:Unix-Install}
Installation on Unix}
%%%%%%%%%%%%%%%%%%%%%%%%%%%%%%%%%%%%%%%%%%%%%%%%%%%%%%%%%%%%%%%%%%%%%%

The HTCondor binary distribution is packaged in the following files
and directories:

\begin{description}
\item[\File{LICENSE-2.0.txt}] the licensing agreement.
                  By installing HTCondor, you agree to the contents of
		  this file
\item[\File{README}] general information
\item[\File{bin}] directory which contains the distribution HTCondor
		  user programs.
\item[\File{bosco\_install}] the Perl script used to install Bosco.
\item[\File{condor\_configure}] the Perl script used to install and
                  configure HTCondor.
\item[\File{condor\_install}] the Perl script used to install HTCondor.
\item[\File{etc}] directory which contains the distribution HTCondor
		  configuration data.
\item[\File{examples}] directory containing C, Fortran and C++ example
		  programs to run with HTCondor.
\item[\File{include}] directory containing HTCondor header files.
\item[\File{lib}] directory which contains the distribution HTCondor
		  libraries.
\item[\File{libexec}] directory which contains the distribution HTCondor
		  auxiliary programs for use internally by HTCondor.
\item[\File{man}] directory which contains the distribution HTCondor
		  manual pages.
\item[\File{sbin}] directory containing HTCondor daemon binaries and admin
		  tools.
\item[\File{src}] directory containing source for some interfaces.
\end{description}

%%%%%%%%%%%%%%%%%%%%%%%%%%%%%%%%%%%%%%%%%%%%%%%%%%%%%%%%%%%%%%%%%%%%%%
\subsubsection{\label{sec:Preparing-to-Install}Preparation} 

Before installation, you need to make a few important
decisions about the basic layout of your pool.
These decisions answer the following questions:

\begin{enumerate}
\item What machine will be the central manager?
\item What machines should be allowed to submit jobs?
\item Will HTCondor run as root or not?
\item Who will be administering HTCondor on the machines in your pool?
\item Will you have a Unix user named condor and will its home directory be
   shared? 
\item Where should the machine-specific directories for HTCondor go?
\item Where should the parts of the HTCondor system be installed? 
	\begin{itemize}
	\item Configuration files
	\item Release directory
		\begin{itemize}
		\item user binaries
		\item system binaries 
		\item \File{lib} directory
	  	\item \File{etc} directory
		\end{itemize}
	\item Documentation
	\end{itemize}
\item Am I using AFS?
\item Do I have enough disk space for HTCondor?
\end{enumerate}

% Note: I guess this is a description rather than an enumerate because
% enumerate isn't bold.  Need to make sure to keep numbers correct!
% wenger 2017-01-12
\begin{description}

\item[1. What machine will be the central manager?]

One machine in your pool must be the central manager.
\index{central manager!installation issues}
Install HTCondor on this machine first.
This is the centralized information repository for the HTCondor pool,
and it is also the
machine that does match-making between available machines and
submitted jobs.
If the central manager machine crashes, any currently active
matches in the system will keep running, but no new matches will be
made.  Moreover, most HTCondor tools will stop working.  Because of the
importance of this machine for the proper functioning of HTCondor,
install the central manager on a machine that is likely to stay up all the
time, or on one that will be rebooted quickly if it does crash.

Also consider
network traffic and your network layout when choosing your central
manager.
All the daemons send updates (by default, every 5 minutes) to this machine.
Memory requirements for the central manager differ by the number of machines
in the pool:
a pool with up to about 100 machines will require approximately
25 Mbytes of memory for the central manager's tasks,
and a pool with about 1000 machines will require approximately
100 Mbytes of memory for the central manager's tasks.

A faster CPU will speed up matchmaking. 

Generally jobs should not be either submitted or run on the central
manager machine.

\item[2. Which machines should be allowed to submit jobs?]

HTCondor can restrict the machines allowed to submit jobs.  Alternatively, 
it can allow any machine the network allows to connect to a submit machine
to submit jobs.  If the HTCondor pool is behind a firewall, and all machines
inside the firewall are trusted, the \Macro{ALLOW\_WRITE} configuration
entry can be set to */*.  Otherwise, it should be set to reflect
the set of machines permitted to submit jobs to this pool.
HTCondor tries to be secure by default: it is shipped with an invalid value
that allows no machine to connect and submit jobs.

\item[3. Will HTCondor run as root or not?]

\index{installation!running as root}
We strongly recommend that the HTCondor daemons be installed and run
as the Unix user root.
Without this,
HTCondor can do very little to enforce security and policy
decisions.
You can install HTCondor as any user;
however there are serious security and performance consequences
do doing a non-root installation.
Please see section~\ref{sec:Non-Root}
in the manual for the details and ramifications of
installing and running HTCondor as a Unix user other than root.

\item[4. Who will administer HTCondor?]

\index{HTCondor!Unix administrator}
\index{Unix administrator}
\index{Unix user!root}

% administrator is a person, not an account
% responsibilities of the administrator needed here
% 1. to install
% 2. to customize policies
% 3. to receive e-mail

%TEMP -- I'd like to re-word this...
Either root will be administering HTCondor directly, or someone else
will be acting as the HTCondor administrator.  If root has delegated
the responsibility to another person, keep in mind that as long as
HTCondor is started up as root, it should be clearly understood that
whoever has the ability to edit the condor configuration files can
effectively run arbitrary programs as root.

The HTCondor administrator will be regularly updating HTCondor by following
these instructions or by using the system-specific installation methods below.
The administrator
will also customize policies of the HTCondor submit and execute nodes.
This person will also receive information from HTCondor if 
something goes wrong with the pool, 
as described in the documentation of the \Macro{CONDOR\_ADMIN}
configuration variable.

\item[5. Will you have a Unix user named condor, and will its home
directory be shared?]

\index{Unix user!condor}
To simplify installation of HTCondor, you should
create a Unix user named condor on all machines in the pool.
The HTCondor daemons will create files
(such as the log files) owned by this user,
and the home directory can be used to specify the location of files
and directories needed by HTCondor.  The home directory of this user can
either be shared among all machines in your pool, or could be a
separate home directory on the local partition of each machine.  Both
approaches have advantages and disadvantages.  Having the directories
centralized can make administration easier, but also concentrates the
resource usage such that you potentially need a lot of space for a
single shared home directory.  See the section below on
machine-specific directories for more details.

Note that the user condor must not be an account into which a person
can log in.
If a person can log in as user condor, 
it permits a major security breach,
in that the user condor could submit jobs that run as any other user,
providing complete access to the user's data by the jobs. 
A standard way of not allowing log in to an account on Unix platforms
is to enter an invalid shell in the password file.

If you choose not to create a user named condor,
then you must specify either via the
\index{CONDOR\_IDS environment variable}
\index{environment variables!CONDOR\_IDS}
\Env{CONDOR\_IDS} environment variable or the \Macro{CONDOR\_IDS}
config file setting which uid.gid pair should be used for
the ownership of various HTCondor files.  
See section~\ref{sec:uids} on UIDs in HTCondor
in the Administrator's Manual for details.

\item[6. Where should the machine-specific directories for
HTCondor go?]

HTCondor needs a few directories that are unique on every machine in
your pool.  These are 
\File{execute},
\File{spool}, 
\File{log},
(and possibly \File{lock}). 
Generally, all
of them are subdirectories of a single machine specific directory called
the local directory (specified by the \Macro{LOCAL\_DIR} macro
in the configuration file).
\index{owner!of directories}
Each should be owned by the user that HTCondor is to be run as.
Do not stage other files in any of these directories;
any files not created by HTCondor in these directories are subject to removal.

If you have a Unix user named \Login{condor} with a local home directory on each
machine, the \MacroNI{LOCAL\_DIR} could just be user \Login{condor}'s home
directory (\MacroNI{LOCAL\_DIR} = \MacroUNI{TILDE} in the 
configuration file).
If this user's home directory is shared among all machines in your
pool, you would want to create a directory for each host (named by
host name) for the local directory (for example, \MacroNI{LOCAL\_DIR} =
\MacroUNI{TILDE}/hosts/\MacroUNI{HOSTNAME}).  If you do not
have a \Login{condor} account on your machines, you can put these directories
wherever you'd like.
However, where to place the directories will require some
thought, as each one has its own resource needs:

\begin{description}
\index{Unix directory!\File{execute}}
\index{disk space requirement!\File{execute} directory}
\item[\File{execute}] This is the directory that acts as the current working
directory for any HTCondor jobs that run on a given execute machine.
The binary for the remote job is copied into this directory, so
there
must be enough space for it.  (HTCondor will not send a job to a
machine that does not have enough disk space to hold the initial
binary..)  In addition, if the remote job dumps core for some reason,
it is first dumped to the execute directory before it is sent back to
the submit machine.  So, put the execute directory on
a partition with enough space to hold a possible core file from the
jobs submitted to your pool.

\index{Unix directory!\File{spool}}
\index{disk space requirement!\File{spool} directory}
\item[\File{spool}] The \File{spool} directory holds the job queue
and history files,
and the checkpoint files for all jobs submitted from a given machine.
As a result, disk space requirements for the \File{spool} directory
can be quite large,
particularly if users are submitting jobs with very large
executables or image sizes.
By using a checkpoint server
(see section~\ref{sec:Ckpt-Server} on Installing a Checkpoint Server on
for details),
you can ease the disk
space requirements, since all checkpoint files are stored on the
server instead of the spool directories for each machine.  However,
the initial checkpoint files (the executables for all the clusters you
submit) are still stored in the spool directory, so you will need
%
% how much?!?
%
some space, even with a checkpoint server. The amount of space will 
depend on how many executables, and what size they are, that need to be stored
in the spool directory.

\index{Unix directory!\File{log}}
\index{disk space requirement!\File{log} directory}
\item[\File{log}] Each HTCondor daemon writes its own log file,
and each log file is placed
in the \File{log} directory.  You can specify what size you want these files
to grow to before they are rotated,
%
% rotated?  Maybe this is talking about wrapping around to
% overwrite the oldest entries first
%
so the disk space requirements of
the directory are configurable.
The larger the log files, the more
historical information they will hold if there is a problem, but the
more disk space they use up.  If you have a network file system
installed at your pool, you might want to place the log directories in
a shared location (such as \File{/usr/local/condor/logs/\$(HOSTNAME)}),
so that you can view the log files from all your machines in a single
location.  However, if you take this approach, you will have to
specify a local partition for the \File{lock} directory (see below).

\index{Unix directory!\File{lock}}
\item[\File{lock}] HTCondor uses a small number of lock files to synchronize
access to certain files that are shared between multiple daemons.
Because of problems encountered with file locking and network
file systems (particularly NFS), these lock files should be placed on a
local partition on each machine.  By default, they are placed in
the \File{log} directory.  If you place your \File{log}
directory on a network file system partition,
specify a local partition for the
lock files with the \Macro{LOCK} parameter in the configuration file (such as
\File{/var/lock/condor}).

\end{description}

\index{disk space requirement!HTCondor files}
Generally speaking, it is recommended that you do not put these directories
(except \File{lock}) on the same partition as \File{/var},
since if the partition
fills up, you will fill up \File{/var} as well. 
This will cause lots of
problems for your machines.  Ideally, you will have a separate partition
for the HTCondor directories. Then, the only consequence of filling up
the directories
will be HTCondor's malfunction, not your whole machine.

\item[7. Where should the parts of the HTCondor system be installed?]

	\begin{itemize}
	\item Configuration Files
	\item Release directory
		\begin{itemize}
		\item User Binaries
		\item System Binaries 
		\item \File{lib} Directory
	  	\item \File{etc} Directory
		\end{itemize}
	\item Documentation
	\end{itemize}

\label{sec:Config-File-Locations}
\begin{description}
\item[Configuration Files] There can be more than one configuration file.
They allow
different levels of control over how HTCondor is configured on each
machine in the pool.  
The global configuration file is shared by all machines in the pool.
For ease of administration, this file should be located on a shared
file system, if possible.
Local configuration files override settings in the
global file permitting different daemons to run,
different policies for when to start and stop HTCondor jobs, and so on.
There may be configuration files specific to each platform in the pool.
See section~\ref{sec:Multiple-Platforms} on about Configuring HTCondor for
Multiple Platforms for details.

\index{configuration files!location}
The location of configuration files is described in
section~\ref{sec:Ordering-Config-File}.

\item[Release Directory]

Every binary distribution contains a contains
five subdirectories: \File{bin}, \File{etc}, \File{lib}, \File{sbin},
and \File{libexec}. Wherever you
choose to install these five directories we call the release directory
(specified by the \Macro{RELEASE\_DIR} macro in the configuration file).
Each
release directory contains platform-dependent binaries and libraries,
so you will need to install a separate one for each kind of machine in
your pool.  For ease of administration, these directories should be
located on a shared file system, if possible.

\begin{itemize}
     \item User Binaries:

     All of the files in the \File{bin} directory are programs that
     HTCondor users should expect to have in their path.  You could
     either put them in a well known location (such as
     \File{/usr/local/condor/bin}) which you have HTCondor users add to
     their \Env{PATH} environment variable, or copy those files
     directly into a well known place already in the user's PATHs (such as
     \File{/usr/local/bin}).  With the above examples, you could also
     leave the binaries in \File{/usr/local/condor/bin} and put in
     soft links from \File{/usr/local/bin} to point to each program.

     \item System Binaries:

     All of the files in the \File{sbin} directory are HTCondor daemons and
     agents, or programs that only the HTCondor administrator would need
     to run.  Therefore, add these programs only
     to the \Env{PATH} of the HTCondor administrator.

     \item Private HTCondor Binaries:

     All of the files in the \File{libexec} directory are HTCondor
     programs that should never be run by hand, but are only used
     internally by HTCondor. 

     \item \File{lib} Directory:

     The files in the \File{lib} directory are the HTCondor libraries that
     must be linked in with user jobs for all of HTCondor's
     checkpointing and migration features to be used.  \File{lib} also
     contains scripts used by the \Condor{compile} program to help
     re-link jobs with the HTCondor libraries.  These files should be
     placed in a location that is world-readable, but they do not need
     to be placed in anyone's \Env{PATH}.  The \Condor{compile} script checks
     the configuration file for the location of the \File{lib} directory.

     \item \File{etc} Directory:

     \File{etc} contains an \File{examples} subdirectory which holds various
     example configuration files and other files used for installing HTCondor.
     \File{etc} is the recommended location to keep the master copy of your
     configuration files.  You can put in soft links from one of the places
     mentioned above that HTCondor checks automatically to find its
     global configuration file. 
\end{itemize}

\item[Documentation]

The documentation provided with HTCondor is currently available in
     HTML, Postscript and PDF (Adobe Acrobat).  It can be locally installed
     wherever is customary at your site.  You can also find the HTCondor
     documentation on the web at:
     \URL{http://htcondor.org/manual}.

\end{description}

\item[8. Am I using AFS?]

If you are using AFS at your site, be sure to read the
section~\ref{sec:HTCondor-AFS} in the
manual.
HTCondor does not currently have a way to authenticate itself to AFS.
A solution is not ready for
\VersionNotice.
This implies that you are probably not going to want
to have the \Macro{LOCAL\_DIR} for HTCondor on AFS.
However, you can
(and probably should) have the HTCondor \MacroNI{RELEASE\_DIR} on AFS, so
that you can share one copy of those files and upgrade them in a
centralized location.  You will also have to do something special if
you submit jobs to HTCondor from a directory on AFS.  Again, read manual
section~\ref{sec:HTCondor-AFS} for all the details.

\item[9. Do I have enough disk space for HTCondor?]

\index{disk space requirement!all versions}
The compressed downloads of HTCondor currently range from a low of about 13
Mbytes for 64-bit Ubuntu 12/Linux to about 115 Mbytes for Windows.  The
compressed source code takes approximately 17 Mbytes.

In addition, you will need a lot of disk space in the local directory
of any machines that are submitting jobs to HTCondor.  See question 6
above for details on this.

\end{description}

%%%%%%%%%%%%%%%%%%%%%%%%%%%%%%%%%%%%%%%%%%%%%%%%%%%%%%%%%%%%%%%%%%%%%%
\subsubsection{\label{sec:install-rpms}
Unix Installation from an RPM}
\index{installation!using Red Hat RPMs}
\index{RPM installation on Red Hat}

RPMs are available for HTCondor \VersionNotice.
We provide a Yum repository, as well as 
installation and configuration in one easy step.
This RPM installation is currently available for Red Hat-compatible
systems only.
As of HTCondor version 7.5.1, 
the HTCondor RPM installs into File Hierachy Standard locations.

Yum repositories and instructions are at
 \URL{http://htcondor.org/yum/} .
The repositories are named to distinguish stable releases from
development releases and by Red Hat version number. 
The 4 repositories are:
\begin{itemize}
  \item \File{htcondor-stable-rhel6.repo}
  \item \File{htcondor-stable-rhel7.repo}
  \item \File{htcondor-development-rhel6.repo}
  \item \File{htcondor-development-rhel7.repo}
\end{itemize}

Here is an ordered set of steps that get HTCondor running using the RPM.
\begin{enumerate}
\item The HTCondor package will automatically add a \Login{condor} user/group,
if it does not exist already.
Sites wishing to control the attributes of this user/group 
should add the \Login{condor} user/group manually before installation.

\item Download and install the meta-data that describes 
the appropriate YUM repository. 
This example is for the stable series, on RHEL 7.
\footnotesize
\begin{verbatim}
  cd /etc/yum.repos.d
  wget http://htcondor.org/yum/repo.d/htcondor-stable-rhel7.repo
\end{verbatim}
\normalsize
Note that this step need be done only once;
do not get the same repository more than once.

\item Import signing key
The RPMs are signed in the Redhat 6 and RedHat 7 repositories. 
\begin{verbatim}
  wget http://htcondor.org/yum/RPM-GPG-KEY-HTCondor
  rpm --import RPM-GPG-KEY-HTCondor
\end{verbatim}

\item Install HTCondor.
\begin{verbatim}
  yum install condor-all
\end{verbatim}

\item As needed, edit the HTCondor configuration files to customize.
The configuration files are in the directory \File{/etc/condor/} .
Do \emph{not} use \Condor{configure} or \Condor{install} for configuration.
The installation will be able to find configuration files without
additional administrative intervention,
as the configuration files are placed in \File{/etc},
and HTCondor searches this directory.

\item Start HTCondor daemons:
\begin{verbatim}
  /sbin/service condor start
\end{verbatim}

\end{enumerate}

% Alain thinks that upgrades DO work.
%RPM upgrade (\Opt{-u} option) does not currently
%work for HTCondor \VersionNotice.

%%%%%%%%%%%%%%%%%%%%%%%%%%%%%%%%%%%%%%%%%%%%%%%%%%%%%%%%%%%%%%%%%%%%%%
\subsubsection{\label{sec:install-debs}
Unix Installation from a Debian Package}
\index{installation!using Debian packages}
\index{Debian installation with Debian packages}

Debian packages are available in HTCondor \VersionNotice.
We provide an APT repository, as well as 
installation and configuration in one easy step.
These Debian packages of HTCondor are currently available for 
Debian 7 (wheezy) and Debian 8 (jessie).
As of HTCondor version 7.5.1, 
the HTCondor Debian package installs into File Hierachy Standard locations.

The HTCondor APT repositories are specified at
 \URL{http://htcondor.org/debian/} .
See this web page for repository information.

Here is an ordered set of steps that get HTCondor running.
\begin{enumerate}
\item The HTCondor package will automatically add a \Login{condor} user/group,
if it does not exist already.
Sites wishing to control the attributes of this user/group 
should add the \Login{condor} user/group manually before installation.

\item If not already present,
set up access to the appropriate APT repository;
they are distinguished as stable or development release,
and by operating system. 
Ensure that the correct one of the following release and 
operating system-specific lines is in 
the file \File{/etc/apt/sources.list} .
\footnotesize
\begin{verbatim}
deb http://htcondor.org/debian/stable/ wheezy contrib
deb http://htcondor.org/debian/development/ wheezy contrib
deb http://htcondor.org/debian/stable/ jessie contrib
deb http://htcondor.org/debian/development/ jessie contrib
\end{verbatim}
\normalsize
Note that this step need be done only once;
do not add the same repository more than once.

\item Install and start HTCondor services:
\begin{verbatim}
  apt-get update
  apt-get install condor
\end{verbatim}

\item As needed, edit the HTCondor configuration files to customize.
The configuration files are in the directory \File{/etc/condor/} .
Do \emph{not} use \Condor{configure} or \Condor{install} for configuration.
The installation will be able to find configuration files without
additional administrative intervention,
as the configuration files are placed in \File{/etc},
and HTCondor searches this directory.

Then, if any configuration changes are made, restart HTCondor with
\begin{verbatim}
  /etc/init.d/condor restart
\end{verbatim}

\end{enumerate}

%%%%%%%%%%%%%%%%%%%%%%%%%%%%%%%%%%%%%%%%%%%%%%%%%%%%%%%%%%%%%%%%%%%%%%
\subsubsection{\label{sec:unix-install-from-tarball}
Unix Installation from a Tarball}

\Bold{Note that installation from a tarball is no longer the preferred
method for installing HTCondor on Unix systems.  Installation
via RPM or Debian package is recommended if available for your
Unix version.}

An overview of the tarball-based installation process is as follows:
\begin{enumerate}
\item Untar the HTCondor software.
\item Run \Condor{install} or \Condor{configure} to install the software.
\end{enumerate}
Details are given below.

\medskip

After download, all the files are in a compressed, tar format.
They need to be untarred, as
\begin{verbatim}
  tar xzf <completename>.tar.gz
\end{verbatim}
After untarring, the directory will have the Perl scripts
\Condor{configure} and \Condor{install} (and \Bosco{install}), as well as 
\File{bin}, \File{etc}, \File{examples}, \File{include},
\File{lib}, \File{libexec}, \File{man}, \File{sbin},
\File{sql} and \File{src} subdirectories.

\index{installation!with \Condor{configure}}
\index{condor\_configure command}
The Perl script \Condor{configure} installs HTCondor.
Command-line arguments specify all needed information to this
script.  The script can be executed multiple times, to modify or further
set the configuration.  \Condor{configure} has been tested using Perl 5.003.
Use this or a more recent version of Perl.

\Condor{configure} and \Condor{install} are the same program, but have
different default behaviors.  \Condor{install} is identical to
running 
\begin{verbatim}
  condor_configure --install=.
\end{verbatim}
\Condor{configure} and \Condor{install} work on the named directories.
As the names imply, \Condor{install} is used to
install HTCondor, whereas \Condor{configure} is used to modify the
configuration of an existing HTCondor install.

\Condor{configure} and \Condor{install} are completely command-line
driven and are not interactive.  Several command-line arguments are
always needed with \Condor{configure} and \Condor{install}.
The argument
\begin{verbatim}
  --install=/path/to/release
\end{verbatim}
specifies the path to the HTCondor release directories.
The default command-line argument for \Condor{install} is 
\begin{verbatim}
  --install=.
\end{verbatim}
The argument
\begin{verbatim}
  --install-dir=<directory>
\end{verbatim}
or
\begin{verbatim}
  --prefix=<directory>
\end{verbatim}
specifies the path to the install directory.

The argument
\begin{verbatim}
  --local-dir=<directory>
\end{verbatim}
specifies the path to the local directory.

The \verb@--@\Opt{type} option to \Condor{configure}
specifies one or more of the roles that a machine can take on
within the HTCondor pool: central manager, submit or execute.
These options are given in a comma separated list.
So, if a machine is both a submit and execute
machine, 
the proper command-line option is
\begin{verbatim}
  --type=submit,execute
\end{verbatim}

Install HTCondor on the central manager machine first.  If HTCondor
will run as root in this pool (Item 3 above), run \Condor{install} 
as root, and it will install and set the file permissions correctly.  
On the central manager machine, run \Condor{install} as follows.
\begin{verbatim}
% condor_install --prefix=~condor \
	--local-dir=/scratch/condor --type=manager
\end{verbatim}

To update the above HTCondor installation, for example, to also be
submit machine:
\begin{verbatim}
% condor_configure --prefix=~condor \
	--local-dir=/scratch/condor --type=manager,submit
\end{verbatim}

As in the above example, the central manager can also be a submit
point or an execute machine, but this is only recommended for very
small pools.  If this is the case, the 
\verb@--@\Opt{type}
option changes to
\Expr{manager,execute} or \Expr{manager,submit}  or 
\Expr{manager,submit,execute}.

After the central manager is installed, the execute and submit machines
should then be configured.  Decisions about whether to run HTCondor as root
should be consistent throughout the pool. For each machine in the pool,
run

\begin{verbatim}
% condor_install --prefix=~condor \
	--local-dir=/scratch/condor --type=execute,submit
\end{verbatim}

See the \Condor{configure} manual 
page~\pageref{man-condor-configure}
for details.


%%%%%%%%%%%%%%%%%%%%%%%%%%%%%%%%%%%%%%%%%%%%%%%%%%%%%%%%%%%%%%%%%%%%%%
\subsubsection{\label{installed-now-what}
Starting HTCondor Under Unix After Installation}

\index{starting HTCondor!Unix platforms}

Now that HTCondor has been installed on the machine(s), there are a few
things to check before starting up HTCondor.

\begin{enumerate}
\item Read through the \Release{etc/condor\_config} file.  There are a
    lot of possible settings and you should at least take a look at
    the first two main sections to make sure everything looks okay.
    In particular, you might want to set up security for
    HTCondor.  See the section~\ref{sec:Security-Model}
    to learn how to do this.

\item For Linux platforms, run the \Condor{kbdd} to monitor keyboard
    and mouse activity on all machines within the pool that will
    run a \Condor{startd}; these are machines that execute jobs.
    To do this, the subsystem \Expr{KBDD} will need to be added to
    the \MacroNI{DAEMON\_LIST} configuration variable definition.

    For Unix platforms other than Linux,
    HTCondor can monitor the activity of your mouse and keyboard,
    provided that you tell it where to look.  You do this with the
    \Macro{CONSOLE\_DEVICES} entry in the \condor{startd} section of
    the configuration file.  On most platforms, reasonable
    defaults are provided.
    For example, the default device for the mouse
    is 'mouse', since most installations have a soft link from
    \File{/dev/mouse} that points to the right device (such as
    \File{tty00} if you have a serial mouse, \File{psaux} if you have
    a PS/2 bus mouse, etc).  If you do not have a \File{/dev/mouse}
    link, you should either create one (you will be glad you did), or
    change the \MacroNI{CONSOLE\_DEVICES} entry in HTCondor's
    configuration file.
    This entry is a comma separated list, so you can have any
    devices in \File{/dev} count as 'console devices' and activity
    will be reported in the \condor{startd}'s ClassAd as
    \AdAttr{ConsoleIdleTime}.

\item  (Linux only) HTCondor needs to be able to find the \File{utmp} file.
    According to the Linux File System Standard, this file should be
    \File{/var/run/utmp}.  If HTCondor cannot find it there, it looks in
    \File{/var/adm/utmp}.  If it still cannot find it, it gives up.  So, if
    your Linux distribution places this file somewhere else, be sure to
    put a soft link from \File{/var/run/utmp} to point to the real location.

\end{enumerate}

To start up the HTCondor daemons, execute the command
\Release{sbin/condor\_master}.  This is the HTCondor master, whose
only job in life is to make sure the other HTCondor daemons are running.
The master keeps track of the daemons, restarts them if they crash,
and periodically checks to see if you have installed new binaries (and,
if so, restarts the affected daemons).

If you are setting up your own pool, you should start HTCondor on your
central manager machine first.  If you have done a submit-only
installation and are adding machines to an existing pool,
the start order does not
matter.

To ensure that HTCondor is running, you can run either:
\begin{verbatim}
        ps -ef | egrep condor_
\end{verbatim}
or
\begin{verbatim}
        ps -aux | egrep condor_
\end{verbatim}
depending on your flavor of Unix.  
On a central manager machine that can submit jobs as well
as execute them, there will be processes for:
\begin{itemize}
	\item \condor{master}
	\item \condor{collector}
	\item \condor{negotiator}
	\item \condor{startd}
	\item \condor{schedd}
\end{itemize}
On a central manager machine that does not submit jobs nor
execute them, there will be processes for:
\begin{itemize}
	\item \condor{master}
	\item \condor{collector}
	\item \condor{negotiator}
\end{itemize}
For a machine that only submits jobs, there will be processes for:
\begin{itemize}
	\item \condor{master}
	\item \condor{schedd}
\end{itemize}
For a machine that only executes jobs, there will be processes for:
\begin{itemize}
	\item \condor{master}
	\item \condor{startd}
\end{itemize}

Once you are sure the HTCondor daemons are running, check to make sure
that they are communicating with each other.  You can run
\Condor{status} to get a one line summary of the status of each
machine in your pool.

Once you are sure HTCondor is working properly, you should add
\Condor{master} into your startup/bootup scripts (i.e. \File{/etc/rc} ) so
that your machine runs \Condor{master} upon bootup.  \Condor{master}
will then fire up the necessary HTCondor daemons whenever your machine
is rebooted.  

If your system uses System-V style init scripts, you can look in
\Release{etc/examples/condor.boot} for a script that can be used
to start and stop HTCondor automatically by init.  Normally, you would
install this script as \File{/etc/init.d/condor} and put in soft link from
various directories (for example, \File{/etc/rc2.d}) that point back to
\File{/etc/init.d/condor}.  The exact location of these scripts and links
will vary on different platforms.

If your system uses BSD style boot scripts, you probably have an
\File{/etc/rc.local} file.  Add a line to start up
\Release{sbin/condor\_master}.


Now that the HTCondor daemons are running, there are a few things you
can and should do:

\begin{enumerate}
\item (Optional) Do a full install for the \Condor{compile} script.
    \condor{compile} assists in linking jobs with the HTCondor libraries
    to take advantage of all of HTCondor's features.  As it is currently
    installed, it will work by placing it in front of any of the
    following commands that you would normally use to link your code:
    gcc, g++, g77, cc, acc, c89, CC, f77, fort77 and ld.  If you
    complete the full install, you will be able to use
    \condor{compile} with any command whatsoever, in particular, make.
    See section~\ref{sec:full-condor-compile} in the manual for directions.

\item Try building and submitting some test jobs.  See
    \File{examples/README} for details.

\item If your site uses the AFS network file system, see
section~\ref{sec:HTCondor-AFS} in the
manual.

\item We strongly recommend that you start up HTCondor (run the
\Condor{master} daemon) as user root.  If you must start HTCondor as
some user other than root, see section~\ref{sec:Non-Root}.

\end{enumerate}

%%%%%%%%%%%%%%%%%%%%%%%%%%%%%%%%%%%%%%%%%%%%%%%%%%%%%%%%%%%%%%%%%%%%%%
%%%%%%%%%%%%%%%%%%%%%%%%%%%%%%%%%%%%%%%%%%%%%%%%%%%%%%%%%%%%%%%%%%%%%%
\subsection{\label{sec:Windows-Install}Installation on Windows}
%%%%%%%%%%%%%%%%%%%%%%%%%%%%%%%%%%%%%%%%%%%%%%%%%%%%%%%%%%%%%%%%%%%%%%

\index{installation!Windows|(}
\index{Windows!installation|(}
This section contains the instructions for installing the Microsoft
Windows version of Condor at your site.  
The install program will set you up with a slightly customized configuration
file that you can further customize after the installation has completed.

Please read the copyright and disclaimer information in 
section~\ref{sec:condor-public-license} on
page~\pageref{sec:condor-public-license} of the manual, or in the
file 
\File{LICENSE.TXT}, before proceeding.  Installation and
use of Condor is acknowledgement that you have read and agreed to these
terms.

Be sure that the Condor tools that get run are of the same version
as the daemons installed.
If they were not (such as 6.5.3 daemons, when running 6.4 \Condor{submit}),
then things will not work.
There may be errors generated by the \Condor{schedd} daemon (in the log).
It is likely that a job would be correctly placed in the queue,
but the job will never run.

The Condor executable for distribution is packaged in
a single file such as:
\begin{verbatim}
  condor-6.7.8-winnt40-x86.msi
\end{verbatim}

\index{Windows!installation!initial file size}
This file is approximately 80 Mbytes in size, and may be
removed once Condor is fully installed.

Before installing Condor, please consider joining the condor-world mailing
list.  Traffic on this list is kept to an absolute minimum.  It is only
used to announce new releases of Condor.
To subscribe, follow the directions given at
\URL{http://www.cs.wisc.edu/condor/mail-lists/}.

\subsubsection{Installation Requirements}

\begin{itemize}

\item Condor for Windows requires Windows 2000 (or better) or Windows XP.

\item 300 megabytes of free disk space is recommended.  Significantly more 
disk space could be desired to be able to run jobs with large data files.

\item Condor for Windows will operate on either an NTFS or FAT filesystem.  However, for security purposes, NTFS is preferred.

\end{itemize}

%%%%%%%%%%%%%%%%%%%%%%%%%%%%%%%%%%%%%%%%%%%%%%%%%%%%%%%%%%%%%%%%%%%%%%
\subsubsection{\label{sec:NT-Preparing-to-Install}Preparing to Install
Condor under Windows } 
%%%%%%%%%%%%%%%%%%%%%%%%%%%%%%%%%%%%%%%%%%%%%%%%%%%%%%%%%%%%%%%%%%%%%%

\index{Windows!installation!preparation}
Before you install the Windows version of Condor at your site,
there are two major
decisions to make about the basic layout of your pool.

\begin{enumerate}
\item What machine will be the central manager?
\item Do I have enough disk space for Condor?
\end{enumerate}

If you feel that you already know the answers to these questions,
skip to the Windows Installation Procedure section below,
section~\ref{sec:nt-install-procedure} on
page~\pageref{sec:nt-install-procedure}.
If you are unsure, read on.

\begin{itemize} 

%%%%%%%%%%%%%%%%%%%%%%%%%%%%%%%%%%%%%%%%%%%%%%%%%%%%%%%%%%%%%%%%%%%%%%
\item{What machine will be the central manager?}
%%%%%%%%%%%%%%%%%%%%%%%%%%%%%%%%%%%%%%%%%%%%%%%%%%%%%%%%%%%%%%%%%%%%%%

One machine in your pool must be the central manager.
This is the
centralized information repository for the Condor pool and is also the
machine that matches available machines with waiting
jobs.  If the central manager machine crashes, any currently active
matches in the system will keep running, but no new matches will be
made.  Moreover, most Condor tools will stop working.  Because of the
importance of this machine for the proper functioning of Condor, we
recommend you install it on a machine that is likely to stay up all the
time, or at the very least, one that will be rebooted quickly if it
does crash.  Also, because all the services will send updates (by
default every 5 minutes) to this machine, it is advisable to consider
network traffic and your network layout when choosing the central
manager.

For Personal Condor, your machine will act as your central manager.

Install Condor on the central manager before installing
on the other machines within the pool.

%%%%%%%%%%%%%%%%%%%%%%%%%%%%%%%%%%%%%%%%%%%%%%%%%%%%%%%%%%%%%%%%%%%%%%
\item{Do I have enough disk space for Condor?}
%%%%%%%%%%%%%%%%%%%%%%%%%%%%%%%%%%%%%%%%%%%%%%%%%%%%%%%%%%%%%%%%%%%%%%

\index{Windows!installation!required disk space}
The Condor release directory takes up a fair amount of space.
The size requirement for the release
directory is approximately 200 Mbytes.

Condor itself, however, needs space to store all of your jobs, and their
input files.  If you will be submitting large amounts of jobs,
you should consider installing Condor on a volume with a large amount
of free space.

\end{itemize}


%%%%%%%%%%%%%%%%%%%%%%%%%%%%%%%%%%%%%%%%%%%%%%%%%%%%%%%%%%%%%%%%%%%%%%
\subsubsection{\label{sec:nt-install-procedure}
Installation Procedure using the included Set Up Program}
%%%%%%%%%%%%%%%%%%%%%%%%%%%%%%%%%%%%%%%%%%%%%%%%%%%%%%%%%%%%%%%%%%%%%%

% condor MUST be run as local system
% 
%  root == administrator
%  to install, must be running with administrator privileges
%  the kernel runs as == local system

Installation of Condor must be done by a user with administrator privileges.
After installation, the Condor services will be run under the local system account.
When Condor is running a user job, however, it will run that User job with normal user permissions.

Download Condor, and start the installation process by running the file (or by double clicking on the file).
The Condor installation is completed by answering questions and choosing options within the following steps.


\begin{description}
\item[If Condor is already installed.]

     For upgrade purposes, you may be running the installation of Condor
     after it has been previously installed.
     In this case, a dialog box will appear before the
     installation of Condor proceeds.
     The question asks if you wish to preserve your current
     Condor configuration files.
     Answer yes or no, as appropriate.
	 
	 If you answer yes, your configuration files will not be changed, and you will proceed to the point where the new binaries will be installed.

     If you answer no, then there will be a second question
     that asks if you want to use answers
     given during the previous installation
     as default answers.

\item[STEP 1: License Agreement.]

     The first step in installing Condor
     is a welcome screen and license agreement.
     You are reminded that it is best to run the installation
     when no other Windows programs are running.
	 If you need to close other Windows programs, it is safe to cancel the
	 installation and close them.
     You are asked to agree to the license.
     Answer yes or no.  If you should disagree with the License, the
	 installation will not continue.

     After agreeing to the license terms, the next Window is where 
     fill in your name and company information,
     or use the defaults as given.

\item[STEP 2: Condor Pool Configuration.]

     The Condor installation will require different
     information depending on whether the installer will
	 be creating a new pool, or joining an existing one.

     If you are creating a new pool, the installation program
	 requires that this machine is the central manager.  
     For the creation of a new Condor pool, you will be asked
	 some basic information about your new pool:
     \begin{description}
     \item[Name of the pool]
     \item[hostname] of this machine.
%  Derek hath declared the Statistics not worthy of prime time.
%     \item[Do you want to keep statistics?]
%       Answer yes or no, as appropriate.
%       If yes, then the maximum amount of data accumulated will
%       be 10 Mbytes.
%       A configurable quantity, \Macro{POOL\_HISTORY\_MAX\_STORAGE}
%       sets the maximum amount of data, and it
%       defaults to 10 Mbytes.
%       If no, then the Condor View client will not have data to display.
     \item[Size of pool]
       Condor needs to know if this a Personal Condor installation,
       or if there will be more than one machine in the pool.
\index{Windows!installation!Personal Condor}
\index{Personal Condor}
       A Personal Condor pool
       implies that there is only one machine in the pool.
       For Personal Condor, several of the following
       steps are omitted as noted.
     \end{description}

     If you are joining an existing pool, all the installation program
	 requires is the hostname of the central manager for your pool.

\item[STEP 3: This Machine's Roles.] 

     This step is omitted for the installation of Personal Condor.

     Each machine within a Condor pool may either
     submit jobs or execute submitted jobs, or both
     submit and execute jobs.
     This step allows the installation on this machine
     to choose if the machine will only submit jobs,
     only execute submitted jobs, or both.
     The common case is both, so the default is both.

\item[STEP 4: Where will Condor be installed?]

\index{Windows!installation!location of files}
The next step is where the destination of the Condor files will be
decided.
It is recommended that Condor be installed in the location shown as the default in the dialog box:
\verb@C:\Condor@.

Installation on the local disk is chosen for several reasons.

The Condor services run as local system, and within Microsoft Windows, local system has no network privileges.
Therefore, for Condor to operate, Condor should be installed on a local hard drive as opposed to a network drive (file server).

The second reason for installation on the local disk is that
the Windows usage of drive letters has implications for where
Condor is placed.
The drive letter used must be not change, even when different users are
logged in.
Local drive letters do not change under normal operation of Windows.

While it is strongly discouraged, it may be possible to place Condor on a hard drive that is not local,  if a dependency is added to the service control manager
such that Condor starts after the required file services
are available.

%  !! goes in C:/condor   (default)
%  !! advice is really should go on local hard drive,
%  as opposed to a network drive (also called file server)
%  Because,
%    1. Condor runs as local system, and accesses to a network
%      drive can't be authenticated  -- local system has
%      no network privileges.
%    2.  it is likely that you don't have this set up:
%    (and you need it to make it work)
%    you can add a dependency in the service control manager
%    that condor should start after the file services are
%    available
%    3. drive letters are "system-wide"
%    Must have dedicated letter (for all users), that remains
%    intact for all time, or condor won't know where
%    things are and can't get access (without its "letter")


\item[STEP 5: Where is the Java Virtual Machine?]
	While not required, it is possible for Condor to run jobs in the
	Java universe. In order for Condor to have support for java,
	you must supply a path to \verb@java.exe@ on your system. The
	installer will tell you if the path is invalid before proceeding
	to the next step. To disable the Java universe, simply leave
	this field blank.

\item[STEP 6: Where should Condor send e-mail if things go wrong?]

     Various parts of Condor will send e-mail to a Condor administrator
     if something goes wrong and requires human attention.
     You specify the e-mail address and the SMTP relay host
     of this administrator.  Please pay close attention to this email
	 since it will indicate problems in your Condor pool.

\item[STEP 7: The domain.]

% not really used right now.  "Things that suck about NT."
% UNIX has 2 domains:  file system domain and user-ID domain
% NT has only 1:  a combination, and so going back to letter
% drives, things get screwed up.
     This step is omitted for the installation of Personal Condor.

     Enter the machine's accounting (or UID) domain.
	 On this version of Condor for Windows, this setting only used for User
	 priorities (see section~\ref{sec:UserPrio} on
	 page~\pageref{sec:UserPrio}) and to form a default email address for
	 the user.

\item[STEP 8: Access permissions.]
     This step is omitted for the installation of Personal Condor.

     Machines within the Condor pool will need
     various types of access permission. 
     The three categories of permission are read, write,
     and administrator. Enter the machines to be given
     access permissions.

     \begin{description}
     \item[Read]
     Read access allows a machine to obtain information about
     Condor such as the status of machines in the pool and the
     job queues.
     All machines in the pool should be given read access. 
     In addition, giving read access to *.cs.wisc.edu 
     will allow the Condor team to obtain information about
     your Condor pool in the event that debugging is needed.
     \item[Write]
     All machines in the pool should be given write access. 
     It allows the machines you specify to send information to your
	 local Condor daemons, for example, to start a Condor Job.
     Note that for a machine to join the Condor pool, it must have both read and write access to all of the machines in the pool.
     \item[Administrator]
     A machine with administrator access will be allowed more
     extended permission to to things such as
     change other user's priorities, modify the job queue,
     turn Condor services on and off,
     and restart Condor.
     The central manager should be given administrator access
     and is the default listed.
	 This setting is granted to the entire machine, so care should be taken not to make this too open.
     \end{description}

	 For more details on these access permissions, and others that can be
	 manually changed in your \File{condor\_config} file, please
	 see the section titled Setting Up IP/Host-Based Security in Condor
	 in section
	 section~\ref{sec:Host-Security}
	 on page~\pageref{sec:Host-Security}.

\item[STEP 9: Job Start Policy.]
     Condor will execute submitted jobs on machines based on
     a preference given at installation.
     Three options are given, and the first is most commonly used
     by Condor pools.
     This specification may be changed or refined in
     the machine ClassAd requirements attribute.

     The three choices:
     \begin{description}
     \item[After 15 minutes of no console activity and low CPU activity.]
     \item[Always run Condor jobs.]
     \item[After 15 minutes of no console activity.]
     \end{description}

\index{Console activity}
     Console activity is the use of the mouse or keyboard.  For instance,
	 if you are reading this document online, and are using either the
	 mouse or the keyboard to change your position, you are generating
	 Console activity.

\index{CPU activity}
     Low CPU activity is defined as a load of less than 30\Percent
	 (and is configurable in your \File{condor\_config} file).  If you have
	 a multiple processor machine, this is the average percentage of
	 CPU activity for both processors.

	For testing purposes, it is often helpful to use use the Always run Condor
	jobs option.  For production mode, however, most people chose the
	After 15 minutes of no console activity and low CPU activity.

\item[STEP 10: Job Vacate Policy.]
     This step is omitted if Condor jobs are always run as
     the option chosen in STEP 9.

     If Condor is executing a job and the user returns,
	 Condor will immediately suspend the job, and after five minutes
	 Condor will decide what to do with the partially completed job.
     There are currently two options for the job.

     \begin{description}
     \item[The job is killed 5 minutes after your return.]
     The job is suspended immediately once there is console activity.
     If the console activity continues, then the job is
     vacated (killed) after 5 minutes. 
     Since this version does not include check-pointing, the job will
     be restarted from the beginning at a later time.
     The job will be placed back into the queue.
     \item[ Suspend job, leaving it in memory.]
     The job is suspended immediately.  At a later time, when the
	 console activity has stopped for ten minutes, the execution of
	 Condor job will be resumed (the job will be unsuspended).
	 The drawback to this option is that since the job will remain
	 in memory, it will occupy swap space.  In many instances, however,
	 the amount of swap space that the job will occupy is small.
     \end{description}

%    Advice on which to choose goes here.
     So which one do you choose?  Killing a job is less intrusive
	 on the workstation owner than leaving it in memory for a later time.
     A suspended job left in memory will require swap space,
     which could possibly be a scarce resource.
     Leaving a job in memory, however, has the benefit that accumulated
     run time is not lost for a partially completed job.

\item[STEP 11: Review entered information.]
     Check that the entered information is correctly entered.
     You have the option to return to previous dialog boxes to fix entries.
\end{description}


%%%%%%%%%%%%%%%%%%%%%%%%%%%%%%%%%%%%%%%%%%%%%%%%%%%%%%%%%%%%%%%%%%%%%%
\subsubsection{\label{sec:nt-unattended-install-procedure}
Unattended Installation Procedure using the included Set Up Program}
%%%%%%%%%%%%%%%%%%%%%%%%%%%%%%%%%%%%%%%%%%%%%%%%%%%%%%%%%%%%%%%%%%%%%%

\index{Windows!installation!unattended install}
This section details how to run the Condor for Windows installer in an
unattended batch mode, i.e. completely from the command prompt without the
GUI interface.

The Condor for Windows installer uses the Microsoft Installer (MSI)
technology, and can be configured for unattended installs just like any
other ordinary MSI installer.

The following is a sample batch file that is used to set all the
properties necessary for an unattended install.

\begin{verbatim}
@echo on
set ARGS=
set ARGS=%ARGS% NEWPOOL=N
set ARGS=%ARGS% POOLNAME=""
set ARGS=%ARGS% RUNJOBS=C
set ARGS=%ARGS% VACATEJOBS=Y
set ARGS=%ARGS% SUBMITJOBS=Y
set ARGS=%ARGS% CONDOREMAIL="you@yours.com"
set ARGS=%ARGS% SMTPSERVER="smtp.localhost"
set ARGS=%ARGS% HOSTALLOWREAD="*"
set ARGS=%ARGS% HOSTALLOWWRITE="*"
set ARGS=%ARGS% HOSTALLOWADMINISTATOR="$(FULL_HOSTNAME)"
set ARGS=%ARGS% INSTALLDIR="C:\Condor"
set ARGS=%ARGS% INSTALLDIR_NTS="C:\Condor"
set ARGS=%ARGS% POOLHOSTNAME="$(FULL_HOSTNAME)"
set ARGS=%ARGS% ACCOUNTINGDOMAIN="none"
set ARGS=%ARGS% JVMLOCATION="C:\Windows\system32\java.exe"
set ARGS=%ARGS% STARTSERVICE="Y"

msiexec /qb /l* condor-install-log.txt /i condor-6.7.18-winnt50-x86.msi %ARGS%
\end{verbatim}

Each property corresponds to answers supplied in the interactive installer
as described above. The following is a brief explanation of each property
as it applies to unattended installations:

\begin{description}
\item [NEWPOOL = $<$ Y \Bar\ N $>$]
determines whether the installer will create a new pool with the target
machine as the central manager.

\item [POOLNAME]
sets the name of the pool if a new pool is to be created. Possible values
are either the name or the empty string \verb@""@.

\item [RUNJOBS = $<$ N \Bar\ A \Bar\ I \Bar\ C $>$]
determines when Condor will run jobs. This can be set to:
\begin{itemize}
\item Never run jobs (N)
\item Always run jobs (A)
\item Only run jobs when the keyboard and mouse are Idle (I)
\item Only run jobs when the keyboard and mouse are idle and the CPU
usage is low (C)
\end{itemize}

\item [VACATEJOBS = $<$ Y \Bar\ N $>$]
determines what Condor should do when it has to stop the execution of
a user job. When set to Y, Condor will vacate the job and start
it somewhere else if possible. When set to N, Condor will merely
suspend the job in memory and wait for the machine to become available
again. 

\item[SUBMITJOBS  = $<$ Y \Bar\ N $>$]
will cause the installer to configure the machine as a submit
node when set to Y. 

\item[CONDOREMAIL]
sets the e-mail address of the Condor admininistrator. Possible values are
an e-mail address or the empty string \verb@""@.

\item[HOSTALLOWREAD]
is a list of host names that are allowed to issue READ commands to
Condor daemons. This value should be set in accordance with the
\Macro{HOSTALLOW\_READ} setting in the configuration file, as described in
section~\ref{sec:Host-Security} on page~\pageref{sec:Host-Security}.

\item[HOSTALLOWWRITE]
is a list of host names that are allowed to issue WRITE commands to
Condor daemons. This value should be set in accordance with the
\Macro{HOSTALLOW\_WRITE} setting in the configuration file, as described in
section~\ref{sec:Host-Security} on page~\pageref{sec:Host-Security}.

\item[HOSTALLOWADMINISTRATOR]
is a list of host names that are allowed to issue ADMINISTRATOR commands to
Condor daemons. This value should be set in accordance with the
\Macro{HOSTALLOW\_ADMINISTRATOR} setting in the configuration file, 
as described in
section~\ref{sec:Host-Security} on page~\pageref{sec:Host-Security}.

\item[INSTALLDIR]
defines the path to where Condor will be installed. 

\item[INSTALLDIR\_NTS]
should be set to whatever INSTALLDIR is set to, with the additional
restriction that it cannot end in a backslash. The installer will be fixed
in an upcoming version of Condor to not require this property.

\item[POOLHOSTNAME]
defines the host name of the pool's central manager. 

\item[ACCOUNTINGDOMAIN] 
defines the accounting (or UID) domain the target machine will be in.

\item[JVMLOCATION]
defines the path to Java virtual machine on the target machine.

\item[SMTPSERVER]
defines the host name of the SMTP server that the target machine is to
use to send e-mail.
\end{description}

\begin{description}
\item [STARTSERVICE = $<$ Y \Bar\ N $>$]
determines whether the Condor service will be started after the installation
completes.
\end {description}

After defining each of these properties for the MSI installer, the
installer can be started with the \verb@msiexec@ command. The following
command starts the installer in unattended mode, and dumps a journal of
the installer's progress to a log file:\\
\verb@msiexec /qb /l* condor-install-log.txt /i condor-6.7.18-winnt50-x86.msi@ [property=value] ... \\

More information on the features of \verb@msiexec@
can be found at Microsoft's website at
\URL{http://www.microsoft.com/resources/documentation/windows/xp/all/proddocs/en-us/msiexec.mspx}.

%%%%%%%%%%%%%%%%%%%%%%%%%%%%%%%%%%%%%%%%%%%%%%%%%%%%%%%%%%%%%%%%%%%%%%
\subsubsection{\label{sec:NT-Manual-Install}Manual Installation Condor on Windows}
%%%%%%%%%%%%%%%%%%%%%%%%%%%%%%%%%%%%%%%%%%%%%%%%%%%%%%%%%%%%%%%%%%%%%%

\index{Windows!manual install}
If you are to install Condor on many different machines, you may wish
to use some other mechanism to install Condor on additional machines
rather than running the Setup program described above on each machine.

\Warn This is for advanced users only!  All others should use the Setup program described above. 

Here is a brief overview of how to install Condor manually without using the provided GUI-based setup program:

\begin{description}
\item [The Service]
The service that Condor will install is called "Condor".  The Startup
Type is Automatic.  The service should log on as System Account, but
\Bold{do not enable} "Allow Service to Interact with Desktop".  The
program that is run is \Condor{master.exe}.

The Condor service can be installed and removed using the
\File{sc.exe} tool, which is included in Windows XP and Windows 2003
Server. The tool is also available as part of the Windows 2000
Resource Kit.

Installation can be done as follows:
\begin{verbatim}
sc create Condor binpath= c:\condor\bin\condor_master.exe
\end{verbatim}

To remove the service, use:
\begin{verbatim}
sc delete Condor
\end{verbatim}

\item [The Registry]
Condor uses a few registry entries in its operation.  The key that Condor
uses is HKEY\_LOCAL\_MACHINE/Software/Condor.  The values that Condor puts
in this registry key serve two purposes.
\begin{enumerate}
\item The values of CONDOR\_CONFIG and RELEASE\_DIR are used for Condor
to start its service.

CONDOR\_CONFIG should point to the \File{condor\_config} file.  In this version
of Condor, it \Bold{must} reside on the local disk.

RELEASE\_DIR should point to the directory where Condor is installed.  This
is typically
\verb@C:\Condor@, and again, this \Bold{must} reside on the
local disk.

\item The other purpose is storing the entries from the last installation
so that they can be used for the next one.
\end{enumerate}

\item [The Filesystem]
The files that are needed for Condor to operate are identical to the Unix
version of Condor, except that executable files end in \File{.exe}.  For
example the on Unix one of the files is \File{condor\_master} and on Condor
the corresponding file is \File{condor\_master.exe}.

These files currently must reside on the local disk for a variety of reasons.
Advanced Windows users might be able to put the files on remote resources.
The main concern is twofold.  First, the files must be there when the service
is started.  Second, the files must always be in the same spot (including
drive letter), no matter who is logged into the machine.  

Note also that when installing manually, you will need to create the
directories that Condor will expect to be present given your
configuration. This normally is simply a matter of creating the
\File{log}, \File{spool}, and \File{execute} directories.

\end{description}


%%%%%%%%%%%%%%%%%%%%%%%%%%%%%%%%%%%%%%%%%%%%%%%%%%%%%%%%%%%%%%%%%%%%%%
\subsubsection{\label{nt-installed-now-what}
Condor Is Installed Under Windows ... Now What?}
%%%%%%%%%%%%%%%%%%%%%%%%%%%%%%%%%%%%%%%%%%%%%%%%%%%%%%%%%%%%%%%%%%%%%%
\index{Windows!starting the Condor service}

After the installation of Condor is completed, the Condor service
must be started.  If you used the GUI-based setup program to install
Condor, the Condor service should already be started.  If you installed
manually, Condor must
be started by hand, or you can simply reboot. \Note The Condor service
will start automatically whenever you reboot your machine.

To start Condor by hand:
\begin{enumerate}
\item From the Start menu, choose Settings.
\item From the Settings menu, choose Control Panel.
\item From the Control Panel, choose Services.
\item From Services, choose Condor, and Start.
\end{enumerate}

Or, alternatively you can enter the following command from a command prompt:
\begin{verbatim}
         net start condor
\end{verbatim}

\index{Windows!Condor daemon names}
Run the Task Manager (Control-Shift-Escape) to check that Condor
services are running.  The following tasks should
be running:  
\begin{itemize}
\item \Condor{master.exe}
\item \Condor{negotiator.exe}, if this machine is a central manager.
\item \Condor{collector.exe}, if this machine is a central manager.
\item \Condor{startd.exe}, if you indicated that this Condor node should start jobs
\item \Condor{schedd.exe}, if you indicated that this Condor node should submit jobs
to the Condor pool.
\end{itemize}

Also, you should now be able to open up a new cmd (DOS prompt) window, and
the Condor bin directory should be in your path, so you can issue the normal
Condor commands, such as \Condor{q} and \Condor{status}.

\index{installation!Windows|)}
\index{Windows!installation|)}

%%%%%%%%%%%%%%%%%%%%%%%%%%%%%%%%%%%%%%%%%%%%%%%%%%%%%%%%%%%%%%%%%%%%%%
\subsubsection{\label{nt-running-now-what}
Condor is Running Under Windows ... Now What?}
%%%%%%%%%%%%%%%%%%%%%%%%%%%%%%%%%%%%%%%%%%%%%%%%%%%%%%%%%%%%%%%%%%%%%%

Once Condor services are running, try building
and submitting some test jobs.  See the \File{README.TXT} file in the
examples directory
for details.

%%%%%%%%%%%%%%%%%%%%%%%%%%%%%%%%%%%%%%%%%%%%%%%%%%%%%%%%%%%%%%%%%%%%%%

%%%%%%%%%%%%%%%%%%%%%%%%%%%%%%%%%%%%%%%%%%%%%%%%%%%%%%%%%%%%%%%%%%%%%%
\subsection{\label{sec:Pool-Upgrade}
Upgrading -- Installing a New Version on an Existing Pool}
%%%%%%%%%%%%%%%%%%%%%%%%%%%%%%%%%%%%%%%%%%%%%%%%%%%%%%%%%%%%%%%%%%%%%%
\index{pool management!installing a new version on an existing pool}
\index{installation!installing a new version on an existing pool}

An upgrade changes the running version of HTCondor
from the current installation to a newer version.
The safe method
to install and start running a newer version of HTCondor
in essence is:
shut down the current installation of HTCondor,
install the newer version,
and then restart HTCondor using the newer version.
To allow for falling back to the current version,
place the new version in a separate directory.
Copy the existing configuration files,
and modify the copy to point to and use the new version,
as well as incorporate any configuration variables that are new or changed
in the new version.
Set the \Env{CONDOR\_CONFIG} environment variable
to point to the new copy of the configuration,
so the new version of HTCondor will use the new configuration when restarted.

As of HTCondor version 8.2.0, 
the default configuration file has been substantially reduced
in size by defining compile-time default values for most configuration
variables.
Therefore,
when upgrading from a version of HTCondor earlier than 8.2.0 
to a more recent version,
the option of reducing the size of the configuration file is an option.
The goal is to identify and use only
the configuration variable values that differ from 
the compile-time default values.  
This is facilitated by using \Condor{config\_val} with 
the \OptArg{-writeconfig:upgrade} argument,
to create a file that behaves the same as the current configuration, 
but is much smaller,
because values matching the default values (as well as some obsolete variables)
have been removed.
Items in the file created by running
\Condor{config\_val} with the \OptArg{-writeconfig:upgrade} argument
will be in the order that they were read from the original configuration files.
This file is a convenient guide to stripping the cruft from 
old configuration files.

When upgrading from a version of HTCondor earlier than 6.8 to more recent version,
note that the configuration settings must be modified for security reasons.
Specifically, the \Macro{HOSTALLOW\_WRITE} configuration variable
must be explicitly changed,
or no jobs can be submitted,
and error messages will be issued by HTCondor tools.

Another way to upgrade leaves HTCondor running. 
HTCondor will automatically restart itself if the \Condor{master} binary
is updated,
and this method takes advantage of this. 
Download the newer version, placing it such that it does not 
overwrite the currently running version.
With the download will be a new set of configuration files;
update this new set with any specializations implemented in the currently
running version of HTCondor.
Then, modify the currently running installation by changing its
configuration such that the path to binaries points instead
to the new binaries.
One way to do that (under Unix) is to use a symbolic link that points 
to the current HTCondor installation directory (for example, \File{/opt/condor}).
Change the symbolic link to point to the new directory. 
If HTCondor is configured to locate its binaries via the symbolic link, 
then after the symbolic link changes, 
the \Condor{master} daemon notices the new binaries and restarts itself. 
How frequently it checks is controlled by the configuration variable 
\Macro{MASTER\_CHECK\_NEW\_EXEC\_INTERVAL}, which defaults 5 minutes.

When the \Condor{master} notices new binaries, 
it begins a graceful restart. 
On an execute machine, 
a graceful restart means that running jobs are preempted. 
Standard universe jobs will attempt to take a checkpoint. 
This could be a bottleneck if all machines in a large pool 
attempt to do this at the same time. 
If they do not complete within the cutoff time specified by the \MacroNI{KILL} 
policy expression (defaults to 10 minutes), 
then the jobs are killed without producing a checkpoint. 
It may be appropriate to increase this cutoff time, 
and a better approach may be to upgrade the pool in stages 
rather than all at once. 

For universes other than the standard universe, jobs are preempted. 
If jobs have been guaranteed a certain amount of uninterrupted run time 
with \MacroNI{MaxJobRetirementTime}, 
then the job is not killed until the specified amount of 
retirement time has been exceeded (which is 0 by default). 
The first step of killing the job is a soft kill signal, 
which can be intercepted by the job so that it can exit gracefully, 
perhaps saving its state. 
If the job has not gone away once the \MacroNI{KILL} expression fires 
(10 minutes by default), 
then the job is forcibly hard-killed. 
Since the graceful shutdown of jobs may rely on shared resources such as disks 
where state is saved, 
the same reasoning applies as for the standard universe: 
it may be appropriate to increase the cutoff time 
for large pools, 
and a better approach may be to upgrade the pool in stages 
to avoid jobs running out of time. 

Another time limit to be aware of is the configuration variable 
\MacroNI{SHUTDOWN\_GRACEFUL\_TIMEOUT}. 
This defaults to 30 minutes. 
If the graceful restart is not completed within this time, 
a fast restart ensues. 
This causes jobs to be hard-killed. 

%%%%%%%%%%%%%%%%%%%%%%%%%%%%%%%%%%%%%%%%%%%%%%%%%%%%%%%%%%%%%%%%%%%%%%
\subsection{\label{sec:Pool-Shutdown-and-Restart}
Shutting Down and Restarting an HTCondor Pool}
%%%%%%%%%%%%%%%%%%%%%%%%%%%%%%%%%%%%%%%%%%%%%%%%%%%%%%%%%%%%%%%%%%%%%%
\index{pool management!shutting down HTCondor}
\index{pool management!restarting HTCondor}

All of the commands described in this section are subject to the
security policy chosen for the HTCondor pool.
As such, the commands must be either run from a
machine that has the proper authorization, 
or run by a user that is authorized to issue the commands.
Section~\ref{sec:Security} details the implementation of 
security in HTCondor.

\begin{description}
\item[Shutting Down HTCondor]
There are a variety of ways to shut down all or parts of an HTCondor pool.
All utilize the \Condor{off} tool.

To stop a single execute machine from running jobs,
the \Condor{off} command specifies the machine by host name.
\begin{verbatim}
  condor_off -startd <hostname>
\end{verbatim}
A running \SubmitCmdNI{standard} universe job will be allowed to 
take a checkpoint before the job is killed.
A running job under another universe will be killed.
If it is instead desired that the machine stops running jobs
only after the currently executing job completes, the command is
\begin{verbatim}
  condor_off -startd -peaceful <hostname>
\end{verbatim}
Note that this waits indefinitely for the running job to finish,
before the \Condor{startd} daemon exits.

Th shut down all execution machines within the pool,
\begin{verbatim}
  condor_off -all -startd
\end{verbatim}

To wait indefinitely for each machine in the pool to finish its current
HTCondor job,
shutting down all of the execute machines as they no longer
have a running job,
\begin{verbatim}
  condor_off -all -startd -peaceful
\end{verbatim}

To shut down HTCondor on a machine from which jobs are submitted,
\begin{verbatim}
  condor_off -schedd <hostname>
\end{verbatim}

If it is instead desired that the submit machine shuts down
only after all jobs that are currently in the queue are finished,
first disable new submissions to the queue 
by setting the configuration variable
\begin{verbatim}
  MAX_JOBS_SUBMITTED = 0
\end{verbatim}
See instructions below in section~\ref{sec:Reconfigure-Pool} for how
to reconfigure a pool.
After the reconfiguration, the command to wait for all jobs to complete
and shut down the submission of jobs is
\begin{verbatim}
  condor_off -schedd -peaceful <hostname>
\end{verbatim}

Substitute the option \Opt{-all} for the host name,
if all submit machines in the pool are to be shut down.

\item[Restarting HTCondor, If HTCondor Daemons Are Not Running]
If HTCondor is not running,
perhaps because one of the \Condor{off} commands was used,
then starting HTCondor daemons back up depends on which part of
HTCondor is currently not running.

If no HTCondor daemons are running, then starting HTCondor is a matter
of executing the \Condor{master} daemon.
The \Condor{master} daemon will then invoke all other specified daemons
on that machine.
The \Condor{master} daemon executes on every machine that is to run HTCondor.

If a specific daemon needs to be started up, and the \Condor{master} daemon
is already running, then issue the command on the specific machine with
\begin{verbatim}
  condor_on -subsystem <subsystemname>
\end{verbatim}
where \verb@<subsystemname>@ is replaced by the daemon's subsystem
name.
Or, this command might be issued from another machine in the pool
(which has administrative authority) with
\begin{verbatim}
  condor_on <hostname> -subsystem <subsystemname>
\end{verbatim}
where \verb@<subsystemname>@ is replaced by the daemon's subsystem
name, and \verb@<hostname>@ is replaced by the host name of the
machine where this \Condor{on} command is to be directed.

\item[Restarting HTCondor, If HTCondor Daemons Are Running]
If HTCondor daemons are currently running, but need to be killed and
newly invoked,
the \Condor{restart} tool does this.
This would be the case for a new value of a configuration variable for
which using \Condor{reconfig} is inadequate.

To restart all daemons on all machines in the pool,
\begin{verbatim}
  condor_restart -all
\end{verbatim}

To restart all daemons on a single machine in the pool,
\begin{verbatim}
  condor_restart <hostname>
\end{verbatim}
where \verb@<hostname>@ is replaced by the host name of the
machine to be restarted.

\end{description}

%%%%%%%%%%%%%%%%%%%%%%%%%%%%%%%%%%%%%%%%%%%%%%%%%%%%%%%%%%%%%%%%%%%%%%
\subsection{\label{sec:Reconfigure-Pool}Reconfiguring an HTCondor Pool}
%%%%%%%%%%%%%%%%%%%%%%%%%%%%%%%%%%%%%%%%%%%%%%%%%%%%%%%%%%%%%%%%%%%%%%
\index{pool management!reconfiguration}

To change a global configuration variable and have all the
machines start to use the new setting, change the value within the file,
and send a \Condor{reconfig} command to each host.
Do this with a \emph{single} command,
\begin{verbatim}
  condor_reconfig -all
\end{verbatim}

If the global configuration file is not shared among all the machines,
as it will be if using a shared file system,
the change must be made to each copy of the global configuration file
before issuing the \Condor{reconfig} command.

Issuing a \Condor{reconfig} command is inadequate for some
configuration variables.
For those, a restart of HTCondor is required.
Those configuration variables that require a restart are listed in
section~\ref{sec:Macros-Requiring-Restart}.
The manual page for \Condor{restart} is at
~\pageref{man-condor-restart}.

