%%%%%%%%%%%%%%%%%%%%%%%%%%%%%%%%%%%%%%%%%%%%%%%%%%%%%%%%%%%%%%%%%%%%%%
\section{Installation of Condor}
\label{sec:install}
%%%%%%%%%%%%%%%%%%%%%%%%%%%%%%%%%%%%%%%%%%%%%%%%%%%%%%%%%%%%%%%%%%%%%%

This section contains the instructions for installing Condor at your
site.  Condor's installation will setup a default configuration which
you can then learn how to customize in the sections which follow.

Please read the the copyright and disclaimer information in 
section~\ref{sec:copyright-and-disclaimer} on
page~\pageref{sec:copyright-and-disclaimer} of the manual, or in the
file LICENSE.TXT, before proceeding.  Installation and
use of Condor is acknowledgement that you have read and agreed to these
terms.

The Condor binary distribution is packaged in the following 5 files
and 2 directories:

\begin{description}
\item[DOC] file containing directions for where to find the 
		  Condor documentation
\item[INSTALL] these installation directions
\item[LICENSE.TXT] by installing Condor, you agree to the contents of
		  the LICENSE.TXT file
\item[README] general info
\item[condor\_install] Perl script to install and configure Condor
\item[examples] directory containing C, Fortran and C++ example
		  programs to run with Condor
\item[release.tar] tar file of the 'release directory', which contains
		  the Condor binaries and libraries
\end{description}

Before you install, please consider joining the condor-world mailing
list.  Traffic on this list is kept to an absolute minimum.  It is only
used to announce new releases of Condor.  To subscribe, send a message
to \Email{majordomo@cs.wisc.edu} with the body:
\begin{verbatim}
   subscribe condor-world 
\end{verbatim}

\subsection{Preparing to Install Condor}

Before you install Condor at your site, there are a few important
decisions you must make about the basic layout of your pool.  These
are:

\begin{enumerate}
\item What machine will be the Central Manager?
\item Will Condor run as root or not?
\item Who will be administering Condor on the machines in your pool?
\item Will you have a 'condor' user and will it's home directory be
   shared? 
\item Where should the machine-specific directories for Condor go?
\item Where should the parts of the Condor system be installed? 
	\begin{itemize}
	\item Config Files
	\item Release directory
		\begin{itemize}
		\item User Binaries
		\item System Binaries 
		\item Lib Directory
	  	\item Etc Directory
		\end{itemize}
	\item Documentation
	\end{itemize}
\item Am I using AFS?
\item Do I have enough disk space for Condor?
\end{enumerate}

If you feel you already know the answers to these questions, you can
skip to the 'Installation Procedure' section below, section~\ref{install:procedure}.
If you are unsure about any of them, read on.

\subsubsection{What machine will be the Central Manager?}

One machine in your pool must be the Central Manager.  You should
setup and install Condor on this machine first.  This is the
centralized information repository for the Condor pool and is also the
machine that does match-making between available machines and waiting
jobs.  If the Central Manager machine crashes, any currently active
matches in the system will keep running, but no new matches will be
made.  Moreover, most Condor tools will stop working.  Because of the
importance of this machine for the proper functioning of Condor, we
recommend you install it on a machine that is likely to stay up all the
time, or at the very least, one that will be rebooted quickly if it
does crash.  Also, because all the daemons will send updates (by
default every 5 minutes) to this machine, it is advisable to consider
network traffic and your network layout when choosing your central
manager.

\subsubsection{Will Condor run as root or not?}

We strongly recommend that you start up the Condor daemons as root.
Otherwise, Condor can do very little to enforce security and policy
decisions.  If you don't have root access and would like to install
Condor, under most platforms you can run Condor under any user you'd
like.  However, there are serious security consequences of this.
Please see section~\ref{sec:Non-Root} on page~\pageref{sec:Non-Root}
in the manual for details on running Condor as non-root.

\subsubsection{Who will be administering Condor on the machines in your pool?}

Either root will be administering Condor directly, or someone else
would be acting as the Condor administrator.  If root has delegated
the responsibility to another person but doesn't want to grant that
person root access, root can specify a condor\_config.root file that
will override settings in the other condor config files.  This way,
the global condor\_config file can be owned and controlled by whoever
is condor-admin, and the condor\_config.root can be owned and
controlled only by root.  Settings that would compromise root security
(such as which binaries are started as root) can be specified in the
condor\_config.root file while other settings that only control policy
or condor-specific settings can still be controlled without root
access.  

\subsubsection{Will you have a 'condor' user and will it's home directory be
   shared? }

To simplify installation of Condor at your site, we recommend that you
create a 'condor' user on all machines in your pool.  The condor
daemons will create files (such as the log files) owned by this user,
and the home directory can be used to specify the location of files
and directories needed by Condor.  The home directory of this user can
either be shared among all machines in your pool, or could be a
separate home directory on the local partition of each machine.  Both
approaches have advantages and disadvantages.  Having the directories
centralized can make administration easier, but also concentrates the
resource usage such that you potentially need a lot of space for a
single shared home directory.  See the section below on
machine-specific directories for more details.

If you choose not to create a condor user, you must specify via the
CONDOR\_IDS environment variable which uid.gid pair should be used for
the ownership of various Condor files.  See section~\ref{sec:uids} on
``UIDs in Condor'' on page~\pageref{sec:uids} in the Administrator's
Manual for details.

\subsubsection{Where should the machine-specific directories for Condor go?}

Condor needs a few directories that are unique on every machine in
your pool.  These are 'spool', 'log', and 'execute'.  Generally, all
three are subdirectories of a single machine specific directory called
the 'local directory' (specified by the \Macro{LOCAL\_DIR} parameter
in the config file).

If you have a 'condor' user with a local home directory on each
machine, the \Macro{LOCAL\_DIR} could just be user condor's home
directory ('\Macro{LOCAL\_DIR} = \MacroU{TILDE}' in the config file).
If this user's home directory is shared among all machines in your
pool, you would want to create a directory for each host (named by
hostname) for the local directory ('\Macro{LOCAL\_DIR} =
\MacroU{TILDE}/hosts/\MacroU{HOSTNAME}' for example).  If you don't
have a condor account on your machines, you can put these directories
wherever you'd like.  However, where to place them will require some
thought, as each one has its own resource needs:

\begin{description}
\item[execute] This is the directory that acts as the current working
directory for any Condor jobs that run on a given execute machine.
The binary for the remote job is copied into this directory, so you
must have enough space for that.  (Condor won't send a job to a
machine that doesn't have enough disk space to hold the initial
binary).  In addition, if the remote job dumps core for some reason,
it is first dumped to the execute directory before it is sent back to
the submit machine.  So, you will want to put the execute directory on
a partition with enough space to hold a possible core file from the
jobs submitted to your pool.

\item[spool] The spool directory holds the job queue and history
files, and the checkpoint files for all jobs submitted from a given
machine.  As a result, disk space requirements for spool can be quite
large, particularly if users are submitting jobs with very large
executables or image sizes.  By using a checkpoint server (see
section~\ref{sec:Ckpt-Server} on ``Installing a Checkpoint Server'' on
page~\pageref{sec:Ckpt-Server} for details), you can ease the disk
space requirements, since all checkpoint files are stored on the
server instead of the spool directories for each machine.  However,
the initial checkpoint files (the executables for all the clusters you
submit) are still stored in the spool directory, so you will need some
space, even with a checkpoint server.

\item[log] Each Condor daemon writes its own log file which is placed
in the log directory.  You can specify what size you want these files
to grow to before they are rotated, so the disk space requirements of
the log directory are configurable.  The larger the logs, the more
historical information they will hold if there's a problem, but the
more disk space they use up.  If you have a network filesystem
installed at your pool, you might want to place the log directories in
a shared location (such as \File{/usr/local/condor/logs/\$(HOSTNAME)})
so that you can view the log files from all your machines in a single
location.  However, if you take this approach, you will have to
specify a local partition for the lock directory (see below).

\item[lock] Condor uses a small number of lock files to synchronize
access to some files that are shared between multiple daemons.
Because of problems we've had with file locking and network
filesystems (particularly NFS), these lock files should be placed on a
local partition on each machine.  By default, they are just placed in
the log directory.  If you place your log directory on a network
filesystem partition, you should specify a local partition for the
lock files with the 'LOCK' parameter in the config file (such as
\File{/var/lock/condor}).

\end{description}

Generally speaking, we recommend that you do not put these directories
(except lock) on the same partition as \File{/var}, since if the partition
fills up, you will fill up \File{/var} as well, which will cause lots of
problems for your machines.  Ideally, you'd have a separate partition
for the Condor directories that the only consequence of filling up
would be Condor's malfunction, not your whole machine.

\subsubsection{Where should the parts of the Condor system be installed?}
	\begin{itemize}
	\item Config Files
	\item Release directory
		\begin{itemize}
		\item User Binaries
		\item System Binaries 
		\item Lib Directory
	  	\item Etc Directory
		\end{itemize}
	\item Documentation
	\end{itemize}

\begin{description}
\item[Config Files] There are a number of config files that allow you
different levels of control over how Condor is configured at each
machine in your pool.  In general, you will have 1 global
configuration file for each platform.  In addition, there is a local
config file for each machine, where you can override settings in the
global file.  This allows you to have different daemons running,
different policies for when to start and stop Condor jobs, and so on.
Beginning with Condor version 6.0.1, you can use a single config file
which is shared among all platforms in your pool, and have both
platform-specific and machine-specific files.  See
section~\ref{sec:Multiple-Platforms} on
page~\pageref{sec:Multiple-Platforms} about ``Configuring Condor for
Multiple Platforms'' for details.

In addition, because we recommend that you start the Condor daemons as
root, we allow you to create config files that are owned and
controlled by root that will override any other condor settings.  This
way, if the condor administrator isn't root, the regular condor config
files can be owned and writable by condor-admin, but root doesn't have
to grant root access to this person.  See
section~\ref{sec:Root-Config} on page~\pageref{sec:Root-Config} in the
manual for a detailed discussion of the root config files, if you
should use them, and what settings should be in them.

In general, there are a number of places that condor will look to find
its config files.  The first file it looks for is the global config
file.  These locations are searched in order until a config file is
found.  If none contain a valid config file, Condor will print an
error message and exit:
\begin{enumerate}
   \item File specified in \Env{CONDOR\_CONFIG} environment variable
   \item \File{/etc/condor/condor\_config}
   \item \File{\Tilde condor/condor\_config}
\end{enumerate}

Next, Condor tries to load the machine-specific, or local config file.
The only way to specify the local config file is in the global config
file, with the \Macro{LOCAL\_CONFIG\_FILE} macro.  If that macro isn't
set, no local config file is used.  Beginning with Condor version
6.0.1, this macro can be a list of files instead of a single file.

The root config files come in last.  The global file is searched for
in the following places:
\begin{enumerate}
   \item \File{/etc/condor/condor\_config.root}
   \item \File{\Tilde condor/condor\_config.root}
\end{enumerate}

The local root config file is found with the
\Macro{LOCAL\_ROOT\_CONFIG\_FILE} macro.  If that isn't set, no local
root config file is used.  Beginning with Condor version 6.0.1, this
macro can also be a list of files instead of a single file.

\item[Release Directory]

Every binary distribution contains a 'release.tar' file that contains
four subdirectories: 'bin', 'etc', 'lib' and 'sbin'.  Wherever you
choose to install these 4 directories we call the 'release directory'
(specified by the 'RELEASE\_DIR' parameter in the config file).  Each
release directory contains platform dependent binaries and libraries,
so you will need to install a separate one for each kind of machine in
your pool.

\begin{itemize}
     \item User Binaries:

     All of the files in the 'bin' directory are programs the end
     Condor users should expect to have in their path.  You could
     either put them in a well known location (such as
     \File{/usr/local/condor/bin}) which you have Condor users add to
     their \Env{PATH} environment variable, or copy those files
     directly into a well known place already in user's PATHs (such as
     \File{/usr/local/bin}).  With the above examples, you could also
     leave the binaries in \File{/usr/local/condor/bin} and put in
     soft links from \File{/usr/local/bin} to point to each program.

     \item System Binaries:

     All of the files in the 'sbin' directory are Condor daemons and
     agents, or programs that only the Condor administrator would need
     to run.  Therefore, we recommend that you only add these programs
     to the \Env{PATH} of the Condor administrator.

     \item Lib Directory:

     The files in the 'lib' directory are the condor libraries that
     must be linked in with user jobs for all of Condor's
     checkpointing and migration features to be used.  'lib' also
     contains scripts used by the \Condor{compile} program to help
     relink jobs with the condor libraries.  These files should be
     placed in a location that is world-readable, but they do not need
     to be placed in anyone's PATH.  The \Condor{compile} script checks
     the config file for the location of the lib directory.

     \item Etc Directory:

     'etc' contains an 'examples' subdirectory which holds various
     example config files and other files used for installing Condor.
     'etc' is the recommended location to keep the master copy of your
     config files.  You can put in soft links from one of the places
     mentioned above that Condor checks automatically to find it's
     global config file. 
\end{itemize}

\item[Documentation]

The documentation provided with Condor is currently only available in
HTML, Postscript and PDF (Adobe Acrobat).  It can be locally installed
wherever is customary at your site.  You can also find the Condor
documentation on the web at:
\Url{http://www.cs.wisc.edu/condor/manual}.

\end{description}

\subsubsection{Am I using AFS?}

If you are using AFS at your site, be sure to read the
section~\ref{sec:Condor-AFS} on page~\pageref{sec:Condor-AFS} in the
manual.  Condor does not currently have a way to authenticate itself
to AFS.  We're working on a solution, it's just not ready for version
6.0.  So, what this means is that you are probably not going to want
to have the \Macro{LOCAL\_DIR} for Condor on AFS.  However, you can
(and probably should) have the Condor \Macro{RELEASE\_DIR} on AFS, so
that you can share one copy of those files and upgrade them in a
centralized location.  You will also have to do something special if
you submit jobs to Condor from a directory on AFS.  Again, read manual
section~\ref{sec:Condor-AFS} for all the gory details.

\subsubsection{Do I have enough disk space for Condor?}

The Condor release directory takes up a fair amount of space.  This is
another reason why it's a good idea to have it on a shared
filesystem.  The rough size requirements for the release
directory on various platforms are listed in table~\ref{install-sizes}.

\begin{center}
\begin{table}
\begin{tabular}{|ll|} \hline
\emph{Platform} 	&	\emph{Size}		\\	\hline \hline
	  Intel/Linux   &	11   megs (statically linked) \\
	  Intel/Linux   &	6.5 megs (dynamically linked) \\
	  Intel/Solaris &	8   megs \\
	  Sparc/Solaris &	10   megs \\
	  SGI/IRIX	    &	17.5 megs \\
	  Alpha/Digital Unix & 	15.5 megs \\ \hline
\end{tabular}
\caption{\label{install-sizes}Release Directory Size Requirements}
\end{table}
\end{center}

In addition, you will need a lot of disk space in the local directory
of any machines that are submitting jobs to Condor.  See question 5
above for details on this.


\subsection{Installation Procedure}
\label{install:procedure}

IF YOU HAVE DECIDED TO CREATE A 'condor' USER AND GROUP, YOU SHOULD DO
THAT ON ALL YOUR MACHINES BEFORE YOU DO ANYTHING ELSE.

The easiest way to install Condor is to use one or both of the scripts
provided to help you: \Condor{install} and \Condor{init}.  You should
run these scripts as the user that you are going to run the Condor
daemons as.  First, run \Condor{install} on the machine that will be a
fileserver for shared files used by Condor, such as the release
directory, and possibly the condor user's home directory.  When you
do, choose the ``full-install'' option in step \#1 described below.

Once you have run \Condor{install} on a file server to setup your
release directory and configure Condor for your site, you should run
\Condor{init} on any other machines in your pool to create any locally
used files that aren't created by \Condor{install}.  In the most
simple case, where nearly all of Condor is installed on a shared file
system, even though Condor{install} will create nearly all the files
and directories you need, you will still need to use \Condor{init} to
create the \Macro{LOCK} directory on the local disk of each machine.
If you have a shared release directory, but the \Macro{LOCAL\_DIR} is
local on each machine, \Condor{init} will create all the directories
and files needed in \Macro{LOCAL\_DIR}.  In addition, \Condor{init}
will create any soft links on each machine that are needed so that
Condor can find its global config file.

If you don't have a shared filesystem, you will need to run
\Condor{install} on each machine in your pool to setup Condor.  In
this case, there is no need to run \Condor{init} at all.

In addition, you will want to run \Condor{install} on your central
manager machine if that machine is different from your file server,
using the ``central-manager'' option in step \#1 described below.  Run
\Condor{install} on your file server first, then on your central
manager.  If this step fails for some reason (NFS permissions, etc),
you can do it manually quite easily.  All this does is copy the
\File{condor\_config.local.central.manager} file from
\Release{etc/examples} to the proper location for the local config
file of your central manager machine.  If your central manager is an
Alpha or an SGI, you might want to add ``KBDD'' to the
\Macro{DAEMON\_LIST} parameter.  See
section~\ref{sec:Configuring-Condor} ``Configuring Condor'' on
page~\pageref{sec:Configuring-Condor} of the manual for details.

\Condor{install} assumes you have perl installed in \File{/usr/bin/perl}.  If
this is not the case, you can either edit the script to put the right
path in, or you will have to invoke perl directly from your shell
(assuming perl is in your PATH):
\begin{verbatim}
% perl condor_install
\end{verbatim}

\Condor{install} breaks down the installation procedure into various
steps.  Each step is clearly numbered.  The following section explains
what each step is for, and suggests how to answer the questions
\Condor{install} will ask you for each one.

\subsubsection{\Condor{install}, step-by-step}

\begin{description}
\item[STEP 1: What type of Condor installation do you want?]

     There are three types of Condor installation you might choose:
     'submit-only', 'full-install', and 'central-manager'.  A
     submit-only machine can submit jobs to a Condor pool, but Condor
     jobs will not run on it.  A full-install machine can both submit
     and run Condor jobs.  

     If you are planning to run Condor jobs on your machines, you
     should either install and run Condor as root, or as user
     'condor'.  

     If you are planning to setup a submit only machine, you can
     either install Condor machine-wide as root or user 'condor', or,
     you can install Condor as yourself into your home directory.

     The other possible installation type is setting up a machine as a
     central manager.  If you do a full-install and you say that you
     want the local host to be your central manager, this step will be
     done automatically.  You should only choose the central-manager
     option at step 1 if you have already run \Condor{install} on your
     file server and you now want to run \Condor{install} on a different
     machine that will be your central manager.

\item[STEP 2: How many machines are you setting up this way?]

     If you are installing Condor for multiple machines, and you have
     a shared file system, \Condor{install} will prompt you for the
     hostnames of each machine you want to add to your Condor pool.
     If you don't have a shared file system, you will have to run
     \Condor{install} locally on each machine, anyway, so it doesn't
     bother asking you for the names.  If you provide a list, it will
     use the names to automatically create directories and files
     later.  At the end, \Condor{install} will dump out this list to a
     'roster' file which can be used by scripts to help maintain your
     Condor pool.

     If you are only installing Condor on 1 machine, you would just
     answer 'no' to the first question, and move on.

\item[STEP 3: Install the Condor release directory] 
     The release directory contains four subdirectories: 'bin', 'etc',
     'lib' and 'sbin'.  bin contains user-level executable programs.
     etc is the recommended location for your Condor config files, and
     also includes an 'examples' directory with default config files
     and other default files used for installing condor.  lib contains
     libraries to link condor user programs and scripts used by the
     Condor system.  sbin contains all administrative executable
     programs and the Condor daemons.  

     If you have multiple machines with a shared filesystem that will
     be running Condor, you should put the release directory on that
     shared filesystem so you only have one copy of all the binaries,
     and so that when you update them, you can do so in one place.
     Note that the release directory is architecture dependent, so you
     will need to download separate binary distributions for every
     platform in your pool.

     \Condor{install} tries to find an already installed release
     directory.  If it can't find one, it asks if you have installed
     one already.  If you have not installed one, it tries to do so
     for you by untarring the release.tar file from the binary
     distribution.  

\underline{NOTE}: If you are only setting up a central manager (you chose 'central
     manager' in step 1) step 3 is the last question you will need to
     answer.

\item[STEP 4: How and where should Condor send email if things go wrong?]

     Various parts of Condor will send email to a condor administrator
     if something goes wrong that needs human attention.  You will
     need to specify the email address of this administrator.  

     You will also need to specify the full path to a mail program
     that Condor will use to send the email.  This program needs to
     understand the '-s' option, which is how you specify a subject
     for the outgoing message.  The default on most platforms will
     probably be correct.  On Linux machines, since there is such
     variation in Linux distributions and installations, you should
     verify that the default works.  If the script complains that it
     cannot find the mail program that was specified, you can try
     'which mail' from your shell prompt to see what 'mail' program is
     currently in your PATH.  If there is none, try 'which mailx'.  If
     you still can't find anything, ask your system administrator.
     You should verify that the program you end up using supports
     '-s'.  The man page for that program will probably tell you.

\item[STEP 5: Where should public programs be installed?]

     It is recommended that you install the user-level condor programs
     in the release directory, (where they go by default).  This way,
     when you want to install a new version of the Condor binaries,
     you can just replace your release directory and everything will
     be updated at once.  So, one option is to have Condor users add
     \Release{bin} to their PATH, so that they can access the
     programs.  However, we recommend putting in soft links from some
     directory already in their PATH (such as \File{/usr/local/bin}) that
     point back to the Condor user programs.  \Condor{install} will do
     this for you, all you have to do is tell it what directory to put
     these links into.  This way, users don't have to change their
     PATH to use Condor but you can still have the binaries installed
     in their own location.

     If you are installing Condor as neither root nor condor, there is
     a perl script wrapper to all the Condor tools that is created
     which sets some appropriate environment variables and
     automatically passes certain options to the tools.  This is all
     created automatically by \Condor{install}.  So, you need to tell
     \Condor{install} where to put this perl script.  The script itself
     is linked to itself with many different names, since it is the
     name that determines the behavior of the script.  This script
     should go somewhere that is in your PATH already, if possible
     (such as \Tilde bin).

\end{description}

At this point, the remaining steps are different depending on what
kind of installation you are doing.  Skip to the appropriate section
depending on what kind of installation you selected in STEP 1 above.

\subsubsection{Full Install}

\begin{description}

\item[STEP 6: What machine will be your central manager?]

     Simply type in the full hostname of the machine you have chosen
     for your central manager.  If \Condor{install} can't find
     information about the host you typed by querying your nameserver,
     it will print out an error message and ask you to confirm.


\item[STEP 7: Where will the 'local directory' go?]

     This is the directory discussed in question \#5 from the
     introduction.  \Condor{install} tries to make some educated guesses
     as to what directory you want to use for the purpose.  Simply
     agree to the correct guess, or (when \Condor{install} has run out
     of guesses) type in what you want.  Since this directory needs to
     be unique, it is common to use the hostname of each machine in
     its name.  When typing in your own path, you can use
     '\MacroU{HOSTNAME}' which \Condor{install} (and the Condor config files)
     will expand to the hostname of the machine you are currently on.
     \Condor{install} will try to create the corresponding directories
     for all the machines you told it about in STEP 2 above.

     Once you have selected the local directory, \Condor{install}
     creates all the needed subdirectories of each one with the proper
     permissions.  They should have the following permissions and
     ownerships:

\begin{verbatim}
     drwxr-xr-x   2 condor   root         1024 Mar  6 01:30 execute/
     drwxr-xr-x   2 condor   root         1024 Mar  6 01:30 log/
     drwxr-xr-x   2 condor   root         1024 Mar  6 01:30 spool/
\end{verbatim}

     If your local directory is on a shared file system,
     \Condor{install} will prompt you for the location of your lock
     files, as discussed in question \#5 above.  In this case, when
     \Condor{install} is finished, you will have to run \Condor{init} on
     each machine in your pool to create the lock directory before you
     can start up Condor.


\item[STEP 8: Where will the local (machine-specific) config files go?]

     As discussed in question \#6 above, there are a few different
     levels of Condor config file.  There's the global config file
     that will be installed in \Release{etc/condor\_config}, and
     there are machine-specific, or local config files that override
     the settings in the global file.  If you are installing on
     multiple machines or are configuring your central manager
     machine, you must select a location for your local config files. 

     The two main options are to have a single directory that holds
     all the local config files, each one named '\MacroU{HOSTNAME}.local',
     or to have the local config files go into the individual local
     directories for each machine.  Given a shared filesystem, we
     recommend the first option, since it makes it easier to configure
     your pool from a centralized location.


\item[STEP 9: How do you want Condor to find its config file?]

     Since there are a few known places Condor looks to find your
     config file, we recommend that you put a soft link from one of
     them to point to \Release{etc/condor\_config}.  This way, you
     can keep your Condor configuration in a centralized location, but
     all the Condor daemons and tools will be able to find their
     config files.  Alternatively, you can set the CONDOR\_CONFIG
     environment variable to contain \Release{etc/condor\_config}.

     \Condor{install} will ask you if you want to create a soft link
     from either of the two fixed locations that Condor searches.

\end{description}

Once you have completed STEP 9, you're done.  \Condor{install} prints
out a messages describing what to do next.  Please skip to 
section~\ref{installed-now-what}.

\subsubsection{Submit Only}
\Todo

\subsection{Condor is installed... now what?}
\label{installed-now-what}

Now that Condor has been installed on your machine(s), there are a few
things you should check before you start up Condor.

\begin{enumerate}
\item Read through the \Release{etc/condor\_config} file.  There are a
    lot of possible settings and you should at least take a look at
    the first two main sections to make sure everything looks okay.
    In particular, you might want to setup host/ip based security for
    Condor.  See the section~\ref{sec:Host-Security} on
    page~\pageref{sec:Host-Security} in the manual to learn how to do
    this.

\item Condor can monitor the activity of your mouse and keyboard,
    provided that you tell it where to look.  You do this with the
    \Macro{CONSOLE\_DEVICES} entry in the \condor{startd} section of
    the config file.  On most platforms, we provide reasonable
    defaults.  For example, the default device for the mouse on Linux
    is 'mouse', since most Linux installations have a soft link from
    '\File{/dev/mouse}' that points to the right device (such as
    \File{tty00} if you have a serial mouse, \File{psaux} if you have
    a PS/2 bus mouse, etc).  If you don't have a \File{/dev/mouse}
    link, you should either create one (you'll be glad you did), or
    change the \Macro{CONSOLE\_DEVICES} entry in Condor's config file.
    This entry is just a comma seperated list, so you can have any
    devices in \File{/dev} count as 'console devices' and activity
    will be reported in the \condor{startd}'s classad as
    \AdAttr{ConsoleIdleTime}.

\item  (Linux only) Condor needs to be able to find the 'utmp' file.
    According to the Linux File System Standard, this file should be
    \File{/var/run/utmp}.  If Condor can't find it there, it looks in
    \File{/var/adm/utmp}.  If it still can't find it, it gives up.  So, if
    your Linux distribution puts this file somewhere else, be sure to
    put a soft link from \File{/var/run/utmp} to point to the real location.

\end{enumerate}

\subsection{Starting up the Condor daemons}

To start up the Condor daemons, all you need to do is execute
\Release{sbin/condor\_master}.  This is the Condor master, whose
only job in life is to make sure the other Condor daemons are running.
The master keeps track of the daemons, restarts them if they crash,
and periodically checks to see if you have installed new binaries (and
if so, restarts the affected daemons).

If you're setting up your own pool, you should start Condor on your
central manager machine first.  If you have done a submit-only
installation and are adding machines to an existing pool, it doesn't
matter what order to start them in.

To ensure that Condor is running, you can run either:
\begin{verbatim}
        ps -ef | egrep condor_
\end{verbatim}
or
\begin{verbatim}
        ps -aux | egrep condor_
\end{verbatim}
depending on your flavor of Unix.  On your central manager machine you
should have processes for:
\begin{itemize}
	\item \condor{master}
	\item \condor{collector}
	\item \condor{negotiator}
	\item \condor{startd}
	\item \condor{schedd}
\end{itemize}
On all other machines in your pool you should have processes for:
\begin{itemize}
	\item \condor{master}
	\item \condor{startd}
	\item \condor{schedd}
\end{itemize}
(\textbf{NOTE}: On Alphas and IRIX machines, there will also be a
	'\Condor{kbdd}' -- see section~\ref{sec:kbdd} on
	page~\pageref{sec:kbdd} of the manual for details.)  If you
	have setup a submit-only machine, you will only see:
\begin{itemize}
	\item \condor{master}
	\item \condor{schedd}
\end{itemize}

Once you're sure the Condor daemons are running, check to make sure
that they are communicating with each other.  You can run
\Condor{status} to get a one line summary of the status of each
machine in your pool.

Once you're sure Condor is working properly, you should add
``condor\_master" into your startup/bootup scripts (i.e. \File{/etc/rc} ) so
that your machine runs \Condor{master} upon bootup.  \Condor{master}
will then fire up the neccesary Condor daemons whenever your machine
is rebooted.  

If your system uses System-V style init scripts, you can look in
\Release{etc/examples/condor.boot} for a script that can be used
to start and stop Condor automatically by init.  Normally, you would
install this script as \File{/etc/init.d/condor} and put in soft link from
various directories (for example, \File{/etc/rc2.d}) that point back to
\File{/etc/init.d/condor}.  The exact location of these scripts and links
will vary on different platforms.

If your system uses BSD style boot scripts, you probably have an
\File{/etc/rc.local} file.  Just add a line in there to start up
\Release{sbin/condor\_master} and you're done.


\subsection{The Condor daemons are running... now what?}

Now that the Condor daemons are running, there are a few things you
can and should do:

\begin{enumerate}
\item (Optional) Do a full install for the \Condor{compile} script.
    \condor{compile} assists in linking jobs with the Condor libraries
    to take advantage of all of Condor's features.  As it is currently
    installed, it will work by placing it in front of any of the
    following commands that you would normally use to link your code:
    gcc, g++, g77, cc, acc, c89, CC, f77, fort77 and ld.  If you
    complete the full install, you will be able to use
    \condor{compile} with any command whatsoever, in particular, make.
    See section~\ref{sec:full-condor-compile} on
    page~\pageref{sec:full-condor-compile} in the manual for
    directions.

\item Try building and submitting some test jobs.  See
    \File{examples/README} for details.

\item If your site uses the AFS network file system, see
section~\ref{sec:Condor-AFS} on page~\pageref{sec:Condor-AFS} in the
manual.

\item We strongly recommend that you start up Condor (i.e. run the
\Condor{master} daemon) as user root.  If you must start Condor as
some user other than root, see section~\ref{sec:Non-Root} on
page~\pageref{sec:Non-Root}.

\end{enumerate}
