%%%      PLEASE RUN A SPELL CHECKER BEFORE COMMITTING YOUR CHANGES!
%%%      PLEASE RUN A SPELL CHECKER BEFORE COMMITTING YOUR CHANGES!
%%%      PLEASE RUN A SPELL CHECKER BEFORE COMMITTING YOUR CHANGES!
%%%      PLEASE RUN A SPELL CHECKER BEFORE COMMITTING YOUR CHANGES!
%%%      PLEASE RUN A SPELL CHECKER BEFORE COMMITTING YOUR CHANGES!

%%%%%%%%%%%%%%%%%%%%%%%%%%%%%%%%%%%%%%%%%%%%%%%%%%%%%%%%%%%%%%%%%%%%%%
\section{\label{sec:History-8-3}Development Release Series 8.3}
%%%%%%%%%%%%%%%%%%%%%%%%%%%%%%%%%%%%%%%%%%%%%%%%%%%%%%%%%%%%%%%%%%%%%%

This is the development release series of HTCondor.
The details of each version are described below.

%%%%%%%%%%%%%%%%%%%%%%%%%%%%%%%%%%%%%%%%%%%%%%%%%%%%%%%%%%%%%%%%%%%%%%
\subsection*{\label{sec:New-8-3-3}Version 8.3.3}
%%%%%%%%%%%%%%%%%%%%%%%%%%%%%%%%%%%%%%%%%%%%%%%%%%%%%%%%%%%%%%%%%%%%%%

\noindent Release Notes:

\begin{itemize}

\item HTCondor version 8.3.3 not yet released.
%\item HTCondor version 8.3.3 released on Month Date, 2014.

\end{itemize}


\noindent New Features:

\begin{itemize}

\item None.

\end{itemize}

\noindent Bugs Fixed:

\begin{itemize}

\item None.

\end{itemize}

%%%%%%%%%%%%%%%%%%%%%%%%%%%%%%%%%%%%%%%%%%%%%%%%%%%%%%%%%%%%%%%%%%%%%%
\subsection*{\label{sec:New-8-3-2}Version 8.3.2}
%%%%%%%%%%%%%%%%%%%%%%%%%%%%%%%%%%%%%%%%%%%%%%%%%%%%%%%%%%%%%%%%%%%%%%

\noindent Release Notes:

\begin{itemize}

\item HTCondor version 8.3.2 not yet released.
%\item HTCondor version 8.3.2 released on Month Date, 2014.

\end{itemize}


\noindent New Features:

\begin{itemize}

\item The new python binding \Procedure{register} has been added 
to allow python functions
be registered with the ClassAds library.  This allows python
functions to be invoked from within ClassAds.
\Ticket{4598}

\item The new python bindings \Procedure{externalRefs} and 
\Procedure{internalRefs} have been added to allow the ClassAd object
to determine internal and external references from an expression.
\Ticket{4681}

\item Added the DAGMan feature of \Arg{ALWAYS-UPDATE} for updates
of a DAGMan node status file. 
Specifying this causes the node status file to be overwritten,
even if no nodes have changed status since the file was last written.
\Ticket{4607}

\item Configuration variable \Macro{MAX\_JOBS\_RUNNING} has been
modified such that it only applies to job universes that require a
\Condor{shadow} process.
Scheduler and local universe jobs are no longer affected by this
variable.
The number of running scheduler and local universe jobs can be controlled
with configuration variables \MacroNI{START\_SCHEDULER\_UNIVERSE} and
\MacroNI{START\_LOCAL\_UNIVERSE}, respectively.
\Ticket{4589}

\item HTCondor now sets environment variable \Env{\_CONDOR\_JOB\_AD} for
scheduler universe jobs. The value is the path to a file which contains
the job ad as it was when the job was started.
This feature already exists for vanilla, parallel, java, and local
universe jobs.
\Ticket{4558}

\item It is now possible run a dual-protocol (IPv4 and IPv6) submit node
submitting to single-protocol execute nodes. This is preliminary work.
\Ticket{4494}

\end{itemize}

\noindent Bugs Fixed:

\begin{itemize}

\item Updated the CREAM client library used in the cream\_gahp.
This fixes the delegation of RFC format proxies, in addition to other
bug fixes.
\Ticket{4596}

\item Fixed a bug that could cause DAGMan to segfault (rather than
printing an appropriate error message) on some DAG file syntax errors.
\Ticket{4616}

\item Fixed a bug that could cause shared port to fail on Mac OS X if
\Macro{LOCK} wasn't explicitly set in the configuration file.
\Ticket{4678}

\item Fixed a bug that caused both \Condor{dagman} and the schedd
to generate commands to remove \Condor{dagman}'s node jobs when
\Condor{dagman} itself is \Condor{rm}'ed.  Now only the schedd
generates the command, avoiding the extra load of running two
identical commands.
\Ticket{4618}

\item Fixed a bug that caused the DAGMan node status file (see
section~\ref{sec:DAG-node-status} for details) to not reflect the
final status of a DAG when the DAG is removed with \Condor{rm} or
aborted by an ABORT-DAG-ON specification.
\Ticket{4686}

\end{itemize}

%%%%%%%%%%%%%%%%%%%%%%%%%%%%%%%%%%%%%%%%%%%%%%%%%%%%%%%%%%%%%%%%%%%%%%
\subsection*{\label{sec:New-8-3-1}Version 8.3.1}
%%%%%%%%%%%%%%%%%%%%%%%%%%%%%%%%%%%%%%%%%%%%%%%%%%%%%%%%%%%%%%%%%%%%%%

\noindent Release Notes:

\begin{itemize}

\item HTCondor version 8.3.1 released on September 11, 2014.

\end{itemize}


\noindent New Features:

\begin{itemize}

\item If cgroups are enabled on Linux platforms, 
the amount of swap space used by a job is now limited to the 
size specified by the machine ClassAd attribute \Attr{VirtualMemory} 
for the slot that the job is running on.
\Ticket{4417}

\item The new configuration variable \Macro{COLLECTOR\_PORT} specifies
the default port used by the \Condor{collector} daemon and command line tools.
The default value is 9618.
This default is the same port as has been used in previous HTCondor versions.
\Ticket{4432}

%\item The \Condor{shared\_port} daemon is now enabled by default 
%for new installations on non-Windows platforms.
%With this change, 
%all HTCondor daemons will share port 9618 for all incoming TCP connections,
%greatly simplifying firewall configuration.  
%Only users of HTCondor-G submitting to Globus GRAM endpoints 
%will need additional incoming TCP ports in the firewall.
%
%Note that this causes \emph{all} HTCondor hosts to use 
%TCP port 9618 by default,
%differing from previous behavior.
%Previously, only the \Condor{collector} host used a fixed port.
%To restore the previous behavior, 
%set the new configuration variable \Macro{SHARED\_PORT\_PORT} to \Expr{0}.
%\Ticket{3813}

\item The \Condor{shared\_port} daemon will now work 
if the default location given by configuration variable
\Macro{DAEMON\_SOCKET\_DIR}, which is \Expr{\$(LOCK)/daemon\_sock},
is longer than 90 characters in length.
On Linux platforms, abstract sockets are now the primary method for
\Condor{shared\_port} to forward an incoming connection to the intended
daemon.
\Ticket{4465}

\item Improvements to CCB increase performance.
\Ticket{4453}

\item The use of a single log file to write events and enforce the 
dependencies of a DAG represented by a \Condor{dagman} instance is mandatory.
To implement this,
the \Opt{-dont\_use\_default\_node\_log} command-line
option to \Condor{submit\_dag} is disabled,
and an attempt to set configuration variable
\Macro{DAGMAN\_ALWAYS\_USE\_NODE\_LOG} to \Expr{False} will generate an
error.
\Ticket{3994}

\item The new \Condor{dagman} configuration variable
\Macro{DAGMAN\_SUPPRESS\_JOB\_LOGS} allows users to prevent DAG node
jobs from writing to the log file specified in their submit description file.
See section~\ref{param:DAGManSuppressJobLogs} for details.
\Ticket{4353}

\item New special variables \Expr{@(OWNER)} and \Expr{@(NODE\_NAME)} are
available when defining configuration variable
\Macro{DAGMAN\_DEFAULT\_NODE\_LOG}.
These values make it easier to avoid log file name collisions.
\Ticket{4334}

\item \Condor{submit} will no longer insert an \Attr{OpSys} requirement
for a job
when one of \Attr{OpSysAndVer}, \Attr{OpSysLongName}, \Attr{OpSysName},
or \Attr{OpSysShortName} is already specified by the user in
the \Attr{Requirements} expression of the submit description file.
\Ticket{4519}

\item The configuration file \Expr{\$(HOME)/.condor/condor\_config}
is no longer considered for the single, initial, global configuration file.
Instead, a user-specific configuration file has been added as the
last file parsed.
The new configuration variable \Macro{USER\_CONFIG\_FILE} may change the
default file name or disable this feature.
Section~\ref{sec:Ordering-Config-File} describes the ordering
in which configuration files are parsed.
\Ticket{3158}

\item Daemons now authenticate many client network connections in
parallel, rather than one at a time.
This improves the scalability of daemons that receive many client
connections, like the \Condor{schedd} and \Condor{collector}.
The improvement is most noticeable when using the FS and GSI
authentication methods.
\Ticket{4137}

\item The GSI security libraries are now loaded into memory only when GSI
authentication is required.
This reduces memory usage when GSI authentication is not used.
The memory reduction will be most noticeable when there are many
\Condor{shadow} processes running.
\Ticket{4483}

\item Implemented fine-grained locking in the HTCondor python module to
allow other python threads to run during HTCondor calls.
\Ticket{4507}

\end{itemize}

\noindent Bugs Fixed:

\begin{itemize}

\item None.

\end{itemize}

%%%%%%%%%%%%%%%%%%%%%%%%%%%%%%%%%%%%%%%%%%%%%%%%%%%%%%%%%%%%%%%%%%%%%%
\subsection*{\label{sec:New-8-3-0}Version 8.3.0}
%%%%%%%%%%%%%%%%%%%%%%%%%%%%%%%%%%%%%%%%%%%%%%%%%%%%%%%%%%%%%%%%%%%%%%

\noindent Release Notes:

\begin{itemize}

\item HTCondor version 8.3.0 released on August 12, 2014.
This release contains all improvements and bug fixes from 
HTCondor version 8.2.2.

\end{itemize}


\noindent New Features:

\begin{itemize}

\item When a daemon creates a child daemon process, it also creates a
security session shared with the child daemon.
This makes the initial communication between the daemons more efficient.
\Ticket{4405}

\item Negotiation cycle performance has been improved, especially
over a wide-area network, by reducing network traffic and latency
between a submit machine and a central manager.
The new configuration variable 
\Macro{NEGOTIATOR\_RESOURCE\_REQUEST\_LIST\_SIZE}
does performance tuning, as defined 
in section~\ref{param:NegotiatorResourceRequestListSize}.
\Ticket{4460}

\item The synchronization of the job event log was improved by only
using \Procedure{fsync} where necessary and 
\Procedure{fdatasync} where sufficient.  
This should provide a small reduction in disk I/O to 
the \Condor{schedd} daemon.
\Ticket{4283}

\item CPU usage by the \Condor{collector} has been reduced when
handling normal queries from \Condor{status},
and CPU usage by the \Condor{schedd} has been reduced when
handling normal queries from \Condor{q}.
\Ticket{4448}

\item HTCondor can now internally cache the result of Globus authorization
callouts.  
The caching behavior is enabled by setting configuration variable
\Macro{GSS\_ASSIST\_GRIDMAP\_CACHE\_EXPIRATION} to a non-zero value.
This feature will be useful for sites that use the Globus authorization
callouts based only on DN and VOMS FQAN, and for sites that have 
performance issues.
\Ticket{4138}

\item The job ClassAd attribute \Attr{DAG\_Status} is included in 
the \File{dagman.out} file.
\Ticket{4381}

\item The new \Opt{-DoRecovery} command line option for \Condor{dagman}
and \Condor{submit\_dag} causes \Condor{dagman} to run in
recovery mode.
\Ticket{2218}

\item The new \Opt{-ads} option to \Condor{status} permits a set of ClassAds
to be read from a file, processing the ClassAds as if they came from
the \Condor{collector}.
\Ticket{4414}

\item Daemon ClassAd hooks implementing Startd Cron functionality  
can now return multiple ClassAds,
and the hooks can specify which ClassAds their output should merge into.
\Ticket{4398}

\item Two new \Condor{schedd} ClassAd statistics attributes are
available: \Attr{JobsRunning} and \Attr{JobsAccumExceptionalBadputTime}.
\Ticket{4409}

\end{itemize}

\noindent Bugs Fixed:

\begin{itemize}

\item Fixed a bug that caused \Condor{dagman} to unnecessarily attempt
to read node job submit description files, 
which could cause spurious warnings when in recovery mode.
Strictly speaking, the bug is fixed only for the
default case in which \MacroNI{DAGMAN\_ALWAYS\_USE\_NODE\_LOG} is set
to \Expr{True}.
\Ticket{3843}

\item Fixed a bug in the \Condor{schedd} daemon that caused the values
of the ClassAd attributes \Attr{JobsRunningSizes} and 
\Attr{ JobsRunningRuntimes} to be much larger than they should have been.
\Ticket{4409}

\end{itemize}

