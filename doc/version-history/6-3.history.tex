%%%%%%%%%%%%%%%%%%%%%%%%%%%%%%%%%%%%%%%%%%%%%%%%%%%%%%%%%%%%%%%%%%%%%%
\section{\label{sec:History-6-3}Development Release Series 6.3}
%%%%%%%%%%%%%%%%%%%%%%%%%%%%%%%%%%%%%%%%%%%%%%%%%%%%%%%%%%%%%%%%%%%%%%

This is the second development release series of Condor.

It contains numerous enhancements over the 6.2 stable series.
For example:

\begin{itemize}

\item Support for Kerberos and X.509 authentication.

\item Support for transferring files needed by jobs (for all universes
except standard and PVM)

\item Support for MPICH jobs.

\item Support for JAVA jobs.

\item 
Condor DAGMan is dramatically more reliable and efficient, and offers
a number of new features.

\end{itemize}

The 6.3 series has many other improvements over the 6.2 series, and
may be available on newer platforms.  The new features, bugs fixed,
and known bugs of each version are described below in detail.


%%%%%%%%%%%%%%%%%%%%%%%%%%%%%%%%%%%%%%%%%%%%%%%%%%%%%%%%%%%%%%%%%%%%%%
\subsection*{\label{sec:New-6-3-3}Version 6.3.3}
%%%%%%%%%%%%%%%%%%%%%%%%%%%%%%%%%%%%%%%%%%%%%%%%%%%%%%%%%%%%%%%%%%%%%%

\noindent New Features:

\begin{itemize}

\item Added support for Kerberos and X.509 authentication in Condor.  

\item Added the ability for vanilla jobs on Unix to use Condor's file
transfer mechanism so that you don't have to rely on a shared file
system.  

\item Added support for MPICH jobs on Windows NT and 2000.

\item Added support for the JAVA universe.

\item When you use \Condor{hold} and \Condor{release}, you now see an
entry about the event in the UserLog file for the job.

\item Whenever a job is removed, put on hold, or released (either by a
Condor user or by the Condor system itself), there is a ``reason''
attribute placed in the job ad and written to the UserLog file.  
If a job is held, \Attr{HoldReason} will be set.
If a job is released, \Attr{ReleaseReason} will be set.
If a job is removed, \Attr{RemoveReason} will be set.
In addition, whenever a job's status changes,
\Attr{EnteredCurrentStatus} will contain the epoch time when the
change took place.

\item The error messages you get from \Condor{rm}, \Condor{hold} and
\Condor{release} have all been updated to be more specific and
accurate. 

\item Condor users can now specify a policy for when their jobs should
leave the queue or be put on hold.
They can specify expressions that are evaluated periodically, and
whenever the job exits.
This policy can be used to ensure that the job remains in the queue
and is re-run until it exits with a certain exit code, that the job
should be put on hold if a certain condition is true, and so on. 
If any of these policy expressions result in the job being removed
from the queue or put on hold, the UserLog entry for the event
includes a string describing why the action was taken.

\item Changed the way Condor finds the various \Condor{shadow} and
\Condor{starter} binaries you have installed on your machine.
Now, you can specify a \Macro{SHADOW\_LIST} and a
\Macro{STARTER\_LIST}.
These are treated much like the \MacroNI{DAEMON\_LIST} setting, they
specify a list of attribute names, each of which point to the actual
binary you want to use.
On startup, Condor will check these lists, make sure all the binaries
specified exist, and find out what abilities each program provides.
This information is used during matchmaking to ensure that a job which
requires a certain ability (like having a new enough version of Condor
to support transferring files on Unix) can find a resource that
provides that ability.

\item Added new security feature to offer fine-grained control over
what configuration values can be modified by \Condor{config\_val}
using \Arg{-set} and related options.
Pool administrators can now define lists of attributes that can be set
by hosts that authenticate to the various permission levels of
Condor's host based security (for example, \DCPerm{WRITE},
\DCPerm{ADMINISTRATOR}, etc).
These lists are defined by attributes with names like
\Macro{SETTABLE\_ATTRS\_CONFIG} and
\Macro{STARTD\_SETTABLE\_ATTRS\_OWNER}. 
For more information about host-based security in Condor, see
section~\ref{sec:Host-Security} on page~\pageref{sec:Host-Security}.
For more information about how to configure the new settings, see the
same section of the manual.
In particular, see section~\ref{sec:Host-Security} on
page~\pageref{sec:Host-Security}. 

\item Greatly improved the handling of the ``soft kill signal'' you
can specify for your job.
This signal is now stored as a signal name, not an integer, so that it
works across different platforms.
Also, fixed some bugs where the signal numbers were getting translated
incorrectly in some circumstances.

\item Added the \Arg{-full} option to \Condor{reconfig}.
The \Arg{-full} option causes the Condor daemon to clear its cache of
DNS information and some other expensive operations.
So, the regular \Condor{reconfig} is now more light-weight, and can
be used more frequently without undue overhead on the Condor daemons. 
The default \Condor{reconfig} has also been changed so that it will
work from any host with \DCPerm{WRITE} permission in your pool,
instead of requiring \DCPerm{ADMINISTRATOR} access.

\item Added the \Macro{EMAIL\_DOMAIN} config file setting.
This allows Condor administrators to define a default domain where
Condor should send email if whatever \Macro{UID\_DOMAIN} is set to
would yield invalid email addresses.
For more information, see section~\ref{param:EmailDomain} on
page~\pageref{param:EmailDomain}.

\item
Added support for Red Hat 7.2.

\item When printing out the UserLog, we now only log a new event for
``Image size of job updated'' when the new value is different than the
existing value.

\end{itemize}

\noindent Bugs Fixed:

\begin{itemize}

\item
Fixed a bug in Condor-PVM where it was possible that a machine would be 
placed into the virtual machine, but then ignored by Condor for the purposes
of scheduling tasks there.

\item
Under Solaris, the checkpointing libraries could segfault while determining
the page size of the machine. 
This has been fixed.

\item
In a heavily loaded submit machine, the \Condor{schedd} would time out
authentication checks with its shadows. 
This would cause the shadows to
exit believing the \Condor{schedd} had died placing jobs into the idle
state and the \Condor{schedd} to exhibit poor performance.
This timeout problem has been corrected.

\item
Removed use of the bfd libary in the Condor Linux distribution. 
This will make the dynamic versions of the Condor executables have a
higher chance of being usable when Red Hat upgrades.

\item
When you specify ``STARTD\_HAS\_BAD\_UTMP = True'' in the config files
on a linux machine with a 2.4+ kernel, the \Condor{startd} would report
an error stating some of the tty entries in /dev. This would result
in incorrect tty activity sampling causing jobs to not be migrated or
incorrectly started on a resource. This has now been corrected.

\item 
When you specify ``GenEnv = True'' in a \Condor{submit} file,
your environment is no longer restricted to 10KB.

\item
The three-digit event numbers which begin each job event in the
userlog were incorrect for some events in Condor 6.3.0 and 6.3.1.
Specifically, ULOG\_JOB\_SUSPENDED, ULOG\_JOB\_UNSUSPENDED,
ULOG\_JOB\_HELD, ULOG\_JOB\_RELEASED, ULOG\_GENERIC, and
ULOG\_JOB\_ABORTED had incorrect event numbers.  This has now been
corrected.

\Note This means userlog-parsing code written for Condor 6.3.0 or
6.3.1 development releases may not work reliably with userlogs
generated by other versions of Condor, and visa-versa.  Userlog events
will remain compatible between all stable releases of Condor, however,
and with post-6.3.1 releases in this development series.

\item
The \Condor{run} script now correctly exits when it sees a job aborted
event, instead of hanging, waiting for a termination event.

\item
Until now, when a DAG node's Condor job failed, the node failed,
regardless of whether its POST script succeeded or failed.  This was a
bug, because it prevented users from using POST scripts to evaluate
jobs with non-zero exit codes and deem them successful anyway.  This
has now been fixed -- a node's success is equal to its POST script's
success -- but the change may affect existing DAGs which rely on the
old, broken behavior.  Users utilizing POST scripts must now be sure
to pass the POST script the job's return value, and return it again,
if they do not wish to alter it; otherwise failed jobs will be masked
by ignorant POST scripts which always succeed.

\end{itemize}

\noindent Known Bugs:

\begin{itemize}
\item The HP-UX Vendor C++ CFront compiler does not work with \Condor{compile}
if exception handling is enabled with +eh.

\item The HP-UX Vendor aCC compiler does not work at all with Condor.
\end{itemize}

%%%%%%%%%%%%%%%%%%%%%%%%%%%%%%%%%%%%%%%%%%%%%%%%%%%%%%%%%%%%%%%%%%%%%%
\subsection*{\label{sec:New-6-3-2}Version 6.3.2}
%%%%%%%%%%%%%%%%%%%%%%%%%%%%%%%%%%%%%%%%%%%%%%%%%%%%%%%%%%%%%%%%%%%%%%

Version 6.3.2 of Condor was only released as a version of
``Condor-G''.
This version of Condor-G is not widely deployed.
However, to avoid confusion, the Condor developers did not want to
release a full Condor distribution with the same version number.


%%%%%%%%%%%%%%%%%%%%%%%%%%%%%%%%%%%%%%%%%%%%%%%%%%%%%%%%%%%%%%%%%%%%%%
\subsection*{\label{sec:New-6-3-1}Version 6.3.1}
%%%%%%%%%%%%%%%%%%%%%%%%%%%%%%%%%%%%%%%%%%%%%%%%%%%%%%%%%%%%%%%%%%%%%%

\noindent New Features:
\begin{itemize}

\item
Added support for an \AdAttr{x509proxy} option in
\Condor{submit}. There is now a seperate \Condor{GridManager} for each
user and proxy pair. This will be detailed in a future release of
Condor.
 
\item
More Condor DAGMan improvements and bug fixes:

\begin{itemize}

\item 
Added a \oArgnm{-dag} flag to \Condor{q} to more succinctly display dags
and their ownership.

\item
Added a new event to the Condor userlog at the completion of a POST
script.  This allows DAGMan, during recovery, to know which POST
scripts have finished succesfully, so it no longer has to re-run them
all to make sure.

\item
Implemented separate \Arg{-MaxPre} and \Arg{-MaxPost} options to limit
the number of simultaneously running PRE and POST scripts.  The
\Arg{-MaxScripts} option is still available, and is equivalent to
setting both \Arg{-MaxPre} and \Arg{-MaxPost} to the same value.

\item
Added support for a new ``Retry'' parameter in the DAG file, which
instructs DAGMan to automatically retry a node a configurable number
of times if its PRE Script, Job, or POST Script fail for any reason.

\item
Added timestamps to all DAGMan log messages.

\item
Fixed a bug whereby DAGMan would clean up its lock file without
creating a rescue file when killed with SIGTERM.

\item
DAGMan no longer aborts the DAG if it encounters executable error or
job aborted events in the userlog, but rather marks the corresponding
DAG nodes as ``failed'' so the rest of the DAG can continue.

\item
Fixed a bug whereby DAGMan could crash if it saw userlog events for
jobs it didn't submit.

\end{itemize}

\item Added port restriction capabilities to Condor so you can specify a range
of ports to use for the communication between Condor Daemons.

\item To improve performance: if there's no \Macro{HISTORY} file
specified, don't connect back to the schedd to report your exit info on
successful compeletion, since the schedd is simply going to discard that
info anyway.

\item Added the macro \Macro{SECONDARY\_COLLECTOR\_LIST} to tell the
master to send classads to an additional list of collectors so you can
do administration commands when the primary collector is down.

\item When a job checkpoints it askes the shadow whether or not it
should and if so where. This fixes some flocking bugs and increases
performance of the pool.

\item Added match rejection diagnostics in \Condor{q} \oArgnm{-analyze} to
give more information on why a particular job hasn't started up yet.

\item Added \oArgnm{--vms} argument to \Condor{glidein} that enables the
control of how many virtual machines to start up on the target platform.

\item Added capability to the config file language to retrieve environment
variables while being processed.

\item Added capability to make default user user priority factor configurable
with the \Macro{DEFAULT\_PRIO\_FACTOR} macro in the config files.

\item Added full support for Red Hat 7.1 and the gcc 2.96 compiler. However,
the standard universe binaries must still be statically linked.

\item When jobs are suspended or unsuspended, an event is now written into
the user job log.

\item Added \oArgnm{-a} flag to \Condor{submit} to add/override attributes
specified in the submit file.

\item Under Unix, added the ability for a submittor of a job to describe when
and how a job is allowed/not allowed to leave the queue. For example, if
a job has only run for 5 minutes, but it was supposed to have run an hour 
minimum, then do not let the job leave the queue but restart it instead.

\item New environment variable available CONDOR\_SCRATCH\_DIR available
in a standard or vanilla job's environment that denotes temporary space
the job can use that will be cleaned up automatically when the job leaves
from the machine.

\item Not exactly a new feature, but some internal parts of Condor had been
fixed up to try and improve the memory footprint of a few of our daemons.

\end{itemize}

\noindent Bugs Fixed:
\begin{itemize}

\item Fixed a bug where \Condor{q} would produce wildly inaccurate run time
reports of jobs in the queue.

\item Fixed it so that if the condor scheduler fails to notify the
administrator through email, just print a warning and do not except.

\item Fixed a bug where \Condor{submit} would incorrectly create the user
log file.

\item Fixed a bug where a job queue sorted by date with \Condor{q} would
be displayed in descending instead of ascending order.

\item Fixed and improved error handling when \Condor{submit} fails.

\item Numerous fixes in the Condor User Log System.

\item Fixed a bug where when Condor inspects its on disk job queue log,
it would do it with case sensitivity. Now there is no case sensitivity.

\item Fixed a bug in \Condor{glidein} where it have trouble figuring out
the architecture of a minimally installed HP-UX machine.

\item Fixed it so that email to the user has the word ``condor'' capitalized
in the subject.

\item Fixed a situation where when a user has multiple schedulers submitting
to the same pool, the Negotiator would starve some of the schedulers.

\item Added a feature whereby if a transfer of an executable
from a submission machine to an execute machine fails, Condor
will retry a configurable numbers of times denotated by the
\Macro{EXEC\_TRANSFER\_ATTEMPTS} macro. This macro defaults to three if
left undefined. This macro exists only for the Unix port of Condor.

\item Fixed a bug where if a schedd had too many rejected clusters during a
match phase, it would ``except'' and have to be restarted by the master.

\end{itemize}

\noindent Known Bugs:
\begin{itemize}
\item The HP-UX Vendor C++ CFront compiler does not work with \Condor{compile}
if exception handling is enabled with +eh.

\item The HP-UX Vendor aCC compiler does not work at all with Condor.
\end{itemize}

%%%%%%%%%%%%%%%%%%%%%%%%%%%%%%%%%%%%%%%%%%%%%%%%%%%%%%%%%%%%%%%%%%%%%%
\subsection*{\label{sec:New-6-3-0}Version 6.3.0}
%%%%%%%%%%%%%%%%%%%%%%%%%%%%%%%%%%%%%%%%%%%%%%%%%%%%%%%%%%%%%%%%%%%%%%

\noindent New Features:
\begin{itemize}

\item Added support for running MPICH jobs under Condor.

\end{itemize}

\noindent
Many Condor DAGMan improvements and bug fixes:

\begin{itemize}

\item
PRE and POST scripts now run asynchronously, rather than synchronously
as in the past.  As a result, DAGMan now supports a \Arg{-MaxScripts}
option to limit the number of simultaneously running PRE and POST
scripts.

\item
Whether or not POST scripts are always executed after failed jobs is
now configurable with the \Arg{-NoPostFail} argument.

\item
Added a \Arg{-r} flag to \Condor{submit\_dag} to submit DAGMan to a
remote \Condor{schedd}.

\item
Made the arguments to \Condor{submit\_dag} case-insensitive.

\item
Fixed a variety of bugs in DAGMan's event handling, so DAGMan should
no longer hang indefinitely after failed jobs, or mistake one job's
userlog events for those of another.

\item
DAGMan's error handling and logging output have been substantially
clarified and improved.  For example, DAGMan now prints a list of
failed jobs when it exits, rather than just saying ``some jobs
failed''.

\item
Jobs submitted by a \Condor{dagman} job now have \AdAttr{DAGManJobId}
and \AdAttr{DAGNodeName} in the job classad.

\item
Fixed a \Condor{submit\_dag} bug preventing the submission of DAGMan
Rescue files.

\item
Improved the handling of userlog errors (less crashing, more coping).

\item
Fixed a bug when recovering from the userlog after a crash or reboot.

\item
Fixed bugs in the handling of \Arg{-MaxJobs}.

\item
Added a \Arg{-a line} argument to \Condor{submit} to add a line to the
submit file before processing (overriding the submit file).

\item
Added a \Arg{-dag} flag to \Condor{q} to format and sort DAG jobs
sensibly under their DAGMan master job.

\end{itemize}

\noindent Known Bugs:

\begin{itemize}

\item \Condor{kbdd} doesn't work properly under Compaq Tru64 5.1, and
as a result, resources may not leave the ``Unclaimed'' state
regardless of keyboard or pty activity.  Compaq Tru64 5.0a and earlier
do work properly.

\end{itemize}
