%%%%%%%%%%%%%%%%%%%%%%%%%%%%%%%%%%%%%%%%%%%%%%%%%%%%%%%%%%%%%%%%%%%%%%
\section{\label{sec:History-6-2}Stable Release Series 6.2}
%%%%%%%%%%%%%%%%%%%%%%%%%%%%%%%%%%%%%%%%%%%%%%%%%%%%%%%%%%%%%%%%%%%%%%

This is the second stable release series of Condor.
All of the new features developed in the 6.1 series are now considered
stable, supported features of Condor.
New releases of 6.2.0 should happen infrequently and will only include
bug fixes and support for new platforms.
New features will be added and tested in the 6.3 development series. 
The details of each version are described below.


%%%%%%%%%%%%%%%%%%%%%%%%%%%%%%%%%%%%%%%%%%%%%%%%%%%%%%%%%%%%%%%%%%%%%%
\subsection{\label{sec:New-6-2-1}Version 6.2.2}
%%%%%%%%%%%%%%%%%%%%%%%%%%%%%%%%%%%%%%%%%%%%%%%%%%%%%%%%%%%%%%%%%%%%%%
\noindent New Features:
\begin{itemize}

\item Condor administrators can now specify
\MacroNI{CONDOR\_SUPPORT\_EMAIL} in their config file.
This address is included at the bottom of all email Condor sends out.
Previously, the \MacroNI{CONDOR\_ADMIN} setting was used for this, but
at many sites, the address where the Condor daemons should send email
about administrative problems is not the same address that users
should use for technical support.
If your site has different addresses for these two things, you can now
specify them properly.

\end{itemize}

\noindent Bugs Fixed:
\begin{itemize}

\item The \MacroNI{FILESYSTEM\_DOMAIN} and \MacroNI{UID\_DOMAIN} settings were
not being automatically append to the requirements of a Vanilla-universe job,
possibly causing it to run on machines where it will not successfully run. This
has been fixed.

\item The getdirentries call was not being trapped, causing a few applications
to fail running inside of Condor.

\item When the NT shadow and the Unix shadow were used in conjunction with each
other during the submission of heterogeneous jobs, they conflicted in where 
they should store their Condor internal files. This would cause extremely
long hangs where the job was listed as running but no job actually started.
This has been fixed and now you can mix the NT shadow and Unix shadow
together just fine during heterogeneous submits.

\item Numerous additions/clarifications to sections of the manual to bring
the manual up to date with the source base.

\item Fixed a bug where if you set \Macro{MAX\_JOBS\_RUNNING} to zero in the
config files, the schedd would fail with a floating point error.

\item PVM support for Solaris 2.8 was mistakenly turned off in 6.2.1,
it has been turned back on again.

\item Added exit status of Condor tools and daemons to the manual.

\item Fixed a bug in the schedd where it would segfault if it manipulated
class ads in a certain way.

\item Fixed \Syscall{stat} and \Syscall{lstat} to ask the shadow where the
file was that it was trying to locate instead of assuming it was always going
to be remote I/O.

\item Fixed a bug where Condor would incorrectly inform you that it
didn't have read/write permission on a file located in NFS when you
actually did have permission to read/write it.

\item Removed mention of \Condor{master\_off} since it was deprecated a
long time ago and is not available anymore.

\item Removed mention of \Condor{reconfig\_schedd} since it was deprecated a
long time ago and is not available anymore.

\item Fixed a bug where the schedd would occasionally EXCEPT with an
``\Macro{ATTR\_JOB\_STATUS} not found'' error soon after a \Condor{rm}
command had been invoked.

\item Fixed a bug in the Negotiator and the Schedd where it would segfault
during reading of a corrupted log file.

\item Fixed a bug where we used to EXCEPT if we couldn't email the
administrator of the pool. We now do not EXCEPT anymore.

\end{itemize}

\noindent Known Bugs:
\begin{itemize}
\item None.
\end{itemize}

%%%%%%%%%%%%%%%%%%%%%%%%%%%%%%%%%%%%%%%%%%%%%%%%%%%%%%%%%%%%%%%%%%%%%%
\subsection{\label{sec:New-6-2-1}Version 6.2.1}
%%%%%%%%%%%%%%%%%%%%%%%%%%%%%%%%%%%%%%%%%%%%%%%%%%%%%%%%%%%%%%%%%%%%%%

\noindent New Features:

\begin{itemize}

\item The \Condor{userlog} command is now available on Windows NT.

\item Jobs run in stand-alone checkpointing mode can now take a -\_condor\_nowarn
argument, which silences the warnings from the system call library when you
perform a checkpoint-unsafe action, such as opening a file for reading and
writing.

\end{itemize}

\noindent Bugs Fixed:

\begin{itemize}

\item When using heterogeneous specifications of an executable between
NT and UNIX, Condor could get confused if the vanilla job had run on
an NT machine, vacated without being completed, and then restarted as
a standard universe job on UNIX. The job would be labeled as running in
the queue, but not perform any work. This has been fixed.

\item The entries in the \AdAttr{environment} 
option in a submit file now correctly override the variables brought in
from the \AdAttr{getenv} option on Windows NT.
In previous version of CondorNT, the job would get an environment with the
variable defined multiple times. This bug did not affect UNIX versions of 
Condor.

\item Some service packs of Windows NT had bugs that prevented Condor
from determining the file permissions on input and output files. 6.2.1
uses a different set of API's to determine the permissions and works 
properly across all service packs

\item In versions of Condor previous to 6.2.0, the registry would slowly
grow on Windows NT and sometimes become corrupted. This was fixed in 6.2.0,
but if a previously-corrupted registry was detected Condor aborted. In 6.2.1,
this has been turned into a warning, as it doesn't need to be a fatal error.

\item Fixed a memory-corruption bug in the \Condor{collector}

\item PVM resources in Condor were unable to have more than one
\Attr{@} symbol in a name. 

\item The \Attr{TRANSFER\_FILES} is now set to \Attr{ON\_EXIT}
on UNIX by default for the vanilla universe. Previously, users submitting
from UNIX to NT needed to explicitly enable it or include the executable in
the list of input files for the job to run.

\item If \Attr{TRANSFER\_FILES} was set to \AdAttr{TRUE}
files created during the job's run would be transfered whenever the job was 
vacated and transfered to the next machine the job ran on, but would not be
transfered back to the submit machine when the job finally exited for the last time.

\item Determining the current working directory was broken in stand-alone
checkpointing. 

\item A job's standard output and standard error can now go to the same file.

\item When the \Macro{START\_HAS\_BAD\_UTMP} is set to TRUE, the
\Condor{startd} now detects activity on the 
\begin{verbatim}/dev/pts\end{verbatim} devices.  

\item The \Condor{negotiator} in 6.2.0 could incorrectly reject a job
that should have been successfully matched if it previously rejected a 
job. If the same jobs were sent to the \Condor{negotiator} in a different
order, the match that should succeed would. In 6.2.1, the order is no longer
important, and previous rejections will not prevent future matches.

\item The getdents, getdirents, and statfs system calls now work correctly in 
cross-platform submissions.

\item \Condor{compile} is better able to detect which version of Linux
it is running on and which flags it should pass to the linker. This should
help Condor users on non-Red Hat distributions.

\item Fixed a bug in the \Condor{startd} that would cause the daemon
to crash if you set the \Macro{POLLING\_INTERVAL} macro to a value
greater than 60.

\item In \Condor{q}, dash-arguments (e.g., -pool, -run, etc.) were being
parsed incorrectly such that the same arguments specified without a
dash would be interpreted as if the dash were present, making it
impossible to specify ``pool'' or ``globus'' or ``run'' as an owner
argument.

\item Fixed bug in \Condor{submit} that would cause certain submit
file directives to be silently ignored if you used the wrong attribute
name.  
Now, all submit file attributes can use the same names you see in the
job ClassAd (what you'd see with \Condor{q} \Opt{-long}.
For example, you can now use ``CoreSize = 0''  or ``core\_size = 0''
in your submit file, and either one would be recognized.

\item A static limit on the number of clusters the \Condor{schedd}
would accept from the \Condor{negotiator} was removed.

\item On Windows NT, if a job's log file was in a non-existent location,
both the \Condor{submit} and the \Condor{schedd} would crash.

\item Encounting unsupported system calls could cause Condor to corrupt the
signal state of the job. 

\item Fixed some of the error messages in \Condor{submit} so that they
are all consistently formatted.

\item Fixed a bug in the Linux standard universe where \Cmd{calloc}{2}
would not return zero filled memory.

\item \Condor{rm}, \Condor{hold} and \Condor{release} will now return
a non-zero exit status on failure, and only return 0 on success.
Previously, they always returned status 0.

\item If a user accidentally put \begin{verbatim}notify_user = false\end{verbatim} in their submit file, Condor used to treat that
as a valid entry.
Now, \Condor{submit} prints out a warning in this case, telling the
user that they probably want to use 
\begin{verbatim}notification = never\end{verbatim} instead.

\end{itemize}

\noindent Known Bugs:

\begin{itemize}

\item It may be possible to checkpoint with an open socket on IRIX 6.2.
On restart, the job will abort and go back into the queue. 

\end{itemize}

%%%%%%%%%%%%%%%%%%%%%%%%%%%%%%%%%%%%%%%%%%%%%%%%%%%%%%%%%%%%%%%%%%%%%%
\subsubsection{\label{sec:New-6-2-0}Version 6.2.0}
%%%%%%%%%%%%%%%%%%%%%%%%%%%%%%%%%%%%%%%%%%%%%%%%%%%%%%%%%%%%%%%%%%%%%%

\noindent New Features Over the 6.0 Release Series
\begin{itemize}

\item Support for running multiple jobs on SMP (Symmetric Multi-Processor)
machines.

\end{itemize}

\noindent New Features Over the Last Development Series: 6.1.17
\begin{itemize}

\item If \Attr{CkptArch} isn't specified in the job submission file's
\Attr{Requirements} attribute, then automatically add this expression:

\begin{verbatim}
CkptRequirements = ((CkptArch == Arch) || (CkptArch =?= UNDEFINED)) &&
	((CkptOpSys == OpSys) || (CkptOpSys =?= UNDEFINED))
\end{verbatim}

to the \Attr{Requirements} expression. This allows for users who specify
a heterogeneous submission to not have to worry about having their checkpoints
incorrectly starting up on architectures for which they were not designed
to run.

\item The \Macro{APPEND\_REQ\_<universe>} config file entries now get
appended to the beginning of the expressions before Condor adds internal
default expressions.  This allows the sysadmin to override any default
policy that Condor enforces.

\item There is now a single \Macro{APPEND\_REQUIREMENTS} attribute
that will get appended to all universe's \Attr{Requirements}
expressions unless a specific \Macro{APPEND\_REQ\_STANDARD} or
\Macro{APPEND\_REQ\_VANILLA} expression is defined.

\item Increased certain networking parameters to help alleviate the 
\Condor{shadow}'s inability to contact the \Condor{schedd} during heavy load
of the system.

\item Added a \Condor{glidein} man page to the manual.

\item Some of the log messages in the \Condor{startd} were modified to
be more clear and to provide more information.

\item Added a new attribute to the \Condor{startd} ClassAd when the
machine is claimed, \AdAttr{RemoteOwner}.

\end{itemize}

\noindent Bugs fixed since 6.1.17
\begin{itemize}

\item On NT, the Registry would increase in size while Condor was
servicing jobs. This has been fixed.

\item Added \File{utmpx} support for Solaris 2.8 to fix a problem where
\AdAttr{KeyBoardIdle} wasn't being set correctly.

\item When doing a \Condor{hold} under NT, the job was removed instead of
held. This has been fixed.

\item When using the \Arg{-master} argument to\Condor{restart}, the
\Condor{master} used to exit instead of restarting.
Now, the \Condor{master} correctly restarts itself in this case.

\end{itemize}

\noindent Known Bugs:
\begin{itemize}

\item \Attr{STARTD\_HAS\_BAD\_UTMP} does not work if set to True on Solaris 
2.8.  However, since \File{utmpx} support is enabled, you shouldn't
need to do this normally.

\item \Condor{kbdd} doesn't work properly under Compaq Tru64 5.1, and
as a result, resources may not leave the ``Unclaimed'' state
regardless of keyboard or pty activity.  Compaq Tru64 5.0a and earlier
do work properly.

\end{itemize}
