%%%      PLEASE RUN A SPELL CHECKER BEFORE COMMITTING YOUR CHANGES!
%%%      PLEASE RUN A SPELL CHECKER BEFORE COMMITTING YOUR CHANGES!
%%%      PLEASE RUN A SPELL CHECKER BEFORE COMMITTING YOUR CHANGES!
%%%      PLEASE RUN A SPELL CHECKER BEFORE COMMITTING YOUR CHANGES!
%%%      PLEASE RUN A SPELL CHECKER BEFORE COMMITTING YOUR CHANGES!

%%%%%%%%%%%%%%%%%%%%%%%%%%%%%%%%%%%%%%%%%%%%%%%%%%%%%%%%%%%%%%%%%%%%%%
\section{\label{sec:History-6-9}Development Release Series 6.9}
%%%%%%%%%%%%%%%%%%%%%%%%%%%%%%%%%%%%%%%%%%%%%%%%%%%%%%%%%%%%%%%%%%%%%%

This is the development release series of Condor.
The details of each version are described below.

%%%%%%%%%%%%%%%%%%%%%%%%%%%%%%%%%%%%%%%%%%%%%%%%%%%%%%%%%%%%%%%%%%%%%%
\subsection*{\label{sec:New-6-9-2}Version 6.9.2}
%%%%%%%%%%%%%%%%%%%%%%%%%%%%%%%%%%%%%%%%%%%%%%%%%%%%%%%%%%%%%%%%%%%%%%

\noindent Release Notes:

\begin{itemize}

%% This is important (and thus, I believe, worth of being a top
%% level release note) because it will surprise anyone upgrading
%% an existing pool or repackaging Condor binaries (say, for
%% custom glideins, or as .deb packages for a local pool.)
% This is part of the privilege seperation work, but the procd
% is required even if you're not turning privsep on.
% Questions should go to the privsep team: psilord, zmiller, etc.
\item As part of ongoing security enhancements, Condor now has a
new, required daemon: \Condor{procd}.  This daemon is
automatically started by the \Condor{master}, you do not need to
add it to \Macro{DAEMON\_LIST}.  New installations should not
need to do anything; the default configuration file is correctly
set.  Installations upgrading to 6.9.2 from previous versions
will need to ensure several things are done.  
1. Be sure to install \Condor{procd} into your Condor \Macro{SBIN} directory. 
2. Add ``\Code{PROCD = \$(SBIN)/condor\_procd}'' to your Condor configuration. 
3. Add ``\Code{PROCD\_ADDRESS = \$(LOCK)/procd\_pipe}'' to your Condor configuration. 
On Windows there are two additional steps:
4. Be sure to install \Condor{softkill} into your Condor \Macro{SBIN} directory. 
5. Add ``\Code{WINDOWS\_SOFTKILL = \$(SBIN)/condor\_softkill}'' to your Condor configuration. 

% This isn't quite so important, but it's not really a feature or
% a bug, just a change.  It is a change that may surprise some
% users.  The full list of settings impacted is
% pretty long.  So far the below is just a small fraction,
% primarily being added because an external user was suprised by
% this when testing a 6.9.2-prerelease. For anyone curious or
% inspired to flesh out the list, here's the checkin that caused
% this:
% http://bonsai.cs.wisc.edu/bonsai/cvsquery.cgi?who=danb&whotype=match&sortby=Date&date=explicit&mindate=02%2F23%2F2007+19%3A15&maxdate=02%2F23%2F2007+19%3A30
% (To do the search yourself, search for checkins by danb between
% 02/23/2007 19:15 and 02/23/2007 19:30 )
% To determine if a variable is impacted, look at the removed
% code and confirm that it used the default if the setting was 0.
% Then if the new code sets a minimum of 1 (the third argument to
% param_integer), it's impacted.
\item Some configuration settings that previously accepted 0 no
longer do so.  Instead the daemon using the setting will exit
with an error message listing the acceptable range to its log.
For these settings 0 was equivalent to requesting the default.
As this was undocumented and confusing behavior it is no longer
present.  To request a setting use its default, either comment it
out, or set it to nothing (``\Code{EXAMPLE\_SETTING=}'').
Setting impacted include but are not limited to: 
% From condor_master.V6/master.C 1.82 to 1.83:
\Macro{MASTER\_BACKOFF\_CONSTANT},
\Macro{MASTER\_BACKOFF\_CEILING},
\Macro{MASTER\_RECOVER\_FACTOR},
\Macro{MASTER\_UPDATE\_INTERVAL},
\Macro{MASTER\_NEW\_BINARY\_DELAY},
\Macro{PREEN\_INTERVAL},
\Macro{SHUTDOWN\_FAST\_TIMEOUT},
\Macro{SHUTDOWN\_GRACEFUL\_TIMEOUT},
% From condor_master.V6/daemon.C 1.68 to 1.69:},
\Macro{MASTER\_<name>\_BACKOFF\_CONSTANT},
\Macro{MASTER\_<name>\_BACKOFF\_CEILING},

\end{itemize}


\noindent New Features:

\begin{itemize}

%%\item added bogus ImageSize to bogus dedicated scheduler
%%jobAd used only for claiming.  This fixes some problems with
%%startd WANT_SUSPEND going to undefined, but we don't document
%%this bogus ad anywhere, so I'm not going to add it here.

\item Added support for EmailAttributes in the parallel universe.  
Previously, it was only valid in the vanilla and standard universes.

\item Added configuration parameter \Macro{DEDICATED\_SCHEDULER\_USE\_FIFO}
which defaults to true.  When false, the dedicated scheduler will
use a best-fit algorithm to schedule parallel jobs.  This setting is
not recommended, as it can cause starvation.  When true, the dedicated
scheduler will schedule jobs in a first-in, first-out manner.

\item Added \Opt{-dump} to \Condor{config\_val} which will print out
all of the macros defined in any of the configuration files found by
the program.
\Condor{config\_val} \Opt{-dump} \Opt{-v} will augment the output
with exactly what line and in what file each configuration variable
was found.
\Note: The output format of the \Opt{-dump} option will most likely
change in a future revision of Condor.

% Gnats PR 671
\item Node names in \Condor{dagman} DAG files can now be DAG
keywords, except for PARENT and CHILD.

\item Improved the log message when \Attr{OnExitRemove} or
\Attr{OnExitHold} evaluates to UNDEFINED.

% Gnats PR 796
\item Added the \Macro{DAGMAN\_ON\_EXIT\_REMOVE} configuration macro,
which allows customization of the \Attr{OnExitRemove} expression
generated by \Condor{submit\_dag}.

\item When using GCB, Condor can now be told to choose from a list of
brokers. \Attr{NET\_REMAP\_INAGENT} is now a space and comma separated
list of brokers. On startup, the \Condor{master} will query all of the
brokers and pick the least-used one for it and its children to use. If
none of the brokers are operational, then the \Condor{master} will wait
until one is working. This waiting can be disabled by setting 
\Attr{MASTER\_WAITS\_FOR\_GCB\_BROKER} to FALSE in the configuration file.
If the chosen broker fails and recovery is not possible or another broker
is available, the \Condor{master} will restart all of the daemons.

\item When using GCB, communications between parent and child
Condor daemons on the same host no longer use the GCB broker.
This improves scalability and also allows a single host to
continue functioning if the GCB broker is unavailable.

\item The \Condor{schedd} now uses non-blocking methods to send the
``alive'' message to the \Condor{startd} when renewing the job lease.
This prevents the \Condor{schedd} from blocking for 20 seconds while
trying to connect to a machine that has become disconnected from the
network.

\item \Condor{advertise} can read the classad to be advertised from
standard input.

\item Unix Condor daemons now reinitialize their DNS
configuration (e.g. IP addresses of the nameservers) on reconfig.

% Gnats PR 777
\item A configuration file for \Condor{dagman} can now be specified
in a DAG file or on the \Condor{submit\_dag} command line.

\item Added \Condor{cod} option \Opt{-lease} for creation of COD claims
with a limited duration lease.  This provides automatic cleanup of COD
claims that are not renewed by the user.  The default lease is infinitely
long, so existing behavior is unchanged unless \Opt{-lease} is explicitly
specified.

\item Added \Condor{cod} command \Opt{delegate\_proxy} which will
delegate an x509 proxy to the requested COD claim.
This is primarily useful for sites wishing to use glexec to spawn the
\Condor{starter} used for COD jobs.
The new command optionally takes an \Opt{-x509proxy} argument to
specify the proxy file.
If this argument is not present, \Condor{cod} will search for the
proxy using the same logic as \Condor{submit} does.

% This is barely a feature, but it's definately not a bug fix. It's
% more of a change in behavior.
\item \Macro{STARTD\_DEBUG} can now be empty, indicating a default, minimal
log level. It now defaults to empty.
Previously it had to be non-empty and defaulted to include D\_COMMAND.

\end{itemize}

\noindent Bugs Fixed:

\begin{itemize}

\item Under various circumstances, condor 6.9.1 daemons would abort
with the message, ``ERROR: Unexpected pending status for fake message
delivery.''  A specific example is when \Attr{OnExitRemove} or
\Attr{OnExitHold} evaluated to UNDEFINED.  This caused the
\Condor{schedd} to abort.

\item In Condor 6.9.1, the \Condor{schedd} would die during startup
when trying to reconnect to running jobs for which the \Condor{schedd}
could not find a startd ClassAd.  This would happen shortly after
logging the following message: ``Could not find machine ClassAds for
one or more jobs.  May be flocking, or machine may be down.
Attempting to reconnect anyway.''

\item Improved Condor's validity checking of configuration values.
For example, in some cases where Condor was expecting an integer but
was given an expression such as 12*60, it would silently interpret
this as 12.  Such cases now result in the condor daemon exiting
after issuing an error message into the log file.

\end{itemize}

\noindent Known Bugs:

\begin{itemize}

\item None.

\end{itemize}



%%%%%%%%%%%%%%%%%%%%%%%%%%%%%%%%%%%%%%%%%%%%%%%%%%%%%%%%%%%%%%%%%%%%%%
\subsection*{\label{sec:New-6-9-1}Version 6.9.1}
%%%%%%%%%%%%%%%%%%%%%%%%%%%%%%%%%%%%%%%%%%%%%%%%%%%%%%%%%%%%%%%%%%%%%%

\noindent Release Notes:

\begin{itemize}

\item The 6.9.1 release contains all of the bug fixes and enhancements
  from the 6.8.x series up to and including version 6.8.3.

\item Version 1.4.0 of the Generic Connection Broker (GCB) library is
  now used for building Condor, and it is the 1.4.0 versions of the
  \Prog{gcb\_broker} and \Prog{gcb\_relay\_server} programs that are
  included in this release.
  This version of GCB includes enhancements used by Condor
  along with a new GCB-related command-line tool:
  \Prog{gcb\_broker\_query}.
  Condor 6.9.1 will not work properly with older versions of the
  \Prog{gcb\_broker} or \Prog{gcb\_relay\_server}.
  For more information about GCB, see section~\ref{sec:GCB} on
  page~\pageref{sec:GCB}. 

\end{itemize}

\noindent New Features:

\begin{itemize}

\item Improved the performance of the ClassAd matching algorithm,
which speeds up the \Condor{schedd} and other daemons.

\item Improved the scalability of the algorithm used by 
the \Condor{schedd} daemon to find runnable jobs.
This makes a noticeable difference in \Condor{schedd} daemon performance,
when there are on the order of thousands of jobs in the queue.

\item the \Dflag{COMMAND} debugging level has been enhanced to
log many more messages. 

\item Updated the version of DRMAA, which contains several great
improvements regarding scalability and race conditions.

% Gnats PR 774
\item Added the \Macro{DAGMAN\_SUBMIT\_DEPTH\_FIRST} configuration macro,
which causes \Condor{dagman} to submit ready nodes in more-or-less depth-first
order, if set to \Expr{True}.  The default behavior is to submit
the ready nodes in breadth-first order.

\item Added configuration parameter \Macro{USE\_PROCESS\_GROUPS}.
If it is set to \Expr{False},
then Condor daemons on Unix machines will not create new 
sessions or process groups. This is intended for use with Glidein, as
we have had reports that some batch systems cannot properly track jobs that
create new process groups. The default value is \Expr{True}.

\item The default value for the submit file command
\SubmitCmd{copy\_to\_spool} has been changed to \Expr{False}, because copying
the executable to the spool directory for each job (or job cluster) is almost
never desired.  Previously, the default was \Expr{True} in all
cases, except for grid universe jobs and remote submissions.

\item More types of file transfer errors now result in the job going
on hold, with a specific error message about what went wrong.  The new
cases involve failures to write output files to disk on the submit
side (for example, when the disk is full).
As always, the specific error number is
recorded in \Attr{HoldReasonSubCode}, so you can enforce an automated
error handling policy using \SubmitCmd{periodic\_release} or
\SubmitCmd{periodic\_remove}.

\item Added the \Macro{<SUBSYS>\_DAEMON\_AD\_FILE}
configuration variable, which is similar to the 
\MacroNI{<SUBSYS>\_ADDRESS\_FILE}.
This new variable will be used in future versions of Condor, but is
not necessary for 6.9.1.


\end{itemize}

\noindent Bugs Fixed:

\begin{itemize}

\item Fixed a bug in the \Condor{master} so that it will now send obituary
e-mails when it kills child processes that it considers hung.

\item \Condor{configure} used to always make a personal Condor with
\Opt{--install} even when \Opt{--type} called for only execute or
submit types.  Now, \Condor{configure} honors the \Opt{--type}
argument, even when using \Opt{--install}.
If \Opt{--type} is not specified, the default is to still install a
full personal Condor with the following daemons: 
\Condor{master}, \Condor{collector},
\Condor{negotiator}, \Condor{schedd}, \Condor{startd}. 

\item While removing, putting on hold, or vacating a large number of
jobs, it was possible for the \Condor{schedd} and the \Condor{shadow} to
temporarily deadlock with each other.  This has been fixed under Unix,
but not yet under Windows.

\item Communication from a \Condor{schedd} to a \Condor{startd}
now occurs in a nonblocking manner.
This fixes the problem of the \Condor{schedd} blocking 
when the claimed machine running the \Condor{startd}
cannot be reached, for example because the machine is turned off.

\end{itemize}

\noindent Known Bugs:

\begin{itemize}

\item Under various circumstances, condor 6.9.1 daemons abort
with the message, ``ERROR: Unexpected pending status for fake message
delivery.''  A specific example is when \Attr{OnExitRemove} or
\Attr{OnExitHold} evaluated to UNDEFINED, which causes the
\Condor{schedd} to abort.

\item In Condor 6.9.1, the \Condor{schedd} will die during startup
when trying to reconnect to running jobs for which the \Condor{schedd}
can not find a startd ClassAd.  This happens shortly after
logging the following message: ``Could not find machine ClassAds for
one or more jobs.  May be flocking, or machine may be down.
Attempting to reconnect anyway.''

\end{itemize}

%%%%%%%%%%%%%%%%%%%%%%%%%%%%%%%%%%%%%%%%%%%%%%%%%%%%%%%%%%%%%%%%%%%%%%
\subsection*{\label{sec:New-6-9-0}Version 6.9.0}
%%%%%%%%%%%%%%%%%%%%%%%%%%%%%%%%%%%%%%%%%%%%%%%%%%%%%%%%%%%%%%%%%%%%%%

\noindent Release Notes:

\begin{itemize}

\item The 6.9.0 release contains all of the bug fixes and enhancements
  from the 6.8.x series up to and including version 6.8.2.

% and a few \condor{gridmanager} bug fixes from 6.8.3.  *sigh* we need
% a real solution to this problem (like pointing to issue node ids) ;)

\end{itemize}


\noindent New Features:

\begin{itemize}


\item Preliminary support for using \Prog{glexec} on execute machines
has been added.  This feature causes the \Condor{startd} to spawn the
\Condor{starter} as the user that \Prog{glexec} determines based on
the user's GSI credential.

\item A ``per-job history files'' feature has been added to the
\Condor{schedd}. When enabled, this will cause the \Condor{schedd} to
write out a copy of each job's ClassAd when it leaves the job
queue. The directory to place these files in is determined by the
parameter \Macro{PER\_JOB\_HISTORY\_DIR}. It is the responsibility of
whatever external entity (for example, an accounting or monitoring system) is
using these files to remove them as it completes its processing.

\item \Condor{chirp} command now supports writing messages to the user log.

\item \Condor{chirp} getattr and putattr now send all classad getattr
and putattr commands to the proc 0 classad, which allows multiple proc
parallel jobs to use proc 0 as a scratch pad.

\item Parallel jobs now support an \Attr{AllRemoteHosts} attribute,
which lists all the hosts across all procs in a cluster.

\item The \Macro{DAGMAN\_ABORT\_DUPLICATES} configuration macro (which causes
\Condor{dagman} to abort itself if it detects another \Condor{dagman}
running on the same DAG) now defaults to \Expr{True} instead of
\Expr{False}.

\end{itemize}

\noindent Bugs Fixed:

\begin{itemize}

\item None.

\end{itemize}

\noindent Known Bugs:

\begin{itemize}

\item None.

\end{itemize}

