%%%      PLEASE RUN A SPELL CHECKER BEFORE COMMITTING YOUR CHANGES!
%%%      PLEASE RUN A SPELL CHECKER BEFORE COMMITTING YOUR CHANGES!
%%%      PLEASE RUN A SPELL CHECKER BEFORE COMMITTING YOUR CHANGES!
%%%      PLEASE RUN A SPELL CHECKER BEFORE COMMITTING YOUR CHANGES!
%%%      PLEASE RUN A SPELL CHECKER BEFORE COMMITTING YOUR CHANGES!

%%%%%%%%%%%%%%%%%%%%%%%%%%%%%%%%%%%%%%%%%%%%%%%%%%%%%%%%%%%%%%%%%%%%%%
\section{\label{sec:History-6-9}Development Release Series 6.9}
%%%%%%%%%%%%%%%%%%%%%%%%%%%%%%%%%%%%%%%%%%%%%%%%%%%%%%%%%%%%%%%%%%%%%%

This is the development release series of Condor.
The details of each version are described below.

%%%%%%%%%%%%%%%%%%%%%%%%%%%%%%%%%%%%%%%%%%%%%%%%%%%%%%%%%%%%%%%%%%%%%%
\subsection*{\label{sec:New-6-9-2}Version 6.9.2}
%%%%%%%%%%%%%%%%%%%%%%%%%%%%%%%%%%%%%%%%%%%%%%%%%%%%%%%%%%%%%%%%%%%%%%

\noindent Release Notes:

\begin{itemize}

\item None.

\end{itemize}


\noindent New Features:

\begin{itemize}

% Gnats PR 671
\item Node names in \Condor{dagman} DAG files can now be DAG
keywords, except for PARENT and CHILD.

\item Improved the log message when \Attr{OnExitRemove} or
\Attr{OnExitHold} evaluates to UNDEFINED.

\end{itemize}

\noindent Bugs Fixed:

\begin{itemize}

\item Under various circumstances, condor 6.9.1 daemons would abort
with the message, ``ERROR: Unexpected pending status for fake message
delivery.''  A specific example is when \Attr{OnExitRemove} or
\Attr{OnExitHold} evaluated to UNDEFINED.  This caused the
\Condor{schedd} to abort.

\item In Condor 6.9.1, the \Condor{schedd} would die during startup
when trying to reconnect to running jobs for which the \Condor{schedd}
could not find a startd ClassAd.  This would happen shortly after
logging the following message: ``Could not find machine ClassAds for
one or more jobs.  May be flocking, or machine may be down.
Attempting to reconnect anyway.''

\end{itemize}

\noindent Known Bugs:

\begin{itemize}

\item None.

\end{itemize}



%%%%%%%%%%%%%%%%%%%%%%%%%%%%%%%%%%%%%%%%%%%%%%%%%%%%%%%%%%%%%%%%%%%%%%
\subsection*{\label{sec:New-6-9-1}Version 6.9.1}
%%%%%%%%%%%%%%%%%%%%%%%%%%%%%%%%%%%%%%%%%%%%%%%%%%%%%%%%%%%%%%%%%%%%%%

\noindent Release Notes:

\begin{itemize}

\item The 6.9.1 release contains all of the bug fixes and enhancements
  from the 6.8.x series up to and including version 6.8.3.

\item Version 1.4.0 of the Generic Connection Broker (GCB) library is
  now used for building Condor, and it is the 1.4.0 versions of the
  \Prog{gcb\_broker} and \Prog{gcb\_relay\_server} programs that are
  included in this release.
  This version of GCB includes enhancements used by Condor
  along with a new GCB-related command-line tool:
  \Prog{gcb\_broker\_query}.
  For more information about GCB, see section~\ref{sec:GCB} on
  page~\pageref{sec:GCB}. 

\end{itemize}

\noindent New Features:

\begin{itemize}

\item Improved the performance of the ClassAd matching algorithm,
which speeds up the \Condor{schedd} and other daemons.

\item Improved the scalability of the algorithm used by 
the \Condor{schedd} daemon to find runnable jobs.
This makes a noticeable difference in \Condor{schedd} daemon performance,
when there are on the order of thousands of jobs in the queue.

\item the \Dflag{COMMAND} debugging level has been enhanced to
log many more messages. 

\item Updated the version of DRMAA, which contains several great
improvements regarding scalability and race conditions.

% Gnats PR 774
\item Added the \Macro{DAGMAN\_SUBMIT\_DEPTH\_FIRST} configuration macro,
which causes \Condor{dagman} to submit ready nodes in more-or-less depth-first
order, if set to \Expr{True}.  The default behavior is to submit
the ready nodes in breadth-first order.

\item Added configuration parameter \Macro{USE\_PROCESS\_GROUPS}.
If it is set to \Expr{False},
then Condor daemons on Unix machines will not create new 
sessions or process groups. This is intended for use with Glidein, as
we have had reports that some batch systems cannot properly track jobs that
create new process groups. The default value is \Expr{True}.

\item The default value for the submit file command
\SubmitCmd{copy\_to\_spool} has been changed to \Expr{False}, because copying
the executable to the spool directory for each job (or job cluster) is almost
never desired.  Previously, the default was \Expr{True} in all
cases, except for grid universe jobs and remote submissions.

\item More types of file transfer errors now result in the job going
on hold, with a specific error message about what went wrong.  The new
cases involve failures to write output files to disk on the submit
side (for example, when the disk is full).
As always, the specific error number is
recorded in \Attr{HoldReasonSubCode}, so you can enforce an automated
error handling policy using \SubmitCmd{periodic\_release} or
\SubmitCmd{periodic\_remove}.

\item Added the \Macro{<SUBSYS>\_DAEMON\_AD\_FILE}
configuration variable, which is similar to the 
\MacroNI{<SUBSYS>\_ADDRESS\_FILE}.
This new variable will be used in future versions of Condor, but is
not necessary for 6.9.1.


\end{itemize}

\noindent Bugs Fixed:

\begin{itemize}

\item Fixed a bug in the \Condor{master} so that it will now send obituary
e-mails when it kills child processes that it considers hung.

\item \Condor{configure} used to always make a personal Condor with
\Opt{--install} even when \Opt{--type} called for only execute or
submit types.  Now, \Condor{configure} honors the \Opt{--type}
argument, even when using \Opt{--install}.
If \Opt{--type} is not specified, the default is to still install a
full personal Condor with the following daemons: 
\Condor{master}, \Condor{collector},
\Condor{negotiator}, \Condor{schedd}, \Condor{startd}. 

\item While removing, putting on hold, or vacating a large number of
jobs, it was possible for the \Condor{schedd} and the \Condor{shadow} to
temporarily deadlock with each other.  This has been fixed under Unix,
but not yet under Windows.

\item Communication from a \Condor{schedd} to a \Condor{startd}
now occurs in a nonblocking manner.
This fixes the problem of the \Condor{schedd} blocking 
when the claimed machine running the \Condor{startd}
cannot be reached, for example because the machine is turned off.

\end{itemize}

\noindent Known Bugs:

\begin{itemize}

\item None.

\end{itemize}

%%%%%%%%%%%%%%%%%%%%%%%%%%%%%%%%%%%%%%%%%%%%%%%%%%%%%%%%%%%%%%%%%%%%%%
\subsection*{\label{sec:New-6-9-0}Version 6.9.0}
%%%%%%%%%%%%%%%%%%%%%%%%%%%%%%%%%%%%%%%%%%%%%%%%%%%%%%%%%%%%%%%%%%%%%%

\noindent Release Notes:

\begin{itemize}

\item The 6.9.0 release contains all of the bug fixes and enhancements
  from the 6.8.x series up to and including version 6.8.2.

% and a few \condor{gridmanager} bug fixes from 6.8.3.  *sigh* we need
% a real solution to this problem (like pointing to issue node ids) ;)

\end{itemize}


\noindent New Features:

\begin{itemize}


\item Preliminary support for using \Prog{glexec} on execute machines
has been added.  This feature causes the \Condor{startd} to spawn the
\Condor{starter} as the user that \Prog{glexec} determines based on
the user's GSI credential.

\item A ``per-job history files'' feature has been added to the
\Condor{schedd}. When enabled, this will cause the \Condor{schedd} to
write out a copy of each job's ClassAd when it leaves the job
queue. The directory to place these files in is determined by the
parameter \Macro{PER\_JOB\_HISTORY\_DIR}. It is the responsibility of
whatever external entity (for example, an accounting or monitoring system) is
using these files to remove them as it completes its processing.

\item \Condor{chirp} command now supports writing messages to the user log.

\item \Condor{chirp} getattr and putattr now send all classad getattr
and putattr commands to the proc 0 classad, which allows multiple proc
parallel jobs to use proc 0 as a scratch pad.

\item Parallel jobs now support an \Attr{AllRemoteHosts} attribute,
which lists all the hosts across all procs in a cluster.

\item The \Macro{DAGMAN\_ABORT\_DUPLICATES} config macro (which causes
\Condor{dagman} to abort itself if it detects another \Condor{dagman}
running on the same DAG) now defaults to \Expr{True} instead of
\Expr{False}.

\end{itemize}

\noindent Bugs Fixed:

\begin{itemize}

\item None.

\end{itemize}

\noindent Known Bugs:

\begin{itemize}

\item None.

\end{itemize}

