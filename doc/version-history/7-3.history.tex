%%%      PLEASE RUN A SPELL CHECKER BEFORE COMMITTING YOUR CHANGES!
%%%      PLEASE RUN A SPELL CHECKER BEFORE COMMITTING YOUR CHANGES!
%%%      PLEASE RUN A SPELL CHECKER BEFORE COMMITTING YOUR CHANGES!
%%%      PLEASE RUN A SPELL CHECKER BEFORE COMMITTING YOUR CHANGES!
%%%      PLEASE RUN A SPELL CHECKER BEFORE COMMITTING YOUR CHANGES!

%%%%%%%%%%%%%%%%%%%%%%%%%%%%%%%%%%%%%%%%%%%%%%%%%%%%%%%%%%%%%%%%%%%%%%
\section{Development Release Series 7.3}\label{sec:History-7-3}
%%%%%%%%%%%%%%%%%%%%%%%%%%%%%%%%%%%%%%%%%%%%%%%%%%%%%%%%%%%%%%%%%%%%%%

This is the development release series of Condor.
The details of each version are described below.

%%%%%%%%%%%%%%%%%%%%%%%%%%%%%%%%%%%%%%%%%%%%%%%%%%%%%%%%%%%%%%%%%%%%%%
\subsection*{\label{sec:New-7-3-2}Version 7.3.2}
%%%%%%%%%%%%%%%%%%%%%%%%%%%%%%%%%%%%%%%%%%%%%%%%%%%%%%%%%%%%%%%%%%%%%%

\noindent Release Notes:

\begin{itemize}

\item The format of the output from \Condor{status} with the \Opt{-grid} option
has been changed to provide more useful information.

% gittrac #12
\item Removed the newline appended to the end of \Condor{status}
\Opt{-format} output.
Therefore, code which parses the output of this command should now
be careful when trimming the last line.

\end{itemize}

\noindent New Features:

\begin{itemize}

\item \Condor{fetchlog} may now fetch the history files of a \Condor{schedd}
daemon.  And, the history file kept by the \Condor{schedd} daemon may
now be rotated daily or monthly.

\item The \Condor{ckpt\_server} will automatically clean up stale
checkpoint files. The configuration variables which control this
behavior are described below.

% gittrac #177
\item The \Condor{ckpt\_server} (either the 32-bit or 64-bit) executable
will now communicate correctly between 32-bit and 64-bit submit nodes.
If by some chance bit width issues arise in the checkpoint protocol
(for example, with file sizes),
clear error messages are logged in the checkpoint server logs.

\item The new \Condor{ssh\_to\_job} tool allows interactive debugging of running
jobs.  See the manual page at~\pageref{man-condor-ssh-to-job} for details.

\item The \Condor{status} command is now substantially faster, 
especially with the \Opt{-format} option.

% gittrac #676
\item Grid universe grid type \Opt{gt5} has been added for submission to
the new Globus GRAM5 service. When a GRAM service is identified as
\SubmitCmd{gt5}, jobmanager throttling and the Grid Monitor are not used.
See section~\ref{sec:Using-gt5} for details.

\item Grid universe grid type \Opt{cream} has been added for submission
to the CREAM job service of \Prog{gLite}.
See section~\ref{sec:CREAM} for details.

\item When low on file descriptors for creating new network sockets,
the \Condor{schedd} daemon now avoids the unlimited stacking up of
messages that it sends periodically to the \Condor{negotiator} 
and \Condor{startd}.

% gittrac #429
\item The performance and failure handling of the Grid Monitor have been
improved.

% gittrac #356
\item For grid type \SubmitCmd{nordugrid} in the grid universe,
job status information
is now obtained using Nordugrid ARC's LDAP server, which should greatly
improve performance. Also, Condor can now tell when these jobs are running.

%gittrac #527
\item The new \Opt{-valgrind} option to \Condor{submit\_dag}
causes \Condor{submit\_dag} to generate a submit description file that
uses \emph{valgrind} on \Condor{dagman}, instead of the \Condor{dagman}
binary as its executable.

%gittrac #328
\item \Condor{dagman} now lazily evaluates and opens node job log files.
Instead of parsing all submit description files and 
immediately opening their specified log files at start up,
\Condor{dagman} now parses
the submit description files just before each job is submitted,
and has each log file open only when relevant jobs are in the queue
or executing POST scripts.
In addition, \Condor{dagman} now automatically generates a default user log
file for any node job that does not specify one.

\item Both the support and documentation for the MPI universe have been removed.
MPI applications are supported through the use of the parallel universe.

% gittrac #551
\item When the \Condor{startd} daemon's test of virtual machine software fails
(for machines configured as capable of running virtual machines),
the \Condor{startd} will periodically retry the test until it succeeds.

%gittrac #654
\item The \Prog{nordugrid\_gahp} now limits the number of connections
made to each NorduGrid ARC server and reuses connections when possible.

\item Added the ClassAd function \Code{eval()}, which takes a string
argument and evaluates the contents of the string as a ClassAd
expression.  An policy example where this is useful is described in
section~\ref{sec:Job-Suspension} on job suspension.

\item The new \Condor{q} option \Opt{-attributes} limits the
attributes which are displayed when using the \Opt{-xml} or \Opt{-long}
options.
Limiting the number of attributes also increases the efficiency of the query.

%gittrac #383
\item Condor's power management capabilities are now implemented as a
  plug-in.  In particular, the \Condor{startd} now runs an
  external program, as specified by the configuration variable
  \Macro{HIBERNATION\_PLUGIN},
  to perform the detection of available low power states and the
  switching to these low power states.

\item The new Condor daemon \Condor{rooster} has been added to wake up
hibernating machines when the expression defined by the configuration variable
\Macro{UNHIBERNATE} becomes \Expr{True}.
The configuration variables relating to \Condor{rooster}
are described in section~\ref{sec:Config-rooster}.

%gittrac #483
\item Added the ability to extract information from the user event log
  reader's state buffer to the user log reader.  This is implemented
  through a new \Code{ReadUserLogStateAccess} C++ class
  as defined in \File{read\_user\_log.h}.

\item Changes to the value of the configuration variable
\MacroNI{CERTIFICATE\_MAPFILE} or the contents
of the file to which it refers no longer require a full restart of Condor.
Instead, the command \Condor{reconfig} will cause the changes to be utilized.

\item The \Condor{master} daemon will now print the path and arguments
  to any daemons it starts if D\_FULLDEBUG is enabled.  Previously,
  there was no way to get it to display the arguments with which it
  was starting a daemon.

% gitrac #308
\item The \Condor{had} daemon now has the ability to control daemons
  other than the \Condor{negotiator}.  This is controlled via the
  \MacroNI{HAD\_CONTROLLEE} macro.

%gittrac #161
\item Condor now recognizes VOMS extensions in X.509 proxies.
The VOMS attributes are encoded in the job ClassAd attribute
\Attr{X509UserProxySubject}.

%gittrac #514
\item The \Condor{startd} can now clean up stranded virtual machines,
following a crash of Condor or its host operating system.

%gittrac #301
\item Following a crash, the \Condor{gridmanager} no longer restarts all
of the jobmanagers for gt2 jobs. This should improve recovery time.

\item Condor works better with the ClassAds categorized as generic
in the \Condor{collector} daemon.
Various daemons that register themselves with generic ClassAds
can now have tools which use the \Opt{-subsystem} option manipulate
their ClassAds properly.

\item Condor now provides a mechanism to enforce strict resource limiting for
some universes of running jobs.

\end{itemize}

\noindent Configuration Variable Additions and Changes:

\begin{itemize}

\item The new configuration variable \Macro{EMAIL\_SIGNATURE} specifies
a custom signature to be appended to e-mail sent by the Condor system.
If defined, then this custom signature replaces the
default one specified internally.

\item The new configuration variable \Macro{CKPT\_SERVER\_CLIENT\_TIMEOUT}
informs the \Condor{schedd} how long in seconds it is willing to wait
to try and talk to a \Condor{ckpt\_server} process before declaring a
\Condor{ckpt\_server} down.
See section~\ref{param:CkptServerClientTimeout} for the complete description.

\item The new configuration variable
\Macro{CKPT\_SERVER\_CLIENT\_TIMEOUT\_RETRY} informs the \Condor{schedd}
that once a \Condor{ckpt\_server} is been marked as down, how may seconds
must pass before the \Condor{schedd} will try and communicate with the
\Condor{ckpt\_server} again.
See section~\ref{param:CkptServerClientTimeoutRetry} 
for the complete description.

\item The new configuration variable
\Macro{CKPT\_SERVER\_REMOVE\_STALE\_CKPT\_INTERVAL} informs the
\Condor{ckpt\_server} to begin removal of stale checkpoints at the specified
interval in seconds.
See section~\ref{param:CkptServerRemoveStaleCkptInterval} 
for the complete description.

\item The new configuration variable
\Macro{CKPT\_SERVER\_STALE\_CKPT\_AGE\_CUTOFF} informs the
\Condor{ckpt\_server} how old a checkpoint file's access time must be
in order to be considered stale. This time is compared against the
current notion of now
when the checkpoint server checks the checkpoint image file.
See section~\ref{param:CkptServerStaleCkptAgeCutoff} 
for the complete description.

% gittrac 331
\item The new configuration variable \Macro{SlotWeight} may be used to
give a slot greater weight when calculating usage, computing fair
shares, and enforcing group quotas.  
See \ref{param:SlotWeight} for the complete description.

\item The new configuration variable \Macro{MAX\_PERIODIC\_EXPR\_INTERVAL}
  implements a ceiling on the time between evaluation of periodic expressions,
  due to the adaptive timing implied by the configuration variable
  \MacroNI{PERIODIC\_EXPR\_TIMESLICE}.
  See \ref{param:MaxPeriodicExprInterval} for the complete description.

% gittrac #475
\item The new configuration variable \Macro{GRIDMANAGER\_SELECTION\_EXPR}
can be used to control how many \Condor{gridmanager} processes will be
spawned to manage grid universe jobs. As a part of this change, removed
the configuration variable and supporting code for 
\Macro{GRIDMANAGER\_PER\_JOB} since the new configuration variable
supersedes it.
See \ref{param:GridManagerSelectionExpr} for the complete description.

% gittrac #207
\item The configuration variable
\Macro{GRIDMANAGER\_MAX\_PENDING\_SUBMITS\_PER\_RESOURCE} and the
corresponding throttle \Macro{GRIDMANAGER\_MAX\_PENDING\_SUBMITS}
have been removed.

% gittrac #429
\item The new configuration variable \Macro{GRID\_MONITOR\_DISABLE\_TIME}
controls how long the \Condor{gridmanager} will wait after encountering
an error before attempting to restart a Grid Monitor job.
See \ref{param:GridMonitorDisableTime} for the complete description.

\item The new pre-defined configuration macro \Macro{DETECTED\_MEMORY}
indicates the amount of physical memory (RAM) detected by Condor.
The value is given in Mbytes.

\item The new pre-defined configuration macro \Macro{DETECTED\_CORES}
indicates the number of CPU cores detected by Condor.

%gittrac #621
\item The new configuration variable
\Macro{DELEGATE\_FULL\_JOB\_GSI\_CREDENTIALS}
controls whether a full or limited X.509 proxy is delegated for grid type
\SubmitCmd{gt2} \SubmitCmd{grid} universe jobs.
See \ref{param:DelegateFullJobGSICredentials}
for the complete description.

%gittrack #383
\item The new configuration variable \Macro{UNHIBERNATE} is used by
the \Condor{startd} to advertise in its ClassAd a boolean expression
specifying when the machine should be woken up, 
for example by \Condor{rooster}.
See \ref{param:Unhibernate} for the complete description.

%gittrac #383
\item The new configuration variable \Macro{HIBERNATION\_PLUGIN} specifies the
  path to the plug-in which the \Condor{startd} uses both to detect
  the low power state capabilities of a machine and to switch the
  machine to a low power state.
  See \ref{param:HibernationPlugin} for the complete description.

%gittrac #383
\item The new configuration variable \Macro{HIBERNATION\_PLUGIN\_ARGS}
  specifies additional command line arguments which the
  \Condor{startd} will pass to the plug-in when invoking it to
  switch the machine to a low power state.
  See \ref{param:HibernationPluginArgs} for the complete description.

%gittrac #383
\item The new configuration variable \Macro{HIBERNATION\_OVERRIDE\_WOL} can be
  used to direct the \Condor{startd} to ignore Wake On LAN (WOL)
  capabilities of the machine's network interface, and to switch to a
  low power state even if the interface does not support WOL, or if
  WOL is disabled on it.
  See \ref{param:HibernationOverrideWOL} for the complete description.

% gittrac 586
\item The new configuration variable \Macro{DAGMAN\_USER\_LOG\_SCAN\_INTERVAL}
controls how long \Condor{dagman} waits between checking job log files
for status updates.
See \ref{param:DAGManUserLogScanInterval} for the complete description.

\item The new configuration variable \Macro{DAGMAN\_DEFAULT\_NODE\_LOG} sets
the default log file name for the new \Condor{dagman}
default node log file feature.
See \ref{param:DAGManDefaultNodeLog}
for the complete description.

\item Removed the configuration variable
\Macro{DAGMAN\_DELETE\_OLD\_LOGS}; new log file reading code makes it
obsolete.

% gittrac #308
\item The new configuration variable \Macro{HAD\_CONTROLLEE} is used
  to specify the name of the daemon which the \Condor{had} controls.
  This name should match the daemon name in the \Condor{master}'s
  \MacroNI{DAEMON\_LIST}.

\end{itemize}

\noindent Bugs Fixed:

\begin{itemize}

\item Fixed a bug in ClassAd functions where arguments which should have been
correctly coerced into strings instead evaluated to \texttt{ERROR}.

\item Fixed a confusing diagnostic message with the JobRouter, which happened
when a job was removed within 5 minutes of being submitted.

% from 7.5.2
% gittrac #581
\item Fixed a bug in which the use of dynamic slots 
(see section~\ref{sec:SMP-dynamicprovisioning})
caused the machine ClassAd attribute \Attr{SLOT<N>\_STARTD\_ATTRS}
to disappear from the ClassAd for some slots.

% gittrac #510
% Karen's rewrite of
%\item Fixed a bug in the Windows port where windows of an application running
%under Condor may never receive a paint message.
\item Fixed a Windows platform bug in which the window belonging to
a Condor job does not receive a paint message.

\item Fixed a bug causing \Condor{q} \Opt{-analyze} to crash when there was no 
\Condor{schedd} daemon ClassAd file.

\item Fixed a \Condor{procd} crash caused when the environment of 
a monitored process exceeded 1MByte in \File{/proc}.

% gittrac #708
\item Fixed a Windows platform bug which could cause the \Condor{credd} 
to crash if a requested credential is not in the password store.

% gittrac #535
\item Fixed a bug that was causing the job event log rotation lock to be
created with incorrect permissions.

% gittrac #601
\item Fixed a bug in the rotation of the job event log which could cause it
never to be rotated in the Windows port of Condor.

% gittrac #691
\item Fixed a potential race condition in the job event log initialization.

% gittrac #695
\item Fixed race condition which could cause a crash of the \Condor{collector}
and \Condor{schedd} on shutdown.

% gittrac #690
\item Fixed a bug in which the \Condor{master} would sometimes die and produce
a \File{dprintf\_failure.MASTER} file when either restarting due to new
binary timestamps or when started initially.

% This is from the unreleased 7.2.5.
\item Fixed a memory leak related to SOAP configuration variables
that occurred when Condor was reconfigured.

% This is from the unreleased 7.2.5
\item Fixed a bug in which the submit description file command
\Attr{cron\_day\_of\_week} was erroneously ignored.

% gittrac #580
\item Fixed bug in which the configuration variables
\Macro{MAX\_JOB\_QUEUE\_LOG\_ROTATIONS} and \Macro{GRIDMANAGER\_SELECTION\_EXPR}
would not work properly at start up; they only worked after a \Condor{reconfig}.

\item Fixed a bug in which SOAP operations were being incorrectly authorized
with the peer IP \Sinful{0.0.0.0}.

\item Fixed a Windows platform bug in which not all Condor daemons were trusted
by the Windows Firewall
(previously known as Internet Connection Firewall or ICF).

% Commented out by Karen until there is enough info to make the item's
% existence useful.
%\item Fixed a race condition in the parallel universe.

\item Fixed a shutdown race condition in the \Condor{master} with respect to
high availability daemons.

\item Fixed a bug in which a Condor daemon incorrectly determined it had
run out of socket descriptors.

% gitrac #552
\item Fixed a bug where the \Condor{schedd} would block for very long
periods of time while trying to connect to a down checkpoint server. Now
the \Condor{schedd} will do a blocking connect with a timeout to the
checkpoint server for a configurable number of seconds. If the connect
fails, the \Condor{schedd} will put a moratorium on connecting to the
checkpoint server until the configurable moratorium period passes. The
configuration file variables that describe this behavior are described
above.

% See gittrac #215
\item Changed the check that \Condor{dagman} does for other
\Condor{dagman} instances
running the same DAG, if it finds a lock file at startup.
Now, if \Condor{dagman} is not sure whether the other DAGMan is alive,
it continues, rather than exiting.

\item Fixed a major file descriptor leak in the Stork daemon.

\item Fixed a bug in which successful Stork transfers were marked as failed.

\item Fixed an uncommon memory leak in the user event log file reading code
when reading badly formatted events.

% gittrac 593
\item Fixed a bug in which multiple machine ClassAds in the
\Condor{collector} with the same \Attr{Name},
but different \Attr{StartdIPAddr} attribute values,
would cause the \Condor{negotiator} to exit with an error.
This is unusual and should not happen in a typical Condor installation.
The most likely cause is using \Condor{advertise}
to advertise custom ClassAds for grid matchmaking.

%gittrac #435
\item Fixed a bug that caused \Condor{dagman} to core dump if all
submit attempts failed on a DAG node having a POST script.
This bug has existed since Condor version 7.1.4.

\item Fixed a memory leak in the \Condor{schedd}, which occurred when
the configuration variable \MacroNI{NEGOTIATOR\_MATCH\_EXPRS} was used.

% gittrac #504
\item Fixed a bug in the Windows platform code that treats scripts as
  executables.
  Unknown file extensions were treated as an error,
  rather than as a Windows executable.

\item The \Condor{job\_router} now correctly sets the ClassAd attribute
\Attr{EnteredCurrentStatus} to the current time when creating a new routed job.
Previously, it copied this attribute from the original job.

\item The \Condor{job\_router} emits a more friendly log message when it
observes that the routed copy of the job was removed.

\item A fix has been made for a problem seen in 7.3.1 in which Condor daemons
using CCB to connect to other Condor daemons would sometimes consume
large amounts of CPU time for no good reason.

\item Fixed a rare failure case bug in which attempts to connect via
CCB could stay in a pending state indefinitely.

\item A Unix only bug caused Condor daemons to fail to start if
\MacroNI{MAX\_FILE\_DESCRIPTORS} was configured higher
than the current hard limit inherited by Condor.  If Condor is running
as root, this is no longer the case.

% gittrac %627
\item The \Condor{gridmanager} now advertises grid ClassAds properly when there
are multiple \Condor{collector} daemons.

%gittrack #647
\item When using \Condor{q} \Opt{-xml} and \Opt{-format} together to
limit the number of ClassAd attributes returned in the query, the XML
\verb@<classads>@ container tag was not generated.  This is fixed, but
now the preferred way to limit the returned attributes is to
use \Condor{q} option \Opt{-attributes}.

\item Fixed a bug in which the Unix \Condor{master} failed
when trying to restart itself,
if the configuration variable \MacroNI{MASTER\_LOCK} was defined,
or if the \Condor{master} was invoked with the \Opt{-t} option.
This bug has existed since the 7.0 series,
and likely has existed much longer than that.

%gittrac #575
\item Fixed a significant memory leak in the \Prog{gahp\_server}. This
leak was only present in previous Condor 7.3.x releases.

%gittrac #570
\item Fixed a bug that can cause a removed job that is held and then
released to return to idle status.

%gittrac #693
\item The Globus jar files distributed with the x86-64 RHEL 5 RPMs were
damaged, causing \SubmitCmd{gt4} grid type jobs to fail. This has been fixed.

\end{itemize}

\noindent Known Bugs:

\begin{itemize}

%gittrac #851
\item The version 7.3.2 \Condor{dagman} binary sometimes has problems running
rescue DAGs.  Probably the best work around for this problem is to use
version 7.3.1 rather than 7.3.2 \Condor{dagman} and \Condor{submit\_dag}
binaries, even if using version 7.3.2 for the rest of the Condor
installation.

\end{itemize}

\noindent Additions and Changes to the Manual:

\begin{itemize}

\item None.

\end{itemize}


%%%%%%%%%%%%%%%%%%%%%%%%%%%%%%%%%%%%%%%%%%%%%%%%%%%%%%%%%%%%%%%%%%%%%%
\subsection*{\label{sec:New-7-3-1}Version 7.3.1}
%%%%%%%%%%%%%%%%%%%%%%%%%%%%%%%%%%%%%%%%%%%%%%%%%%%%%%%%%%%%%%%%%%%%%%

\noindent Release Notes:

\begin{itemize}

\item None.

\end{itemize}


\noindent New Features:

\begin{itemize}

\item Added the STARTD\_HISTORY configuration parameter.  If set, this
is a pathname to a history file, just like the condor\_schedd maintains,
but only for jobs run on that startd.

\item Added the JavaSpecificationVersion attribute to startds which
support Java.  This allows users to request machines which support
a particular major version of Java, without specifying the exact
specific version.  So, Java versions 1.6.0\_01, 1.6.1\_02 and 1.6.2\_03
all advertise JavaSpecificationVersion of 1.6.

\item Implemented a performance increase to \Condor{dagman} which can
decrease the parsing times of DAG input files by up to 60 times.
This performance increase works for certain common DAG geometries.
This will help in submission and recovery
time for DAGs whose nodes have a very large number of dependency edges
associated with them.

\item \Condor{q} -analyze and -better-analyze now emit warnings
if the \Condor{schedd} will not run jobs when it is out of swap space or
has hit the limit imposed by the configuration variable
\MacroNI{MAX\_JOBS\_RUNNING}.

\item When matching Condor-G jobs to resources, if multiple jobs
match multiple resources, and every job has identical job rank, the
matchmaker would always fill up one particular resource first.  Now,
the resources will be matched in a round robin fashion.  This can be
overridden by setting job rank appropriately.

\item Made the \Condor{schedd} more efficient in how it stores
information about \verb@$$()@ expansions in the job ClassAd.
Also made the \Condor{schedd} more efficient in how it contacts
the \Condor{negotiator} to submit reschedule requests.

\item Improved the Job Router's heuristic for site throttle adjustment.  It
is now quicker to release the throttle when the failure rate drops
below the configured threshold.

\item Made the Job Router more efficient on startup by improving the way it
reads the job queue log file.

\item Added an accessor class to the user log reader API to allow the
  application to query about reader state, including the
  difference in the event numbers and log position of two states.  This
  can be used by the application to determine the number of events
  missed when missed events are detected.

\item Added the ability to throttle the rate at which jobs are
stopped via \Condor{rm}, \Condor{hold}, \Condor{vacate\_job},
and during a graceful shutdown of the \Condor{schedd} daemon.

\item In the configuration file, Condor now accepts expressions for
the values of configuration variables that are required to be 
numeric literals or boolean constants.  
Note that this does not imply that the
expressions may freely reference ClassAd values in places where they
could not before.  
See section~\ref{sec:Intro-to-Config-Files} for an example with
further explanation.

\end{itemize}

\noindent Configuration Variable Additions and Changes:

\begin{itemize}

\item Added the STARTD\_HISTORY configuration parameter.  If set, this
is a pathname to a history file, just like the condor\_schedd maintains,
but only for jobs run on that startd.

\item The new configuration variable \Macro{UPDATE\_OFFSET} 
  causes the \Condor{startd} to
  delay the initial (and all further) updates that it sends to the
  \Condor{collector}.  See \ref{param:UpdateOffset} for more details.

\item The new configuration variables
  \Macro{JOB\_STOP\_COUNT} and \Macro{JOB\_STOP\_DELAY}
  limit the rate at which jobs are stopped via \Condor{rm},
  \Condor{hold}, \Condor{vacate\_job}, and during a graceful shutdown of
  the \Condor{schedd} daemon.
  See \ref{param:JobStopCount} and \ref{param:JobStopDelay} 
  for full definitions.

\end{itemize}

\noindent Bugs Fixed:

\begin{itemize}

\item Fixed a problem with job removal in the local universe that 
  would cause spurious error messages to be written to the log of the
  \Condor{schedd} daemon.

\item The \Condor{schedd} was failing to send `reschedule' commands to
flocked negotiators, so unless some other schedd in the negotiator's
pool sent it a reschedule command, negotiation cycles would only
happen every \Macro{NEGOTIATOR\_INTERVAL}.

\end{itemize}

\noindent Known Bugs:

\begin{itemize}

\item When using CCB to connect to other Condor daemons, Condor 7.3.1
daemons can sometimes consume large amounts of CPU, potentially
causing performance problems.  Condor 7.3.0 did not suffer from this
problem.

\end{itemize}

\noindent Additions and Changes to the Manual:

\begin{itemize}

\item None.

\end{itemize}

%%%%%%%%%%%%%%%%%%%%%%%%%%%%%%%%%%%%%%%%%%%%%%%%%%%%%%%%%%%%%%%%%%%%%%
\subsection*{\label{sec:New-7-3-0}Version 7.3.0}
%%%%%%%%%%%%%%%%%%%%%%%%%%%%%%%%%%%%%%%%%%%%%%%%%%%%%%%%%%%%%%%%%%%%%%

\noindent Release Notes:

\begin{itemize}

\item This release is incompatible when communicating with
previous versions of Condor if CCB is enabled or if
\Macro{PRIVATE\_NETWORK\_NAME} is configured.

\item Updated the DRMAA version.
This new version is compliant with GFD.133,
the DRMAA 1.0 grid recommendation standard.
Three new functions were added to meet the specification's requirements,
and several bugs were fixed.

\end{itemize}


\noindent New Features:

\begin{itemize}

\item Added support for using any recognized script as an executable
in a submit file on Windows. For more information please see
section~\ref{sec:windows-scripts-as-executables} on
page~\pageref{sec:windows-scripts-as-executables}.

\item Improved support for private networks:
Added CCB, the Condor Connection Broker.  It is similar in
functionality to GCB, the Generic Connection Broker, but it has
several advantages, including ease of use and working on Windows as
well as Unix platforms.
GCB continues to work, but we may remove
it some time in the 7.3 development series.  The main missing feature
in CCB at the moment that prevents it from replacing GCB,
is support for connectivity from one private network to another.
CCB only works
when connecting from a public network to a private one.  For example,
jobs may be sent from a \Condor{schedd} on the public Internet to 
\Condor{startd} daemons on a
private network, if the \Condor{startd} daemons are configured
to use a CCB server that is accessible to the \Condor{schedd} daemon.
However, if the \Condor{schedd} daemon is on one private
network and the \Condor{startd} daemons are on a different private network,
CCB does not help.  For more information on CCB, see section~ \ref{sec:CCB}.

\item Added support for a CPU affinity on both Windows and Linux platforms.

\item Added support for the \Condor{q} \Opt{-better-analyze} option on Windows.

\item Added \MacroNI{WANT\_HOLD}.  When \MacroNI{PREEMPT} becomes
true, if \MacroNI{WANT\_HOLD} is true, the job is put on hold for the
reason (optionally) specified by \MacroNI{WANT\_HOLD\_REASON} and
\MacroNI{WANT\_HOLD\_SUBCODE}.  These policy expressions are evaluated
by the execute machine.  As usual, the job owner may specify
\AdAttr{periodic\_release} and/or \AdAttr{periodic\_remove}
expressions to react to specific hold states automatically.

\item Added the ClassAd function \Procedure{debug}.
See section~ \ref{sec:classadFunctions} for the details of this function.

% Commented out by Karen, as this is useless to a user taking the
% time to read a version history. More info is needed. 
%\item Log messages have been made more clear.
% Includes: Give a clear warning instead of a terse error, when lacking a COLLECTOR.

\item The \Condor{schedd} can now use MD5 check sums to avoid storing
multiple copies of the same executable in its \Macro{SPOOL} directory.
Note that this feature only affects executables sent to the
\Condor{schedd} via the \SubmitCmd{copy\_to\_spool} command within
a submit description file.

% gittrac #197
\item Reduced the number of sleeps \Condor{dagman} does to maintain log
file consistency when a DAG uses multiple user logs for node jobs.
DAGMan now does one sleep per submit cycle,
instead of one sleep for each submit.

% gittrac #166, #208
\item Added the \Opt{-import\_env} command-line flag to
\Condor{submit\_dag}.  This explicitly puts the submittor's environment
into the \File{.condor.sub} file.

\item Optimized the removal of large numbers of jobs.  
Previously, removal of tens of thousands of jobs caused the
\Condor{schedd} daemon to consume
a lot of CPU time for several minutes.

\item Reduced memory usage by the \Condor{shadow} daemon.  Since there is one
\Condor{shadow} process per running job, this helps increase the
number of running jobs that a submit machine can handle.  Under Linux 2.6,
we found that running 10,000 jobs from a single submit machine
requires about 10GBytes of system RAM.  We also found in this case that to
run more than 10,000 simultaneous jobs requires a 64-bit submit
machine.  On a 32-bit Linux platform, kernel memory is exhausted,
regardless of how much additional RAM the system has.

\item Reduced the memory usage of the \Condor{collector} daemon,
when \Expr{UPDATE\_COLLECTOR\_WITH\_TCP = True}.

\end{itemize}

\noindent Configuration Variable Additions and Changes:

\begin{itemize}

\item The new configuration variable \Macro{OPEN\_VERB\_FOR\_<EXT>\_FILES}
allows the default interpreter for scripts with an extension \textit{EXT} to
be changed.  For more information please see
section~\ref{sec:windows-scripts-as-executables} on
page~\pageref{sec:windows-scripts-as-executables}.

\item The new configuration variable \Macro{CCB\_ADDRESS}
configures a daemon to use one or more
CCB servers to allow communication with Condor components outside of
the private network.  See page~\pageref{sec:CCB}.

\item The new configuration variable \Macro{MAX\_FILE\_DESCRIPTORS}
(on Unix platforms only) specifies the
required file descriptor limit for a Condor daemon.  File descriptors
are a system resource used for open files and for network connections.
Condor daemons that make many simultaneous network connections may
require an increased number of file descriptors.  For example, see
page~\pageref{sec:CCB} for information on file descriptor requirements
of CCB.

\item The new configuration variables \Macro{ENFORCE\_CPU\_AFFINITY} and 
\Macro{SLOT<N>\_CPU\_AFFINITY} on Linux platforms allow for
Condor to lock slots to given CPUs.
Definitions for these variables are at \ref{param:EnforceCpuAffinity}.

\item The new configuration variable \Macro{DEBUG\_TIME\_FORMAT}
  allows a custom specification for the format of the time
  printed at the start of each line in a daemon's log file.
  See \ref{param:DebugTimeFormat} for the complete definition of
  this variable.

\item The new configuration variable \Macro{SHARE\_SPOOLED\_EXECUTABLES}
  is a boolean value that determines whether the \Condor{schedd} daemon will
  use MD5 check sums to avoid storing multiple copies of the same
  executable in the \MacroNI{SPOOL} directory. The default setting is
  \Expr{True}.

\item The new boolean configuration variable
  \Macro{EVENT\_LOG\_FSYNC} provides control of the behavior of
  Condor when writing events to the event log.  Previously,
  the behavior was as if this parameter were set to \Expr{False}.
  See \ref{param:EventLogFsync} for the complete definition of
  this variable.

\item The new boolean configuration variable
  \Macro{EVENT\_LOG\_LOCKING} provides control of the behavior of
  Condor when writing events to the event log.  Previously,
  the behavior was controlled by \MacroNI{ENABLE\_USERLOG\_LOCKING}.
  See \ref{param:EventLogLocking} for the complete definition of
  this variable.

\end{itemize}

\noindent Bugs Fixed:

\begin{itemize}

\item None.

\end{itemize}

\noindent Known Bugs:

\begin{itemize}

\item None.

\end{itemize}

\noindent Additions and Changes to the Manual:

\begin{itemize}

\item None.

\end{itemize}
