%%%      PLEASE RUN A SPELL CHECKER BEFORE COMMITTING YOUR CHANGES!
%%%      PLEASE RUN A SPELL CHECKER BEFORE COMMITTING YOUR CHANGES!
%%%      PLEASE RUN A SPELL CHECKER BEFORE COMMITTING YOUR CHANGES!
%%%      PLEASE RUN A SPELL CHECKER BEFORE COMMITTING YOUR CHANGES!
%%%      PLEASE RUN A SPELL CHECKER BEFORE COMMITTING YOUR CHANGES!

%%%%%%%%%%%%%%%%%%%%%%%%%%%%%%%%%%%%%%%%%%%%%%%%%%%%%%%%%%%%%%%%%%%%%%
\section{\label{sec:History-7-3}Development Release Series 7.3}
%%%%%%%%%%%%%%%%%%%%%%%%%%%%%%%%%%%%%%%%%%%%%%%%%%%%%%%%%%%%%%%%%%%%%%

This is the development release series of Condor.
The details of each version are described below.

%%%%%%%%%%%%%%%%%%%%%%%%%%%%%%%%%%%%%%%%%%%%%%%%%%%%%%%%%%%%%%%%%%%%%%
\subsection*{\label{sec:New-7-3-1}Version 7.3.1}
%%%%%%%%%%%%%%%%%%%%%%%%%%%%%%%%%%%%%%%%%%%%%%%%%%%%%%%%%%%%%%%%%%%%%%

\noindent Release Notes:

\begin{itemize}

\item None.

\end{itemize}


\noindent New Features:

\begin{itemize}

\item None.

\end{itemize}

\noindent Configuration Variable Additions and Changes:

\begin{itemize}

\item None.

\end{itemize}

\noindent Bugs Fixed:

\begin{itemize}

\item None.

\end{itemize}

\noindent Known Bugs:

\begin{itemize}

\item None.

\end{itemize}

\noindent Additions and Changes to the Manual:

\begin{itemize}

\item None.

\end{itemize}

%%%%%%%%%%%%%%%%%%%%%%%%%%%%%%%%%%%%%%%%%%%%%%%%%%%%%%%%%%%%%%%%%%%%%%
\subsection*{\label{sec:New-7-3-0}Version 7.3.0}
%%%%%%%%%%%%%%%%%%%%%%%%%%%%%%%%%%%%%%%%%%%%%%%%%%%%%%%%%%%%%%%%%%%%%%

\noindent Release Notes:

\begin{itemize}

\item Updated the DRMAA version.
This new version is compliant with GFD.133,
the DRMAA 1.0 grid recommendation standard.
Three new functions were added to meet the specification's requirements,
and several bugs were fixed.

\end{itemize}


\noindent New Features:

\begin{itemize}

\item Added support for using any recognized script as an executable
in a submit file on Windows. For more information please see
section~\ref{sec:windows-scripts-as-executables} on
page~\pageref{sec:windows-scripts-as-executables}.

\item Added CCB, the Condor Connection Broker.  It is similar in
functionality to GCB, the Generic Connection Broker, but it has
several advantages, including ease of use and working on Windows as
well as Unix platforms.
GCB continues to work, but we may remove
it some time in the 7.3 development series.  The main missing feature
in CCB at the moment that prevents it from replacing GCB,
is support for connectivity from one private network to another.
CCB only works
when connecting from a public network to a private one.  For example,
jobs may be sent from a \Condor{schedd} on the public internet to 
\Condor{startd} daemons on a
private network, if the \Condor{startd} daemons are configured
to use a CCB server that is accessible to the \Condor{schedd} daemon.
However, if the \Condor{schedd} daemon is on one private
network and the \Condor{startd} daemons are on a different private network,
CCB does not help.  For more information on CCB, see page~\pageref{sec:CCB}.

\item Added support for CPU affinity on Linux.

\item Added support for \Condor{q}'s \Opt{-better-analyze} on Windows.

\item Added \MacroNI{WANT\_HOLD}.  When \MacroNI{PREEMPT} becomes
true, if \MacroNI{WANT\_HOLD} is true, the job is put on hold for the
reason (optionally) specified by \MacroNI{WANT\_HOLD\_REASON} and
\MacroNI{WANT\_HOLD\_SUBCODE}.  These policy expressions are evaluated
by the execute machine.  As usual, the job owner may specify
\AdAttr{periodic\_release} and/or \AdAttr{periodic\_remove}
expressions to react to specific hold states automatically.

\item Added the ClassAd function \Procedure{debug}.
See section~ \ref{sec:classadFunctions} for the details of this function.

\item Log messages have been made more clear.
% Includes: Give a clear warning instead of a terse error, when lacking a COLLECTOR.

\item The \Condor{schedd} can now use MD5 checksums to avoid storing
multiple copies of the same executable in its \Macro{SPOOL} directory.
Note that this feature only affects executables sent to the
\Condor{schedd} via the \SubmitCmd{copy\_to\_spool} submit macro.

% gittrac #197
\item Reduced the number of sleeps \Condor{dagman} does to maintain log
file consistency when a DAG uses multiple user logs for node jobs.
(DAGMan now does one sleep per submit cycle instead of one sleep for
each submit.)

% gittrac #166, #208
\item Added the \Opt{-import\_env} command-line flag to
\Condor{submit\_dag}.  This explicitly puts the submittor's environment
into the \File{.condor.sub} file.

\end{itemize}

\noindent Configuration Variable Additions and Changes:

\begin{itemize}

\item \MacroNI{OPEN\_VERB\_FOR\_<EXT>\_FILES} has been added to allow
the default interpreter for scripts with an extension \textit{EXT} to
be changed.  For more information please see
section~\ref{sec:windows-scripts-as-executables} on
page~\pageref{sec:windows-scripts-as-executables}.

\item \MacroNI{CCB\_ADDRESS} configures a daemon to use one or more
CCB servers to allow communication with Condor components outside of
the private network.  See page~\pageref{sec:CCB}.

\item \MacroNI{MAX\_FILE\_DESCRIPTORS} (UNIX only) specifies the
required file descriptor limit for a Condor daemon.  File descriptors
are a system resource used for open files and for network connections.
Condor daemons that make many simultaneous network connections may
require an increased number of file descriptors.  For example, see
page~\pageref{sec:CCB} for information on file descriptor requirements
of CCB.

\item The new configuration variables \Macro{ENFORCE\_CPU\_AFFINITY} and 
\Macro{SLOTx\_CPU\_AFFINITY} have been added on Linux to allow for
condor to lock slots to given CPUs.

\item The new configuration variable \Macro{DEBUG\_TIME\_FORMAT}
  allows a custom specification for the format of the time
  printed at the start of each line in a daemon's log file.
  See \ref{param:DebugTimeFormat} for the complete definition of
  this variable.

\item The new configuration variable \Macro{SHARE\_SPOOLED\_EXECUTABLES}
  is a boolean value that determines whether the \Condor{schedd} will
  use MD5 checksums to avoid storing multiple copies of the same
  executable in the \Macro{SPOOL} directory. The default setting is
  \Expr{True}.

\end{itemize}

\noindent Bugs Fixed:

\begin{itemize}

\item This release is incompatible when communicating with previous
versions of Condor if the new CCB feature is enabled or if
\Macro{PRIVATE\_NETWORK\_NAME} is configured.

\end{itemize}

\noindent Known Bugs:

\begin{itemize}

\item None.

\end{itemize}

\noindent Additions and Changes to the Manual:

\begin{itemize}

\item None.

\end{itemize}
