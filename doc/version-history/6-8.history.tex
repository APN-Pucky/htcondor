%%%%%%%%%%%%%%%%%%%%%%%%%%%%%%%%%%%%%%%%%%%%%%%%%%%%%%%%%%%%%%%%%%%%%%
\section{\label{sec:History-6-8}Stable Release Series 6.8}
%%%%%%%%%%%%%%%%%%%%%%%%%%%%%%%%%%%%%%%%%%%%%%%%%%%%%%%%%%%%%%%%%%%%%%

This is a stable release series of Condor.
It is based on the 6.7 development series.
All new features added or bugs fixed in the 6.7 series are available
in the 6.8 series.
As usual, only bug fixes (and potentially, ports to new platforms)
will be provided in future 6.8.x releases.
New features will be added in the forthcoming 6.9.x development series.

%%%%%%
% we need a summary of major new features since 6.6.x here.  trying to
% sort through the 21 different 6.7.x releases and all the new
% features is a huge amount of noise.  people just want to see a
% summary of the major new functionality.
% \Todo
%%%%%%

The 6.8.x series supports a different set of platforms than 6.6.x.
Please see the updated table of available platforms in
section~\ref{sec:Availability} on page~\pageref{sec:Availability}.

The details of each version are described below.


%%%%%%%%%%%%%%%%%%%%%%%%%%%%%%%%%%%%%%%%%%%%%%%%%%%%%%%%%%%%%%%%%%%%%%
\subsection*{\label{sec:New-6-8-1}Version 6.8.1}
%%%%%%%%%%%%%%%%%%%%%%%%%%%%%%%%%%%%%%%%%%%%%%%%%%%%%%%%%%%%%%%%%%%%%%

\noindent Release Notes:

\begin{itemize}

\item The PCRE (Perl Compatible Regular Expressions) library used by
Condor is now dynamically linked and shipped as a DLL with Condor for
Windows, rather than being statically linked.

\end{itemize}


\noindent New Features:

\begin{itemize}

% Gnats PR 610
\item Added an optional argument to the \Condor{dagman} ABORT-DAG-ON
command that allows the DAGMan exit code to be specified separately
from the node value that causes the abort; also, a DAG can now be
aborted on a zero exit code from a node.

\end{itemize}

\noindent Bugs Fixed:

\begin{itemize}

\item Fixed a Quill bug that prevented it from running on Windows.  The
symptom was errors in the QuillLog like
\begin{verbatim}
POLLING RESULT: ERROR
\end{verbatim}

\item Fixed a bug in Quill where it would cause errors like
\begin{verbatim}
duplicate key violates unique constraint "history_vertical_pkey"
\end{verbatim}
in the QuillLog and the postgres log file.  These errors would then trigger
a significant slowdown in the performance of Quill and the database.  This
would only happpen when a job attribute would change type from a string
type to a numeric type, or vice versa.

\item When a large number of jobs (roughly 200 or more) are running from a
single schedd and those jobs are using job leases (the default in 6.8), it is
possible the schedd to enter a state where it crashes on startup until all of
the job leases expire.  This bug is fixed.
% The bug is more generic; it potentially hit any user of GenericQuery or
% CondorQuery, but we know of no users encountering the bug in any other cases.

%% This is the change to datathread.C
\item In those unusual cases where Condor is unable to create a new process,
it will shut down cleanly, eliminating a small possibility of data corruption.

\item Fixed a bug with the gt4 and nordugrid grid universe job types that
caused the stdout and stderr of a job to not be transferred correctly if
the filenames given had absolute paths.

% Gnats PR 711
\item \Condor{dagman} now echos warnings from \condor{submit} and
stork\_submit to the \File{dagman.out} file.

\item Fixed a bug introduced in 6.7.20 causing \Condor{ckpt\_server}
to exit immediately after starting up, unless Condor's security
negotation was disabled.

% This was a bug because the default configuration file and manual
% both claimed it defaulted to 1MB.
\item \Macro{MAX\_<SUBSYS>\_LOG} defaults to one megabyte, even if the
setting is missing from the configuration.  Previously it was 64 kilobytes.

% this is the change to condor_secman.C by zmiller
\item Fixed a bug related to non-blocking connect that could occasionally
cause Condor daemons to crash.

% collector.C change: eliminated useless fprintf(stderr).
\item Fixed a rare bug where an exceptionally large query to the
\Condor{collector} could cause it to crash.  The most common cause is a single
schedd restarting and trying to recover a large number of job leases at once.
More than approximately 250 running jobs on a single schedd would be necessary
to trigger this bug.

\item When using the \Macro{JOB\_PROXY\_OVERRIDE\_FILE} configuration
parameter, the X509 proxy will now be properly forwarded for Condor-C jobs.

\item Greatly reduced the chance that a Condor-C job in the REMOVED state
will become HELD due to an expired proxy or failure to talk to the remote
\Condor{schedd}.

\item Fixed a number of error and debug messages added in 6.7.20 that
were incorrectly reporting IP and port numbers.  These messages were
intended to report the peer's address, but they were reporting the
local address of the network socket instead.

\item Fixed the problems relating to credential cache problems in the Kerberos
authentication mechanism.  The current version of Kerberos is 1.4.3.

\item Some of the binaries required to use Condor-C on Windows were
mistakenly not included in previous releases of Condor. This has been
fixed.

\item Fixed problem on Windows where the \Condor{startd} could fail to
include some attributes in its ClassAd. This would result in some jobs
incorrectly not being matched to that machine.  This only happened if
\Macro{CREDD\_HOST} was defined and Condor daemons on the execute
machine were unable to authenticate with the \Condor{credd}.

\item Fixed a \Condor{dagman} bug which had prevented the
\MacroU{DAGManJobId} attribute from being expanded in job submit files
(e.g., when used as the value of the \Macro{Priority} command).

\end{itemize}

\noindent Known Bugs:

\begin{itemize}

\item None.

\end{itemize}




%%%%%%%%%%%%%%%%%%%%%%%%%%%%%%%%%%%%%%%%%%%%%%%%%%%%%%%%%%%%%%%%%%%%%%
\subsection*{\label{sec:New-6-8-0}Version 6.8.0}
%%%%%%%%%%%%%%%%%%%%%%%%%%%%%%%%%%%%%%%%%%%%%%%%%%%%%%%%%%%%%%%%%%%%%%

\noindent Release Notes:

\begin{itemize}

\item The default configuration for Condor now requires that
\Macro{HOSTALLOW\_WRITE} be explicitly set.  Condor will refuse
to start if the default configuration is used unmodified.
Existing installations should not need to change anything.  For
those who desire the earlier default, you can set it to "*", but
note that this is potentially a security hole allowing anyone to
submit jobs or machines to your pool.

\item Most Linux distributions are now supported using dynamically
  linked binaries built on a RedHat Enterprise Linux 3 machine.
  Recent security patches to a number of Linux distributions have
  rendered the binaries built on RedHat 9 machines ineffective.
  The download pages have been changed to reflect this, but Linux users
  should be aware of this change.
  The recommended download for most x86 Linux users is now:
  \File{condor-6.8.0-linux-x86-rhel3-dynamic.tar.gz}.

\item Some log messages have been clarified or moved to different
  debugging levels.
  For example, certain messages that looked like errors were printed
  to \MacroNI{D\_ALWAYS}, even though nothing was wrong and the system was
  behaving as expected.

\item The new features and bugs fixed in the rest of this section only
  refer to changes made since the 6.7.20 release, not the last stable
  release (6.6.11).
  For a complete list of changes since 6.6.11, read the 6.7 version
  history in section~\ref{sec:History-6-7} on
  page~\pageref{sec:History-6-7}. 

\end{itemize}


\noindent New Features:

\begin{itemize}

\item Version 1.4 of the Condor DRMAA libraries are now included 
  with the Condor release.
  For more information about DRMAA, see section~\ref{API-DRMAA} on
  page~\pageref{API-DRMAA}.

\item Version 1.0.15 of the Condor GAHP is now used for Condor-G and
  Condor-C. 

% Gnats PR 710
\item Added the \Opt{-outfile\_dir} command-line argument to
\Condor{submit\_dag}.  This allows you to change the directory in which
\Condor{dagman} writes the \File{dagman.out} file.

\item Added a new \Opt{--summary} (also \Opt{-s}) option to the
\Condor{update\_stats} tool.  If enabled, this prevents it from
displaying the entire history for each machine and only displays the
summary info.

\end{itemize}

\noindent Bugs Fixed:

\begin{itemize}

\item Fixed a number of potential static buffer overflows in various
  Condor daemons and libraries.

\item Fixed some small memory leaks in the \Condor{startd},
  \Condor{schedd}, and a potential leak that effected all Condor
  daemons.

\item Fixed a bug in Quill which caused it to crash when certain
long attributes appeared in a job ad.

\item The startd would crash after a reconfig if the address of a
collector had not been resolved since the previous reconfig
(e.g. because DNS was down during that time).

\item Once a Condor daemon failed to lookup the IP address of the
collector (e.g. because DNS was down), it would fail to contact the
collector from that time until the next reconfig.  Now, each time Condor
tries to contact the collector, it generates a fresh DNS query if the
previous attempt failed.

% Gnats PR 707
\item When using Condor-C or the -s or -r command-line options to
\condor{submit}, the job's standard output and error would be placed
in the job's initial working directory, even if the job ad said to
place them in a different directory.

% Gnats PRs 501 and 663
\item Greatly sped up the parsing of large DAGs (by a factor of 50
or so) by using a hash table instead of linear search to find DAG nodes.

% Gnats PR 697
\item Fixed a bug in \Condor{dagman} that caused an EXECUTABLE\_ERROR
event from a node job to abort the DAG instead of just marking the
relevant node as failed.

\item Fixed a bug in \Condor{collector} that caused it to discard
machine ads that don't have an IP address field (either StartdIpAddr
or STARTD\_IP\_ADDR).  The \Condor{startd} will always produce a
StartdIpAddr field, but machine ads published through
\Condor{advertise} may not.

\item When using \MacroNI{BIND\_ALL\_INTERFACES} on a dual-homed
machine, a bug introduced in 6.7.18 was causing Condor daemons to
sometimes incorrectly report their IP addresses, which could cause
jobs to fail to start running.

\item Made the event checking in \Condor{dagman} less strict: 
added the new "allow duplicate events" value to the
\MacroNI{DAGMAN\_ALLOW\_EVENTS} macro (this value is part of the
default); 16 value now also allows terminate event before submit;
changed "allow all events" to "allow almost all events"
(all except "run after terminal event"), so it is more useful.

% Gnats PR 712
\item \Condor{dagman} and \Condor{submit\_dag} now report
\Opt{-NoEventChecks} as ignored rather than deprecated.

\item Fixed a bug in the \Condor{dagman} \Opt{-maxidle} feature:
a shadow exception event now puts the corresponding job into the
idle state in \Condor{dagman}'s internal count.

\item Fixed a problem on Windows where daemons would sometimes crash
when dealing with UNC path names.

\item Fixed a problem where the \Condor{schedd} on Windows would
incorrectly reject a job if the client provided an \Attr{Owner}
attribute that was correct but differed in case from the authenticated
name.

\item Fixed a \Condor{startd} crash introduced in version 6.7.20. This
crash would appear if an execute machine was matched for preemption
but then not claimed in time by the appropriate \Condor{schedd}.

\item Resolved an issue where the \Condor{startd} was unable to clean
up jobs' execute directories on Windows when the \Condor{master} was
started from the command line rather than as a service.

\item Added more patches to Condor's DRMAA interface to make it more
compatible with Sun Grid Engine's DRMAA interface.

\item Removed the unused \MacroNI{D\_UPDOWN} debug level and added the
  \MacroNI{D\_CONFIG} debug level.

\item Fixed a bug that caused \Condor{q} with the \Opt{-l} or \Opt{-xml}
arguments to print out duplicate attributes when using Quill.

\item Fixed a bug that prevented Condor-C jobs (universe grid jobs of type condor)
from submitting correctly if \MacroNI{QUEUE\_ALL\_USERS\_TRUSTED} is set to
True.

\item Fixed a bug that could cause the \Condor{negotiator} to crash if the
pool contains several different versions of the \Condor{schedd} and in the
config file \MacroNI{NEGOTIATOR\_MATCHLIST\_CACHING} is set to True.

\item Changed the default value for config file entry
\MacroNI{NEGOTIATOR\_MATCHLIST\_CACHING} from False to True.  When set to
True, this will instruct the negotiator to safely cache data in order to
improve matchmaking performance.

\item The Condor{master} now recognizes \Condor{quill} as a valid
  Condor daemon without any manual configuration on the part of site
  administrators.
  This simplifies the configuration changes required to enable Quill. 

\item Fixed a rare bug in the \Condor{starter} where if there was a
  failure transferring job output files back to the submitting host,
  it could hang indefinitely, and the job appeared as if it was
  continuing to run.

\end{itemize}


\noindent Known Bugs:

\begin{itemize}

\item There are known scalability problems when using Condor's Kerberos
authentication mechanism in large pools.  If your installation of Condor is
more than a couple dozen machines, and you need to use Kerberos for Condor
authentication, we recommend you wait for Condor version 6.8.1 or use Condor
version 6.7.17 (which does not suffer from these problems).

\item There are known problems with Condor's SSL authentication mechanism.
While the HTTPS support in Condor (which also uses SSL) works fine for the
SOAP/Birdbath interface, there are bugs with the SSL support when SSL is
listed in \MacroNI{SEC\_DEFAULT\_AUTHENTICATION\_METHODS}.  We expect to fix
these issues for version 6.8.1.

\end{itemize}

