%%%%%%%%%%%%%%%%%%%%%%%%%%%%%%%%%%%%%%%%%%%%%%%%%%%%%%%%%%%%%%%%%%%%%%
\section{\label{sec:History-6-8}Stable Release Series 6.8}
%%%%%%%%%%%%%%%%%%%%%%%%%%%%%%%%%%%%%%%%%%%%%%%%%%%%%%%%%%%%%%%%%%%%%%

This is a stable release series of Condor.
It is based on the 6.7 development series.
All new features added or bugs fixed in the 6.7 series are available
in the 6.8 series.
The details of each version are described below.


%%%%%%%%%%%%%%%%%%%%%%%%%%%%%%%%%%%%%%%%%%%%%%%%%%%%%%%%%%%%%%%%%%%%%%
\subsection*{\label{sec:New-6-8-1}Version 6.8.1}
%%%%%%%%%%%%%%%%%%%%%%%%%%%%%%%%%%%%%%%%%%%%%%%%%%%%%%%%%%%%%%%%%%%%%%

\noindent Release Notes:

\begin{itemize}

\item None.

\end{itemize}


\noindent New Features:

\begin{itemize}

\item None.

\end{itemize}

\noindent Bugs Fixed:

\begin{itemize}

\item Fixed a bug in Quill where it would cause errors like
\begin{verbatim}
duplicate key violates unique constraint "history_vertical_pkey"
\end{verbatim}
in the QuillLog and the postgres log file.  These errors would then trigger
a significant slowdown in the performance of Quill and the database.  This
would only happpen when a job attribute would change type from a string
type to a numeric type, or vice versa.

\item When a large number of jobs (roughly 200 or more) are running from a
single schedd and those jobs are using job leases (the default in 6.8), it is
possible the schedd to enter a state where it crashes on startup until all of
the job leases expire.  This bug is fixed.
% The bug is more generic; it potentially hit any user of GenericQuery or
% CondorQuery, but we know of no users encountering the bug in any other cases.

%% This is the change to datathread.C
\item In those unusual cases where Condor is unable to create a new process,
it will shut down cleanly, eliminating a small possibility of data corruption.

\end{itemize}

\noindent Known Bugs:

\begin{itemize}

\item None.

\end{itemize}




%%%%%%%%%%%%%%%%%%%%%%%%%%%%%%%%%%%%%%%%%%%%%%%%%%%%%%%%%%%%%%%%%%%%%%
\subsection*{\label{sec:New-6-8-0}Version 6.8.0}
%%%%%%%%%%%%%%%%%%%%%%%%%%%%%%%%%%%%%%%%%%%%%%%%%%%%%%%%%%%%%%%%%%%%%%

\noindent Release Notes:

\begin{itemize}

\item None.

\end{itemize}


\noindent New Features:

\begin{itemize}

% Gnats PR 710
\item Added the \Opt{-outfile\_dir} command-line argument to
\Condor{submit\_dag}.  This allows you to change the directory in which
\Condor{dagman} writes the \File{dagman.out} file.

\item Added a new \Opt{--summary} (also \Opt{-s}) option to the
\Condor{update\_stats} tool.  If enabled, this prevents it from
displaying the entire history for each machine and only displays the
summary info.

\item The default configuration for Condor now requires that
\Macro{HOSTALLOW\_WRITE} be explicitly set.  Condor will refuse
to start if the default configuration is used unmodified.
Existing installations should not need to change anything.  For
those who desire the earlier default, you can set it to "*", but
note that this is potentially a security hole allowing anyone to
submit jobs or machines to your pool.

\end{itemize}

\noindent Bugs Fixed:

\begin{itemize}

\item Fixed a bug in Quill which caused it to crash when certain
long attributes appeared in a job ad.

\item The startd would crash after a reconfig if the address of a
collector had not been resolved since the previous reconfig
(e.g. because DNS was down during that time).

\item Once a Condor daemon failed to lookup the IP address of the
collector (e.g. because DNS was down), it would fail to contact the
collector from that time until the next reconfig.  Now, each time Condor
tries to contact the collector, it generates a fresh DNS query if the
previous attempt failed.

% Gnats PR 707
\item When using Condor-C or the -s or -r command-line options to
\condor{submit}, the job's standard output and error would be placed
in the job's initial working directory, even if the job ad said to
place them in a different directory.

% Gnats PRs 501 and 663
\item Greatly speeded up the parsing of large DAGs (by a factor of 50
or so) by using a hash table instead of linear search to find DAG nodes.

% Gnats PR 697
\item Fixed a bug in \Condor{dagman} that caused an EXECUTABLE\_ERROR
event from a node job to abort the DAG instead of just marking the
relevant node as failed.

\item Fixed a bug in \Condor{collector} that caused it to discard
machine ads that don't have an IP address field (either StartdIpAddr
or STARTD\_IP\_ADDR).  The \Condor{startd} will always produce a
StartdIpAddr field, but machine ads published through
\Condor{advertise} may not.

\item When using \MacroNI{BIND\_ALL\_INTERFACES} on a dual-homed
machine, a bug introduced in 6.7.18 was causing Condor daemons to
sometimes incorrectly report their IP addresses, which could cause
jobs to fail to start running.

\item Made the event checking in \Condor{dagman} less strict: 
added the new "allow duplicate events" value to the
\MacroNI{DAGMAN\_ALLOW\_EVENTS} macro (this value is part of the
default); 16 value now also allows terminate event before submit;
changed "allow all events" to "allow almost all events"
(all except "run after terminal event"), so it is more useful.

% Gnats PR 712
\item \Condor{dagman} and \Condor{submit\_dag} now report
\Opt{-NoEventChecks} as ignored rather than deprecated.

\item Fixed a bug in the \Condor{dagman} \Opt{-maxidle} feature:
a shadow exception event now puts the corresponding job into the
idle state in \Condor{dagman}'s internal count.

\item Fixed a problem on Windows where daemons would sometimes crash
when dealing with UNC path names.

\item Fixed a problem where the \Condor{schedd} on Windows would
incorrectly reject a job if the client prvoided an \Attr{Owner}
attribute that was correct but differed in case from the authenticated
name.

\item Fixed a \Condor{startd} crash introduced in version 6.7.20. This
crash would appear if an execute machine was matched for preemption
but then not claimed in time by the appropriate \Condor{schedd}.

\item Resolved an issue where the \Condor{startd} was unable to clean
up jobs' execute directories on Windows when the \Condor{master} was
started from the command line rather than as a service.

\end{itemize}

\noindent Known Bugs:

\begin{itemize}

\item None.

\end{itemize}

