%%%%%%%%%%%%%%%%%%%%%%%%%%%%%%%%%%%%%%%%%%%%%%%%%%%%%%%%%%%%%%%%%%%%%%
\section{\label{sec:gotchas}Upgrade Surprises}
%%%%%%%%%%%%%%%%%%%%%%%%%%%%%%%%%%%%%%%%%%%%%%%%%%%%%%%%%%%%%%%%%%%%%%

\index{upgrading!items to be aware of}
Occasional changes to Condor can cause unexpected errors or results
to users.
Here is a list of changes to note and be aware of.

\begin{itemize}

\item 
We believe that Condor 7.0.x and 6.8.x are wire-compatible,
and can be freely mixed between computers in a Condor pool. 
However, we do not regularly test this compatibility and cannot guarantee it, 
so we recommend using a single release of Condor when possible. 
Please note that although you can mix Condor 7.0.x and 6.8.x in a pool, 
you cannot mix them on a single computer. 
That is, a \Condor{master} daemon running 6.8.x cannot run Condor daemons 
from version 7.0.x, or vice-versa.

\item When upgrading from 6.6.x 6.7.x,
  jobs that rely on the environment variable \Env{CONDOR\_SCRATCH\_DIR}
  need to be changed to use \Env{\_CONDOR\_SCRATCH\_DIR}.
  An underscore was added to the beginning of this variable.

\item A necessary change was introduced in version 6.7.15, such that
  standard universe jobs are better identified and have more
  correct resumption semantics on other machines with proper platform
  identification signatures (which contain things such as operating system,
  OS distribution, and kernel memory model). This change affects all
  standard universe jobs that remain in the job queue across an upgrade
  from any Condor release previous to 6.7.15 to any Condor release of
  6.7.15 or more recent.

  The resulting policy changes are documented in
  section~\ref{sec:Is-Valid-Checkpoint-Platform}.

  Suggestions for dealing with this backwards compatibility issue
  are given in the Frequently Asked Questions in
  section~\ref{sec:checkpoint-platform-faq}.


\end{itemize}

