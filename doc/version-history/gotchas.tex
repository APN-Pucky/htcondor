%%%%%%%%%%%%%%%%%%%%%%%%%%%%%%%%%%%%%%%%%%%%%%%%%%%%%%%%%%%%%%%%%%%%%%
\section{\label{sec:gotchas}Upgrading from the 7.4 series to the 7.6 series of Condor}
%%%%%%%%%%%%%%%%%%%%%%%%%%%%%%%%%%%%%%%%%%%%%%%%%%%%%%%%%%%%%%%%%%%%%%

\index{upgrading!items to be aware of}
While upgrading from the 7.4 series of Condor to the 7.6 series will
bring many new features and improvements introduced in the 7.5 series of
Condor, it will also introduce changes that administrators of sites
running from an older Condor version should be aware of when
planning an upgrade.
Here is a list of items that administrators should be aware of.

\begin{itemize}

\item  Previously, Condor's RPM packaging installed into \File{/opt}.
  Now, the RPM install paths adhere to the Linux Standard Base File System
  Hierarchy Standard (LSB-FHS), meaning Condor will be installed by default
  into appropriate subdirectories within \File{/usr}.  
  See section~\ref{sec:install-rpms} for updated RPM information. 

\item  The feature set once known as the Startd Cron or as Hawkeye is now
  called \Term{Daemon ClassAd Hooks}.  Besides the new name, the mechanisms
  have been updated, and the configuration has changed.
  Sites using this functionality are strongly encouraged to
  update their configuration file to adhere to the new syntax. 
  Configuration file syntax introduced in Condor version 7.2 
  is still compatible with Condor version 7.6,
  but the syntax of Condor version 7.0 is unlikely to work as expected.
  See section~\ref{sec:daemon-classad-hooks} for documentation.

\item  For those who compile Condor from the source code, 
  rather than using packages of pre-built executables, 
  Condor is now built using \Prog{cmake} on all platforms.
  This means the process for building Condor from source code has changed.
  For instructions, 
  see the Condor Wiki at \URL{http://wiki.condorproject.org}, and follow 
  the Wiki link to the section on Building and Testing Condor.

\item  Quill and Stork are no longer included in the binary packages
  released from the Condor Project. Instead, both are available as
  \Term{Contribution Modules}. 
  Contribution Modules are optional packages that add functionality to Condor,
  but are provided and maintained outside of the core code base, 
  often by third parties. 
  To add Quill and/or Stork to Condor, 
  see the Condor Wiki at \URL{http://wiki.condorproject.org}, follow 
  the Wiki link, and see the section titled User FAQs.

\item The layout in the \MacroNI{SPOOL} directory has changed.
  As a result,
  be certain to upgrade all Condor binaries on a given host at the same time.
  That is, do not try to run Condor version 7.6 \Condor{schedd} daemons 
  and Condor version 7.4 \Condor{shadow} daemons on the same machine.
  Also, should you need to revert back to an older version of
  Condor after running Condor version 7.6 (hopefully not the case!),
  be aware of release notes for Condor version 7.5.5 regarding the 
  \File{SPOOL} directory in section~\ref{sec:New-7-5-5}.

\item Issues regarding the detection of keyboard activity on Linux and
  especially on Windows Vista/7 have been addressed. Bottom line: if you
  want Condor to detect keyboard/mouse activity on either of these
  platforms, ensure that the subsystem \Expr{KBDD} is listed in the
  \MacroNI{DAEMON\_LIST} configuration variable definition on start up.

\item On Windows platforms,
  the MSI installer has been rewritten (using WiX) to better
  support Windows 7. While the installation process is essentially the same,
  a handful of arguments that control an \emph{unattended} MSI installation
  have changed, such as the arguments controlling HDFS installation.  See
  section~\ref{sec:nt-unattended-install-procedure} for the updated list of
  unattended installation arguments.

\item The submit description file commands for grid jobs using a
\SubmitCmd{grid\_resource} type of amazon has changed. Users will need to
edit their submit description files for jobs destined for Amazon to
include an EC2 URL. See section~\ref{sec:Amazon-submit}.

\end{itemize}

