%%%      PLEASE RUN A SPELL CHECKER BEFORE COMMITTING YOUR CHANGES!
%%%      PLEASE RUN A SPELL CHECKER BEFORE COMMITTING YOUR CHANGES!
%%%      PLEASE RUN A SPELL CHECKER BEFORE COMMITTING YOUR CHANGES!
%%%      PLEASE RUN A SPELL CHECKER BEFORE COMMITTING YOUR CHANGES!
%%%      PLEASE RUN A SPELL CHECKER BEFORE COMMITTING YOUR CHANGES!

%%%%%%%%%%%%%%%%%%%%%%%%%%%%%%%%%%%%%%%%%%%%%%%%%%%%%%%%%%%%%%%%%%%%%%
\section{\label{sec:History-7-4}Stable Release Series 7.4}
%%%%%%%%%%%%%%%%%%%%%%%%%%%%%%%%%%%%%%%%%%%%%%%%%%%%%%%%%%%%%%%%%%%%%%

This is a stable release series of Condor.
As usual, only bug fixes (and potentially, ports to new platforms)
will be provided in future 7.4.x releases.
New features will be added in the 7.5.x development series.

The details of each version are described below.

%%%%%%%%%%%%%%%%%%%%%%%%%%%%%%%%%%%%%%%%%%%%%%%%%%%%%%%%%%%%%%%%%%%%%%
\subsection*{\label{sec:New-7-4-3}Version 7.4.3}
%%%%%%%%%%%%%%%%%%%%%%%%%%%%%%%%%%%%%%%%%%%%%%%%%%%%%%%%%%%%%%%%%%%%%%

\noindent Release Notes:

\begin{itemize}

\item None.

\end{itemize}


\noindent New Features:

\begin{itemize}

\item None.

\end{itemize}

\noindent Configuration Variable and ClassAd Attribute Additions and Changes:

\begin{itemize}

\item None.

\end{itemize}

\noindent Bugs Fixed:

\begin{itemize}

\item None.

\end{itemize}

\noindent Known Bugs:

\begin{itemize}

\item None.

\end{itemize}

\noindent Additions and Changes to the Manual:

\begin{itemize}

\item None.

\end{itemize}


%%%%%%%%%%%%%%%%%%%%%%%%%%%%%%%%%%%%%%%%%%%%%%%%%%%%%%%%%%%%%%%%%%%%%%
\subsection*{\label{sec:New-7-4-2}Version 7.4.2}
%%%%%%%%%%%%%%%%%%%%%%%%%%%%%%%%%%%%%%%%%%%%%%%%%%%%%%%%%%%%%%%%%%%%%%

\noindent Release Notes:

\begin{itemize}

\item None.

\end{itemize}


\noindent New Features:

\begin{itemize}

\item None.

\end{itemize}

\noindent Configuration Variable and ClassAd Attribute Additions and Changes:

\begin{itemize}

\item None.

\end{itemize}

\noindent Bugs Fixed:

\begin{itemize}

% gittrack #1167
\item Fixed a problem with GSI authentication in Condor that would cause
daemons to consume more and more memory over time.  The biggest source
of trouble was introduced in Condor version 7.3.2.
However, a smaller memory leak that
existed in all previous versions of Condor has also been fixed.

% gitrack #553
\item Fixed a bug where if \Condor{compile} is invoked in a manner such as:
\begin{verbatim}
  condor_compile gcc -print-prog-name=ld 
\end{verbatim}
an error would be emitted,
and \Condor{compile} would exit with a bad exit code.

% gittrack #1093
\item The sort based on \Condor{status} output accidentally changed in 
Condor version 7.3,
so that the output was based on the slot name first, then machine name.
The behavior is now restored to the original sorting: first on machine name,
then slot name.

% gittrack #728
\item If one machine running a parallel job crashed,
and job leases are enabled (which they are by default),
the job would not exit until the job lease duration expired.
As the \Condor{starter} will not get respawned,
there is no need to wait.
Many sites set long job lease durations,
to prevent jobs from being killed when the machine running
the \Condor{schedd} daemon reboots.
Now, if one node goes away, the whole computation is shut down immediately.

\item Fixed the verbosity level of some \Condor{dagman} messages written to
the \File{dagman.out} file.

% gittrack #1137
\item Fixed a bug introduced in Condor version 7.3.2 that resulted in
  messages such as the following even in cases where no problem in
  communicating with the \Condor{collector} had been encountered:

\begin{verbatim}
Collector <X> is still being avoided if an alternative succeeds.
\end{verbatim}

This problem was believed to be fixed in Condor 7.4.1, but some cases
of the problem remained in that version.

% gittrack 1160
\item Fixed a bug from Condor version 6.1.14,
that resulted in the \Condor{schedd} performing
the operation scheduled via \MacroNI{WALL\_CLOCK\_CKPT\_INTERVAL} at the
specified frequency (default time of 1 hour),
multiplied by the number of times the
\Condor{schedd} daemon had been reconfigured during its lifetime.
This could lead to degraded performance,
especially prior to Condor version 7.4.1,
when this operation was more disk-intensive.

% gittrack 1184
% gittrack 1181
\item 32-bit Linux versions of Condor running in a 64-bit environment would
sometimes not detect the existence of some processes and sometimes
wrongly detect that a tracked process belonged to root when it
actually belonged to some other user.  This could lead to failure to run
jobs or failure to properly monitor and clean up after them.  When the wrong
process ownership problem happened,
the following message appeared in the \Condor{master} and/or \Condor{procd}
logs:

\begin{verbatim}
ProcAPI: fstat failed in /proc! (errno=75)
\end{verbatim}

If \Condor{procd} failed to detect the existence of its own parent process,
it would exit with the following message in its log:

\begin{verbatim}
ERROR: master has exited
\end{verbatim}

% gittrack 1186
\item Fixed a problem in the \Condor{job\_router} daemon,
  introduced in Condor version 7.2.2,
  that could cause the daemon to crash when failing to carry out the change
  of state dictated by a job's periodic policy expressions,
  for example, the failure to put a job on hold when \AdAttr{periodic\_hold}
  becomes \Expr{True}.

% gittrac #1209
\item Fixed a bug introduced in Condor 7.3.2 that caused Grid Monitor
jobs to receive a full X.509 proxy. Now, it always receives a limited
proxy, which was the previous behavior.

% gittrac #1070
\item Fixed a bug that could cause the nordugrid\_gahp to crash.

\end{itemize}

\noindent Known Bugs:

\begin{itemize}

\item None.

\end{itemize}

\noindent Additions and Changes to the Manual:

\begin{itemize}

\item Descriptions of all the commands that may be placed into a
submit description file are now located within the \Condor{submit}
manual page, instead of within Chapter 2, the Users' Manual.

\end{itemize}


%%%%%%%%%%%%%%%%%%%%%%%%%%%%%%%%%%%%%%%%%%%%%%%%%%%%%%%%%%%%%%%%%%%%%%
\subsection*{\label{sec:New-7-4-1}Version 7.4.1}
%%%%%%%%%%%%%%%%%%%%%%%%%%%%%%%%%%%%%%%%%%%%%%%%%%%%%%%%%%%%%%%%%%%%%%

\noindent Release Notes:

\begin{itemize}

% gittrac #1018
\item \Security A flaw was found that could allow a user who already is authorized to
submit jobs into Condor, to queue a job under the guise of  a different
user.  In this way, someone who has access to a Condor submission
service and is allowed to submit jobs into Condor could gain access to
another non-root or non-administrator account on the system.
This flaw was discovered during the development process; no incidents
have been reported.  Details of the problem will be made available on Feb 1st,
2010.

% gittrack #918
\item The default value of \MacroNI{JOB\_ROUTER\_NAME} has changed
  from an empty string to \verb|jobrouter| in order to address
  problems caused by the previous default.  Without special handling,
  this means that jobs being managed by \Condor{job\_router} before
  upgrading will not be adopted by the new version of
  \Condor{job\_router} if the default \MacroNI{JOB\_ROUTER\_NAME} was
  being used.  To correct this, follow the instructions given in the
  description of \MacroNI{JOB\_ROUTER\_NAME} on
  page~\pageref{JobRouterName}.

\end{itemize}


\noindent New Features:

\begin{itemize}

% gittrac #921, #999
\item Condor allows submit files to specify an \SubmitCmd{IwdFlushNFSCache}
expression,
to control whether or not Condor tries to flush the NFS cache for 
a job's initial working directory on job completion.

% gittrac #929, #943
\item The new \Opt{-attributes} option to \Condor{status}
  explicitly specifies the attributes to be listed when using the
  \Opt{-xml} or \Opt{-long} options.

\end{itemize}

\noindent Configuration Variable and ClassAd Attribute Additions and Changes:

\begin{itemize}

% gittrac #161, #935, #936
\item New VOMS attributes have been introduced into the job ad to keep them
separate from the X509UserProxySubjectName.

\item The default for \MacroNI{JOB\_ROUTER\_NAME} has changed from an
  empty string to \verb|jobrouter|.  See the release notes for more
  information about upgrading from an old version.

\item The configuration variable \Macro{TCP\_FORWARDING\_HOST}
  has existed in Condor since version 7.0.0, but was not documented.
  See section~\ref{param:TcpForwardingHost} for documentation.

% gittrac #933
\item The new configuration variable \MacroNI{STARTD\_PER\_JOB\_HISTORY\_DIR}
allows ClassAds of completed jobs to be stored in a directory separate 
from the existing one specified with \MacroNI{PER\_JOB\_HISTORY\_DIR}.

\end{itemize}

\noindent Bugs Fixed:

\begin{itemize}

% gittrac #749
\item  Condor no longer creates the job sandbox in its \MacroNI{SPOOL}
directory if it is not needed.

% gittrac #1019
\item Fixed a problem introduced in Condor version 7.4.0 that caused GSI
authentication between Condor processes to fail with using a
non-legacy format X.509 proxy.

% gittrack #1028
\item Fixed a problem with CCB under Windows platforms that has existed since
Condor version 7.3.0.  
This problem caused CCB-enabled daemons to become unresponsive
after the exit of a child process.

% gittrac #931 -- Fixed minor spelling errors, not worthy of listing.

% gittrac #923
\item Improved the handling of previously-submitted gt2 grid jobs upon
release from hold, when there is no Globus job manager for the job running
on the remote resource.

% gittrac #453
\item Fixed a problem with job leases for jobs that use a \Condor{shadow}.
Previously, while these jobs were running, lease renewals from the 
submitter would not be
noticed, and the job would be aborted when the original lease expired.

% gittrac #870
\item Fixed a bug that only allowed approximately 50 splices to be included into
a DAG input file. There is now no limit to the number of splices
one may include into a DAG input file except, of course, for the
implicit memory allocation limit of the \Condor{dagman} process.

% gittrac #909
\item Removed attempted limiting of swap space via \Prog{ulimit -v} using the
\Attr{VirtualMemory} machine ClassAd attribute in the script
\File{condor\_limits\_wrapper.sh}.

% gittrack #899
\item Fixed a bug that caused \MacroNI{ALLOW\_CONFIG} and
  \MacroNI{HOSTALLOW\_CONFIG}, as well as the corresponding
  \MacroNI{DENY} configuration variables to incorrectly handle a
  setting consisting of a single \Expr{*} or the equivalent \Expr{*/*}.  This
  also fixes a bug that caused incorrect merging of \MacroNI{ALLOW}
  and \MacroNI{HOSTALLOW} settings when one, but not both, consisted of
  a single \Expr{*} or the equivalent \Expr{*/*}.
  These bugs have existed since before Condor version 6.8.

% gittrack #905
\item Fixed a bug introduced in Condor version 7.3.0 that could cause 
Condor daemons to crash when reading malformed network addresses.

% gittrac #883
\item Removed a check for root ownership of a script specified by
the configuration variable \MacroNI{VM\_SCRIPT}.

% gittrac #884
\item Fixed a bug in writing the header of the file identified by
the configuration variable \MacroNI{EVENT\_LOG}.

% gittrac #891
\item Fixed a bug that could cause the \Condor{startd} to segfault on shutdown
when using dynamic slots.

% gittrack #871
\item Fixed a problem introduced in Condor version 7.3.2 that changed 
  the behavior of
  an undocumented method for selecting attributes to be displayed in
  \Condor{q} \Opt{-xml}.  Prior to this bug, the following command
  would produce XML output with the attributes \Attr{A} and \Attr{B},
  plus a few other attributes that were always shown.

\begin{verbatim}
condor_q -xml -format "%s" A+B
\end{verbatim}

In Condor versions 7.3.2 and 7.4.0,
this same command produced an empty XML ClassAd.
The workaround was to use multiple \Opt{-format} options, each listing
just one desired attribute, rather than a single one with an
expression of all desired attributes.  Although this is now fixed, the
more straightforward way to select attributes since Condor version 7.3.2
is to use the \Opt{-attributes} option.

% gittrack #907
\item Fixed a bug introduced in Condor version 7.3.2 that resulted in 
  messages such
  as the following even in cases where no problem in communicating
  with the \Condor{collector} had been encountered:

\begin{verbatim}
Collector <X> is still being avoided if an alternative succeeds.
\end{verbatim}

% gittrack #859
\item Fixed a bug that has been in the \Condor{startd} since before
  Condor version 6.8.  If the \Condor{startd} ever failed to send signals to the
  \Condor{starter} process, it could fail to properly clean up the
  machine ClassAd, leaving attributes from
  \MacroNI{STARTD\_JOB\_EXPRS} in the ClassAd but not making them visible
  in \Condor{status} queries.  One possible problem resulting from
  this could be matches being made by the \Condor{negotiator} that are then
  rejected by the \Condor{startd}.  Repeated messages such as the following
  would then result in the \Condor{startd} log:

\begin{verbatim}
slot1: Request to claim resource refused.
\end{verbatim}

%gittrack #908
\item Fixed a problem that resulted in the following message in the
  \Condor{startd} log:

\begin{verbatim}
Timer -1 not found
\end{verbatim}

%gittrack #937
\item Fixed a problem in which security sessions were not cached
  correctly when using CCB.  This resulted in re-authentication in
  some cases where a cached security session could have been used.

% gittrac #161, #935, #936, #1020
\item Fixed multiple problems with the handling of VOMS attributes in GSI
proxies.

% gittrac #934
\item Fixed a bug that caused \Condor{dagman} to hang when running a
DAG with POST scripts, if the global event log is turned on.

% gittrack #973
\item Improved how the private network address is published when using
  the configuration variables \MacroNI{PRIVATE\_NETWORK\_NAME} and
  \MacroNI{PRIVATE\_NETWORK\_INTERFACE}.  In some cases, this
  information was not being used, and therefore connections were made
  to the public address when they could have been made to the private
  address.

%gittrac #801
\item Fixed a bug exhibited under Windows XP,
where using \MacroNI{USE\_VISIBLE\_DESKTOP}
would cause strange behavior after a job completed.

% gittrack #713
\item CCB now works with \MacroNI{TCP\_FORWARDING\_HOST}.  Previously,
  the reverse connection was made to the private address rather than
  to the host defined by \MacroNI{TCP\_FORWARDING\_HOST}.

% gittrac #852
\item Removed a bad optimization that caused some information about job
execution to be lost during job completion or removal,
if a history file was not configured.

% gittrac #893
\item Condor now checks whether the configuration variable
\MacroNI{GRIDFTP\_URL\_BASE} is set before
submitting cream grid jobs, as that variable is required for cream jobs
to function properly. If the variable is not set, cream jobs are put on
hold with an appropriate message.

% gittrac #920
\item Fixed a bug that allowed running virtual machines to be leaked
if the \Condor{startd} crashed.

% gittrac #912
\item Fixed a bug in \Prog{cream\_gahp} which could cause crashes when
there were more than 500 cream jobs queued.

% gittrac #972
\item Improved recovery when Condor crashes during the submission of a cream
grid job. Before, affected jobs would remain in \Expr{REGISTERED} state
on the cream server, but never run.

% gittrac #954
\item Improved the \Attr{HoldReason} message when cream grid jobs are
held by the \Condor{gridmanager}.

% gittrac #895
\item When naming a resource for a cream grid job, Condor now properly
recognizes the format used by the standard cream client UI:
\File{https://foo.edu:8443/cream-pbs-cream\_queue}.

% gittrac #795
% The memory leak is not worth documenting.
\item The configuration variable \MacroNI{SOAP\_SSL\_CA\_FILE} is now 
consulted in addition to
\MacroNI{SOAP\_SSL\_CA\_DIR} when authenticating
an https proxy for Amazon EC2, when \MacroNI{AMAZON\_HTTP\_PROXY} is defined.

% gittrac #485
\item Previously, if \Condor{rm} and friends were given both a constraint
and a user name or cluster id, they would act on all jobs matching the
constraint and all jobs associated with the user or cluster. Now, this
combination of arguments results in an error.

% gittrac #1062
\item Failure to purge a cream grid universe job from the remote server
because it was previously purged no longer results in the job being held.

% gittrac #1044
\item The \Condor{gridmanager} now recognizes VOMS attributes in X.509
proxies and will handle them appropriately. For example, it recognizes
that two proxies with the same identity but different VOMS attributes may
be mapped to different accounts on a remote machine.

% gittrac #947
% Not documenting, as the parameter being removed was added for a specifc
% customer and never documented.

% gittrac #932
% Not documenting, as the bug hasn't caused any problems.

% gittrac #1043
% Not documenting, as the problem never made it into a release.

% gittrac #979
\item Fixed a bug in \Condor{dagman}, introduced in 7.3.2, that will
cause \Condor{dagman} running on Windows to hang on any DAG using
more than one log file for the node jobs.

% gittrac #967
\item Fixed a bug in \Condor{dagman}, introduced in 7.3.2, that could
cause \Condor{dagman} to fail on a DAG using node job log files on
multiple devices, if log files on different devices happened to have
the same inode number.

% gittrac #981
\item Fixed a bug that caused the \Condor{schedd} daemon to segfault when
spooling more than 9 files.

% gittrac #1011
\item Fixed a bug that caused the \Condor{startd} daemon to crash on
Debian Stable.

% gittrac #1033
\item Fixed keyboard activity detection on the Windows XP platform.

% gittrac #1068
\item Fixed a bug in the \Condor{had} daemon that caused it to not start
the controlled daemon if CCB was enabled.

\end{itemize}

\noindent Known Bugs:

\begin{itemize}

% gittrac #983
\item \Condor{dagman} may fail on Windows if the set of node job log
file names includes multiple paths that are hard links (not symbolic links)
to the same file.

% gittrac #1081
\item \Condor{dagman} PRE and POST script arguments (and the names of
the scripts themselves) cannot contain spaces.

% gittrac #1082
\item \Condor{dagman} VARS values cannot contain single quotes.

\end{itemize}

\noindent Additions and Changes to the Manual:

\begin{itemize}

% gittrac #725
\item Added documentation about how to include spaces (and other
special characters) in \Condor{dagman} VARS values.

\end{itemize}


%%%%%%%%%%%%%%%%%%%%%%%%%%%%%%%%%%%%%%%%%%%%%%%%%%%%%%%%%%%%%%%%%%%%%%
\subsection*{\label{sec:New-7-4-0}Version 7.4.0}
%%%%%%%%%%%%%%%%%%%%%%%%%%%%%%%%%%%%%%%%%%%%%%%%%%%%%%%%%%%%%%%%%%%%%%

\noindent Release Notes:

\begin{itemize}

\item The default configuration file within the release now uses
  \MacroNI{ALLOW}/\MacroNI{DENY} in place of
  \MacroNI{HOSTALLOW}/\MacroNI{HOSTDENY} for security related settings.
  We recommend making this
  same change throughout all configuration files.  That way,
  a policy that depends on the default policy should continue to
  work as it did before.  The behavior of these configuration variables
  remains unchanged.  The \MacroNI{ALLOW}/\MacroNI{DENY} lists are
  added to the \MacroNI{HOSTALLOW}/\MacroNI{HOSTDENY} lists to form the
  authorization policy.  Both styles support the same syntax.  
  This change permits an anticipated
  phasing out of the \MacroNI{HOSTALLOW}/\MacroNI{HOSTDENY}  configuration
  variables, in order to simplify configuration.

\item As of Condor version 7.3.2, \Condor{q} \Opt{-xml} output no longer 
  begins with the non-XML consisting of two blank lines followed by a line
  of the following form:

\begin{verbatim}
-- Submitter: schedd-name : <IP> : hostname
\end{verbatim}

\item All \Prog{Stork} data placement is now supported by the Stork
project at the 
LSU Center for Computation and Technology
(\URL{http://www.cct.lsu.edu/www.cct.lsu.edu}).
Please see the home page of the Stork project at
\URL{http://www.cct.lsu.edu/~kosar/stork/index.php} for details and
software.

\end{itemize}


\noindent New Features:

\begin{itemize}

\item Condor is now integrated with the Hadoop Distributed File System (HDFS). 
See documentation in section~\ref{sec:Condor-HDFS} and 
section~\ref{sec:HDFS-Config-File-Entries}.

% commit af65de7ccc1a281c2b05b8f68ac70bcfa56b2dd1
\item \Condor{q} using the options \Opt{-analyze} and \Opt{-better-analyze}
  now provide analysis for scheduler and local universe jobs.
  Specifically, the \MacroNI{START\_SCHEDULER\_UNIVERSE} and
  \MacroNI{START\_LOCAL\_UNIVERSE} expressions are checked.

% #824
\item Added the ClassAd attributes
\Attr{TotalLocalRunningJobs}, \Attr{TotalLocalIdleJobs},
\Attr{TotalSchedulerRunningJobs}, and \Attr{TotalSchedulerIdleJobs}
to the published ClassAd for the \Condor{schedd}.  This means that
\Condor{q} \Opt{-analyze} can still give helpful information about
why local or scheduler universe jobs are idle when
the configuration variables \MacroNI{START\_LOCAL\_UNIVERSE} or
\MacroNI{START\_SCHEDULER\_UNIVERSE} refer to these attributes.
These attributes were already present internally within 
the \Condor{schedd} daemon, 
just not published.

% #688
\item The \Condor{vm-gahp} now supports KVM and links with libvirt, rather 
than calling virsh command-line tools.

% #760 #771 #769 #772 #773 #775
\item Greatly improved the \Condor{gridmanager}'s scalability when handling
many grid type gt2 grid universe jobs.  Improvements include more quickly
processing updated X.509 certificates, not checking jobs for status updates if 
they have not been submitted to the remote site, and eliminating unnecessary 
updates to the \Condor{schedd} daemon.

% commit 75f6b2fe920b88717712a0d41765d16665ad7fe6
\item Latency in the submission and cleaning up of Condor-C jobs
has been improved by changing the default value of
\Macro{C\_GAHP\_CONTACT\_SCHEDD\_DELAY} from 20 to 5.

% commit 8c2d88c695d6981be3bdab7e10c5d92e9f6bb55b
\item The \Expr{eval()} ClassAd function added in Condor version 7.3.2
is now also understood by the \Condor{job\_router} and
\Condor{q} using the \Opt{-better-analyze} option.

\item The submit command \SubmitCmd{run\_as\_owner} is now supported
for Unix platforms. Previously, it was only supported on Windows platforms.

% #795
\item When setting \MacroNI{AMAZON\_HTTP\_PROXY}, a username and password
can now be given as part of the proxy URL.
The value of \MacroNI{SOAP\_SSL\_CA\_DIR} is now consulted when authenticating
an https proxy for Amazon EC2, when \MacroNI{AMAZON\_HTTP\_PROXY} is defined.

% #694
\item The \Condor{collector} daemon now advertises to itself, and will appear
in the output of \Condor{status} \Opt{-collector}.

% #775, cf02764d9d0fdd2b36ef3629f862f856ee41a717, and more
\item Optimizations in core Condor systems should provide minor speed 
improvements.

% 823
\item Updated the maximum log size to the maximum operating system's file size.

\end{itemize}

\noindent Configuration Variable and ClassAd Attribute Additions and Changes:

\begin{itemize}

% commit 0e8800c201f81eac54cba925b3d7f6d81a83aeca
\item The undocumented configuration variable 
  \Macro{TOOLS\_PROVIDE\_OLD\_MESSAGES} is no longer recognized by Condor.

% #768
\item The new configuration variable 
  \Macro{SCHEDD\_JOB\_QUEUE\_LOG\_FLUSH\_DELAY} sets an
  upper bound in seconds on how long it takes for changes to the job
  ClassAd to be visible to the Condor Job Router and to Quill.
  The default value is 5 seconds.
  Previously, there was no upper limit.  Typically, other activity in
  the job queue, such as jobs being submitted or completed would cause
  buffered data to be flushed to disk, such that the effective upper bound was
  a function of how busy the job queue was.

% commit 55525e0a338be8b2ba2d9173ce832e43d05413c3
\item The default configuration file now uses
  \MacroNI{ALLOW}/\MacroNI{DENY} in place of
  \MacroNI{HOSTALLOW}/\MacroNI{HOSTDENY}.  See the release notes above
  for more information.

% commit 7199e217f9228082a8465b85aaee18c2ebb19787
\item The default value for \Macro{MAX\_JOBS\_RUNNING} has changed.
  Previously, it was 200.  Now it is defined by an expression that depends 
  on the total amount of memory and the operating system.  The default
  expression requires 1MByte of RAM per running job, on the submit machine.
  In some environments and configurations, this is overly
  generous and can be cut by as much as 50\%.  Under Windows, the
  number of running jobs is still capped at 200.
  A 64-bit version of Windows  is recommended in order to raise the value
  above the default.
  Under Unix, the maximum default is now 10,000.  To scale higher, we
  recommend that the system ephemeral port range is extended
  such that there are at least 2.1 ports per running job.

% #767 commit 18296bfdfa92f16684a73d8d57a54d231b48dc33
\item The default value of \MacroNI{RESERVED\_SWAP} has changed to
  the value 0, which
  disables the \Condor{schedd} daemon's check for sufficient swap space
  before starting more jobs.  The new expression defined with 
  \MacroNI{MAX\_JOBS\_RUNNING} has a more appropriate memory check, so
  the configuration variable \MacroNI{RESERVED\_SWAP} will no longer
  be used in the near future.
  For cases where 
  \MacroNI{RESERVED\_SWAP} is not set to 0, the default value
  of \MacroNI{SHADOW\_SIZE\_ESTIMATE} has changed to 800 Kbytes.
  Previously, it was 200 if not set,
  but it was set to 1800 in the example configuration file.

% #767 commit c80e8a40e67ef4faa4e2b32b3671877eae1e1a19
\item The default values of \Macro{START\_LOCAL\_UNIVERSE} and
  \Macro{START\_SCHEDULER\_UNIVERSE} have changed.  Previously,
  these were set to \Expr{True}.  Now, they are set using an expression
  that allows
  up to 200 local universe and 200 scheduler universe jobs to run.

% #767 commit c4f4d911a808e1bdb18552e1cdeb0407b6344969
\item The default value of
  \Macro{GRIDMANAGER\_MAX\_SUBMITTED\_JOBS\_PER\_RESOURCE} has
  changed from 100 to 1000.

% #767 commit 9e6dfa463c71c28c8dc2c0c0c215b51d6762e811
% commit b4fd08ad1a8c69da24c371565796ef73522a61fc
\item The default value of \Macro{NEGOTIATOR\_INTERVAL}
   has changed from 300 to 60.

% #767 commit 8b91877ec8186810887402e1dd1df07b6341ade7
% Probably at least one other commit
\item The default value of \Macro{ENABLE\_GRID\_MONITOR} has been
  changed from \Expr{False} to \Expr{True}.  This variable
  was assigned to \Expr{True} in the example configuration file, so
  the change in default value now matches the value set in the example
  configuration.

% #631
\item The configuration variable \MacroNI{VM\_VERSION} has been removed,
as has the machine ClassAd attribute of the same name.
When the virtual machine version information is needed in the machine ClassAd,
the configuration variable \MacroNI{STARTD\_ATTRS} can be used to
add it.
 
% #861
\item The default configuration now uses
  \MacroNI{VM\_BRIDGE\_SCRIPT} and \MacroNI{VM\_SCRIPT} in place of
  \MacroNI{XEN\_BRIDGE\_SCRIPT} and \MacroNI{XEN\_SCRIPT} due to the
  support of KVM. 
  Submit description file commands have also been added, and they include:
  \SubmitCmd{kvm\_disk}, \SubmitCmd{kvm\_transfer\_files},
   and \SubmitCmd{kvm\_cd\_rom\_device}.

% #872
\item The configuration variables \MacroNI{XEN\_DEFAULT\_KERNEL}
  and \MacroNI{XEN\_DEFAULT\_INITRD} have been removed.
  Corresponding to this, the submit description file command
  \Expr{xen\_kernel = any} is no longer valid.

\end{itemize}

\noindent Bugs Fixed:

\begin{itemize}

\item Fixed a bug that prevented parallel universe jobs from running 
  on \Condor{startd} daemons with dynamic slots enabled.

% #706
\item Fixed a race condition bug in the \Condor{startd} which allowed
it to send Unix signals, intended for \Condor{starter} processes, as
root to non-Condor related processes.

% 735
\item A Windows platform bug has been fixed.
The bug caused a 20-second interval in which
the \Condor{shadow}, \Condor{startd}, and \Condor{starter} daemons
appeared as deadlocked. 
The bug was visible if a job ClassAd update from the \Condor{starter} caused
the job's periodic hold or remove policy to become \Expr{True}.

%gittrac #622
\item Fixed a bug that could cause \Condor{dagman} to generate an
illegal rescue DAG, if it read events incorrectly in recovery mode.
\Condor{dagman} now checks for events that violate DAG semantics
when reading events in recovery mode, and it exits without creating a
rescue DAG if it reads such an event.

% gittrac #744
\item Fixed a bug that could cause \Condor{dagman} to abort if it saw
the combination of a terminated event and an aborted event on a node with
retries.

% commit 5039a08cf00b0d0fafcc3fd8241518d1854ec3a1
\item Changed some logged warnings in \Condor{dagman} to not be
printed at the default verbosity setting.

% gittrac #825
\item The version compatibility checking between a \File{.condor.sub}
file and the \Condor{dagman} binary which is done at DAG startup
is now much more permissive.
Currently, \File{.condor.sub} files with
Condor versions of 7.1.2 and later accepted by \Condor{dagman}.

% gittrac #851
\item Fixed a bug introduced with the new \Condor{dagman} lazy log file
evaluation code in Condor version 7.3.2.
The bug sometimes caused failure when running rescue DAGs.

% #211 commit d6c0144739000523e94205a192be3cf9afe9ca9f
\item Fixed a bug originating in Condor version 7.1.4.
When a user submitted a job
with an executable that did not have execute permission enabled,
Condor was running as root, and file transfer was specified in the job,
Condor failed to automatically turn on execute permission after
transferring the file.

% commit 3bb847691bfda4f26d2f570bed1a412fb3afb439
\item Fixed a bug that appeared in Condor version 7.3.2.
The configuration variable
\MacroNI{COUNT\_HYPERTHREAD\_CPUS} was ignored and was effectively
treated as \Expr{False} in all cases.

% #761
\item Fixed a bug in which the Condor Job Router was not able
to see matchmaking diagnostic attributes such as \Attr{LastRejMatchTime}.
Therefore, when evaluating policy
expressions that referred to these attributes, they were effectively
treated as though \Expr{Undefined}.
Quill was also not able to see these attributes.

% #822
\item Fixed a bug introduced in Condor version 7.3.2 that could cause the
\Condor{gridmanager} to crash repeatedly on startup,
if the job queue
contained grid type gt2 jobs that had been previously submitted.

% #724, #774, #786
\item Fixed two bugs introduced in Condor version 7.3.2,
and related to VOMS. 
The first bug
prevented jobs with X.509 proxies from being submitted on platforms
on which Condor does not support VOMS.
The second bug prevented submission
of jobs with VOMS proxies, if the authenticity of the VOMS extensions
could not be verified.
At the same time, improved memory usage when VOMS extensions are not used.

\item Fixed a bad default in the file \File{batch\_gahp.config},
that prevented
Condor from observing job state changes for grid universe jobs
with a grid type of pbs or lsf.

% #748
\item Fixed a bug that caused Condor-C jobs to fail if
the submit description file command \SubmitCmd{transfer\_executable}
was set to \Expr{False}.

% #784
\item Fixed a bug that caused Condor-C jobs to fail if the executable
or one of the \File{stdin}, \File{stdout}, or \File{stderr} file names
contained a comma.

% #460
\item File transfer for grid type gt4 jobs requires an empty directory
within \File{/tmp}, which the \Condor{gridmanager} creates. 
If this directory is deleted, the \Condor{gridmanager} will now recreate it.

%gittrac #790
\item Fixed a bug that could cause the user job log to become
  corrupted on Windows platforms.  This bug would manifest itself only if the
  same log file was specified with different paths.  For example, the
  following submit file could have triggered this bug:
\begin{verbatim}
...
initialdir = /data/job1
log = ../JobLog
queue

initialdir = /data/job2
log = ../JobLog
queue
\end{verbatim}


% commit a26fcd9fe54cd3920fe777d5d8e0b2ffefbc3b1f
\item Fixed a memory leak introduced into Condor version 7.3.2.
The leak was in the \Condor{collector} daemon.

% commit 1663b7e183e6bf1df8152af98d9387412c2ae146
\item Fixed a bug introduced in Condor version 7.3.2
that resulted in the \Condor{negotiator} daemon
refusing to run, if the configuration variable \MacroNI{GROUP\_QUOTA}
for any group was set to 0.

% gittrac #731
\item Fixed a bug that caused the \Code{ctime} in the event log header
  to always be zero.

% #862 commit 9a432e2f3497e5dce120db5c733e79212257f6a5
\item Fixed the output of \Condor{status} when used with the command-line
  options \Opt{-java} or \Opt{-vm}.

\item Fixed a problem in the \Condor{schedd} daemon introduced in
  7.3.2.  For \Condor{schedd} daemons with lots of jobs having periodic release
  expressions, this bug could result in the \Condor{schedd} taking a long
  time while evaluating periodic expressions, causing it to become
  unresponsive to queries and other tasks.
  With a job queue of 30,000 jobs,
  a period of unresponsiveness of an hour was observed,
  whereas the evaluation of periodic expressions in this same environment
  normally takes less than 5 seconds.

\item Potential bugs and memory leaks were identified and 
fixed throughout Condor.  The Condor Team is not aware of anyone having 
encountered these bugs.

% #692 commit 8bc6bb4e06f11b2fdca28214d98c68c34c0ab9a4
\item The \Condor{starter} cleans up working directories in more
situations.  Previously during some error conditions, the working
directory within \MacroUNI{EXECUTE} might be left behind.

% #692 commit 8bc6bb4e06f11b2fdca28214d98c68c34c0ab9a4
\item If the user log cannot be accessed when a local universe
job starts, the job would fail and immediately be retried.  Now
the job is placed on hold.

% 826 
\item Fixed a bug in the \Condor{startd} in which vacating jobs would not 
respect the value of \Attr{JobLeaseDuration}.

% 802
\item Updated the detection of \Attr{HasVM} within the \Condor{startd}
 to publish an update to the \Condor{collector},
 when the configuration variable \MacroNI{VM\_RECHECK\_INTERVAL} is specified.

% commit 68f06088fa36eb0eb332a4f72a5c48ccd48b1d5a
\item Fixed a bug in which the \Condor{gridmanager} could, in rare cases,
waste a
small amount of memory and processor time checking for proxy files no longer
being used by any active jobs.

% commit bc66aa432e1f4e69d88a5b769204a4fce0648bfc
\item The setting \Macro{CREAM\_GAHP} was added to the default configuration 
file with a value of \File{\$(SBIN)/cream\_gahp}.
Existing installations desiring to 
submit jobs to CREAM should add this setting.

% #702
\item Fixed a bug where \Condor{restart} would fail on a \Condor{collector}
daemon configured for high availability with multiple \Condor{collector}
daemons.

% commit f44a68fb351e528ea5b251dd2c3cf9767b0c1fba
\item Fixed a bug in which multiple entries of output from 
the command
\Condor{status} \Opt{-negotiator}
would be on a single line.  They are now listed one per line.

% #778
\item Fixed a bug in which the command
\Condor{submit} \Opt{-dump} would crash if multiple
jobs were queued from within a single submit file.

% #742
\item Fixed a bug in which a slot whose associated job disappeared
could remain in the Claimed/Idle state until the claim lease expired.
The slot should now promptly return to the Unclaimed/Idle state.

% commit 0d5e3ad8fc85f0cd0dc58f73b503c76c0ad49bc4
\item Fixed a bug in which a \Condor{startd} using dynamic slots could
crash on shutdown or reconfiguration.



\end{itemize}

\noindent Known Bugs:

\begin{itemize}

% gittrac #161, #935, #936, #1020
\item There are multiple bugs related to using VOMS attributes.
In Condor version 7.4.0, VOMS support should be disabled by setting
the configuration variable \Expr{USE\_VOMS\_ATTRIBUTES = FALSE}.

\item A configuration variable of  \Macro{USE\_VISIBLE\_DESKTOP} set 
to \Expr{True} will corrupt the visible desktop.
  This bug is present back through Condor version 7.2.4.
This configuration variable did not work at all in 7.2 releases
prior to 7.2.4.  This bug will be fixed in Condor version 7.4.1.

% gittrac #934
\item If the global event log (see section~\ref{param:EventLog}) is
turned on, \Condor{dagman} will hang when running any DAG that has
POST scripts.

% gittrac #979
\item \Condor{dagman} will hang on Windows when running any DAG that
uses more than one log file for the node jobs.

\end{itemize}

\noindent Additions and Changes to the Manual:

\begin{itemize}

\item See section~\ref{sec:Condor-HDFS} and 
section~\ref{sec:HDFS-Config-File-Entries} for preliminary documentation of
Condor's integration with the Hadoop Distributed File System (HDFS). 

\end{itemize}

