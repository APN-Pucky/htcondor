%%%      PLEASE RUN A SPELL CHECKER BEFORE COMMITTING YOUR CHANGES!
%%%      PLEASE RUN A SPELL CHECKER BEFORE COMMITTING YOUR CHANGES!
%%%      PLEASE RUN A SPELL CHECKER BEFORE COMMITTING YOUR CHANGES!
%%%      PLEASE RUN A SPELL CHECKER BEFORE COMMITTING YOUR CHANGES!
%%%      PLEASE RUN A SPELL CHECKER BEFORE COMMITTING YOUR CHANGES!

%%%%%%%%%%%%%%%%%%%%%%%%%%%%%%%%%%%%%%%%%%%%%%%%%%%%%%%%%%%%%%%%%%%%%%
\section{\label{sec:History-7-4}Stable Release Series 7.4}
%%%%%%%%%%%%%%%%%%%%%%%%%%%%%%%%%%%%%%%%%%%%%%%%%%%%%%%%%%%%%%%%%%%%%%

This is a stable release series of Condor.
As usual, only bug fixes (and potentially, ports to new platforms)
will be provided in future 7.4.x releases.
New features will be added in the 7.5.x development series.

The details of each version are described below.

%%%%%%%%%%%%%%%%%%%%%%%%%%%%%%%%%%%%%%%%%%%%%%%%%%%%%%%%%%%%%%%%%%%%%%
\subsection*{\label{sec:New-7-4-0}Version 7.4.0}
%%%%%%%%%%%%%%%%%%%%%%%%%%%%%%%%%%%%%%%%%%%%%%%%%%%%%%%%%%%%%%%%%%%%%%

\noindent Release Notes:

\begin{itemize}

\item The default configuration file within the release now uses
  \MacroNI{ALLOW}/\MacroNI{DENY} in place of
  \MacroNI{HOSTALLOW}/\MacroNI{HOSTDENY} for security related settings.
  We recommend making this
  same change throughout all configuration files.  That way,
  a policy that depends on the default policy should continue to
  work as it did before.  The behavior of these configuration variables
  remains unchanged.  The \MacroNI{ALLOW}/\MacroNI{DENY} lists are
  added to the \MacroNI{HOSTALLOW}/\MacroNI{HOSTDENY} lists to form the
  authorization policy.  Both styles support the same syntax.  
  This change permits an anticipated
  phasing out of the \MacroNI{HOSTALLOW}/\MacroNI{HOSTDENY}  configuration
  variables, in order to simplify configuration.

\item As of Condor version 7.3.2, \Condor{q} \Opt{-xml} output no longer 
  begins with the non-XML consisting of two blank lines followed by a line
  of the following form:

\begin{verbatim}
-- Submitter: schedd-name : <IP> : hostname
\end{verbatim}

\item All \Prog{Stork} data placement is now supported by the Stork
project at the 
LSU Center for Computation and Technology
(\URL{http://www.cct.lsu.edu/www.cct.lsu.edu}).
Please see the home page of the Stork project at
\URL{http://www.cct.lsu.edu/~kosar/stork/index.php} for details and
software.

\end{itemize}


\noindent New Features:

\begin{itemize}

% commit af65de7ccc1a281c2b05b8f68ac70bcfa56b2dd1
\item \Condor{q} \Opt{-analyze} and \Opt{-better-analyze} now provide
  analysis for scheduler and local universe jobs.  Specifically, the
  \MacroNI{START\_SCHEDULER\_UNIVERSE} and
  \MacroNI{START\_LOCAL\_UNIVERSE} expressions are checked.

% #824
\item Added the ClassAd attributes
\Attr{TotalLocalRunningJobs}, \Attr{TotalLocalIdleJobs},
\Attr{TotalSchedulerRunningJobs}, and \Attr{TotalSchedulerIdleJobs}
to the published ClassAd for the \Condor{schedd}.  This means that
\Condor{q} \Opt{-analyze} can still give helpful information about
why local or scheduler universe jobs are idle when
the configuration variables \MacroNI{START\_LOCAL\_UNIVERSE} or
\MacroNI{START\_SCHEDULER\_UNIVERSE} refer to these attributes.
These attributes were already present internally within 
the \Condor{schedd} daemon, 
just not published.

% #688
\item The \Condor{vm-gahp} now supports KVM and links with libvirt, rather 
than calling virsh command-line tools.

% #760 #771 #769 #772 #773 #775
\item Greatly improved the \Condor{gridmanager}'s scalability when handling
many grid type gt2 grid universe jobs.  Improvements include more quickly
processing updated x509 certificates, not checking jobs for status updates if 
they have not been submitted to the remote site, and eliminating unnecessary 
updates to the \Condor{schedd}.

% commit 75f6b2fe920b88717712a0d41765d16665ad7fe6
\item Latency in the submission and cleaning up of Condor-C jobs
has been improved by changing the default value of
\Macro{C\_GAHP\_CONTACT\_SCHEDD\_DELAY} from 20 to 5.

% commit 8c2d88c695d6981be3bdab7e10c5d92e9f6bb55b
\item The \Expr{eval()} ClassAd function added in Condor version 7.3.2
is now also understood by the \Condor{job\_router} and
\Condor{q} using the \Opt{-better-analyze} option.

\item The submit command \SubmitCmd{run\_as\_owner} is now supported
for Unix platforms. Previously, it was only supported on Windows platforms.

% #795
\item When setting \MacroNI{AMAZON\_HTTP\_PROXY}, a username and password
can now be given as part of the proxy URL.
The value of \MacroNI{SOAP\_SSL\_CA\_DIR} is now consulted when authenticating
an https proxy for Amazon EC2, when \MacroNI{AMAZON\_HTTP\_PROXY} is defined.

% commit 0e8800c201f81eac54cba925b3d7f6d81a83aeca
\item The deprecated setting \Macro{TOOLS\_PROVIDE\_OLD\_MESSAGES} is no longer
recognized.

\end{itemize}

\noindent Configuration Variable Additions and Changes:

\begin{itemize}

% #768
\item The new configuration variable 
  \Macro{SCHEDD\_JOB\_QUEUE\_LOG\_FLUSH\_DELAY} sets an
  upper bound in seconds on how long it takes for changes to the job
  ClassAd to be visible to the Condor Job Router and to Quill.
  The default value is 5 seconds.
  Previously, there was no upper limit.  Typically, other activity in
  the job queue, such as jobs being submitted or completed would cause
  buffered data to be flushed to disk, such that the effective upper bound was
  a function of how busy the job queue was.

% commit 55525e0a338be8b2ba2d9173ce832e43d05413c3
\item The default configuration file now uses
  \MacroNI{ALLOW}/\MacroNI{DENY} in place of
  \MacroNI{HOSTALLOW}/\MacroNI{HOSTDENY}.  See the release notes above
  for more information.

% commit 7199e217f9228082a8465b85aaee18c2ebb19787
\item The default value for \Macro{MAX\_JOBS\_RUNNING} has changed.
  Previously, it was 200.  Now it is defined by an expression that depends 
  on the total amount of memory and the operating system.  The default
  expression requires 1MByte of RAM per running job, on the submit machine.
  In some environments and configurations, this is overly
  generous and can be cut by as much as 50\%.  Under Windows, the
  number of running jobs is still capped at 200.
  A 64-bit version of Windows  is recommended in order to raise the value
  above the default.
  Under Unix, the maximum default is now 10,000.  To scale higher, we
  recommend that the system ephemeral port range is extended
  such that there are at least 2.1 ports per running job.

% #767 commit 18296bfdfa92f16684a73d8d57a54d231b48dc33
\item The default value of \MacroNI{RESERVED\_SWAP} has changed to
  the value 0, which
  disables the \Condor{schedd} daemon's check for sufficient swap space
  before starting more jobs.  The new expression defined with 
  \MacroNI{MAX\_JOBS\_RUNNING} has a more appropriate memory check, so
  the configuration variable \MacroNI{RESERVED\_SWAP} will no longer
  be used in the near future.
  For cases where 
  \MacroNI{RESERVED\_SWAP} is not set to 0, the default value
  of \MacroNI{SHADOW\_SIZE\_ESTIMATE} has changed to 800 Kbytes.
  Previously, it was 200 if not set,
  but it was set to 1800 in the example configuration file.

% #767 commit c80e8a40e67ef4faa4e2b32b3671877eae1e1a19
\item The default values of \Macro{START\_LOCAL\_UNIVERSE} and
  \Macro{START\_SCHEDULER\_UNIVERSE} have changed.  Previously,
  these were set to \Expr{True}.  Now, they are set using an expression
  that allows
  up to 200 local universe and 200 scheduler universe jobs to run.

% #767 commit c4f4d911a808e1bdb18552e1cdeb0407b6344969
\item The default value of
  \Macro{GRIDMANAGER\_MAX\_SUBMITTED\_JOBS\_PER\_RESOURCE} has
  changed from 100 to 1000.

% #767 commit 9e6dfa463c71c28c8dc2c0c0c215b51d6762e811
% commit b4fd08ad1a8c69da24c371565796ef73522a61fc
\item The default \Macro{NEGOTIATOR\_INTERVAL} has changed from 300 to 60.

% #767 commit 8b91877ec8186810887402e1dd1df07b6341ade7
% Probably at least one other commit
\item The default value of \Macro{ENABLE\_GRID\_MONITOR} has been
  changed from \Expr{False} to \Expr{True}.  This variable
  was assigned to \Expr{True} in the example configuration file, so
  the change in default value now matches the value set in the example
  configuration.

% #631
\item The configuration variable \MacroNI{VM\_VERSION} has been removed,
as has the machine ClassAd attribute of the same name.
When the virtual machine version information is needed in the machine ClassAd,
the configuration variable \MacroNI{STARTD\_ATTRS} can be used to
add it.
 
% #861
\item The default configuration now uses
  \MacroNI{VM\_BRIDGE\_SCRIPT} and \MacroNI{VM\_SCRIPT} in place of
  \MacroNI{XEN\_BRIDGE\_SCRIPT} and \MacroNI{XEN\_SCRIPT} due to KVM additions. 
  Submit description file commands have also been added, and they include:
  \SubmitCmd{kvm\_disk}, \SubmitCmd{kvm\_transfer\_files},
   and \SubmitCmd{kvm\_cd\_rom\_device}.

% #872
\item The configuration variables \MacroNI{XEN\_DEFAULT\_KERNEL}
  and \MacroNI{XEN\_DEFAULT\_INITRD} have been removed.
  Corresponding to this, the submit description file command
  \Expr{xen\_kernel = any} is no longer valid.

% #775, cf02764d9d0fdd2b36ef3629f862f856ee41a717, and more
\item Optimizations in core Condor systems should provide minor speed 
improvements.

% 823
\item Updated the maximum log size to the maximum operating system's file size.  
\end{itemize}

\noindent Bugs Fixed:

\begin{itemize}

% #706
\item Fixed a race condition bug in the \Condor{startd} which allowed
it to send Unix signals, intended for \Condor{starter} processes, as
root to non-Condor related processes.

% 735
\item A Windows platform bug has been fixed.
The bug caused a 20-second interval in which
the \Condor{shadow}, \Condor{startd}, and \Condor{starter} daemons
appeared as deadlocked. 
The bug was visible if a job ClassAd update from the \Condor{starter} caused
the job's periodic hold or remove policy to become \Expr{True}.

%gittrac #622
\item Fixed a bug that could cause \Condor{dagman} to generate an
illegal rescue DAG, if it read events incorrectly in recovery mode.
\Condor{dagman} now checks for events that violate DAG semantics
when reading events in recovery mode, and it exits without creating a
rescue DAG if it reads such an event.

% gittrac #744
\item Fixed a bug that could cause \Condor{dagman} to abort if it saw
the combination of a terminated event and an aborted event on a node with
retries.

% commit 5039a08cf00b0d0fafcc3fd8241518d1854ec3a1
\item Changed some logged warnings in \Condor{dagman} to not be
printed at the default verbosity setting.

% gittrac #825
\item The version compatibility checking between a \File{.condor.sub}
file and the \Condor{dagman} binary which is done at DAG startup
is now much more permissive.
Currently, \File{.condor.sub} files with
Condor versions of 7.1.2 and later accepted by \Condor{dagman}.

% gittrac #851
\item Fixed a bug introduced with the new \Condor{dagman} lazy log file
evaluation code in Condor version 7.3.2.
The bug sometimes caused failure when running rescue DAGs.

\item Fixed a bug originating in Condor version 7.1.4.
When a user submitted a job
with an executable that did not have execute permission enabled,
Condor was running as root, and file transfer was specified in the job,
Condor failed to automatically turn on execute permission after
transferring the file.

% commit 3bb847691bfda4f26d2f570bed1a412fb3afb439
\item Fixed a bug that appeared in Condor version 7.3.2.
The configuration variable
\MacroNI{COUNT\_HYPERTHREAD\_CPUS} was ignored and was effectively
treated as \Expr{False} in all cases.

% #761
\item Fixed a bug in which the Condor Job Router was not able
to see matchmaking diagnostic attributes such as \Attr{LastRejMatchTime}.
Therefore, when evaluating policy
expressions that referred to these attributes, they were effectively
treated as though \Expr{Undefined}.
Quill was also not able to see these attributes.

% #822
\item Fixed a bug introduced in Condor version 7.3.2 that could cause the
\Condor{gridmanager} to crash repeatedly on startup,
if the job queue
contained grid type gt2 jobs that had been previously submitted.

% #724, #774, #786
\item Fixed two bugs introduced in Condor version 7.3.2,
and related to VOMS. 
The first bug
prevented jobs with X.509 proxies from being submitted on platforms
on which Condor does not support VOMS.
The second bug prevented submission
of jobs with VOMS proxies, if the authenticity of the VOMS extensions
could not be verified.
At the same time, improved memory usage when VOMS extensions are not used.

\item Fixed a bad default in the \Attr{batch\_gahp.config} file,
that prevented
Condor from observing job state changes for grid universe jobs
with grid type pbs and lsf.

% #748
\item Fixed a bug that caused Condor-C jobs to fail if
the submit description file command \SubmitCmd{transfer\_executable}
was set to \Expr{False}.

% #784
\item Fixed a bug that caused Condor-C jobs to fail if the executable
or one of the \File{stdin}, \File{stdout}, or \File{stderr} file names
contained a comma.

% #460
\item File transfer for grid type gt4 jobs requires an empty directory
within \File{/tmp}, which the \Condor{gridmanager} creates. 
If this directory is deleted, the \Condor{gridmanager} will now recreate it.

%gittrac #790
\item Fixed a bug that could cause the user job log to become
  corrupted on Windows platforms.  This bug would manifest itself only if the
  same log file was specified with different paths.  For example, the
  following submit file could have triggered this bug:
\begin{verbatim}
...
initialdir = /data/job1
log = ../JobLog
queue

initialdir = /data/job2
log = ../JobLog
queue
\end{verbatim}


% commit a26fcd9fe54cd3920fe777d5d8e0b2ffefbc3b1f
\item Fixed a memory leak introduced into Condor version 7.3.2.
The leak was in the \Condor{collector} daemon.

% #211 commit d6c0144739000523e94205a192be3cf9afe9ca9f
\item Fixed a problem introduced in Condor version 7.1.4 
that resulted in the failure of forcing
the execute bit to be set on the transferred job executable.

% commit 1663b7e183e6bf1df8152af98d9387412c2ae146
\item Fixed a bug introduced in Condor version 7.3.2
that resulted in the \Condor{negotiator} daemon
refusing to run, if the configuration variable \MacroNI{GROUP\_QUOTA}
for any group was set to 0.

% gittrac #731
\item Fixed a bug that caused the \Code{ctime} in the event log header
  to always be zero.

% #862 commit 9a432e2f3497e5dce120db5c733e79212257f6a5
\item Fixed the output of \Condor{status} when \Opt{-java} or \Opt{-vm} is
used.

\item Fixed a problem in the \Condor{schedd} daemon introduced in
  7.3.2.  For \Condor{schedd} daemons with lots of jobs having periodic release
  expressions, this bug could result in the \Condor{schedd} taking a long
  time while evaluating periodic expressions, causing it to become
  unresponsive to queries and other tasks.
  With a job queue of 30,000 jobs,
  a period of unresponsiveness of an hour was observed,
  whereas the evaluation of periodic expressions in this same environment
  normally takes less than 5 seconds.

\item A number of potential bugs and memory leaks were identified and 
fixed throughout Condor.  The Condor Team is not aware of anyone having 
encountered these bugs.

% #692 commit 8bc6bb4e06f11b2fdca28214d98c68c34c0ab9a4
\item The \Condor{starter} cleans up working directories in more
situations.  Previously during some error conditions, the working
directory within \MacroUNI{EXECUTE} might be left behind.

% #692 commit 8bc6bb4e06f11b2fdca28214d98c68c34c0ab9a4
\item If the user log cannot be accessed when a local universe
job starts, the job would fail and immediately be retried.  Now
the job is placed on hold.

% 826 
\item Fixed a bug in the \Condor{startd} in which vacating jobs would not 
respect the value of \MacroNI{job\_lease\_duration}.

% 802
\item Updated the detection of \Attr{HasVM} within the \Condor{startd}
 to publish an update to the \Condor{collector} when the configuration variable
\MacroNI{VM\_RECHECK\_INTERVAL} is specified.

% commit 68f06088fa36eb0eb332a4f72a5c48ccd48b1d5a
\item Fixed a bug where the \Condor{gridmanager} could, in rare cases, waste a
small amount of memory and processor time checking for proxy files no longer
being used by any active jobs.

% commit bc66aa432e1f4e69d88a5b769204a4fce0648bfc
\item The setting \Macro{CREAM\_GAHP} was added to the default configuration 
file with a value of \$(SBIN)/cream\_gahp. Existing installations desiring to 
submit jobs to CREAM should add this setting.

\end{itemize}

\noindent Known Bugs:

\begin{itemize}

\item \Macro{USE\_VISIBLE\_DESKTOP}, when set to to \Expr{TRUE} will corrupt 
the visible desktop.  This bug is present back through 7.2.4.
(\MacroNI{USE\_VISIBLE\_DESKTOP} didn't work at all in earlier 7.2 releases.)
This bug will be fixed in 7.4.1.

\end{itemize}

\noindent Additions and Changes to the Manual:

\begin{itemize}

\item None.

\end{itemize}

