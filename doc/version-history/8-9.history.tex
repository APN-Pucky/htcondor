%%%      PLEASE RUN A SPELL CHECKER BEFORE COMMITTING YOUR CHANGES!
%%%      PLEASE RUN A SPELL CHECKER BEFORE COMMITTING YOUR CHANGES!
%%%      PLEASE RUN A SPELL CHECKER BEFORE COMMITTING YOUR CHANGES!
%%%      PLEASE RUN A SPELL CHECKER BEFORE COMMITTING YOUR CHANGES!
%%%      PLEASE RUN A SPELL CHECKER BEFORE COMMITTING YOUR CHANGES!

%%%%%%%%%%%%%%%%%%%%%%%%%%%%%%%%%%%%%%%%%%%%%%%%%%%%%%%%%%%%%%%%%%%%%%
\section{Development Release Series 8.9}\label{sec:History-8-9}
%%%%%%%%%%%%%%%%%%%%%%%%%%%%%%%%%%%%%%%%%%%%%%%%%%%%%%%%%%%%%%%%%%%%%%

This is the development release series of HTCondor.
The details of each version are described below.

%%%%%%%%%%%%%%%%%%%%%%%%%%%%%%%%%%%%%%%%%%%%%%%%%%%%%%%%%%%%%%%%%%%%%%
\subsection*{\label{sec:New-8-9-1}Version 8.9.1}
%%%%%%%%%%%%%%%%%%%%%%%%%%%%%%%%%%%%%%%%%%%%%%%%%%%%%%%%%%%%%%%%%%%%%%

\noindent Release Notes:

\begin{itemize}

\item HTCondor version 8.9.1 not yet released.
%\item HTCondor version 8.9.1 released on Month Date, 2019.

\end{itemize}


\noindent New Features:

\begin{itemize}

\item Added a new submit file option, docker\_network\_type = host
which causes a docker universe job to use the host's network,
instead of the default NATed interface.
\Ticket{6906}

\item Added a new config knob, \Macro{DOCKER\_EXTRA\_ARGUMENTS}
to allow admins to add arbritrary docker command line options to
the docker create command.
\Ticket{6900}

\item If a job RequestGpus and is a Docker Universe job , HTCondor
automatically mounts the nvidia gpu devices.
\Ticket{6910}

\item If a job RequestGpus, and Singularity is enabled, HTCondor
automatically passes the --nv flag to Singularity to tell it to mount
the nvidia gpus.
\Ticket{6898}

\item When jobs are run without file transfer on, usually because there is a 
shared filesytem, HTCondor used to unconditionally set the jobs argv[0]
to the string condor\_exe.exe.  This breaks jobs that look at their own
argv[0], in ways that are very hard to debug.  In this release of HTCondor, 
we no longer do this.
\Ticket{6943}

\item The performance of HTCondor's File Transfer mechanism has improved when
	sending multiple files, especially in wide-area network settings.
\Ticket{6884}

\item We've added six new events to the job event log, recording details
about file transfer.  For both file transfer -in (before/to the job) and
-out (after/from the job), we log if the transfer was queued, when it started,
and when it finished.  If the event was queued, the start event will note
for how long; the first transfer event written will additionally include
the starter's address, which has not otherwise been printed.

We've also added several transfer-related attributes to the job ad.  For jobs
which do file transfer, we now set
\Attr{JobCurrentFinishTransferOutputDate}, to complement
\Attr{JobCurrentStartTransferOutputDate}, as well as the corresponding
attributes for input transfer:
\Attr{JobCurrentStartTransferInputDate} and
\Attr{JobCurrentFinishTransferInputDate}.  The new attributes are added at
the same time as \Attr{JobCurrentStartTransferOutputDate}, that is, at
job termination.  This set of attributes use the older and more deceptive
definitions of file transfer timing.  To obtain the times recorded by the
new events, instead reference \Attr{TransferInQueued},
\Attr{TransferInStarted}, \Attr{TransferInFinished},
\Attr{TransferOutQueued}, \Attr{TransferOutStarted}, and
\Attr{TransferOutFinished}.  HTCondor sets these attributes (roughly) at
the time they occur.

\Ticket{6854}

\item Implemented a new version of the curl\_plugin with multi-file support,
allowing it to transfer many files in a single invocation of the plugin.
\Ticket{6499} \Ticket{6859}

\item Added support for passing HTTPS authentication credentials to  
file transfer plugins, using specially custozimed protocols.
\Ticket{6858}

\item Added support for output file remaps for URLs. This allows users to 
specify a URL where they want individual output files to go, and once a job 
is complete, we automatically uploads the files there. We are preserving the 
older implementation (OutputDestination), which puts all output files in the 
same place, for backwareds compatibility.
\Ticket{6876}

\end{itemize}

\noindent Bugs Fixed:

\begin{itemize}

\item Avoid killing jobs using between 90\% and 99\% of memory limit. 
\Ticket{6925}

\item Improved how \AdStr{Chirp} handles a network disconneciton between
the \Condor{starter} and \Condor{shadow}.
\AdStr{Chirp} commands now return a special error and no longer cause the
\Condor{starter} to exit (killing the job).
\Ticket{6873}

\end{itemize}

%%%%%%%%%%%%%%%%%%%%%%%%%%%%%%%%%%%%%%%%%%%%%%%%%%%%%%%%%%%%%%%%%%%%%%
\subsection*{\label{sec:New-8-9-0}Version 8.9.0}
%%%%%%%%%%%%%%%%%%%%%%%%%%%%%%%%%%%%%%%%%%%%%%%%%%%%%%%%%%%%%%%%%%%%%%

\noindent Release Notes:

\begin{itemize}

\item HTCondor version 8.9.0 released on February 28, 2019.

\end{itemize}

\noindent Known Issues:

This release may require configuration changes to work as before.
During this release series, we are making changes to make it easier to deploy
secure pools. This release contains two security related configuration changes.

\begin{itemize}

\item Absent any configuration, the default behavior is to deny authorization
to all users.

\item In the configuration files, if \MacroNI{ALLOW\_DAEMON} or
\MacroNI{DENY\_DAEMON} are omitted, \MacroNI{ALLOW\_WRITE} or
\MacroNI{DENY\_WRITE} are no longer used in their place.

On most pools, the easiest way to get the previous behavior is to add the
following to your configuration:

\begin{verbatim}
ALLOW_READ = *
ALLOW_DAEMON = $(ALLOW_WRITE)
\end{verbatim}

The main configuration file (\File{/etc/condor/condor\_config}) already
implements the above change by calling \Expr{use SECURITY : HOST\_BASED}.

With the addition of the automatic security session for a family of HTCondor
daemons and the existing match password authentication between the execute
and submit daemons, most hosts in a pool may not require changes to the
configuration files.
On the central manager, you do need to ensure \Expr{DAEMON} level access
for your submit nodes.
Also, CCB requires \Expr{DAEMON} level access.

\end{itemize}

\noindent New Features:

\begin{itemize}

\item Changed the default security behavior to deny authorization by default.
Also, neither \MacroNI{ALLOW\_DAEMON} nor \MacroNI{DENY\_DAEMON} fall back
to using the corresponding \MacroNI{ALLOW\_WRITE} or \MacroNI{DENY\_WRITE}
when reading configuration files.
\Ticket{6824}

\item A family of HTCondor daemons can now share a security session that
allows them to trust each other without doing a security negotiation
when a network connection is made amongst them.
This ``family'' security session can be disabled by setting the new
configuration parameter \MacroNI{SEC\_USE\_FAMILY\_SESSION} to \Expr{False}.
\Ticket{6788}

\item Scheduler Universe jobs now start in order of priority, instead of random
order. This is most typically used for DAGMan. When running \Condor{submit\_dag}
against a .dag file, you can use the -priority <N> flag to set the priority
for the overall \Condor{dagman} job. When the \Condor{schedd} is starting new
Scheduler Universe jobs, the highest priority queued job will start first. If
all queued Scheduler Universe jobs have equal priority, they get started in
order of submission.
\Ticket{6703}

\item  Normally, HTCondor requires the user to specify their credentials
when using EC2 (via the grid universe or via \Condor{annex}).  This allows
users to use different accounts from the same machine.  However,
if a user started an EC2 instance with the privileges necessary to start
other instances, and ran HTCondor in that instance, HTCondor was unable to
use that instance's privileges; the user still had to specify their
credentials.  Instead, the user may now specify \Expr{FROM INSTANCE} instead
of the name of a credential file to indicate that HTCondor should use the
instance's credentials.

By default, any user with access to a privileged EC2 instance has access to
that instance's privileges.  If you would like to make use of this feature,
please read \ref{sec:Instance Roles} before adding privileges (an instance
role) to an instance which allows access by other users, specifically
including the submitting of jobs to or running jobs on that instance.
\Ticket{6789}

\item The \Condor{now} tool now supports vacating more than one job; the
additional jobs' resources will be coalesced into a single slot, on which
the now-job will be run.
\Ticket{6694}

\item In the Python bindings, the \Expr{JobEventLog} class now has a
\Expr{close} method.  It is also now its own iterable context manager
(implements \Expr{\_\_enter\_\_} and \Expr{\_\_exit\_\_}).
The \Expr{JobEvent} class now implements \Expr{\_\_str\_\_} and \Expr{\_\_repr\_\_}.
\Ticket{6814}

\item the \Condor{hdfs} daemon which allowed the hdfs daemons to run under
the \Condor{master} has been removed from the contributed source.
\Ticket{6809}

\end{itemize}

\noindent Bugs Fixed:

\begin{itemize}

\item Fixed potential authentication failures between the \Condor{schedd}
and \Condor{startd} when multiple \Condor{startd}s are using the same
shared port server.
\Ticket{5604}

\end{itemize}

