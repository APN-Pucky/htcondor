%%%      PLEASE RUN A SPELL CHECKER BEFORE COMMITTING YOUR CHANGES!
%%%      PLEASE RUN A SPELL CHECKER BEFORE COMMITTING YOUR CHANGES!
%%%      PLEASE RUN A SPELL CHECKER BEFORE COMMITTING YOUR CHANGES!
%%%      PLEASE RUN A SPELL CHECKER BEFORE COMMITTING YOUR CHANGES!
%%%      PLEASE RUN A SPELL CHECKER BEFORE COMMITTING YOUR CHANGES!

%%%%%%%%%%%%%%%%%%%%%%%%%%%%%%%%%%%%%%%%%%%%%%%%%%%%%%%%%%%%%%%%%%%%%%
\section{\label{sec:History-7-7}Development Release Series 7.7}
%%%%%%%%%%%%%%%%%%%%%%%%%%%%%%%%%%%%%%%%%%%%%%%%%%%%%%%%%%%%%%%%%%%%%%

This is the development release series of Condor.
The details of each version are described below.

%%%%%%%%%%%%%%%%%%%%%%%%%%%%%%%%%%%%%%%%%%%%%%%%%%%%%%%%%%%%%%%%%%%%%%
\subsection*{\label{sec:New-7-7-0}Version 7.7.0}
%%%%%%%%%%%%%%%%%%%%%%%%%%%%%%%%%%%%%%%%%%%%%%%%%%%%%%%%%%%%%%%%%%%%%%

\noindent Release Notes:

\begin{itemize}

\item Condor version 7.7.0 not yet released.
%\item Condor version 7.7.0 released on Month Date, 2011.

\end{itemize}


\noindent New Features:

\begin{itemize}

\item \Condor{q} now supports the new option \Opt{-stream-results}.
  When this option is specified, \Condor{q} displays results as they
  are fetched from the job queue, rather than buffering up the query
  results before displaying anything.

% #1871 commit e4cce6996764a5eaabb28106c44f901c23dc4bae
\item The new submit description file command \SubmitCmd{stack\_size} 
  applies to Linux jobs that are not running in the standard universe. 
  It sets the allocation of stack space to be other than the default
  value of 512 Mbytes.
  It also advertises the job ClassAd attribute \AdAttr{StackSize}.

% gittrac #1550
\item The new ClassAd function \Code{stringListsIntersect} evaluates to 
  \Expr{True} if two strings of delimited elements have any matching elements,
  and it evaluates to \Expr{False} otherwise.

% gittrac #1821
\item The grid universe now supports the \SubmitCmd{ec2} resource type,
  which uses the EC2 Query (REST) API to start virtual machines on cloud
  resources.

% gittrac #2090 
\item The behavior of DAGMan has changed, 
such that if multiple definitions of a VARS macroname 
for a specific node within a DAG input exist,
a warning is written to the log, of the format
\begin{verbatim}
Warning: VAR <macroname> is already defined in job <JobName>
Discovered at file "<DAG input file name>", line <line number>
\end{verbatim}
See section ~\ref{dagman:VARS} for details.

% gittrac #659 commit 551785beaa1c0d8689d44368030b9f561b5ddfee
\item Filip Krikava supplied a patch that limits the number of file descriptors
that DAGman has open at a time. The reason for creating this capability is that
DAGman tends to fail on ``wide'' dags, where many jobs are independent, rather
than being ``linear'', where jobs are highly interdependent.
\end{itemize}

\noindent Configuration Variable and ClassAd Attribute Additions and Changes:

\begin{itemize}

\item For a job with an X.509 proxy credential, the new job ClassAd
attribute \AdAttr{X509UserProxyEmail} is the email address extracted
from the proxy.

% gittrac 2067
\item On Linux execute machines with kernel version more recent than 2.6.27,
the proportional set size (PSS) in kilobytes summed across all
processes in the job is now reported in the attribute
\AdAttr{ProportionalSetSizeKb}.  If the execute machine does not
support monitoring of PSS or PSS has not yet been measured, this
attribute will be undefined.  PSS differs from \AdAttr{ImageSize} in
how memory shared between processes is accounted.  The PSS for one
process is the sum of that processes memory pages divided by the
number of processes sharing each of the pages.  \AdAttr{ImageSize} is
the same except there is no division by the number of processes
sharing the pages.

% gittrac #1755
\item The new configuration variable \Macro{DAGMAN\_USE\_STRICT} 
turns warnings into errors, as defined in section~\ref{param:DAGManUseStrict}.

% gittrac #2006
\item The \Condor{schedd} now publishes performance-related statistics.
  Page~\pageref{sec:Scheduler-ClassAd-Attributes} in Appendix A contains
  definitions for these new attributes:
  \begin{itemize}
    \item \Attr{DetectedMemory}
    \item \Attr{DetectedCpus}
    \item \Attr{UpdateInterval}
    \item \Attr{WindowedStatWidth}
    \item \Attr{ExitCode<N>}
    \item \Attr{ExitCodeCumulative<N>}
    \item \Attr{JobsSubmitted}
    \item \Attr{JobsSubmittedCumulative}
    \item \Attr{JobsStarted}
    \item \Attr{JobsStartedCumulative}
    \item \Attr{JobsCompleted}
    \item \Attr{JobsCompletedCumulative}
    \item \Attr{JobsExited}
    \item \Attr{JobsExitedCumulative}
    \item \Attr{ShadowExceptions}
    \item \Attr{ShadowExceptionsCumulative}
    \item \Attr{JobSubmissionRate}
    \item \Attr{JobStartRate}
    \item \Attr{JobCompletionRate}
    \item \Attr{MeanTimeToStart}
    \item \Attr{MeanTimeToStartCumulative}
    \item \Attr{MeanRunningTime}
    \item \Attr{MeanRunningTimeCumulative}
    \item \Attr{SumTimeToStartCumulative}
    \item \Attr{SumRunningTimeCumulative}
  \end{itemize}

\end{itemize}

\noindent Bugs Fixed:

\begin{itemize}

% gittrack 1928
\item \AdAttr{JobUniverse} was effectively a required attribute for
  jobs created via the Fetch Work hook,
  due to the need to set the \MacroNI{IS\_VALID\_CHECKPOINT\_PLATFORM}
  expression, such that it would not evaluate to \Expr{Undefined}.
  Now the default \MacroNI{IS\_VALID\_CHECKPOINT\_PLATFORM} expression
  evaluates to \Expr{True} when \AdAttr{JobUniverse} is not defined.

% gittrack 1943
\item When there are multiple cpus but only one slot, the slot name no
longer begins with \Expr{slot1@}.

% gittrac 1805 commit 931ca85d6bc483dc6af82d7000a3f5c9ee64dc07
% gittrac 1805 commit e45ec5ff35e77533bfba3239dc3b71315d9d5afd
% gittrac 1805 commit 7fe5d0c64562f59a7efafe77227995e4ada24e20
\item The tool \Condor{advertise} seemed to be trying too hard to resolve
host names. This was fixed to only do the minimally necessary 
number of look ups.

\end{itemize}

\noindent Known Bugs:

\begin{itemize}

\item None.

\end{itemize}

\noindent Additions and Changes to the Manual:

\begin{itemize}

\item None.

\end{itemize}

