%%%%%%%%%%%%%%%%%%%%%%%%%%%%%%%%%%%%%%%%%%%%%%%%%%%%%%%%%%%%%%%%%%%%%%
\section{Upgrading from the 8.0 series to the 8.2 series of HTCondor}\label{sec:to-8.2}
%%%%%%%%%%%%%%%%%%%%%%%%%%%%%%%%%%%%%%%%%%%%%%%%%%%%%%%%%%%%%%%%%%%%%%

\index{upgrading!items to be aware of}
Upgrading from the 8.0 series of HTCondor to the 8.2 series 
will bring new features introduced in the 8.1 series of HTCondor.
These new features include:
configuration is more powerful with new syntax and features, and
the default configuration policy does not preempt jobs,
monitoring is enhanced and now integrates with Ganglia,
automated detection and management of GPUs,
numerous scalability enhancements improve performance,
an improved Python API including support for Python 3,
new native packaging and ports are available for the latest Linux 
distributions including Red Hat 7 Beta and Debian 7,
cloud computing improvements including support for
EC2 spot instances, OpenStack, and \Condor{ssh\_to\_job} 
directly into EC2 jobs,
grid universe jobs can now target Google Compute Engine and BOINC servers,
partitionable slots now are compatible with \Condor{startd} \MacroNI{RANK}
expressions, and consumption policies permit partitionable slots 
to be split into dynamic slots at negotiation time,
improved data management including dynamic adjustment of the level of 
file transfer concurrency based on disk load 
(see section~\ref{param:FileTransferDiskLoadThrottle}), 
and experimental support to allow the execution of a job to be overlaid 
with the transfer of output files from the previous different job,
and
the new \Condor{sos} tool helps administrators manage overloaded daemons. 

Upgrading from the 8.0 series of HTCondor to the 8.2 series will
also introduce changes that administrators of sites running from an older
HTCondor version should be aware of when planning an upgrade.  
Here is a list of items that administrators should be aware of.

\begin{itemize}

\item New configuration syntax and features change:
  \begin{itemize}
  \item The interaction of comments and the line continuation character
   has changed.  See  section~\ref{sec:Other-Syntax} for the current
   interaction. 
  \item The use of a colon character (\verb@:@) instead of the
   equals sign (\verb@=@) in assigning a value to a configuration variable
   causes tools that parse configuration to output a warning.
   Therefore, any custom parsing of tool output may need to be updated to
   handle this warning.
   Previous versions of the default configuration set variable
   \MacroNI{RUNBENCHMARKS} using a colon character;
   HTCondor code explicitly suppresses the warning in this case.
  \end{itemize}

\item The default user priority factor for new users has changed 
from 1 to 1000.
Therefore, unless the accountant log is discarded,
existing users will still have a priority factor of 1,
while new users will have a priority factor of 1000.
Use \Condor{userprio} to change the priority factor of existing users
if the accountant log is maintained across the upgrade. 
\Ticket{4282}

\item For Windows platforms,
HTCondor has switched to use the newer 2012 Microsoft compiler,
which uses the Visual C++ 2012 Runtime components.
Therefore, the HTCondor MSI installer will acquire this Runtime,
if it is not already installed.

\item The meaning of \Expr{cpus=auto} when there are more 
slots than CPUs has changed within the configuration. 
In the \Expr{SLOT\_TYPE\_<N>} configuration variable,
\Expr{cpus=auto} previously resulted in 1 CPU per slot. 
Now, all slots with \Expr{cpu=auto} get an equal share of the CPUs, 
rounded down.
\Ticket{3249}

\item The DAGMan node status file formatting has changed.
The format of the DAG node status file is now New ClassAds,
and the amount of information in the file has increased.

\item Setting configuration variable
\Macro{DAGMAN\_ALWAYS\_USE\_NODE\_LOG} to \Expr{False}
or using the corresponding \Opt{-dont\_use\_default\_node\_log} option
to \Condor{submit\_dag} is no longer recommended.
Note that at strictness setting 1 (the default), setting
\MacroNI{DAGMAN\_ALWAYS\_USE\_NODE\_LOG} to \Expr{False}
will cause a fatal error. 
If the DAG must be run with \MacroNI{DAGMAN\_ALWAYS\_USE\_NODE\_LOG} 
set to \Expr{False},
a good way to deal with upgrading is to use DAGMan Halt files 
to cause all of the running DAGs to drain from the queue, 
and then do the upgrade after the DAGs have stopped.  
After the upgrade is done, 
edit the per-DAG configuration files to have 
\MacroNI{DAGMAN\_ALWAYS\_USE\_NODE\_LOG} set to \Expr{True},
or set \MacroNI{DAGMAN\_USE\_STRICT} to 0 and 
re-submit the DAGs, which will then run the Rescue DAGs.

\item If using \Expr{ENABLE\_IPV6 = True}, the configuration must
also set \Expr{ENABLE\_IPV4 = False}.
If both are enabled simultaneously,
daemons will listen on both IPv4 and IPv6, 
but will only advertise one of the two addresses.

\item Globus 5.2.2 or a more recent version is now required 
for grid universe jobs of grid-type nordugrid and cream.
Globus version 5.2.5 is included in this 8.2.0 release of HTCondor.
HTCondor will prefer to use libraries already installed in \File{/usr/lib[64]},
when present.
\Ticket{4243}

\item If referencing attribute \AdAttr{SubmittorUserPrio} in
configuration, such as in the \MacroNI{PREEMPTION\_REQUIREMENTS} expression,
you will need to change it to \AdAttr{SubmitterUserPrio} 
Note the spelling difference in the ClassAd attribute name.
\Ticket{4369}

\item HTCondor can not distinguish normal from abnormal job exit
for Nordugrid ARC grids.
Therefore, all grid-type nordugrid jobs will be recorded as 
terminating normally, with an exit code from 0 to 255.
\Ticket{4342}

\item For configuration, parameter substitution now honors per-daemon 
overrides.  This long standing bug's fix may result in subtle changes
to the way that your configuration files are processed.

\end{itemize}

