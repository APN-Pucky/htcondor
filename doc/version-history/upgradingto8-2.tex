%%%%%%%%%%%%%%%%%%%%%%%%%%%%%%%%%%%%%%%%%%%%%%%%%%%%%%%%%%%%%%%%%%%%%%
\section{\label{sec:to-8.2}Upgrading from the 8.0 series to the 8.2 series of HTCondor}
%%%%%%%%%%%%%%%%%%%%%%%%%%%%%%%%%%%%%%%%%%%%%%%%%%%%%%%%%%%%%%%%%%%%%%

\index{upgrading!items to be aware of}
While upgrading from the 8.0 series of HTCondor to the 8.2 series 
will bring 
new features and improvements introduced in the 8.1 series of HTCondor,
it will
also introduce changes that administrators of sites running from an older
HTCondor version should be aware of when planning an upgrade.  
Here is a list of items that administrators should be aware of.

\begin{itemize}

\item New syntax and features with respect to configuration
make configuration more powerful.
  \begin{itemize}
  \item The default configuration file is much smaller in size.
   Defaults for most configuration variables are set at compile time.
   Directions for converting to the new, small size file are in
   section~\ref{sec:Pool-Upgrade}.
  \item A feature to include configuration from elsewhere
   is described in section~\ref{sec:Config-Include}.
  \item Metaknobs permit the incorporation of predefined configuration.
   Metaknobs are available for setting a machine's role within
   a pool, enabling a feature belonging to a machine, the policy under which
   machines start jobs, or the security configuration.
   Section~\ref{sec:Config-Metaknobs} lists the complete set of available
   metaknobs.
  \item Limited conditional configuration is described in 
   section~\ref{sec:Config-IfElse}.
  \end{itemize}
This introduction of new features causes a few items that pool
administrators must be aware of and work with:
  \begin{itemize}
  \item The interaction of comments and the line continuation character
   has changed.  See  section~\ref{sec:Other-Syntax} for the current
   interaction. 
  \item The use of a colon character (\verb@:@) instead of the
   equals sign (\verb@=@) in assigning a value to a configuration variable
   causes tools that parse configuration to output a warning.
   Therefore, any custom parsing of tool output may need to be updated to
   handle this warning.
   Previous versions of the default configuration set variable
   \MacroNI{RUNBENCHMARKS} using a colon character;
   HTCondor code explicitly suppresses the warning in this case.
  \end{itemize}

\item Support for ganglia monitoring.

\item Support for GPUs.

\item The default user priority factor for new users has changed 
from 1 to 1000.
Therefore, unless the accountant log is discarded,
existing users will still have a priority factor of 1,
while new users will have a priority factor of 1000.
Use \Condor{userprio} to change the priority factor of existing users
if the accountant log is maintained across the upgrade. 
\Ticket{4282}

\item The meaning of \Expr{cpus=auto} when there are more 
slots than CPUs has changed within the configuration. 
In the \Expr{SLOT\_TYPE\_<N>} configuration variable,
\Expr{cpus=auto} previously resulted in 1 CPU per slot. 
Now, all slots with \Expr{cpu=auto} get an equal share of the CPUs, 
rounded down.
\Ticket{3249}

\item The DAGMan node status file formatting has changed.
The format of the DAG node status file is now New ClassAds,
and the amount of information in the file has increased.

\item Setting configuration variable
\Macro{DAGMAN\_ALWAYS\_USE\_NODE\_LOG} to \Expr{False}
or using the corresponding \Opt{-dont\_use\_default\_node\_log} option
to \Condor{submit\_dag} is no longer recommended.
Note that at strictness setting 1 (the default), setting
\MacroNI{DAGMAN\_ALWAYS\_USE\_NODE\_LOG} to \Expr{False}
will cause a fatal error. 
If the DAG must be run with \MacroNI{DAGMAN\_ALWAYS\_USE\_NODE\_LOG} 
set to \Expr{False},
a good way to deal with upgrading is to use DAGMan Halt files 
to cause all of the running DAGs to drain from the queue, 
and then do the upgrade after the DAGs have stopped.  
After the upgrade is done, 
edit the per-DAG configuration files to have 
\MacroNI{DAGMAN\_ALWAYS\_USE\_NODE\_LOG} set to \Expr{True},
or set \MacroNI{DAGMAN\_USE\_STRICT} to 0 and 
re-submit the DAGs, which will then run the Rescue DAGs.

\item If using \Expr{ENABLE\_IPV6 = True}, the configuration must
also set \Expr{ENABLE\_IPV4 = False}.
If both are enabled simultaneously,
daemons will listen on both IPv4 and IPv6, 
but will only advertise one of the two addresses.

\item Globus 5.2.2 or a more recent version is now required 
for grid universe jobs of grid-type nordugrid and cream.
Globus version 5.2.5 is included in this 8.2.0 release of HTCondor.
HTCondor will prefer to use libraries already installed in \File{/usr/lib[64]},
when present.
\Ticket{4243}

\item ClassAd integer and real values are 64 bits.
\Ticket{3928}

\item HTCondor can not distinguish normal from abnormal job exit
for Nordugrid ARC grids.
Therefore, all grid-type nordugrid jobs will be recorded as 
terminating normally, with an exit code from 0 to 255.
\Ticket{4342}

\end{itemize}

