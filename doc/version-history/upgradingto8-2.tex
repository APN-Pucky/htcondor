%%%%%%%%%%%%%%%%%%%%%%%%%%%%%%%%%%%%%%%%%%%%%%%%%%%%%%%%%%%%%%%%%%%%%%
\section{\label{sec:to-8.2}Upgrading from the 8.0 series to the 8.2 series of HTCondor}
%%%%%%%%%%%%%%%%%%%%%%%%%%%%%%%%%%%%%%%%%%%%%%%%%%%%%%%%%%%%%%%%%%%%%%

\index{upgrading!items to be aware of}
While upgrading from the 8.0 series of HTCondor to the 8.2 series 
will bring 
new features and improvements introduced in the 8.1 series of HTCondor,
it will
also introduce changes that administrators of sites running from an older
HTCondor version should be aware of when planning an upgrade.  
Here is a list of items that administrators should be aware of.

\begin{itemize}

\item New syntax and features with respect to configuration.

\item Support for ganglia monitoring.

\item The DAGMan node status file formatting has changed.
The format of the DAG node status file is now New ClassAds,
and the amount of information in the file has increased.

\item Setting configuration variable
\Macro{DAGMAN\_ALWAYS\_USE\_NODE\_LOG} to \Expr{False}
or using the corresponding \Opt{-dont\_use\_default\_node\_log} option
to \Condor{submit\_dag} is no longer recommended.
Note that at strictness setting 1 (the default), setting
\MacroNI{DAGMAN\_ALWAYS\_USE\_NODE\_LOG} to \Expr{False}
will cause a fatal error. 
If the DAG must be run with \MacroNI{DAGMAN\_ALWAYS\_USE\_NODE\_LOG} 
set to \Expr{False},
a good way to deal with upgrading is to use DAGMan Halt files 
to cause all of the running DAGs to drain from the queue, 
and then do the upgrade after the DAGs have stopped.  
After the upgrade is done, 
edit the per-DAG configuration files to have 
\MacroNI{DAGMAN\_ALWAYS\_USE\_NODE\_LOG} set to \Expr{True},
or set \MacroNI{DAGMAN\_USE\_STRICT} to 0 and 
re-submit the DAGs, which will then run the Rescue DAGs.

\end{itemize}

