%%%      PLEASE RUN A SPELL CHECKER BEFORE COMMITTING YOUR CHANGES!
%%%      PLEASE RUN A SPELL CHECKER BEFORE COMMITTING YOUR CHANGES!
%%%      PLEASE RUN A SPELL CHECKER BEFORE COMMITTING YOUR CHANGES!
%%%      PLEASE RUN A SPELL CHECKER BEFORE COMMITTING YOUR CHANGES!
%%%      PLEASE RUN A SPELL CHECKER BEFORE COMMITTING YOUR CHANGES!

%%%%%%%%%%%%%%%%%%%%%%%%%%%%%%%%%%%%%%%%%%%%%%%%%%%%%%%%%%%%%%%%%%%%%%
\section{\label{sec:History-7-1}Development Release Series 7.1}
%%%%%%%%%%%%%%%%%%%%%%%%%%%%%%%%%%%%%%%%%%%%%%%%%%%%%%%%%%%%%%%%%%%%%%

This is the development release series of Condor.
The details of each version are described below.

%%%%%%%%%%%%%%%%%%%%%%%%%%%%%%%%%%%%%%%%%%%%%%%%%%%%%%%%%%%%%%%%%%%%%%
\subsection*{\label{sec:New-7-1-1}Version 7.1.1}
%%%%%%%%%%%%%%%%%%%%%%%%%%%%%%%%%%%%%%%%%%%%%%%%%%%%%%%%%%%%%%%%%%%%%%

\noindent Release Notes:

\begin{itemize}

\item None.

\end{itemize}


\noindent New Features:

\begin{itemize}

\item Included some Windows example jobs (submit files and binaries).

\item Added a new feature to the DAGMan language called splicing. Please
read section~\ref{sec:DAGSplicing} on page \pageref{sec:DAGSplicing}.

\item The Prepare Job Hook can now modify the job ClassAd before execution.
For a complete description of the new hook system, read
section~\ref{sec:job-hooks} on page~\pageref{sec:job-hooks}.

\item Condor now coerces the result of \$\$([]) expressions within
submit description files to strings.
This means that submit files can do simple arithmetic.
For example, you can describe a command-line argument as:

arguments = \$\$([\$(PROCESS)+100])

and \Condor{submit} will expand the argument to be the expected value.

\item Condor daemons now periodically update the \Code{ctime} of their
  log files, instead of the \Code{mtime}, as they previously did.
  At start up, the daemons use this \Code{ctime} 
  to determine how long they may have been down.

\item Added the capability to the \Condor{startd} to allow it to power 
  down machines based a user specified policy.  See 
  section~\ref{sec:power-man} on \pageref{sec:power-man} on
  Power Management for more details.

\item \Condor{off} now supports the \Opt{-peaceful} option for the
  \Condor{schedd}, in addition to the existing support that already existed for
  the \Condor{startd}.  When peacefully shut down,
  the \Condor{schedd} stops starting new
  jobs and waits for all running jobs to finish before exiting.  The
  default shut down behavior is still \Opt{-graceful}, which checkpoints
  and stops all running standard universe jobs and gracefully
  disconnects from other types of jobs in the hopes of later restarting
  and reconnecting to them without any disturbance to the running job.

\item The \Condor{job\_router} now supports deletion of attributes
  when transforming job ClassAds from vanilla to grid universe.  It also
  behaves more deterministically when choosing from multiple possible
  routes.  Rather than picking one at random, it uses a round-robin
  selection.

\end{itemize}

\noindent Configuration Variable Additions and Changes:

\begin{itemize}

\item The existing \Macro{BIND\_ALL\_INTERFACES} configuration variable
  now defaults to \Expr{True}.

\item Added the \Macro{HIBERNATE} expression, which, when evaluated in
  the context of each slot, determines if a machine should enter
  a low power state. See page~\pageref{param:Hibernate} for more 
  information.

\item Added the \Macro{HIBERNATE\_CHECK\_INTERVAL} configuration variable,
  which, if set to a non-zero value, enables the \Condor{startd} to place the 
  machine in a low power state based on the evaluation of the
  \MacroNI{HIBERNATE} expression.  See 
  page~\pageref{param:HibernateCheckInterval} for more information.

\item The existing \Macro{VALID\_SPOOL\_FILES} configuration variable
  now automatically includes \File{SCHEDD.lock},
  the lock file used for high availability \Condor{schedd} fail over.
  Other high availability lock files are not currently included.

\item Added the \Macro{SEC\_DEFAULT\_AUTHENTICATION\_TIMEOUT} configuration
  variable, where the definition \Expr{DEFAULT} may be replaced
  by the usual list of contexts for security settings
  (for example, \Expr{CLIENT}, \Expr{READ}, and \Expr{WRITE}).
  This specifies the number of seconds that Condor should
  allow for the authentication of network connections to complete.
  Previously, GSI authentication was hard-coded to allow 5 minutes
  for authentication.
  Now it uses the same default as all other methods: 20 seconds.

\item Added the \Macro{STARTER\_UPDATE\_INTERVAL\_TIMESLICE} configuration
  variable, which
  specifies the highest fraction of time that the \Condor{starter} should spend
  collecting monitoring information about the job, such as disk usage.
  It defaults to 0.1.  If checking the disk usage of the job takes a
  long time, the \Condor{starter} will monitor less frequently than 
  specified by \MacroNI{STARTER\_UPDATE\_INTERVAL}.

\end{itemize}

\noindent Bugs Fixed:

\begin{itemize}

\item Fixed a bug in Java universe where each slot would be told to
  potentially use all the memory on the machine.  Now, each JVM 
  receives the physical memory divided by the number of slots.

\item On Windows, slot users would sometimes show up in the Windows Welcome
  Screen.  This has now been resolved.
  The slot users need to be manually
  removed for this to take effect and the machine may need to be rebooted for
  the setting to be honored.

\item Fixed a bug in the ClassAd \Procedure{string} function.
  The function now properly converts integers and floats
  to their string representation.

\item The Windows Installer is now completely internationalized: it will no 
  longer fail to install because of a missing "Users" group; instead, it
  will use the regionally appropriate group.

\item Interoperability with Samba (as a PDC) has been improved.  Condor 
  uses a fast form of login during credential validation.  Unfortunately, 
  this login procedure fails under Samba, even if the credentials are 
  valid.  The new behavior is to attempt the fast login, and on failure, 
  fall back to the slower form.

\item Windows slot users no longer have the Batch Privilege added, nor 
  does Condor first attempt a Batch login for slot users.  This was 
  causing permission problems on hardened versions of Windows, such 
  as Windows Sever 2003, in that not interactive users lacked the 
  permission to run batch files (via the \Prog{cmd.exe} tool). This affected 
  any user submitting jobs that used batch files as the executable.

% issue [#1516]
\item If the \AdAttr{IWD} is not defined in a job classified
  ad that was either fetched by the \Condor{startd} via job hooks, or
  pushed to the \Condor{startd} via COD, the \Condor{starter} no
  longer treats this as a fatal error, and instead uses the temporary
  job execution sandbox as the initial working directory.

% Fixes requested by LIGO
\item Made some fixes to the new-style rescue DAG feature:
\begin{itemize}
\item \Condor{submit\_dag} no longer needs the \Opt{-force} flag if a rescue
DAG will be run, even if the files generated by \Condor{submit\_dag}
already exist.
\item \Condor{submit\_dag} with the \Opt{-force} flag now renames any
existing new-style rescue DAG files, and therefore runs the original DAG.
\end{itemize}

% PR 942
\item Fixed a problem that caused new-style rescue DAGs to fail when
\Condor{submit\_dag} is invoked with the \Opt{-usedagdir} flag.

\end{itemize}

\noindent Known Bugs:

\begin{itemize}

\item None.

\end{itemize}

\noindent Additions and Changes to the Manual:

\begin{itemize}

\item The manual now contains Windows installation instructions for
  controlling the configuration for the \SubmitCmd{vm} universe.

\end{itemize}



%%%%%%%%%%%%%%%%%%%%%%%%%%%%%%%%%%%%%%%%%%%%%%%%%%%%%%%%%%%%%%%%%%%%%%
\subsection*{\label{sec:New-7-1-0}Version 7.1.0}
%%%%%%%%%%%%%%%%%%%%%%%%%%%%%%%%%%%%%%%%%%%%%%%%%%%%%%%%%%%%%%%%%%%%%%

\noindent Release Notes:

\begin{itemize}

\item Upgrading to 7.1.0 from previous versions of Condor will make
existing Standard Universe jobs that have already run fail to match to
machines running Condor 7.1.0 unless the job previously ran on a
machine using the Red Hat 5.0 release of Condor.  This is because the
value of the \Attr{CheckpointPlatform} attribute of the machine
ClassAd has changed in order to better represent checkpoint
compatibility.  If this affects you, you can use \Condor{qedit} to
change the \Attr{LastCheckpointPlatform} attribute of existing
Standard Universe jobs to match the new \Attr{CheckpointPlatform}
advertised by the machine ClassAd where the job last ran.

\item Condor no longer supports root configuration files
(for example, \File{/etc/condor/condor\_config.root},
\File{~condor/condor\_config.root}, and
the file defined by the configuration variable
\MacroNI{LOCAL\_ROOT\_CONFIG\_FILE}).  This feature was intended to
give limited powers to a Unix administrator to configure some aspects
of Condor without gaining root powers.  However, given the flexibility
of the configuration system, we decided that this was not practical.
As long as Condor is started up as root, it should be clearly
understood that whoever has the ability to edit the Condor
configuration files can effectively run arbitrary programs as root.

\end{itemize}


\noindent New Features:

\begin{itemize}

\item In the past, Condor has always sent work to the execute machines
  by pushing jobs to the \Condor{startd}, either from the
  \Condor{schedd} or via \Condor{cod}.
  As of version 7.1.0, The \Condor{startd} now has the ability to pull
  work by fetching jobs via a system of plug-ins or hooks.
  Additional hooks are invoked by the \Condor{starter} to help manage
  work (especially for fetched jobs, but the \Condor{starter} hooks
  can be defined and invoked for other kinds of jobs as well).
  For a complete description of the new hook system, read
  section~\ref{sec:job-hooks} on page~\pageref{sec:job-hooks}.

% PR 888/921
\item Added the capability to insert commands into the \File{.condor.sub}
  file produced by \Condor{submit\_dag} with the \Opt{-append} and
  \Opt{-insert\_sub\_file} command-line arguments to \Condor{submit\_dag} and
  the \Macro{DAGMAN\_INSERT\_SUB\_FILE} configuration variable.
  See the \Condor{submit\_dag} manual page on
  page~\pageref{man-condor-submit-dag}
  and the configuration variable definition on
  page~\pageref{param:DAGManInsertSubFile} for more information.

\item For platforms running a Windows operating system, the \Attr{Arch}
  machine ClassAd attribute more correctly reflects the architectures
  supported.  Instead of values \AdStr{INTEL} and \AdStr{UNDEFINED},
  the values will now be: \AdStr{INTEL} for x86,
  \AdStr{IA64} for Intel Itanium,
  and \AdStr{X86\_64} for both AMD and Intel 64-bit processors.
  These values are listed in the unnumbered subsection labeled
  Machine ClassAd Attributes on page~\pageref{sec:Machine-ClassAd-Attributes}.

\item The Windows MSI installer now supports extended \SubmitCmd{vm} universe 
  options. These new options include: the ability to set the 
  networking type, how much memory the \SubmitCmd{vm} universe can use 
  on a host, and
  the ability to set the version of \Prog{VMware} installed on the host.

\item The \Condor{status} and \Condor{q} command line tools now have a
  version option which prints the version of those specific tools.  This
  can be useful when multiple versions of Condor are installed on the
  same machine.

\item The configuration variable \MacroNI{CONDOR\_VIEW\_HOST} may now
  contain a port number and may (if desired) refer to a
  \Condor{collector} daemon running on the same host as the
  \Condor{collector} that is forwarding ads.  It is also now possible to
  use the forwarded ads for matchmaking purposes.  For example, several
  collectors could forward ads to a single aggregating collector which
  a \Condor{negotiator} then uses as its source of information for
  matchmaking.

\item Added client-side authorization controls
\MacroNI{ALLOW\_CLIENT}, \MacroNI{DENY\_CLIENT}.  When using a mutual
authentication method (e.g. GSI, SSL, Kerberos), this allows you to
specify what authenticated servers Condor tools and daemons should
trust when they form a connection to the server.  This deprecates
\MacroNI{GSI\_DAEMON\_NAME}, which provided rudimentary support for
client-side authorization in a GSI-specific way.

% PR 598/788
\item \Condor{dagman} deals with rescue DAGs in a more sophisticated
way; this is especially helpful for nested DAGs.
See the rescue DAG subsection~\pageref{sec:DAGRescue} of the \Condor{dagman}
manual section for more information.

\item Additional logging details for unusual error cases to help 
identify problems.

\item A new (optional) daemon named \Condor{job\_router} has been
added, so far only on unix.  It may be configured to transform vanilla
universe jobs into grid universe jobs, for example to send excess jobs
to other sites via Condor-C or Condor-G.  For details, see
page~\pageref{sec:JobRouter}.

\item Previously, \condor{q} \Opt{-better-analyze} was supported on most
but not all versions of Linux.  It is now supported on all Unix platforms
but not yet on Windows.

\end{itemize}

\noindent Configuration Variable Additions and Changes:

\begin{itemize}

% PR 921
\item Added the \Macro{DAGMAN\_INSERT\_SUB\_FILE} variable, which allows a file
  of commands to be inserted into \File{.condor.sub} files generated
  by \Condor{submit\_dag}.  See page~\pageref{param:DAGManInsertSubFile}
  for more information.

\item The semantics of \MacroNI{CLAIM\_WORKLIFE} were previously not
clearly defined before the start of the first job.  A delay between
the \Condor{schedd} claiming a slot and the \Condor{shadow} starting a
job could be caused by the submit machine being very busy or by
\MacroNI{JOB\_START\_DELAY}.  Previously, such a delay would
unpredictably result in the first job being rejected if
\MacroNI{CLAIM\_WORKLIFE} expired during that time.  Now,
\MacroNI{CLAIM\_WORKLIFE} is defined to apply only after the first job
has started.  Therefore, setting it to zero has the effect of allowing
exactly one job per claim to run.  The default is still the special
value -1, which places no limit on how long the slot may continue
accepting new jobs from the \Condor{schedd} that claimed it.

% PR 598/788
\item Added the \Macro{DAGMAN\_OLD\_RESCUE} variable, which controls whether
\Condor{dagman} writes rescue DAGs in the old way.  See
page~\pageref{param:DAGManOldRescue} for more information.

% PR 598/788
\item Added the \Macro{DAGMAN\_AUTO\_RESCUE} variable, which controls
whether \Condor{dagman} automatically runs an existing rescue DAG.
See page~\pageref{param:DAGManAutoRescue} for more information.

% PR 598/788
\item Added the \Macro{DAGMAN\_MAX\_RESCUE\_NUM} variable, which
controls the maximum "new-style" rescue DAG number written or
automatically run by \Condor{dagman}.
See page~\pageref{param:DAGManMaxRescueNum} for more information.

\end{itemize}

\noindent Bugs Fixed:

\begin{itemize}

\item The Condor Build ID is now printed by \Condor{version} and placed 
  in the logs for machines running a Windows operating system.

\item \Condor{quill} and the \Condor{dbmsd} correctly register 
  themselves with the Windows firewall.

% PR 926
\item \Condor{submit\_dag} now avoids possibly running off the end
of the argument list if an argument requiring a value does not have one.

\item The \Condor{submit\_dag} \Opt{-debug} argument now must be
specified with at least \Opt{-de} to avoid conflict with the
\Opt{-dagman} argument.

\item Added missing information about the \Opt{-config} argument to
\Condor{submit\_dag}'s usage message.

% PR 927
\item \Condor{dagman} no longer considers duplicate edges in a DAG a
fatal error (it is now a warning).

\end{itemize}

\noindent Known Bugs:

\begin{itemize}

\item No hook is invoked if a fetched job does not contain enough data
  to be spawned by a \Condor{starter} or if other errors prevent the
  job from being run after the \Condor{startd} agrees to accept the
  work.
  This limitation will be addressed in a future version of Condor,
  most likely via the addition of a new hook invoked whenever the
  \Condor{starter} fails to spawn a job.
  For more information about the new hook system included in Condor
  version 7.1.0, read section~\ref{sec:job-hooks} on
  page~\pageref{sec:job-hooks}.

\end{itemize}

\noindent Additions and Changes to the Manual:

\begin{itemize}

\item Added \AdStr{WINNT60} for the Vista operating system to
  the documented list of possible values for the machine ClassAd
  attribute \AdAttr{OpSys}.

\end{itemize}

