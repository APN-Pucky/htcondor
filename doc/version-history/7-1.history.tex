%%%      PLEASE RUN A SPELL CHECKER BEFORE COMMITTING YOUR CHANGES!
%%%      PLEASE RUN A SPELL CHECKER BEFORE COMMITTING YOUR CHANGES!
%%%      PLEASE RUN A SPELL CHECKER BEFORE COMMITTING YOUR CHANGES!
%%%      PLEASE RUN A SPELL CHECKER BEFORE COMMITTING YOUR CHANGES!
%%%      PLEASE RUN A SPELL CHECKER BEFORE COMMITTING YOUR CHANGES!

%%%%%%%%%%%%%%%%%%%%%%%%%%%%%%%%%%%%%%%%%%%%%%%%%%%%%%%%%%%%%%%%%%%%%%
\section{\label{sec:History-7-1}Development Release Series 7.1}
%%%%%%%%%%%%%%%%%%%%%%%%%%%%%%%%%%%%%%%%%%%%%%%%%%%%%%%%%%%%%%%%%%%%%%

This is the development release series of Condor.
The details of each version are described below.


%%%%%%%%%%%%%%%%%%%%%%%%%%%%%%%%%%%%%%%%%%%%%%%%%%%%%%%%%%%%%%%%%%%%%%
\subsection*{\label{sec:New-7-1-0}Version 7.1.0}
%%%%%%%%%%%%%%%%%%%%%%%%%%%%%%%%%%%%%%%%%%%%%%%%%%%%%%%%%%%%%%%%%%%%%%

\noindent Release Notes:

\begin{itemize}

\item Condor no longer supports root configuration files
(e.g. \File{/etc/condor/condor\_config.root},
\File{~condor/condor\_config.root}, and
\MacroNI{LOCAL\_ROOT\_CONFIG\_FILE}).  This feature was intended to
give limited powers to a unix administrator to configure some aspects
of Condor without gaining root powers.  However, given the flexibility
of the configuration system, we decided that this was not practical.
As long as Condor is started up as root, it should be clearly
understood that whoever has the ability to edit the condor
configuration files can effectively run arbitrary programs as root.

\end{itemize}


\noindent New Features:

\begin{itemize}

% PR 888/921
\item Added the capability to insert commands into the \File{.condor.sub}
  file produced by \Condor{submit\_dag} with the \Opt{-append} and
  \Opt{-insert\_sub\_file} command-line arguments to \Condor{submit\_dag} and
  the \Macro{DAGMAN\_INSERT\_SUB\_FILE} configuration variable.
  See the \Condor{submit\_dag} manual page on
  page~\pageref{man-condor-submit-dag}
  and the configuration variable definition on
  page~\pageref{param:DAGManInsertSubFile} for more information.

\item For platforms running a Windows operating system, the \Attr{Arch}
  machine ClassAd attribute more correctly reflects the architectures
  supported.  Instead of values \AdStr{INTEL} and \AdStr{UNDEFINED},
  the values will now be: \AdStr{INTEL} for x86,
  \AdStr{IA64} for Intel Itanium,
  and \AdStr{X86\_64} for both AMD and Intel 64-bit processors.
  These values are listed in the unnumbered subsection labeled
  Machine ClassAd Attributes on page~\pageref{sec:Machine-ClassAd-Attributes}.

\item The Windows MSI installer now supports extended \SubmitCmd{vm} universe 
  options. These new options include: the ability to set the 
  networking type, how much memory the \SubmitCmd{vm} universe can use 
  on a host, and
  the ability to set the version of \Prog{VMware} installed on the host.

\item The \Condor{status} and \Condor{q} command line tools now have a
  version option which prints the version of those specific tools.  This
  can be useful when multiple versions of Condor are installed on the
  same machine.

\item The configuration variable \MacroNI{CONDOR\_VIEW\_HOST} may now
  contain a port number and may (if desired) refer to a
  \Condor{collector} daemon running on the same host as the
  \Condor{collector} that is forwarding ads.  It is also now possible to
  use the forwarded ads for matchmaking purposes.  For example, several
  collectors could forward ads to a single aggregating collector which
  a \Condor{negotiator} then uses as its source of information for
  matchmaking.

\item Added client-side authorization controls
\MacroNI{ALLOW\_CLIENT}, \MacroNI{DENY\_CLIENT}.  When using a mutual
authentication method (e.g. GSI, SSL, Kerberos), this allows you to
specify what authenticated servers Condor tools and daemons should
trust when they form a connection to the server.  This deprecates
\MacroNI{GSI\_DAEMON\_NAME}, which provided rudamentary support for
client-side authorization in a GSI-specific way.

% PR 598/788
\item \Condor{dagman} deals with rescue DAGs in a more sophisticated
way; this is especially helpful for nested DAGs.

\end{itemize}

\noindent Configuration Variable Additions and Changes:

\begin{itemize}

% PR 921
\item Added the \Macro{DAGMAN\_INSERT\_SUB\_FILE} variable, which allows a file
  of commands to be inserted into \File{.condor.sub} files generated
  by \Condor{submit\_dag}.  See page~\pageref{param:DAGManInsertSubFile}
  for more information.

\item The semantics of \MacroNI{CLAIM\_WORKLIFE} were previously not
clearly defined before the start of the first job.  A delay between
the \Condor{schedd} claiming a slot and the \Condor{shadow} starting a
job could be caused by the submit machine being very busy or by
\MacroNI{JOB\_START\_DELAY}.  Previously, such a delay would
unpredictably result in the first job being rejected if
\MacroNI{CLAIM\_WORKLIFE} expired during that time.  Now,
\MacroNI{CLAIM\_WORKLIFE} is defined to apply only after the first job
has started.  Therefore, setting it to zero has the effect of allowing
exactly one job per claim to run.  The default is still the special
value -1, which places no limit on how long the slot may continue
accepting new jobs from the \Condor{schedd} that claimed it.

% PR 598/788
\item Added the \Macro{DAGMAN\_OLD\_RESCUE} variable, which controls whether
\Condor{dagman} writes rescue DAGs in the old way.

% PR 598/788
\item Added the \Macro{DAGMAN\_AUTO\_RESCUE} variable, which controls
whether \Condor{dagman} automatically runs an existing rescue DAG.

\end{itemize}

\noindent Bugs Fixed:

\begin{itemize}

\item The Condor Build ID is now printed by \Condor{version} and placed 
  in the logs for machines running a Windows operating system.

\item \Condor{quill} and the \Condor{dbmsd} correctly register 
  themselves with the Windows firewall.

% PR 926
\item \Condor{submit\_dag} now avoids possibly running off the end
of the argument list if an argument requiring a value does not have one.

\item The \Condor{submit\_dag} \Opt{-debug} argument now must be
specified with at least \Opt{-de} to avoid conflict with the
\Opt{-dagman} argument.

\item Added missing information about the \Opt{-config} argument to
\Condor{submit\_dag}'s usage message.

% PR 927
\item \Condor{dagman} no longer considers duplicate edges in a DAG a
fatal error (it is now a warning).

\end{itemize}

\noindent Known Bugs:

\begin{itemize}

\item None.

\end{itemize}

\noindent Additions and Changes to the Manual:

\begin{itemize}

\item Added \AdStr{WINNT60} for the Vista operating system to
  the documented list of possible values for the machine ClassAd
  attribute \AdAttr{OpSys}.

\end{itemize}

