%%%%%%%%%%%%%%%%%%%%%%%%%%%%%%%%%%%%%%%%%%%%%%%%%%%%%%%%%%%%%%%%%%%%%%
\section{\label{sec:History-6-7}Development Release Series 6.7}
%%%%%%%%%%%%%%%%%%%%%%%%%%%%%%%%%%%%%%%%%%%%%%%%%%%%%%%%%%%%%%%%%%%%%%

This is the development release series of Condor,
The details of each version are described below.

%%%%%%%%%%%%%%%%%%%%%%%%%%%%%%%%%%%%%%%%%%%%%%%%%%%%%%%%%%%%%%%%%%%%%%
\subsection{\label{sec:New-6-7-0}Version 6.7.0}
%%%%%%%%%%%%%%%%%%%%%%%%%%%%%%%%%%%%%%%%%%%%%%%%%%%%%%%%%%%%%%%%%%%%%%

\noindent Release Notes:

\begin{itemize}

\item Version 6.7.0 contains all of the features, ports, and bug fixes
  from the previous stable series, up to and including version 6.6.4.
  In addition, a number of new features and some bug fixes have been
  made, which are described below in more detail.

\end{itemize}


\noindent New Features:

\begin{itemize}

\item Added support for vanilla and Java jobs to reconnect when the
  connection between the submitting and execution nodes is lost for
  any reason.
  Possible reasons for this disconnect include: network outages,
  rebooting the submit machine, restarting the Condor daemons on the
  submit machine, etc.
  If the execution machine is rebooted or the Condor daemons are
  restarted, reconnection is not possible.
  To take advantage of this reconnect feature, jobs must be submitted
  with a \Attr{JobLeaseDuration}.
  There are new events in the UserLog related to disconnect and
  reconnect.

\item Added a new Condor tool, \Condor{vacate\_job}.
  This command is similar to \Condor{vacate}, except the kinds of
  arguments it takes define jobs in a job queue, not machines to
  vacate.
  For example, a user can vacate a specific job id, all the jobs in a
  given cluster, all the jobs matching a job queue constraint, or even
  all jobs owned by that user.
  The owner of a job can always vacate their own jobs, regardless of
  the pool security policy controlling \Condor{vacate} (which is an
  administrative command which acts directly on machines).
  See the new command reference, section~\ref{man-condor-vacate-job}
  on page~\pageref{man-condor-vacate-job} for details.
  
\item Added a new ``High Availability'' service to the \Condor{master}.
   You can now specify a daemon which can have ``fail over'' capabilities
   (i.e. the master on another machine can start a matching daemon if the
   first one fails).  Currently, this is only available over a shared
   file system (i.e. NFS), and has only been tested for the \Condor{schedd}.

\item Scheduler universe jobs on UNIX can now specify a
  \Attr{HoldKillSig}, the signal that should be sent when the job is
  put on hold.
  If not specified, the default is to use the \Attr{KillSig}, and if
  that is not defined, the job will be sent a SIGTERM.
  The submit file keyword to use for defining this signal is
  \AdAttr{hold\_kill\_sig}, for example,
  \verb@hold_kill_sig = SIGUSR1@.

\item The \Condor{startd} can now support policies on SMP machines
  where each virtual machine (VM) has knowledge of the other VMs on
  the same host.
  For example, if a job starts running on one of the VMs, a job
  running on another VM could immediately be suspended.
  This is accomplished by using the new configuration variable
  \Macro{STARTD\_VM\_EXPRS}, which is a list of ClassAd attribute
  names that should be shared across all VMs on the machine.
  For each VM on the machine, every attribute in this list is looked
  up in the VM-specific machine ClassAd, the attribute name is given a
  prefix indicating what VM it came from, and then inserted into the
  machine ClassAds of all the other VMs.

\item The \Condor{startd} publishes four new attributes into the
  machine ClassAds it generates when it is in the Claimed state:
  \Attr{TotalJobRunTime}, \Attr{TotalJobSuspendTime},
  \Attr{TotalClaimRunTime}, \Attr{TotalClaimSuspendTime}.
  These attributes keep track of the total time the resource was
  either running a job (in the Busy activity) or had a job suspended,
  regardless of how many suspend/resume cycles the job went through.
  The first two attributes (with ``Job'' in the name) keep track for a
  single job (i.e. since the last time the resource was
  Claimed/Idle). 
  The last two attributes (with ``Claim'' in the name) keep track of
  these totals across all jobs that ran under the same claim
  (i.e. since the last state change into the Claimed state).

\item Added a \Opt{-num} option to the \Condor{wait} tool to wait for
   a specified number of jobs to finish.

\item Added a configuration option \Macro{STARTER\_JOB\_ENVIRONMENT}
   so the admin can configure the default environment inherited by
   user jobs.

\item Added a (configurable, defaults to off) feature to the \Condor{schedd}
   to allow backup the spool file before doing anything else.

\item The "Continous" option of the \Condor{startd} ``cron'' jobs is
being deprecated.   It's being replaced by two new options which
control separate aspects of it's behaviour:
\begin{itemize}
\item "WaitForExit" specifies the "exit timing" mode
\item "ReConfig" specifies that the job can handle SIGHUPs, and it should 
be sent a SIGHUP when the \Condor{startd} is reconfigured.
\end{itemize}

\item A lot of the items logged by the \Condor{startd} ``cron'' logic,
changed to D\_FULLDEBUG (from D\_ALWAYS), etc.

\item Added \Macro{NEGOTIATOR\_PRE\_JOB\_RANK} and
\Macro{NEGOTIATOR\_POST\_JOB\_RANK}.  These expressions are applied
respectively before and after the user-supplied job rank when deciding
which of the possible matches to choose.  (The existing expression
\Macro{PREEMPTION\_RANK} is applied after
\Macro{NEGOTIATOR\_POST\_JOB\_RANK}.)  The pool administrator may use
these expressions to steer jobs in ways that improve the overall
performance of the pool.  For example, using the pre job rank,
preemption may be avoided as long as there are idle machines, even
when the user-supplied rank expression prefers a machine that happens
to be busy.  Using the post job rank, one could steer jobs towards
machines that are known to be dedicated to batch jobs, or one could
enforce breadth-first instead of depth-first filling of a cluster of
multi-processor machines.

\item Added the ability for Condor to transfer files larger than 2G on
platforms that support large files.  This works automatically for
transferred executables, input files and output files.

\item Added the ability for jobs to stream back standard input, output, and
error files while running.  This is activated by the \Opt{stream\_input},
\Opt{stream\_output}, and \Opt{stream\_error} options to \Condor{submit}.
Note that this feature is incompatible with the new feature described
above where the shadow and starter can reconnect in certain
circumstances. 

\end{itemize}


\noindent Bugs Fixed:

\begin{itemize}

\item Fixed a bug in the \Condor{startd} ``cron'' logic which caused the
\Condor{startd} to except when trying to delete a job that could never
be run (i.e. invalid executable, etc).

\item Fixed a bug in \Condor{startd} ``cron'' logic which caused it to
not detect when the starting of a ``job'' failed.

\item Fixed several bugs in the reconfiguration handling of the
\Condor{startd} ``cron'' logic.  In particular, even if the job has
the "reconfig" option set (or "continuous"), the job(s) won't be sent
a SIGHUP when the startd first starts, or when the job itself is first
run (until it output's it's first "output block", defined by the "-"
separator).

\end{itemize}


\noindent Known Bugs:

\begin{itemize}

\item None.

\end{itemize}
