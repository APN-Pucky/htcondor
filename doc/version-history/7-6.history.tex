%%%      PLEASE RUN A SPELL CHECKER BEFORE COMMITTING YOUR CHANGES!
%%%      PLEASE RUN A SPELL CHECKER BEFORE COMMITTING YOUR CHANGES!
%%%      PLEASE RUN A SPELL CHECKER BEFORE COMMITTING YOUR CHANGES!
%%%      PLEASE RUN A SPELL CHECKER BEFORE COMMITTING YOUR CHANGES!
%%%      PLEASE RUN A SPELL CHECKER BEFORE COMMITTING YOUR CHANGES!

%%%%%%%%%%%%%%%%%%%%%%%%%%%%%%%%%%%%%%%%%%%%%%%%%%%%%%%%%%%%%%%%%%%%%%
\section{\label{sec:History-7-6}Stable Release Series 7.6}
%%%%%%%%%%%%%%%%%%%%%%%%%%%%%%%%%%%%%%%%%%%%%%%%%%%%%%%%%%%%%%%%%%%%%%

This is a stable release series of Condor.
As usual, only bug fixes (and potentially, ports to new platforms)
will be provided in future 7.6.x releases.
New features will be added in the 7.7.x development series.

The details of each version are described below.

%%%%%%%%%%%%%%%%%%%%%%%%%%%%%%%%%%%%%%%%%%%%%%%%%%%%%%%%%%%%%%%%%%%%%%
\subsection*{\label{sec:New-7-6-0}Version 7.6.0}
%%%%%%%%%%%%%%%%%%%%%%%%%%%%%%%%%%%%%%%%%%%%%%%%%%%%%%%%%%%%%%%%%%%%%%

\noindent Release Notes:

\begin{itemize}

\item Condor version 7.6.0 not yet released.
%\item Condor version 7.6.0 released on Month Date, 2011.

% gittrac #2016
\item Prior to Condor version 7.5.0, commenting out \MacroNI{PREEN} in the
  default configuration file disabled \Condor{preen}.  
  Starting in Condor version 7.5.0,
  there was an internal default value for \MacroNI{PREEN}, so if
  the configuration variable was not set in any configuration file,
  \Condor{preen} would still run.
  To disable this functionality, \MacroNI{PREEN} can be explicitly set to
  nothing.

\end{itemize}


\noindent New Features:

\begin{itemize}

\item Condor can now create and manage virtual machine instances in a
cloud service via Deltacloud. This is done via the new
\SubmitCmd{deltacloud} grid type in the grid universe.
See section ~\ref{sec:Deltacloud} for details.

% gittrac #1931
\item Improved scalability of submission of cream grid type jobs.

\end{itemize}

\noindent Configuration Variable and ClassAd Attribute Additions and Changes:

\begin{itemize}

\item The new configuration variable \Macro{DELTACLOUD\_GAHP} specifies
where the \Prog{deltacloud\_gahp} binary is located. This binary is used to
manage deltacloud grid type jobs in the grid universe.
In a normal Condor installation, the value should be
\File{\$(SBIN)/deltacloud\_gahp}.

\item Several new job ClassAd attributes have been added to support
the deltacloud grid type in the grid universe.
These attributes are taken from the submit description file:
\AdAttr{DeltacloudUsername},
\AdAttr{DeltacloudPasswordFile},
\AdAttr{DeltacloudImageId},
\AdAttr{DeltacloudRealmId},
\AdAttr{DeltacloudHardwareProfile},
\AdAttr{DeltacloudHardwareProfileCpu},
\AdAttr{DeltacloudHardwareProfileMemory},
\AdAttr{DeltacloudHardwareProfileStorage},
\AdAttr{DeltacloudKeyname}, and
\AdAttr{DeltacloudUserData}.
%\AdAttr{DeltacloudRetryTimeout},
These attributes are set by Condor when the instance runs:
\AdAttr{DeltacloudAvailableActions},
\AdAttr{DeltacloudPrivateNetworkAddresses},
\AdAttr{DeltacloudPublicNetworkAddresses}.
See section ~\ref{sec:Deltacloud} for details of running jobs under
Deltacloud, and see section ~\ref{sec:Job-ClassAd-Attributes}
for definitions of these job ClassAd attributes.

% gittrac #2024
\item The configuration variable \Macro{JAVA\_MAXHEAP\_ARGUMENT} 
  has been removed. 
  This means that Java universe jobs will now run with the JVM's 
  default maximum heap setting,
  unless specified otherwise by the administrator using the configuration
  of \Macro{JAVA\_EXTRA\_ARGUMENTS},
  or by the job via 
  \SubmitCmd{java\_vm\_args} in the submit description file
  as described in section~\ref{sec:Java}.

\end{itemize}

\noindent Bugs Fixed:

\begin{itemize}

% gittrac #1957
\item Fixed a bug in \Condor{dagman} that caused it to fail when in recovery
mode in the face of a specific combination of node job failures with
retries.

% gittrac #1991
\item Fixed a bug that resulted in the spooled user log not being
  fetched by \Condor{transfer\_data}.  Prior to Condor version 7.5.4, this
  problem affected spooled jobs submitted with an explicit list of
  output files to transfer.  In Condor version 7.5.4, this problem also
  affected spooled jobs that auto-detected output files.

% gittrac #1985
\item When a job is submitted with the \Opt{-spool} option to \Condor{submit},
the \Condor{schedd} now correctly writes the submit event to the user log 
in its spool directory. 
Previously, it would try to write the event using the user
log path given to \Condor{submit} by the user, 
which \Condor{submit} may not have access to.

% gittrac #2001
\item Fixed a file descriptor leak in the \Condor{vm-gahp}. The leak would
cause the daemon to fail if a VMware job ran for more than 16 hours on a
Linux machine.

%gittrac #2017
\item Fixed a bug in \Condor{dagman} that caused it to treat any dollar
sign in the log file name of a node job's submit description file
as an illegal \Condor{dagman} macro.
Now only the sequence of characters \Expr{\$(} delimits a macro.

\end{itemize}

\noindent Known Bugs:

\begin{itemize}

\item There are two known issues related to the automatic creation
of checkpoints with the Condor checkpointing library in 
Condor version 7.6.0.
The first is that compression of
standalone checkpoints is disabled for 32-bit binaries.
It was always disabled previously, for 64-bit binaries.
A standalone checkpoint is one that is run outside
of Condor's standard universe.  The second problem has to do with compressed
32-bit checkpoint files within the standard universe.
If a user requests a compressed 32-bit checkpoint in the standard universe,
the resulting checkpoint will not be compressed.
As with standalone checkpoints, this has never been supported
in 64-bit binaries.  We hope to fix both problems in 
Condor version 7.6.1.

\end{itemize}

\noindent Additions and Changes to the Manual:

\begin{itemize}

\item None.

\end{itemize}

