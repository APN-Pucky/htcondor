%%%%%%%%%%%%%%%%%%%%%%%%%%%%%%%%%%%%%%%%%%%%%%%%%%%%%%%%%%%%%%%%%%%%%%
\section{\label{sec:History-6-5}Development Release Series 6.5}
%%%%%%%%%%%%%%%%%%%%%%%%%%%%%%%%%%%%%%%%%%%%%%%%%%%%%%%%%%%%%%%%%%%%%%

This is the development release series of Condor,
The details of each version are described below.

%%%%%%%%%%%%%%%%%%%%%%%%%%%%%%%%%%%%%%%%%%%%%%%%%%%%%%%%%%%%%%%%%%%%%%
\subsection{\label{sec:New-6-5-5}Version 6.5.5}
%%%%%%%%%%%%%%%%%%%%%%%%%%%%%%%%%%%%%%%%%%%%%%%%%%%%%%%%%%%%%%%%%%%%%%

\noindent New Features:

\begin{itemize}

\item Added initial support for the \Condor{gridshell}.
  Documentation is not yet available, but will be coming soon.

\item Added the \Macro{TRUST\_UID\_DOMAIN} config file setting.
  For more information about this feature, see
  section~\ref{param:TrustUidDomain} on
  page~\pageref{param:TrustUidDomain}. 

\end{itemize}

\noindent Bugs Fixed:

\begin{itemize}

\item The \Condor{starter} was reporting ImageSize to be much bigger
than reality for multi-threaded jobs in Linux.  If the jobs were ever evicted,
this could cause them to never match to another machine or to be
unnecessarily restricted.

\item The \Condor{starter} no longer seg-faults when attempting to run
  a job as nobody on HPUX.

\item Fixed a bug related to defining \Macro{CONDOR\_IDS} in the
  configuration file.
  This new feature was added in version 6.5.3, but it did not work
  properly with the Standard and PVM universes.
  Now, all parts of Condor should work correctly without permission
  problems if \MacroNI{CONDOR\_IDS} is defined in the config file.

\item Under certain rare circumstances (e.g., running \Condor{rm} on a
      \Condor{dagman} job which was running a POST script for a node
      which had no PRE script defined), DAGMan could dereference a
      nonexistent pointer and segfault.  This has now been fixed.

\end{itemize}

\noindent Known Bugs:

\begin{itemize}

\item There is a suspected-but-not-yet-confirmed bug in DAGMan's
      \Opt{-MaxJobs} feature that can lead to a segfault which
      re-occurs during recovery.  Users are advised to avoid using
      this feature until the bug is found and fixed.  \Condor{dagman}
      jobs submitted without the \Opt{-MaxJobs} feature are not
      affected.  Likewise, the \Opt{-MaxPre} and \Opt{-MaxPost}
      features are not affected and can be safely used.

\end{itemize}



%%%%%%%%%%%%%%%%%%%%%%%%%%%%%%%%%%%%%%%%%%%%%%%%%%%%%%%%%%%%%%%%%%%%%%
\subsection{\label{sec:New-6-5-4}Version 6.5.4}
%%%%%%%%%%%%%%%%%%%%%%%%%%%%%%%%%%%%%%%%%%%%%%%%%%%%%%%%%%%%%%%%%%%%%%

\noindent New Features:

\begin{itemize}

\item Added the \Opt{-jobad} option to the \Condor{cod} command-line
  tool when activiating Computing On Demand (COD) jobs.
  This allows the user to specify dynamic attributes at the point when
  they request the job to start, instead of having to define
  everything ahead of time in the Condor configuration files at the
  execution site.

\end{itemize}

\noindent Bugs Fixed:

\begin{itemize}

\item The \Condor{startd} will now immediately exit with a fatal error
  if there is a syntax error in any of the policy expressions
  (\MacroNI{START}, \MacroNI{SUSPEND}, \MacroNI{PREEMPT}, etc) as
  defined in the configuration file.
  Before, the \Condor{startd} might run for a long time before
  reporting the error, or would silently ignore the expression.
  For example, if the \MacroNI{START} expression contained a syntax
  error, the \Condor{startd} would never match against any resources
  without warning the administrator that the \MacroNI{START}
  expression could not be parsed.

\item The \Condor{starter} now checks to make sure that the value of
      the \MacroNI{JOB\_RENICE\_INCREMENT} expression is within the
      valid range of 0 to 19, and adjusts out-of-range values to the
      closest valid value, printing a warning to the StarterLog.
      Previously, no warnings were printed.  Also, under Windows,
      values above 19 used to be incorrectly adjusted to 0 (i.e.,
      normal priority) instead of 19 (i.e., idle priority class).

\item When Condor is testing to see if a given submit machine is in
  the same \MacroNI{UID\_DOMAIN} as the execution machine, it now uses 
  a case-insensitive test.
  Previous versions of Condor that used a case-sensitive comparison
  caused problems for sites that have mixed-case hostnames in their
  DNS records.

\item When the \Condor{shadow} is trying to claim a \Condor{startd},
  if the IP address of the \Condor{shadow} cannot be resolved, the
  \Condor{startd} used to refuse the request.
  Now, the \Condor{startd} accepts the request and uses the IP address
  of the \Condor{shadow} in all log messages, tool output, and the
  \Attr{ClientMachine} ClassAd attribute that are normally set to the
  hostname.

\item Process handles would intermittantly be lost on Windows, causing
  the \Condor{startd} and other daemons to EXCEPT(). This has been fixed.

\end{itemize}

\noindent Known Bugs:

\begin{itemize}

\item None.

\end{itemize}


%%%%%%%%%%%%%%%%%%%%%%%%%%%%%%%%%%%%%%%%%%%%%%%%%%%%%%%%%%%%%%%%%%%%%%
\subsection{\label{sec:New-6-5-3}Version 6.5.3}
%%%%%%%%%%%%%%%%%%%%%%%%%%%%%%%%%%%%%%%%%%%%%%%%%%%%%%%%%%%%%%%%%%%%%%

% Karen's table
\begin{center}
\begin{table}[hbt]
\begin{tabular}{|ll|} \hline
\emph{Architecture} & \emph{Operating System} \\ \hline \hline
Hewlett Packard PA-RISC (both PA7000 and PA8000 series) & HPUX 10.20 \\ \hline
Sun SPARC Sun4m,Sun4c, Sun UltraSPARC & Solaris 2.6, 2.7, 8, 9 \\ \hline
Silicon Graphics MIPS (R5000, R8000, R10000) & IRIX 6.5 \\ \hline
Intel x86 & RedHat Linux 6.2, 7.2 \\
 & RedHat Linux 8 (clipped) \\ \hline
 & RedHat Linux 9 (clipped) \\ \hline
 & Windows NT 4.0 (clipped) \\ \hline
 & Windows NT 2000 (clipped) \\ \hline
 & Windows NT XP (clipped) \\ \hline
ALPHA & Digital Unix 4.0 \\
 & RedHat Linux 7.2 (clipped) \\ \hline
 & Tru64 5.1 (clipped) \\ \hline
PowerPC & Macintosh OS X (clipped) \\
Itanium & RedHat Linux 7.2 (clipped) \\
\end{tabular}
\caption{\label{vers-hist-sup-plat}Condor 6.5.3 supported platforms}
\end{table}
\end{center}

\noindent New Features:

\begin{itemize}

\item Added a new installation and configuration tool to Condor.
  Instead of using the complicated \Condor{install} script to install
  and configure Condor, users can now try the simplified
  \Condor{configure} script.
  The new method does not ask any interactive questions.
  Instead, you set the few installation options you need to specify
  for your site as command-line arguments.
  For most sites, this method is much better than the question-driven
  \Condor{install} script, and we recommend that new users of Condor
  try \Condor{configure}.
  All the options it supports are documented in the new
  \Condor{configure} man page, section~\ref{man-condor-configure} on
  page~\pageref{man-condor-configure}. 
  However, since this is the first public release of the new script,
  there may be some problems with how it works under certain
  conditions, so we are still including \Condor{install} in the 6.5.3 
  release if you run into problems with \Condor{configure} and would
  prefer to use the old method. 

\item You can now define \MacroNI{CONDOR\_IDS} in the Condor
  configuration files, not just as an environment variable.
  This setting is used to specify what effective user id the Condor
  daemons should run as when they are spawned with root privileges.
  This is the effective user id that Condor daemons write to their log
  files as, manipulate the job queue, and so on.
  Therefore, the \File{log} and \File{spool} directories should be
  owned by this user.
  If the Condor daemons are spawned as root and \verb@CONDOR_IDS@ is
  not set, Condor searches for a ``condor'' user in the local user
  information database (the \File{/etc/passwd} file, NIS, etc).
  For more information about the \verb@CONDOR_IDS@ setting, see
  section~\ref{sec:uids} on page~\pageref{sec:uids} and/or
  section~\ref{param:CondorIds} on page~\pageref{param:CondorIds}.

\item Added the \Dflag{HOSTNAME} debugging level to print out verbose
  messages regarding how Condor resolves hostnames, domain names,
  IP addresses, and so on.
  If set, the Condor daemons also print out messages showing how they
  initialize their own local hostname and IP address.
  This is useful for sites that are having trouble getting Condor to
  work because of problems with their DNS or NIS installation.

\item Improvements for Computing On Demand (COD) support: 
  \begin{itemize}
    \item Simplified the interface to the \Condor{cod} command-line tool
      for managing COD claims.
      All commands that require a ClaimID argument no longer require
      \Opt{-name} or \Opt{-addr}, since the ClaimID contains the
      contact information for the \Condor{startd} that created it. 
    \item If you request a claim from a specific virtual machine with
      the \Opt{-name} option to \Condor{cod\_request} (for example
      \verb$condor_cod_request -name vm3@hostname$)
      the tool will now automatically append a requirement to your
      request to ensure that your claim comes from the virtual machine
      you requested.
    \item \verb@condor_status -cod -long@ now provides the ClaimID
      string for any COD claims in your pool. 
      This is useful in case you misplace or do not save the ClaimID
      returned to you from \Condor{cod\_request}.
  \end{itemize}

\item Improvements for PVM support:
  \begin{itemize}
  \item Added support for \Procedure{pvm\_export}.
  \item Added support for any tasks the Master spawns to also spawn other
	tasks. Before this fix, only the master process could spawn tasks.
  \item Added the ability for multiple tasks to be run on a resource
	simultaneously. 
  \item Added support for Condor-PVM to run multiple tasks under a single
	Condor-PVM starter. Currently, this feature must be turned on
	by a specific option in the submit description file.
  \end{itemize}

\item Improvements for Windows:
  \begin{itemize}
  \item Added support for encrypting the execute directory
    on Windows 2000/XP. This feature is enabled by setting
    \MacroNI{ENCRYPT\_EXECUTE\_DIRECTORY} = True.
  \item Added the \MacroNI{VMx\_USER} config file parameter for specifying
    an account to run jobs as (instead of the condor nobody accounts). A
    different account can be specified for each VM.
  \end{itemize}

\item Added new option \Opt{-dagman} to \Condor{submit\_dag} which allows 
specification of an alternate \Condor{dagman} executable. 

\item Added new entry in a DAG input file that adds a macro (using the
"+" syntax) for use in the submit description file.  The new entry
is called VARS.

\item Improved \Opt{-config} option to \Condor{config\_val} to print
      descriptive lines to stderr and filenames to stdout, so that its
      output can be cleanly piped into tools like xargs.

\item On Unix, added the \MacroNI{VMx\_USER} config file parameter for
specifying an account to run jobs as instead of ``nobody''. (i.e. when
\Macro{UID\_DOMAIN}s do not match and \Macro{SOFT\_UID\_DOMAIN} =
false). A different account can be specified for each VM.

\item Added \Macro{EXECUTE\_LOGIN\_IS\_DEDICATED}. When set to True,
this tells the \Condor{starter} to track job processes by login, instead
of by process tree, which prevents so-called lurker processes. This is
turned off by default.

\end{itemize}

\noindent Bugs Fixed:

\begin{itemize}

\item Fixed the serious bug with periodic checkpointing that was
  introduced in version 6.5.2.

\item Fixed a security hole in the \Condor{schedd} when it was running
  not as root on UNIX.
  Starting with Condor version 6.5.1, if you did not run the
  \Condor{schedd} as root, it would incorrectly allow other users to
  modify the job queue.
  In particular, a user could remove another user's jobs, to submit
  jobs to another user's personal \Condor{schedd}, and so on.
  This bug has now been fixed, and the \Condor{schedd} will no longer
  allow users to remove or modify other user's jobs.

\item Fixed some bugs in the \Condor{kbdd} that prevented it from
  communicating with the \Condor{startd} to send updates about console
  activity (key presses and mouse movements).
  The \Condor{kbdd} is only needed on Digital Unix and IRIX platforms,
  so this bug did not effect most Condor users.

\item Fixed a minor bug in the \Condor{startd} that caused it to exit
  with a fatal error when it fails to execute the \Condor{starter}. 
  This problem is now handled gracefully and is not considered a fatal
  error. 

\item Fixed a minor problem introduced in version 6.5.2 in
  \Condor{submit}.
  New attributes for specifying Condor's File Transfer mechanism were
  added in version 6.5.2.
  \Condor{submit} was supposed to be backwards compatible with old job
  description files, but in a few cases, it would incorrectly give a
  fatal error when it saw certain combinations of the old syntax.
  Now, all old job description files are properly recognized by the
  new \Condor{submit}.
  For more information on using the Condor File Transfer mechanism,
  see section~\ref{sec:file-transfer} on
  page~\pageref{sec:file-transfer}, or the \Condor{submit} man
  page on page~\pageref{man-condor-submit}.

\item Fixed a minor problem (introduced in version 6.5.2 while fixing
  another bug) that made it difficult to use a \Condor{master} to
  spawn the Hawkeye daemons.
  Now, if you install Hawkeye on a machine, add \verb@HAWKEYE@ to your
  \MacroNI{DAEMON\_LIST}, and define \MacroNI{HAWKEYE} to be the path to
  the \Prog{hawkeye\_master} program, the \Condor{master} will spawn
  everything correctly.

\item PVM-related bugs:
  \begin{itemize}
    \item Fixed the \Attr{Environment} attribute in the submit description 
		to be honored by both the master process and the slave tasks.
    \item Fixed a few PVM deadlock scenarios during PVM startup.
  \end{itemize}

\item Fixed a bug which caused the ``KFlops'' published in the
startd's ClassAd to have a large variance on fast CPUs.

\item Fixed a bug which could cause inter-daemon communication
problems.  A side effect of this fix, however, causes all daemons to
use 3 more file descriptors.

\item Fixed a misleading error message generated by \Condor{q} when it
  can not find the address of the \Condor{schedd} to query.
  Previously, it suggested a fairly obscure problem as the likely
  source of the error.
  The same error occurs when the \Condor{schedd} is not running, which
  is a far more common situation.

\item Fixed a couple of related of bugs which could cause the
 \Condor{collector} to seg-fault under unusual circumstances.

\item Fixed a minor bug where Condor could get confused if a machine
  had only a fully qualified domain name in DNS, but no simple
  hostname without the domain name.
  In this case, Condor daemons and tools used to exit with a fatal
  error.
  Now, they function properly, since there's no reason for them to
  consider this a problem.

 \item Fixed a minor bug on Windows where the \Condor{starter} would
   incorrectly determine the VM number if StarterLog path contained
   periods.

\end{itemize}

\noindent Known Bugs:

\begin{itemize}

\item The VARS entry within a DAG input file is not propagated
to the rescue DAG.

\end{itemize}


%%%%%%%%%%%%%%%%%%%%%%%%%%%%%%%%%%%%%%%%%%%%%%%%%%%%%%%%%%%%%%%%%%%%%%
\subsection{\label{sec:New-6-5-2}Version 6.5.2}
%%%%%%%%%%%%%%%%%%%%%%%%%%%%%%%%%%%%%%%%%%%%%%%%%%%%%%%%%%%%%%%%%%%%%%

\noindent Known Bugs:
\begin{itemize}

\item There is a serious bug with periodic checkpointing that was
  introduced in version 6.5.2.
  This bug only effects jobs submitted to the Standard universe, since
  those are the only jobs that can perform periodic checkpoints in
  Condor. 
  As soon as the \Macro{PERIODIC\_CHECKPOINT} expression evaluates to
  TRUE on a machine, the \Condor{startd} will get in a loop where it
  continues to request a periodic checkpoint every 5 seconds until the
  job is evicted from the machine.
  The work-around for this bug in version 6.5.2 is to set
  \verb@PERIODIC_CHECKPOINT = False@ in your \File{condor\_config}
  file.

\end{itemize}

\noindent New Features:
\begin{itemize}

\item The collector and negotiator can now run on configurable
ports, instead of relying on hard-coded values.
To use this feature, many places in Condor where you could previously
only provide a hostname now understand ``hostname:port'' notation.
For example, in your config file, you can now use:

\begin{verbatim}
  COLLECTOR_HOST = $(CONDOR_HOST):9650
  NEGOTIATOR_HOST = $(CONDOR_HOST):9651
  FLOCK_TO = your-other.collector.domain.org:7002
\end{verbatim}

If you define \Macro{COLLECTOR\_HOST} in this way, all Condor tools
will automatically use the specified port if you are using them in the
local pool (so you do not need to use any special options to the tools
to get them to find your \Condor{collector} listening on the new port).

In addition, the \Opt{-pool} option to all Condor tools now
understands the ``hostname:port'' notation for remote pools.
To query a remote pool with a collector listening on a non-standard
port, you can use this:

\begin{verbatim}
condor_status -pool your-other.collector.domain.org:7002
\end{verbatim}

\item When the Condor daemons start up, they now log the names of the
Configuration files they are using, right after the startup banner.

\item The \Condor{config\_val} program now has a \Opt{-verbose} option
  which will tell you in which configuration file and line number a
  condor configuration parameter is defined, and a \Opt{-config}
  option which will simply list all of the configuration files in use
  by Condor.

\item There are new attributes to control Condor's file transfer
  mechanism.
  Not only will they hopefully be more clear and easy to use than the
  old ones, they also provide new functionality.
  There is now an option to only transfer files ``IF\_NEEDED''.
  In this case, if the job is matched with a machine in the same
  \Attr{FileSystemDomain}, the files are not transfered, but if the
  job runs at a machine in a different \Attr{FileSystemDomain}, the
  files are transfered automatically.
  For more information on using the Condor File Transfer mechanism,
  see section~\ref{sec:file-transfer} on
  page~\pageref{sec:file-transfer}, or the \Condor{submit} man
  page on page~\pageref{man-condor-submit}.

\item Added support for Condor's new ``Computing on Demand''
  functionality.
  Documentation is not yet available, but will be coming soon.

\item DAGMan no longer requires that all jobs in a DAG have the same
  log file.

\item The value of the \Macro{JOB\_RENICE\_INCREMENT} configuration
      parameter can now be an arbitrary ClassAd expression rather than
      just a static integer.  The expression will be evaluated by the
      \Condor{starter} for each job just before it runs, and can refer
      to any attribute in the job ClassAd.

\item The \Condor{collector} now publishes statistics about the running
  jobs.  In particular, it now publishes the number of jobs running in
  each Universe, both as a snapshot, and the maximum of the snapshots
  for each month.

\item The UNIX \Condor{collector} has the ability to fork off child
  processes to handle queries.  The \Macro{COLLECTOR\_QUERY\_WORKERS}
  parameter is used to specify the maximum number of these worker
  processes and defaults to zero.

\item All daemons now publish ``sequence number'' and ``start time''
information in their ClassAds.

\item The Collector now maintains and publishes update statistics
using the above ClassAd ``sequence number'' and ``start time''
information.  History information is stored for the past
COLLECTOR\_CLASS\_HISTORY\_SIZE updates and is also published in the
Collector's ClassAd as a hex string.

\item The \Condor{schedd} can now use TCP connections to send updates
  to pools that it is configured to flock to.
  You can now define \Macro{TCP\_UPDATE\_COLLECTORS} list and any
  collectors listed there, including ones the \Condor{schedd} is
  flocking with, will be updated with TCP.
  Also, the \Condor{master} uses the same list to decide if it should
  use TCP to update any collectors listed in the
  \Macro{SECONDARY\_COLLECTORS\_LIST}.
  For more infomation on TCP collector updates in Condor and if your
  site would want to enable them, read
  section~\ref{sec:tcp-collector-update} on ``Using TCP to Send 
  Collector Updates'' on page~\pageref{sec:tcp-collector-update}.

\item You no longer need to define \Macro{FLOCK\_VIEW\_SERVERS} in
  your config file if you have configured a \Condor{schedd} to flock
  to other pools.
  This is now handled automatically, so you only have to define
  \Macro{FLOCK\_TO}.

\item If no \Macro{STARTD\_EXPRS} is specified, the \Condor{startd}
  now defaults to ``JobUniverse''.

\item Globus universe jobs under Condor-G now send email on job
completion based on the notification setting in the submit file.

\item When submitting a Globus universe job to Condor-G the
input, output, and error files can now optionally be an http,
https, ftp, or gsiftp URL.  (The actual file transfer is handled
by globus-url-copy on the remote site.)

\end{itemize}

\noindent Bugs Fixed:
\begin{itemize}

\item DAGMan now correctly reports an error and rejects DAGs which
      contain two nodes with the same name, regardless of their case.
      (DAGMan has rejected duplicate node names since Condor 6.4.6,
      but until now it would fail to do so if there was any difference
      in their case.)

\item When \Condor{submit\_dag} checks job submit files for proper
      ``log'' statements, it now correctly recognizes lines with
      leading whitespace.

\item Fixed a minor bug whereby DAGMan was not removing its lock file
      after successful completion.

\item Fixed a bug introduced in version 6.5.0 whereby the
      \Macro{UID\_DOMAIN} attributes of jobs and resources were being
      compared in a case-sensitive manner, resulting in erroneous
      failures.

\item The \Condor{master} used to always pass a \Opt{-f} on the
  command line to all daemons defined in the \Macro{DAEMON\_LIST}
  config file setting.
  However, if you include entries which are not Condor daemons, the
  \Condor{master} will no longer add a \Opt{-f}.

\item The files specified in \Opt{transfer\_input\_files} can now
  contain spaces in the filenames.

\end{itemize}


%%%%%%%%%%%%%%%%%%%%%%%%%%%%%%%%%%%%%%%%%%%%%%%%%%%%%%%%%%%%%%%%%%%%%%
\subsection{\label{sec:New-6-5-1}Version 6.5.1}
%%%%%%%%%%%%%%%%%%%%%%%%%%%%%%%%%%%%%%%%%%%%%%%%%%%%%%%%%%%%%%%%%%%%%%

\noindent New Features:
\begin{itemize}

\item DAGMan now supports both Condor computational jobs and Stork data
placement (DaP) jobs.  (See \URL{http://www.cs.wisc.edu/condor/stork/}
for more info on Stork.)

\item Starter exceptions, such as failure to open standard input, are
now recorded in the user log.

\item Added new \Opt{-forcex} argument to \Condor{rm} to force the
immediate local removal of (typically Globus universe) jobs in the 'X'
state, regardless of their remote state.

\end{itemize}

\noindent Bugs Fixed:
\begin{itemize}

\item When transfer\_files is being used, the path to the stdout/stderr
files was not being respected.  After hese files have been transferred,
they are now copied to the location specified in the submit file.

\item A DAGMan bug introduced in Condor 6.5.0 has been fixed, where
DAGMan could crash (with a failed assertion) when recovering from a
rescue DAG.

\item Fixed a bug in the example condor\_config.generic and
hawkeye\_config files, where COLLECTOR\_HOST was being included in the
default STARTD\_EXPRS in non-string form, resulting in an invalid value
for that attribute in machine classads.

\end{itemize}

\noindent Known Bugs:
\begin{itemize}

\item DAGMan doesn't detect when users mistakenly specify two
DAG nodes with the same node name; instead it waits for the
same node to complete twice, which never happens, and so DAGMan
goes off into never-never land.

\end{itemize}

%%%%%%%%%%%%%%%%%%%%%%%%%%%%%%%%%%%%%%%%%%%%%%%%%%%%%%%%%%%%%%%%%%%%%%
\subsection{\label{sec:New-6-5-0}Version 6.5.0}
%%%%%%%%%%%%%%%%%%%%%%%%%%%%%%%%%%%%%%%%%%%%%%%%%%%%%%%%%%%%%%%%%%%%%%
\noindent New Features:
\begin{itemize}

\item A fresh value of RemoteWallClock is now used when evaluating
user policy expressions, such as periodic\_remove.

\item The IOProxy handler now handles escaped characters (whitespace)
in filenames.

\item condor.boot is now configured to work automatically with RedHat
chkconfig.

\item A new log\_xml option has been added to condor\_submit. It is
documented in the condor\_submit portion of the manual.

\item A new DAGMan option to produce dot files was added. Dot is a
program that creates visualizations of DAGs. This feature is
documented in Section~\ref{sec:DAGMan}.

\item The email report from condor\_preen is now less cryptic, and
more self-explanatory.

\item Specifying full device paths (e.g., ``/dev/mouse'') instead of bare
device names (e.g., ``mouse'') in CONSOLE\_DEVICES in the config file is no
longer an error.

\item The condor\_submit tool now prints a more helpful, specific error if
the specified job executable is not found, or can't be accessed.

\item The startd ``cron'' (Hawkeye) now permits zero length ``prefix''
strings.

\item A number of new Hawkeye modules have been added, and most have
various bug fixes and improvements.

\item Added support for a new config parameter, Q\_QUERY\_TIMEOUT, which
defines the timeout that \Condor{q} uses when communicating with the
\Condor{schedd}.

\item Added the ability to use TCP to send ClassAd updates to the
Condor{collector}, though the feature is disabled by default.
Read section~\ref{sec:tcp-collector-update} on ``Using TCP to Send
Collector Updates'' on page~\pageref{sec:tcp-collector-update} for
more details and a discussion of when a site would need to enable this 
functionality.

\item Scheduler Universe has been ported to Windows. This enables
\Condor{dagman} to run on Windows as well.

\end{itemize}

\noindent Bugs Fixed:
\begin{itemize}

\item Hawkeye will no longer busy-loop if a ``continuous mode''
module with a period of 0 fails to execute for some reason.  (Now, for
continuous mode modules, a period of 0 is automatically reset to be 1, and
a warning appears in the log.)

\item Fixed a very rare potential bug when initializing a user log
file.
Improved the error messages generated when there are problems
initializing the user log to include string descriptions of the
errors, not just the error number (errno).

\item The default value in the config files for \Macro{LOCK} is now
defined in terms of \Macro{LOG}, instead of using \Macro{LOCAL\_DIR}
and appending ``log''.
This is a very minor correction, but in the rare cases where the log
directory is being redefined for some reason, we usually want that to
apply to the lock files, as well.
Of course, if the log directory is on NFS, \Macro{LOCK} should still
be customized to point to a directoy on a local file system.

\item Fixed a bug in the \Condor{schedd} where Scheduler Universe jobs
  with \Macro{copy\_to\_spool} = false would fail.

\end{itemize}

\noindent Known Bugs:
\begin{itemize}

\item DAGMan doesn't detect when users mistakenly specify two
DAG nodes with the same node name; instead it waits for the
same node to complete twice, which never happens, and so DAGMan
goes off into never-never land.

\end{itemize}
