%%%%%%%%%%%%%%%%%%%%%%%%%%%%%%%%%%%%%%%%%%%%%%%%%%%%%%%%%%%%%%%%%%%%%%
\section{\label{sec:History-Intro}Introduction to HTCondor Versions}
%%%%%%%%%%%%%%%%%%%%%%%%%%%%%%%%%%%%%%%%%%%%%%%%%%%%%%%%%%%%%%%%%%%%%%

This chapter provides descriptions of what features have been added or
bugs fixed for each version of HTCondor.
The first section describes the HTCondor version numbering scheme, what
the numbers mean, and what the different \Term{release series} are.
The rest of the sections each describe a specific release series, and
all the HTCondor versions found in that series.

%%%%%%%%%%%%%%%%%%%%%%%%%%%%%%%%%%%%%%%%%%%%%%%%%%%%%%%%%%%%%%%%%%%%%%
\subsection{\label{sec:Version-Number-Scheme}
HTCondor Version Number Scheme}
%%%%%%%%%%%%%%%%%%%%%%%%%%%%%%%%%%%%%%%%%%%%%%%%%%%%%%%%%%%%%%%%%%%%%%

Starting with version 6.0.1, HTCondor adopted a new, hopefully easy to
understand version numbering scheme.
It reflects the fact that HTCondor is both a production system and a
research project.
The numbering scheme was primarily taken from the Linux kernel's
version numbering, so if you are familiar with that, it should seem
quite natural.

There will usually be two HTCondor versions available at any given time,
the \Term{stable} version, and the \Term{development} version.
Gone are the days of ``patch level 3'', ``beta2'', or any other random
words in the version string.
All versions of HTCondor now have exactly three numbers, separated by
``.''   

\begin{itemize}

\item The first number represents the major version number, and will
change very infrequently.

\item \emph{The thing that determines whether a version of HTCondor is
\Term{stable} or \Term{development} is the second digit.
Even numbers represent stable versions, while odd numbers represent
development versions.}

\item The final digit represents the minor version number, which
defines a particular version in a given release series.

\end{itemize}


%%%%%%%%%%%%%%%%%%%%%%%%%%%%%%%%%%%%%%%%%%%%%%%%%%%%%%%%%%%%%%%%%%%%%%
\subsection{\label{sec:Stable-Series}The Stable Release Series}
%%%%%%%%%%%%%%%%%%%%%%%%%%%%%%%%%%%%%%%%%%%%%%%%%%%%%%%%%%%%%%%%%%%%%%

People expecting the stable, production HTCondor system should download
the stable version, denoted with an even number in the second digit of
the version string.
Most people are encouraged to use this version.  
We will only offer our paid support for versions of HTCondor from the
stable release series.

\emph{On the stable series, new minor version releases will only
be made for bug fixes and to support new platforms.}
No new features will be added to the stable series.
People are encouraged to install new stable versions of HTCondor when
they appear, since they probably fix bugs you care about.
Hopefully, there will not be many minor version releases for any given
stable series.


%%%%%%%%%%%%%%%%%%%%%%%%%%%%%%%%%%%%%%%%%%%%%%%%%%%%%%%%%%%%%%%%%%%%%%
\subsection{\label{sec:Developement-Series}
The Development Release Series}
%%%%%%%%%%%%%%%%%%%%%%%%%%%%%%%%%%%%%%%%%%%%%%%%%%%%%%%%%%%%%%%%%%%%%%

Only people who are interested in the latest research, new features
that haven't been fully tested, etc, should download the development
version, denoted with an odd number in the second digit of the version
string.  
We will make a best effort to ensure that the development series will
work, but we make no guarantees.

On the development series, new minor version releases will probably
happen frequently.
People should not feel compelled to install new minor versions unless
they know they want features or bug fixes from the newer development
version.

\emph{Most sites will probably never want to install a development
version of HTCondor for any reason.}
Only if you know what you are doing (and like pain), or were
explicitly instructed to do so by someone on the HTCondor Team, should
you install a development version at your site.

After the feature set of the development series is satisfactory to the
HTCondor Team, we will put a code freeze in place, and from that point
forward, only bug fixes will be made to that development series.
When we have fully tested this version, we will release a new stable
series, resetting the minor version number, and start work on a new
development release from there.

%%%%%%%%%%%%%%%%%%%%%%%%%%%%%%%%%%%%%%%%%%%%%%%%%%%%%%%%%%%%%%%%%%%%%%
% The rest of this file just inputs other files which contain sections
% describing each release series in detail.
%%%%%%%%%%%%%%%%%%%%%%%%%%%%%%%%%%%%%%%%%%%%%%%%%%%%%%%%%%%%%%%%%%%%%%

% upgrade instructions are in the Pool Management section
%%%%%%%%%%%%%%%%%%%%%%%%%%%%%%%%%%%%%%%%%%%%%%%%%%%%%%%%%%%%%%%%%%%%%%
\section{\label{sec:gotchas}Upgrading from the 7.6 series to the 7.8 series of Condor}
%%%%%%%%%%%%%%%%%%%%%%%%%%%%%%%%%%%%%%%%%%%%%%%%%%%%%%%%%%%%%%%%%%%%%%

\index{upgrading!items to be aware of}
While upgrading from the 7.6 series of Condor to the 7.8 series will
bring many new features and improvements introduced in the 7.7 series of
Condor, it will also introduce changes that administrators of sites
running from an older Condor version should be aware of when
planning an upgrade.
Here is a list of items that administrators should be aware of.

\begin{itemize}

\item  Placeholder.  First item goes here.

\end{itemize}


%%%      PLEASE RUN A SPELL CHECKER BEFORE COMMITTING YOUR CHANGES!
%%%      PLEASE RUN A SPELL CHECKER BEFORE COMMITTING YOUR CHANGES!
%%%      PLEASE RUN A SPELL CHECKER BEFORE COMMITTING YOUR CHANGES!
%%%      PLEASE RUN A SPELL CHECKER BEFORE COMMITTING YOUR CHANGES!
%%%      PLEASE RUN A SPELL CHECKER BEFORE COMMITTING YOUR CHANGES!

%%%%%%%%%%%%%%%%%%%%%%%%%%%%%%%%%%%%%%%%%%%%%%%%%%%%%%%%%%%%%%%%%%%%%%
\section{\label{sec:History-7-9}Development Release Series 7.9}
%%%%%%%%%%%%%%%%%%%%%%%%%%%%%%%%%%%%%%%%%%%%%%%%%%%%%%%%%%%%%%%%%%%%%%

This is the development release series of Condor.
The details of each version are described below.

%%%%%%%%%%%%%%%%%%%%%%%%%%%%%%%%%%%%%%%%%%%%%%%%%%%%%%%%%%%%%%%%%%%%%%
\subsection*{\label{sec:New-7-9-2}Version 7.9.2}
%%%%%%%%%%%%%%%%%%%%%%%%%%%%%%%%%%%%%%%%%%%%%%%%%%%%%%%%%%%%%%%%%%%%%%

\noindent Release Notes:

\begin{itemize}

\item Condor version 7.9.2 not yet released.
%\item Condor version 7.9.2 released on Month Date, 2012.

\end{itemize}


\noindent New Features:

\begin{itemize}

\item None.

\end{itemize}

\noindent Configuration Variable and ClassAd Attribute Additions and Changes:

\begin{itemize}

\item None.

\end{itemize}

\noindent Bugs Fixed:

\begin{itemize}

\item None.

\end{itemize}

\noindent Known Bugs:

\begin{itemize}

\item None.

\end{itemize}

\noindent Additions and Changes to the Manual:

\begin{itemize}

\item None.

\end{itemize}


%%%%%%%%%%%%%%%%%%%%%%%%%%%%%%%%%%%%%%%%%%%%%%%%%%%%%%%%%%%%%%%%%%%%%%
\subsection*{\label{sec:New-7-9-1}Version 7.9.1}
%%%%%%%%%%%%%%%%%%%%%%%%%%%%%%%%%%%%%%%%%%%%%%%%%%%%%%%%%%%%%%%%%%%%%%

\noindent Release Notes:

\begin{itemize}

\item Condor version 7.9.1 not yet released.
%\item Condor version 7.9.1 released on Month Date, 2012.

\item Condor no longer looks for its main configuration file in the
location \File{\MacroUNI{GLOBUS\_LOCATION}/etc/condor\_config}.
\Ticket{2830}

\end{itemize}


\noindent New Features:

\begin{itemize}

\item \Condor{job\_router} can now submit the routed copy of jobs to a
different \Condor{schedd} than the one that serves as the source of
jobs to be routed.  The spool directories of the two
\Condor{schedds} must still be directly accessible to
\Condor{job\_router}.  This feature is enabled by using the new
optional configuration settings:

\begin{itemize}
\item \Macro{JOB\_ROUTER\_SCHEDD1\_SPOOL}
See definition at section~\ref{param:JobRouterSchedd1Spool}.
\item \Macro{JOB\_ROUTER\_SCHEDD2\_SPOOL}
See definition at section~\ref{param:JobRouterSchedd2Spool}.
\item \Macro{JOB\_ROUTER\_SCHEDD1\_NAME}
See definition at section~\ref{param:JobRouterSchedd1Name}.
\item \Macro{JOB\_ROUTER\_SCHEDD2\_NAME}
See definition at section~\ref{param:JobRouterSchedd2Name}.
\item \Macro{JOB\_ROUTER\_SCHEDD1\_POOL}
See definition at section~\ref{param:JobRouterSchedd1Pool}.
\item \Macro{JOB\_ROUTER\_SCHEDD2\_POOL}
See definition at section~\ref{param:JobRouterSchedd2Pool}.
\end{itemize}
\Ticket{3030}

\item The \Condor{job\_router} can now optionally transform jobs in place,
rather than creating a second transformed version (copy) of the job.
\Ticket{3185}

\item The \Condor{defrag} daemon now has a policy option implemented
by configuration to cancel the draining
of a machine that is in the Draining mode.  This can be used to effect
partial draining of machines.
\Ticket{2993}

\item Communication between the \Condor{c-gahp} and \Condor{schedd} has
been improved. A large number of Condor-C jobs should no longer cause
other clients of the remote \Condor{schedd} to time out trying to get the
\Condor{schedd}'s attention.
\Ticket{2575}

\end{itemize}

\noindent Configuration Variable and ClassAd Attribute Additions and Changes:

\begin{itemize}

\item Dynamic slots now fill the values for attributes of the form TotalSlotXxx, 
for configured local resources, in a way consistent with standard resources
such as TotalSlotCpus.  Previously those values were all filled with zero on
dynamic slots.
\Ticket{3229}

\item The \Condor{schedd} now advertises the value of configuration variable
\MacroNI{COLLECTOR\_HOST} as attribute \Attr{CollectorHost} in 
its daemon ClassAd.  This allows one to determine if a given
\Condor{schedd} reporting to a \Condor{collector} is flocking to that 
\Condor{collector} or not.
\Ticket{3202}

\item Added the attribute \Attr{DAGManNodesMask} to control the verboseness of
the log referred to by \Attr{DAGManNodesLog}.
\Ticket{2807}

\item The new configuration variable
\Macro{QUEUE\_SUPER\_USER\_MAY\_IMPERSONATE} specifies a regular
expression that matches the user names that
the queue super user may impersonate when managing jobs.  When not
set, the default behavior is to allow impersonation of any user who
has had a job in the queue during the life of the \Condor{schedd}.  For
proper functioning of the \Condor{shadow}, the \Condor{gridmanager}, and
the \Condor{job\_router}, this expression, if set, must match the owner
names of all jobs that these daemons will manage.
\Ticket{3030}


\item The new configuration variable \Macro{DEFRAG\_CANCEL\_REQUIREMENTS}
is an expression that specifies which draining machines should have 
draining be canceled.  
This defaults to \MacroUNI{DEFRAG\_WHOLE\_MACHINE\_EXPR}.  
This could be used to drain partial rather than whole machines.
\Ticket{2993}

\item The new submit command \SubmitCmd{use\_x509userproxy} can be set
to \Expr{True} to indicate that an X.509 user proxy is required for the job. 
If \SubmitCmd{x509userproxy} is not set, 
then the proxy file will be looked for in the standard locations.
\Ticket{3025}

\end{itemize}

\noindent Bugs Fixed:

\begin{itemize}

\item Fixed a bug in all daemons wherein the \Attr{DaemonStartTime} attribute in
the ad for all daemons would be reset to the current time when they are
reconfigured.
\Ticket{3235}

\item \Security Although not user-visible, there were multiple updates to remove places
in the code where potential buffer overruns could occur, thus removing
potential attacks.  None were known to be exploitable.

\item \Security Although not user-visible, there were updates to the code to improve
error checking of system calls, removing some potential security threats.  None
were known to be exploitable.

\item \Security Although not user-visible, some code that was no longer used was removed.
The presence of this code could have lead to a Denial-of-Service attack which
would allow an attacker to stop another user's jobs from running.

\item \Security Filesystem (FS) authentication was improved to check the UNIX permissions
of the directory used for authentication.  Without this, an attacker may have
been able to impersonate another submitter on the same submit machine.

\item The \Condor{negotiator} now checks the accountant log file for sanity
once only on start up,  
thereby increasing efficiency of iteration through 
the accountant ClassAd log structure.
\Ticket{3011}

\item The ClassAd functions \Procedure{splitUserName} and 
\Procedure{splitSlotName}
no longer leak a small amount of memory each time they are evaluated.  
This bug was introduced when these functions were added in Condor version 7.7.6.
\Ticket{3082}

\item There are several bug fixes for grid-type batch jobs:
  \begin{itemize}
  \item Monitoring the status of jobs submitted to PBS and SGE has been
    improved. \Ticket{3067} \Ticket{3157} \Ticket{3181}
  \item Job command-line arguments containing 
    left parenthesis, \verb@(@, right parenthesis, \verb@)@, 
    and ampersand, \verb@&@, characters are now handled properly. 
    \Ticket{3057}
  \item Removing PBS jobs that have just completed no longer causes the jobs
    to become held. \Ticket{3016}
  \item Added a work-around for a bug when submitting jobs to
    a Condor pool running Condor versions 7.7.6 through 7.8.2.
    A bug in \Condor{history} \Opt{-f} caused an error in determining
    a job's status.
    \Ticket{3133}
  \item Improved the handling of job files when the batch system has a shared
    file system. \Ticket{3195}
  \end{itemize}

\item Changes introduced in Condor version 7.9.0 caused jobs submitted by
\Condor{dagman} in the local universe to not write to the default node log file,
when \Macro{DAGMAN\_ALWAYS\_USE\_NODE\_LOG} was \Expr{True} (the default),
and a user log was also defined. This is fixed. 
\Ticket{3111}

\item Fixed a bug introduced in Condor version 7.9.0 that caused grid type
cream jobs to be held with a hold reason of 
\footnotesize
\begin{verbatim}
  CREAM_Delegate Error: Cannot set credentials in the gsoap-plugin context.
\end{verbatim}
\normalsize
\Ticket{3234}

\end{itemize}

\noindent Known Bugs:

\begin{itemize}

\item None.

\end{itemize}

\noindent Additions and Changes to the Manual:

\begin{itemize}

\item None.

\end{itemize}


%%%%%%%%%%%%%%%%%%%%%%%%%%%%%%%%%%%%%%%%%%%%%%%%%%%%%%%%%%%%%%%%%%%%%%
\subsection*{\label{sec:New-7-9-0}Version 7.9.0}
%%%%%%%%%%%%%%%%%%%%%%%%%%%%%%%%%%%%%%%%%%%%%%%%%%%%%%%%%%%%%%%%%%%%%%

\noindent Release Notes:

\begin{itemize}

\item Condor version 7.9.0 released on August 16, 2012.

\end{itemize}


\noindent New Features:

\begin{itemize}

\item Machine slots can now be configured to identify and
divide customized local resources.
Jobs may then request these resources.
See section~\ref{sec:Configuring-SMP} for details.
\Ticket{2905}

\item Condor now supports and implements the caching of ClassAds 
to reduce memory footprints. 
This feature is experimental and is currently disabled by default.
It can be enabled by setting
the new configuration variable \Macro{ENABLE\_CLASSAD\_CACHING}
to \Expr{True}.
\Ticket{2541}
\Ticket{3127}

\item \Condor{status} now returns the \Condor{schedd} ClassAd directly 
from the \Condor{schedd} daemon,
if both options \Opt{-direct} and \Opt{-schedd} are given on the command line.
\Ticket{2492}

\item The new \Opt{-status} and \Opt{-echo} command line options to 
\Condor{wait} command cause it to show job start and terminate information,
and to print events to \Code{stdout}.
\Ticket{2926}

\item Added a \Expr{DEBUG} logging level output flag \Dflag{CATEGORY},
which causes Condor to include the logging level
flags in effect for each line of logged output.
\Ticket{2712}

\item \Condor{status} and \Condor{q} each have a new \Opt{-autoformat} option
to make some output format specifications easier than the existing
\Opt{-format} option.
See the \Condor{status} manual page located on page~\pageref{man-condor-status}
and the \Condor{q} manual page located on page~\pageref{man-condor-q} 
for details.
\Ticket{2941}

\item Enhanced the ClassAd log system to report the log line number 
on parse failures, 
and improved the ability to detect parse failures closer to 
the point of corruption.
\Ticket{2934}

\item Added an \Opt{-evaluate} option to \Condor{config\_val}, which causes the configured value queried from
a given daemon to be evaluated with respect to that daemon's ClassAd.
\Ticket{856}

\item Added code to \Condor{dagman},
such that a \Expr{VARS} assignment in a top-level DAG is applied to splices.
\Ticket{1780}

\item Condor now uses libraries from Globus 5.2.1.
\Ticket{2838}

\item When authenticating Condor daemons with GSI and
configuration variable \MacroNI{GSI\_DAEMON\_NAME} is undefined, 
Condor checks that the server name in the certificate matches the 
host name that the client is connecting to. 
When \MacroNI{GSI\_DAEMON\_NAME} is defined,
the old behavior is preserved: only certificates matching
\MacroNI{GSI\_DAEMON\_NAME} pass the authentication step, 
and no host name check is performed.  
The behavior of the host name check
may be further controlled with the new configuration variables
\MacroNI{GSI\_SKIP\_HOST\_CHECK} and
\MacroNI{GSI\_SKIP\_HOST\_CHECK\_CERT\_REGEX}.
\Ticket{1605}

\item Added new capability to \Condor{submit} to allow recursive macros in
submit description files. 
This facility allows one to update variables recursively. 
Before this new capability was added,
recursive definition would send \Condor{submit} into an
infinite loop of expanding the macro,
such that the expansion would fill up memory.
See section~\ref{macro-in-submit-description-file} for details.
\Ticket{406}

\item A DAGMan limitation and restriction has been removed.  
It is now permitted to define a \SubmitCmd{log} command using a macro,
within a node job's submit description file.
\Ticket{2428}

\item To enforce the dependencies of a DAG,
DAGMan now uses and watches only the default node
user log of the \Condor{dagman} job for events.  
DAGMan requests the \Condor{schedd} and \Condor{shadow} daemons to write each
event to this default log, 
in addition to writing to a log specified by the node job.
\Condor{dagman} writes POST script terminate events only to its default log;
these terminate events are not written to the user log.
This behavior can be turned off by setting the configuration variable
\Macro{DAGMAN\_ALWAYS\_USE\_NODE\_LOG} to \Expr{False}.
For correct behavior,
\MacroNI{DAGMAN\_ALWAYS\_USE\_NODE\_LOG} should be set to \Expr{False}
if \Condor{dagman} version 7.9.0 or later is submitting jobs 
to an older version of
a \Condor{schedd} daemon or of a \Condor{submit} executable.
\Ticket{2807}

\item \Condor{submit} has a new \Opt{-interactive} option for
platforms other than Windows,
which schedules and runs a job that provides a shell prompt
on the execute machine.
Documentation of this feature is not yet available.
\Ticket{3088}

\end{itemize}

\noindent Configuration Variable and ClassAd Attribute Additions and Changes:

\begin{itemize}

\item The new configuration variables \Macro{MACHINE\_RESOURCE\_NAMES}
(see section~\ref{param:MachineResourceNames})
and \Macro{MACHINE\_RESOURCE\_<name>}
(see section~\ref{param:MachineResourceResourcename})
identify and specify the use of customized local machine resources.
\Ticket{2905}

\item The new configuration variable \MacroNI{ENABLE\_CLASSAD\_CACHING}
controls whether the new caching feature of ClassAds is used.
The default value is \Expr{False}.
\Ticket{3127}

\item The new configuration variable \Macro{CLASSAD\_LOG\_STRICT\_PARSING}
controls whether ClassAd log files such as the job queue
log are read with strict parse checking on ClassAd expressions.
\Ticket{3069}

\item The default value for configuration variable \Macro{USE\_PROCD}
is now \Expr{True} for the \Condor{master} daemon.  
This means that by
default the \Condor{master} will start a \Condor{procd} daemon to be used 
by it and all other daemons on that machine.
\Ticket{2911}

\item There is a new configuration variable used by the \Condor{starter}.
If \Macro{STARTER\_RLIMIT\_AS} is set to an integer value, 
the \Condor{starter}
will use the \Procedure{setrlimit} system call with the 
\Code{RLIMIT\_AS} parameter to
limit the virtual memory size of each process in the user job.  
The value of this configuration variable is a ClassAd expression, 
evaluated in the context of both the machine and job ClassAds, 
where the machine ClassAd is the \Expr{my} ClassAd, 
and the job ClassAd is the \Expr{target} ClassAd.
\Ticket{1663}

\item New configuration variables were added to to the \Condor{schedd} to
define statistics that count subsets of jobs. 
These variables have the form \Macro{SCHEDD\_COLLECT\_STATS\_BY\_<Name>},
and should be defined by a ClassAd expression that evaluates to a string.
See section~\ref{param:ScheddCollectStatsByName}
for the complete definition.
The optional configuration variable of the form
\Macro{SCHEDD\_EXPIRE\_STATS\_BY\_<Name>} can be used to set an expiration time,
in seconds, for each set of statistics.
\Ticket{2862}

\item The new \SubmitCmd{batch\_queue} submit description file command
and new job ClassAd attribute \Attr{BatchQueue} specify which job
queue to use for grid universe jobs of type
\SubmitCmd{pbs}, \SubmitCmd{lsf}, and \SubmitCmd{sge}.
\Ticket{2996}

\item The new configuration variable \Macro{GSI\_SKIP\_HOST\_CHECK} is
a boolean that controls whether a check is performed during
GSI authentication of a Condor daemon.  
When the default value \Expr{False},
the check is not skipped, so the daemon host name must match the
host name in the daemon's certificate, unless otherwise exempted
by values of \MacroNI{GSI\_DAEMON\_NAME} or
\MacroNI{GSI\_SKIP\_HOST\_CHECK\_CERT\_REGEX}.
When \Expr{True}, this check is skipped, and hosts will not be rejected
due to a mismatch of certificate and host name.
\Ticket{1605}

\item The new configuration variable
\MacroNI{GSI\_SKIP\_HOST\_CHECK\_CERT\_REGEX} may be set to a
regular expression.  GSI certificates of Condor daemons with a
subject name that are matched in full by this regular expression
are not required to have a matching daemon host name and certificate
host name.  The default is an empty regular expression, which will
not match any certificates, even if they have an empty subject name.
\Ticket{1605}

\end{itemize}

\noindent Bugs Fixed:

\begin{itemize}

\item Fixed a bug in which usage of cgroups incorrectly included the page cache 
in the maximum memory usage.
This bug fix is also included in Condor version 7.8.2.
\Ticket{3003}

\item The EC2 GAHP will now respect the value of the environment variable
\Env{X509\_CERT\_DIR} and the configuration variable
\Macro{GSI\_DAEMON\_TRUSTED\_CA\_DIR} for \emph{all} secure connections.
\Ticket{2823}

\item Condor will avoid selecting down (disabled) network interfaces.  Previously Condor could select a down interface over an up (active) interface.
\Ticket{2893}

\item Made logic in the \Condor{negotiator} that computes submitter limits 
properly aware of the configuration variable
\Macro{NEGOTIATOR\_CONSIDER\_PREEMPTION}.
\Ticket{2952}


\item Condor no longer back-dates file modification times by 3 minutes
when transferring job input files into the job spool directory or the job
execute directory.
\Ticket{2423}

\item Fixed a bug in which the use of a pipe in the configuration file 
on Windows platforms would cause a visible console window 
to show up whenever the configuration was read.
\Ticket{1534}

\end{itemize}

\noindent Known Bugs:

\begin{itemize}

\item None.

\end{itemize}

\noindent Additions and Changes to the Manual:

\begin{itemize}

\item Machine ClassAd attribute string values relating to \Attr{OpSys} have
been documented for Scientific Linux platforms.
\Ticket{2366}

\end{itemize}



%%%      PLEASE RUN A SPELL CHECKER BEFORE COMMITTING YOUR CHANGES!
%%%      PLEASE RUN A SPELL CHECKER BEFORE COMMITTING YOUR CHANGES!
%%%      PLEASE RUN A SPELL CHECKER BEFORE COMMITTING YOUR CHANGES!
%%%      PLEASE RUN A SPELL CHECKER BEFORE COMMITTING YOUR CHANGES!
%%%      PLEASE RUN A SPELL CHECKER BEFORE COMMITTING YOUR CHANGES!

%%%%%%%%%%%%%%%%%%%%%%%%%%%%%%%%%%%%%%%%%%%%%%%%%%%%%%%%%%%%%%%%%%%%%%
\section{\label{sec:History-7-8}Stable Release Series 7.8}
%%%%%%%%%%%%%%%%%%%%%%%%%%%%%%%%%%%%%%%%%%%%%%%%%%%%%%%%%%%%%%%%%%%%%%

This is a stable release series of Condor.
As usual, only bug fixes (and potentially, ports to new platforms)
will be provided in future 7.8.x releases.
New features will be added in the 7.9.x development series.

The details of each version are described below.

%%%%%%%%%%%%%%%%%%%%%%%%%%%%%%%%%%%%%%%%%%%%%%%%%%%%%%%%%%%%%%%%%%%%%%
\subsection*{\label{sec:New-7-8-6}Version 7.8.6}
%%%%%%%%%%%%%%%%%%%%%%%%%%%%%%%%%%%%%%%%%%%%%%%%%%%%%%%%%%%%%%%%%%%%%%

\noindent Release Notes:

\begin{itemize}

\item Condor version 7.8.6 not yet released.
%\item Condor version 7.8.6 released on Month Date, 2012.

\end{itemize}


\noindent New Features:

\begin{itemize}

\item None.

\end{itemize}

\noindent Configuration Variable and ClassAd Attribute Additions and Changes:

\begin{itemize}

\item None.

\end{itemize}

\noindent Bugs Fixed:

\begin{itemize}

	\item Fixed a problem where a ssh\_to\_job or interactive job session would be terminated prematurely if the execute machine was configured to track process trees via a dedicated login (aka \Macro{DEDICATED\_EXECUTE\_ACCOUNT\_REGEXP} is being used).  \Ticket{3232}

\item ClassAd functions int() and real() now ignore trailing characters
when a string argument containing a valid number is given.
\Ticket{3102}

\end{itemize}

\noindent Known Bugs:

\begin{itemize}

\item None.

\end{itemize}

\noindent Additions and Changes to the Manual:

\begin{itemize}

\item None.

\end{itemize}

%%%%%%%%%%%%%%%%%%%%%%%%%%%%%%%%%%%%%%%%%%%%%%%%%%%%%%%%%%%%%%%%%%%%%%
\subsection*{\label{sec:New-7-8-5}Version 7.8.5}
%%%%%%%%%%%%%%%%%%%%%%%%%%%%%%%%%%%%%%%%%%%%%%%%%%%%%%%%%%%%%%%%%%%%%%

\noindent Release Notes:

\begin{itemize}

\item Condor version 7.8.5 not yet released.
%\item Condor version 7.8.5 released on Month Date, 2012.

\end{itemize}


\noindent New Features:

\begin{itemize}

\item Condor now contains a tool, accountant_log_fixer, that can fix the
damage to the Accountantnew.log caused by the version 7.8.4 \Condor{negotiator}.
\Ticket{3221}

\end{itemize}

\noindent Configuration Variable and ClassAd Attribute Additions and Changes:

\begin{itemize}

\item None.

\end{itemize}

\noindent Bugs Fixed:

\begin{itemize}

\item The \Condor{schedd} daemon would mark scheduler universe jobs
as completed and remove them from the job queue,
even when they should have been requeued, according to policy
\Ticket{3207}

\item The \Condor{negotiator} version 7.8.4 would write corrupt resource entries
to the Accountantnew.log that it would not be able to read back.
The \Condor{negotiator} version 7.9.0 will abort when trying to read
corrupted resource entries. The \Condor{negotiator} will now correct corrupt.
resource entries created by 7.8.4 over time.
\Ticket{3221}

\item The \Condor{schedd} statistics \Attr{JobsAccumExecuteTime}
and \Attr{JobsAccumPostExecuteTime} were sometimes much too large for jobs
that had been vacated and then restarted.
\Ticket{3227}

\end{itemize}

\noindent Known Bugs:

\begin{itemize}

\item None.

\end{itemize}

\noindent Additions and Changes to the Manual:

\begin{itemize}

\item None.

\end{itemize}


%%%%%%%%%%%%%%%%%%%%%%%%%%%%%%%%%%%%%%%%%%%%%%%%%%%%%%%%%%%%%%%%%%%%%%
\subsection*{\label{sec:New-7-8-4}Version 7.8.4}
%%%%%%%%%%%%%%%%%%%%%%%%%%%%%%%%%%%%%%%%%%%%%%%%%%%%%%%%%%%%%%%%%%%%%%

\noindent Release Notes:

\begin{itemize}

\item Condor version 7.8.4 released on September 19, 2012.

\item This release contains several important security fixes and all users should upgrade as soon as possible.

\end{itemize}


\noindent New Features:

\begin{itemize}

\item None.

\end{itemize}

\noindent Configuration Variable and ClassAd Attribute Additions and Changes:

\begin{itemize}

\item The new configuration variable \Macro{GSI\_AUTHZ\_CONF}
fixes a bug in which an instance of Condor may utilize the  
wrong Globus mapping.
The configuration variable defines a path and file name 
to the file that contains the Globus mapping library. 
See the complete definition at
~\ref{param:GSIAuthzConf}.
\Ticket{2103}


\end{itemize}

\noindent Bugs Fixed:

\begin{itemize}

\item \Security Some code that was no longer used was removed.  The presence
of this code could expose information which would allow an attacker to control
another user's job.  (CVE-2012-3493)

\item \Security Some code that was no longer used was removed.  The presence
of this code could have lead to a Denial-of-Service attack which would allow
an attacker to remove another user's idle job.  (CVE-2012-3491)

\item \Security Filesystem (FS) authentication was improved to check the UNIX
permissions of the directory used for authentication.  Without this, an
attacker may have been able to impersonate another submitter on the same submit
machine.  (CVE-2012-3492)

\item \Security Although not user-visible, there were multiple updates to
remove places in the code where potential buffer overruns could occur, thus
removing potential attacks.  None were known to be exploitable.

\item \Security Although not user-visible, there were updates to the code to
improve error checking of system calls, removing some potential security
threats.  None were known to be exploitable.


% https://access.redhat.com/security/cve/CVE-2012-349X


\item Fixed the \Condor{schedd} daemon; 
it would crash when a submit description file
contained a malformed \verb@$$()@ expansion macro that contained
a period.
\Ticket{3216}

\item Fixed a case in which a daemon could crash and leave behind a log
file owned by \Login{root}.  This \Login{root}-owned file would then cause
subsequent attempts to restart the daemon to fail.
\Ticket{2894}

\item Fixed a special case bug in which configuration variables
defined utilizing initial substrings of \Expr{\$(DOLLAR)},
for example  \Expr{\$(D)} and  \Expr{\$(DO)},  
were not expanded properly.
\Ticket{3217}

\item The command \Condor{q} \Opt{-run} now displays correct HOST field 
information for local universe jobs.
\Ticket{3150}

\end{itemize}

\noindent Known Bugs:

\begin{itemize}

\item None.

\end{itemize}

\noindent Additions and Changes to the Manual:

\begin{itemize}

\item None.

\end{itemize}


%%%%%%%%%%%%%%%%%%%%%%%%%%%%%%%%%%%%%%%%%%%%%%%%%%%%%%%%%%%%%%%%%%%%%%
\subsection*{\label{sec:New-7-8-3}Version 7.8.3}
%%%%%%%%%%%%%%%%%%%%%%%%%%%%%%%%%%%%%%%%%%%%%%%%%%%%%%%%%%%%%%%%%%%%%%

\noindent Release Notes:

\begin{itemize}

\item Condor version 7.8.3 released on September 6, 2012.

\end{itemize}


\noindent New Features:

\begin{itemize}

\item The \File{libcondorapi} library for reading and writing job event
logs is again available as a shared library on Linux and Mac OS platforms.
Since Condor 7.5.x, it had only been available as a static library.
\Ticket{3047}

\end{itemize}

\noindent Configuration Variable and ClassAd Attribute Additions and Changes:

\begin{itemize}

\item To avoid the output of an unnecessary DAGMan error message,
the value of \Macro{DAGMAN\_LOG\_ON\_NFS\_IS\_ERROR}
is ignored when both \MacroNI{CREATE\_LOCKS\_ON\_LOCAL\_DISK}
and \MacroNI{ENABLE\_USERLOG\_LOCKING} are \Expr{True}.
\Ticket{3087}

\end{itemize}

\noindent Bugs Fixed:

\begin{itemize}

\item Fixed a bug in which usage of cgroups incorrectly included the
page cache in the maximum memory usage.
This bug fix is also included in Condor version 7.9.0.
\Ticket{3003}

\item Jobs from a hook to fetch work, 
where the hook is defined by configuration variable 
\MacroNI{<Keyword>\_HOOK\_FETCH\_WORK},
now correctly receive dynamic slots from a partitionable slot 
instead of claiming the entire partitionable slot.
\Ticket{2819}

\item Fixed a bug in which a slot might become stuck in the Preempting state
when a \Condor{startd} is configured with a hook to fetch work,
as defined by \Macro{<Keyword>\_HOOK\_FETCH\_WORK}.
\Ticket{3076}

\item Fixed a bug that caused Condor to transfer a job's input files from
the execute machine back to the submit machine as if they were output files.
This would happen if the
job's input files were stored in Condor's spool directory;
occurred if the job was submitted via Condor-C or via 
\Condor{submit} with the \Opt{-spool} or \Opt{-remote} options.
\Ticket{2406}

\item Fixed a bug that could cause the first grid-type cream jobs destined 
for a particular CREAM server to never be submitted to that server.
This bug was probably introduced in Condor version 7.6.5.
\Ticket{3054}

\item Fixed several problems with the XML parsing class
\Code{ClassAdXMLParser} in the ClassAds library:
  \begin{itemize}
  \item Several methods named \Procedure{ParseClassAd} were declared, 
  but never implemented. 
\Ticket{3049}
  \item The parser silently dropped leading white space in string values.
\Ticket{3042}
  \item The parser could go into an infinite loop or leak memory when
    reading a malformed ClassAd XML document. 
\Ticket{3045}
  \end{itemize}

\item Fixed a bug that prevented the \Opt{-f} command line option to
\Condor{history} from being recognized.
The \Opt{-f} option was being interpreted as \Opt{-forward}. 
At least four letters are now required for the \Opt{-forward} option
(\Opt{-forw}) to prevent ambiguity.
\Ticket{3044}

\item The implementation of the \Condor{history} \Opt{-backwards} option, 
which is the default ordering for reading the history file,
in the 7.7 series did not work on Windows platforms.
This has been fixed.
\Ticket{3055}

\item Fixed a bug that caused an invalid proxy to be delegated when
refreshing the job's X.509 proxy when configuration variable
\Macro{DELEGATE\_JOB\_GSI\_CREDENTIALS\_LIFETIME} was set to 0.
\Ticket{3059}

\item Fixed a bug in which DAGMan did not account properly for jobs being
suspended and then unsuspended.
\Ticket{3108}

\item \Condor{dagman} now takes note of job reconnect failed 
events (event code 24) in the user log, for counting idle jobs.
\Ticket{3189}

\item Job IDs generated by NorduGrid ARC 12.05 and above are now
properly recognized.
\Ticket{3062}

\item Fixed a bug in which Condor would not mark grid-type nordugrid jobs
as Running due to variation in the format of the job status value.
NorduGrid ARC job statuses of the form \Expr{INLRMS: ?} are now
properly recognized both with and without the space after the colon.
\Ticket{3118}

\item The \Condor{gridmanager} now properly handles X.509 proxy files
that are specified in the job ClassAd with a relative path name.
\Ticket{3027}

\item Fixed a bug that caused daemon names,
as set in configuration variables such as \MacroNI{STARTD\_NAME},
containing a period character to be ignored.
\Ticket{3172}

\item Fixed a bug that prevented the \Condor{schedd} from removing old
execute directories for local universe jobs on start up.
\Ticket{3176}

\item The \Condor{defrag} daemon sometimes scheduled fewer draining attempts 
than specified.
\Ticket{3199}

\item Fixed a bug that could cause the \Condor{gridmanager} to crash if a
grid universe job's X.509 user certificate did not contain an e-mail
address.
\Ticket{3203}

\item Fixed a bug introduced in Condor version 7.7.5 that caused multiple
\Condor{schedd} daemons running on the same machine to share the job queue
with each other due to way in which the default value of configuration
variable \MacroNI{JOB\_QUEUE\_LOG} was set.
\Ticket{3196}

\item Fixed a bug that could cause \Condor{q} to not print all jobs when
it thought it was querying an old \Condor{schedd} daemon.
\Ticket{3206}

\item Fixed a bug that could cause a job's standard output and standard
error files to be written in the job's initial working directory,
despite the submit description file's specification to write them 
to a different directory.
This would happen when the file transfer mechanism was used, 
the execution machine was running Condor version 7.7.1 or earlier, 
and either Condor's security negotiation
was disabled or the configuration variable
\MacroNI{SEC\_ENABLE\_MATCH\_PASSWORD\_AUTHENTICATION} was set to \Expr{True}.
\Ticket{3208}

\item The log message generated when the \MacroNI{EXECUTE} directory
is missing is now more helpful.
\Ticket{3194}

\item The load average was incorrect for non-English versions on 
Windows platforms.
This has been fixed for Windows Vista and more recent versions.
\Ticket{3182}

\end{itemize}

\noindent Known Bugs:

\begin{itemize}

\item None.

\end{itemize}

\noindent Additions and Changes to the Manual:

\begin{itemize}

\item There is now documentation for the submit description file commands
\SubmitCmd{encrypt\_input\_files},
\SubmitCmd{encrypt\_output\_files},
\SubmitCmd{dont\_encrypt\_input\_files}, and
\SubmitCmd{dont\_encrypt\_output\_files} in the \Condor{submit}
manual page.
These commands have been available since Condor version 6.7.2,
but were never documented.
See descriptions starting at
~\ref{man-condor-submit-dont-encrypt-input-files}.
\Ticket{3174}


\end{itemize}


%%%%%%%%%%%%%%%%%%%%%%%%%%%%%%%%%%%%%%%%%%%%%%%%%%%%%%%%%%%%%%%%%%%%%%
\subsection*{\label{sec:New-7-8-2}Version 7.8.2}
%%%%%%%%%%%%%%%%%%%%%%%%%%%%%%%%%%%%%%%%%%%%%%%%%%%%%%%%%%%%%%%%%%%%%%

\noindent Release Notes:

\begin{itemize}

\item Condor version 7.8.2 released on August 14, 2012.

\item \Security Fixed a critical problem with DNS handling.

\end{itemize}

\noindent New Features:

\begin{itemize}

\item None.

\end{itemize}

\noindent Configuration Variable and ClassAd Attribute Additions and Changes:

\begin{itemize}

\item None.

\end{itemize}

\noindent Bugs Fixed:

\begin{itemize}

\item \Security Fixed a critical problem with DNS handling.

\end{itemize}

\noindent Known Bugs:

\begin{itemize}

\item None.

\end{itemize}

\noindent Additions and Changes to the Manual:

\begin{itemize}

\item None.

\end{itemize}

%%%%%%%%%%%%%%%%%%%%%%%%%%%%%%%%%%%%%%%%%%%%%%%%%%%%%%%%%%%%%%%%%%%%%%
\subsection*{\label{sec:New-7-8-1}Version 7.8.1}
%%%%%%%%%%%%%%%%%%%%%%%%%%%%%%%%%%%%%%%%%%%%%%%%%%%%%%%%%%%%%%%%%%%%%%

\noindent Release Notes:

\begin{itemize}

\item Condor version 7.8.1 released on June 15, 2012.

\end{itemize}


\noindent New Features:

\begin{itemize}

\item None.

\end{itemize}

\noindent Configuration Variable and ClassAd Attribute Additions and Changes:

\begin{itemize}

\item (Added in 7.8.0.) The new configuration variable
\Macro{ENABLE\_DEPRECATION\_WARNINGS} causes \Condor{submit} to issue
warnings when a job requests features that are no longer supported.
\Ticket{2968}

\item (Added in 7.7.6) The new configuration variable
\Macro{BATCH\_GAHP} should be used instead of \Macro{PBS\_GAHP},
\Macro{LSF\_GAHP} and \Macro{SGE\_GAHP}. These older configuration
variables are still recognized, but their use is now discouraged.
\Ticket{2670}

\item The default value for \Macro{GROUP\_SORT\_EXPR} was changed 
so that the \Expr{<none>} group would always negotiate last 
when using hierarchical group quotas.
Associated with that, 
the default value for \Macro{NEGOTIATOR\_ALLOW\_QUOTA\_OVERSUBSCRIPTION} 
was changed to \Expr{True}.  
These changes were made to make negotiation behave more like it did 
in the stable 7.4 series of Condor,
before hierarchical group quotas were added.
\Ticket{3040}

\end{itemize}

\noindent Bugs Fixed:

\begin{itemize}

\item Fixed a bug that caused events to not be written to the job event
log when the log is written in XML and a job policy expression triggering
the event contains any double quote marks.
\Ticket{3048}

\item Fixed a bug in the Condor init script that would cause
the init script to hang if Condor was not running.
\Ticket{2872}

\item Fixed a bug that caused parallel universe jobs using
Parallel Scheduling Groups 
(see section ~\ref{sec:Configure-Dedicated-Groups})
to occasionally stay idle even when
there were available machines to run them.
\Ticket{3017}

\item Fixed a bug that caused the \Condor{gridmanager} to crash when
attempting to submit jobs to a local PBS, LSF, or SGI cluster.
\Ticket{3014}

\item Fixed a bug in the handling of local universe jobs which caused
the \Condor{schedd} to log a spurious \Expr{ERROR} message
every time a local universe job exited, 
and then further caused the statistics for local universe jobs to be 
incorrectly computed.
\Ticket{3008}

\item Changed the internally used \Condor{ckpt\_probe} executable
to link statically, which should make the
checkpoint signature more resistant to non-significant changes in the system
configuration.
\Ticket{2901}

\item Restored Globus and VOMS support for the Mac OS X platform.
\Ticket{2991}

\item Fixed a bug when Condor runs under the PrivSep model,
in which if a job created a hard link from one file to another,
Condor was unable to transfer the files back to the submit side,
and the job was put on hold.
\Ticket{2987}

\item When configuration variables \MacroNI{MaxJobRetirementTime} or
\MacroNI{MachineMaxVacateTime} were very large, estimates of machine
draining badput and completion time were sometimes nonsensical
because of integer overflow.
\Ticket{3001}

\item Fixed a bug where per-job subdirectories and their contents
in \MacroUNI{SPOOL} would not be removed when the associated job
left the queue.
\Ticket{2942}

\item Fixed a bug that could cause the \Condor{schedd} to 
occasionally crash due to a race condition when running local universe jobs.
Associated with the bug would be the error message
\footnotesize
\begin{verbatim}
No local universe jobs were expected to be running, but one just exited!
\end{verbatim}
\normalsize
\Ticket{3009}

\end{itemize}

\noindent Known Bugs:

\begin{itemize}

\item None.

\end{itemize}

\noindent Additions and Changes to the Manual:

\begin{itemize}

\item Submit description file commands introduced in Condor version 7.7.1
have now been documented.
See the \Condor{submit} manual page at ~\ref{man-condor-submit} for
the newly added definitions of
\begin{description}
  \item[\SubmitCmd{ec2\_availability\_zone}]
  \item[\SubmitCmd{ec2\_ebs\_volumes}]
  \item[\SubmitCmd{ec2\_elastic\_ip}]
  \item[\SubmitCmd{ec2\_keypair\_file}]
  \item[\SubmitCmd{ec2\_vpc\_ip}]
  \item[\SubmitCmd{ec2\_vpc\_subnet}]
\end{description}

\item There is now a manual page for \Condor{router\_rm}, 
a script that provides additional features convenient for removing
jobs managed by the Condor Job Router.

\item Documentation not completed for the 7.7.6 release is now available.
The use of configuration variable \MacroNI{BATCH\_GAHP},
as well as the use of the new \SubmitCmd{grid\_resource} of
type \Expr{batch} for local submission of PBS, LSF, and SGE
jobs is documented.
See section ~\ref{sec:batch} for details.
\Ticket{2670}

\end{itemize}


%%%%%%%%%%%%%%%%%%%%%%%%%%%%%%%%%%%%%%%%%%%%%%%%%%%%%%%%%%%%%%%%%%%%%%
\subsection*{\label{sec:New-7-8-0}Version 7.8.0}
%%%%%%%%%%%%%%%%%%%%%%%%%%%%%%%%%%%%%%%%%%%%%%%%%%%%%%%%%%%%%%%%%%%%%%

\noindent Release Notes:

\begin{itemize}

\item Condor version 7.8.0 released on May 10, 2012.

\end{itemize}


\noindent New Features:

\begin{itemize}

\item (Added in 7.7.6.)  The new \Arg{-\_condor\_relocatable} argument
may be given as part of the invocation of a program that uses
standalone checkpointing.  This allows checkpointed programs to restart
without attempting to change to their original directory.
\Ticket{2877}

\item (Added in 7.7.5.) Added the \Arg{-absent} flag to \Condor{status},
which displays absent ClassAds.
\Ticket{2690}

\item (Added in 7.7.5.) Implement absent ads, which help track pool membership
in a persistent way.
\Ticket{2608} 

\end{itemize}

\noindent Configuration Variable and ClassAd Attribute Additions and Changes:

\begin{itemize}

\item The job ClassAd attribute \Attr{RemotePool} is now saved in
  \Attr{LastRemotePool} when the job finishes running.

\end{itemize}

\noindent Bugs Fixed:

\begin{itemize}

\item (Fixed in 7.7.6.) Fix \Arg{-absent}, \Arg{-vm}, and \Arg{-java}
flags to \Condor{status} so that they work with the \Arg{-long} option.
\Ticket{2943}

\item Support glob() on Scientific Linux 6 and others using the new
Linux system call fstatat(), but only when not using remote system calls.
\Ticket{2945}

\item Fixed potential startd crash introduced in v7.7.5 when claiming 
a partitionable slot that was in the Owner state. 
\Ticket{2936}

\item When ClassAd function stringListMember() is called with an empty
string as the second argument, it now evaluates to \Expr{False}.
Previously, it incorrectly evaluated to \Expr{Undefined}.
\Ticket{2953}

\item Format tags \%v and \%V for the \Opt{-format} option now properly
print all ClassAd value types. Previously, \Expr{True} and \Expr{False}
were printed as integers, and new ClassAd types like lists and nested
ClassAds could not be printed.
\Ticket{2960}

\item Restored RCS keyword strings CondorVersion and CondorPlatform to
the Condor binaries. These strings are found and printed by the 
\Opt{ident} program on Unix. They were missing in Condor versions 7.7.3
and later.
\Ticket{2932}

\item \Condor{job\_router} failed to route spooled source jobs.
\Ticket{2955}

\item Fixed a bug on Debian 6 and RHEL 6 that could cause standard
universe jobs to never checkpoint. This would happen if the job
triggered a call to NSCD (Name Service Caching Daemon) but NSCD 
wasn't running. 
Calls to NSCD can be triggered by a look up of a user account or
resolving a machine hostname to an IP address.
Now, NSCD is never consulted by a standard universe
job (this was already the behavior on other platforms).
\Ticket{2973}

\end{itemize}

\noindent Known Bugs:

\begin{itemize}

\item None.

\end{itemize}

\noindent Additions and Changes to the Manual:

\begin{itemize}

\item None.

\end{itemize}



%%%      PLEASE RUN A SPELL CHECKER BEFORE COMMITTING YOUR CHANGES!
%%%      PLEASE RUN A SPELL CHECKER BEFORE COMMITTING YOUR CHANGES!
%%%      PLEASE RUN A SPELL CHECKER BEFORE COMMITTING YOUR CHANGES!
%%%      PLEASE RUN A SPELL CHECKER BEFORE COMMITTING YOUR CHANGES!
%%%      PLEASE RUN A SPELL CHECKER BEFORE COMMITTING YOUR CHANGES!

%%%%%%%%%%%%%%%%%%%%%%%%%%%%%%%%%%%%%%%%%%%%%%%%%%%%%%%%%%%%%%%%%%%%%%
\section{\label{sec:History-7-7}Development Release Series 7.7}
%%%%%%%%%%%%%%%%%%%%%%%%%%%%%%%%%%%%%%%%%%%%%%%%%%%%%%%%%%%%%%%%%%%%%%

This is the development release series of Condor.
The details of each version are described below.

%%%%%%%%%%%%%%%%%%%%%%%%%%%%%%%%%%%%%%%%%%%%%%%%%%%%%%%%%%%%%%%%%%%%%%
\subsection*{\label{sec:New-7-7-0}Version 7.7.0}
%%%%%%%%%%%%%%%%%%%%%%%%%%%%%%%%%%%%%%%%%%%%%%%%%%%%%%%%%%%%%%%%%%%%%%

\noindent Release Notes:

\begin{itemize}

\item Condor version 7.7.0 not yet released.
%\item Condor version 7.7.0 released on Month Date, 2011.

\end{itemize}


\noindent New Features:

\begin{itemize}

\item \Condor{q} now supports a new option \Opt{-stream-results}.
  When this option is specified, \Condor{q} displays results as they
  are fetched from the job queue, rather than buffering up the query
  results before displaying anything.

\end{itemize}

\noindent Configuration Variable and ClassAd Attribute Additions and Changes:

\begin{itemize}

\item None.

\end{itemize}

\noindent Bugs Fixed:

\begin{itemize}

% gittrack 1928
\item \AdAttr{JobUniverse} was documented as an optional attribute for
  jobs created via the fetch-work hook, but it was effectively
  required, because the default \AdAttr{IsValidCheckpointPlatform}
  expression evaluated to \Expr{undefined} without it.  Now the
  default \AdAttr{IsValidCheckpointPlatform} expression evaluates to
  \Expr{true} when \AdAttr{JobUniverse} is not defined.

% gittrack 1943
\item When there are multiple cpus but only one slot, the slot name no
longer begins with ``slot1@''.

% #1871 commit e4cce6996764a5eaabb28106c44f901c23dc4bae
\item \AdAttr{stack\_size} was added to the commands that can be put in
  a job description file for \Condor{submit}. This command only applies to
  Linux jobs that are not running in the standard universe. This also
  adds ``\AdAttr{StackSize}'' to the job Ad.

\end{itemize}

\noindent Known Bugs:

\begin{itemize}

\item None.

\end{itemize}

\noindent Additions and Changes to the Manual:

\begin{itemize}

\item None.

\end{itemize}


%%%      PLEASE RUN A SPELL CHECKER BEFORE COMMITTING YOUR CHANGES!
%%%      PLEASE RUN A SPELL CHECKER BEFORE COMMITTING YOUR CHANGES!
%%%      PLEASE RUN A SPELL CHECKER BEFORE COMMITTING YOUR CHANGES!
%%%      PLEASE RUN A SPELL CHECKER BEFORE COMMITTING YOUR CHANGES!
%%%      PLEASE RUN A SPELL CHECKER BEFORE COMMITTING YOUR CHANGES!

%%%%%%%%%%%%%%%%%%%%%%%%%%%%%%%%%%%%%%%%%%%%%%%%%%%%%%%%%%%%%%%%%%%%%%
\section{\label{sec:History-7-6}Stable Release Series 7.6}
%%%%%%%%%%%%%%%%%%%%%%%%%%%%%%%%%%%%%%%%%%%%%%%%%%%%%%%%%%%%%%%%%%%%%%

This is a stable release series of Condor.
As usual, only bug fixes (and potentially, ports to new platforms)
will be provided in future 7.6.x releases.
New features will be added in the 7.7.x development series.

The details of each version are described below.

%%%%%%%%%%%%%%%%%%%%%%%%%%%%%%%%%%%%%%%%%%%%%%%%%%%%%%%%%%%%%%%%%%%%%%
\subsection*{\label{sec:New-7-6-5}Version 7.6.5}
%%%%%%%%%%%%%%%%%%%%%%%%%%%%%%%%%%%%%%%%%%%%%%%%%%%%%%%%%%%%%%%%%%%%%%

\noindent Release Notes:

\begin{itemize}

\item Condor version 7.6.5 not yet released.
%\item Condor version 7.6.5 released on Month Date, 2011.

\end{itemize}


\noindent New Features:

\begin{itemize}

\item Added support for detecting loadavg from the v3 Linux kernel.
\Ticket{2579}

\end{itemize}

\noindent Configuration Variable and ClassAd Attribute Additions and Changes:

\begin{itemize}

\item None.

\end{itemize}

\noindent Bugs Fixed:

\begin{itemize}

\item Fixed a performance problem on Windows platforms
that caused claim activations to
fail when more than about 8 jobs were already running on that machine.
\Ticket{2441}

\item Fixed a bug in which the submit event would not be written to the user
job log,
if the job was submitted with the \Opt{-remote} or \Opt{-spool} option to
\Condor{submit}.
\Ticket{2569}

\item Fixed a bug that caused \Opt{condor\_q -analyze} to fail if a job or
machine ad contained a string attribute ending in a backslash.
\Opt{condor\_q -analyze} would print the error 
``Unable to process machine ClassAds'' or ``Unable to process job ClassAd''.
\Ticket{2603}

\item Fixed a bug that caused \Condor{startd} to crash after being
reconfigured if a startd cron job was removed and was running at the
time.
\Ticket{2636}

\item Configuration parameters of the form \Macro{MAX\_<SUBSYS>\_<LEVEL>\_LOG}
now work properly on 32-bit linux. Previously, the corresponding log file
would grow without bound.
\Ticket{2638}

\end{itemize}

\noindent Known Bugs:

\begin{itemize}

\item None.

\end{itemize}

\noindent Additions and Changes to the Manual:

\begin{itemize}

\item None.

\end{itemize}


%%%%%%%%%%%%%%%%%%%%%%%%%%%%%%%%%%%%%%%%%%%%%%%%%%%%%%%%%%%%%%%%%%%%%%
\subsection*{\label{sec:New-7-6-4}Version 7.6.4}
%%%%%%%%%%%%%%%%%%%%%%%%%%%%%%%%%%%%%%%%%%%%%%%%%%%%%%%%%%%%%%%%%%%%%%

\noindent Release Notes:

\begin{itemize}

%\item Condor version 7.6.4 not yet released.
\item Condor version 7.6.4 released on October 21, 2011.

\end{itemize}


\noindent New Features:

\begin{itemize}

\item The new Windows-only \Condor{rmdir} was included in Condor version 7.6.0,
but there was no version history entry for this introduced tool at release.
This item attempts to correct that oversight, 
as well as identify that usage of \Condor{rmdir},
instead of the built-in Windows \Prog{rmdir}, 
is enabled by default.
\Condor{rmdir} worked correctly in Condor version 7.6.0, 
contained a bug in Condor version 7.6.1,
and was fixed in Condor version 7.6.2.
\Ticket{1877}


\end{itemize}

\noindent Configuration Variable and ClassAd Attribute Additions and Changes:

\begin{itemize}

\item The new configuration variable \Macro{<Keyword>\_HOOK\_JOB\_EXIT\_TIMEOUT}
defines the number of seconds that the \Condor{starter} will wait
before continuing with a shut down,
if a hook defined by \MacroNI{<Keyword>\_HOOK\_JOB\_EXIT} has not completed.
The addition of this configuration variable fixes the bug described below.
\Ticket{2543}

\item The new configuration variable \Macro{SKIP\_WINDOWS\_LOGON\_NETWORK} 
is a boolean value which specifies whether the Windows
\Expr{LOGON\_NETWORK} authentication is skipped or not.
If skipped, Condor tries \Expr{LOGON\_INTERACTIVE} authentication first.
The addition of this configuration variable fixes the bug described below.
\Ticket{2549}  

% This actually should have gone into 7.6.1, where it was committed.
% It appears here in the 7.6.4 history by wrangler's order, without
% owning up to the improper appearance in this version.
\item The new configuration variable \Macro{SHADOW\_RUN\_UNKNOWN\_USER\_JOBS} 
defaults to \Expr{False}.
When \Expr{True}, 
it allows the \Condor{shadow} daemon to run jobs remotely submitted from 
users not in the local password file.
\Ticket{2004}

\end{itemize}
\noindent Bugs Fixed:

\begin{itemize}

%\item Properly support \Attr{MAX\_<subsys>\_LOG} values >= 2GB.
\item Implemented proper support of values greater than or equal to  2 GBytes
set for the configuration variable \Macro{MAX\_<SUBSYS>\_LOG}.
\Ticket{2471}

\item Updated the \Condor{negotiator} daemon's assessment of pool size 
to properly take partitionable slots into account.
See section ~\ref{sec:Configuring-SMP} for an explanation of 
partitionable slots on SMP machines.
\Ticket{2440}

\item Provided an informative error message 
when the \Condor{userprio} tool cannot locate the \Condor{negotiator} daemon.
\Ticket{2478}

\item \Condor{userprio} and the \Condor{negotiator} daemon 
did not correctly handle the names of submitters, 
when these names exceeded 63 characters in length.
The fix handles submitter names of arbitrary length.
\Ticket{2445}

\item Removed a spurious boolean flag reset in \Condor{q},
which resulted in an order dependency between the command line arguments
\Opt{-long} and \Opt{-format}.
\Ticket{2498}

\item Fixed a bug in which a graceful shutdown of a \Condor{startd}
did not correctly handle jobs using job deferral
which have landed on an execute machine but have not yet
reached their deferral time.
These jobs would appear to be running, despite the lack of
a \Condor{starter} daemon. 
These jobs now correctly transition to the idle state.
\Ticket{2486}

\item Corrected a hierarchical group quota bug in which
the user accounting mechanism in the \Condor{negotiator} daemon allowed 
submitter records to be deleted,
if the submitter's priority factor was explicitly set and
the value was equal to that defined by \MacroNI{DEFAULT\_PRIO\_FACTOR}.
\Ticket{2442}

\item Fixed CPU detection on Windows, such that the correct number of CPUs
is detected when there are more than 32 CPUs.
\Ticket{2381}

\item Fixed a memory leak in the \Condor{negotiator},
caused by the failure to
free memory returned from some calls to \Procedure{param\_without\_default}.
\Ticket{2299}

\item Jobs run via \Prog{glexec} always had their \Env{PATH} environment
variable cleared.  Now, if \Env{PATH} was specified for the job, 
in any of the ways that job environment may be specified, 
this setting is used.
\Ticket{2096}

\item Fixed an infinite loop that could happen in Condor daemons
shortly after the receipt of a new connection.  
This problem was introduced in Condor version 7.5.6.
\Ticket{2413}

\item Fixed a problem in \Condor{hdfs} that caused it to exit shortly
after starting up,
if the configuration variables 
\MacroNI{HDFS\_DENY}, \MacroNI{HOSTDENY\_WRITE}, or \MacroNI{HOSTDENY\_READ} 
were not defined.
Previously, if \MacroNI{HDFS\_DENY} was
not defined, \MacroNI{HOSTDENY\_WRITE} and \MacroNI{HOSTDENY\_READ}
were used to build the deny list.  
Now \MacroNI{DENY\_WRITE} and \MacroNI{DENY\_READ} are also used.
\Ticket{2414}

\item Removed an extra copy of the java files required to run gt4 and gt42
grid universe jobs. This does not affect Condor's operation.
\Ticket{2435}

\item Fixed a problem that caused the \Condor{schedd} to crash when
writing to some user logs with specific names.  The specific names that
caused crashes are not easy to describe.
\Ticket{2439}

\item Fixed a bug in which the \Condor{schedd} failed to start up
when any job ClassAd attribute value contained the ASCII character 255.
\Ticket{2450}

\item Fixed a bug in which \Condor{preen} failed to honor the 
\Opt{-remove} option, and would always remove lock files.
\Ticket{2497}

\item \Condor{preen} expected an old format for local lock file paths;
it now understands the proper format.
\Ticket{2496}

\item \Condor{preen} would EXCEPT when processing multiple 
subdirectories for local locks.
\Ticket{2495}

\item In 32-bit Condor binaries, the \Attr{ImageSize} of processes larger than 
4 GBytes was reported as 4 GBytes.  This limit has been raised to 4095 GBytes.

\item \SubmitCmd{vm} universe jobs using Xen or KVM would fail to start,
if the disk image files were transferred from the submit machine
and the default value defined for \Macro{LIBVIRT\_XML\_SCRIPT} was used.
The script did not provide absolute path names for the files.
\Ticket{2511}

\item Fixed a bug in which a completed Xen or KVM \SubmitCmd{vm} universe 
job's modified disk image files would not be transferred back 
to the submit machine.
\Ticket{2530}

\item Fixed a bug in which a \Condor{starter} configured to use job hooks 
could fail to run a job, 
but not wait for the job exit hook to complete before exiting.  
The bug fix introduces the new configuration variable
\Macro{<Keyword>\_HOOK\_JOB\_EXIT\_TIMEOUT},
which defines the number of seconds the \Condor{starter} will wait
before continuing with a shut down,
if the job exit hook has not completed.
\Ticket{2543}

\item In Condor version 7.5.4, an improvement was made to avoid reliance on
the machine specified by \MacroNI{NEGOTIATOR\_HOST} 
matching a reverse DNS look up of the \Condor{negotiator}.
However, this improvement was not made to the dedicated scheduler,
so matchmaking of parallel jobs was still subject to the
problems associated with the old algorithm.  
Now, the dedicated scheduler benefits from the same improved algorithm as the
non-dedicated scheduler.
\Ticket{2540}
  
\item Occasionally there have been problems with Windows 
\Expr{LOGON\_NETWORK} authentication,
leading to users being locked out from their account.
The new configuration variable \MacroNI{SKIP\_WINDOWS\_LOGON\_NETWORK},
when set to \Expr{True},
fixes the problem by allowing this mechanism to be skipped entirely,
instead proceeding straight to the \Expr{LOGON\_INTERACTIVE} authentication. 
This bug only affected users using the \Condor{credd}. 
\Ticket{2549}  

\item Condor now correctly groups CREAM jobs based on how CREAM servers 
authorize and map them.
The servers map them based on X.509 proxy subject name 
and first VOMS attribute. 
Previously, all VOMS attributes were considered.
This could cause unexpected behavior due to the aliasing of CREAM leases
and proxy delegations.
\Ticket{2271}

\item Communication errors in the job lease renewal protocol were
sometimes not correctly handled.  This resulted in the job being
killed.
\Ticket{2563}

\end{itemize}

\noindent Known Bugs:

\begin{itemize}

\item None.

\end{itemize}

\noindent Additions and Changes to the Manual:

\begin{itemize}

\item The manual now contains a manual page for \Condor{rmdir},
a Windows only replacement for the built-in Windows \Prog{rmdir}
introduced in Condor version 7.6.0.

\end{itemize}


%%%%%%%%%%%%%%%%%%%%%%%%%%%%%%%%%%%%%%%%%%%%%%%%%%%%%%%%%%%%%%%%%%%%%%
\subsection*{\label{sec:New-7-6-3}Version 7.6.3}
%%%%%%%%%%%%%%%%%%%%%%%%%%%%%%%%%%%%%%%%%%%%%%%%%%%%%%%%%%%%%%%%%%%%%%

\noindent Release Notes:

\begin{itemize}

\item Condor version 7.6.3 released on August 23, 2011.

\end{itemize}


\noindent New Features:

\begin{itemize}

\item None.

\end{itemize}

\noindent Configuration Variable and ClassAd Attribute Additions and Changes:

\begin{itemize}

\item None.

\end{itemize}

\noindent Bugs Fixed:

\begin{itemize}

\item Fixed a bug causing parallel universe jobs to be preempted upon 
renewal of the job lease, 
which by default happens within 20 minutes. 
This meant that essentially no parallel universe job that took
longer than 20 minutes would ever finish.
\Ticket{2317}

\item When the specified job requirements expression contained a
reference to \Attr{RequestMemory}, there was inconsistent behavior:
in some cases the default \Attr{RequestMemory} requirements were
suppressed, and in other cases not.  Now, the default
\Attr{RequestMemory} requirements are always suppressed when there
are explicit references to \Attr{RequestMemory} in the job
requirements.

\item Fixed a bug that could cause Condor to crash when using 
the Local Credential Mapping Service (LCMAPS) with
GSI authentication.
\Ticket{2340}

\item Fixed a bug that caused the \Condor{collector} daemon to crash
upon receiving a CCB command,
when \Macro{ENABLE\_CCB\_SERVER} was changed from \Expr{True} to \Expr{False}
without restarting the daemon.
\Ticket{2357}

\item The GT2 GAHP no longer consumes all of the CPU when compiled
with threaded Globus libraries.
\Ticket{2345}

\item Fixed a problem introduced in Condor version 7.5.6, 
which led to local lock files for user log locking always being created 
whether or 
not \MacroNI{ENABLE\_USERLOG\_LOCKING} was set to \Expr{False}.
\Ticket{2116}

\item Installation as a service by the MSI installer on Windows platforms 
now sets the default of Automatic Delayed.
\Ticket{2318}

\item In PrivSep mode, if started as \Login{root}, 
the \Condor{master} re-executes itself as the \Login{condor} user.
Previously, supplementary groups were preserved.
Now supplementary groups are cleared and set to the list of groups
to which the \Login{condor} user belongs.
\Ticket{2376}

% commit 3d145180fd55b0d50e74656765cebe561c840830
% commit fea686687f5dda08908e03b5595c517b3ddda03a
\item Fixed a bug where setting \Macro{DAGMAN\_PROHIBIT\_MULTI\_JOBS} to
\Expr{True} caused SUBDAGs to stop working.
\Ticket{2331}

\item Fixed a bug that caused scheduler universe jobs submitted via
Condor-C or \Condor{submit} \Opt{-spool} to be held and be unable to run.
The hold reason given was \Expr{File <filename> is missing or not executable}.
\Ticket{2396}

\item \Condor{submit} now exits with an error,
if the command \Expr{hold = True} is in the submit description file
when using \Opt{-spool} or \Opt{-remote} as command-line arguments. 
This combination of settings resulted in jobs being unable to run.
\Ticket{2398}

\end{itemize}

\noindent Known Bugs:

\begin{itemize}

\item None.

\end{itemize}

\noindent Additions and Changes to the Manual:

\begin{itemize}

\item None.

\end{itemize}


%%%%%%%%%%%%%%%%%%%%%%%%%%%%%%%%%%%%%%%%%%%%%%%%%%%%%%%%%%%%%%%%%%%%%%
\subsection*{\label{sec:New-7-6-2}Version 7.6.2}
%%%%%%%%%%%%%%%%%%%%%%%%%%%%%%%%%%%%%%%%%%%%%%%%%%%%%%%%%%%%%%%%%%%%%%

\noindent Release Notes:

\begin{itemize}

\item Condor version 7.6.2 released on July 19, 2011.

\end{itemize}


\noindent New Features:

\begin{itemize}

\item Improved how \Condor{dagman} deals with certain parse errors:
missing node name or submit description file in JOB lines.
Also, \Condor{dagman}
now prints DAG input file lines as they are parsed, 
if the debug verbosity setting is 6 or above,
as set with the \Condor{submit\_dag} command line option \Opt{-debug}.

\end{itemize}

\noindent Configuration Variable and ClassAd Attribute Additions and Changes:

\begin{itemize}

\item None.

\end{itemize}

\noindent Bugs Fixed:

\begin{itemize}

% gittrac #2275 
\item Fixed a bug in the \Condor{negotiator} that impacted the processing 
of machine \MacroNI{RANK} such that \Condor{startd} \MacroNI{RANK}
preemption only occurred if the preempting user had sufficient user priority 
to claim another machine. 

% gittrac #2235 
\item \Condor{ssh\_to\_job} did not work on systems using the 
dash shell for \Prog{/bin/sh}.

% gittrac #2263 
\item \Condor{ssh\_to\_job} now works with jobs that are run via 
\Prog{glexec}. Previously, it did not. 

% gittrac #1642 
\item When \Prog{glexec} was configured with \Expr{linger=on},
the \Condor{starter} would become unresponsive for the duration of the job. 
For jobs longer than the value set by configuration variable
\MacroNI{NOT\_RESPONDING\_TIMEOUT},
this caused the job to be aborted. 
This also prevented job resource usage monitoring from working 
while the job was running.

% gittrac #2262 
\item Fixed a bug in hierarchical group quotas that caused 
a warning to be logged, despite correct implementation.

% gittrac #2261 
\item \Condor{preen} now properly respects the convention that
the \Opt{-debug} option causes \Procedure{dprintf} logging to \Code{stderr}. 

% gittrac #2253 
% gittrac #2294 
\item Fixed a problem introduced in Condor version 7.5.5 
that could cause the \Condor{schedd} to crash when a job was removed 
during negotiation or when an idle parallel universe job left the queue. 

% gittrac #2247 
\item Fixed a problem that sometimes caused the \Condor{procd} to die.
The chain of events for this fixed bug were that
the \Condor{startd} killed the \Condor{starter} due to unresponsiveness,
and the \Condor{procd} would die.
Then \Condor{startd} logged the message
\Expr{ProcD has failed} and the \Condor{startd} exited. 

% gittrac #2233 
\item Fixed a problem introduced in Condor version 7.6.1 
that caused the \Condor{shadow} to crash without successfully putting the job 
on hold when the user log could not be opened for writing. 

% gittrac #2210 
\item \Condor{history} no longer crashes when given a constraint expression 
longer than 512 characters. 

% gittrac #2248 
\item PBS and LSF grid jobs that arrive in a queue via Condor-C
or remote submission again work properly. 

% gittrac #2210 
\item Fix a bug that can cause the \Condor{gridmanager} to crash 
when a CREAM job ClassAd is missing the \Attr{X509UserProxy} attribute. 

% gittrac #2202 
\item Fix a bug that caused CREAM jobs to have incomplete input files,
if the \Condor{gridmanager} crashed during transfer of those input files. 

% gittrac #2201 
\item Fix a bug in \Condor{submit} that caused grid jobs intended for 
CREAM services whose names contain extra dashes to become held. 

\item Fixed a bug in which \Condor{submit} would try, 
but fail to open the Deltacloud password file,
when the file name was dependent on an expression specified with \Expr{\$\$()}.

% gittrac #2173 
\item If the \Attr{Owner} attribute was not set in the ClassAd associated
with a cluster of jobs,
shared spooled executables were not correctly cleaned up.

% gittrac #2238 
\item Fixed a bug for 64-bit versions of Windows in which the
user \Login{condor-reuse-slot<N>} showed up on the login screen.

% gittrac #2288 
\item Fixed a bug introduced in Condor version 7.5.5,
which changed the default value of the configuration variable
\Macro{INVALID\_LOG\_FILES} from the empty set to a file called \File{core}.
This resulted in core files being removed unexpectedly by \Condor{preen},
and that complicated debugging of Condor.
Previous behavior has been restored.

% gittrac #2278 
\item Fixed a bug existing since Condor version 7.5.5 on Windows platforms.
The installer installed Java jar files in the correct \verb|$(BIN)| directory,
while the value for the configuration variable 
\MacroNI{JAVA\_CLASSPATH\_DEFAULT} utilized the obsolete \verb|$(LIB)|
directory.
The installer now correctly sets \MacroNI{JAVA\_CLASSPATH\_DEFAULT} 
to the \verb|$(BIN)| directory.

% gittrac #2308
\item Fixed a problem causing Condor to fail to start when
privsep was enabled and the environment had any variables
containing newlines.

\end{itemize}

\noindent Known Bugs:

\begin{itemize}

\item For Condor versions 7.6.2, 7.6.1, and 7.6.0,
a bug causes parallel universe jobs to be preempted upon 
renewal of the job lease, 
which by default will happen within 20 minutes, 
essentially meaning that no parallel universe job that takes
longer than 20 minutes can ever finish.
The work around for this bug is to place the following
configuration variable in the configuration of the submit machine:
\begin{verbatim}
  STARTD_SENDS_ALIVES = FALSE
\end{verbatim}
A \Condor{reconfig} is required, 
after which the preempted parallel universe jobs will then be
able to run to completion.

\end{itemize}

\noindent Additions and Changes to the Manual:

\begin{itemize}

\item None.

\end{itemize}


%%%%%%%%%%%%%%%%%%%%%%%%%%%%%%%%%%%%%%%%%%%%%%%%%%%%%%%%%%%%%%%%%%%%%%
\subsection*{\label{sec:New-7-6-1}Version 7.6.1}
%%%%%%%%%%%%%%%%%%%%%%%%%%%%%%%%%%%%%%%%%%%%%%%%%%%%%%%%%%%%%%%%%%%%%%

\noindent Release Notes:

\begin{itemize}

\item Condor version 7.6.1 released on June 3, 2011.

\end{itemize}


\noindent New Features:

\begin{itemize}

\item None.

\end{itemize}

\noindent Configuration Variable and ClassAd Attribute Additions and Changes:

\begin{itemize}

\item None.

\end{itemize}

\noindent Bugs Fixed:

\begin{itemize}

% gittrac #2170 
\item A bug introduced in Condor version 7.5.5 caused the \Condor{schedd}
to die when its attempt to claim a slot for a parallel universe job 
was rejected by the \Condor{startd}. 

% gittrac #2059
\item \Condor{q} \Opt{-analyze} failed to provide detailed analysis of
the job's requirements expression when the expression contained ClassAd
function calls in some cases. 

% gittrac #2192
\item Fixed an incorrect exit code from \Condor{q} 
when invoked with the \Opt{-name} option and using Quill.

%gittrac #2013
\item Fixed a segmentation fault bug introduced in Condor version 7.5.5,
in the checkpoint and restart of jobs using compressed checkpoint images
under the standard universe.
By default, Condor will not compress checkpoints under the standard universe.
Jobs which do not compress their checkpoints were immune to this bug.  
Compressed checkpoints are only available in 32-bit versions of Condor.
Generation of checkpoints in 64-bit versions of Condor are unaffected.

% gittrac #2069
\item In Condor version 7.6.0, the \Condor{schedd} would create a 
spool directory for every job. The corrected and previous behavior 
has now been restored, 
which is to create a spool directory only when needed.

%gittrac #2086
\item Fixed a bug introduced in Condor version 7.5.2,
that caused the \Condor{negotiator} daemon to crash
if any machine ClassAds contained cyclical attribute references.

%gittrac #2101
\item Fixed a bug that caused usage by \SubmitCmd{nice\_user} jobs to
be charged to the user directly rather than `nice-user.\emph{user}'.
This bug was introduced in the 7.5 series.

%gittrac #2081
\item Fixed bugs in the RPM init script that could cause some 
shutdown failures to be unreported, 
and they could cause status requests,
such as \Expr{service condor status},
to always report Condor as inactive.

\item Fixed a bug in the \Condor{gridmanager} that could cause a crash 
when a grid type \SubmitCmd{amazon} job was missing required attributes.

%gittrac #2105
\item Fixed bug in the \Condor{shadow}, in which it would treat 
the closed socket to the execute machine as an error,
after both it had asked for the claim to be deactivated and the 
\Condor{schedd} daemon was busy.  
As a result, a busy \Condor{schedd} could result in the job being re-run.

%gittrac #2109
\item The matchmaking attributes 
\Attr{SubmitterUserResourcesInUse} and \Attr{RemoteUserResourcesInUse} 
no longer ignore \Attr{SlotWeight}, if used by the \Condor{negotiator}.

%gittrac #2102
\item On Windows, the \Condor{kbdd} daemon was missing changes to the
port on which the \Condor{startd} was listening.
This resulted in failure of the \Condor{kbdd} to send updates in 
keyboard and mouse activity,
further causing the failure of policy implementation that relied upon 
knowledge of the activity.

%gittrac #2163
\item Fixed a bug present throughout ClassAds,
in which expressions expecting a floating point value returned an error,
if the expression actually evaluated to a boolean.
This is most common in machine \MacroNI{RANK} expressions.

%gittrac #2172
\item Fixed a bug in the \Condor{negotiator} daemon,
which caused a crash if the \Condor{negotiator} was reconfigured 
during a negotiation cycle, 
but only if hierarchical group quotas were in use.

%gittrac #2162
\item Fixed a bug in which when submitting a job into the \Condor{schedd}
remotely, or with spooling, 
the file transfer plug-ins activated on the submit machine 
and pulled down all the specified URLs in the transfer list 
to the spool directory. 
This behavior has been changed so that URLs are only downloaded 
when the job is actually running with a \Condor{starter} above it. 
This is usually on an execute node, but could also be in the local universe. 

%gittrac #2178
\item Removed the requirement that the Condor GAHP needs DAEMON-level 
authorization access to the \Condor{gridmanager}. 

%gittrac #2181
\item On Windows platforms only, 
fixed a bug that could cause a sporadic access violation 
when a Condor daemon spawned another process.

%gittrac #2191
\item Fixed a bug that would cause the \Condor{startd} to 
incorrectly report \Expr{Benchmarking} as its activity, instead of \Expr{Idle}
when there was a problem launching the benchmarking programs. 

%gittrac #2193
\item Fixed a bug in which the \Condor{startd} can get stuck in a loop,
trying to execute an invalid, non-existent Daemon ClassAd Hook job. 

%gittrac #2088
\item Fixed a bug in which the dedicated scheduler did not correctly 
deactivate claims,
tending to result in jobs that appear to move back and forth between
the \Expr{Idle} and \Expr{Running} states,
with the \Condor{shadow} daemon exiting each time with status 108.

\end{itemize}

\noindent Known Bugs:

\begin{itemize}

\item None.

\end{itemize}

\noindent Additions and Changes to the Manual:

\begin{itemize}

\item None.

\end{itemize}


%%%%%%%%%%%%%%%%%%%%%%%%%%%%%%%%%%%%%%%%%%%%%%%%%%%%%%%%%%%%%%%%%%%%%%
\subsection*{\label{sec:New-7-6-0}Version 7.6.0}
%%%%%%%%%%%%%%%%%%%%%%%%%%%%%%%%%%%%%%%%%%%%%%%%%%%%%%%%%%%%%%%%%%%%%%

\noindent Release Notes:

\begin{itemize}

\item Condor version 7.6.0 released on April 19, 2011.

% gittrac #2016
\item Prior to Condor version 7.5.0, commenting out \MacroNI{PREEN} in the
  default configuration file disabled \Condor{preen}.  
  Starting in Condor version 7.5.0,
  there was an internal default value for \MacroNI{PREEN}, so if
  the configuration variable was not set in any configuration file,
  \Condor{preen} would still run.
  To disable this functionality, \MacroNI{PREEN} can be explicitly set to
  nothing.

\end{itemize}


\noindent New Features:

\begin{itemize}

\item Condor can now create and manage virtual machine instances in a
cloud service via Deltacloud. This is done via the new
\SubmitCmd{deltacloud} grid type in the grid universe.
See section ~\ref{sec:Deltacloud} for details.

% gittrac #1931
\item Improved scalability of submission of cream grid type jobs.

\end{itemize}

\noindent Configuration Variable and ClassAd Attribute Additions and Changes:

\begin{itemize}

\item The new configuration variable \Macro{DELTACLOUD\_GAHP} specifies
where the \Prog{deltacloud\_gahp} binary is located. This binary is used to
manage deltacloud grid type jobs in the grid universe.
In a normal Condor installation, the value should be
\File{\$(SBIN)/deltacloud\_gahp}.

\item Several new job ClassAd attributes have been added to support
the deltacloud grid type in the grid universe.
These attributes are taken from the submit description file:
\AdAttr{DeltacloudUsername},
\AdAttr{DeltacloudPasswordFile},
\AdAttr{DeltacloudImageId},
\AdAttr{DeltacloudRealmId},
\AdAttr{DeltacloudHardwareProfile},
\AdAttr{DeltacloudHardwareProfileCpu},
\AdAttr{DeltacloudHardwareProfileMemory},
\AdAttr{DeltacloudHardwareProfileStorage},
\AdAttr{DeltacloudKeyname}, and
\AdAttr{DeltacloudUserData}.
%\AdAttr{DeltacloudRetryTimeout},
These attributes are set by Condor when the instance runs:
\AdAttr{DeltacloudAvailableActions},
\AdAttr{DeltacloudPrivateNetworkAddresses},
\AdAttr{DeltacloudPublicNetworkAddresses}.
See section ~\ref{sec:Deltacloud} for details of running jobs under
Deltacloud, and see section ~\ref{sec:Job-ClassAd-Attributes}
for definitions of these job ClassAd attributes.

% gittrac #2024
\item The configuration variable \Macro{JAVA\_MAXHEAP\_ARGUMENT} 
  has been removed. 
  This means that Java universe jobs will now run with the JVM's 
  default maximum heap setting,
  unless specified otherwise by the administrator using the configuration
  of \Macro{JAVA\_EXTRA\_ARGUMENTS},
  or by the job via 
  \SubmitCmd{java\_vm\_args} in the submit description file
  as described in section~\ref{sec:Java}.

% gittrac #2066
\item The configuration variable \Macro{TRUST\_UID\_DOMAIN}
  was set to \Expr{True} in the default \File{condor\_config.local}
  in the rpm and Debian packages.  This is no longer the case.
  This setting will therefore use the default value \Expr{False}.

\item The configuration variable \Macro{NEGOTIATOR\_INTERVAL} was set
  to 20 in the default \File{condor\_config.local} in the rpm and
  Debian packages.  This is no longer the case.  This setting
  therefore will use the default value 60.

\end{itemize}

\noindent Bugs Fixed:

\begin{itemize}

% gittrac #1957
\item Fixed a bug in \Condor{dagman} that caused it to fail when in recovery
mode in the face of a specific combination of node job failures with
retries.

% gittrac #1991
\item Fixed a bug that resulted in the spooled user log not being
  fetched by \Condor{transfer\_data}.  Prior to Condor version 7.5.4, this
  problem affected spooled jobs submitted with an explicit list of
  output files to transfer.  In Condor version 7.5.4, this problem also
  affected spooled jobs that auto-detected output files.

% gittrac #1985
\item When a job is submitted with the \Opt{-spool} option to \Condor{submit},
the \Condor{schedd} now correctly writes the submit event to the user log 
in its spool directory. 
Previously, it would try to write the event using the user
log path given to \Condor{submit} by the user, 
which \Condor{submit} may not have access to.

% gittrac #2001
\item Fixed a file descriptor leak in the \Condor{vm-gahp}. The leak would
cause the daemon to fail if a VMware job ran for more than 16 hours on a
Linux machine.

%gittrac #2017
\item Fixed a bug in \Condor{dagman} that caused it to treat any dollar
sign in the log file name of a node job's submit description file
as an illegal \Condor{dagman} macro.
Now only the sequence of characters \Expr{\$(} delimits a macro.

\end{itemize}

\noindent Known Bugs:

\begin{itemize}

\item There are two known issues related to the automatic creation
of checkpoints with the Condor checkpointing library in 
Condor version 7.6.0.
The first is that compression of
standalone checkpoints is disabled for 32-bit binaries.
It was always disabled previously, for 64-bit binaries.
A standalone checkpoint is one that is run outside
of Condor's standard universe.  The second problem has to do with compressed
32-bit checkpoint files within the standard universe.
If a user requests a compressed 32-bit checkpoint in the standard universe,
the resulting checkpoint will not be compressed.
As with standalone checkpoints, this has never been supported
in 64-bit binaries.  We hope to fix both problems in 
Condor version 7.6.1.

\end{itemize}

\noindent Additions and Changes to the Manual:

\begin{itemize}

\item None.

\end{itemize}


% as of April 2012, Karen no longer wants to include these older
% version histories with the 7.4 and 7.5 manuals.
%%%%      PLEASE RUN A SPELL CHECKER BEFORE COMMITTING YOUR CHANGES!
%%%      PLEASE RUN A SPELL CHECKER BEFORE COMMITTING YOUR CHANGES!
%%%      PLEASE RUN A SPELL CHECKER BEFORE COMMITTING YOUR CHANGES!
%%%      PLEASE RUN A SPELL CHECKER BEFORE COMMITTING YOUR CHANGES!
%%%      PLEASE RUN A SPELL CHECKER BEFORE COMMITTING YOUR CHANGES!

%%%%%%%%%%%%%%%%%%%%%%%%%%%%%%%%%%%%%%%%%%%%%%%%%%%%%%%%%%%%%%%%%%%%%%
\section{\label{sec:History-7-5}Development Release Series 7.5}
%%%%%%%%%%%%%%%%%%%%%%%%%%%%%%%%%%%%%%%%%%%%%%%%%%%%%%%%%%%%%%%%%%%%%%

This is the development release series of Condor.
The details of each version are described below.

%%%%%%%%%%%%%%%%%%%%%%%%%%%%%%%%%%%%%%%%%%%%%%%%%%%%%%%%%%%%%%%%%%%%%%
\subsection*{\label{sec:New-7-5-2}Version 7.5.2}
%%%%%%%%%%%%%%%%%%%%%%%%%%%%%%%%%%%%%%%%%%%%%%%%%%%%%%%%%%%%%%%%%%%%%%

\noindent Release Notes:

\begin{itemize}

\item Condor version 7.5.2 not yet released.
%\item Condor version 7.5.2 released on Month Date, 2010.

\end{itemize}


\noindent New Features:

\begin{itemize}

% gittrack 1231
\item The \Condor{schedd} daemon uses less disk bandwidth when logging
updates to job ClassAds from running jobs and also when removing jobs
from the queue.  This should improve performance in situations where
disk bandwidth is a limiting factor.  Updates to the job queue log
have also been optimized in a number of cases to be less disk intensive.

\end{itemize}

\noindent Configuration Variable and ClassAd Attribute Additions and Changes:

\begin{itemize}

\item None.

\end{itemize}

\noindent Bugs Fixed:

\begin{itemize}

\item None.

\end{itemize}

\noindent Known Bugs:

\begin{itemize}

\item None.

\end{itemize}

\noindent Additions and Changes to the Manual:

\begin{itemize}

\item None.

\end{itemize}


%%%%%%%%%%%%%%%%%%%%%%%%%%%%%%%%%%%%%%%%%%%%%%%%%%%%%%%%%%%%%%%%%%%%%%
\subsection*{\label{sec:New-7-5-1}Version 7.5.1}
%%%%%%%%%%%%%%%%%%%%%%%%%%%%%%%%%%%%%%%%%%%%%%%%%%%%%%%%%%%%%%%%%%%%%%

\noindent Release Notes:

\begin{itemize}

\item Condor version 7.5.1 released on March 2, 2010.

\item All bug fixes and features which are in Condor version 7.4.2
are in this 7.5.1 release.

\item The Condor release is now available as a proper RPM or Debian
package.

\item Condor now internally uses the version of New ClassAds provided
as a stand-alone library (\URL{http://www.cs.wisc.edu/condor/classad/}).
Previously, Condor 
used an older version of ClassAds that was heavily tied to the Condor 
development libraries. This change should be transparent in the 
current development series. In the next development series (7.7.x),
Condor  will begin to use features of New ClassAds that were unavailable in 
Old ClassAds. 
Section~\ref{sec:classad-newandold} details the differences.

\item HPUX 11.00 is no longer a supported platform.

\end{itemize}


\noindent New Features:

\begin{itemize}

% gittrac #1102
\item A port number defined within \Macro{CONDOR\_VIEW\_HOST} may now use 
  a shared port.

% gittrac #1104
\item The \Condor{master} no longer pauses for 3 seconds after starting
  the \Condor{collector}.  However, if the configuration variable
  \MacroNI{COLLECTOR\_ADDRESS\_FILE} defines a file, 
  the \Condor{master} will wait for that file to be created
  before starting other daemons.

% gittrac #1144
\item In the grid universe, Condor can now automatically distinguish
between GRAM2 and GRAM5 servers, that is grid types \SubmitCmd{gt2} and
\SubmitCmd{gt5}.
Users can submit jobs using a grid type of \SubmitCmd{gt2} or \SubmitCmd{gt5}
for either type of server.

% gittrac #938
\item Grid universe jobs using the CREAM grid system now batch up
common requests into larger single requests.  This
reduces network traffic, increases the number of parallel tasks
the Condor can handle at once, and reduces the load on the remote
gatekeeper.

% gittrac #1100
\index{submit commands!cream\_attributes}
\item The new submit description file command \SubmitCmd{cream\_attributes}
sets additional attribute/value pairs for the CREAM job description
that Condor creates when submitting a grid universe job 
destined for the CREAM grid system.

% gittrac #1138
\item The \Condor{q} command with option \Opt{-analyze} is now performs
the same analysis as previously occurred with the \Opt{-better-analyze} option.
Therefore, the output of \Condor{q} with the \Opt{-analyze} option
has different output than before.
The \Opt{-better-analyze} option is still recognized and behaves the same
as before, though it may be removed from a future version.

% gittrack #1169
\item Security sessions that are not used for longer than an hour are
now removed from the security session cache to limit memory usage.

% gittrack #1169
\item The number of security sessions in the cache is now advertised in
the daemon ClassAd as \Attr{MonitorSelfSecuritySessions}.

% gittrac #1078
\item \Condor{dagman} now has the capability to run DAGs containing nodes
that are declared to be NOOPs -- for these nodes, a job is never actually
submitted.  See section~\ref{dagman:NOOP} for information.

% gittrac #1128
\index{submit commands!vm\_macaddr}
\item The submit file attribute \SubmitCmd{vm\_macaddr} can now be used to set
the MAC address for vm universe jobs that use VMware. The range of valid
MAC addresses is constrained by limits imposed by VMware.

% gittrac #1173
\item The \Condor{q} command with option \Opt{-globus}
is now much more efficient in its communication with the \Condor{schedd}.

\end{itemize}

\noindent Configuration Variable and ClassAd Attribute Additions and Changes:

\begin{itemize}

% gittrac #1242
\item The new configuration variable \MacroNI{STRICT\_CLASSAD\_EVALUATION}
controls whether new or old ClassAd expression evaluation semantics are
used. In new ClassAd semantics, an unscoped attribute reference is only
looked up in the local ad. The default is False (use old ClassAd semantics).

% gittrac #221
\item The configuration variable
\MacroNI{DELEGATE\_FULL\_JOB\_GSI\_CREDENTIALS} now applies to all proxy
delegations done between Condor daemons and tools.
The value is a boolean and defaults to \Expr{False},
which means that when doing delegation Condor will now create a limited proxy
instead of a full proxy.

\item The new configuration variable
  \MacroNI{SEC\_<access-level>\_SESSION\_LEASE} specifies the maximum
  number of seconds an unused security session will be kept in a daemon's
  session cache before being removed to save memory.  The default is 3600.
  If the server and client have different configurations, the smaller
  one will be used.

\end{itemize}

\noindent Bugs Fixed:

\begin{itemize}

% gittrack #1141
\item The default value for \Macro{SEC\_DEFAULT\_SESSION\_DURATION}
  was effectively 3600 in Condor version 7.5.0.
  This produced longer than desired
  cached sessions for short-lived tools such as \Condor{status}.
  It also produced shorter than possibly desired cached sessions for
  long-lived daemons such as \Condor{startd}.  
  The default has been restored to what it was before Condor version 7.5.0,
  with the exception of \Condor{submit},
  which has been changed from 1 hour to 60 seconds.
  For command line tools, the default is 60 seconds,
  and for daemons it is 1 day.

% gittrack #1142
\item \MacroNI{SEC\_<access-level>\_SESSION\_DURATION} previously did
  not support integer expressions, but it did not detect invalid
  input, so the use of an expression could produce unexpected results.
  Now, like other integer configuration variables,
  a constant expression can be used and input is fully validated.

% gittrac #1196
\item The configuration variable \MacroNI{LOCAL\_CONFIG\_DIR} is no longer
ignored if defined in a local configuration file.

% gittrack #767
\item Removed the incorrect issuing of the following Condor version 7.5.0 
  warning to the
  \Condor{starter}'s log, even when the outdated, and no longer used
  configuration
  variable \MacroNI{EXECUTE\_LOGIN\_IS\_DEDICATED} was not defined.

\begin{verbatim}
WARNING: EXECUTE_LOGIN_IS_DEDICATED is deprecated.
Please use DEDICATED_EXECUTE_ACCOUNT_REGEXP instead.
\end{verbatim}


\end{itemize}

\noindent Known Bugs:

\begin{itemize}

\item None.

\end{itemize}

\noindent Additions and Changes to the Manual:

\begin{itemize}

\item None.

\end{itemize}


%%%%%%%%%%%%%%%%%%%%%%%%%%%%%%%%%%%%%%%%%%%%%%%%%%%%%%%%%%%%%%%%%%%%%%
\subsection*{\label{sec:New-7-5-0}Version 7.5.0}
%%%%%%%%%%%%%%%%%%%%%%%%%%%%%%%%%%%%%%%%%%%%%%%%%%%%%%%%%%%%%%%%%%%%%%

\noindent Release Notes:

\begin{itemize}

\item All bug fixes and features which are in 7.4.1 are in this 7.5.0 release.

\end{itemize}


\noindent New Features:

\begin{itemize}

% gittrack #892
\item Added the new daemon \Condor{shared\_port} for Unix platforms 
  (except for HPUX).
  It allows Condor daemons to share a
  single network port.  This makes opening access to Condor through a
  firewall easier and safer.  It also increases the scalability of a
  submit node by decreasing port usage. See
  section~\ref{sec:Config-shared-port} for more information.

% gittrack #960
\item Improved CCB's handling of rude NAT/firewalls that silently drop
TCP connections.

% gittrack #968
\item Simplified the publication of daemon addresses.
  \Attr{PublicNetworkIpAddr} and \Attr{PrivateNetworkIpAddr} have been removed.
  \Attr{MyAddress} contains both public and private addresses.  For now,
  \Attr{<Subsys>IpAddr} contains the same information.  In a future release,
  the latter may be removed.

% gittrack #975
\item Changes to \MacroNI{TCP\_FORWARDING\_HOST},
  \MacroNI{PRIVATE\_NETWORK\_ADDRESS}, and
  \MacroNI{PRIVATE\_NETWORK\_NAME} can now be made without requiring a
  full restart.  It may take up to one \Condor{collector} update interval 
  for the changes to become visible.

% gittrack #1002
\item Network compatibility with Condor prior to 6.3.3 is no longer
  supported unless \MacroNI{SEC\_CLIENT\_NEGOTIATION} is set to
  \Expr{NEVER}.  This change removes the risk of communication errors
  causing performance problems resulting from automatic fall-back to the
  old protocol.

% gittrack #930
\item For efficiency, authentication between the \Condor{shadow} and
  \Condor{schedd} daemons is now able to be cached and reused in more
  cases.  Previously, authentication for updating job information was
  only cached if read access was configured to require authentication.

\item \Condor{config\_val} will now report the default value for
  configuration variables that are not set in the configuration files.

% gittrac #939
\item The \Condor{gridmanager} now uses a single status call to obtain
the status of all CREAM grid universe jobs from the remote server.

% gittrac #955
\item The \Condor{gridmanager} will now retry CREAM commands that time out.

% gittrac #941
\item Forwarding a renewed proxy for CREAM grid universe jobs to the
remote server is now much more efficient.

\end{itemize}

\noindent Configuration Variable and ClassAd Attribute Additions and Changes:

\begin{itemize}

% gittrack #997
\item Removed the configuration variable 
  \MacroNI{COLLECTOR\_SOCKET\_CACHE\_SIZE}.
  Configuration of this parameter used to be mandatory to enable TCP updates
  to the \Condor{collector}.  Now no special configuration of the
  \Condor{collector} is required to allow TCP updates, but it is
  important to ensure that there are sufficient file descriptors for
  efficient operation.  See section~\ref{sec:tcp-collector-update} for
  more information.

% gittrack #892
\item The new configuration variable \MacroNI{USE\_SHARED\_PORT} 
  is a boolean value that specifies
  whether a Condor process should rely on the \Condor{shared\_port} daemon for
  receiving incoming connections.  Write access to
  \Macro{DAEMON\_SOCKET\_DIR} is required for this to take effect.
  The default is \Expr{False}.  If set to \Expr{True}, \MacroNI{SHARED\_PORT}
  should be added to \MacroNI{DAEMON\_LIST}.  See
  section~\ref{sec:Config-shared-port} for more information.

% gittrack #960
\item Added the new configuration variable \MacroNI{CCB\_HEARTBEAT\_INTERVAL}.
  It is the maximum
  number of seconds of silence on a daemon's connection to the CCB server
  after which it will ping the server to verify that the connection still
  works.  
  The default value is 1200 (20 minutes).
  This feature serves to both speed
  up detection of dead connections and to generate a guaranteed minimum
  frequency of activity to attempt to prevent the connection from being
  dropped.

\end{itemize}

\noindent Bugs Fixed:

\begin{itemize}

\item Fixed problem with a ClassAd debug function,
so it now properly emits debug information for ClassAd \Code{IfThenElse}
clauses.

\end{itemize}

\noindent Known Bugs:

\begin{itemize}

\item None.

\end{itemize}

\noindent Additions and Changes to the Manual:

\begin{itemize}

\item None.

\end{itemize}

%%%%      PLEASE RUN A SPELL CHECKER BEFORE COMMITTING YOUR CHANGES!
%%%      PLEASE RUN A SPELL CHECKER BEFORE COMMITTING YOUR CHANGES!
%%%      PLEASE RUN A SPELL CHECKER BEFORE COMMITTING YOUR CHANGES!
%%%      PLEASE RUN A SPELL CHECKER BEFORE COMMITTING YOUR CHANGES!
%%%      PLEASE RUN A SPELL CHECKER BEFORE COMMITTING YOUR CHANGES!

%%%%%%%%%%%%%%%%%%%%%%%%%%%%%%%%%%%%%%%%%%%%%%%%%%%%%%%%%%%%%%%%%%%%%%
\section{\label{sec:History-7-4}Stable Release Series 7.4}
%%%%%%%%%%%%%%%%%%%%%%%%%%%%%%%%%%%%%%%%%%%%%%%%%%%%%%%%%%%%%%%%%%%%%%

This is a stable release series of Condor.
As usual, only bug fixes (and potentially, ports to new platforms)
will be provided in future 7.4.x releases.
New features will be added in the 7.5.x development series.

The details of each version are described below.

%%%%%%%%%%%%%%%%%%%%%%%%%%%%%%%%%%%%%%%%%%%%%%%%%%%%%%%%%%%%%%%%%%%%%%
\subsection*{\label{sec:New-7-4-3}Version 7.4.3}
%%%%%%%%%%%%%%%%%%%%%%%%%%%%%%%%%%%%%%%%%%%%%%%%%%%%%%%%%%%%%%%%%%%%%%

\noindent Release Notes:

\begin{itemize}

\item Condor version 7.4.3 not yet released.
%\item Condor version 7.4.3 released on Month Date, 2010.

\end{itemize}


\noindent New Features:

\begin{itemize}

\item None.

\end{itemize}

\noindent Configuration Variable and ClassAd Attribute Additions and Changes:

\begin{itemize}

\item None.

\end{itemize}

\noindent Bugs Fixed:

\begin{itemize}

% gittrac 1329
\item Network connections for Condor file transfers were ignoring
  private network settings.  The connection from the execute node to
  the submit node always attempted to use the public network address
  of the submit machine.

% gittrac 1346
\item A single job could match multiple offline slots in a single
negotiation cycle.  This problem could cause \Condor{rooster} to
wake up too many offline machines for the number of jobs available
to run on them.  The fix for this problem requires updating both
the \Condor{negotiator} and the \Condor{schedd}.

% gittrac 1349
\item Fixed a problem that caused the \Condor{startd} daemon to
crash in some cases when \MacroNI{STARTD\_SENDS\_ALIVES} was \Expr{True}.
This setting is \Expr{False} by default.

% gittrac #1337
\item Fixed a problem where the \Condor{kbdd} has a chance of
entering an infinite loop on platforms that use X-Windows.
Microsoft Windows and Mac OS X were not affected.  This bug is
present in all earlier 7.4.x Condor releases.

\item The default path to \Prog{sftp-server} has been improved so that
  \Condor{ssh\_to\_job} can use \Prog{sftp} out-of-the-box on RHEL 5.

% gittrac #1383
\item A \Prog{nordugrid\_gahp} binary built on RHEL3 no longer crashes
when run on a RHEL4 or Scientific Linux 4 machine.

\end{itemize}

\noindent Known Bugs:

\begin{itemize}

\item None.

\end{itemize}

\noindent Additions and Changes to the Manual:

\begin{itemize}

\item None.

\end{itemize}


%%%%%%%%%%%%%%%%%%%%%%%%%%%%%%%%%%%%%%%%%%%%%%%%%%%%%%%%%%%%%%%%%%%%%%
\subsection*{\label{sec:New-7-4-2}Version 7.4.2}
%%%%%%%%%%%%%%%%%%%%%%%%%%%%%%%%%%%%%%%%%%%%%%%%%%%%%%%%%%%%%%%%%%%%%%

\noindent Release Notes:

\begin{itemize}

\item Condor version 7.4.2 released on April 6, 2010.

\end{itemize}


\noindent New Features:

\begin{itemize}

\item None.

\end{itemize}

\noindent Configuration Variable and ClassAd Attribute Additions and Changes:

\begin{itemize}

\item None.

\end{itemize}

\noindent Bugs Fixed:

\begin{itemize}

% gittrac 1217
\item Fixed a bug in which the \Condor{schedd} would sometimes negotiate
  for and try to run
  more jobs than specified by \MacroNI{MAX\_RUNNING\_JOBS}.  Once the
  jobs started running, it would then kill them off to get back below
  the limit.  This was more likely to happen with slow preemption
  caused by \MacroNI{MaxJobRetirementTime} or by a large timeout
  imposed by \MacroNI{KILL}.  This problem has existed since before
  Condor 6.5.  When this problem happened, the following message
  appeared in the \Condor{schedd} log:

\begin{verbatim}
Preempting X jobs due to MAX_JOBS_RUNNING change
\end{verbatim}

% gittrac 1250
\item Fixed a problem that caused \Condor{ssh\_to\_job} to fail to connect
to a job running on a slot with multiple '@' signs in its name.  This bug
has existed since the introduction of \Condor{ssh\_to\_job} in 7.3.2.

% gittrac 116
\item In all previous versions of Condor, \Condor{status} refused to
  accept \Opt{-long}, \Opt{-xml}, and \Opt{-format} when followed by
  an argument such as \Opt{-master} that specified which type of
  daemon to look at.  The order of the arguments had to be reversed or
  it would produce a message such as the following:

\begin{verbatim}
Error:  arg 4 (-master) contradicts arg 1 (-format)
\end{verbatim}

% gittrac #1201
\item Fixed a bug which caused the \Condor{master} to crash if
\MacroNI{VIEW\_SERVER} was included in \MacroNI{DAEMON\_LIST} and
\MacroNI{CONDOR\_VIEW\_HOST} was unset.

% gittrac #1196
\item Fixed a bug that caused configuration parameter
\MacroNI{LOCAL\_CONFIG\_DIR} to be ignored if it was set in a local
configuration file, as opposed to the top-level configuration file.

% gittrac #1202
\item Fixed a bug that could cause the \Condor{schedd} to behave
incorrectly when reading an invalid job queue log on startup.

% gittrac #1215
\item Fixed a bug that could corrupt the job queue log
if the \Condor{schedd} daemon's attempt to compact it fails.

% gittrac #1256
\item Fixed a problem that in rare cases caused the \Condor{schedd} to
crash shortly after the \Condor{gridmanager} exited.
This bug has existed since before Condor version 6.8.

% gittrac #1270
\item Fixed a problem that was resulting in messages such as the following:

\footnotesize
\begin{verbatim}
ERROR: receiving new UDP message but found a long message still waiting
to be closed (consumed=0). Closing it now.
\end{verbatim}
\normalsize

\item The file extension specified to \Condor{fetch\_log} can no longer
contain a path delimiter.

% gittrac 1299
\item When in graceful shutdown mode, the \Condor{schedd} was
  sometimes starting idle scheduler universe jobs.  With a large
  enough number of scheduler universe jobs, this could lead to a cycle
  of stopping and restarting jobs until the graceful shutdown time
  expired.

% gittrac #1259
\item Fixed multiple bugs that prevented Condor from building on or
  running correctly on OpenSolaris X86/64 version 2009.06.

% gittrac #1238
\item Fixed a bug which caused the \Condor{startd} to incorrectly
  count the number of processors on some machines with
  Hyper-threading enabled.  This bug was introduced in
  Condor version 7.3.2, and exists in 7.4.0 and 7.4.1.

% gittrac #1167
\item Fixed a problem with GSI authentication in Condor that would cause
daemons to consume more and more memory over time.  The biggest source
of trouble was introduced in Condor version 7.3.2.
However, a smaller memory leak that
existed in all previous versions of Condor has also been fixed.

% gitrack #553
\item Fixed a bug where if \Condor{compile} is invoked in a manner such as:
\begin{verbatim}
  condor_compile gcc -print-prog-name=ld 
\end{verbatim}
an error would be emitted,
and \Condor{compile} would exit with a bad exit code.

% gittrac #1093
\item The sort based on \Condor{status} output accidentally changed in 
Condor version 7.3,
so that the output was based on the slot name first, then machine name.
The behavior is now restored to the original sorting: first on machine name,
then slot name.

% gittrac #728
\item If one machine running a parallel job crashed,
and job leases are enabled (which they are by default),
the job would not exit until the job lease duration expired.
As the \Condor{starter} will not get respawned,
there is no need to wait.
Many sites set long job lease durations,
to prevent jobs from being killed when the machine running
the \Condor{schedd} daemon reboots.
Now, if one node goes away, the whole computation is shut down immediately.

\item Fixed the verbosity level of some \Condor{dagman} messages written to
the \File{dagman.out} file.

% gittrac #1137
\item Fixed a bug introduced in Condor version 7.3.2 that resulted in
  messages such as the following even in cases where no problem in
  communicating with the \Condor{collector} had been encountered:

\begin{verbatim}
Collector <X> is still being avoided if an alternative succeeds.
\end{verbatim}

This problem was believed to be fixed in Condor 7.4.1, but some cases
of the problem remained in that version.

% gittrac 1160
\item Fixed a bug from Condor version 6.1.14,
that resulted in the \Condor{schedd} performing
the operation scheduled via \MacroNI{WALL\_CLOCK\_CKPT\_INTERVAL} at the
specified frequency (default time of 1 hour),
multiplied by the number of times the
\Condor{schedd} daemon had been reconfigured during its lifetime.
This could lead to degraded performance,
especially prior to Condor version 7.4.1,
when this operation was more disk-intensive.

% gittrac 1184
% gittrac 1181
\item 32-bit Linux versions of Condor running in a 64-bit environment would
sometimes not detect the existence of some processes and sometimes
wrongly detect that a tracked process belonged to root when it
actually belonged to some other user.  This could lead to failure to run
jobs or failure to properly monitor and clean up after them.  When the wrong
process ownership problem happened,
the following message appeared in the \Condor{master} and/or \Condor{procd}
logs:

\begin{verbatim}
ProcAPI: fstat failed in /proc! (errno=75)
\end{verbatim}

If \Condor{procd} failed to detect the existence of its own parent process,
it would exit with the following message in its log:

\begin{verbatim}
ERROR: master has exited
\end{verbatim}

% gittrac 1186
\item Fixed a problem in the \Condor{job\_router} daemon,
  introduced in Condor version 7.2.2,
  that could cause the daemon to crash when failing to carry out the change
  of state dictated by a job's periodic policy expressions,
  for example, the failure to put a job on hold when \AdAttr{periodic\_hold}
  becomes \Expr{True}.

% gittrac #1209
\item Fixed a bug introduced in Condor 7.3.2 that caused Grid Monitor
jobs to receive a full X.509 proxy. Now, it always receives a limited
proxy, which was the previous behavior.

% gittrac #1070
\item Fixed a bug that could cause the nordugrid\_gahp to crash.

% gittrac #742
\item Fixed a problem introduced in 7.4.0 that could cause two 
  \Condor{schedd} daemons
  with a match to the same slot to both fail to claim it, rather than
  letting the first one to claim it succeed.  This sort of situation
  can happen when the \Condor{negotiator} has a stale view of the pool,
  either because the gap between negotiation cycles is configured to
  be shorter than usual, or because updates from the \Condor{startd}
  to the \Condor{collector}
  are not reliably delivered and processed.

% gittrac #1251
\item The \Condor{kbdd} is no longer ignored by the \Condor{startd}
when the configuration variable \Macro{CONSOLE\_DEVICES} is defined.

% gittrac #92
\item When using the \Opt{-d} command line argument with a daemon,
the values of \MacroNI{LOG}, \MacroNI{SPOOL}, and \MacroNI{EXECUTE}
no longer change every time a \Condor{reconfig} command is received.

\end{itemize}

\noindent Known Bugs:

\begin{itemize}

% gittrac #1337
\item The \Condor{kbdd} has a chance of entering an infinite loop
on platforms that use X-Windows.  Microsoft Windows and Mac OS X
are not affected.  Removing KBDD from \MacroNI{DAEMON\_LIST} is a
workaround, although this impairs Condor's ability to detect
console usage.  This bug is fixed in Condor version 7.4.3.

\end{itemize}

\noindent Additions and Changes to the Manual:

\begin{itemize}

\item Descriptions of all the commands that may be placed into a
submit description file are now located within the \Condor{submit}
manual page, instead of within Chapter 2, the Users' Manual.

\item An initial, but not yet complete set of configuration variables
that require a restart when changed,
is listed in section~\ref{sec:Macros-Requiring-Restart}.
Using \Condor{reconfig} to change these variables' values is not sufficient.

\end{itemize}


%%%%%%%%%%%%%%%%%%%%%%%%%%%%%%%%%%%%%%%%%%%%%%%%%%%%%%%%%%%%%%%%%%%%%%
\subsection*{\label{sec:New-7-4-1}Version 7.4.1}
%%%%%%%%%%%%%%%%%%%%%%%%%%%%%%%%%%%%%%%%%%%%%%%%%%%%%%%%%%%%%%%%%%%%%%

\noindent Release Notes:

\begin{itemize}

% gittrac #1018
\item \Security A flaw was found that could allow a user who already is authorized to
submit jobs into Condor, to queue a job under the guise of  a different
user.  In this way, someone who has access to a Condor submission
service and is allowed to submit jobs into Condor could gain access to
another non-root or non-administrator account on the system.
This flaw was discovered during the development process; no incidents
have been reported.  Details of the problem will be made available on Feb 1st,
2010.

% gittrac #918
\item The default value of \MacroNI{JOB\_ROUTER\_NAME} has changed
  from an empty string to \verb|jobrouter| in order to address
  problems caused by the previous default.  Without special handling,
  this means that jobs being managed by \Condor{job\_router} before
  upgrading will not be adopted by the new version of
  \Condor{job\_router} if the default \MacroNI{JOB\_ROUTER\_NAME} was
  being used.  To correct this, follow the instructions given in the
  description of \MacroNI{JOB\_ROUTER\_NAME} on
  page~\pageref{JobRouterName}.

\end{itemize}


\noindent New Features:

\begin{itemize}

% gittrac #921, #999
\item Condor allows submit files to specify an \SubmitCmd{IwdFlushNFSCache}
expression,
to control whether or not Condor tries to flush the NFS cache for 
a job's initial working directory on job completion.

% gittrac #929, #943
\item The new \Opt{-attributes} option to \Condor{status}
  explicitly specifies the attributes to be listed when using the
  \Opt{-xml} or \Opt{-long} options.

\end{itemize}

\noindent Configuration Variable and ClassAd Attribute Additions and Changes:

\begin{itemize}

% gittrac #161, #935, #936
\item New VOMS attributes have been introduced into the job ad to keep them
separate from the X509UserProxySubjectName.

\item The default for \MacroNI{JOB\_ROUTER\_NAME} has changed from an
  empty string to \verb|jobrouter|.  See the release notes for more
  information about upgrading from an old version.

\item The configuration variable \Macro{TCP\_FORWARDING\_HOST}
  has existed in Condor since version 7.0.0, but was not documented.
  See section~\ref{param:TcpForwardingHost} for documentation.

% gittrac #933
\item The new configuration variable \MacroNI{STARTD\_PER\_JOB\_HISTORY\_DIR}
allows ClassAds of completed jobs to be stored in a directory separate 
from the existing one specified with \MacroNI{PER\_JOB\_HISTORY\_DIR}.

\end{itemize}

\noindent Bugs Fixed:

\begin{itemize}

% gittrac #749
\item  Condor no longer creates the job sandbox in its \MacroNI{SPOOL}
directory if it is not needed.

% gittrac #1019
\item Fixed a problem introduced in Condor version 7.4.0 that caused GSI
authentication between Condor processes to fail with using a
non-legacy format X.509 proxy.

% gittrac #1028
\item Fixed a problem with CCB under Windows platforms that has existed since
Condor version 7.3.0.  
This problem caused CCB-enabled daemons to become unresponsive
after the exit of a child process.

% gittrac #931 -- Fixed minor spelling errors, not worthy of listing.

% gittrac #923
\item Improved the handling of previously-submitted gt2 grid jobs upon
release from hold, when there is no Globus job manager for the job running
on the remote resource.

% gittrac #453
\item Fixed a problem with job leases for jobs that use a \Condor{shadow}.
Previously, while these jobs were running, lease renewals from the 
submitter would not be
noticed, and the job would be aborted when the original lease expired.

% gittrac #870
\item Fixed a bug that only allowed approximately 50 splices to be included into
a DAG input file. There is now no limit to the number of splices
one may include into a DAG input file except, of course, for the
implicit memory allocation limit of the \Condor{dagman} process.

% gittrac #909
\item Removed attempted limiting of swap space via \Prog{ulimit -v} using the
\Attr{VirtualMemory} machine ClassAd attribute in the script
\File{condor\_limits\_wrapper.sh}.

% gittrac #899
\item Fixed a bug that caused \MacroNI{ALLOW\_CONFIG} and
  \MacroNI{HOSTALLOW\_CONFIG}, as well as the corresponding
  \MacroNI{DENY} configuration variables to incorrectly handle a
  setting consisting of a single \Expr{*} or the equivalent \Expr{*/*}.  This
  also fixes a bug that caused incorrect merging of \MacroNI{ALLOW}
  and \MacroNI{HOSTALLOW} settings when one, but not both, consisted of
  a single \Expr{*} or the equivalent \Expr{*/*}.
  These bugs have existed since before Condor version 6.8.

% gittrac #905
\item Fixed a bug introduced in Condor version 7.3.0 that could cause 
Condor daemons to crash when reading malformed network addresses.

% gittrac #883
\item Removed a check for root ownership of a script specified by
the configuration variable \MacroNI{VM\_SCRIPT}.

% gittrac #884
\item Fixed a bug in writing the header of the file identified by
the configuration variable \MacroNI{EVENT\_LOG}.

% gittrac #891
\item Fixed a bug that could cause the \Condor{startd} to segfault on shutdown
when using dynamic slots.

% gittrac #871
\item Fixed a problem introduced in Condor version 7.3.2 that changed 
  the behavior of
  an undocumented method for selecting attributes to be displayed in
  \Condor{q} \Opt{-xml}.  Prior to this bug, the following command
  would produce XML output with the attributes \Attr{A} and \Attr{B},
  plus a few other attributes that were always shown.

\begin{verbatim}
condor_q -xml -format "%s" A+B
\end{verbatim}

In Condor versions 7.3.2 and 7.4.0,
this same command produced an empty XML ClassAd.
The workaround was to use multiple \Opt{-format} options, each listing
just one desired attribute, rather than a single one with an
expression of all desired attributes.  Although this is now fixed, the
more straightforward way to select attributes since Condor version 7.3.2
is to use the \Opt{-attributes} option.

% gittrac #907
\item Fixed a bug introduced in Condor version 7.3.2 that resulted in 
  messages such
  as the following even in cases where no problem in communicating
  with the \Condor{collector} had been encountered:

\begin{verbatim}
Collector <X> is still being avoided if an alternative succeeds.
\end{verbatim}

% gittrac #859
\item Fixed a bug that has been in the \Condor{startd} since before
  Condor version 6.8.  If the \Condor{startd} ever failed to send signals to the
  \Condor{starter} process, it could fail to properly clean up the
  machine ClassAd, leaving attributes from
  \MacroNI{STARTD\_JOB\_EXPRS} in the ClassAd but not making them visible
  in \Condor{status} queries.  One possible problem resulting from
  this could be matches being made by the \Condor{negotiator} that are then
  rejected by the \Condor{startd}.  Repeated messages such as the following
  would then result in the \Condor{startd} log:

\begin{verbatim}
slot1: Request to claim resource refused.
\end{verbatim}

%gittrac #908
\item Fixed a problem that resulted in the following message in the
  \Condor{startd} log:

\begin{verbatim}
Timer -1 not found
\end{verbatim}

%gittrac #937
\item Fixed a problem in which security sessions were not cached
  correctly when using CCB.  This resulted in re-authentication in
  some cases where a cached security session could have been used.

% gittrac #161, #935, #936, #1020
\item Fixed multiple problems with the handling of VOMS attributes in GSI
proxies.

% gittrac #934
\item Fixed a bug that caused \Condor{dagman} to hang when running a
DAG with POST scripts, if the global event log is turned on.

% gittrac #973
\item Improved how the private network address is published when using
  the configuration variables \MacroNI{PRIVATE\_NETWORK\_NAME} and
  \MacroNI{PRIVATE\_NETWORK\_INTERFACE}.  In some cases, this
  information was not being used, and therefore connections were made
  to the public address when they could have been made to the private
  address.

% gittrac #801
\item Fixed a bug exhibited under Windows XP,
where using \MacroNI{USE\_VISIBLE\_DESKTOP}
would cause strange behavior after a job completed.

% gittrac #713
\item CCB now works with \MacroNI{TCP\_FORWARDING\_HOST}.  Previously,
  the reverse connection was made to the private address rather than
  to the host defined by \MacroNI{TCP\_FORWARDING\_HOST}.

% gittrac #852
\item Removed a bad optimization that caused some information about job
execution to be lost during job completion or removal,
if a history file was not configured.

% gittrac #893
\item Condor now checks whether the configuration variable
\MacroNI{GRIDFTP\_URL\_BASE} is set before
submitting cream grid jobs, as that variable is required for cream jobs
to function properly. If the variable is not set, cream jobs are put on
hold with an appropriate message.

% gittrac #920
\item Fixed a bug that allowed running virtual machines to be leaked
if the \Condor{startd} crashed.

% gittrac #912
\item Fixed a bug in \Prog{cream\_gahp} which could cause crashes when
there were more than 500 cream jobs queued.

% gittrac #972
\item Improved recovery when Condor crashes during the submission of a cream
grid job. Before, affected jobs would remain in \Expr{REGISTERED} state
on the cream server, but never run.

% gittrac #954
\item Improved the \Attr{HoldReason} message when cream grid jobs are
held by the \Condor{gridmanager}.

% gittrac #895
\item When naming a resource for a cream grid job, Condor now properly
recognizes the format used by the standard cream client UI:
\File{https://foo.edu:8443/cream-pbs-cream\_queue}.

% gittrac #795
% The memory leak is not worth documenting.
\item The configuration variable \MacroNI{SOAP\_SSL\_CA\_FILE} is now 
consulted in addition to
\MacroNI{SOAP\_SSL\_CA\_DIR} when authenticating
an https proxy for Amazon EC2, when \MacroNI{AMAZON\_HTTP\_PROXY} is defined.

% gittrac #485
\item Previously, if \Condor{rm} and friends were given both a constraint
and a user name or cluster id, they would act on all jobs matching the
constraint and all jobs associated with the user or cluster. Now, this
combination of arguments results in an error.

% gittrac #1062
\item Failure to purge a cream grid universe job from the remote server
because it was previously purged no longer results in the job being held.

% gittrac #1044
\item The \Condor{gridmanager} now recognizes VOMS attributes in X.509
proxies and will handle them appropriately. For example, it recognizes
that two proxies with the same identity but different VOMS attributes may
be mapped to different accounts on a remote machine.

% gittrac #947
% Not documenting, as the parameter being removed was added for a specifc
% customer and never documented.

% gittrac #932
% Not documenting, as the bug hasn't caused any problems.

% gittrac #1043
% Not documenting, as the problem never made it into a release.

% gittrac #979
\item Fixed a bug in \Condor{dagman}, introduced in 7.3.2, that will
cause \Condor{dagman} running on Windows to hang on any DAG using
more than one log file for the node jobs.

% gittrac #967
\item Fixed a bug in \Condor{dagman}, introduced in 7.3.2, that could
cause \Condor{dagman} to fail on a DAG using node job log files on
multiple devices, if log files on different devices happened to have
the same inode number.

% gittrac #981
\item Fixed a bug that caused the \Condor{schedd} daemon to segfault when
spooling more than 9 files.

% gittrac #1011
\item Fixed a bug that caused the \Condor{startd} daemon to crash on
Debian Stable.

% gittrac #1033
\item Fixed keyboard activity detection on the Windows XP platform.

% gittrac #1068
\item Fixed a bug in the \Condor{had} daemon that caused it to not start
the controlled daemon if CCB was enabled.

\end{itemize}

\noindent Known Bugs:

\begin{itemize}

% gittrac #1337
\item The \Condor{kbdd} has a chance of entering an infinite loop
on platforms that use X-Windows.  Microsoft Windows and Mac OS X
are not affected.  Removing KBDD from \MacroNI{DAEMON\_LIST} is a
workaround, although this impairs Condor's ability to detect
console usage.  This bug is fixed in Condor version 7.4.3.

% gittrac #983
\item \Condor{dagman} may fail on Windows if the set of node job log
file names includes multiple paths that are hard links (not symbolic links)
to the same file.

% gittrac #1081
\item \Condor{dagman} PRE and POST script arguments (and the names of
the scripts themselves) cannot contain spaces.

% gittrac #1082
\item \Condor{dagman} VARS values cannot contain single quotes.

\end{itemize}

\noindent Additions and Changes to the Manual:

\begin{itemize}

% gittrac #725
\item Added documentation about how to include spaces (and other
special characters) in \Condor{dagman} VARS values.

\end{itemize}


%%%%%%%%%%%%%%%%%%%%%%%%%%%%%%%%%%%%%%%%%%%%%%%%%%%%%%%%%%%%%%%%%%%%%%
\subsection*{\label{sec:New-7-4-0}Version 7.4.0}
%%%%%%%%%%%%%%%%%%%%%%%%%%%%%%%%%%%%%%%%%%%%%%%%%%%%%%%%%%%%%%%%%%%%%%

\noindent Release Notes:

\begin{itemize}

\item The default configuration file within the release now uses
  \MacroNI{ALLOW}/\MacroNI{DENY} in place of
  \MacroNI{HOSTALLOW}/\MacroNI{HOSTDENY} for security related settings.
  We recommend making this
  same change throughout all configuration files.  That way,
  a policy that depends on the default policy should continue to
  work as it did before.  The behavior of these configuration variables
  remains unchanged.  The \MacroNI{ALLOW}/\MacroNI{DENY} lists are
  added to the \MacroNI{HOSTALLOW}/\MacroNI{HOSTDENY} lists to form the
  authorization policy.  Both styles support the same syntax.  
  This change permits an anticipated
  phasing out of the \MacroNI{HOSTALLOW}/\MacroNI{HOSTDENY}  configuration
  variables, in order to simplify configuration.

\item As of Condor version 7.3.2, \Condor{q} \Opt{-xml} output no longer 
  begins with the non-XML consisting of two blank lines followed by a line
  of the following form:

\begin{verbatim}
-- Submitter: schedd-name : <IP> : hostname
\end{verbatim}

\item All \Prog{Stork} data placement is now supported by the Stork
project at the 
LSU Center for Computation and Technology
(\URL{http://www.cct.lsu.edu/www.cct.lsu.edu}).
Please see the home page of the Stork project at
\URL{http://www.cct.lsu.edu/~kosar/stork/index.php} for details and
software.

\end{itemize}


\noindent New Features:

\begin{itemize}

\item Condor is now integrated with the Hadoop Distributed File System (HDFS). 
See documentation in section~\ref{sec:Condor-HDFS} and 
section~\ref{sec:HDFS-Config-File-Entries}.

% commit af65de7ccc1a281c2b05b8f68ac70bcfa56b2dd1
\item \Condor{q} using the options \Opt{-analyze} and \Opt{-better-analyze}
  now provide analysis for scheduler and local universe jobs.
  Specifically, the \MacroNI{START\_SCHEDULER\_UNIVERSE} and
  \MacroNI{START\_LOCAL\_UNIVERSE} expressions are checked.

% #824
\item Added the ClassAd attributes
\Attr{TotalLocalRunningJobs}, \Attr{TotalLocalIdleJobs},
\Attr{TotalSchedulerRunningJobs}, and \Attr{TotalSchedulerIdleJobs}
to the published ClassAd for the \Condor{schedd}.  This means that
\Condor{q} \Opt{-analyze} can still give helpful information about
why local or scheduler universe jobs are idle when
the configuration variables \MacroNI{START\_LOCAL\_UNIVERSE} or
\MacroNI{START\_SCHEDULER\_UNIVERSE} refer to these attributes.
These attributes were already present internally within 
the \Condor{schedd} daemon, 
just not published.

% #688
\item The \Condor{vm-gahp} now supports KVM and links with libvirt, rather 
than calling virsh command-line tools.

% #760 #771 #769 #772 #773 #775
\item Greatly improved the \Condor{gridmanager}'s scalability when handling
many grid type gt2 grid universe jobs.  Improvements include more quickly
processing updated X.509 certificates, not checking jobs for status updates if 
they have not been submitted to the remote site, and eliminating unnecessary 
updates to the \Condor{schedd} daemon.

% commit 75f6b2fe920b88717712a0d41765d16665ad7fe6
\item Latency in the submission and cleaning up of Condor-C jobs
has been improved by changing the default value of
\Macro{C\_GAHP\_CONTACT\_SCHEDD\_DELAY} from 20 to 5.

% commit 8c2d88c695d6981be3bdab7e10c5d92e9f6bb55b
\item The \Expr{eval()} ClassAd function added in Condor version 7.3.2
is now also understood by the \Condor{job\_router} and
\Condor{q} using the \Opt{-better-analyze} option.

\item The submit command \SubmitCmd{run\_as\_owner} is now supported
for Unix platforms. Previously, it was only supported on Windows platforms.

% #795
\item When setting \MacroNI{AMAZON\_HTTP\_PROXY}, a username and password
can now be given as part of the proxy URL.
The value of \MacroNI{SOAP\_SSL\_CA\_DIR} is now consulted when authenticating
an https proxy for Amazon EC2, when \MacroNI{AMAZON\_HTTP\_PROXY} is defined.

% #694
\item The \Condor{collector} daemon now advertises to itself, and will appear
in the output of \Condor{status} \Opt{-collector}.

% #775, cf02764d9d0fdd2b36ef3629f862f856ee41a717, and more
\item Optimizations in core Condor systems should provide minor speed 
improvements.

% 823
\item Updated the maximum log size to the maximum operating system's file size.

\end{itemize}

\noindent Configuration Variable and ClassAd Attribute Additions and Changes:

\begin{itemize}

% commit 0e8800c201f81eac54cba925b3d7f6d81a83aeca
\item The undocumented configuration variable 
  \Macro{TOOLS\_PROVIDE\_OLD\_MESSAGES} is no longer recognized by Condor.

% #768
\item The new configuration variable 
  \Macro{SCHEDD\_JOB\_QUEUE\_LOG\_FLUSH\_DELAY} sets an
  upper bound in seconds on how long it takes for changes to the job
  ClassAd to be visible to the Condor Job Router and to Quill.
  The default value is 5 seconds.
  Previously, there was no upper limit.  Typically, other activity in
  the job queue, such as jobs being submitted or completed would cause
  buffered data to be flushed to disk, such that the effective upper bound was
  a function of how busy the job queue was.

% commit 55525e0a338be8b2ba2d9173ce832e43d05413c3
\item The default configuration file now uses
  \MacroNI{ALLOW}/\MacroNI{DENY} in place of
  \MacroNI{HOSTALLOW}/\MacroNI{HOSTDENY}.  See the release notes above
  for more information.

% commit 7199e217f9228082a8465b85aaee18c2ebb19787
\item The default value for \Macro{MAX\_JOBS\_RUNNING} has changed.
  Previously, it was 200.  Now it is defined by an expression that depends 
  on the total amount of memory and the operating system.  The default
  expression requires 1MByte of RAM per running job, on the submit machine.
  In some environments and configurations, this is overly
  generous and can be cut by as much as 50\%.  Under Windows, the
  number of running jobs is still capped at 200.
  A 64-bit version of Windows  is recommended in order to raise the value
  above the default.
  Under Unix, the maximum default is now 10,000.  To scale higher, we
  recommend that the system ephemeral port range is extended
  such that there are at least 2.1 ports per running job.

% #767 commit 18296bfdfa92f16684a73d8d57a54d231b48dc33
\item The default value of \MacroNI{RESERVED\_SWAP} has changed to
  the value 0, which
  disables the \Condor{schedd} daemon's check for sufficient swap space
  before starting more jobs.  The new expression defined with 
  \MacroNI{MAX\_JOBS\_RUNNING} has a more appropriate memory check, so
  the configuration variable \MacroNI{RESERVED\_SWAP} will no longer
  be used in the near future.
  For cases where 
  \MacroNI{RESERVED\_SWAP} is not set to 0, the default value
  of \MacroNI{SHADOW\_SIZE\_ESTIMATE} has changed to 800 Kbytes.
  Previously, it was 200 if not set,
  but it was set to 1800 in the example configuration file.

% #767 commit c80e8a40e67ef4faa4e2b32b3671877eae1e1a19
\item The default values of \Macro{START\_LOCAL\_UNIVERSE} and
  \Macro{START\_SCHEDULER\_UNIVERSE} have changed.  Previously,
  these were set to \Expr{True}.  Now, they are set using an expression
  that allows
  up to 200 local universe and 200 scheduler universe jobs to run.

% #767 commit c4f4d911a808e1bdb18552e1cdeb0407b6344969
\item The default value of
  \Macro{GRIDMANAGER\_MAX\_SUBMITTED\_JOBS\_PER\_RESOURCE} has
  changed from 100 to 1000.

% #767 commit 9e6dfa463c71c28c8dc2c0c0c215b51d6762e811
% commit b4fd08ad1a8c69da24c371565796ef73522a61fc
\item The default value of \Macro{NEGOTIATOR\_INTERVAL}
   has changed from 300 to 60.

% #767 commit 8b91877ec8186810887402e1dd1df07b6341ade7
% Probably at least one other commit
\item The default value of \Macro{ENABLE\_GRID\_MONITOR} has been
  changed from \Expr{False} to \Expr{True}.  This variable
  was assigned to \Expr{True} in the example configuration file, so
  the change in default value now matches the value set in the example
  configuration.

% #631
\item The configuration variable \MacroNI{VM\_VERSION} has been removed,
as has the machine ClassAd attribute of the same name.
When the virtual machine version information is needed in the machine ClassAd,
the configuration variable \MacroNI{STARTD\_ATTRS} can be used to
add it.
 
% #861
\item The default configuration now uses
  \MacroNI{VM\_BRIDGE\_SCRIPT} and \MacroNI{VM\_SCRIPT} in place of
  \MacroNI{XEN\_BRIDGE\_SCRIPT} and \MacroNI{XEN\_SCRIPT} due to the
  support of KVM. 
  Submit description file commands have also been added, and they include:
  \SubmitCmd{kvm\_disk}, \SubmitCmd{kvm\_transfer\_files},
   and \SubmitCmd{kvm\_cd\_rom\_device}.

% #872
\item The configuration variables \MacroNI{XEN\_DEFAULT\_KERNEL}
  and \MacroNI{XEN\_DEFAULT\_INITRD} have been removed.
  Corresponding to this, the submit description file command
  \Expr{xen\_kernel = any} is no longer valid.

\end{itemize}

\noindent Bugs Fixed:

\begin{itemize}

\item Fixed a bug that prevented parallel universe jobs from running 
  on \Condor{startd} daemons with dynamic slots enabled.

% #706
\item Fixed a race condition bug in the \Condor{startd} which allowed
it to send Unix signals, intended for \Condor{starter} processes, as
root to non-Condor related processes.

% 735
\item A Windows platform bug has been fixed.
The bug caused a 20-second interval in which
the \Condor{shadow}, \Condor{startd}, and \Condor{starter} daemons
appeared as deadlocked. 
The bug was visible if a job ClassAd update from the \Condor{starter} caused
the job's periodic hold or remove policy to become \Expr{True}.

%gittrac #622
\item Fixed a bug that could cause \Condor{dagman} to generate an
illegal rescue DAG, if it read events incorrectly in recovery mode.
\Condor{dagman} now checks for events that violate DAG semantics
when reading events in recovery mode, and it exits without creating a
rescue DAG if it reads such an event.

% gittrac #744
\item Fixed a bug that could cause \Condor{dagman} to abort if it saw
the combination of a terminated event and an aborted event on a node with
retries.

% commit 5039a08cf00b0d0fafcc3fd8241518d1854ec3a1
\item Changed some logged warnings in \Condor{dagman} to not be
printed at the default verbosity setting.

% gittrac #825
\item The version compatibility checking between a \File{.condor.sub}
file and the \Condor{dagman} binary which is done at DAG startup
is now much more permissive.
Currently, \File{.condor.sub} files with
Condor versions of 7.1.2 and later accepted by \Condor{dagman}.

% gittrac #851
\item Fixed a bug introduced with the new \Condor{dagman} lazy log file
evaluation code in Condor version 7.3.2.
The bug sometimes caused failure when running rescue DAGs.

% #211 commit d6c0144739000523e94205a192be3cf9afe9ca9f
\item Fixed a bug originating in Condor version 7.1.4.
When a user submitted a job
with an executable that did not have execute permission enabled,
Condor was running as root, and file transfer was specified in the job,
Condor failed to automatically turn on execute permission after
transferring the file.

% commit 3bb847691bfda4f26d2f570bed1a412fb3afb439
\item Fixed a bug that appeared in Condor version 7.3.2.
The configuration variable
\MacroNI{COUNT\_HYPERTHREAD\_CPUS} was ignored and was effectively
treated as \Expr{False} in all cases.

% #761
\item Fixed a bug in which the Condor Job Router was not able
to see matchmaking diagnostic attributes such as \Attr{LastRejMatchTime}.
Therefore, when evaluating policy
expressions that referred to these attributes, they were effectively
treated as though \Expr{Undefined}.
Quill was also not able to see these attributes.

% #822
\item Fixed a bug introduced in Condor version 7.3.2 that could cause the
\Condor{gridmanager} to crash repeatedly on startup,
if the job queue
contained grid type gt2 jobs that had been previously submitted.

% #724, #774, #786
\item Fixed two bugs introduced in Condor version 7.3.2,
and related to VOMS. 
The first bug
prevented jobs with X.509 proxies from being submitted on platforms
on which Condor does not support VOMS.
The second bug prevented submission
of jobs with VOMS proxies, if the authenticity of the VOMS extensions
could not be verified.
At the same time, improved memory usage when VOMS extensions are not used.

\item Fixed a bad default in the file \File{batch\_gahp.config},
that prevented
Condor from observing job state changes for grid universe jobs
with a grid type of pbs or lsf.

% #748
\item Fixed a bug that caused Condor-C jobs to fail if
the submit description file command \SubmitCmd{transfer\_executable}
was set to \Expr{False}.

% #784
\item Fixed a bug that caused Condor-C jobs to fail if the executable
or one of the \File{stdin}, \File{stdout}, or \File{stderr} file names
contained a comma.

% #460
\item File transfer for grid type gt4 jobs requires an empty directory
within \File{/tmp}, which the \Condor{gridmanager} creates. 
If this directory is deleted, the \Condor{gridmanager} will now recreate it.

%gittrac #790
\item Fixed a bug that could cause the user job log to become
  corrupted on Windows platforms.  This bug would manifest itself only if the
  same log file was specified with different paths.  For example, the
  following submit file could have triggered this bug:
\begin{verbatim}
...
initialdir = /data/job1
log = ../JobLog
queue

initialdir = /data/job2
log = ../JobLog
queue
\end{verbatim}


% commit a26fcd9fe54cd3920fe777d5d8e0b2ffefbc3b1f
\item Fixed a memory leak introduced into Condor version 7.3.2.
The leak was in the \Condor{collector} daemon.

% commit 1663b7e183e6bf1df8152af98d9387412c2ae146
\item Fixed a bug introduced in Condor version 7.3.2
that resulted in the \Condor{negotiator} daemon
refusing to run, if the configuration variable \MacroNI{GROUP\_QUOTA}
for any group was set to 0.

% gittrac #731
\item Fixed a bug that caused the \Code{ctime} in the event log header
  to always be zero.

% #862 commit 9a432e2f3497e5dce120db5c733e79212257f6a5
\item Fixed the output of \Condor{status} when used with the command-line
  options \Opt{-java} or \Opt{-vm}.

\item Fixed a problem in the \Condor{schedd} daemon introduced in
  7.3.2.  For \Condor{schedd} daemons with lots of jobs having periodic release
  expressions, this bug could result in the \Condor{schedd} taking a long
  time while evaluating periodic expressions, causing it to become
  unresponsive to queries and other tasks.
  With a job queue of 30,000 jobs,
  a period of unresponsiveness of an hour was observed,
  whereas the evaluation of periodic expressions in this same environment
  normally takes less than 5 seconds.

\item Potential bugs and memory leaks were identified and 
fixed throughout Condor.  The Condor Team is not aware of anyone having 
encountered these bugs.

% #692 commit 8bc6bb4e06f11b2fdca28214d98c68c34c0ab9a4
\item The \Condor{starter} cleans up working directories in more
situations.  Previously during some error conditions, the working
directory within \MacroUNI{EXECUTE} might be left behind.

% #692 commit 8bc6bb4e06f11b2fdca28214d98c68c34c0ab9a4
\item If the user log cannot be accessed when a local universe
job starts, the job would fail and immediately be retried.  Now
the job is placed on hold.

% 826 
\item Fixed a bug in the \Condor{startd} in which vacating jobs would not 
respect the value of \Attr{JobLeaseDuration}.

% 802
\item Updated the detection of \Attr{HasVM} within the \Condor{startd}
 to publish an update to the \Condor{collector},
 when the configuration variable \MacroNI{VM\_RECHECK\_INTERVAL} is specified.

% commit 68f06088fa36eb0eb332a4f72a5c48ccd48b1d5a
\item Fixed a bug in which the \Condor{gridmanager} could, in rare cases,
waste a
small amount of memory and processor time checking for proxy files no longer
being used by any active jobs.

% commit bc66aa432e1f4e69d88a5b769204a4fce0648bfc
\item The setting \Macro{CREAM\_GAHP} was added to the default configuration 
file with a value of \File{\$(SBIN)/cream\_gahp}.
Existing installations desiring to 
submit jobs to CREAM should add this setting.

% #702
\item Fixed a bug where \Condor{restart} would fail on a \Condor{collector}
daemon configured for high availability with multiple \Condor{collector}
daemons.

% commit f44a68fb351e528ea5b251dd2c3cf9767b0c1fba
\item Fixed a bug in which multiple entries of output from 
the command
\Condor{status} \Opt{-negotiator}
would be on a single line.  They are now listed one per line.

% #778
\item Fixed a bug in which the command
\Condor{submit} \Opt{-dump} would crash if multiple
jobs were queued from within a single submit file.

% #742
\item Fixed a bug in which a slot whose associated job disappeared
could remain in the Claimed/Idle state until the claim lease expired.
The slot should now promptly return to the Unclaimed/Idle state.

% commit 0d5e3ad8fc85f0cd0dc58f73b503c76c0ad49bc4
\item Fixed a bug in which a \Condor{startd} using dynamic slots could
crash on shutdown or reconfiguration.



\end{itemize}

\noindent Known Bugs:

\begin{itemize}

% gittrac #1337
\item The \Condor{kbdd} has a chance of entering an infinite loop
on platforms that use X-Windows.  Microsoft Windows and Mac OS X
are not affected.  Removing KBDD from \MacroNI{DAEMON\_LIST} is a
workaround, although this impairs Condor's ability to detect
console usage.  This bug is fixed in Condor version 7.4.3.

% gittrac #161, #935, #936, #1020
\item There are multiple bugs related to using VOMS attributes.
In Condor version 7.4.0, VOMS support should be disabled by setting
the configuration variable \Expr{USE\_VOMS\_ATTRIBUTES = FALSE}.

\item A configuration variable of  \Macro{USE\_VISIBLE\_DESKTOP} set 
to \Expr{True} will corrupt the visible desktop.
  This bug is present back through Condor version 7.2.4.
This configuration variable did not work at all in 7.2 releases
prior to 7.2.4.  This bug will be fixed in Condor version 7.4.1.

% gittrac #934
\item If the global event log (see section~\ref{param:EventLog}) is
turned on, \Condor{dagman} will hang when running any DAG that has
POST scripts.

% gittrac #979
\item \Condor{dagman} will hang on Windows when running any DAG that
uses more than one log file for the node jobs.

\end{itemize}

\noindent Additions and Changes to the Manual:

\begin{itemize}

\item See section~\ref{sec:Condor-HDFS} and 
section~\ref{sec:HDFS-Config-File-Entries} for preliminary documentation of
Condor's integration with the Hadoop Distributed File System (HDFS). 

\end{itemize}


% as of April 2011, Karen no longer wants to include these older
% version histories with the 7.6 and beyond manuals.
%%%%      PLEASE RUN A SPELL CHECKER BEFORE COMMITTING YOUR CHANGES!
%%%      PLEASE RUN A SPELL CHECKER BEFORE COMMITTING YOUR CHANGES!
%%%      PLEASE RUN A SPELL CHECKER BEFORE COMMITTING YOUR CHANGES!
%%%      PLEASE RUN A SPELL CHECKER BEFORE COMMITTING YOUR CHANGES!
%%%      PLEASE RUN A SPELL CHECKER BEFORE COMMITTING YOUR CHANGES!

%%%%%%%%%%%%%%%%%%%%%%%%%%%%%%%%%%%%%%%%%%%%%%%%%%%%%%%%%%%%%%%%%%%%%%
\section{\label{sec:History-7-3}Development Release Series 7.3}
%%%%%%%%%%%%%%%%%%%%%%%%%%%%%%%%%%%%%%%%%%%%%%%%%%%%%%%%%%%%%%%%%%%%%%

This is the development release series of Condor.
The details of each version are described below.

%%%%%%%%%%%%%%%%%%%%%%%%%%%%%%%%%%%%%%%%%%%%%%%%%%%%%%%%%%%%%%%%%%%%%%
\subsection*{\label{sec:New-7-3-1}Version 7.3.1}
%%%%%%%%%%%%%%%%%%%%%%%%%%%%%%%%%%%%%%%%%%%%%%%%%%%%%%%%%%%%%%%%%%%%%%

\noindent Release Notes:

\begin{itemize}

\item None.

\end{itemize}


\noindent New Features:

\begin{itemize}

\item Added the STARTD\_HISTORY configuration parameter.  If set, this
is a pathname to a history file, just like the condor\_schedd maintains,
but only for jobs run on that startd.

\item Added the JavaSpecificationVersion attribute to startds which
support Java.  This allows users to request machines which support
a particular major version of Java, without specifying the exact
specific version.  So, Java versions 1.6.0\_01, 1.6.1\_02 and 1.6.2\_03
all advertise JavaSpecificationVersion of 1.6.

\item Implemented a performance increase to \Condor{dagman} which can
decrease the parsing times of DAG input files by up to 60 times.
This performance increase works for certain common DAG geometries.
This will help in submission and recovery
time for DAGs whose nodes have a very large number of dependency edges
associated with them.

\item \Condor{q} -analyze and -better-analyze now emit warnings
if the \Condor{schedd} will not run jobs when it is out of swap space or
has hit the limit imposed by the configuration variable
\MacroNI{MAX\_JOBS\_RUNNING}.

\item When matching Condor-G jobs to resources, if multiple jobs
match multiple resources, and every job has identical job rank, the
matchmaker would always fill up one particular resource first.  Now,
the resources will be matched in a round robin fashion.  This can be
overridden by setting job rank appropriately.

\item Made the \Condor{schedd} more efficient in how it stores
information about \verb@$$()@ expansions in the job ClassAd.
Also made the \Condor{schedd} more efficient in how it contacts
the \Condor{negotiator} to submit reschedule requests.

\item Improved the Job Router's heuristic for site throttle adjustment.  It
is now quicker to release the throttle when the failure rate drops
below the configured threshold.

\item Made the Job Router more efficient on startup by improving the way it
reads the job queue log file.

\item Added an accessor class to the user log reader API to allow the
  application to query about reader state, including the
  difference in the event numbers and log position of two states.  This
  can be used by the application to determine the number of events
  missed when missed events are detected.

\item Added the ability to throttle the rate at which jobs are
stopped via \Condor{rm}, \Condor{hold}, \Condor{vacate\_job},
and during a graceful shutdown of the \Condor{schedd} daemon.

\item In the configuration file, Condor now accepts expressions for
the values of configuration variables that are required to be 
numeric literals or boolean constants.  
Note that this does not imply that the
expressions may freely reference ClassAd values in places where they
could not before.  
See section~\ref{sec:Intro-to-Config-Files} for an example with
further explanation.

\end{itemize}

\noindent Configuration Variable Additions and Changes:

\begin{itemize}

\item The new configuration variable \Macro{UPDATE\_OFFSET} 
  causes the \Condor{startd} to
  delay the initial (and all further) updates that it sends to the
  \Condor{collector}.  See \ref{param:UpdateOffset} for more details.

\item The new configuration variables
  \Macro{JOB\_STOP\_COUNT} and \Macro{JOB\_STOP\_DELAY}
  limit the rate at which jobs are stopped via \Condor{rm},
  \Condor{hold}, \Condor{vacate\_job}, and during a graceful shutdown of
  the \Condor{schedd} daemon.
  See \ref{param:JobStopCount} and \ref{param:JobStopDelay} 
  for full definitions.

\end{itemize}

\noindent Bugs Fixed:

\begin{itemize}

\item Fixed a problem with job removal in the local universe that 
  would cause spurious error messages to be written to the log of the
  \Condor{schedd} daemon.

\item The \Condor{schedd} was failing to send `reschedule' commands to
flocked negotiators, so unless some other schedd in the negotiator's
pool sent it a reschedule command, negotiation cycles would only
happen every \Macro{NEGOTIATOR\_INTERVAL}.

\end{itemize}

\noindent Known Bugs:

\begin{itemize}

\item When using CCB to connect to other Condor daemons, Condor 7.3.1
daemons can sometimes consume large amounts of CPU, potentially
causing performance problems.  Condor 7.3.0 did not suffer from this
problem.

\end{itemize}

\noindent Additions and Changes to the Manual:

\begin{itemize}

\item None.

\end{itemize}

%%%%%%%%%%%%%%%%%%%%%%%%%%%%%%%%%%%%%%%%%%%%%%%%%%%%%%%%%%%%%%%%%%%%%%
\subsection*{\label{sec:New-7-3-0}Version 7.3.0}
%%%%%%%%%%%%%%%%%%%%%%%%%%%%%%%%%%%%%%%%%%%%%%%%%%%%%%%%%%%%%%%%%%%%%%

\noindent Release Notes:

\begin{itemize}

\item This release is incompatible when communicating with
previous versions of Condor if CCB is enabled or if
\Macro{PRIVATE\_NETWORK\_NAME} is configured.

\item Updated the DRMAA version.
This new version is compliant with GFD.133,
the DRMAA 1.0 grid recommendation standard.
Three new functions were added to meet the specification's requirements,
and several bugs were fixed.

\end{itemize}


\noindent New Features:

\begin{itemize}

\item Added support for using any recognized script as an executable
in a submit file on Windows. For more information please see
section~\ref{sec:windows-scripts-as-executables} on
page~\pageref{sec:windows-scripts-as-executables}.

\item Improved support for private networks:
Added CCB, the Condor Connection Broker.  It is similar in
functionality to GCB, the Generic Connection Broker, but it has
several advantages, including ease of use and working on Windows as
well as Unix platforms.
GCB continues to work, but we may remove
it some time in the 7.3 development series.  The main missing feature
in CCB at the moment that prevents it from replacing GCB,
is support for connectivity from one private network to another.
CCB only works
when connecting from a public network to a private one.  For example,
jobs may be sent from a \Condor{schedd} on the public Internet to 
\Condor{startd} daemons on a
private network, if the \Condor{startd} daemons are configured
to use a CCB server that is accessible to the \Condor{schedd} daemon.
However, if the \Condor{schedd} daemon is on one private
network and the \Condor{startd} daemons are on a different private network,
CCB does not help.  For more information on CCB, see section~ \ref{sec:CCB}.

\item Added support for a CPU affinity on both Windows and Linux platforms.

\item Added support for the \Condor{q} \Opt{-better-analyze} option on Windows.

\item Added \MacroNI{WANT\_HOLD}.  When \MacroNI{PREEMPT} becomes
true, if \MacroNI{WANT\_HOLD} is true, the job is put on hold for the
reason (optionally) specified by \MacroNI{WANT\_HOLD\_REASON} and
\MacroNI{WANT\_HOLD\_SUBCODE}.  These policy expressions are evaluated
by the execute machine.  As usual, the job owner may specify
\AdAttr{periodic\_release} and/or \AdAttr{periodic\_remove}
expressions to react to specific hold states automatically.

\item Added the ClassAd function \Procedure{debug}.
See section~ \ref{sec:classadFunctions} for the details of this function.

% Commented out by Karen, as this is useless to a user taking the
% time to read a version history. More info is needed. 
%\item Log messages have been made more clear.
% Includes: Give a clear warning instead of a terse error, when lacking a COLLECTOR.

\item The \Condor{schedd} can now use MD5 check sums to avoid storing
multiple copies of the same executable in its \Macro{SPOOL} directory.
Note that this feature only affects executables sent to the
\Condor{schedd} via the \SubmitCmd{copy\_to\_spool} command within
a submit description file.

% gittrac #197
\item Reduced the number of sleeps \Condor{dagman} does to maintain log
file consistency when a DAG uses multiple user logs for node jobs.
DAGMan now does one sleep per submit cycle,
instead of one sleep for each submit.

% gittrac #166, #208
\item Added the \Opt{-import\_env} command-line flag to
\Condor{submit\_dag}.  This explicitly puts the submittor's environment
into the \File{.condor.sub} file.

\item Optimized the removal of large numbers of jobs.  
Previously, removal of tens of thousands of jobs caused the
\Condor{schedd} daemon to consume
a lot of CPU time for several minutes.

\item Reduced memory usage by the \Condor{shadow} daemon.  Since there is one
\Condor{shadow} process per running job, this helps increase the
number of running jobs that a submit machine can handle.  Under Linux 2.6,
we found that running 10,000 jobs from a single submit machine
requires about 10GBytes of system RAM.  We also found in this case that to
run more than 10,000 simultaneous jobs requires a 64-bit submit
machine.  On a 32-bit Linux platform, kernel memory is exhausted,
regardless of how much additional RAM the system has.

\item Reduced the memory usage of the \Condor{collector} daemon,
when \Expr{UPDATE\_COLLECTOR\_WITH\_TCP = True}.

\end{itemize}

\noindent Configuration Variable Additions and Changes:

\begin{itemize}

\item The new configuration variable \Macro{OPEN\_VERB\_FOR\_<EXT>\_FILES}
allows the default interpreter for scripts with an extension \textit{EXT} to
be changed.  For more information please see
section~\ref{sec:windows-scripts-as-executables} on
page~\pageref{sec:windows-scripts-as-executables}.

\item The new configuration variable \Macro{CCB\_ADDRESS}
configures a daemon to use one or more
CCB servers to allow communication with Condor components outside of
the private network.  See page~\pageref{sec:CCB}.

\item The new configuration variable \Macro{MAX\_FILE\_DESCRIPTORS}
(on Unix platforms only) specifies the
required file descriptor limit for a Condor daemon.  File descriptors
are a system resource used for open files and for network connections.
Condor daemons that make many simultaneous network connections may
require an increased number of file descriptors.  For example, see
page~\pageref{sec:CCB} for information on file descriptor requirements
of CCB.

\item The new configuration variables \Macro{ENFORCE\_CPU\_AFFINITY} and 
\Macro{SLOTx\_CPU\_AFFINITY} on Linux platforms allow for
Condor to lock slots to given CPUs.
Definitions for these variables are at \ref{param:EnforceCpuAffinity}.

\item The new configuration variable \Macro{DEBUG\_TIME\_FORMAT}
  allows a custom specification for the format of the time
  printed at the start of each line in a daemon's log file.
  See \ref{param:DebugTimeFormat} for the complete definition of
  this variable.

\item The new configuration variable \Macro{SHARE\_SPOOLED\_EXECUTABLES}
  is a boolean value that determines whether the \Condor{schedd} daemon will
  use MD5 check sums to avoid storing multiple copies of the same
  executable in the \MacroNI{SPOOL} directory. The default setting is
  \Expr{True}.

\item The new boolean configuration variable
  \Macro{EVENT\_LOG\_FSYNC} provides control of the behavior of
  Condor when writing events to the event log.  Previously,
  the behavior was as if this parameter were set to \Expr{False}.
  See \ref{param:EventLogFsync} for the complete definition of
  this variable.

\item The new boolean configuration variable
  \Macro{EVENT\_LOG\_LOCKING} provides control of the behavior of
  Condor when writing events to the event log.  Previously,
  the behavior was controlled by \MacroNI{ENABLE\_USERLOG\_LOCKING}.
  See \ref{param:EventLogLocking} for the complete definition of
  this variable.

\end{itemize}

\noindent Bugs Fixed:

\begin{itemize}

\item None.

\end{itemize}

\noindent Known Bugs:

\begin{itemize}

\item None.

\end{itemize}

\noindent Additions and Changes to the Manual:

\begin{itemize}

\item None.

\end{itemize}

%%%%      PLEASE RUN A SPELL CHECKER BEFORE COMMITTING YOUR CHANGES!
%%%      PLEASE RUN A SPELL CHECKER BEFORE COMMITTING YOUR CHANGES!
%%%      PLEASE RUN A SPELL CHECKER BEFORE COMMITTING YOUR CHANGES!
%%%      PLEASE RUN A SPELL CHECKER BEFORE COMMITTING YOUR CHANGES!
%%%      PLEASE RUN A SPELL CHECKER BEFORE COMMITTING YOUR CHANGES!

%%%%%%%%%%%%%%%%%%%%%%%%%%%%%%%%%%%%%%%%%%%%%%%%%%%%%%%%%%%%%%%%%%%%%%
\section{\label{sec:History-7-2}Stable Release Series 7.2}
%%%%%%%%%%%%%%%%%%%%%%%%%%%%%%%%%%%%%%%%%%%%%%%%%%%%%%%%%%%%%%%%%%%%%%

This is a stable release series of Condor.
As usual, only bug fixes (and potentially, ports to new platforms)
will be provided in future 7.2.x releases.
New features will be added in the 7.3.x development series.

The details of each version are described below.

%%%%%%%%%%%%%%%%%%%%%%%%%%%%%%%%%%%%%%%%%%%%%%%%%%%%%%%%%%%%%%%%%%%%%%
\subsection*{\label{sec:New-7-2-3}Version 7.2.3}
%%%%%%%%%%%%%%%%%%%%%%%%%%%%%%%%%%%%%%%%%%%%%%%%%%%%%%%%%%%%%%%%%%%%%%

\noindent Release Notes:

\begin{itemize}

\item None.

\end{itemize}


\noindent New Features:

\begin{itemize}

\item Enhanced the Debian 5.0 Condor port on the x86\_64 platform to 
include support for standard universe. 

\item The \Condor{install} script now sets the default central manager
to the current host when the type of installation is `manager'.

\end{itemize}

\noindent Configuration Variable Additions and Changes:

\begin{itemize}

\item Added \MacroNI{SEC\_TCP\_SESSION\_DEADLINE}.  This specifies the
number of seconds after which the client should give up its attempt to
establish a security session with a daemon that it is connecting to.
The default is 120 seconds.

\end{itemize}

\noindent Bugs Fixed:

\begin{itemize}

\item Fixed a memory leak in the \Condor{collector} daemon.  The growth
in memory over time was approximately 10MB/day per 1000 slots.  This
bug was introduced in 7.2.0.

\item Fixed a problem that caused integrity checking of most UDP packets
longer than about 40k to fail.  This bug affects all previous versions of
Condor.

\item By adding \MacroNI{SEC\_TCP\_SESSION\_DEADLINE}, fixed a problem
that has existed since 7.1.2.  The problem is that non-blocking read
operations in the security handshake had no timeout and could therefore
lead to a socket remaining allocated indefinitely if the other side of
the connection didn't respond.  When this problem was observed, the following
message appeared in the SchedLog:

\begin{verbatim}
file descriptor safety level exceeded
\end{verbatim}

\item Fixed a bug in the event log reader code that could cause it to
  not detect missed events in rare cases.

\item A bug in the Chirp java client has been fixed.  ChirpInputStream's
read() method was returning negative values when encountering binary data.

% Gnats PR 872
\item \Condor{dagman} now rejects negative node retry values.

% Gnats PR 946
\item \Condor{dagman} no longer generates a rescue DAG if the DAG is
aborted but is considered successful (ABORT-DAG-ON with return value
of 0).

\end{itemize}

\noindent Known Bugs:

\begin{itemize}

\item None.

\end{itemize}

\noindent Additions and Changes to the Manual:

\begin{itemize}

\item None.

\end{itemize}



%%%%%%%%%%%%%%%%%%%%%%%%%%%%%%%%%%%%%%%%%%%%%%%%%%%%%%%%%%%%%%%%%%%%%%
\subsection*{\label{sec:New-7-2-2}Version 7.2.2}
%%%%%%%%%%%%%%%%%%%%%%%%%%%%%%%%%%%%%%%%%%%%%%%%%%%%%%%%%%%%%%%%%%%%%%

\noindent Release Notes:

\begin{itemize}

\item None.

\end{itemize}


\noindent New Features:

\begin{itemize}

\item Added a full port of Condor to Debian 5.0 on the x86 platform.

\item Added a clipped port of Condor to Debian 5.0 on the x86\_64 platform.

\item Added the \Opt{-DumpRescue} command-line flag to \Condor{dagman}
and \Condor{submit\_dag}.  This flag is intended mainly for testing.

\item Added support for the \Opt{-debug} option to \Condor{qedit}.

\item The Job Router now uses a time slice timer for periodic expression
  evaluation, similar to the \Condor{schedd} daemon.
  The evaluation interval is controlled by 
  the configuration variable \MacroNI{PERIODIC\_EXPR\_INTERVAL},
  and defaults to 60 seconds, the same default value used by
  the \Condor{schedd} daemon.

\item The Job Router now resets the source job, if a failure occurs when
  updating the \Condor{schedd} daemon for a periodic expression that
  evaluated to \Expr{True}.  The job's periodic expressions should be
  evaluated again some time in the future with a successful update.

\end{itemize}

\noindent Configuration Variable Additions and Changes:

\begin{itemize}

\item The new boolean configuration variable
  \Macro{EVENT\_LOG\_FSYNC} provides control of the behavior of
  Condor when writing events to the event log.  Previously,
  the behavior was as if this parameter were set to \Expr{False}.
  See \ref{param:EventLogFsync} for the complete definition of
  this variable.

\item The new boolean configuration variable
  \Macro{EVENT\_LOG\_LOCKING} provides control of the behavior of
  Condor when writing events to the event log.  Previously,
  the behavior was controlled by \MacroNI{ENABLE\_USERLOG\_LOCKING}.
  See \ref{param:EventLogLocking} for the complete definition of
  this variable.

% gittrac #314
\item The new string configuration variable \Macro{TRANSFERER}
  specifies the path to the \Condor{transferer} program which is
  invoked by the \Condor{replication} daemon to perform the actual
  transfer of the file set by \MacroNI{STATE\_FILE}.
  This is part of the high availability framework.
  Prior to Condor 7.2.2, the value of \MacroNI{TRANSFERER} was hard coded to
  \File{\MacroUNI{RELEASE\_DIR}/sbin/condor\_transferer}.  The use of
  this hard coded behavior should be considered obsolete behavior, and
  will be removed in a future version of Condor.

\item The \MacroNI{PREEMPTION\_REQUIREMENTS} and the \MacroNI{RANK}
  expression in the matchmaker can now reference many more ClassAd
  attributes than just \Attr{SubmittorPrio}.  New attributes allow
  this expression to take into account resources currently in use, as
  well as group usage and quota info.  New attributes are:
  \MacroNI{SubmitterUserResourcesInUse},
  \MacroNI{RemoteUserResourcesInUse},
  \MacroNI{RemoteGroupResourcesInUse}, \MacroNI{RemoteGroupQuota},
  \MacroNI{SubmitterGroupResourcesInUse},
  \MacroNI{SubmitterGroupQuota}.

\item Added \MacroNI{JOB\_ROUTER\_ATTRS\_TO\_COPY} configuration
  option. This is a comma separated list of attributes that the Job
  Router should copy from the routed ad to the source ad in addition
  to internally hard coded attributes that are copied.

\item Added \MacroNI{JOB\_ROUTER\_RELEASE\_ON\_HOLD}. configuration
  option that will control whether the Job Router will reset the
  source job to an untouched state if it needs to yield the job
  because the routed job went on hold.  The option defaults to
  resetting the source job.

\item The new configuration variables \Macro{PREEMPTION\_REQUIREMENTS\_STABLE}
  and \Macro{PREEMPTION\_RANK\_STABLE} identify for Condor if all
  attributes in the variables \MacroNI{PREEMPTION\_REQUIREMENTS} and
  \MacroNI{PREEMPTION\_RANK} will not change within
  a negotiation interval.

\item The new configuration variables \Macro{OFFLINE\_LOG}
  and \Macro{OFFLINE\_EXPIRE\_ADS\_AFTER} specify the location of
  persistent machine ClassAds for hibernating machines,
  as well as the lifetime of the persistent ClassAds.

\end{itemize}

\noindent Bugs Fixed:

\begin{itemize}

\item Fixed the \Condor{collector} daemon such that hibernating machines
  never time out.

\item Fixed incorrectly set ClassAd attribute values of machines
  entering a hibernation state.
  All hibernating machines are unclaimed and idle,
  they have no load, the CPU is not busy, and
  the keyboard and console appear as if they had been idle for a long time.

\item Fixed a bug where if any idle slot satisfied the
  \MacroNI{HIBERNATE} expression, Condor would put the machine into a
  sleep state irrespective of any active slots.

\item Fixed a bug on Windows that made it impossible to use
  the defined string \verb@"S5"@ for hibernation.

\item Fixed a bug in the \Condor{starter} where it would be running as
  real uid condor after job hooks are invoked which causes issues when
  accessing files.

\item Fixed a bug where some machines would send a final update ad to
  the \Condor{collector}, invalidating the persistent one that was
  previously sent (when \MacroNI{HIBERNATE} evaluates to \Expr{True}).
  This had the effect of dropping the machine out of the pool once the
  ad had grown stale.

\item Fixed a bug where any two Condor daemons on Windows were able to
  bind to the same port at the same time.

\item Fixed the behavior of the \Condor{negotiator} so that when a
  Condor-G matchmaking ad matches, the machine's ad will be shuffled
  to the end for round-robin matching to multiple gatekeepers with the
  same rank.

\item Resolved a bug in which the submit description file command
  \SubmitCmd{vm\_macaddr} was improperly parsed,
  and thus ignored, by \Condor{submit} for vm universe jobs.

\item Condor's Windows zip file distribution now includes the new
  C/C++ runtime libraries.

\item Fixed a Windows platform bug for jobs that enable streaming I/O.
  The bug caused the \Condor{starter} to crash upon invocation of the
  job.

\item Fixed a bug in which an ill-formed network packet could crash a
Condor daemon.  This would not be seen in normal Condor operation, but
sometimes port-scanning software could trigger such a crash.

\item Fixed a bug in which \Condor{q} would sometimes exit with 
  the value zero, indicating success,
  when it could not connect to a \Condor{schedd} daemon.
  It now exits with an error code.

\item Fixed two seemingly small memory leaks in Condor's SOAP
interface. A small amount of memory was lost per SOAP transaction. On
a high traffic machine, this leak would eventually render the
\Condor{schedd} daemon unresponsive.

\item Fixed a bug in the parallel universe where periodic expressions
involving the \Attr{JobStatus} attribute would not function properly.

\item Fixed a bug where Condor daemons could segmentation fault while trying
to write a core file to disk in the Unix ports.

\item Fixed a bug in which the use of dedicated execute accounts
(indicated by use of the configuration variable
\MacroNI{DEDICATED\_EXECUTE\_ACCOUNT\_REGEXP}) did not work properly
in PrivSep mode: those with the configuration variable
\MacroNI{PRIVSEP\_ENABLED} set to \Expr{True}.

\item Fixed an erroneous log message that reported that
the hook defined by \Macro{HOOK\_UPDATE\_JOB\_INFO} had run,
but would print the \MacroUNI{HOOK\_PREPARE\_JOB} path.
The correct hook ran, so this was only a logging error.
The log message is only visible at the \Expr{D\_FULLDEBUG} level.

% PR 953
\item Fixed a bug that caused \Condor{dagman} to crash if the
\File{dagman.out} file reached a size of 2 GBytes.

\item Fixed a problem affecting the \Condor{starter} when in PrivSep mode.
After the user job exited, an error was printed in the
\Condor{starter} log file complaining that it failed to \Prog{chown} the
sandbox to Condor ownership.  This error was not actually harmful,
just noisy.

\item Fixed a bug in the \Condor{master} that caused it to not have
  \MacroNI{REPLICATION} in its default list for \MacroNI{DC\_DAEMON\_LIST}.
  The example
  configuration file for HAD has been updated to match, as well.

\item Fixed the \Condor{transferer} daemon and documentation to consistently
  use the value of the configuration variable 
  \MacroNI{MAX\_TRANSFERER\_LIFETIME} in High Availability code.

% gittrac #285
\item Fixed a bug that caused \Condor{dagman} to crash,
if a splice DAG has node categories.

\item Changed splice-related \Condor{dagman} debug messages
to \emph{not} be printed at the default verbosity.
They are now mostly printed at debug level 4.
For definitions of the debug levels, see the \Condor{dagman} manual
page at section~ \ref{man-condor-dagman}.

% gittrac #313
\item Fixed a bug that caused the \Condor{replication} daemon,
  as part of the high availability framework,
  to start the \Condor{transferer} client incorrectly; the end result was
  that the \Condor{transferer} was unable to authenticate via GSI
  using host-based certificates.

\item Fixed a bug in which the ClassAd attribute \AdAttr{RemoteWallClockTime}
  could get too big after a restart of the \Condor{schedd} daemon,
  for jobs that were running at the time of the restart.

% gittrac #231
\item Fixed a bug that was causing the \Condor{startd} to log the
  error message 
\begin{verbatim}
  ioctl(SIOCETHTOOL/GWOL) failed: Operation not permitted (1)
\end{verbatim}
  when started as a Personal Condor on Linux.
  The message is now suppressed in this case.  When the message is
  printed, an additional message is logged informing the user that
  this error can be ignored, unless hibernation is being used.

% gittrac #330
\item Fixed a bug that was causing the \Condor{startd} to sometimes
  publish the network adapter's hardware address incorrectly in its
  ClassAd.

\item Fixed a case in which \Condor{history} could get into an infinite
loop when searching through a corrupted history file.

% gittrac #355
\item Fixed a bug in the user log reader code that could cause it to
  get into an inconsistent state after detecting missed events.

\item Condor version 7.2.2 and previous releases do not support 
  communication with Condor 7.3.x daemons using the new 7.3.x
  configuration variables \MacroNI{CCB\_ADDRESS} or
  \MacroNI{PRIVATE\_NETWORK\_NAME}.
  The version 7.2.2 \Condor{collector} daemon now
  recognizes when it is receiving ClassAds from such daemons,
  and it will reject them.
  In prior versions, Condor would accept the ClassAds,
  but attempts to use them led to unexpected behavior.

\end{itemize}

\noindent Known Bugs:

\begin{itemize}

\item None.

\end{itemize}

\noindent Additions and Changes to the Manual:

\begin{itemize}

\item Reorganized the user manual section that describes DAGMan.

\item Added a note about the fact that environment values specified
with the \Opt{environment} submit description file command override values from
the submittor's environment, as imported with \Opt{getenv = True}.

\item Added new information to the section on Power Management
  pertaining to the handling of hibernating machines.
  

\end{itemize}


%%%%%%%%%%%%%%%%%%%%%%%%%%%%%%%%%%%%%%%%%%%%%%%%%%%%%%%%%%%%%%%%%%%%%%
\subsection*{\label{sec:New-7-2-1}Version 7.2.1}
%%%%%%%%%%%%%%%%%%%%%%%%%%%%%%%%%%%%%%%%%%%%%%%%%%%%%%%%%%%%%%%%%%%%%%

\noindent Release Notes:

\begin{itemize}

\item This release addresses reported 7.2.0 problems with the
Windows distribution.

\end{itemize}


\noindent New Features:

\begin{itemize}

\item Condor now has a clipped port to i386 Debian 5.0 (Lenny).

\item Added standard universe support for \Prog{gfortran}.

\item Added support for standard output and standard error to be greater
than 2 Gigabytes.

\end{itemize}

\noindent Configuration Variable Additions and Changes:

\begin{itemize}

\item The configuration variable \Macro{JAVA\_MAXHEAP\_ARGUMENT} now
defaults to the value \Opt{-Xmx1024m}.  The installation process of
Condor resets this value to \Expr{UNDEFINED} in the local
configuration file, if the detected JVM is not from Sun Microsystems.

\item A new feature has been added to the \Condor{master} that makes
it possible to append to the \MacroNI{DC\_DAEMON\_LIST} configuration
variable, instead of overwriting it.  To take advantage of this, place
the plus character ('\verb@+@') as the first character in the
\MacroNI{DC\_DAEMON\_LIST} definition.  For example:
\begin{verbatim}
  DAEMON_LIST     = NEW_DAEMON
  DC_DAEMON_LIST  = +NEW_DAEMON
\end{verbatim}

% PR 959
\item The new configuration variable \Macro{DAGMAN\_COPY\_TO\_SPOOL}
controls whether the \Condor{dagman} binary gets copied to the spool
directory when a DAG is submitted.  See \ref{param:DAGManCopyToSpool}
for details.

% PR 964
\item Added \Opt{-version} and \Opt{-help} command line options to
\Condor{submit\_dag}.

\end{itemize}

\noindent Bugs Fixed:

\begin{itemize}

\item Fixed a bug in the \Condor{collector} which could cause it
to hang indefinitely while reading network input in rare conditions.

\item Fixed a bug in \Condor{chirp} for Windows which was causing it
to crash on invocation.

\item Fixed a bug in the Windows \Condor{mail} program, which was causing
it to become unresponsive when run.  If left running, the application also
increased its memory consumption.

\item Fixed a bug that could cause the \Condor{schedd} to never
evaluate periodic expressions.

\item Fixed a bug on Unix platforms where \Condor{configure} would
provide incorrect defaults for the \MacroNI{JAVA\_MAXHEAP\_ARGUMENT}
attribute in the installed configuration files. The new current
default for Sun Java JVMs is \Opt{-Xmx1024m}.

\item Fixed a bug on Unix platforms where \Condor{configure} would
imply that using the Unix user \Login{root} or UID 0 for the
\Opt{--owner} option is a good thing.  It is not, and would then complain
that it could not find user \Login{root} in the password file.

\item Fixed a bug on Unix platforms where \Condor{configure} would
emit errors about not being able to execute \Prog{ldd} when installing
Condor on the Mac OS X 10.5 platform.  \Condor{configure} now
correctly detects shared library requirements when installing the
Condor binaries on the Mac OS X 10.5 platform.

\item Fixed a bug where execute-side daemons started before the
\Condor{credd} would fail to match with Windows jobs with
\SubmitCmd{run\_as\_owner} set.  This condition persisted until the
execute-side daemons were either restarted or reconfigured.

\item Fixed a problem affecting the Job Router and Condor-C.  When jobs
spool input files, they enter a temporary hold state, which could
trigger actions by a naive periodic remove or release expression.
Periodic expressions are no longer evaluated when in this temporary
hold state, which has the hold reason \AdStr{Spooling input data files}.

\item The example init script \Prog{condor.boot.generic} erroneously claimed
that the \Condor{master} would begin sending SIGKILL to child
processes after 20 seconds if SIGQUIT (the fast shutdown) failed.  The
\Condor{master} will actually wait \MacroUNI{SHUTDOWN\_FAST\_TIMEOUT}
seconds, a value that currently defaults to 300 seconds.

\item Environment variable names are now properly treated as
case-insensitive on Windows. The most common symptom of this bug was
the inability to specify a custom \Env{PATH} environment variable
for a job from its submit description file.

\item Changed \Condor{submit} \Opt{-debug} to issue a warning when ignoring
environment variables. This occurs with \SubmitCmd{getenv = True} set
in a submit description file.

\item Fixed a long-standing memory leak in SOAP interface.
This caused the leak of a few hundred bytes of memory for each connection.
This could eventually have caused the \Condor{schedd} daemon to crash.

\item Fixed Job Router hooks so that their output is properly
propagated where appropriate.

\item Implemented a fix for the \Condor{startd} that prevents it from
crashing if the user specified the configuration variable
\MacroNI{NUM\_SLOTS\_TYPE\_N}, without also specifying \MacroNI{SLOT\_TYPE\_N}.

\item The sample configuration files now correctly set the default
universe to vanilla.  This default has been true since 7.2.0,
but was not reflected in the sample configuration files.

\item Fixed a bug that incorrectly set the value of the
job ClassAd attribute \Attr{RequestMemory} to be 1024 times its
correct size due to a mismatch in units;
the attribute \Attr{RequestMemory} is given in Mbytes, while
the attribute \Attr{ImageSize} is given in Kbytes.

\item Fixed a memory leak in \Condor{dagman} that leaked a small
amount of memory for each job submitted.

\item Fixed a bug that was causing the network mask to be advertised
as a Condor sinful string, rather than a dotted-quad.

\item Fixed a handle leak in the \Condor{procd} on Windows.

\end{itemize}

\noindent Known Bugs:

\begin{itemize}

\item None.

\end{itemize}

\noindent Additions and Changes to the Manual:

\begin{itemize}

\item Added a FAQ entry for Windows describing how machines
with miss-configured performance counters may cause the \Condor{procd}
to crash.

\item Added a manual page for the command \Condor{router\_history}.

\end{itemize}



%%%%%%%%%%%%%%%%%%%%%%%%%%%%%%%%%%%%%%%%%%%%%%%%%%%%%%%%%%%%%%%%%%%%%%
\subsection*{\label{sec:New-7-2-0}Version 7.2.0}
%%%%%%%%%%%%%%%%%%%%%%%%%%%%%%%%%%%%%%%%%%%%%%%%%%%%%%%%%%%%%%%%%%%%%%

\noindent Release Notes:

\begin{itemize}

\item A bug in some older Xen kernels can result in Condor errors
due to a broken assumption in the \Condor{procd} daemon.
See the FAQ entry at section~ \ref{sec:xen-jiffies-bug} for details.

\item A problem has been discovered when using snapshot disks with 
\SubmitCmd{vm} universe VMware jobs,
if the path that the \Condor{vm-gahp} uses to refer to the
virtual machine's VMX file contains a symbolic link.
See the FAQ entry at section~ \ref{sec:vmware-symlink-bug} for details.

\item The name of the Amazon EC2 GAHP binary has changed from
\Prog{amazon-gahp} to \Prog{amazon\_gahp}. This makes it consistent
with the naming of other Condor binaries.

\end{itemize}


\noindent New Features:

\begin{itemize}

\item The default \SubmitCmd{universe} for jobs is now 
\SubmitCmd{vanilla}, instead of \SubmitCmd{standard}.
The default can be changed using the configuration variable
\Macro{DEFAULT\_UNIVERSE}.

\item VMware \SubmitCmd{vm} universe jobs now have any BIOS settings saved in
an \File{nvram} file in the \SubmitCmd{vmware\_dir} given in the
job's submit file transferred to the execute machine, so that they
apply to the job's execution.

\item Daemons that become unresponsive are now killed using the
SIGABRT signal, which causes a core file to be dropped.
Setting the configuration variable \Macro{NOT\_RESPONDING\_WANT\_CORE}
to \Expr{False} will revert to the previous behavior that used
the SIGKILL signal.

\item The \Condor{job\_router} and the
\Condor{q} command with the \Opt{-better-analyze} option now
support more ClassAd functions than they previously did.  They now
support all ClassAd functions, except for those with names beginning
with the string \Code{stringList}.

\item \Condor{status} given the options \Opt{-submitters} \Opt{-xml}
no longer emits a single blank line when there are no submitters,
instead it prints valid XML output with an empty body.

\end{itemize}

\noindent Configuration Variable Additions and Changes:

\begin{itemize}

\item The HAD configuration variable \MacroNI{NEGOTIATOR\_STATE\_FILE}
has changed its name to \MacroNI{STATE\_FILE}.

\end{itemize}

\noindent Bugs Fixed:

\begin{itemize}

\item \Security A flaw was found and fixed that could allow an unauthenticated
user to cause Condor daemons to shut down,
and could allow running jobs to be removed from the queue.

% PR 952
\item Fixed a bug that caused \Condor{dagman} to stay in the Condor
queue, if \Condor{dagman} was accidentally submitted with an empty DAG
input file.

% PR 959
\item \Condor{submit\_dag} now generates a \File{.condor.sub} file with
the submit description file command \SubmitCmd{copy\_to\_spool}
set to \Expr{True}, to ease version upgrades while
large DAGs are running.

\item Fixed a problem in the \Condor{startd} when using
\MacroNI{STARTD\_SLOT\_EXPRS} for attributes that are sometimes
present and sometimes absent from the machine ClassAd.  This is most
typical of attributes that enter the machine ClassAd from the job, via
\MacroNI{STARTD\_JOB\_EXPRS}.  When the attribute went away from slot X
(for example, because the job on slot X finished), the corresponding
\MacroNI{SlotX\_<AttributeName>} attribute was not reliably removed from
all of the other slots.

\item Removed some redundant information from the \Condor{startd} 
advertisements to the \Condor{collector}, 
from within the private ClassAd that is not user-visible.
This fix reduces UDP traffic and memory usage generated by
the \Condor{startd} by about 20\Percent\
in the \Condor{collector} and \Condor{negotiator} daemons.

\item Fixed the \Condor{master} daemon to correctly preserve all command-line
arguments when restarting itself.  In some cases, not preserving \Code{argv[0]}
confused external utilities that monitor the \Condor{master} process by looking
at the output of \Prog{ps} or similar programs.  Also, not preserving
\Opt{-pid} and \Opt{-runfor} could cause unexpected behavior.

\item Fixed a bug that exhibited itself when
the configuration variable \MacroNI{NEGOTIATOR\_CONSIDER\_PREEMPTION}
was set to \Expr{False}, in which jobs
would not be matched to slots in the backfill state.  Corrected, slots doing
backfill are included in the matchmaking process.

\item The \Condor{job\_router} did not work while managing jobs from
multiple users when read access to the \Condor{schedd} required
authentication.  The \Condor{job\_router} was also not able to use
authentication methods other than FS.  Now it can use any
authentication method, as long as the resulting identity is listed in
the configuration variable
\MacroNI{QUEUE\_SUPER\_USERS} or the \Condor{job\_router} and
\Condor{schedd} are running as a Personal Condor in non-root mode.

% Commented out by Karen, as it provides no relevant information
% in the given form.
% \item Fixed a number of memory leaks.

\item Fixed a bug in the \Condor{schedd} daemon that could cause it to write
  an incorrect Unique ID to the event log's header.

\item Fixed a bug in the user log reader API that could cause it to
  incorrectly return a ULOG\_NO\_EVENT in rare cases.

\item Fixed a bug in the user log reader API that could cause it to
  crash if the application attempted to re-initialize the ReadUserLog
  object.  The code now detects this condition, and returns an error
  when the application attempts to re-initialization an already
  initialized ReadUserLog object.

\item Fixed a bug that limited the size of \File{stdin}, \File{stdout},
and \File{stderr} files in the vanilla universe to 2GBytes.

\item Fixed a bug that could cause the \Condor{starter} to EXCEPT upon 
completion or eviction of a \SubmitCmd{vm} universe job.
The error message that appeared in the \File{StarterLog} file was
\begin{verbatim}
  Write_Pipe: invalid pipe end
\end{verbatim}

\item When a held job is removed, the values of the attributes
\Attr{HoldReason}, \Attr{HoldReasonCode} and \Attr{HoldReasonSubCode}
are moved to \Attr{LastHoldReason}, \Attr{LastHoldReasonCode} and
\Attr{LastHoldReasonSubCode}. Before, a hold reason could be lost if a
removed job was subsequently held.

\item The executable attribute for amazon grid universe jobs no longer
needs to be a valid file path.

\item Improved error reporting when a Xen or VMware command fails in the
\SubmitCmd{vm} universe.

\item For \SubmitCmd{vm} universe jobs,
virtual floppy disks are no longer disabled.

\item Fixed a bug introduced in Condor 7.1.4 that caused Condor to
ignore the virtual machine status reported by Xen in the \SubmitCmd{vm} universe.

\item Fixed a 20-second delay in the start up of the \Condor{c-gahp} and
the \Condor{vm-gahp}.

\item Fixed a bug which caused the net mask to be published
  into the machine ClassAd incorrectly.

\item Fixed a bug introduced in Condor 7.1.4 which could cause any
  Condor daemon to crash if the level of debugging output \MacroNI{D\_ALL}
  is enabled when a \Condor{reconfig} command is issued.

\item Fixed a bug introduced in Condor 7.1.4 which caused standard universe
jobs to fail to start up, if security authentication, but not encryption was
enabled between the submit side and the execute side.

% Commented out by Karen, as it gives no relevant information to any
% reader of this version history.  
%\item Many bugs fixed in the \Condor{job\_router} hooks.

\item Fixed a bug with streaming \File{stdin}, \File{stdout}, and
\File{stderr} when using \Prog{glexec}.

% Commented out by Karen, as it gives no relevant information to any
% reader of this version history, and has nothing to do with bugs fixed.
% \item Many improvements in error propagation and debugging output.

\end{itemize}

\noindent Known Bugs:

\begin{itemize}

\item None.

\end{itemize}

\noindent Additions and Changes to the Manual:

\begin{itemize}

\item Initial documentation for dynamic provisioning is available
in section~ \ref{sec:SMP-dynamicprovisioning}.

\item Documentation for Kerberos authentication
(see section~ \ref{sec:Kerberos-Authentication})
and associated configuration variables has been updated.

\end{itemize}


%%%%      PLEASE RUN A SPELL CHECKER BEFORE COMMITTING YOUR CHANGES!
%%%      PLEASE RUN A SPELL CHECKER BEFORE COMMITTING YOUR CHANGES!
%%%      PLEASE RUN A SPELL CHECKER BEFORE COMMITTING YOUR CHANGES!
%%%      PLEASE RUN A SPELL CHECKER BEFORE COMMITTING YOUR CHANGES!
%%%      PLEASE RUN A SPELL CHECKER BEFORE COMMITTING YOUR CHANGES!

%%%%%%%%%%%%%%%%%%%%%%%%%%%%%%%%%%%%%%%%%%%%%%%%%%%%%%%%%%%%%%%%%%%%%%
\section{\label{sec:History-7-1}Development Release Series 7.1}
%%%%%%%%%%%%%%%%%%%%%%%%%%%%%%%%%%%%%%%%%%%%%%%%%%%%%%%%%%%%%%%%%%%%%%

This is the development release series of Condor.
The details of each version are described below.

%%%%%%%%%%%%%%%%%%%%%%%%%%%%%%%%%%%%%%%%%%%%%%%%%%%%%%%%%%%%%%%%%%%%%%
\subsection*{\label{sec:New-7-1-1}Version 7.1.1}
%%%%%%%%%%%%%%%%%%%%%%%%%%%%%%%%%%%%%%%%%%%%%%%%%%%%%%%%%%%%%%%%%%%%%%

\noindent Release Notes:

\begin{itemize}

\item None.

\end{itemize}


\noindent New Features:

\begin{itemize}

\item Included some Windows example jobs (submit files and binaries).

\item Added a new feature to the DAGMan language called splicing. Please
read section~\ref{sec:DAGSplicing} on page \pageref{sec:DAGSplicing}.

\item The Prepare Job Hook can now modify the job ClassAd before execution.
For a complete description of the new hook system, read
section~\ref{sec:job-hooks} on page~\pageref{sec:job-hooks}.

\item Condor now coerces the result of \$\$([]) expressions within
submit description files to strings.
This means that submit files can do simple arithmetic.
For example, you can describe a command-line argument as:

arguments = \$\$([\$(PROCESS)+100])

and \Condor{submit} will expand the argument to be the expected value.

\item Condor daemons now periodically update the \Code{ctime} of their
  log files, instead of the \Code{mtime}, as they previously did.
  At start up, the daemons use this \Code{ctime} 
  to determine how long they may have been down.

\item Added the capability to the \Condor{startd} to allow it to power 
  down machines based a user specified policy.  See 
  section~\ref{sec:power-man} on \pageref{sec:power-man} on
  Power Management for more details.

\item \Condor{off} now supports the \Opt{-peaceful} option for the
  \Condor{schedd}, in addition to the existing support that already existed for
  the \Condor{startd}.  When peacefully shut down,
  the \Condor{schedd} stops starting new
  jobs and waits for all running jobs to finish before exiting.  The
  default shut down behavior is still \Opt{-graceful}, which checkpoints
  and stops all running standard universe jobs and gracefully
  disconnects from other types of jobs in the hopes of later restarting
  and reconnecting to them without any disturbance to the running job.

\item The \Condor{job\_router} now supports deletion of attributes
  when transforming job ClassAds from vanilla to grid universe.  It also
  behaves more deterministically when choosing from multiple possible
  routes.  Rather than picking one at random, it uses a round-robin
  selection.

\end{itemize}

\noindent Configuration Variable Additions and Changes:

\begin{itemize}

\item The existing \Macro{BIND\_ALL\_INTERFACES} configuration variable
  now defaults to \Expr{True}.

\item Added the \Macro{HIBERNATE} expression, which, when evaluated in
  the context of each slot, determines if a machine should enter
  a low power state. See page~\pageref{param:Hibernate} for more 
  information.

\item Added the \Macro{HIBERNATE\_CHECK\_INTERVAL} configuration variable,
  which, if set to a non-zero value, enables the \Condor{startd} to place the 
  machine in a low power state based on the evaluation of the
  \MacroNI{HIBERNATE} expression.  See 
  page~\pageref{param:HibernateCheckInterval} for more information.

\item The existing \Macro{VALID\_SPOOL\_FILES} configuration variable
  now automatically includes \File{SCHEDD.lock},
  the lock file used for high availability \Condor{schedd} fail over.
  Other high availability lock files are not currently included.

\item Added the \Macro{SEC\_DEFAULT\_AUTHENTICATION\_TIMEOUT} configuration
  variable, where the definition \Expr{DEFAULT} may be replaced
  by the usual list of contexts for security settings
  (for example, \Expr{CLIENT}, \Expr{READ}, and \Expr{WRITE}).
  This specifies the number of seconds that Condor should
  allow for the authentication of network connections to complete.
  Previously, GSI authentication was hard-coded to allow 5 minutes
  for authentication.
  Now it uses the same default as all other methods: 20 seconds.

\item Added the \Macro{STARTER\_UPDATE\_INTERVAL\_TIMESLICE} configuration
  variable, which
  specifies the highest fraction of time that the \Condor{starter} should spend
  collecting monitoring information about the job, such as disk usage.
  It defaults to 0.1.  If checking the disk usage of the job takes a
  long time, the \Condor{starter} will monitor less frequently than 
  specified by \MacroNI{STARTER\_UPDATE\_INTERVAL}.

\end{itemize}

\noindent Bugs Fixed:

\begin{itemize}

\item Fixed a bug in Java universe where each slot would be told to
  potentially use all the memory on the machine.  Now, each JVM 
  receives the physical memory divided by the number of slots.

\item On Windows, slot users would sometimes show up in the Windows Welcome
  Screen.  This has now been resolved.
  The slot users need to be manually
  removed for this to take effect and the machine may need to be rebooted for
  the setting to be honored.

\item Fixed a bug in the ClassAd \Procedure{string} function.
  The function now properly converts integers and floats
  to their string representation.

\item The Windows Installer is now completely internationalized: it will no 
  longer fail to install because of a missing "Users" group; instead, it
  will use the regionally appropriate group.

\item Interoperability with Samba (as a PDC) has been improved.  Condor 
  uses a fast form of login during credential validation.  Unfortunately, 
  this login procedure fails under Samba, even if the credentials are 
  valid.  The new behavior is to attempt the fast login, and on failure, 
  fall back to the slower form.

\item Windows slot users no longer have the Batch Privilege added, nor 
  does Condor first attempt a Batch login for slot users.  This was 
  causing permission problems on hardened versions of Windows, such 
  as Windows Sever 2003, in that not interactive users lacked the 
  permission to run batch files (via the \Prog{cmd.exe} tool). This affected 
  any user submitting jobs that used batch files as the executable.

% issue [#1516]
\item If the \AdAttr{IWD} is not defined in a job classified
  ad that was either fetched by the \Condor{startd} via job hooks, or
  pushed to the \Condor{startd} via COD, the \Condor{starter} no
  longer treats this as a fatal error, and instead uses the temporary
  job execution sandbox as the initial working directory.

\end{itemize}

\noindent Known Bugs:

\begin{itemize}

\item None.

\end{itemize}

\noindent Additions and Changes to the Manual:

\begin{itemize}

\item The manual now contains Windows installation instructions for
  controlling the configuration for the \SubmitCmd{vm} universe.

\end{itemize}



%%%%%%%%%%%%%%%%%%%%%%%%%%%%%%%%%%%%%%%%%%%%%%%%%%%%%%%%%%%%%%%%%%%%%%
\subsection*{\label{sec:New-7-1-0}Version 7.1.0}
%%%%%%%%%%%%%%%%%%%%%%%%%%%%%%%%%%%%%%%%%%%%%%%%%%%%%%%%%%%%%%%%%%%%%%

\noindent Release Notes:

\begin{itemize}

\item Upgrading to 7.1.0 from previous versions of Condor will make
existing Standard Universe jobs that have already run fail to match to
machines running Condor 7.1.0 unless the job previously ran on a
machine using the Red Hat 5.0 release of Condor.  This is because the
value of the \Attr{CheckpointPlatform} attribute of the machine
ClassAd has changed in order to better represent checkpoint
compatibility.  If this affects you, you can use \Condor{qedit} to
change the \Attr{LastCheckpointPlatform} attribute of existing
Standard Universe jobs to match the new \Attr{CheckpointPlatform}
advertised by the machine ClassAd where the job last ran.

\item Condor no longer supports root configuration files
(for example, \File{/etc/condor/condor\_config.root},
\File{~condor/condor\_config.root}, and
the file defined by the configuration variable
\MacroNI{LOCAL\_ROOT\_CONFIG\_FILE}).  This feature was intended to
give limited powers to a Unix administrator to configure some aspects
of Condor without gaining root powers.  However, given the flexibility
of the configuration system, we decided that this was not practical.
As long as Condor is started up as root, it should be clearly
understood that whoever has the ability to edit the Condor
configuration files can effectively run arbitrary programs as root.

\end{itemize}


\noindent New Features:

\begin{itemize}

\item In the past, Condor has always sent work to the execute machines
  by pushing jobs to the \Condor{startd}, either from the
  \Condor{schedd} or via \Condor{cod}.
  As of version 7.1.0, The \Condor{startd} now has the ability to pull
  work by fetching jobs via a system of plug-ins or hooks.
  Additional hooks are invoked by the \Condor{starter} to help manage
  work (especially for fetched jobs, but the \Condor{starter} hooks
  can be defined and invoked for other kinds of jobs as well).
  For a complete description of the new hook system, read
  section~\ref{sec:job-hooks} on page~\pageref{sec:job-hooks}.

% PR 888/921
\item Added the capability to insert commands into the \File{.condor.sub}
  file produced by \Condor{submit\_dag} with the \Opt{-append} and
  \Opt{-insert\_sub\_file} command-line arguments to \Condor{submit\_dag} and
  the \Macro{DAGMAN\_INSERT\_SUB\_FILE} configuration variable.
  See the \Condor{submit\_dag} manual page on
  page~\pageref{man-condor-submit-dag}
  and the configuration variable definition on
  page~\pageref{param:DAGManInsertSubFile} for more information.

\item For platforms running a Windows operating system, the \Attr{Arch}
  machine ClassAd attribute more correctly reflects the architectures
  supported.  Instead of values \AdStr{INTEL} and \AdStr{UNDEFINED},
  the values will now be: \AdStr{INTEL} for x86,
  \AdStr{IA64} for Intel Itanium,
  and \AdStr{X86\_64} for both AMD and Intel 64-bit processors.
  These values are listed in the unnumbered subsection labeled
  Machine ClassAd Attributes on page~\pageref{sec:Machine-ClassAd-Attributes}.

\item The Windows MSI installer now supports extended \SubmitCmd{vm} universe 
  options. These new options include: the ability to set the 
  networking type, how much memory the \SubmitCmd{vm} universe can use 
  on a host, and
  the ability to set the version of \Prog{VMware} installed on the host.

\item The \Condor{status} and \Condor{q} command line tools now have a
  version option which prints the version of those specific tools.  This
  can be useful when multiple versions of Condor are installed on the
  same machine.

\item The configuration variable \MacroNI{CONDOR\_VIEW\_HOST} may now
  contain a port number and may (if desired) refer to a
  \Condor{collector} daemon running on the same host as the
  \Condor{collector} that is forwarding ads.  It is also now possible to
  use the forwarded ads for matchmaking purposes.  For example, several
  collectors could forward ads to a single aggregating collector which
  a \Condor{negotiator} then uses as its source of information for
  matchmaking.

\item Added client-side authorization controls
\MacroNI{ALLOW\_CLIENT}, \MacroNI{DENY\_CLIENT}.  When using a mutual
authentication method (e.g. GSI, SSL, Kerberos), this allows you to
specify what authenticated servers Condor tools and daemons should
trust when they form a connection to the server.  This deprecates
\MacroNI{GSI\_DAEMON\_NAME}, which provided rudimentary support for
client-side authorization in a GSI-specific way.

% PR 598/788
\item \Condor{dagman} deals with rescue DAGs in a more sophisticated
way; this is especially helpful for nested DAGs.
See the rescue DAG subsection~\pageref{sec:DAGRescue} of the \Condor{dagman}
manual section for more information.

\item Additional logging details for unusual error cases to help 
identify problems.

\item A new (optional) daemon named \Condor{job\_router} has been
added, so far only on unix.  It may be configured to transform vanilla
universe jobs into grid universe jobs, for example to send excess jobs
to other sites via Condor-C or Condor-G.  For details, see
page~\pageref{sec:JobRouter}.

\item Previously, \condor{q} \Opt{-better-analyze} was supported on most
but not all versions of Linux.  It is now supported on all Unix platforms
but not yet on Windows.

\end{itemize}

\noindent Configuration Variable Additions and Changes:

\begin{itemize}

% PR 921
\item Added the \Macro{DAGMAN\_INSERT\_SUB\_FILE} variable, which allows a file
  of commands to be inserted into \File{.condor.sub} files generated
  by \Condor{submit\_dag}.  See page~\pageref{param:DAGManInsertSubFile}
  for more information.

\item The semantics of \MacroNI{CLAIM\_WORKLIFE} were previously not
clearly defined before the start of the first job.  A delay between
the \Condor{schedd} claiming a slot and the \Condor{shadow} starting a
job could be caused by the submit machine being very busy or by
\MacroNI{JOB\_START\_DELAY}.  Previously, such a delay would
unpredictably result in the first job being rejected if
\MacroNI{CLAIM\_WORKLIFE} expired during that time.  Now,
\MacroNI{CLAIM\_WORKLIFE} is defined to apply only after the first job
has started.  Therefore, setting it to zero has the effect of allowing
exactly one job per claim to run.  The default is still the special
value -1, which places no limit on how long the slot may continue
accepting new jobs from the \Condor{schedd} that claimed it.

% PR 598/788
\item Added the \Macro{DAGMAN\_OLD\_RESCUE} variable, which controls whether
\Condor{dagman} writes rescue DAGs in the old way.  See
page~\pageref{param:DAGManOldRescue} for more information.

% PR 598/788
\item Added the \Macro{DAGMAN\_AUTO\_RESCUE} variable, which controls
whether \Condor{dagman} automatically runs an existing rescue DAG.
See page~\pageref{param:DAGManAutoRescue} for more information.

% PR 598/788
\item Added the \Macro{DAGMAN\_MAX\_RESCUE\_NUM} variable, which
controls the maximum "new-style" rescue DAG number written or
automatically run by \Condor{dagman}.
See page~\pageref{param:DAGManMaxRescueNum} for more information.

\end{itemize}

\noindent Bugs Fixed:

\begin{itemize}

\item The Condor Build ID is now printed by \Condor{version} and placed 
  in the logs for machines running a Windows operating system.

\item \Condor{quill} and the \Condor{dbmsd} correctly register 
  themselves with the Windows firewall.

% PR 926
\item \Condor{submit\_dag} now avoids possibly running off the end
of the argument list if an argument requiring a value does not have one.

\item The \Condor{submit\_dag} \Opt{-debug} argument now must be
specified with at least \Opt{-de} to avoid conflict with the
\Opt{-dagman} argument.

\item Added missing information about the \Opt{-config} argument to
\Condor{submit\_dag}'s usage message.

% PR 927
\item \Condor{dagman} no longer considers duplicate edges in a DAG a
fatal error (it is now a warning).

\end{itemize}

\noindent Known Bugs:

\begin{itemize}

\item No hook is invoked if a fetched job does not contain enough data
  to be spawned by a \Condor{starter} or if other errors prevent the
  job from being run after the \Condor{startd} agrees to accept the
  work.
  This limitation will be addressed in a future version of Condor,
  most likely via the addition of a new hook invoked whenever the
  \Condor{starter} fails to spawn a job.
  For more information about the new hook system included in Condor
  version 7.1.0, read section~\ref{sec:job-hooks} on
  page~\pageref{sec:job-hooks}.

\end{itemize}

\noindent Additions and Changes to the Manual:

\begin{itemize}

\item Added \AdStr{WINNT60} for the Vista operating system to
  the documented list of possible values for the machine ClassAd
  attribute \AdAttr{OpSys}.

\end{itemize}


%%%%      PLEASE RUN A SPELL CHECKER BEFORE COMMITTING YOUR CHANGES!
%%%      PLEASE RUN A SPELL CHECKER BEFORE COMMITTING YOUR CHANGES!
%%%      PLEASE RUN A SPELL CHECKER BEFORE COMMITTING YOUR CHANGES!
%%%      PLEASE RUN A SPELL CHECKER BEFORE COMMITTING YOUR CHANGES!
%%%      PLEASE RUN A SPELL CHECKER BEFORE COMMITTING YOUR CHANGES!

%%%%%%%%%%%%%%%%%%%%%%%%%%%%%%%%%%%%%%%%%%%%%%%%%%%%%%%%%%%%%%%%%%%%%%
\section{\label{sec:History-7-0}Stable Release Series 7.0}
%%%%%%%%%%%%%%%%%%%%%%%%%%%%%%%%%%%%%%%%%%%%%%%%%%%%%%%%%%%%%%%%%%%%%%

This is a stable release series of Condor.
It is based on the 6.9 development series.
All new features added or bugs fixed in the 6.9 series are available
in the 7.0 series.
As usual, only bug fixes (and potentially, ports to new platforms)
will be provided in future 7.0.x releases.
New features will be added in the 7.1.x development series.

On backwards compatibility:
we believe that Condor 7.0.x and 6.8.x are wire-compatible, 
and can be freely mixed between computers in a Condor pool. 
However, we do not regularly test this compatibility and cannot guarantee it, 
so we recommend using a single release of Condor when possible. 
Please note that although you can mix Condor 7.0.x and 6.8.x in a pool, 
you cannot mix them on a single computer. 
That is, a \Condor{master} daemon running 6.8.x cannot run Condor daemons 
from version 7.0.x, or vice-versa.

The details of each version are described below.


%%%%%%%%%%%%%%%%%%%%%%%%%%%%%%%%%%%%%%%%%%%%%%%%%%%%%%%%%%%%%%%%%%%%%%
\subsection*{\label{sec:New-7-0-6}Version 7.0.6}
%%%%%%%%%%%%%%%%%%%%%%%%%%%%%%%%%%%%%%%%%%%%%%%%%%%%%%%%%%%%%%%%%%%%%%

\noindent Release Notes:

\begin{itemize}

\item None.

\end{itemize}


\noindent New Features:

\begin{itemize}

\item None.

\end{itemize}

\noindent Configuration Variable Additions and Changes:

\begin{itemize}

\item None.

\end{itemize}

\noindent Bugs Fixed:

\begin{itemize}

\item In some rare cases, the \Condor{startd} failed to fully preempt jobs.
The job itself was killed, but the \Condor{starter} process watching over
it would not be killed.  The slot would then stay in the Preempting state
indefinitely.

\end{itemize}

\noindent Known Bugs:

\begin{itemize}

\item None.

\end{itemize}

\noindent Additions and Changes to the Manual:

\begin{itemize}

\item None.

\end{itemize}


%%%%%%%%%%%%%%%%%%%%%%%%%%%%%%%%%%%%%%%%%%%%%%%%%%%%%%%%%%%%%%%%%%%%%%
\subsection*{\label{sec:New-7-0-5}Version 7.0.5}
%%%%%%%%%%%%%%%%%%%%%%%%%%%%%%%%%%%%%%%%%%%%%%%%%%%%%%%%%%%%%%%%%%%%%%

\noindent Release Notes:

This release contains many bug fixes and some improvements to error handling
of Local Universe jobs. Note that some of the bug fixes are
security-related; therefore, we recommend sites either upgrade Condor, or
restrict permissions on who is allowed to submit Condor jobs to trusted
users. Bug fixes that are security related are clearly marked in the Bugs
Fixed section below along with a description of the potential security
impact. The Condor Project believes in the full disclosure of information,
and therefore complete vulnerability details can be found at
\URL{http://www.cs.wisc.edu/condor/security/}. However, in order to give an
adequate upgrade window for production installations, we will delay posting
the full vulnerability details fixed in this release for 30 days (until the
week of November 3rd 2008). 


\noindent New Features:

\begin{itemize}

\item Local universe jobs now go on hold for the same specific reasons that
vanilla jobs may go on hold.  Examples are missing input or executable files.
Previously, when local universe jobs failed in this manner,
the jobs returned to the idle state in the job queue,
repetitively attempting to run, 
and failing over and over until the job is removed.

\item Local universe jobs now have the ClassAd attribute \Attr{NumShadowStarts}.
Although local universe jobs do not have a \Condor{shadow} process, 
this attribute
is introduced to keep management of local universe as similar to
vanilla universe as possible.  For local universe jobs, this attribute
is identical to the attribute \Attr{JobRunCount}, 
which indicates how many times a
local \Condor{starter} process has been created to run the job.

\end{itemize}

\noindent Configuration Variable Additions and Changes:

\begin{itemize}

\item None.

\end{itemize}

\noindent Bugs Fixed:

\begin{itemize}

\item \Security A flaw was found and fixed in the way Condor processes user submitted
jobs. It was possible for a user who had permissions to submit jobs into
Condor to do so in a way that could cause that job to run as any other
non-root user. We have not had any reported incidents exploiting this
flaw. (CVE-2008-3826)

\item \Security A stack-based buffer overflow flaw was found and fixed in the
\Condor{schedd} daemon. A user who had permissions to submit a job could
do so in a manner that could cause the \Condor{schedd} to crash, or
potentially, execute arbitrary code on the submit machine with the
\Condor{schedd}'s identity. We are not aware of any known exploits for
this flaw. We have not had any reported incidents exploiting this flaw.
(CVE-2008-3828)

\item \Security A denial-of-service flaw was found and fixed in the \Condor{schedd}
daemon. A user who had permissions to submit a job could have done so in a
manner that would cause \Condor{schedd} to crash. 
We have not had any reported incidents exploiting this flaw. (CVE-2008-3829)

\item \Security A flaw was found and fixed in the way Condor processes allow and deny 
net masks for access control. 
If Condor's configuration file contained overlapping net masks in 
the allow or deny rules, it could have caused those rules to be ignored, 
potentially allowing unintended access to users in Condor's 
deny authorization lists. 
We have not had any reported incidents exploiting this flaw. (CVE-2008-3830) 

% Fixed by Red Hat
\item Fixed a segmentation fault bug with \Condor{submit} \Opt{-dump} when
\Expr{universe=grid} or \Expr{x509userproxy=<anything>}.

\item Fixed a stack overflow bug in the \Condor{negotiator} daemon.

\item Fixed \Condor{submit} \Opt{-dump} such that it would function with the
standard universe.

\item Fixed a memory leak in the \Condor{startd}, which occurred during the
handling of a \Condor{reconfig} command.

\item When the configuration variable \Macro{NEGOTIATOR\_CONSIDER\_PREEMPTION}
is defined to be \Expr{False},
this no longer results in machines in the Owner state being ignored
during matchmaking.  Previously, even if \MacroNI{START} was \Expr{True},
machines in the Owner state were disregarded.

\item Setting \Attr{JobLeaseDuration} to be less than 15 minutes caused the
\Condor{schedd} daemon to abort and restart the next time a
\Condor{reconfig} command was executed.
The error message in the \Condor{schedd} log appeared as:

\footnotesize
\begin{verbatim}
ALIVE_INTERVAL in the condor configuration is too high (300).
\end{verbatim}
\normalsize

\item Fixed a slow memory leak affecting the \Condor{startd},
\Condor{schedd}, and \Condor{collector} daemons.  This leak would probably
require many months of continuous operation before causing noticeable problems.

\item Fixed a bug that caused a \Condor{schedd} daemon crash.
The crash occurred during a fast shut down of the
\Condor{schedd} daemon as it dealt with local universe
jobs or with any job that required reconnection when
the \Condor{schedd} daemon started up.

\item Local and scheduler universe jobs were failing to increment the
\Attr{JobRunCount} attribute in the job ClassAd when an attempt to run
the job was made.  This problem was introduced in 6.9.5.

\item Some rare types of failures during file transfer caused the
Condor daemon conducting the transfer to hang indefinitely.  For
example, if the file transfer process created by the \Condor{schedd}
was killed by an administrator or crashed due to an internal error,
the \Condor{schedd} would become unresponsive.

\item GCB was updated, fixing minor bugs with GCB temporary files 
(typically the file(s) \File{/tmp/gcb-inherit-*}).
These bugs did not impact GCB functionality.  Earlier
versions would leave temporary files behind. Temporary files would have
permissions of 000.  With the fix, under normal operations the files should be
deleted, and the \Login{condor} user should have read and write access to the files.

\item Evaluation of the configuration variable \Macro{STARTD\_AD\_REEVAL\_EXPR}
did not work for many types of expressions.
The problem resulted in the following message in the 
\Condor{negotiator} daemon log:

\begin{verbatim}
Can't evaluate STARTD_AD_REEVAL_EXPR  ...
\end{verbatim}

\item Reconnecting to parallel universe jobs after a restart of the
\Condor{schedd} daemon, would sometimes fail.  The failure was caused
by the \Condor{shadow} trying to connect to the address of the
previous instance of the \Condor{schedd} rather than the address of
the current instance.

\item Made the \Condor{gridmanager} less aggressive in forwarding refreshed
proxies for gt2 grid universe jobs. Now, the refreshed proxy will not be
forwarded until the old proxy has less than six hours of life until
expiration.

\item Fixed a bug in the \Condor{gridmanager} that could result in job
status updates from the Grid Monitor to be ignored.

\item The Grid Monitor no longer changes the last-modified time of GRAM
state files whose job's status is FAILED. This should make it easier for
file cleaners to remove the the GRAM state files of old, abandoned jobs.

\item Fixed a problem that could cause flocked jobs to fail due to
authorization errors in the \Condor{starter}. Such failures were more
likely to occur for long-running jobs or if the \Condor{schedd} were
issued a full reconfig during the job's execution.

\item Fixed a \Condor{gridmanager} crash on Windows. This crash only
appeared if \Macro{GRIDMANAGER\_DEBUG} were set to a higher level than
the default.

\item In PrivSep mode, a job would previously fail if it created a
symlink in its sandbox pointing to a file owned by a UID other than
that used to run the job. This behavior has been fixed.

\end{itemize}

\noindent Known Bugs:

\begin{itemize}

\item None.

\end{itemize}

\noindent Additions and Changes to the Manual:

\begin{itemize}

\item Descriptions of previously undocumented Condor Perl module subroutines
  have been added to the manual.  See section~\ref{sec:condor-pm}.

\end{itemize}



%%%%%%%%%%%%%%%%%%%%%%%%%%%%%%%%%%%%%%%%%%%%%%%%%%%%%%%%%%%%%%%%%%%%%%
\subsection*{\label{sec:New-7-0-4}Version 7.0.4}
%%%%%%%%%%%%%%%%%%%%%%%%%%%%%%%%%%%%%%%%%%%%%%%%%%%%%%%%%%%%%%%%%%%%%%

\noindent Release Notes:

\begin{itemize}

\item This release fixes a problem causing possible incorrect handling
of wild cards in authorization lists.
Examples of the configuration variables that specify
authorization lists are
\begin{verbatim}
  ALLOW_WRITE
  DENY_WRITE
  HOSTALLOW_WRITE
  HOSTDENY_WRITE
\end{verbatim}
If a configuration variable uses the asterisk character
(\verb@*@) in configuration variables that specify the
authorization policy, it is advisable to upgrade.
This is especially true for the 
use of wild cards in any \MacroNI{DENY} list,
since this problem could result in
access being allowed, when it should have been denied.
This issue affects all previous versions of Condor.

\item The default daemon-to-daemon security session duration has been
changed from 100 days to 1 day. This should reduce memory usage in the
\Condor{collector} in pools with short-lived \Condor{startd}s (e.g. 
glidein pools or pools whose machines are rebooted every night).

\end{itemize}


\noindent New Features:

\begin{itemize}

\item Added functionality to periodically update timestamps on lock files. 
This prevents administrative programs from deleting in-use lock files and
causing undefined behavior.

\item When the configuration variable \Macro{SCHEDD\_NAME} ends in 
the \verb$@$ symbol,
Condor will no longer append the fully qualified
host name to the value.
This makes it possible to configure a high availability
job queue that works with the remote submission of jobs.

\end{itemize}

\noindent Configuration Variable Additions and Changes:

\begin{itemize}

\item Added configuration variable: \Macro{LOCK\_FILE\_UPDATE\_INTERVAL}.
Please see page~\pageref{param:LockFileUpdateInterval} for a complete
description.

\item Changed the default value of configuration variable
\Macro{SEC\_DEFAULT\_SESSION\_DURATION} from 8640000 seconds (100 days)
to 86400 seconds (1 day).

\end{itemize}

\noindent Bugs Fixed:

\begin{itemize}

\item Fixed a bug in the \Condor{c-gahp} that caused it to fail repeatedly
on Windows, if more than two Condor-C jobs were submitted at the same time.

\item Fixed a problem that caused the \Condor{collector}'s memory usage
to increase dramatically, if \Condor{findhost} was run repeatedly.

\item Fixed a bug where Windows jobs suspended by Condor would never
be continued, despite log files indicating successful continuation.
This problem has existed since the 6.9.2 release of Condor.

%PR 937
\item Fixed a problem that could cause \Condor{dagman} to core dump
if straced, especially if the \File{dagman.out} file is on a shared
file system.

\item Fixed a problem introduced in 7.0.1 that could cause the \Condor{schedd}
daemon to crash when starting parallel or MPI universe jobs.  In some cases,
the problem would result in the following log message:

\footnotesize
\begin{verbatim}
ERROR ``Assertion ERROR on (mrec->request_claim_sock == sock)'' \
  at line 1361 in file dedicated_scheduler.C
\end{verbatim}
\normalsize

\item The \Condor{procd} daemon now periodically updates the timestamps on
the named pipe file system objects that it uses for communication.
This prevents these objects from being cleaned up by programs like
\Prog{tmpwatch}, which would result in Condor daemon exceptions.

\item Fixed a problem introduced in Condor 7.0.2 that would cause daemons
to fail on start up on Windows 2000.

\item Fixed a problem where standard universe jobs would fail to start
when using PrivSep, if the \Macro{PROCD\_ADDRESS} configuration variable was not
defined.

\item If the X509 proxy of a vanilla universe job has been refreshed, the
updated file will no longer be returned when the job completes.

\item If ClassAd attributes \Attr{StreamOut} or \Attr{StreamErr} are
missing from the job ClassAd of a grid universe job,
the default value for these attributes is now \Expr{False}.

\end{itemize}

\noindent Known Bugs:

\begin{itemize}

\item A bug in 7.0.4 affects jobs using Condor file transfer on submit
machines that are configured to deny write access from execute
machines.  The result is that output from jobs may fail to be copied
back to the submit machine.  The problem may or may not affect jobs
that run for less than eight hours, but it definitely will affect jobs
that run for more than eight hours.  An example of a configuration
vulnerable to this problem is one where DAEMON level access is allowed
to all execute nodes but WRITE level access is not.  When the problem
happens, the \Condor{shadow} log will contain a line like the following:

\begin{verbatim}
DaemonCore: PERMISSION DENIED to unknown user from host ...
for command 61001 (FILETRANS_DOWNLOAD), access level WRITE
\end{verbatim}

The workaround for this problem is to allow WRITE access from the
execute nodes.  If the existing configuration requires WRITE access to
be authenticated, then simply add WRITE access by the authenticated
condor identities associated with all execute nodes.  If WRITE access
is not currently required to be authenticated, then allow
unauthenticated WRITE access from all worker nodes.  Note that this
does \emph{not} imply that execute nodes will be able to modify the
job queue without authenticating.  Remote commands that modify the job
queue (for example, \Condor{submit} or \Condor{qedit}) always require that the
user be authenticated, no matter what configuration options are used;
if no method of remote authentication can succeed in the pool for
WRITE operations, then commands that modify the job queue can only run
on the submit machine.

\end{itemize}

\noindent Additions and Changes to the Manual:

\begin{itemize}

\item None.

\end{itemize}

%%%%%%%%%%%%%%%%%%%%%%%%%%%%%%%%%%%%%%%%%%%%%%%%%%%%%%%%%%%%%%%%%%%%%%
\subsection*{\label{sec:New-7-0-3}Version 7.0.3}
%%%%%%%%%%%%%%%%%%%%%%%%%%%%%%%%%%%%%%%%%%%%%%%%%%%%%%%%%%%%%%%%%%%%%%

\noindent Release Notes:

\begin{itemize}

\item This is a bug fix release.  A bug in Condor version 7.0.2 sometimes caused
the \Condor{schedd} to become unresponsive for 20 seconds when starting
the \Condor{shadow} to run a job.
Therefore, anyone running 7.0.2 is strongly encouraged to upgrade.

\end{itemize}


\noindent New Features:

\begin{itemize}

\item None.

\end{itemize}

\noindent Configuration Variable Additions and Changes:

\begin{itemize}

\item The configuration variable \Macro{VALID\_SPOOL\_FILES} now automatically 
includes \File{SCHEDD.lock},
the lock file used for high availability \Condor{schedd} fail over.  Other
high availability lock files are not currently included.

\end{itemize}

\noindent Bugs Fixed:

\begin{itemize}

\item Fixed a problem sometimes causing minutes or more of lag between
the time of job suspension or unsuspension and the corresponding entries
in the job user log.

\item Fixed a problem in \Condor{q} \Opt{-better-analyze} handling
requirements expressions containing  the expression \Expr{=!= UNDEFINED}.

\item Configuration variable \Macro{GRIDMANAGER\_GAHP\_CALL\_TIMEOUT}
is now recognized for nordugrid grid universe jobs.

\item Fixed a bug that could cause the \Condor{schedd} daemon to abort
and restart some time after a graceful restart,
when jobs to which the \Condor{schedd} daemon reconnected were preempted.

\item Fixed a bug causing failure to reconnect to jobs which use
\Expr{\$\$([\textit{expression}])}
in their ClassAds.  The jobs would go on
hold with the hold reason:
\AdStr{Cannot expand \$\$([\textit{expression}]).}

\item Fixed a bug in Condor version 7.0.2 that sometimes caused 
the \Condor{schedd} daemon to become
unresponsive for 20 seconds when starting the \Condor{shadow} daemon
to run a job.

\end{itemize}

\noindent Known Bugs:

\begin{itemize}

\item None.

\end{itemize}

\noindent Additions and Changes to the Manual:

\begin{itemize}

\item See 
  section~\ref{sec:WebService-Implementation}
  for documentation on finding the port number the \Condor{schedd} daemon
  is listening on for use with the web service API.

\end{itemize}


%%%%%%%%%%%%%%%%%%%%%%%%%%%%%%%%%%%%%%%%%%%%%%%%%%%%%%%%%%%%%%%%%%%%%%
\subsection*{\label{sec:New-7-0-2}Version 7.0.2}
%%%%%%%%%%%%%%%%%%%%%%%%%%%%%%%%%%%%%%%%%%%%%%%%%%%%%%%%%%%%%%%%%%%%%%

\noindent Release Notes:

\begin{itemize}

\item On Unix, Condor no longer requires its \Macro{EXECUTE} directory to
be world-writable, as long as it is not on a root-squashed NFS mount and is
owned by the user given in the \Macro{CONDOR\_IDS} setting (or by Condor's
real UID, if not started as \Login{root}). Condor will automatically remove
world-writability from existing \MacroNI{EXECUTE} directories where possible.
Note: The \MacroNI{EXECUTE} directory has never been required to be
world-writable on Windows.

\item With this release, a binary package for IA64 SUSE Linux Enterprise 8
will no longer be made available.

\end{itemize}


\noindent New Features:

\begin{itemize}

\item A clipped port to FreeBSD 7.0 x86 and x86\_64 is available, but at this
time, it is not available for download as a binary package.

\item Previously, \Condor{q} \Opt{-better-analyze} was supported on most
but not all versions of Linux.  It is now supported on all Unix platforms,
but not yet on Windows platforms.

\end{itemize}

\noindent Configuration Variable Additions and Changes:

\begin{itemize}

\item The new configuration variable
\Macro{GRIDMANAGER\_MAX\_WS\_DESTROYS\_PER\_RESOURCE} limits the number
of simultaneous WS destroy commands issued to a given server for grid
universe jobs of type gt4. The default value is 5.

\end{itemize}

\noindent Bugs Fixed:

\begin{itemize}

\item Fixed a bug in the standard universe where if a Linux machine was
  configured to use the Network Service Cache Daemon (nscd), taking
  a checkpoint would be deferred indefinitely.

\item Fixed a bug that caused the Quill daemon to crash.

\item Fixed bug that prevented Quill, when running on a
  Windows host, from successfully updating the database.

\item Fixed a bug that prevented Quill's \Condor{dbmsd} daemon from proper
  shutting down upon request when running on Windows platforms.

% condor-admin 17847
\item Fixed a bug that caused Stork to be completely broken.

\item As a back port from Condor versions 7.1,
  the Windows Installer is now completely
  internationalized: it will no longer fail to install because of a
  missing "Users" group; instead, it will use the regionally appropriate
  group.

\item As a back port from Condor versions 7.1,
  interoperability with Samba (as a PDC) has been improved.
  Condor uses a fast form of login during credential validation.
  Unfortunately, this login procedure fails under Samba,
  even if the credentials are valid.  The new behavior is to attempt
  the fast login, and on failure, fall back to the slower form.

\item As a back port from Condor versions 7.1,
  Windows slot users no longer have the
  Batch Privilege added, nor does Condor first attempt a Batch login
  for slot users.  This was causing permission problems on hardened
  versions of Windows, such as Windows Sever 2003, in that not
  interactive users lacked the permission to run batch files 
  (via the \Prog{cmd.exe} tool). 
  This affected any user submitting jobs that used
  batch files as the executable.

\item Fixed a bug that could sometimes cause the \Condor{schedd}
  to either EXCEPT or crash shortly after a user issues a \Condor{rm}
  command with the \Opt{-forcex} option.

\item \Condor{history} in a Quill environment,
  when given the \Opt{-constraint} option,
  would ignore attributes from the vertical schema.  This has been fixed.

\item In Unix, when started as \Login{root},
  the \Condor{master} now changes the
  effective user id back to \Login{root} (instead of condor)
  when restarting itself.
  This occurs for example due to the command \Condor{restart}.
  This makes no difference unless the \Condor{master} is wrapped
  with a script, and the script expects to be run as \Login{root}
  not only on initial start up, but on restart as well.

\item The dedicated scheduler would sometimes take two negotiation cycles
  to acquire all the machines it needed to run a job.
  This has been now fixed.

% PR 938
\item \Condor{dagman} no longer prints "Argument added" and
  "Retry Abort Value" diagnostic messages at the default verbosity,
  to reduce the size of the \File{dagman.out} file and the start up time
  for very large DAGs.

\item \Condor{dagman} now prints a few fatal parse errors at lower
  verbosity settings than it did previously.

\item \Condor{preen} no longer deletes \Prog{MyProxy} password files in the
  Condor spool directory.

\item When using TCP updates (UDP updates are the default), the
  \Condor{collector} would sometimes freeze for 20 seconds when
  receiving an invalidation notice.  
  The notice is received when Condor is being turned off
  on a machine in the pool.

\item Fixed a case in which the \Condor{schedd}'s job queue log file
could get corrupted when encountering errors writing to the disk such
as `out of space'.  This type of corruption was detected by the
\Condor{schedd} the next time it restarted and read the file to
restore the job queue, so you would only have been affected by this
problem if your \Condor{schedd} refused to start up until you fixed or
removed the job queue log file.  This bug has existed in all versions
of Condor, but it became more likely to occur in 6.9.4.

\item The configuration setting \MacroNI{JAVA} may now contain spaces.
Previously, this did not work.

\item Fixed a problem that caused occasional failure to detect hung
Condor daemons.

\item Fixed a file descriptor leak in the negotiator.  The leak happened
  whenever the negotiator failed to initiate the NEGOTIATE command to
  a \Condor{schedd}, for example if security negotiation failed
  with the \Condor{schedd}.
  Under Unix, this would eventually cause the \Condor{negotiator} to run out of
  file descriptors, exit, and restart.  This bug affected all previous
  versions of Condor.

\item Fixed several bugs in the user log reader that caused it to
  generate an invalid persisted state if no events had been read in.
  When read back in, this persisted state would cause the reader to
  segfault during initialization.

\item Fixed a bug causing communication problems if different portions
of a Condor pool were configured with different values of
\MacroNI{SEC\_DEFAULT\_SESSION\_DURATION}.  This bug affects all
previous versions of Condor.  The client side of the connection was
always using its own security session duration, even if the server's
duration was shorter.  Among other potential problems, this was
observed to cause file transfer failures when the starter was
configured with a longer session duration than the shadow.

\item Fixed a bug in the user log writer that was causing the writing
  of events to the global event log fail in some conditions.

\item In the grid universe, submission of nordugrid jobs is now properly
throttled by configuration parameters
\Macro{GRIDMANAGER\_MAX\_SUBMITTED\_JOBS\_PER\_RESOURCE} and
\Macro{GRIDMANAGER\_MAX\_PENDING\_SUBMITS\_PER\_RESOURCE}.

\item The NorduGrid GAHP server can now properly extract job execution
information from newer NorduGrid servers. Previously, the GAHP could crash
when talking to newer servers.

\item Fixed a bug that caused \Condor{config\_val} \Opt{-set} or
  \Opt{-rset} to fail if security negotiation was turned off.
  This happens, for example, if
  \Expr{SEC\_DEFAULT\_NEGOTIATION = NEVER}.
  This bug was introduced in Condor 7.0.0.

\item Fixed a bug that could cause incorrect IP addresses to be advertised
when the \Condor{collector} was on a multi-homed host.

\item Fixed a problem where unexpected ownership and permissions on files
inside a job's working directory could cause the \Condor{starter} to EXCEPT.

\item Improved the speed at which the \Condor{startd} can handle claim
requests, particularly when the \Condor{startd} manages a large number
of slots.

\item Fixed an error in the way the \Condor{procd} calculates image size for
jobs that involve multiple processes. Previously the maximum image size for
any single process was being used. Now the image size sum across all
processes is used.

\item The \Condor{procd} no longer truncates its log file on start up.
  Enabling a log file for the \Condor{procd} is only recommended for
  debugging, since it is not rotated to conserve disk space.

\item Fixed a problem present in Condor 7.0.1 and 7.1.0 where the
\Condor{startd} will crash upon deactivating or releasing a COD claim.

\item Condor on Windows can now correctly handle job image size when
processes are created that allocate more than 2GB of address space.

\item The \Macro{JOB\_INHERITS\_STARTER\_ENVIRONMENT} setting now works when
the \Macro{GLEXEC\_STARTER} feature is in use.

\item Fixed a problem causing \Condor{schedd} to perform poorly when
handling large job queues in which there are any idle local or
scheduler universe jobs (for example, Condor cron jobs).

\item Sped up \Condor{schedd} graceful shutdown when disconnecting
from running jobs that have job leases.  Previously, it would only
disconnect from one such job at a time, so if there were a lot of jobs
running, \Condor{schedd} could take so long to shut down that job leases
expire before it has a chance to restart and reconnect to the jobs.

\item Fixed a bug that could cause incorrect IP addresses to be advertised
when the \Condor{collector} was on a multi-homed host.

\end{itemize}

\noindent Known Bugs:

\begin{itemize}

\item None.

\end{itemize}

\noindent Additions and Changes to the Manual:

\begin{itemize}

\item None.

\end{itemize}


%%%%%%%%%%%%%%%%%%%%%%%%%%%%%%%%%%%%%%%%%%%%%%%%%%%%%%%%%%%%%%%%%%%%%%
\subsection*{\label{sec:New-7-0-1}Version 7.0.1}
%%%%%%%%%%%%%%%%%%%%%%%%%%%%%%%%%%%%%%%%%%%%%%%%%%%%%%%%%%%%%%%%%%%%%%

\noindent Release Notes:

\begin{itemize}

\item Fixed a bug in Condor's authorization policy reader.  The bug
affects cases where the policy (\MacroNI{ALLOW}/\MacroNI{DENY} and
\MacroNI{HOSTALLOW}/\MacroNI{HOSTDENY} settings) mixes host-based
authorizations with authorizations that refer to the authenticated
user name.  In some cases, this bug would result in host-based
settings not being applied to authenticated users.

\end{itemize}

\noindent New Features:

\begin{itemize}

\item Support for Backfill Jobs is now available on Windows platforms.
For more information on this, please see
section~\ref{sec:Backfill-BOINC-Windows} on
page~\pageref{sec:Backfill-BOINC-Windows}.

\item Condor has been ported to Red Hat Enterprise Linux
5.0 running on the 32-bit x86 architecture and on the 64-bit x86\_64
architecture.

% This feature was added in 6.7, but condor_submit wasn't changed.
% Until now, the user had to set this via the '+' notation in the submit
% file.
\item The command \SubmitCmd{email\_attributes} in a job submit
description file defines a set of job ClassAd attributes whose values
should be included in the e-mail notification of job completion.

\item The configuration variable \Macro{CONDOR\_VIEW\_HOST} may now
contain a port number, and may refer to a
\Condor{collector} daemon running on the same host as the
\Condor{collector} that is forwarding ClassAds.  It is also now possible to
use the forwarded ClassAds for matchmaking purposes.  For example, several
\Condor{collector} daemons could forward ClassAds to 
a single aggregating \Condor{collector} daemon which
a \Condor{negotiator} then uses as its source of information for
matchmaking.

\item \Condor{configure} and \Condor{install} now detect missing
  shared libraries (such as \File{libstdc++.so.5} on Linux), and print
  messages and exit if missing libraries are detected.  The new command
  line option \Opt{--ignore-missing-libs} causes it not to exit
  after the messages have been printed, and to proceed with the
  installation.

\item Added a \Opt{--force} command line option to \Condor{configure}
  (and \Condor{install}) which will turn on \Opt{--overwrite} and
  \Opt{--ignore-missing-libs}.

\item \Condor{configure} now writes simple sh and csh shell scripts
  which can be sourced by their respective shells to set the user's
  \Env{PATH} and \Env{CONDOR\_CONFIG} environment variables.  By default, these
  are created in the root of the Condor installation, but this can be
  changed via the \Opt{--env-scripts-dir} command line option.  Also,
  the creation of these scripts can be disabled with the
  \Opt{--no-env-scripts} command line option.

\end{itemize}

\noindent Configuration Variable Additions and Changes:

\begin{itemize}

\item The new configuration variables \Macro{PREEMPTION\_REQUIREMENTS\_STABLE}
  and \Macro{PREEMPTION\_RANK\_STABLE} are boolean values to
  identify whether or not attributes used within the definition of
  \Macro{PREEMPTION\_REQUIREMENTS} and \Macro{PREEMPTION\_RANK} remain
  unchanged during a negotiation cycle.
  See section~\ref{param:PreemptionRequirementsStable} on
  page~\pageref{param:PreemptionRequirementsStable} for 
  complete definitions.

\item The configuration variable \Macro{STARTER\_UPLOAD\_TIMEOUT}
  changed its default value to 300 seconds.

\item The new configuration variable \Macro{CKPT\_PROBE} 
specifies an internal to Condor
executable which determines information about how a process is laid out
in memory, in addition to other information. This executable is not yet
available on Windows platforms.

\item The new configuration variable 
\MacroNI{CKPT\_SERVER\_CHECK\_PARENT\_INTERVAL} sets an interval
of time between checks by the checkpoint server to see if 
its parent, the \Condor{master} daemon, has gone away unexpectedly.
The checkpoint server shuts itself down if this happens.
The default interval for checking is 120 seconds.
Setting this parameter to 0 disables the check.

\end{itemize}

\noindent Bugs Fixed:

\begin{itemize}

\item Upgrade from PCRE v5.0 to PCRE v7.6, due to security vulnerabilities 
found in PCRE v5.0.

\item Fixed file descriptor leak in the \Condor{schedd} when using the SOAP
interface.

\item Fixed a bug that primarily affected pools with
\MacroNI{MaxJobRetirementTime} (0 by default) set larger than
\MacroNI{REQUEST\_CLAIM\_TIMEOUT} (30 minutes by default).  Since
6.9.3, when the \Condor{schedd} timed out requesting a claim to a slot, the
\Condor{startd} was not made aware of the canceled request.  This
resulted in some wasted time (up to \MacroNI{ALIVE\_INTERVAL}) in
which the \Condor{startd} would wait for a job to run.

\item A problem with \Condor{history} in a Quill environment incorrectly
interpreting the \Opt{-name} option has been fixed.

\item A memory leak that prevented \Condor{load\_history} from running
with large history files has been fixed.

\item A bug in \Condor{history} when running in a quill environment has been fixed.  This bug would cause the command to crash in some situations.

\item The job ClassAd attribute \Attr{EmailAttributes} now works 
for grid universe jobs.

\item On 32-bit Linux platforms, the job queue database file may now exceed 2GB.
Previously, the \Condor{schedd} would halt with an error when trying
to write past the 2GB mark.

\item On 32-bit Linux platforms, \Condor{history} can now read from history
files larger than 2GB \emph{except} when using the \Opt{-backwards} option.

\item Local universe jobs are now scheduled to run more promptly.  Previously,
new local universe jobs would sometimes take up to \MacroNI{SCHEDD\_INTERVAL}
(default 5 minutes) to be considered for running.

\item The memory usage of the \Condor{collector} used to grow over time if
daemons with new names kept joining and then leaving the pool
(for example, in a Glidein pool).
This was due to statistics on dropped updates that
accumulated for all daemons that ever advertised themselves to the
\Condor{collector}.  These statistics are now periodically purged of
information about daemons which have not reported in a long time.  How
long is controlled by \Macro{COLLECTOR\_STATS\_SWEEP}, which
defaults to 2 days.

\item Condor daemons would die when trying to send ClassAd
advertisements to a host name that could not be resolved by DNS.

\item Since 6.9.5, file transfer errors for vanilla, java, or parallel
jobs would sometimes not result in the job going on hold as it should.
This was most likely for very small files that failed to be written
for some reason.

\item The \AdAttr{ImageSize} reported for jobs on AIX was too big by a factor
of 1024.

\item Since 6.9.5, \Condor{glidein} failed in the set up stage, due to the
change in syntax of quoting rules
in the Condor submit description file for gt2 argument strings.

\item Fixed a bug in the \Condor{gridmanager} that could prevent refreshed
X509 proxies from being forwarded to the remote machine for grid universe
jobs of type gt4.

\item Fixed a bug in Condor's authorization policy reader.  The bug
affects cases where the policy (\MacroNI{ALLOW}/\MacroNI{DENY} and
\MacroNI{HOSTALLOW}/\MacroNI{HOSTDENY} settings) mixes host-based
authorizations with authorizations that refer to the authenticated
user name.  In some cases, this bug would result in host-based
settings not being applied to authenticated users.

\item Fixed a bug in \Condor{history} which causes a crash 
when \Condor{quill} is enabled.

\item Fixed a problem affecting the GSI and SSL authentication
methods.  When these methods successfully authenticated the user but
failed to find a mapping of the X509 name to a condor user id, they
were setting the authenticated name to \verb|gsi| and \verb|ssl|
respectively.  However, these names contain no domain, so they could
not be referred to in the authorization policy.  Now these anonymous
mappings are \verb|gsi@unmappeduser| and \verb|ssl@unmappeduser|.
Therefore, configuration to deny access by users who are not explicitly mapped
in the map file appears as:

\begin{verbatim}
DENY_READ = *@unmappeduser
DENY_WRITE = *@unmappeduser
\end{verbatim}

\end{itemize}

\noindent Known Bugs:

\begin{itemize}

\item When using \Condor{compile} with the RHEL5 x86 port of Condor to
produce a standard universe executable, one will see a warning message
about how linking with dynamic libraries is not portable. This warning
is erroneous and should be ignored. It will be fixed in a future version
of Condor.

\end{itemize}

\noindent Additions and Changes to the Manual:

\begin{itemize}

\item The existing configuration variables 
\Macro{SYSTEM\_PERIODIC\_HOLD}, \Macro{SYSTEM\_PERIODIC\_RELEASE}, and
\Macro{SYSTEM\_PERIODIC\_REMOVE} have documented definitions.
See section~\ref{param:SystemPeriodicHold} for definitions.

\item A manual page for \Condor{load\_history} has been added.

\end{itemize}


%%%%%%%%%%%%%%%%%%%%%%%%%%%%%%%%%%%%%%%%%%%%%%%%%%%%%%%%%%%%%%%%%%%%%%
\subsection*{\label{sec:New-7-0-0}Version 7.0.0}
%%%%%%%%%%%%%%%%%%%%%%%%%%%%%%%%%%%%%%%%%%%%%%%%%%%%%%%%%%%%%%%%%%%%%%

\noindent Release Notes:

\begin{itemize}

\item PVM support has been dropped.

\item The time zone for the \Prog{PostgreSQL} 8.2 database
  used with Quill on Windows machines must be explicitly set
  to use an abbreviation.
  This Windows environment variable is \verb@TZ@.
  Proper abbreviations for the value of this variable may be found
  within the \Prog{PostgreSQL} installation in a file,
  \File{share/timezonesets/<continent>.txt}, where
  \verb@<continent>@ is replaced by the continent of the 
  desired time zone.

\end{itemize}


\noindent New Features:

\begin{itemize}

\item The Windows MSI installer now supports VM Universe.

\item Eliminated the ``tarball in a tarball'' in our distribution.
  The contents of \File{release.tar} from the distribution tarball
  (for example, \File{condor-6.9.6-linux-x86-centos45-dynamic.tar.gz}) is now
  included in the distribution tarball.

\item Updated \Condor{configure} to match the above change.  The
  \Opt{--install} option now takes a directory path as its parameter,
  for example \Opt{--install=/path/to/release}.
  It previously took the path to
  the \File{release.tar} tarball.

\item Added \Condor{install}, which is a symlink to \Condor{configure}.
  Invoking 
\begin{verbatim}
    condor_install
\end{verbatim}
  is identical to running
\begin{verbatim}
    condor_configure --install=.
\end{verbatim}

\item Added the option \Opt{--prefix=dir} to \Condor{configure} and
  \Condor{install}.  This is an alias for \Opt{--install-dir=dir}.

\item Added the option \Opt{--backup} option to \Condor{configure} and
  \Condor{install}.  This option renames the target \File{sbin} directory,
  if the \Condor{master} daemon exits while in the target \File{sbin} directory.
  Previous versions of \Condor{configure} did this by default.

\item Changed the default behavior of \Condor{install} to exit with a
  warning if the target \File{sbin} directory exists,
  the \Condor{master} daemon is in the \File{sbin} directory,
  and neither the \Opt{--backup} nor \Opt{--overwrite} options are specified.
  This prevents \Condor{install} from improperly moving an \File{sbin}
  directory out of the way.
  For example,
\begin{verbatim}
    condor_install --prefix=/usr
\end{verbatim}
  will not move \File{/usr/sbin} out of the way unless
  the \Opt{--backup} option is also specified.

\item Updated the usage summary of \Condor{configure} and
  \Condor{install} to be much more readable.

\end{itemize}

\noindent Configuration Variable Additions and Changes:

\begin{itemize}

\item The new configuration variable
  \Macro{DEAD\_COLLECTOR\_MAX\_AVOIDANCE\_TIME} defines the maximum
  time in seconds that a daemon will fail over from a primary
  \Condor{collector} to a secondary \Condor{collector}.
  See section~\ref{param:DeadCollectorMaxAvoidanceTime} on
  page~\pageref{param:DeadCollectorMaxAvoidanceTime} for a
  complete definition.

\end{itemize}

\noindent Bugs Fixed:

\begin{itemize}

\item Fixed a memory leak in the \Condor{procd} daemon on Windows.

\item Fixed a problem that could cause Condor daemons to crash if a
  failure occurred when communicating with the \Condor{procd}.

\item Fixed a couple of problems that were preventing the
  \Condor{startd} from properly removing per-job directories
  when running with PrivSep.

\item The \Condor{startd} will no longer fail to initialize, 
  claiming the \MacroNI{EXECUTE} directory has improper permissions,
  when PrivSep is enabled.

\item Look ups of ClassAd attribute \Attr{CurrentTime} are now
  case-insensitive, just like all other attributes.

\item Fixed problems causing the following error message in the log file:

\footnotesize
\begin{verbatim}
ERROR: receiving new UDP message but found a short message still waiting to be closed (consumed=1). Closing it now.
\end{verbatim}
\normalsize

\item The existence of the executable given in the submit file is now 
  enforced (when transferring the executable and not using VM 
  universe).

\item The copy of \Condor{dagman} that ships with Condor is now automatically 
  added to the list of trusted programs in the Windows Firewall.

\item Removed \SubmitCmd{remove\_kill\_sig} from the submission file
  generated by \Condor{submit\_dag} on Windows.

\item Fixed the algorithm in the \Condor{negotiator} daemon, which
  with large numbers of machine ClassAds (for example, 10,000) 
  was causing long delays at the 
  beginning of each negotiation cycle.

\item Use of \MacroNI{MAX\_CONCURRENT\_UPLOADS} was resulting in a
  connection attempt from the \Condor{shadow} to the \Condor{schedd} with a
  fixed 10 second timeout, which is sometimes too small.  This timeout
  has been increased to be the same as other connection timeouts between
  the \Condor{shadow} and the \Condor{schedd}, and it now respects
  \MacroNI{SHADOW\_TIMEOUT\_MULTIPLIER}, so it can be adjusted if necessary.

\item Fixed a problem with \Macro{MAX\_CONCURRENT\_UPLOADS} and
  \Macro{MAX\_CONCURRENT\_DOWNLOADS}, which was sometimes allowing
  more than the configured number of concurrent transfers to happen.

\item Fixed a bug in the \Condor{schedd} that could cause it to crash due
  to file descriptor exhaustion when trying to send messages to hundreds of
  \Condor{startd}s simultaneously.

\item Fixed a 6.9.4 bug in the \Condor{startd} that would cause it to crash
  when a BOINC backfill job exited.

\item Since 6.9.4, when using glExec, configuring \MacroNI{SLOT<N>\_EXECUTE}
  would cause \Condor{starter} to fail when starting the job.

\item Fixed a bug from 6.9.5 which caused authentication failure for
  the pool password authentication method.

\item Fixed a bug that caused Condor daemons to crash when encountering
  some types of invalid ClassAd expressions.

\item Fixed a bug under Linux that could cause multi-process daemons
  lacking a log lock file to crash while rotating logs that have reached
  their maximum configured size.

\item Fixed a bug under Windows that sometimes caused connection attempts
  between Condor daemons to fail with Windows error number 10056.

\item Fixed a problem in which there are multiple 
  \Condor{collector} daemons in a pool
  for fault tolerance.  If the primary \Condor{collector} failed, the
  \Condor{negotiator} would fail over to the secondary \Condor{collector}
  indefinitely (or until the secondary \Condor{collector} also failed or the
  administrator ran \Condor{reconfig}).  This was a problem for users
  flocking jobs to the pool, because flocking currently only works with
  the primary \Condor{collector}.  Now, the \Condor{negotiator} will fail over
  for a restricted amount of time, up to
  \Macro{DEAD\_COLLECTOR\_MAX\_AVOIDANCE\_TIME} seconds.  The default
  is one hour, but if querying the dead primary \Condor{collector}
  takes very little
  time to fail, the \Condor{negotiator} may retry more frequently
  in order to remain
  responsive to flocked users.

\item Fixed a problem preventing the use of \Condor{q} \Opt{-analyze}
  with the \Opt{-pool} option.

\item Fixed a problem in the \Condor{negotiator} in which machines go
  unassigned when user priorities result in the machines getting split
  into shares that are rounded down to 0.  For example if there are 10
  machines and 100 equal priority submitters, then each submitter was
  getting 0.1 machines, which got rounded down to 0, so no machines were
  assigned to anybody.  The message in the \Condor{negotiator} log in this case
  was this:

\footnotesize
\begin{verbatim}
Over submitter resource limit (0) ... only consider startd ranks
\end{verbatim}
\normalsize

\item Fixed a problem introduced in 6.9.3 that would cause daemons to
  run out of file descriptors if they create sub-processes and are
  configured to use a lock file for the debug log.

\item Standard universe jobs now work properly when using PrivSep.

\item Fixed problem with PrivSep mode where a job that dumps core would
  not get the core file transferred back to the the submit host if the
  \SubmitCmd{transfer\_output\_files} submit option were used.

\item Fixed a bug that caused the \Condor{starter} to crash if a job
called \Condor{chirp} with the \Expr{get\_job\_attr} option.

\end{itemize}

\noindent Known Bugs:

\begin{itemize}

\item None.

\end{itemize}

\noindent Additions and Changes to the Manual:

\begin{itemize}

\item None.

\end{itemize}


% Oct 2009, as we release 7.4, Karen commented out inclusion of the
% 6.9 and 6.8 histories
%%%%      PLEASE RUN A SPELL CHECKER BEFORE COMMITTING YOUR CHANGES!
%%%      PLEASE RUN A SPELL CHECKER BEFORE COMMITTING YOUR CHANGES!
%%%      PLEASE RUN A SPELL CHECKER BEFORE COMMITTING YOUR CHANGES!
%%%      PLEASE RUN A SPELL CHECKER BEFORE COMMITTING YOUR CHANGES!
%%%      PLEASE RUN A SPELL CHECKER BEFORE COMMITTING YOUR CHANGES!

%%%%%%%%%%%%%%%%%%%%%%%%%%%%%%%%%%%%%%%%%%%%%%%%%%%%%%%%%%%%%%%%%%%%%%
\section{\label{sec:History-6-9}Development Release Series 6.9}
%%%%%%%%%%%%%%%%%%%%%%%%%%%%%%%%%%%%%%%%%%%%%%%%%%%%%%%%%%%%%%%%%%%%%%

This is the development release series of Condor.
The details of each version are described below.

%%%%%%%%%%%%%%%%%%%%%%%%%%%%%%%%%%%%%%%%%%%%%%%%%%%%%%%%%%%%%%%%%%%%%%
\subsection*{\label{sec:New-6-9-5}Version 6.9.5}
%%%%%%%%%%%%%%%%%%%%%%%%%%%%%%%%%%%%%%%%%%%%%%%%%%%%%%%%%%%%%%%%%%%%%%

\noindent Release Notes:

\begin{itemize}

\item The suggested configuration value for
\MacroNI{SHADOW\_RENICE\_INCREMENT} has been changed from 10 to 0.  If
you have the old value in an existing configuration file, we recommend
changing it.  This improves performance of Condor on busy submit nodes
where other processes would cause low priority shadows to become
starved for CPU time.

\end{itemize}


\noindent New Features:

\begin{itemize}

\item Added new Windows specific ClassAd attributes:
 \begin{itemize}
 \item \AdAttr{WindowsMajorVersion}
 \item \AdAttr{WindowsMinorVersion}
 \item \AdAttr{WindowsBuildNumber}
 \end{itemize}
For more information, please see their descriptions in 
section~\ref{user-man-jobad} on page~\pageref{user-man-jobad}.

\item Added new authorization levels to allow fine-grained control
over the security settings that are used by the collector when
receiving ClassAd updates by different types of daemons:
\DCPerm{ADVERTISE\_MASTER}, \DCPerm{ADVERTISE\_STARTD}, and
\DCPerm{ADVERTISE\_SCHEDD}.  An example of what you can do with this
is to require that all \Condor{startds} that join the pool be
authenticated with a pool password and exist within a restricted set
of IP addresses, while schedds may join the pool from a broader set
of IP addresses and must authenticate with X509 credentials.

\item Condor-C now uses a more efficient protocol when querying the
status of jobs from Condor 6.9.5 and newer \Condor{schedd} daemons.

\item Added 4 new counters to the job ClassAd:
 \begin{itemize}
 \item \AdAttr{NumJobStarts}
 \item \AdAttr{NumJobReconnects}
 \item \AdAttr{NumShadowExceptions}
 \item \AdAttr{NumShadowStarts}
 \end{itemize}
For more information, please see their descriptions in
section~\ref{user-man-jobad} on page~\pageref{user-man-jobad}.

\item Added a new attribute, \AdAttr{GridJobStatus}, to the ClassAds of
\SubmitCmd{grid} universe jobs. This string shows the job's status as reported
by the remote job management system.

\item \Condor{q} \Opt{-analyze} now shows the full hold reason for jobs
that are on hold.

\item Increased efficiency of \Condor{preen} when there are large
numbers of jobs in the job queue.  Without this, the \Condor{schedd}
would become unresponsive for a long time (e.g. 10 minutes with 20,000
jobs in the queue) whenever \Condor{preen} was activated.

\item A 6.9.5 \Condor{q} can now query an older \Condor{quill} daemon directly 
for job information.

\item Reduced memory requirements of \Condor{shadow}.

\item Added the ability to \Condor{submit} to list unused or unexpanded 
  variables in submission file.

% Gnats PR 372.
\item Added the capability to assign priorities to DAG nodes.  Ready nodes
within a DAG are submitted in priority order by \Condor{dagman}.

% Gnats PR 852
\item Added the capability to assign categories to DAG nodes, and
throttle submission of node jobs by category.

\item \MacroNI{USE\_CLONE\_TO\_CREATE\_PROCESSES} (which defaults to
True) is now supported on ppc64, SUSE 9.  This also fixes a bug in which
the Yellow Dog Linux version of Condor installed on a ppc64 SUSE 9 machine
would fail to start jobs.

\item When the \Condor{preen} sends email about old files being found, it
now includes the name of the machine and the number of files found in the
subject of the message.

\item The user log reading code is now able to handle global event log
  rotations correctly.  The API is backwards compatible, but with
  several new method, it is able to invisibly handle rotated eventlog
  files.

\item The user log writer code now generates a header record (as a
  ``generic'' event) with some meta information to the event log.
  This header is not written to the ``user log'', only to the global
  event log.  Some of the information stored in this header is used by
  the enhanced log reader (see above) to more reliably detect rotated
  log files.

\end{itemize}

\noindent Configuration Variable Additions and Changes:

\begin{itemize}

\item The new variable \Macro{WANT\_UDP\_COMMAND\_SOCKET} controls
  if Condor daemons should create a UDP command socket in addition to
  the TCP command socket (which is required).
  The default is \Expr{True}, but it is now possible to completely
  disable UDP command sockets by defining this to \Expr{False}.
  See section~\ref{param:WantUDPCommandSocket} on
  page~\pageref{param:WantUDPCommandSocket} for more information.

\item The new variable \Macro{NEGOTIATOR\_INFORM\_START} controls if
  the \Condor{negotiator} should inform the \Condor{startd} when it
  has been matched with a job.
  The default is \Expr{True}.
  See section~\ref{param:NegotiatorInformStartd} on
  page~\pageref{param:NegotiatorInformStartd} for more information.

\item The new variable \Macro{SHADOW\_LAZY\_QUEUE\_UPDATE} controls if
  the \Condor{shadow} should immediately update the job queue for
  certain attributes (for example, the new \AdAttr{NumJobStarts} and
  \AdAttr{NumJobReconnects} counters) or if it should wait and only
  update the job queue on the next periodic update.
  The default is \Expr{True} to do lazy, periodic updates.
  See section~\ref{param:ShadowLazyQueueUpdate} on
  page~\pageref{param:ShadowLazyQueueUpdate} for more information.

\item The new variable \Macro{WARN\_ON\_UNUSED\_SUBMIT\_FILE\_MACROS} 
  controls if \Condor{submit} should warn when there are unused or
  unexpanded variables in a submit file.  The default is \Expr{True}
  to list unused or unexpanded variables.

\end{itemize}

\noindent Bugs Fixed:

\begin{itemize}

\item Patched the parallel universe scripts lamscript and mp1script
so that they work with newer versions of the GNU textutils.

\item Fixed a bad bug in the standard universe--introduced in Condor
6.9.4,  which would cause corruption of any binary data being written to
any fd the application opened. If an application only writes ASCII data
to an fd, the application will not encounter this bug.

\item The \Condor{negotiator} now prints out the value of the configuration
	parameter \MacroNI{PREEMPTION\_REQUIREMENTS} if it is set.  Previously,
	it always logged that it was unset, even when it was.

\item Fixed bug in the master that occurred if the collector
	was configured to use an ephemeral command port 
	(i.e. by explicitly setting the port to 0).	  The collector
	is now more reliable in this situation.

\item Fixed a bug introduced in Condor 6.9.4 that caused grid universe
jobs of type gt4 to not work when the Condor daemons were started as
root and any file transfer was associated with the job.

\item Fixed a bug introduced in Condor 6.9.4 that caused the
\Condor{gridmanager} to exit immediately on startup when the Condor
daemons were started as root and a condor username didn't exist.

\item Removed race condition that was causing the \Condor{schedd} to core
  dump on Windows when the Condor service was stopped.

\item When grid universe jobs of grid-type condor, lsf, or pbs are running,
\Condor{q} will now show the correct accumulated runtime.

\item When removing grid universe jobs of type gt2 that have just finished
executing, the chance of encountering Globus GRAM error 31 (the job manager
failed to cancel the job as requested) is now much reduced.

\item Fixed a problem introduced in 6.9.4: the \Condor{schedd} would hang
when given a constraint with \Condor{hold} that included jobs that the user
did not have permission to modify.

\item Fixed a problem from 6.9.4 in which the schedd would not
relinquish a claimed startd after reconnecting to a disconnected job.
After the job finished, the startd would remain in the claimed idle
state until the claim lease expired (20 minutes by default).

\item Applied the \MacroNI{QUERY\_TIMEOUT} to fix problem where the schedd 
would block for a long time when doing negotiation with flocked or HAD 
negotiator, and one of the collectors was not routeable (for instance, when 
the machine is powered off). Previously, there was no time-out and would
result in a the schedd waiting for the connection attempt to fail, which may
take a long time.

\item If both \MacroNI{EXECUTE\_LOGIN\_IS\_DEDICATED} and
\MacroNI{DEDICATED\_EXECUTE\_ACCOUNT\_REGEXP} are defined, the latter
now takes precedence, whereas previously the reverse was true.

\item Fixed a problem on Windows where if
\Macro{STARTD\_RESOURCE\_PREFIX} was set to anything besides
\texttt{slot} (the default), all jobs would run using the
\Username{condor-reuse-slot1} account, regardless of the actual slot
used for execution. This problem existed in versions 6.9.3 and 6.9.4
of Condor.

\item Undocumented ``DAGMan helper'' functionality has been removed
  due to disuse

\item Reworked Condor's detection of CPUs and ``Hyper Threads'' under
  Linux.  It now correctly detects these on all machines that we've
  been able to test against.  No configuration changes are involved in
  this fix.

% Gnats PR 668
\item When a standard universe job becomes held due to user job policy or
a version mismatch, a hold reason is now set in the job ad.

\item Invalid \MacroNI{QUILL\_DB\_TYPE} settings could result in a 
segmentation fault in \Condor{q}. 
Condor now ignores invalid settings and assumes \Prog{PostgreSQL}.

\item In rare cases, \Condor{reconfig} could cause \Condor{master} and
one of its children to become deadlocked.  This problem was only
possible with security negotiation enabled, and it has therefore
existed in all versions of Condor since security negotiation was
added.

\item Fixed a potential crash in the \Condor{starter} if it's told to
shutdown while it's disconnected from the \Condor{shadow}.

\item Fixed the global event log rotation code.  Previously, if two or
  more processes were concurrently writing to the event log, they
  didn't correctly detect that another writer process had rotated the
  file, and would do their own rotation, resulting in data loss.
 
\end{itemize}

\noindent Known Bugs:

\begin{itemize}

\item Condor on MacOSX 10.4 on the PowerPC architecture.
will report zero image size and resident set size for jobs. This
is due to bugs in the MacOSX 10.4 kernel on the PowerPC.

\end{itemize}


%%%%%%%%%%%%%%%%%%%%%%%%%%%%%%%%%%%%%%%%%%%%%%%%%%%%%%%%%%%%%%%%%%%%%%
\subsection*{\label{sec:New-6-9-4}Version 6.9.4}
%%%%%%%%%%%%%%%%%%%%%%%%%%%%%%%%%%%%%%%%%%%%%%%%%%%%%%%%%%%%%%%%%%%%%%

\noindent Release Notes:

\begin{itemize}

\item The default in standard universe for \SubmitCmd{copy\_to\_spool}
is now \Expr{true}.  In 6.9.3, it was changed to \Expr{false} for all
universes for performance reasons, but this is deemed too risky for
standard universe, because any modification of the executable is
likely to make it impossible to resume execution using checkpoint
files made from the original version of the executable.

\item Version 1.5.0 of the Generic Connection Broker (GCB) is
  now used for building Condor.
  This version of GCB fixes a few critical bugs.
  \begin{itemize}
    \item GCB was unable to pass information about sockets registered
      at a GCB broker to child processes due to a bug in the way a
      special environment variable was being set.
    \item All sockets for outbound connections were being registered
      at the GCB broker, which was putting severe strain on the GCB
      broker even under relatively low load.
      Now, only sockets that are listening for inbound connections are
      registered at the broker.
    \item The \Macro{USE\_CLONE\_TO\_CREATE\_PROCESSES} setting was
      causing havoc for applications linked with GCB.
      This configuration setting is now always disabled if GCB is enabled.
    \item Fixed a race condition in GCB\_connect() that would
      frequently cause connect() attempts to fail, especially
      non-blocking connections.
    \item Fixed bugs in GCB\_select() when GCB changes the direction
      of a connection from active to passive (for example, so that a
      Condor daemon running behind a firewall will use an outbound
      connection to communicate with a public client that had
      attempted to initiate contact via the GCB broker).
    \item Also improved logging at the GCB broker.
  \end{itemize}

  Additionally, there was a bug in how Condor was publishing the
  classified ads for GCB-enabled daemons.
  Condor used to be re-writing any attributes containing an IP address
  when a classified ad was sent over a network connection (in an
  effort to provide correct behavior for multi-homed machines).
  Now, this re-writing is disabled whenever GCB is enabled, since GCB
  already has logic to determine the correct IP addresses to advertise.

  For more information about GCB, see section~\ref{sec:GCB} on
  page~\pageref{sec:GCB}. 

\item The owner of the log file for the \Condor{gridmanager}
  has changed to the \Login{condor} user.
  In Condor 6.9.3 and previous versions, it was owned by the
  user submitting the job.
  Therefore, the owner of and permissions on an existing log file 
  are likely to be incorrect.
  Condor issues an error if the \Condor{gridmanager} is unable
  to read and write the existing file.
  To correct the problem, an administrator may modify file 
  permissions such that the \Login{condor} user may read and
  write the log file. 
  Alternatively, an administrator may delete the file, and
  Condor will create a new file with the expected owner and
  permissions.
  In addition, the definition for \Macro{GRIDMANAGER\_LOG}
  in the \File{condor\_config.generic} file has changed for
  Condor 6.9.4.

\end{itemize}


\noindent New Features:

\begin{itemize}

\item Condor has been ported to Yellow Dog 5.0 Linux on the
PPC architecture. This port of Condor will also run on the
Sony Playstation 3 running said distribution of Linux.

\item Enhanced the standard universe to utilize Condor's privilege separation
mechanism.

\item Implemented a completely new version of Quill.  Quill can now
record information about all the daemons into a relational database.
See section~\ref{sec:Quill} for details on Quill.

\item Jobs in the mpi universe now can have \$\$ expanded in their
ads in the same way as other universes.

\item Added the \SubmitCmd{vm} universe, to facilitate
running jobs under Xen or VMware virtual machines.

\item Added the \Opt{-subsystem} command to \Condor{status} that queries all
ClassAds of a given type.

\item Improved the speed at which the \Condor{schedd} writes to its database
file \File{job\_queue.log} and the job history file.  In benchmark tests,
this roughly doubles the maximum throughput rate to approximately 
20 jobs per second, although actual
performance depends on the specific hardware used.

\item The \Condor{startd} now records historical statistics about the
  total time (in seconds) that it spends in every state/activity pair.
  If a given slot spent more than 0 seconds in any of the possible
  pairs, the specifically-named ClassAd attribute for that
  pair is defined in the slot's ClassAd.
  The list of possible new machine attributes (alphabetically):
\begin{verbatim}
TotalTimeBackfillBusy
TotalTimeBackfillIdle
TotalTimeBackfillKilling
TotalTimeClaimedBusy
TotalTimeClaimedIdle
TotalTimeClaimedRetiring
TotalTimeClaimedSuspended
TotalTimeMatchedIdle
TotalTimeOwnerIdle
TotalTimePreemptingKilling
TotalTimePreemptingVacating
TotalTimeUnclaimedBenchmarking
TotalTimeUnclaimedIdle
\end{verbatim}

\item The \Condor{shadow} now waits and retries after failing to
commit the final update to the job ClassAd in the \Condor{schedd}'s
job queue, rather than immediately aborting and causing the job to be
requeued to run again.  See
page~\pageref{param:ShadowMaxJobCleanupRetries} for the related
configuration options.

\item If the \Condor{starter} fails with a core dump on Unix, the
core dump file is now put in the LOG directory.  Previously, it
was deleted by the \Condor{startd}.

\item Added a small amount of randomization to the default values of
\MacroNI{PERIODIC\_CHECKPOINT} (in the example config file) and
\MacroNI{PASSWD\_CACHE\_REFRESH} (in Condor's internal default) in
order to decreases the chances of synchronized timing across many
processes causing overloading of servers.

\item Added the new submit command \SubmitCmd{cron\_window}. It is an alias
to \SubmitCmd{deferral\_window}.

\item Optimized the submission of grid-type gt4 grid universe jobs to the
remote resource.Submission now takes one operation instead of three.

\item Added new functionality for multi-homed machines (those with
  multiple network interfaces) to allow Condor to handle private
  networks in some cases without having to use the Generic Connection
  Broker (GCB).
  See the entries below that describe the new
  \MacroNI{PRIVATE\_NETWORK\_NAME} and
  \MacroNI{PRIVATE\_NETWORK\_INTERFACE} configuration variables.

\end{itemize}

\noindent Configuration Variable Additions and Changes:

\begin{itemize}

\item Added \MacroNI{SLOTx\_EXECUTE}.  This allows the execute directory
to be configured independently for each batch slot.  You could use this,
for example, to have jobs on a multi-CPU machine using scratch space on
different disks so that there is less chance of them interfering with
each other.  See page~\pageref{param:SlotXExecute} for more details.

\item The semantics of \MacroNI{SLOT\_TYPE\_<N>} has changed slightly.
Previously, any resource shares left undefined would default to a fractional
share equal to \Expr{1/NUM\_CPUS}.  Now, the default is \Expr{auto}, which
causes all remaining resources to be evenly divided.  This is more convenient
in cases where some slots are configured to take more or less than their
``fair'' share and the rest are desired to evenly split the remainder.
The underlying reason for this change was to be able to better support
the specification of disk partition shares in all the possible cases:
The \Expr{auto} share takes into account how many slots are sharing the
same disk partition.

\item When set to \Expr{True}, the new configuration variable
\Macro{LOGS\_USE\_TIMESTAMP} will cause Condor to print all daemon
log messages using a Unix timestamp instead of a formatted date
string. This feature is useful for debugging Condor Glideins that may
be executing in different timezones. It should be noted that this does
not affect job user logs. The default is \Expr{False}.

\item The existing configuration variable \Macro{LOG\_ON\_NFS\_IS\_ERROR} has 
changed behavior. When set to \Expr{False},
\Condor{submit} does not emit a warning about user logs files being on NFS.

% Gnat PR 652
\item The existing 
configuration variables \Macro{DAEMON\_LIST},
\Macro{DC\_DAEMON\_LIST}, and \Macro{MASTER\_HA\_LIST} 
have changed behavior.
Trailing commas are now ignored.
Previously, trailing commas could cause the \Condor{master} to misbehave,
including exiting with an error.

\item The \Macro{<SUBSYS>\_DAEMON\_AD\_FILE} was defined for the \Condor{schedd}.
This setting was first made available in Condor 6.9.1 but was not used for
any daemon.  
It appears in the configuration file as \Macro{SCHEDD\_DAEMON\_AD\_FILE} and
is set to the file .schedd\_classad in the LOG directory. This setting is 
not necessary unless you are using the Quill functionality, and pools may 
upgrade to 6.9.4 without setting it if they are not using Quill. 
  
\item Added new configuration variables
\Macro{DBMSD}, \Macro{DBMSD\_ARGS}, and \Macro{DBMSD\_LOG},
which define the location of the \Condor{dbmsd} daemon,
the command line arguments to that daemon,
and the location of the daemon's log.
Default values are
\begin{verbatim}
DBMSD = $(SBIN)/condor_dbmsd
DBMSD_ARGS = -f
DBMSD_LOG = $(LOG)/DbmsdLog
\end{verbatim}
These configuration variables are only necessary when using Quill,
and then must be defined on only one machine in the Condor pool.

\item Added new configuration variables
\Macro{PRIVATE\_NETWORK\_NAME} and
\Macro{PRIVATE\_NETWORK\_INTERFACE},
which allow Condor daemons to function more properly on multi-homed
machines and in certain network configurations that involve private
networks.
There are no default values, both must be explicitly defined to have
any effect.
See section~\ref{param:PrivateNetworkName} on
page~\pageref{param:PrivateNetworkName} for more information about
these two new settings.

\item Added new configuration variables
  \Macro{EVENT\_LOG}, \Macro{MAX\_EVENT\_LOG}, \Macro{EVENT\_LOG\_USE\_XML}, 
  and \Macro{EVENT\_LOG\_JOB\_AD\_INFORMATION\_ATTRS}
  to specify the new event log which logs job user log events,
  but across all users.
  See section~\ref{param:EventLog} on
  page~\pageref{param:EventLog} for definitions of these configuration
  variables.
\end{itemize}


\noindent Bugs Fixed:

\begin{itemize}

\item Trailing commas in lists of items in submit files and 
configuration files are now ignored.  Previously, Condor would treat trailing
commas in various surprising ways.

\item Numerous bugs in GCB and the interaction between Condor and GCB.
  See the release notes above for details.

\item The submit file entry ``coresize'' was not being honored properly on
many universe. It is now honored on all universes except pvm and the grid
universes (except where the grid type is Condor). For the java universe,
it controls the core file size for the JVM itself.

\item The \Condor{configure} installation script now allows Condor
to be installed on hosts without a fully-qualified domain name.

% Gnats PR 857/condor-admin 15687
\item Fixed a bug in \Condor{dagman}: if a DAG run with a per-DAG
configuration file specification generated a rescue DAG, the rescue
DAG file did not contain the appropriate DAG configuration file line.
(This bug was introduced when the per-DAG configuration file option
was added in version 6.9.2.)

\item Fixed a bug introduced in 6.9.3 when handling local universe jobs.
The starter ignored failures in contacting the \Condor{schedd} in the
final update to the job queue.

\item When the \Condor{schedd} is issued a graceful shutdown command, any jobs
that running with a job lease are allowed to keep running. When the \Condor{schedd}
starts back up at a later time, it will spawn \Condor{shadow} to reconnect
to the jobs if they are still executing. This mimics the same behavior as
a fast shutdown.  This also fixes a bug in 6.9.3 in which the \Condor{schedd}
would fail to reconnect to jobs that were left running during a graceful
shutdown.

\item When the \Condor{starter} is gracefully shutting down and if it
has become disconnected from the \Condor{shadow}, it will wait for the
job lease time to expire before giving up on telling the \Condor{shadow}
that the job was evicted.  Previously, the \Condor{starter} would exit
as soon as it was done evicting the job.

% Gnat PR 718
\item Job ad attribute \Attr{HoldReasonCode} is now properly set when
\Condor{hold} is called and when jobs are submitted on hold.

\item If a job specified a job lease duration, and the \Condor{schedd}
  was killed or crashed, the \Condor{shadow} used to notice when the
  \Condor{schedd} was gone, and gracefully shutdown the job (evicting
  the job at the remote site).
  Now, the \Condor{shadow} honors the job lease duration, and if the
  lease has not yet expired, it simply exists without evicting the
  job, in the hopes that the \Condor{schedd} will be restarted in time
  to reconnect to the still-running job and resume computation.

\item Fixed a bug from 6.9.3 in which \Condor{q} \Opt{-format} no longer
worked when given an expression (as opposed to simple attribute reference).
The expression was always treated as being undefined.

\item When a condor daemon such as the \Condor{schedd} or
\Condor{negotiator} tried to establish many new security sessions for
UDP messages in a short span of time, it was possible for the daemon
to run out of file descriptors, causing it to abort execution and be
restarted by the \Condor{master}.  A problem was found and fixed in the
mechanism that protects against this.

\item Improved error descriptions when Condor-C encounters failures when
sending input files to the remote schedd.

\item Rare failure conditions during stage in would cause Condor-C to put
the state of the job in the remote schedd into an invalid state in which
it would run but later fail during stage out.  This now results in the
job on the submit side going on hold with a staging failure.

\item Fixed a bug which could cause \Condor{store\_cred} to crash during
common use.

\item Fixed a bug where the vanilla universe \Condor{starter} would possibly
crash when running a job not as the owner of the job.

\item Fixed a bug which would cause a \Condor{starter} being used for
the local universe to core dump.

\item Fixed a bug which caused the \Condor{schedd} to core dump while
processing a job's crontab entries in the submit description file.

\item Fixed a privilege separation bug in the standard universe 
\Condor{starter}.

\end{itemize}

\noindent Known Bugs:

\begin{itemize}

\item \SubmitCmd{grid} universe jobs for the gt4 grid type do not work, 
if Condor daemons are started as root
and there is file transfer associated with or specified by the job.
These jobs are placed on hold.

\item The \Macro{STARTD\_RESOURCE\_PREFIX} setting on Windows results
in broken behavior on both Condor 6.9.3 and 6.9.4. Specifically, when
this setting is given a value other than its default (``slot''), all
jobs will run using the ``condor-reuse-slot1'' user account,
regardless of the actual slot used for execution.

\end{itemize}

\noindent Additions and Changes to the Manual:

\begin{itemize}

\item New documentation for the new \SubmitCmd{vm} universe
in the User's Manual, section~\ref{sec:vmuniverse}.
Definitions of configuration variables for the \SubmitCmd{vm} universe are in
section~\ref{sec:Config-VMs}.

\item New RDBMS schema tables added for Quill in
section~\ref{sec:Quill-Schema}.

\end{itemize}

%%%%%%%%%%%%%%%%%%%%%%%%%%%%%%%%%%%%%%%%%%%%%%%%%%%%%%%%%%%%%%%%%%%%%%
\subsection*{\label{sec:New-6-9-3}Version 6.9.3}
%%%%%%%%%%%%%%%%%%%%%%%%%%%%%%%%%%%%%%%%%%%%%%%%%%%%%%%%%%%%%%%%%%%%%%

\noindent Release Notes:

\begin{itemize}

\item As of version 6.9.3, the entire Condor system has undergone a
  major terminology change.
  For almost 10 years, Condor has used the term \Term{virtual machine}
  or \Term{vm} to refer to each distinct resource that could run a
  Condor job (for example, each of the CPUs on an SMP machine).
  Back when we chose this terminology, it made sense, since each of
  these resource was like an independent machine in a pool, with
  its own state, ClassAd, claims, and so on.
  However, in recent years, the term \Term{virtual machine} is now
  almost universally associated with the kinds of virtual machines
  created using tools such as VMware and Xen.  Entire operating systems
  run inside a given process, usually emulating the underlying
  hardware on a host machine.
  So, to avoid confusion with these other kinds of virtual machines,
  the old  \Term{virtual machine} terminology has been replaced by
  the term \Term{slot}.

  Numerous configuration settings, command-line arguments to Condor
  tools, ClassAd attribute names, and so on, have all been
  modified to reflect the new \Term{slot} terminology.
  In general, the old settings and options will still work, but are
  now retired and may disappear in the future.

\item The \Condor{install} installation script has
  been removed.
  All sites should use \Condor{configure} when setting up a new Condor
  installation.

\item The \Macro{SECONDARY\_COLLECTOR\_LIST} configuration variable has
  been removed.
  Sites relying on this variable should instead use the configuration
  variable \Macro{COLLECTOR\_HOST}. It may be used to
  define a list of \Condor{collector} daemon hosts.

\item Cleaned up and improved help information for \Condor{history}.

\end{itemize}


\noindent New Features:

\begin{itemize}

\item Numerous scalability and performance improvements.  Given enough
memory, the schedd can now handle much larger job queues (e.g. 10s of
thousands) without the severe degradation in performance that used to
be the case.

\item Added the \Macro{START\_LOCAL\_UNIVERSE} and \Macro{START\_SCHEDULER\_UNIVERSE}
parameters for the \Condor{schedd}. This allows administrators to control whether
a Local/Scheduler universe job will be started. This expression is evaluated
against the job's ClassAd before the \AdAttr{Requirements} expression.

\item All Local and Scheduler universe jobs now have their \AdAttr{Requirements} 
expressions evaluated before execution. If the expression evaluates to false, the
job will not be allowed to begin running. In previous versions of Condor, Local 
and Scheduler universe jobs could begin execution without the \Condor{schedd} checking
the validity of the \AdAttr{Requirements}.

\item Added \MacroNI{SCHEDD\_INTERVAL\_TIMESLICE} and
\MacroNI{PERIODIC\_EXPR\_TIMESLICE}.  These indicate the maximum
fraction of time that the schedd will spend on the respective
activities.  Previously, these activities were done on a fixed
interval, so with very large job queue sizes, the fraction of time
spent was increasing to unreasonable levels.

\item Under Intel Linux, added \Macro{USE\_CLONE\_TO\_CREATE\_PROCESSES}.
This defaults to true and results in scalability improvements for processes
using large amounts of memory (e.g. a schedd with a lot of jobs in the queue).

\item Jobs in the parallel universe now can have \$\$ expanded in their
ads in the same way as other universes.

\item Local universe jobs now support policy expression evaluation, which includes
the \AdAttr{ON\_EXIT\_REMOVE}, \AdAttr{ON\_EXIT\_HOLD}, \AdAttr{PERIODIC\_REMOVE},
\AdAttr{PERIODIC\_HOLD}, and \AdAttr{PERIODIC\_RELEASE} attributes. The periodic
expressions are evaluated at intervals determined by the
\Macro{PERIODIC\_EXPR\_INTERVAL} configuration macro.

\item Jobs can be scheduled to executed periodically, similar to the crontab
functionality found in Unix systems. The \Condor{schedd} calculates the next
runtime for a job based on the new \AdAttr{CRON\_MINUTE}, \AdAttr{CRON\_HOUR},
\AdAttr{CRON\_DAY\_OF\_MONTH}, \AdAttr{CRON\_MONTH}, and
\AdAttr{CRON\_DAY\_OF\_WEEK} attributes. A preparation time defined by the
\AdAttr{CRON\_PREP\_TIME} attribute allows a job to be submitted to the
execution machine before the actual time the job is to begin execution.
Jobs that would like to be run repeatedly will need to define the
the \AdAttr{ON\_EXIT\_REMOVE} attribute properly so that they are
re-queued after executing each time.

\item Condor now looks for its configuration file in \File{/usr/local/etc}
if the \Macro{CONDOR\_CONFIG} environment variable is not set and there is
no condor\_config file located in \File{/etc/condor}. This allows a default
Condor installation to be more compatible with Free BSD.

\item If a user job requests streaming input or output in the submit
file, the job can now run with job leases and the job will continue
to run for the lease duration should the submit machine crash.  Previously,
jobs with streaming i/o would be evicted if the submit machine crashed.
While the submit machine is down, if the job tried to issue a streaming
read or write, the job will block until the submit machine returns or the
job lease expires.

\item Ever since version 6.7.19, \Condor{submit} has added a default
  job lease duration of 20 minutes to all jobs that support these
  leases.
  However, there was no way to disable this functionality if a user
  did not want job lease semantics.
  Now, a user can place \verb@job_lease_duration = 0@ in their submit
  file to manually disable the job lease.

% condor-admin 15254
\item Added new configuration knob \Macro{STARTER\_UPLOAD\_TIMEOUT}
which sets the timeout for the starter to upload output files to the
shadow on job exit.  The default value is 200 seconds, which should
be sufficient for serial jobs.  For parallel jobs, this may need to
be increased if many large output files are sent back to the shadow
on job exit.

% Gnats PR 806
\item \Condor{dagman} now aborts the DAG on ``scary'' submit events.
These are submit events in which
the Condor ID of the event does not match the
expected value.
Previously, \Condor{dagman} printed a warning, but continued.
To restore Condor to the previous behavior,
set the new \Macro{DAGMAN\_ABORT\_ON\_SCARY\_SUBMIT} configuration variable
to \Expr{False}.

\item When the \Condor{master} detects that its GCB broker is unavailable
and there is a list of alternative brokers,
it will restart immediately if \Macro{MASTER\_WAITS\_FOR\_GCB\_BROKER} is
set to \Expr{False} instead of waiting for another broker to became available.
\Condor{glidein} now sets \MacroNI{MASTER\_WAITS\_FOR\_GCB\_BROKER}
to  \Expr{False} in its configuration file.

\item When using GCB and a list of brokers is available, the
\Condor{master} will now pick a random broker rather than the least-loaded
one.

\item All Condor daemons now evaluate some ClassAd expressions
  whenever they are about to publish an update to the
  \Condor{collector}.
  Currently, the two supported expressions are:
  \begin{description}
  \item[\Macro{DAEMON\_SHUTDOWN}]
    If \Expr{True}, the daemon will gracefully shut itself down and will not
    be restarted by the \Condor{master} (as if it sent itself a
    \Condor{off} command).
  \item[\Macro{DAEMON\_SHUTDOWN\_FAST}]
    If \Expr{True}, the daemon will quickly shut itself down and will not be
    restarted by the \Condor{master} (as if it sent itself a
    \Condor{off} command using the \Opt{-fast} option).
  \end{description}
  For more information about these expressions, see
  section~\ref{param:DaemonShutdown} on
  page~\pageref{param:DaemonShutdown}.

\item When the \Condor{master} sends email announcing that another daemon has
died, exited, or been killed, it now notes the name of the machine, the
daemon's name, and a summary of the situation in the Subject line.

\item Anyplace in a Condor configuration or submit description file where
wild cards may be used, you can now place wild cards at both the beginning
and end of the string pattern (i.e. match strings that contain the text
between the wild cards anywhere in the string). Previously, only one
wild card could appear in the string pattern.

\item Added optional configuration setting
\Macro{NEGOTIATOR\_MATCH\_EXPRS}.  This allows the negotiator to
insert expressions into the matched ClassAd.  See
page~\pageref{param:NegotiatorMatchExprs} for more information.

\item Increased speed of ClassAd parsing.

\item Added \Macro{DEDICATED\_EXECUTE\_ACCOUNT\_REGEXP} and
deprecated the boolean setting
\Macro{EXECUTE\_LOGIN\_IS\_DEDICATED}, because the latter could not
handle a policy where some jobs run as the job owner and some run as
dedicated execution accounts.  Also added support for
\Macro{STARTER\_ALLOW\_RUNAS\_OWNER} under Unix.  See
Section~\ref{param:DedicatedExecuteAccountRegexp} and
Section~\ref{sec:RunAsNobody} for more information.

\item All Condor daemons now publish a \Attr{MyCurrentTime} attribute
  which is the current local time at the time the update was generated
  and sent to the \Condor{collector}.
  This is in addition to the \Attr{LastHeardFrom} attribute which is
  inserted by the \Condor{collector} (the current local time at the
  collector when the update is received).

\item \Condor{history} now accepts partial command line
arguments.  For example, -constraint can be abbreviated -const.
This brings \Condor{history} in line with other Condor command
line tools.

\item \Condor{history} can now emit ClassAds formatted as XML with
the new -xml option.
This brings \Condor{history} more in line \Condor{q}.

\item The \verb@$$@ substitution macro syntax now supports the insertion
of literal \verb@$$@ characters through the use of \verb@$$(DOLLARDOLLAR)@.
Also, \verb@$$@ expansion is no longer recursive, so if the value being
substituted in place of a \verb@$$@ macro itself contains \verb@$$@ characters,
these are no longer interpreted as substitution macros but are instead
inserted literally.

\item When started as root on a Linux 64-bit x86 machine, Condor daemons will
now leave core files in the log directory when they crash.  This matches
Condor's behavior on most other Unix-like operating systems, including
32-bit x86 versions of Linux.
% Code: Google's coredumper library is now used on Linux x86-64.

\item The \Env{\_CONDOR\_SLOT} variable is now placed into the
  environment for jobs of all universes.
  This variable indicates what slot a given job is running on, and
  will have the same value as the \AdAttr{SlotID} from the machine
  classified ad where the job is running.
  The \Env{\_CONDOR\_SLOT} variable replaces the deprecated
  \Env{CONDOR\_VM} environment variable, which was only defined for
  standard universe jobs.

\item Added a \Macro{USE\_PROCD} configuration parameter. If this
parameter is set to true for a given daemon, the daemon will use the
\Condor{procd} program to monitor process families. If set to false,
the daemon will execute process family monitoring logic on its
own. The \Condor{procd} is more scalable and is also an essential
piece in the ongoing privilege separation effort. The disadvantage of
using the ProcD is that it is newer, less-hardened code.

\end{itemize}

\noindent Configuration Variable Additions and Changes:

\begin{itemize}

\item The \Macro{SECONDARY\_COLLECTOR\_LIST} configuration variable has
  been removed.
  Sites relying on this variable should instead use the configuration
  variable \Macro{COLLECTOR\_HOST} to
  define a list of \Condor{collector} daemon hosts.

\item Added new configuration variables \Macro{START\_LOCAL\_UNIVERSE}
  and \Macro{START\_SCHEDULER\_UNIVERSE} for the \Condor{schedd} daemon.
  These boolean expressions default to \Expr{True}.
  \MacroNI{START\_LOCAL\_UNIVERSE} is relevant only to local universe jobs.
  \MacroNI{START\_SCHEDULER\_UNIVERSE} is relevant only to scheduler 
  universe jobs.
  These new variables allow an administrator to define
  a \MacroNI{START} expression specific to these jobs. 
  The expression is evaluated
  against the job's ClassAd before the \AdAttr{Requirements} expression.

\item Added new configuration variables \Macro{SCHEDD\_INTERVAL\_TIMESLICE}
  and \Macro{PERIODIC\_EXPR\_TIMESLICE}.  These configuration variables
  address a scalability issue for very large job queues.
  Previously, the \Condor{schedd} daemon handled an activity related
  to counting jobs, as well as the activity related to evaluating
  periodic expressions for jobs at the fixed time interval of 5 minutes.
  With large job queues, the fraction of the \Condor{schedd} daemon
  execution time devoted to these two activities became excessive,
  such that it could be doing little else.
  The fixed time interval is now gone, and Condor calculates the amount
  of time spent on the two activities, using these new configuration
  variables to calculate an appropriate time interval.
  
  Each is a floating point value within the range
  (noninclusive) 0.0 to 1.0.
  Each determines the maximum fraction of the time interval that the 
  \Condor{schedd} daemon  will spend on the respective
  activity.
  \MacroNI{SCHEDD\_INTERVAL\_TIMESLICE} defaults to the value 0.05,
  such that the calculated time interval will be 20 * the amount
  of time spent on the counting jobs activity.
  \MacroNI{PERIODIC\_EXPR\_TIMESLICE} defaults to the value 0.01,
  such that the calculated time interval will be 100 * the amount
  of time spent on the periodic expression evaluation activity.

\item Added new configuration variable 
  \Macro{USE\_CLONE\_TO\_CREATE\_PROCESSES}, relevant only to the
  Intel Linux platform.  
  This boolean value defaults to \Expr{True}, and it results in scalability
  improvements for Condor processes using large amounts of memory.
  These processes may clone themselves instead of forking themselves.
  An example of the improvement occurs for a \Condor{schedd}
  daemon with a lot of jobs in the queue.

\item Added new configuration variable \Macro{STARTER\_UPLOAD\_TIMEOUT},
  which allows a configurable time (in seconds) for a timeout used by the 
  \Condor{starter}.
  The default value of 200 seconds replaces the previously hard coded
  value of 20 seconds.
  This timeout before job failure is to upload output files to the
  \Condor{shadow} upon job exit.
  The default value should be sufficient for serial jobs.
  For parallel jobs, it may need to
  be increased if there are many large output files.

\item Added new configuration variable \Macro{DAGMAN\_ABORT\_ON\_SCARY\_SUBMIT}.
  This boolean variable defaults to \Expr{True}, and causes
  \Condor{dagman} to abort the DAG on ``scary'' submit events.
  These are submit events in which
  the Condor ID of the event does not match the expected value.
  Previously, \Condor{dagman} printed a warning, but continued.
  To restore Condor to the previous behavior,
  set \MacroNI{DAGMAN\_ABORT\_ON\_SCARY\_SUBMIT} to \Expr{False}.

\item Added new configuration variable \Macro{NEGOTIATOR\_MATCH\_EXPRS}.
  It causes the \Condor{negotiator} to
  insert expressions into the matched ClassAd.  See
  page~\pageref{param:NegotiatorMatchExprs} for details.

\item Added new configuration variable
  \Macro{DEDICATED\_EXECUTE\_ACCOUNT\_REGEXP} to replace the retired 
  \Macro{EXECUTE\_LOGIN\_IS\_DEDICATED},
  because \MacroNI{EXECUTE\_LOGIN\_IS\_DEDICATED} could not
  handle a policy where some jobs run as the job owner and others run as
  dedicated execution accounts.  Also added support for
  the existing configuration variable
  \Macro{STARTER\_ALLOW\_RUNAS\_OWNER} under Unix.  See
  Section~\ref{param:DedicatedExecuteAccountRegexp} and
  Section~\ref{sec:RunAsNobody} for more information.

\item Added new configuration variable \Macro{USE\_PROCD}.
  This boolean variable defaults to \Expr{False} for the
  \Condor{master}, and \Expr{True} for all other daemons.
  When \Expr{True}, the daemon will use the
  \Condor{procd} program to monitor process families.
  When \Expr{False}, a daemon will execute process family
  monitoring logic on its own.
  The \Condor{procd} is more scalable and is also an essential
  piece in the ongoing privilege separation effort. The disadvantage of
  using the \Condor{procd} is that it is newer, less-hardened code.

\end{itemize}

\noindent Bugs Fixed:

\begin{itemize}

\item On Unix systems, Condor can now handle file descriptors larger than
FD\_SETSIZE when using the select system call. Previously, file descriptors
larger than FD\_SETSIZE would cause memory corruption and crashes.

\item When an update to the \Condor{collector} from the
\Condor{startd} is lost, it is possible for multiple claims to the
same resource to be handed out by the \Condor{negotiator}.  This is
still true.  What is fixed is that these multiple claims will not
result in mutual annihilation of the various attempts to use the
resource.  Instead, the first claim to be successfully requested will
proceed and the others will be rejected.

\item \Condor{glidein} was setting \Macro{PREEN\_INTERVAL}=0 in the default
configuration, but this is no longer a legal value, as of 6.9.2.

\item \Condor{glidein} was not setting necessary configuration parameters
for \Condor{procd} in the default glidein configuration.

\item In 6.9.2, Condor daemons crashed after failing to authenticate a
network connection.

\item \Condor{status} will now accurately report the \Attr{ActvtyTime}
  (activity time) value in Condor pools where not all machines are in
  the same timezone, or if there is clock-skew between the hosts.

\item Fixed the known issue in Condor 6.9.2 where using the
\Macro{EXECUTE\_LOGIN\_IS\_DEDICATED} setting on UNIX platforms would
cause the \Condor{procd} to crash.

\item Failure when activating a COD claim no longer will result in an
opportunistic job running on the same \Condor{startd} being left
suspended. This problem was most likely to be seen when using the
\Macro{GLEXEC\_STARTER} feature.

\item In Condor 6.9.2 for Tru64 UNIX, the \Condor{master} would
immediately fail if started as root. This problem has been fixed.

\item Condor 6.9.2 introduced a problem where the \Condor{master}
would fail if started as root with the UID part of the
\Macro{CONDOR\_IDS} parameter set to 0 (root). This issue has been
fixed.

\end{itemize}

\noindent Known Bugs:

\begin{itemize}

\item The 6.9.3 \Condor{schedd} daemon incorrectly handles jobs with leases
(true by default for vanilla, java, and parallel universe jobs) when
shutting down gracefully.  These jobs are allowed to continue running,
but when the \Condor{schedd} daemon is started back up, it fails to reconnect
to them.  The result is that the orphaned jobs are left running for
the duration of the job's lease time (a default time of 20 minutes).
The state of the jobs in the restarted queue is independent of any
orphaned running jobs, so these queued jobs may begin running on another
machine while orphans are still running.

\item \Condor{q} \Opt{-format} in 6.9.3 does not work with expressions.  It
behaves as if the expression evaluates to an undefined result.

\end{itemize}

%%%%%%%%%%%%%%%%%%%%%%%%%%%%%%%%%%%%%%%%%%%%%%%%%%%%%%%%%%%%%%%%%%%%%%
\subsection*{\label{sec:New-6-9-2}Version 6.9.2}
%%%%%%%%%%%%%%%%%%%%%%%%%%%%%%%%%%%%%%%%%%%%%%%%%%%%%%%%%%%%%%%%%%%%%%

\noindent Release Notes:

\begin{itemize}

%% This is important (and thus, I believe, worth of being a top
%% level release note) because it will surprise anyone upgrading
%% an existing pool or repackaging Condor binaries (say, for
%% custom glideins, or as .deb packages for a local pool.)
% This is part of the privilege separation work, but the procd
% is required even if you're not turning privsep on.
% Questions should go to the privsep team: psilord, zmiller, etc.
\item As part of ongoing security enhancements, Condor now has a
new, required daemon: \Condor{procd}.  This daemon is
automatically started by the \Condor{master}, you do not need to
add it to \Macro{DAEMON\_LIST}.  
However, you must be certain to update the \Condor{master}
if you update any of the other Condor daemons.
%Commented out the below since the defaults are in the code.
%New installations should not
%need to do anything; the default configuration file is correctly
%set.  Installations upgrading to 6.9.2 from previous versions
%will need to ensure several things are done.  
%1. Be sure to install \Condor{procd} into your Condor \Macro{SBIN} directory. 
%2. Add ``\Code{PROCD = \$(SBIN)/condor\_procd}'' to your Condor configuration. 
%3. Add ``\Code{PROCD\_ADDRESS = \$(LOCK)/procd\_pipe}'' to your Condor configuration. 
%On Windows there are two additional steps:
%4. Be sure to install \Condor{softkill} into your Condor \Macro{SBIN} directory. 
%5. Add ``\Code{WINDOWS\_SOFTKILL = \$(SBIN)/condor\_softkill}'' to your Condor configuration. 

% This isn't quite so important, but it's not really a feature or
% a bug, just a change.  It is a change that may surprise some
% users.  The full list of settings impacted is
% pretty long.  So far the below is just a small fraction,
% primarily being added because an external user was surprised by
% this when testing a 6.9.2-prerelease. For anyone curious or
% inspired to flesh out the list, here's the checkin that caused
% this:
% http://bonsai.cs.wisc.edu/bonsai/cvsquery.cgi?who=danb&whotype=match&sortby=Date&date=explicit&mindate=02%2F23%2F2007+19%3A15&maxdate=02%2F23%2F2007+19%3A30
% (To do the search yourself, search for checkins by danb between
% 02/23/2007 19:15 and 02/23/2007 19:30 )
% To determine if a variable is impacted, look at the removed
% code and confirm that it used the default if the setting was 0.
% Then if the new code sets a minimum of 1 (the third argument to
% param_integer), it's impacted.
\item Some configuration settings that previously accepted 0 no
  longer do so.  Instead the daemon using the setting will exit
  with an error message listing the acceptable range to its log.
  For these settings 0 was equivalent to requesting the default.
  As this was undocumented and confusing behavior it is no longer
  present.  To request a setting use its default, either comment it
  out, or set it to nothing (``\Code{EXAMPLE\_SETTING=}'').
  Setting impacted include but are not limited to: 
  % From condor_master.V6/master.C 1.82 to 1.83:
  \Macro{MASTER\_BACKOFF\_CONSTANT},
  \Macro{MASTER\_BACKOFF\_CEILING},
  \Macro{MASTER\_RECOVER\_FACTOR},
  \Macro{MASTER\_UPDATE\_INTERVAL},
  \Macro{MASTER\_NEW\_BINARY\_DELAY},
  \Macro{PREEN\_INTERVAL},
  \Macro{SHUTDOWN\_FAST\_TIMEOUT},
  \Macro{SHUTDOWN\_GRACEFUL\_TIMEOUT},
  % From condor_master.V6/daemon.C 1.68 to 1.69:},
  \Macro{MASTER\_<name>\_BACKOFF\_CONSTANT},
  \Macro{MASTER\_<name>\_BACKOFF\_CEILING},

\item Version 1.4.1 of the Generic Connection Broker (GCB) is
  now used for building Condor.  This version of GCB fixes a timing bug
  where a client may incorrectly think a network connection has been established,
  and also guards against an unresponsive client from causing a denial of
  service by the broker.
  For more information about GCB, see section~\ref{sec:GCB} on
  page~\pageref{sec:GCB}. 

% I'm checking this in commented since I'm not sure what disclosure
% policy we want to use. Only CDF (Igor) uses the GLEXEC_STARTER
% functionality, so I think it'd be wise to run it by him before
% documenting this publicly.
%
%\item Fixed a security vulnerability in the \Macro{GLEXEC\_STARTER}
%feature. In previous versions when the \Condor{startd} received the
%user proxy, it placed it in a temporary file that for a short window
%of time could be opened for reading by any user on the system.

\end{itemize}


\noindent New Features:

\begin{itemize}
\item On UNIX, an execute-side Condor installation can run without
root privileges and still execute jobs as different users, properly
clean up when a job exits, and correctly enforce policies specified by
the Condor administrator and resource owners. Privileged functionality
has been separated into a well-defined set of functions provided by a
setuid helper program. This feature currently does not work for the
standard or PVM universes.

%%\item added bogus ImageSize to bogus dedicated scheduler
%%jobAd used only for claiming.  This fixes some problems with
%%startd WANT_SUSPEND going to undefined, but we don't document
%%this bogus ad anywhere, so I'm not going to add it here.

\item Added support for EmailAttributes in the parallel universe.  
Previously, it was only valid in the vanilla and standard universes.

\item Added configuration parameter \Macro{DEDICATED\_SCHEDULER\_USE\_FIFO}
which defaults to true.  When false, the dedicated scheduler will
use a best-fit algorithm to schedule parallel jobs.  This setting is
not recommended, as it can cause starvation.  When true, the dedicated
scheduler will schedule jobs in a first-in, first-out manner.

\item Added \Opt{-dump} to \Condor{config\_val} which will print out
all of the macros defined in any of the configuration files found by
the program.
\Condor{config\_val} \Opt{-dump} \Opt{-v} will augment the output
with exactly what line and in what file each configuration variable
was found.
\Note: The output format of the \Opt{-dump} option will most likely
change in a future revision of Condor.

% Gnats PR 671
\item Node names in \Condor{dagman} DAG files can now be DAG
keywords, except for PARENT and CHILD.

\item Improved the log message when \Attr{OnExitRemove} or
\Attr{OnExitHold} evaluates to UNDEFINED.

% Gnats PR 796
\item Added the \Macro{DAGMAN\_ON\_EXIT\_REMOVE} configuration macro,
which allows customization of the \Attr{OnExitRemove} expression
generated by \Condor{submit\_dag}.

\item When using GCB, Condor can now be told to choose from a list of
brokers. \Macro{NET\_REMAP\_INAGENT} is now a space and comma separated
list of brokers. On start up, the \Condor{master} will query all of the
brokers and pick the least-used one for it and its children to use. If
none of the brokers are operational, then the \Condor{master} will wait
until one is working. This waiting can be disabled by setting 
\Macro{MASTER\_WAITS\_FOR\_GCB\_BROKER} to FALSE in the configuration file.
If the chosen broker fails and recovery is not possible or another broker
is available, the \Condor{master} will restart all of the daemons.

\item When using GCB, communications between parent and child
Condor daemons on the same host no longer use the GCB broker.
This improves scalability and also allows a single host to
continue functioning if the GCB broker is unavailable.

\item The \Condor{schedd} now uses non-blocking methods to send the
``alive'' message to the \Condor{startd} when renewing the job lease.
This prevents the \Condor{schedd} from blocking for 20 seconds while
trying to connect to a machine that has become disconnected from the
network.

\item \Condor{advertise} can read the classad to be advertised from
standard input.

\item Unix Condor daemons now reinitialize their DNS
configuration (e.g. IP addresses of the name servers) on reconfig.

% Gnats PR 777
\item A configuration file for \Condor{dagman} can now be specified
in a DAG file or on the \Condor{submit\_dag} command line.

\item Added \Condor{cod} option \Opt{-lease} for creation of COD claims
with a limited duration lease.  This provides automatic cleanup of COD
claims that are not renewed by the user.  The default lease is infinitely
long, so existing behavior is unchanged unless \Opt{-lease} is explicitly
specified.

\item Added \Condor{cod} command \Opt{delegate\_proxy} which will
delegate an x509 proxy to the requested COD claim.
This is primarily useful for sites wishing to use glexec to spawn the
\Condor{starter} used for COD jobs.
The new command optionally takes an \Opt{-x509proxy} argument to
specify the proxy file.
If this argument is not present, \Condor{cod} will search for the
proxy using the same logic as \Condor{submit} does.

% This is barely a feature, but it's definitely not a bug fix. It's
% more of a change in behavior.
\item \Macro{STARTD\_DEBUG} can now be empty, indicating a default, minimal
log level. It now defaults to empty.
Previously it had to be non-empty and defaulted to include D\_COMMAND.

\item The addition of the \Condor{procd} daemon means that all process
family monitoring and control logic is no longer replicated in each
Condor daemon that needs it. This improves Condor's scalability,
particularly on machines with many processes.

\end{itemize}

\noindent Bugs Fixed:

\begin{itemize}

\item Under various circumstances, condor 6.9.1 daemons would abort
with the message, ``ERROR: Unexpected pending status for fake message
delivery.''  A specific example is when \Attr{OnExitRemove} or
\Attr{OnExitHold} evaluated to UNDEFINED.  This caused the
\Condor{schedd} to abort.

\item In Condor 6.9.1, the \Condor{schedd} would die during startup
when trying to reconnect to running jobs for which the \Condor{schedd}
could not find a startd ClassAd.  This would happen shortly after
logging the following message: ``Could not find machine ClassAds for
one or more jobs.  May be flocking, or machine may be down.
Attempting to reconnect anyway.''

\item Improved Condor's validity checking of configuration values.
For example, in some cases where Condor was expecting an integer but
was given an expression such as 12*60, it would silently interpret
this as 12.  Such cases now result in the condor daemon exiting
after issuing an error message into the log file.

\item When sending a \Code{WM\_CLOSE} message to a process on Windows,
Condor daemons now invoke the helper program \Condor{softkill} to do
so. This prevents the daemon from needing to temporarily switch away
from its dedicated service Window Station and Desktop. It also fixes a
bug where daemons would leak Window Station and Desktop handles. This
was mainly a problem in the \Condor{schedd} when running many scheduler
universe jobs.

\end{itemize}

\noindent Known Bugs:

\begin{itemize}

\item \Condor{glidein} generates a default config file that sets
\Macro{PREEN\_INTERVAL} to an invalid value (0).  To fix this,
remove the setting of \MacroNI{PREEN\_INTERVAL}.

\item There are a couple of known issues with Condor's
\Macro{GLEXEC\_STARTER} feature when used in conjunction with
COD. First, the \Condor{cod} tool invoked with the
\Opt{delegate\_proxy} option will sometimes incorrectly report that the
operation has failed. In addition, the \MacroNI{GLEXEC\_STARTER}
feature will not work properly with COD unless the UID that the each
COD job runs as is different than the UID of the opportunistic job or
any other COD jobs that are running on the execute machine when the
COD claim is activated.

\item The \Macro{EXECUTE\_LOGIN\_IS\_DEDICATED} feature has been found
to be broken on UNIX platforms. Its use will cause the \Condor{procd}
to crash, bringing down the other Condor daemons with it.

\end{itemize}



%%%%%%%%%%%%%%%%%%%%%%%%%%%%%%%%%%%%%%%%%%%%%%%%%%%%%%%%%%%%%%%%%%%%%%
\subsection*{\label{sec:New-6-9-1}Version 6.9.1}
%%%%%%%%%%%%%%%%%%%%%%%%%%%%%%%%%%%%%%%%%%%%%%%%%%%%%%%%%%%%%%%%%%%%%%

\noindent Release Notes:

\begin{itemize}

\item The 6.9.1 release contains all of the bug fixes and enhancements
  from the 6.8.x series up to and including version 6.8.3.

\item Version 1.4.0 of the Generic Connection Broker (GCB) library is
  now used for building Condor, and it is the 1.4.0 versions of the
  \Prog{gcb\_broker} and \Prog{gcb\_relay\_server} programs that are
  included in this release.
  This version of GCB includes enhancements used by Condor
  along with a new GCB-related command-line tool:
  \Prog{gcb\_broker\_query}.
  Condor 6.9.1 will not work properly with older versions of the
  \Prog{gcb\_broker} or \Prog{gcb\_relay\_server}.
  For more information about GCB, see section~\ref{sec:GCB} on
  page~\pageref{sec:GCB}. 

\end{itemize}

\noindent New Features:

\begin{itemize}

\item Improved the performance of the ClassAd matching algorithm,
which speeds up the \Condor{schedd} and other daemons.

\item Improved the scalability of the algorithm used by 
the \Condor{schedd} daemon to find runnable jobs.
This makes a noticeable difference in \Condor{schedd} daemon performance,
when there are on the order of thousands of jobs in the queue.

\item the \Dflag{COMMAND} debugging level has been enhanced to
log many more messages. 

\item Updated the version of DRMAA, which contains several great
improvements regarding scalability and race conditions.

% Gnats PR 774
\item Added the \Macro{DAGMAN\_SUBMIT\_DEPTH\_FIRST} configuration macro,
which causes \Condor{dagman} to submit ready nodes in more-or-less depth-first
order, if set to \Expr{True}.  The default behavior is to submit
the ready nodes in breadth-first order.

\item Added configuration parameter \Macro{USE\_PROCESS\_GROUPS}.
If it is set to \Expr{False},
then Condor daemons on Unix machines will not create new 
sessions or process groups. This is intended for use with Glidein, as
we have had reports that some batch systems cannot properly track jobs that
create new process groups. The default value is \Expr{True}.

\item The default value for the submit file command
\SubmitCmd{copy\_to\_spool} has been changed to \Expr{False}, because copying
the executable to the spool directory for each job (or job cluster) is almost
never desired.  Previously, the default was \Expr{True} in all
cases, except for grid universe jobs and remote submissions.

\item More types of file transfer errors now result in the job going
on hold, with a specific error message about what went wrong.  The new
cases involve failures to write output files to disk on the submit
side (for example, when the disk is full).
As always, the specific error number is
recorded in \Attr{HoldReasonSubCode}, so you can enforce an automated
error handling policy using \SubmitCmd{periodic\_release} or
\SubmitCmd{periodic\_remove}.

\item Added the \Macro{<SUBSYS>\_DAEMON\_AD\_FILE}
configuration variable, which is similar to the 
\Macro{<SUBSYS>\_ADDRESS\_FILE}.
This new variable will be used in future versions of Condor, but is
not necessary for 6.9.1.


\end{itemize}

\noindent Bugs Fixed:

\begin{itemize}

\item Fixed a bug in the \Condor{master} so that it will now send obituary
e-mails when it kills child processes that it considers hung.

\item \Condor{configure} used to always make a personal Condor with
\Opt{--install} even when \Opt{--type} called for only execute or
submit types.  Now, \Condor{configure} honors the \Opt{--type}
argument, even when using \Opt{--install}.
If \Opt{--type} is not specified, the default is to still install a
full personal Condor with the following daemons: 
\Condor{master}, \Condor{collector},
\Condor{negotiator}, \Condor{schedd}, \Condor{startd}. 

\item While removing, putting on hold, or vacating a large number of
jobs, it was possible for the \Condor{schedd} and the \Condor{shadow} to
temporarily deadlock with each other.  This has been fixed under Unix,
but not yet under Windows.

\item Communication from a \Condor{schedd} to a \Condor{startd}
now occurs in a nonblocking manner.
This fixes the problem of the \Condor{schedd} blocking 
when the claimed machine running the \Condor{startd}
cannot be reached, for example because the machine is turned off.

\end{itemize}

\noindent Known Bugs:

\begin{itemize}

\item Under various circumstances, condor 6.9.1 daemons abort
with the message, ``ERROR: Unexpected pending status for fake message
delivery.''  A specific example is when \Attr{OnExitRemove} or
\Attr{OnExitHold} evaluated to UNDEFINED, which causes the
\Condor{schedd} to abort.

\item In Condor 6.9.1, the \Condor{schedd} will die during startup
when trying to reconnect to running jobs for which the \Condor{schedd}
can not find a startd ClassAd.  This happens shortly after
logging the following message: ``Could not find machine ClassAds for
one or more jobs.  May be flocking, or machine may be down.
Attempting to reconnect anyway.''

\end{itemize}

%%%%%%%%%%%%%%%%%%%%%%%%%%%%%%%%%%%%%%%%%%%%%%%%%%%%%%%%%%%%%%%%%%%%%%
\subsection*{\label{sec:New-6-9-0}Version 6.9.0}
%%%%%%%%%%%%%%%%%%%%%%%%%%%%%%%%%%%%%%%%%%%%%%%%%%%%%%%%%%%%%%%%%%%%%%

\noindent Release Notes:

\begin{itemize}

\item The 6.9.0 release contains all of the bug fixes and enhancements
  from the 6.8.x series up to and including version 6.8.2.

% and a few \condor{gridmanager} bug fixes from 6.8.3.  *sigh* we need
% a real solution to this problem (like pointing to issue node ids) ;)

\end{itemize}


\noindent New Features:

\begin{itemize}


\item Preliminary support for using \Prog{glexec} on execute machines
has been added.  This feature causes the \Condor{startd} to spawn the
\Condor{starter} as the user that \Prog{glexec} determines based on
the user's GSI credential.

\item A ``per-job history files'' feature has been added to the
\Condor{schedd}. When enabled, this will cause the \Condor{schedd} to
write out a copy of each job's ClassAd when it leaves the job
queue. The directory to place these files in is determined by the
parameter \Macro{PER\_JOB\_HISTORY\_DIR}. It is the responsibility of
whatever external entity (for example, an accounting or monitoring system) is
using these files to remove them as it completes its processing.

\item \Condor{chirp} command now supports writing messages to the user log.

\item \Condor{chirp} getattr and putattr now send all classad getattr
and putattr commands to the proc 0 classad, which allows multiple proc
parallel jobs to use proc 0 as a scratch pad.

\item Parallel jobs now support an \Attr{AllRemoteHosts} attribute,
which lists all the hosts across all procs in a cluster.

\item The \Macro{DAGMAN\_ABORT\_DUPLICATES} configuration macro (which causes
\Condor{dagman} to abort itself if it detects another \Condor{dagman}
running on the same DAG) now defaults to \Expr{True} instead of
\Expr{False}.

\end{itemize}

\noindent Bugs Fixed:

\begin{itemize}

\item None.

\end{itemize}

\noindent Known Bugs:

\begin{itemize}

\item None.

\end{itemize}


%%%%      PLEASE RUN A SPELL CHECKER BEFORE COMMITTING YOUR CHANGES!
%%%      PLEASE RUN A SPELL CHECKER BEFORE COMMITTING YOUR CHANGES!
%%%      PLEASE RUN A SPELL CHECKER BEFORE COMMITTING YOUR CHANGES!
%%%      PLEASE RUN A SPELL CHECKER BEFORE COMMITTING YOUR CHANGES!
%%%      PLEASE RUN A SPELL CHECKER BEFORE COMMITTING YOUR CHANGES!

%%%%%%%%%%%%%%%%%%%%%%%%%%%%%%%%%%%%%%%%%%%%%%%%%%%%%%%%%%%%%%%%%%%%%%
\section{\label{sec:History-6-8}Stable Release Series 6.8}
%%%%%%%%%%%%%%%%%%%%%%%%%%%%%%%%%%%%%%%%%%%%%%%%%%%%%%%%%%%%%%%%%%%%%%

This is a stable release series of Condor.
It is based on the 6.7 development series.
All new features added or bugs fixed in the 6.7 series are available
in the 6.8 series.
As usual, only bug fixes (and potentially, ports to new platforms)
will be provided in future 6.8.x releases.
New features will be added in the forthcoming 6.9.x development series.

%%%%%%
% we need a summary of major new features since 6.6.x here.  trying to
% sort through the 21 different 6.7.x releases and all the new
% features is a huge amount of noise.  people just want to see a
% summary of the major new functionality.
% \Todo
%%%%%%

The 6.8.x series supports a different set of platforms than 6.6.x.
Please see the updated table of available platforms in
section~\ref{sec:Availability} on page~\pageref{sec:Availability}.

The details of each version are described below.

%%%%%%%%%%%%%%%%%%%%%%%%%%%%%%%%%%%%%%%%%%%%%%%%%%%%%%%%%%%%%%%%%%%%%%
\subsection*{\label{sec:New-6-8-5}Version 6.8.5}
%%%%%%%%%%%%%%%%%%%%%%%%%%%%%%%%%%%%%%%%%%%%%%%%%%%%%%%%%%%%%%%%%%%%%%

\noindent Release Notes:

\begin{itemize}

\item The Globus libraries used by Condor now include the following advisory
packages:
  \begin{itemize}
  \item globus\_gss\_assist-3.23
  \item globus\_xio-0.35
  \item globus\_gram\_protocol-6.5
  \item globus\_gass\_transfer-2.12
  \end{itemize}
See \URL{http://www.globus.org/toolkit/advisories.html} for details on the
bugs fixed by these updated packages.
The patch given in Globus Bugzilla 5091
(\URL{http://bugzilla.mcs.anl.gov/globus/show\_bug.cgi?id=5091}) is also
included.

\end{itemize}


\noindent New Features:

\begin{itemize}

\item The functionality embodied in \Condor{q} \Opt{-better-analyze} is now
available for X86\_64 native ports of Condor.

\item We now supply distinct native ports for MacOSX 10.3 and 10.4.

\end{itemize}

\noindent Bugs Fixed:

\begin{itemize}

\item Removed periodic re-indexing of the quill history\_vertical table.
This should not be needed with the current schema, and should speed
up database re-indexing operations.

\item Fixed a bug that would cause the dedicated scheduler to
crash if the schedd was suspended or blocked for more than roughly 10 minutes.
The most likely cause of this suspension is if the schedd executable was
mounted from a remote NFS file system.

\item Fixed a bug where if \Opt{-lc} was specified multiple times for
the compiler when using \Condor{compile} (some tools like \Prog{pgf90}
do this), \Condor{compile} would fail to link the application and emit
a multiply defined symbol error for many symbols.

\item Fixed a bug where Condor erroneously indicates that a scheduler
universe's job executable is missing or not executable, if the scheduler
universe job had been submitted with
\SubmitCmd{CopyToSpool = false} in the submit
description file, and the user had a umask which prevented the user
named \Login{condor} from following the search path to the
user owned executable.

\item Fixed a bug that could cause the \Condor{schedd} to crash if it
received too many matches in one negotiation cycle (more than 1000 on a
Linux platform).

\item Fixed a bug where \Condor{history} did not honor the \Opt{-format} flag
properly when Quill was in use.

\item Fixed a bug where a java property which included surrounding double quotes
caused the detection of a java virtual machine to go awry. 
The fix, which may change in the future, changes any extra double quotes
within a property value to single quotes.

\item Fixed a bug in which the \Condor{quill} daemon 
crashed occasionally when the Postgres database
server was unavailable.

\item The Solaris 9 Condor package can be used under Solaris 10 again.
Changes in 6.7.20 broke this compatibility.

% PR 806
\item \Condor{dagman} now does a better job, especially in recovery mode,
of detecting potentially incorrect submit events.
Those have Condor IDs not matching what is expected.

% PR 814
\item \Condor{dagman} now truncates existing node job user log files
to zero length, rather than deleting the log files.  This prevents breaking the
link if a user log file is set up as a link.

\item When starting a GridFTP server to handle file transfers for gt4
grid jobs, the \Condor{gridmanager} now properly sets the
GLOBUS\_TCP\_PORT\_RANGE and GLOBUS\_TCP\_SOURCE\_RANGE environment
variables if appropriate.

\item Fixed a bug that could cause a security session to get deleted
by the server (for example, the \Condor{schedd}) before the client
(for example, the \Condor{shadow}) was done using it.
This bug can be observed as
communication failure the next time the client tried to connect to
the server.  In some cases, this caused jobs to be requeued to be run
again, because the final update of the job queue failed.

\item If a grid job becomes held while it's still submitted to the remote
resource and is then removed, the \Condor{gridmanager} will now attempt
to remove the job from the remote resource before letting it leave the
local job queue.

\item Fixed a bug in the \Condor{c-gahp} that caused it to not use the 
user's credential for authentication with the remote schedd on some 
connections.

\item The \Condor{c-gahp} now properly lists all of the commands it
supports in response to the COMMANDS command.

% Gnats PR 815
\item Removed the 5096-character restriction on the length of DAG
macro values (and names) in \Condor{dagman}.

\item Condor-G will now notice when jobs are missing from the status
reports sent by the Grid Monitor.
Jobs can disappear for short periods of time under normal circumstances,
but a prolonged absence is often a sign of problems on the remote machine.
The amount of time that a job can go missing from the Grid Monitor
status reports before the \Condor{gridmanager} reacts can be set by the
new configuration parameter \Macro{GRID\_MONITOR\_NO\_STATUS\_TIMEOUT}.
The default is 15 minutes.

\item \Condor{q} -analyze will now print a warning if a job being analyzed
is already completed or if a grid universe job being analyzed has already
been matched.

\item In \Condor{shadow}, when forwarding an updated X509 proxy to an
executing job, the logic for whether to delegate or copy the proxy 
(determined by configuration parameter
\Macro{DELEGATE\_JOB\_GSI\_CREDENTIALS}) was reversed.

\item Made a small improvement to the reliability of Condor's process
ancestry tracking under Linux.  However, jobs that create children
with more than 4096 bytes of environment are still problematic, due to
a Linux kernel limitation that prevents reading more than 4k from
/proc/<pid>/environ.  The only truly reliable way to ensure that
Condor is aware of all processes spawned by a unix job is to use
\MacroNI{VMx\_USER}.

\item \Condor{glidein} option \Opt{-run\_here} no longer fails when the
current working directory is not in PATH.

\item \Condor{glidein} option \Opt{-runtime} would cause runtime errors
at startup under some batch systems.  The problematic parentheses characters
are no longer generated as part of the environment value that is set by
this option.

\end{itemize}

\noindent Known Bugs:

\begin{itemize}

\item None.

\end{itemize}


%%%%%%%%%%%%%%%%%%%%%%%%%%%%%%%%%%%%%%%%%%%%%%%%%%%%%%%%%%%%%%%%%%%%%%
\subsection*{\label{sec:New-6-8-4}Version 6.8.4}
%%%%%%%%%%%%%%%%%%%%%%%%%%%%%%%%%%%%%%%%%%%%%%%%%%%%%%%%%%%%%%%%%%%%%%

\noindent Release Notes:

\begin{itemize}

\item None.

\end{itemize}


\noindent New Features:

\begin{itemize}

\item Added new tool \Condor{dump\_history} which will
enable schema migration to future Quill schema versions.

\item Quill can now automatically rebuild the indexes on the
\Prog{PostgreSQL} database tables.  Some sites reported that even
with auto vacuuming turned on, the indexes on the tables were 
growing without bounds.  Rebuilding the indexes fixes that problem.
Rebuilding is disabled by setting the parameter 
\Macro{QUILL\_SHOULD\_REINDEX}
to \Expr{False}.  Re-indexing happens immediately after the history file
is purged of old data. So, if Quill is configured to never delete
history data, the tables are never re-indexed.  Also, \Condor{quill}
was changed so that the history deletion also happens at start time.
This ensures that old history rows are deleted if Quill crashes
before the scheduled deletion time.

\item Added more information to StarterLog for an error message
involved in file transfers: 
\begin{verbatim}
Download acknowledgment missing attribute: Result.
\end{verbatim}
The extra information is a full dump of the
ClassAd that was received, in order to help determine why the expected
attribute was not found.

\item Added output to the \File{dagman.out} file documenting when
\Condor{dagman} shortcuts node retries because of \Condor{submit}
failures or a helper command failure.

\end{itemize}

\noindent Bugs Fixed:

\begin{itemize}

\item Fixed a bug in \Condor{q} that only happened when running 
with a Quill database and using the long (-l) option.  The bug was 
introduced in 6.8.3.  The bug truncated the output of \Condor{q}, and
only displayed some of the job attributes.

\item Fixed a bug in \Condor{submit} that caused standard universe jobs
to be unable to open their standard output or standard error, if
\SubmitCmd{should\_transfer\_files} is \SubmitCmd{YES} or 
\SubmitCmd{IF\_NEEDED} in the submit description file.

\item Fixed a bug in \Condor{glidein} that could cause it to request the
queue unknown when submitting its setup job to GRAM, leading to
failures.

\item The \Attr{OnExitRemove} expression generated for DAGMan by
\Condor{submit\_dag} evaluated to UNDEFINED for some values of
\Attr{ExitCode}, causing \Condor{dagman} to go on hold.

\item Fixed a bug in which garbage values (random bits from memory)
were sometimes
written to the pool history file in the field representing the
backfill state.

% Gnats PR 798
\item \Condor{submit\_dag} now generates a submit file
(\File{.condor.sub}) for \Condor{dagman} that sends \File{stdout} and
\File{stderr} to separate files.  This has always been recommended,
and recent versions of Condor cause \File{stdout} and \File{stderr} to
overwrite each other if they are directed to the same file.

\item Fixed several bugs for grid type \SubmitCmd{nordugrid} jobs.
The \Condor{gridmanager} would create an invalid RSL for these jobs
and save their output to the wrong location in some cases.

\item \Condor{glidein} now properly escapes glidein tarball URLs that
contain characters that have special meaning to GRAM RSL. It also turns
on TCP updates to the \Condor{collector},
if they are enabled on the submit machine.

\item When using the submit file option \SubmitCmd{getenv=true},
environment
variables containing a newline in their value are no longer inserted
into the job's environment.  The \Condor{schedd} daemon
does not allow newlines
within ClassAd values, so the attempt to insert such values resulted
in failure of job submission and caused the \Condor{schedd} daemon
to abort.

% Gnats PR 799
\item Fixed a bug that caused \Condor{dagman} to hang if a node
with a POST script and retries initially runs but fails, and then
has all \Condor{submit} attempts fail on the retry.

\item Fixed a problem in the Windows installer where the
\Macro{DAEMON\_LIST} parameter would be incorrectly set if the ``Join
an existing Condor pool'' option was selected or the ``Submit jobs to
Condor pool'' option was unchecked.  In the first case, a
\Condor{collector} and \Condor{negotiator} would incorrectly be run on
the machine. In the second case, a \Condor{schedd} would incorrectly
be run. The problem exists in all previous 6.8 and 6.9 series
releases.

\item Fixed a bug in the handling of local universe jobs
for a very busy \Condor{schedd} daemon.
When a local universe job completed, the \Condor{starter} might not
be able to connect to the \Condor{schedd} daemon to update final information
about the job, such as the exit status.
Under this circumstance,
the \Condor{starter} would hang indefinitely.
The bug is fixed by having the \Condor{starter} attempt
to retry a few times (with a delay in between each attempt) before
exiting with a fatal error.
The fatal error causes the job to restart.

\end{itemize}

\noindent Known Bugs:

\begin{itemize}

% Gnats PR 813
\item Setting \MacroNI{DAGMAN\_DELETE\_OLD\_LOGS} to false can cause
\Condor{dagman} to have problems (including hanging), especially
when running a rescue DAG.  If you want to keep your old user log
files, the best thing to do is to rename them before each
\Condor{dagman} run.  If you do run with
\MacroNI{DAGMAN\_DELETE\_OLD\_LOGS} set to false, check your
\File{dagman.out} file for error messages about submit event
Condor IDs not matching the expected value.  If you get such an
error, you will probably have to \Condor{rm} the \Condor{dagman}
job, remove or rename the old user log file(s) and run the rescue DAG.
(Note: this bug also applies to earlier versions of \Condor{dagman}.)

\end{itemize}




%%%%%%%%%%%%%%%%%%%%%%%%%%%%%%%%%%%%%%%%%%%%%%%%%%%%%%%%%%%%%%%%%%%%%%
\subsection*{\label{sec:New-6-8-3}Version 6.8.3}
%%%%%%%%%%%%%%%%%%%%%%%%%%%%%%%%%%%%%%%%%%%%%%%%%%%%%%%%%%%%%%%%%%%%%%

\noindent Release Notes:

\begin{itemize}

\item In this release,
the command \Condor{q} \Arg{-long} does not work when querying
the Quill database.
Instead, use the command
\Condor{q} \Arg{-direct quilld} \Arg{-long},
or use a previous version of \Condor{q}.

\item Performed a security audit of all places where Condor opens files,
to make certain files are opened with a reasonable permission mode
and with the
O\_EXCL flag whenever possible.

\end{itemize}


\noindent New Features:

\begin{itemize}

\item Added the \Macro{JOB\_INHERITS\_STARTER\_ENVIRONMENT}configuration
macro.  When set
to \Expr{True}, jobs inherit all environment variables from
the \Condor{starter}.  This is useful for glidein jobs that need to access
environment variables from the batch system running the glidein daemons.
The default for this configuration macro is \Expr{False}, so existing behavior
is unchanged.  This feature does not apply to standard and pvm universe
jobs.

\item Changed the default UDP receive buffer for the
\Condor{collector} from 1M to 10M.  This value can be configured with
the (existing) \MacroNI{COLLECTOR\_SOCKET\_BUFSIZE} macro.

\Note For some Linux distributions, it may be necessary to configure
a larger value than the default; this parameter is
/proc/sys/net/core/rmem\_max .  You can see the values that the
\Condor{collector} actually used by enabling D\_FULLDEBUG for the
\Condor{collector} and looking at the log line that looks like this:

Reset OS socket buffer size to 2048k (UDP), 255k (TCP).

\item Added a new configuration macro to control the size of the
TCP send buffers for the \Condor{collector}.  This macro used to
be the same as \MacroNI{COLLECTOR\_SOCKET\_BUFSIZE}.  The new macro is
\Macro{COLLECTOR\_TCP\_SOCKET\_BUFSIZE}, and it defaults to 128K.

\item Added a clipped port for SuSE Linux Enterprise Server 9 running on the 
PowerPC architecture.  Note the known bug below.

\item The \Condor{schedd} now maintains a birth date for the job queue. 
Nothing in Condor currently uses this feature, but future versions of \Condor{quill} may require it. 

\item There is a new configuration file macro
\MacroNI{RANDOM\_INTEGER}(min,max[,step]).  It produces a
pseudo-random integer within the range \verb@min@ and \verb@max@,
inclusive at configuration time.

\end{itemize}

\noindent Bugs Fixed:

\begin{itemize}

\item Fixed a deadlock situation between the \Condor{schedd} and
the \Condor{startd} that can
significantly impact the \Condor{schedd}'s performance.  The likelihood of the
deadlock increased based upon the number of VMs advertised by the
\Condor{startd}.

\item Fixed a bug reading the user job log on Windows that caused
occasional DAGMan confusion.
Thanks to Fairview Software, Inc. for
both finding the bug and writing a patch.

\item Fixed a denial of service problem: Condor daemons no longer freeze
for 20 seconds when a client connects to them and then sends no data.
This behavior is common with port scanners.

\item Fixed a race condition with \Condor{quill} caused by
\Prog{PostgreSQL}'s default transaction isolation level being ``read
committed''. 
This bug would cause truncated \Condor{q} reads when using Quill.

\item Fixed a bug where the \Condor{ckpt\_server} would segfault when
turned off with \Condor{off} \Opt{-fast}.

\item Fixed a bug in the \Condor{startd} where it could die with
  SIGABRT when a \Condor{starter} exited under certain rare
  circumstances.
  The bug seems to have been most likely to appear on x86\_64 Linux
  machines, but could potentially affect all platforms.

\item Fixed a problem with \Condor{history} when running with Quill enabled,
which caused it to allocate an unbounded amount of memory.

\item Fixed a problem with \Condor{q} when running with Quill, which caused
it to silently truncate the printing of the job queue.

\item Fixed a bug in the \Condor{gridmanager} that caused the following
configuration files parameters to be ignored for grid types condor and
nordugrid jobs: \MacroNI{GRIDMANAGER\_RESOURCE\_PROBE\_INTERVAL},
\MacroNI{GRIDMANAGER\_MAX\_PENDING\_SUBMITS\_PER\_RESOURCE}, and
\MacroNI{GRIDMANAGER\_MAX\_SUBMITTED\_JOBS\_PER\_RESOURCE}.

\item Fixed a bug in \Condor{run} that caused it to abort on non-fatal
warnings from \Condor{submit} and print incorrect error messages.

\item Fixed a bug in the \Condor{gridmanager} dealing with grid type gt4
grid universe jobs. If the job's standard output or error was not specified
in the job ClassAd, the \Condor{gridmanager} would create an improper GRAM
RSL string, causing the job to fail.

\item Fixed a bug in the \Condor{gridmanager} that could cause it to
delegate the wrong credential when refreshing the credentials for a
grid type gt4 grid universe job.

\item The \Condor{gridmanager} could get into a state where it would no
longer start up Globus jobmanagers for grid type gt2 grid universe jobs,
if previous requests failed due to connection errors. This bug has been
fixed.

\item The \Condor{c-gahp} now properly exits when the pipe to its parent
goes away. Before, it would fill its log with large amounts of useless
messages, before exiting several minutes later.

\item Fixed a bug where a problem opening standard input, output, or error,
the standard universe might generate an incorrect warning in the 
\Condor{shadow}'s log.

\item The \Condor{gridmanager} now recovers properly when a proxy refresh
fails for a gt2 grid universe job in the stage-out state. Before, the job
would become held with a hold reason of ``Globus error 3: an I/O operation
failed''.

\item A number of fixes to minor typos and incorrect formatting in
Condor's log files.

\item When \MacroNI{REQUEST\_CLAIM\_TIMEOUT} was reached and the
\Condor{schedd}
failed to contact the \Condor{startd} to release the claim, the 
\Condor{schedd} would
periodically try releasing the claim indefinitely, possibly resulting in
a lengthy communication delay each time.

\item Under Windows, Condor daemons such as the \Condor{schedd} were sometimes
limiting their use of pending connect operations more than they should
have.  This would result in the message, ``file descriptor safety level
exceeded''.

\item \Condor{fetchlog} no longer allows or documents the -dagman option.
The option's appearance was an error.  The option never worked.

\item The \Condor{schedd} ensures that the initial job queue log file
contains a sequence number for use by Quill.  This fixes a case in
which no sequence number was inserted, because the initial rotation of
this (empty) file failed.  Quill also now reports exactly what the
problem is if it reads a job queue log in this state, rather than
simply crashing.  This problem has so far only been observed under
Windows.

\item Fixed a problem on Windows where, when submitting a job with a
sandbox (for example, using the \Opt{-s} or \Opt{-r} option to
\Condor{submit}), an erroneous file permissions check in the
\Condor{schedd} would result in a failed submission.

\item The \Condor{startd} would crash shortly after start up if the
\MacroNI{RANK} expression contained any use of the unary minus
operator.  This patch should also fix any other cases where Condor
daemons crashed due to the use of the unary minus operator in ClassAd
expressions.

\item Stork now writes a terminated event to the user log when it removes
a transfer job from its queue because of failures to invoke a transfer
module. Without this event, DAGMan would not notice that these jobs had
left the queue.

\item Fixed a problem where the \Condor{schedd} on Windows would
incorrectly reject a job if the client provided an \Attr{Owner}
attribute that was correct but differed in case from the authenticated
name. This bug was thought to have been fixed in Condor 6.8.0.

\item Fixed problems with \Condor{store\_cred} behaving strangely when
storing or removing a user name that is some initial substring of
``condor\_pool''. Specifying such a user name would be incorrectly
interpreted as equivalent to specifying the \Opt{-c} option.

\item Fixed a problem with \Condor{glidein} spewing lots of text to
the screen when checking the status of a job it submitted.

\item A new version of the GT4 GAHP is included, with the following changes:

  \begin{itemize}

  \item A new \File{axis.jar} from Globus fixes a thread safety bug that
  can cause lockups in subscriptions for WS notifications. See Globus
  Bugzilla 4858
  (\URL{http://bugzilla.globus.org/bugzilla/show\_bug.cgi?id=4858}).

  \item Fixed bugs that caused memory related to destroyed jobs to not
  be reclaimed in both the client and the server.

  \item Removed redundant usage of Secure Message, Secure Conversation,
  and Transport Security when talking to a WS GRAM service. Now, only
  Transport Security is used.

  \end{itemize}

\item Fixed memory leaks in \Condor{quill}.

\item Fixed a bug that might have caused \Condor{startd} problems
launching the \Condor{starter} for the standard universe on 64-bit systems.

\item Improved Condor's file transfer.  If you request that Condor
automatically transfer back your output, it now detects changes better.
Previously, it would only transfer back files that had a more recent timestamp 
than the spool date.  Now, it will transfer back any file that has changed
in date (including being dated in the past) or changed in size.

\end{itemize}

\noindent Known Bugs:

\begin{itemize}

\item SuSE Linux Enterprise Server 9 on PowerPC only: The default Java
interpreter on SuSE Linux Enterprise Server 9 running on the PowerPC
architecture has compatibility problems with this release of Condor.  The
problem exhibits itself as the \Condor{startd} hanging, never reporting itself
to the \Condor{collector}.  The workaround is to either disable the Java
universe (set \MacroNI{JAVA} to an empty string), or disable just-in-time
compilation when running in the Java universe with the following configuration
setting:
\begin{verbatim}
  JAVA_EXTRA_ARGUMENTS = -Djava.compiler=NONE
\end{verbatim}

\end{itemize}




%%%%%%%%%%%%%%%%%%%%%%%%%%%%%%%%%%%%%%%%%%%%%%%%%%%%%%%%%%%%%%%%%%%%%%
\subsection*{\label{sec:New-6-8-2}Version 6.8.2}
%%%%%%%%%%%%%%%%%%%%%%%%%%%%%%%%%%%%%%%%%%%%%%%%%%%%%%%%%%%%%%%%%%%%%%

\noindent Release Notes:

\begin{itemize}

\item Condor now uses Globus 4.0.3 for GSI, GRAM, and GridFTP support.
This includes a patch for the OpenSSL vulnerability detailed in 
CVE-2006-4339 and \URL{http://www.openssl.org/news/secadv\_20060905.txt}.
It also includes fixes for Globus Bugzilla 4689 
(\URL{http://bugzilla.globus.org/bugzilla/show\_bug.cgi?id=4689}) and a 
bug that can cause duplicate UUIDs to be generated for WS GRAM jobs.

\item The \Condor{schedd} daemon no longer forks separate processes to 
change ownership of job directories in the spool.
Previously on Unix-like systems, this would create a
new process before a job started running and after it finished running.   Some
sites with very busy \Condor{schedd} daemons were encountering scaling problems.

\end{itemize}

\noindent New Features:

\begin{itemize}

\item Because, by default, the \Condor{startd} daemon references the job
ClassAd attribute \AdAttr{NumCkpts}, Condor's default configuration
will now round up the value of \AdAttr{NumCkpts}, in order to improve 
matchmaking performance.  See the entry on \Macro{SCHEDD\_ROUND\_ATTR}
in section~\ref{param:ScheddRoundAttr}.

\item Enhanced the RHEL3 x86\_64 port of Condor to include the standard
universe.

% Gnats PR 757
\item \Condor{submit\_dag} \Opt{-f} no longer deletes the
\File{dagman.out} file.  \Condor{submit\_dag} without the \Opt{-f}
option will now submit a DAGMan run even if the \File{dagman.out}
file exists.  In this case, the file will be appended to.

\item Added a property to the Windows installer program to determine
whether the Condor service will be started after installation. The
property name is STARTSERVICE, and the default value is ``Y''.

\end{itemize}

\noindent Bugs Fixed:

\begin{itemize}

\item A bug caused the \Condor{master} daemon to kill
only immediate children within the process tree,
upon an abnormal exit of the \Condor{master} daemon. 
The \Condor{master} daemon now kills all descendant processes.

\item Fixed a bug where if the file system was full, the debugging log
files (for example \File{SchedLog}) would silently lose messages.  Now,
if the disk is full, the Condor daemons will 
exit.

\item Fixed a bug in the \Condor{schedd} daemon that caused it to stop
negotiating for grid universe jobs in the case that it decided
it could not spawn any new \Condor{shadow} processes.

% Gnats PR 751
\item Added the ProcessId class (which more uniquely identifies a
process than a PID does) to the \Condor{dagman} abort duplicate
runs feature.  This makes it less likely that a given instance of
\Condor{dagman} will mistakenly conclude that another instance of
\Condor{dagman} is already running on the same DAG.  Also fixed an
unrelated bug in the abort duplicate runs feature that could cause
a \Condor{dagman} to not abort itself when it should.

\item Condor daemons leaked memory (consuming more and more memory over time)
when parsing ClassAds that use functions with arguments.

% Gnats PR 743
\item Fixed a bug in the \Condor{starter} daemon,
which caused it to look in the
wrong place for the job's executable, if \Attr{TransferExecutable} was set
to \Expr{True} in the job ClassAd.

\item \Condor{history} no longer crashes if \Attr{HISTORY} is not defined
in the Condor configuration file.

\item Fixed an unintentional change to the value of \Opt{-Condorlog}
in a \Condor{dagman} submit description file: it is once again the log file of
the first node job.

\item Fixed a bug in \Condor{q} that would cause \Condor{q} \Opt{-hold} or
\Condor{q} \Opt{-run} to exit with an error on some platforms.

\item Fixed a bug on Unix platforms, in which a misconfiguration of
\MacroNI{MAIL} would cause the \Condor{master} daemon to restart
all of its child
daemons whenever it tried (and failed) to send e-mail to the
administrator.

\item Network related error messages have been improved to make debugging
easier.  For example, when timing out on a read or write operation, the
peer's address is now included in the error message.

\item An invalid value for \MacroNI{UPDATE\_INTERVAL} now causes
the \Condor{startd} daemon to abort.  Previously, it would continue running,
but some invalid values (for example, 0) could cause it to stop sending
periodic ClassAd updates to the \Condor{collector}, even after being
reconfigured with a valid value.  Only a complete restart of
the \Condor{startd} daemon was sufficient to get it out of this state.

\item Fixed a bug that caused X.509 limited proxies to be delegated as 
impersonation (i.e. non-limited) proxies. Any authentication attempted
with the resulting proxies would fail.

\item Fixed a couple bugs that would cause Condor to lose track of
some Condor-related processes and subsequently fail to clean up (kill)
these processes.

\item Fixed a bug that would cause \Condor{history} to crash when
dealing with rotated history files. Note that history file rotation is
turned on by default. (See
Section~\ref{sec:Condor-wide-Config-File-Entries} for descriptions of
\Macro{ENABLE\_HISTORY\_ROTATION} and
\Macro{MAX\_HISTORY\_ROTATIONS}.)

\end{itemize}

\noindent Known Bugs:
\begin{itemize}

\item None.

\end{itemize}


%%%%%%%%%%%%%%%%%%%%%%%%%%%%%%%%%%%%%%%%%%%%%%%%%%%%%%%%%%%%%%%%%%%%%%
\subsection*{\label{sec:New-6-8-1}Version 6.8.1}
%%%%%%%%%%%%%%%%%%%%%%%%%%%%%%%%%%%%%%%%%%%%%%%%%%%%%%%%%%%%%%%%%%%%%%

\noindent Release Notes:

\begin{itemize}

\item Version 6.8.1 fixes important bugs, some of which have
security implications.  All users are encouraged to upgrade, and full
disclosure of the vulnerabilities will be given at the end of October 2006.

\item Condor is now linked against GSI from Globus 4.0.2. This includes
a patch for Globus Security Advisories 2006-01 
(\URL{http://www.globus.org/mail\_archive/security-announce/2006/08/msg00000.html})
and 2006-02 
(\URL{http://www.globus.org/mail\_archive/security-announce/2006/08/msg00001.html}).
It also includes a
patch for the OpenSSL vulnerability detailed in CVE-2006-4339 and
\URL{http://www.openssl.org/news/secadv\_20060905.txt}.

\item The PCRE (Perl Compatible Regular Expressions) library used by
Condor is now dynamically linked and shipped as a DLL with Condor for
Windows, rather than being statically linked.

\end{itemize}


\noindent New Features:

\begin{itemize}

% Gnats PR 610
\item Added an optional argument to the \Condor{dagman} ABORT-DAG-ON
command that allows the DAGMan exit code to be specified separately
from the node value that causes the abort; also, a DAG can now be
aborted on a zero exit code from a node.

% I implemented in condor_rm.  Todd is handling condor_schedd implementation.
\item Added the \Macro{ALLOW\_FORCE\_RM} configuration variable.
If this expression evaluates to \Expr{True},
then an \Condor{rm} -f attempt is allowed.  If it evaluated to \Expr{False},
the attempt is disallowed.
The expression is evaluated in the context of the job ClassAd.
If not defined, the value defaults to \Expr{True}, matching the behavior of
previous Condor releases.

% Gnats PR 664
\item \Condor{dagman} will now reject DAGs for which any of the nodes'
user job log files are on NFS (because of the unreliability of NFS
file locking, this can cause DAGs to fail).  This feature can be
turned off by setting the \MacroNI{DAGMAN\_LOG\_ON\_NFS\_IS\_ERROR}
configuration macro to \Expr{False} (the default is \Expr{True}).

\item \Condor{submit} can now be configured to reject jobs for which
the log file is on NFS.
To do this, set the \MacroNI{LOG\_ON\_NFS\_IS\_ERROR}
configuration macro to \Expr{True}.
The default is that \condor{submit} will issue a warning
for a log file on NFS.

\item Added the \MacroNI{DAGMAN\_ABORT\_DUPLICATES} configuration macro,
which causes
\Condor{dagman} to attempt to detect at startup whether another
\Condor{dagman} is already running on the same DAG; if so, the second
\Condor{dagman} will abort itself.

\item The new configuration variable
\MacroNI{NETWORK\_MAX\_PENDING\_CONNECTS} may be used to limit the
maximum number of simultaneous network connection attempts.  This is
primarily relevant to the \Condor{schedd} daemon, which may try to connect to
large numbers of \Condor{startd} daemons when claiming them.
The \Condor{negotiator} may also
connect to large numbers of \Condor{startd} daemons when initiating
security sessions
used for sending MATCH messages.  On Unix, the default is to allow up to
eighty percent of the process file descriptor limit.  On Windows, the
default is 1600.

% Gnats 734/condor-admin 13872
\item Added some more debug output to \Condor{dagman} to clarify
fatal errors.

\item The -format argument to \Condor{q} and \Condor{status} can now take an expression in addition to a simple attribute name.

\item DRMAA is now available on most Linux platforms, Windows and PPC MacOS.

\end{itemize}

\noindent Bugs Fixed:

\begin{itemize}

\item When a large number of jobs (roughly 200 or more) are running from a
single \Condor{schedd} daemon, and those jobs are using job leases
(the default in 6.8), it is
possible for the \Condor{schedd} daemon to enter a state 
where it crashes on startup until all of
the job leases expire.
% The bug is more generic; it potentially hit any user of GenericQuery or
% CondorQuery, but we know of no users encountering the bug in any other cases.

\item Condor jobs submitted with the \AdAttr{NiceUser} priority were
  not being matched if the \Macro{NEGOTIATOR\_MATCHLIST\_CACHING}
  setting was TRUE (which is enabled by default).

\item Fixed a Quill bug that prevented it from running on Windows.  The
symptom showed with errors in the QuillLog such as
\begin{verbatim}
POLLING RESULT: ERROR
\end{verbatim}

\item Fixed a bug in Quill where it would cause errors such as
\begin{verbatim}
duplicate key violates unique constraint "history_vertical_pkey"
\end{verbatim}
in the QuillLog and the \Prog{PostgreSQL} log file.  These errors
triggered
a significant slowdown in the performance of Quill and the database.  This
would only happen when a job attribute changed type from a string
type to a numeric type, or vice versa.

%% This is the change to datathread.C
\item In those unusual cases where Condor is unable to create a new process,
it shuts down cleanly, eliminating a small possibility of data corruption.

\item Fixed a bug with the gt4 and nordugrid grid universe jobs that
caused the \File{stdout} and \File{stderr} of a job to not be 
transferred correctly, if the given file names had absolute paths.

% Gnats PR 711
\item \Condor{dagman} now echos warnings from \Condor{submit} and
\Stork{submit} to the \File{dagman.out} file.

\item Fixed a bug introduced in 6.7.20, causing the \Condor{ckpt\_server}
to exit immediately after starting up, unless Condor's security
negotiation was disabled.

% This was a bug because the default configuration file and manual
% both claimed it defaulted to 1MB.
\item \Macro{MAX\_<SUBSYS>\_LOG} defaults to one Megabyte, even if the
setting is missing from the configuration.  Previously it was 64 Kilobytes.

% this is the change to condor_secman.C by zmiller
\item Fixed a bug related to non-blocking connect that could occasionally
cause Condor daemons to crash.

% collector.C change: eliminated useless fprintf(stderr).
\item Fixed a rare bug where an exceptionally large query to the
\Condor{collector} could cause it to crash.  The most common cause was a single
\Condor{schedd} daemon restarting,
and trying to recover a large number of job leases at once.
More than approximately 250 running jobs on a single \Condor{schedd} daemon
would be necessary to trigger this bug.

\item When using the \Macro{JOB\_PROXY\_OVERRIDE\_FILE} configuration
parameter, the X.509 proxy will now be properly forwarded for Condor-C jobs.

\item Greatly reduced the chance that a Condor-C job in the REMOVED state
will be HELD due to an expired proxy or failure to talk to the remote
\Condor{schedd}.

\item Fixed error and debug messages added in Condor version 6.7.20 that
incorrectly reported IP and port numbers.  These messages were
intended to report the peer's address, but they were instead reporting the
local address of the network socket.

\item Fixed a bug introduced in Condor version 6.7.20
which could cause Condor daemons to
die with the message 
\begin{verbatim}
PANIC -- OUT OF FILE DESCRIPTORS
\end{verbatim}
The conditions
causing this related to failed attempts to send updated status
to the \Condor{collector} daemon,
with both non-blocking updates and security negotiation
enabled (the defaults).

\item Also fixed a bug in the negotiator with the same effect as
above, except it only happened with the configuration setting
\MacroNI{NEGOTIATOR\_USE\_NONBLOCKING\_STARTD\_CONTACT}=False.

\item Fixed a bug in \Condor{schedd} under Solaris that could also
cause file descriptors to become exhausted over time when many
machines were claimed in a short spans of time (e.g. over 100) and the
\Condor{schedd} process file descriptor limit was near 256.

\item Fixed a bug in \Condor{schedd} under Windows that could cause
network sockets to be allocated and never released back to the system.
The circumstances that could cause this were very rare.  The error
message in the logs indicating that this problem was happening is
\begin{verbatim}
ERROR: DuplicateHandle() failed in Sock::set_inheritable
\end{verbatim}
In cases where this error message is displayed, the network socket
is closed.

\item Under some conditions, when making TCP connections, Condor was
still trying to connect for the full duration of the operation timeout
(often 10 or 20 seconds), even if the connection attempt was refused
(for example, because the port being accessed is not accepting connections).
Now, the connect operation finishes immediately after the first such
failure, allowing the Condor process to continue with other tasks.

\item Fixed the problems relating to credential cache problems in the Kerberos
authentication mechanism.  The current version of Kerberos is 1.4.3.

% changes to condor_auth_ssl.C by zmiller
\item Fixed bugs in the SSL authentication mechanism that caused the
\Condor{schedd} to crash when submitting a job (on Unix) and caused
all tools and daemons to crash on Windows when using SSL.

\item Some of the binaries required to use Condor-C on Windows were
mistakenly not included in previous releases of Condor. This has been
fixed.

\item Fixed a problem on Windows where the \Condor{startd} could fail to
include some attributes in its ClassAd. This would result in some jobs
incorrectly not being matched to that machine.  This only happened if
\Macro{CREDD\_HOST} was defined and Condor daemons on the execute
machine were unable to authenticate with the \Condor{credd}.

\item Fixed a \Condor{dagman} bug which had prevented the
\MacroUNI{DAGManJobId} attribute from being expanded in job submit files
(for example,
when used as the value to define the \SubmitCmd{Priority} command).

\item Fixed a bug in \Condor{submit} that caused parallel universe jobs
submitted via Condor-C to become mpi universe jobs.

\item Fixed a bug which could cause Condor daemons to hang if they try
to write to the standard error stream (\File{stderr}) on some platforms.  In
general, this should never happen, but can, due to third party
libraries (beyond our control) trying to write error or other messages.

\item Fixed \Condor{status} to report error messages.

\item Fixed a bug in which setting the configuration variable 
\begin{verbatim}
NEGOTIATOR_CONSIDER_PREEMPTION = False
\end{verbatim}
caused an incorrect calculation.
The fraction of the pool already being claimed by a user was
calculated using the wrong total number of \Condor{startd} daemons.
This could cause some \Condor{startd} daemons to remain unclaimed,
even when there were jobs available to run on them.

% zmiller's changes to condor_io/condor_auth_fs.C
\item Fixed a security vulnerability in Condor's FS and FS\_REMOTE
authentication methods.  The vulnerability allowed an attacker to impersonate
another user on the system, potentially allowing submission of jobs as a
different user.  This may allow escalation to root privilege if the Condor
binaries and configuration files have improper permissions.  The fix is not
backwards compatible, which means all daemons and tools using FS authentication
must be running Condor 6.8.1 or greater.  The same applies to FS\_REMOTE; All
daemons and tools using FS\_REMOTE must be using Condor 6.8.1 or greater.  In
practice, this means that for FS, all Condor binaries on one host must be
version 6.8.1 or greater, but versions can be different from host to host.  For
FS\_REMOTE it means all binaries across all hosts must be 6.8.1 or greater.

% zmiller's changes to condor_credd/credd.C and stork/dap_server.C
\item Fixed a couple race conditions in stork and the credd where credential
files were possibly created with improper permissions before being set to owner
permissions.

\item Fixed a bug in the \Condor{gridmanager} that caused it to delegate
12-hour proxies for grid-type gt4 jobs and then not refresh them.

\item Fixed a bug in the \Condor{gridmanager} that caused a directory
needed for staging-in of grid-type gt4 job files to be removed when
the \Condor{Gridmanager} exited, causing the stage-in to fail.

% Condor-admin 14075
\item Fixed a bug that caused the \Term{checkpoint server} to restart
because of (ostensibly) getting an unexpected errno from select().

\item Fixed a bug on Windows where setting \SubmitCmd{output} or
\SubmitCmd{error} to a relative or absolute path (as opposed to a
simple file name without path information) would not work properly.

\item History file rotation did not previously work on Windows because
the name of a rotated files would contain an ISO 8601 extended format
timestamp, which contains colon characters. The naming convention for
rotated files has been modified to use ISO 8601 basic format, avoiding
this problem.

\item The CLAIMTOBE authentication method (which is inherently
insecure and should only be used for testing or other special
circumstances) previously would authenticate without providing the
``domain'' portion of the user name. As an example, a user would be
authenticated as simply ``user'' rather than
``user@cs.wisc.edu''. This problem has been fixed, but the new
protocol is not backwards compatible so the fix is turned off by
default. Correct behavior can be enabled by setting the
\Macro{SEC\_CLAIMTOBE\_INCLUDE\_DOMAIN} parameter to \Expr{True}.

\item Fixed a bug with the \MacroNI{NEGOTIATOR\_MATCHLIST\_CACHING} that
would cause very low-priority jobs (like jobs submitted with
\MacroNI{nice\_user=True}) to not match even if resources were available.

\item Fixed a buffer overflow that could crash the \Condor{negotiator}.

\item \MacroNI{SCHEDD\_ROUND\_ATTR\_<xxxx>} preserves the value being
rounded up when it is a multiple of the power of 10 specified for
rounding.  Previously, the value would be incremented; now it remains
the same.  For example, if SCHEDD\_ROUND\_ATTR\_<xxxx>=2 and the value
being rounded up is 100, it now remains 100, rather than being
incremented to 200.

\item Fixed \Condor{updates\_stats} to report it's version number
correctly.

\end{itemize}

\noindent Known Bugs:

\begin{itemize}

\item The \Opt{-completedsince} option to \Condor{history} works
when Quill is enabled.  The behavior of \Condor{history}
\Opt{-completedsince} is undefined when Quill is \emph{not}
enabled.

\end{itemize}




%%%%%%%%%%%%%%%%%%%%%%%%%%%%%%%%%%%%%%%%%%%%%%%%%%%%%%%%%%%%%%%%%%%%%%
\subsection*{\label{sec:New-6-8-0}Version 6.8.0}
%%%%%%%%%%%%%%%%%%%%%%%%%%%%%%%%%%%%%%%%%%%%%%%%%%%%%%%%%%%%%%%%%%%%%%

\noindent Release Notes:

\begin{itemize}

\item The default configuration for Condor now requires that
\Macro{HOSTALLOW\_WRITE} be explicitly set.  Condor will refuse
to start if the default configuration is used unmodified.
Existing installations should not need to change anything.  For
those who desire the earlier default, you can set it to "*", but
note that this is potentially a security hole allowing anyone to
submit jobs or machines to your pool.

\item Most Linux distributions are now supported using dynamically
  linked binaries built on a RedHat Enterprise Linux 3 machine.
  Recent security patches to a number of Linux distributions have
  rendered the binaries built on RedHat 9 machines ineffective.
  The download pages have been changed to reflect this, but Linux users
  should be aware of this change.
  The recommended download for most x86 Linux users is now:
  \File{condor-6.8.0-linux-x86-rhel3-dynamic.tar.gz}.

\item Some log messages have been clarified or moved to different
  debugging levels.
  For example, certain messages that looked like errors were printed
  to \MacroNI{D\_ALWAYS}, even though nothing was wrong and the system was
  behaving as expected.

\item The new features and bugs fixed in the rest of this section only
  refer to changes made since the 6.7.20 release, not the last stable
  release (6.6.11).
  For a complete list of changes since 6.6.11, read the 6.7 version
  history in section~\ref{sec:History-6-7} on
  page~\pageref{sec:History-6-7}. 

\end{itemize}


\noindent New Features:

\begin{itemize}

\item Version 1.4 of the Condor DRMAA libraries are now included 
  with the Condor release.
  For more information about DRMAA, see section~\ref{API-DRMAA} on
  page~\pageref{API-DRMAA}.

\item Version 1.0.15 of the Condor GAHP is now used for Condor-G and
  Condor-C. 

% Gnats PR 710
\item Added the \Opt{-outfile\_dir} command-line argument to
\Condor{submit\_dag}.  This allows you to change the directory in which
\Condor{dagman} writes the \File{dagman.out} file.

\item Added a new \Opt{--summary} (also \Opt{-s}) option to the
\Condor{update\_stats} tool.  If enabled, this prevents it from
displaying the entire history for each machine and only displays the
summary info.

\end{itemize}

\noindent Bugs Fixed:

\begin{itemize}

\item Fixed a number of potential static buffer overflows in various
  Condor daemons and libraries.

\item Fixed some small memory leaks in the \Condor{startd},
  \Condor{schedd}, and a potential leak that effected all Condor
  daemons.

\item Fixed a bug in Quill which caused it to crash when certain
long attributes appeared in a job ad.

\item The startd would crash after a reconfig if the address of a
collector had not been resolved since the previous reconfig
(e.g. because DNS was down during that time).

\item Once a Condor daemon failed to lookup the IP address of the
collector (e.g. because DNS was down), it would fail to contact the
collector from that time until the next reconfig.  Now, each time Condor
tries to contact the collector, it generates a fresh DNS query if the
previous attempt failed.

% Gnats PR 707
\item When using Condor-C or the -s or -r command-line options to
\condor{submit}, the job's standard output and error would be placed
in the job's initial working directory, even if the job ad said to
place them in a different directory.

% Gnats PRs 501 and 663
\item Greatly sped up the parsing of large DAGs (by a factor of 50
or so) by using a hash table instead of linear search to find DAG nodes.

% Gnats PR 697
\item Fixed a bug in \Condor{dagman} that caused an EXECUTABLE\_ERROR
event from a node job to abort the DAG instead of just marking the
relevant node as failed.

\item Fixed a bug in \Condor{collector} that caused it to discard
machine ads that don't have an IP address field (either StartdIpAddr
or STARTD\_IP\_ADDR).  The \Condor{startd} will always produce a
StartdIpAddr field, but machine ads published through
\Condor{advertise} may not.

\item When using \MacroNI{BIND\_ALL\_INTERFACES} on a dual-homed
machine, a bug introduced in 6.7.18 was causing Condor daemons to
sometimes incorrectly report their IP addresses, which could cause
jobs to fail to start running.

\item Made the event checking in \Condor{dagman} less strict: 
added the new "allow duplicate events" value to the
\MacroNI{DAGMAN\_ALLOW\_EVENTS} macro (this value is part of the
default); 16 value now also allows terminate event before submit;
changed "allow all events" to "allow almost all events"
(all except "run after terminal event"), so it is more useful.

% Gnats PR 712
\item \Condor{dagman} and \Condor{submit\_dag} now report
\Opt{-NoEventChecks} as ignored rather than deprecated.

\item Fixed a bug in the \Condor{dagman} \Opt{-maxidle} feature:
a shadow exception event now puts the corresponding job into the
idle state in \Condor{dagman}'s internal count.

\item Fixed a problem on Windows where daemons would sometimes crash
when dealing with UNC path names.

\item Fixed a problem where the \Condor{schedd} on Windows would
incorrectly reject a job if the client provided an \Attr{Owner}
attribute that was correct but differed in case from the authenticated
name.

\item Fixed a \Condor{startd} crash introduced in version 6.7.20. This
crash would appear if an execute machine was matched for preemption
but then not claimed in time by the appropriate \Condor{schedd}.

\item Resolved an issue where the \Condor{startd} was unable to clean
up jobs' execute directories on Windows when the \Condor{master} was
started from the command line rather than as a service.

\item Added more patches to Condor's DRMAA interface to make it more
compatible with Sun Grid Engine's DRMAA interface.

\item Removed the unused \MacroNI{D\_UPDOWN} debug level and added the
  \MacroNI{D\_CONFIG} debug level.

\item Fixed a bug that caused \Condor{q} with the \Opt{-l} or \Opt{-xml}
arguments to print out duplicate attributes when using Quill.

\item Fixed a bug that prevented Condor-C jobs (universe grid jobs of type condor)
from submitting correctly if \MacroNI{QUEUE\_ALL\_USERS\_TRUSTED} is set to
True.

\item Fixed a bug that could cause the \Condor{negotiator} to crash if the
pool contains several different versions of the \Condor{schedd} and in the
config file \MacroNI{NEGOTIATOR\_MATCHLIST\_CACHING} is set to True.

\item Changed the default value for config file entry
\MacroNI{NEGOTIATOR\_MATCHLIST\_CACHING} from False to True.  When set to
True, this will instruct the negotiator to safely cache data in order to
improve matchmaking performance.

\item The Condor{master} now recognizes \Condor{quill} as a valid
  Condor daemon without any manual configuration on the part of site
  administrators.
  This simplifies the configuration changes required to enable Quill. 

\item Fixed a rare bug in the \Condor{starter} where if there was a
  failure transferring job output files back to the submitting host,
  it could hang indefinitely, and the job appeared as if it was
  continuing to run.

\end{itemize}


\noindent Known Bugs:

\begin{itemize}

\item The \Opt{-completedsince} option to \Condor{history} works
when Quill is enabled.  The behavior of \Condor{history}
\Opt{-completedsince} is undefined when Quill is \emph{not}
enabled.

\end{itemize}


% Dec 2007, as we release 7.x, Karen commented out the older histories
%%%%%%%%%%%%%%%%%%%%%%%%%%%%%%%%%%%%%%%%%%%%%%%%%%%%%%%%%%%%%%%%%%%%%%%
\section{\label{sec:History-6-7}Development Release Series 6.7}
%%%%%%%%%%%%%%%%%%%%%%%%%%%%%%%%%%%%%%%%%%%%%%%%%%%%%%%%%%%%%%%%%%%%%%

This is the development release series of Condor.
The details of each version are described below.

%%%%%%%%%%%%%%%%%%%%%%%%%%%%%%%%%%%%%%%%%%%%%%%%%%%%%%%%%%%%%%%%%%%%%%
\subsection*{\label{sec:New-6-7.17}Version 6.7.17}
%%%%%%%%%%%%%%%%%%%%%%%%%%%%%%%%%%%%%%%%%%%%%%%%%%%%%%%%%%%%%%%%%%%%%%

\noindent Release Notes:

\begin{itemize}

\item None.

\end{itemize}

\noindent New Features:

\begin{itemize}

\item Added a \Opt{-format} option to the \Condor{history} command which
behaves just like the -format option to \Condor{status} and \Condor{q}
commands.

\item Added remove and get\_job\_attr options to the \Condor{chirp}
command line tool.

\item When the Grid Monitor encounters problems, Condor no longer tries
to restart the Globus JobManagers for all of the affected grid universe
jobs. Restarting the JobManagers can easily bring down a remote headnode.
Condor will attempt to restart the Grid Monitor, but there will be
no update of job status in the mean time. 

\item When started as root on a Linux 32-bit x86 machine, Condor daemons will
leave core files in the log directory when they crash.  Recent changes to the
Linux kernel default to blocking these core files.  This change means
Condor behaves more consistently across different Unix-like operating systems.

\end{itemize}

\noindent Bugs Fixed:

\begin{itemize}

\item When running a gridftp server for grid-type gt4 jobs, Condor will now
start the server so as to ignore /etc/grid-security/gridftp.conf and
\$GLOBUS\_LOCATION/etc/gridftp.conf. These files may contain options that
would cause the gridftp server to fail when not run as root.

\item Fixed a bug where jobs using the grid (or globus) universe that specified
an AccountingGroup would never run because the \Condor{gridmanager} would fail
to start.

\item Fixed a bug introduced in 6.7.14 where the job attributes RemoteUserCpu 
and RemoteSysCpu were incorrectly reported as 0 in the history file and the job queue
for non-standard universe jobs.

\item Fixed a physical memory reporting bug for the Mac OSX port of Condor.

\item Since the addition of the ``new'' cron syntax (introduced in
version 6.7.11), the \Condor{startd} has (silently) ignored any jobs
defined with the ``old'' syntax if any jobs are defined with the
``new'' syntax.  Now, the \Condor{startd} will honor both definitions,
but will log a warning to it's log file if any jobs with the ``old''
syntax are found (whether or not any new jobs are found).
The \Condor{schedd} (which also has the ``cron'' logic) will behave in
the same way.

\item The bug which was causing the ``Cron'' job command lines to have
the name added each invocation has been fixed.

\item Fixed some messages about keyboard and mouse idle time had been logged
too often in the \Condor{startd} logs under certain conditions to be logged
less often.

% Gnats PR 589
\item Fixed the \Opt{-dag} option to \Condor{q}.  Previously, this did not
print DAG node names as it should have.  (This bug has existed since
approximately v6.7.11.)

\item Enabled COLLECTOR\_QUERY\_WORKERS in the default
\Condor{collector} configuration, and set this value to 16.  Note this
COLLECTOR\_QUERY\_WORKERS has no effect on non-UNIX systems (Windows).

\item Fixed a bug that could cause the \Condor{gridmanager} to crash
if the GridJobId attribute for a gt2 job became mangled. The cause of
mangling seen by some users is still unknown.

\end{itemize}

\noindent Known Bugs:

\begin{itemize}

\item None.

\end{itemize}

%%%%%%%%%%%%%%%%%%%%%%%%%%%%%%%%%%%%%%%%%%%%%%%%%%%%%%%%%%%%%%%%%%%%%%
\subsection*{\label{sec:New-6-7.16}Version 6.7.16}
%%%%%%%%%%%%%%%%%%%%%%%%%%%%%%%%%%%%%%%%%%%%%%%%%%%%%%%%%%%%%%%%%%%%%%

\noindent Release Notes:

\begin{itemize}

\item None.

\end{itemize}


\noindent New Features:

\begin{itemize}

\item Support for running a personal Condor on Windows using
\Condor{master} \Opt{-f}

\end{itemize}

\noindent Bugs Fixed:

\begin{itemize}

\item Support for NorduGrid jobs was accidently left out of the
\Condor{gridmanager} in previous releases. This has been corrected.

\item The \Condor{starter} was refusing to run jobs if it could not
perform a reverse-DNS lookup of the submit-host.  It is now possible
to run without the reverse-DNS lookup, but you may still have to
enable \Macro{TRUST\_UID\_DOMAIN}, since the authenticity of the
submit-host's uid domain cannot be verified without the DNS
information.

\item Fixed a few bugs with \AdAttr{transfer\_output\_remaps} that caused
files to be remapped while in a temporary sandbox. Now, the remapping
occurs only when the files are returned to the job submitter.

\item Fixed some minor memory leaks in the \Condor{gridmanager}.

\item Fixed a bug in 6.7.15 that was causing startd cron jobs to fail to run
if the old-style configuration setting \Macro{STARTD\_CRON\_JOBS} was used
instead of the new-style configuration setting \Macro{STARTD\_CRON\_JOBLIST}.

\end{itemize}

\noindent Known Bugs:

\begin{itemize}

\item None.

\end{itemize}

%%%%%%%%%%%%%%%%%%%%%%%%%%%%%%%%%%%%%%%%%%%%%%%%%%%%%%%%%%%%%%%%%%%%%%
\subsection*{\label{sec:New-6-7.15}Version 6.7.15}
%%%%%%%%%%%%%%%%%%%%%%%%%%%%%%%%%%%%%%%%%%%%%%%%%%%%%%%%%%%%%%%%%%%%%%

\noindent Release Notes:

\begin{itemize}

\item If you have not used the undocumented configuration setting
\Macro{SIGNIFICANT\_ATTRIBUTES}, there is no need to read the rest of
this paragraph.  For sites that have been using
\Macro{SIGNIFICANT\_ATTRIBUTES} in the config file, we suggest
removing that setting, because Condor now automatically selects the
list of attributes that are used to cluster job ClassAds into distinct
ads for negotiation.  In 6.7.15, any setting of
\Macro{SIGNIFICANT\_ATTRIBUTES} will be combined with the automated
list of attributes that Condor produces.  In the future, this behavior
may change (e.g. it might override the automated behavior rather than
combining with it).  If you know in advance that your use of Condor
heavily depends on \Macro{SIGNIFICANT\_ATTRIBUTES} \emph{not}
including some attributes that \emph{are} used in requirements
expressions (e.g.  ImageSize), then you should be aware that 6.7.15
provides \emph{no} way for you to suppress such attributes.  In
that case, we recommend that you wait for 6.7.16 before upgrading.
This should not concern most users--especially anyone who is not even
using \Macro{SIGNIFICANT\_ATTRIBUTES}, or who has defined
\Macro{SIGNIFICANT\_ATTRIBUTES} to include all attributes that are
used in requirements expressions (which is the normal usage case).

\item Added a clipped port of Condor to YellowDog Linux 3.0 on the
PowerPC architecture.

\item ``Cron'' jobs defined with the ``old'' configuration syntax
(usually through ``STARTD\_CRON\_JOBS'' or ``HAWKEYE\_CRON\_JOBS'' --
see the \Condor{startd} manual section for more details) are broken.
Using the ``new'' syntax (``STARTD\_CRON\_JOBLIST'') will work around
this problem.

\end{itemize}

\noindent New Features:

\begin{itemize}

\item For those platforms which support it, libcondorapi.so is now
produced and available in the lib/ directory after installing Condor.

\item The negotiation protocol between the \Condor{schedd} and
the \Condor{negotiator} daemons has been improved for both scalability and
correctness.  In general, most sites will see faster negotiation 
cycles when many jobs are submitted after upgrading both the negotiator
and all schedd daemons to version 6.7.15.  This means the scheduling overhead
per job is reduced.  If you have used the undocumented macro
\Macro{SIGNIFICANT\_ATTRIBUTES}, please read the note above in the release
notes, because this new automated behavior affects the use of that
configuration setting--in most cases making it unnecessary.

\item Due to kernel bugs between the Linux 2.4.x and 2.6.x kernels,
Condor now implements "checkpointing signatures" which allow more fine
grained and automatic control over whether or not a particular machine
is willing to resume a previously created checkpoint. This functionality
is homogenized across all platforms which provide the standard universe
feature set.

\item Grid matchmaking ads are now aged and replaced by the negotiator 
based on a configurable classad expression from the condor config file. This
configuration parameter is called \Macro{STARTD\_AD\_REEVAL\_EXPR}.  
In previous versions, this was done strictly based on the 
UpdateSequenceNumber field in the ad.  The default value for the new 
parameter behaves the same as the older, hard-coded algorithm.

\item Condor can now dynamically start its own gridftp server to handle
file transfers for grid-type gt4 jobs. The gridftp server appears
as a job in the queue and disappears when it's no longer needed.

\item Automatic renewal of job proxies from a MyProxy server now works for
all grid universe jobs. Before, it only worked for grid-type gt2 jobs.

\item \Condor{dagman} now reports to its POST scripts uniquely
distinguishable return codes for non-exe job failures (e.g.,
\Condor{dagman}, batch-system, or other external errors such as failed
batch job submission, or batch job removal).  In the past these errors
were reported as various signals (e.g., SIGABRT for job removal or
SIGUSR1 for failed job submission), making it impossible to
distinugish them from the real signals as which they were
masquerading.  We now represent these errors using the
previously-unused return-code space below -64 (we start below -1000,
in fact).  As before, 0-255 reflect normal exe return codes, and -1 to
-64 represent signals 1 to 64 -- but now -1000 and below represent
DAGMan, batch-system, or other external errors.

\item Added the \Macro{DAGMAN\_RETRY\_NODE\_FIRST} configuration macro to
\Condor{dagman} to control whether failed nodes are retried before
or after other ready nodes.  The default is FALSE (\Condor{dagman}'s
previous behavior), which means that failed nodes will be retried
after other ready nodes.

\item Added a new (backward compatible) syntax for job arguments and
environment, allowing special characters to be escaped in a uniform
way.  The old limit of 4096 characters in the job arguments has also
been removed.  See \Condor{submit} manual for details of the new
syntax.

\item Added more configuration parameters to the \Condor{master}'s
restart / backoff mechanism.  You can now configure the initial value
of the backoff time (via \Macro{MASTER\_BACKOFF\_CONSTANT}).
Additionally, you can now set daemon specific values for all of these
parameters.  See the \Condor{master} entry in the manual for more
details.

\item \Condor{userprio} now supports \Opt{-setaccum} \Opt{-setbegin} 
\Opt{-setlast}  options to set the Accumulated Usage, Begin Usage Time, and
Last Usage time of a submitter. This is in addition to the existing
\Opt{-setprio} and \Opt{-setfactor} options.
These options can be used to safely reconstruct priority information if
the only backup data available is the output from \Condor{userprio} \Opt{-l}

\item An updated DRMAA version is available on supported platforms.  The 
previous DRMAA implementation has been removed.

% Gnats PR 248, new per-job Stork user logs.
% Implicitly added PR 278, which requires Stork user logs to
% be owned by user, and not Stork server daemon.  Implicitly added PR 572, now
% Stork user logs use standard Condor user log API.
\item Added new per-job Stork user logs.  Stork user logs are now optional, and
specified in the job submit file.  Stork now uses Condor user log output
format, including optional XML format.  Previous, per-server Stork user log in
\File{LOG/Stork.user\_log} is now deprecated, and will be removed in a future
release.

\item \Condor{dagman} now supports the new, per-job Stork user logs.
"Old-style" Stork logs (specified with \Opt{-Storklog} on the
\Condor{submit\_dag} command line) are supported for now, but this
support will probably be eliminated in the 6.7.16 release.

% Gnats PR 600.
\item Added new per-job Stork input, output and error output file
specifications.  Stork job output is now optional, and
specified in the job submit file.  Previous, per-server Stork user log in 
\File{LOG/Stork-module.stderr} and \File{LOG/Stork-module.stdout} has been
removed.

\item The Condor installer for Windows is now MSI compliant.

\end{itemize}

\noindent Bugs Fixed:

\begin{itemize}

\item Configuration parameters \Macro{LOWPORT} and \Macro{HIGHPORT} are
now respected for ports created for interaction with Globus GRAM servers.

\item \Condor{status} \Opt{-any} now reports quill ads when quill is enabled.

\item \Condor{restart} \Opt{-peaceful} was causing \Condor{master} to only
do a graceful shutdown, rather than a peaceful one.  This means that
\Macro{GRACEFUL\_SHUTDOWN\_TIMEOUT} would come into effect if jobs running
under the startd took too long to finish.  However, \Opt{-peaceful} restart
did work in the case where a specific subsystem (e.g. \Opt{-startd}) was
specified.

% Gnats PR 588
\item When run from a privileged (root) Stork server, modules lose
\Macro{LD\_LIBRARY\_PATH} and other key environments, for security
reasons.  This is not actually a Stork bug, but a feature of glibc.
When run with a dynamically linked \File{globus-url-copy}, the
contributed modules for the HTTP, FTP and GSIFTP transfer protocols
will fail.  To compensate, these modules can now restore their
environment via the pre-existing \Macro{STORK\_ENVIRONMENT}
configuration macro.  Unprivileged (user level) Storks are not
affected by this behavior.

\item  Jobs that are are placed on held because of \AdAttr{on\_exit\_hold}
evaluated to TRUE or jobs that stay in the queue after finishing because
\AdAttr{on\_exit\_remove} evaluated to FALSE again correctly report the
expression as being a "job attribute", not "UNKNOWN (never set)".

\item \Condor{glidein} was creating a default configuration with
\Macro{UPDATE\_INTERVAL}=20, which causes unnecessary scaling problems in large
glidein pools.  It now simply leaves this value undefined so that
the default behavior may be assumed.

\item Fixed a bug that could cause the \Condor{gridmanager} to crash when
a grid-type condor grid universe job left the queue.

\item When using job leases with the condor grid-type, a completed job will
now leave the remote \Condor{schedd}'s queue when the lease expires.

\item Fixed a bug in the \Code{fullpath()} function that tests whether
a file path is a full path -- paths of the form \File{"c:/"} were not
recognized as full paths, which could lead to something being prepended
to what was already a full path, thereby creating an invalid path.

\item Fixed a problem with \AdAttr{WhenToTransferOutput}=ALWAYS.  The
bug affected jobs that were evicted after producing one or more
intermediate files that were removed by the job before finally running
to completion in a subsequent run.  Condor was treating the missing
intermediate files as an error and the job would typically keep
running and failing until the user intervened.  In addition to fixing
this bug, file transfer error messages are now propagated back to the
shadow log and the user log, making it easier to debug problems
related to file-transfers.

\item \Condor{submit} was not paying attention to
\AdAttr{transfer\_output\_remaps} when doing permissions checks on
output files.

\end{itemize}

\noindent Known Bugs:

\begin{itemize}

\item None.

\end{itemize}


%%%%%%%%%%%%%%%%%%%%%%%%%%%%%%%%%%%%%%%%%%%%%%%%%%%%%%%%%%%%%%%%%%%%%%
\subsection*{\label{sec:New-6-7.14}Version 6.7.14}
%%%%%%%%%%%%%%%%%%%%%%%%%%%%%%%%%%%%%%%%%%%%%%%%%%%%%%%%%%%%%%%%%%%%%%

\noindent Release Notes:

\begin{itemize}

\item None.

\end{itemize}

\noindent New Features:

\begin{itemize}

\item The Condor grid universe can now be used to submit jobs to
Nordugrid and Unicore resources.

\item The Condor daemons now automatically restart when the
system clock jumps more than 20 minutes in either
direction.  This may happen if the machine running Condor entered
a "sleep" state.  This resolves a variety of minor problems.

\item Added a \Arg{-direct} debugging option to \Condor{q} which, when
using or querying a quill installation, allows talking directly to the
rdbms, the quill daemon, or the schedd without performing the queue
location discovery algorithm.

\item \Condor{schedd} provides more flexibility in how local and
scheduler universe jobs are started. The new configuration macros
\Macro{START\_LOCAL\_UNIVERSE} and \Macro{START\_SCHEDULER\_UNIVERSE}
allow administrators to control whether \Condor{schedd} will start
an idle local or scheduler universe job. If a job's respective universe
macro evaluates to true, \Condor{schedd} will then evaluate the 
\Macro{Requirements} expression for the job. Only if both conditions are
met will a job be allowed to begin execution.

\item \Condor{schedd} advertises how many local and scheduler
universe jobs are currently running or idle in its ClassAd. The
total number of running jobs is denoted by the
\Attr{TotalLocalJobsRunning} and \Attr{TotalSchedulerJobsRunning}
attributes. The total number of idle jobs is denoted by the
\Attr{TotalLocalJobsIdle} and \Attr{TotalSchedulerJobsIdle}.

\item A job submission can now specify the exact time that it should be
executed at using the \Attr{DeferralTime} attribute. The time is specified
as the number seconds since the Unix epoch (00:00:00 UTC, Jan 1, 1970).
An additional attribute \Attr{DeferralWindow} can be specified along with
the deferral time that will allow a job to run even if it misses the
execution time. The window is the number of seconds in the past that
Condor will allow for a missed job to execute. This feature is not 
supported for scheduler universe jobs.

\item Added the concept of a ``controlling'' daemon to the
\Condor{master}.  This feature is currently used only for ``High
Availability'' (HA) configurations involving the \Condor{had} daemon.
To properly use these Condor HA features you must set this macro.  

To configure the \Condor{negotiator} daemon to be controlled by the
\Condor{had}, you should add an entry to your condor\_config:

\begin{verbatim}
MASTER_NEGOTIATOR_CONTROLLER = HAD
\end{verbatim}

This will cause the \Condor{master} to treat the \Condor{had} as the
``controller'' of the \Condor{negotiator}.

\item Grid-type condor grid universe jobs now respect configuration
parameters \Macro{GRIDMANAGER\_MAX\_PENDING\_SUBMIT\_PER\_RESOURCE} and
\Macro{GRIDMANAGER\_MAX\_SUBMITTED\_JOBS\_PER\_RESOURCE}.

\item Grid universe jobs can now determine their \SubmitCmd{grid\_type}
via matchmaking,
in addition to which resource they will be submitted to.
A \SubmitCmd{grid} universe job may become any \SubmitCmd{grid\_type} job,
depending on what resource ad it is matched with.

\item Added support for a new configuration value,
  \Macro{STARTD\_CRON\_AUTOPUBLISH}.
  This setting can be used to tell the \Condor{startd} to
  automatically publish a new update to the \Condor{collector}
  whenever any of the \Term{cron} modules it is configured to run have
  produced output.
  For more information, see the description of
  \MacroNI{STARTD\_CRON\_AUTOPUBLISH} in
  section~\ref{param:StartdCronAutopublish} on
  page~\pageref{param:StartdCronAutopublish}. 

\item Reduced delay in negotiation when a job is released.  A reschedule
request is sent to the negotiator when a job is released from hold.  This
reduces the delay in several cases, most notably when using Condor-C or
"condor\_submit -s".  Previously the negotiator would not be notified and
would normally wait until the next scheduled negotiation cycle.

\item Added three new user log events: GridResourceUp, GridResourceDown,
and GridSubmit. They are equivalent to the existing Globus-specific log
events, but are used for all grid universe jobs.

\item When known, CPU-usage information will be reflected in the Terminated
user log event for grid universe jobs.

\item Changed ClassAd expression evaluation so that logical and and
logical or are short-circuited. This means that an expression like
\verb@TARGET.foo && TARGET.bar@ will not evaluate \verb@TARGET.bar@
if \verb@TARGET.foo@ evaluates to false. This will speed up some
expressions, particularly those involving user-defined
functions. Although this was thoroughly tested, this is the sort of
change that could have subtle, unexpected behavior, so please be on
the lookout for problems that might be caused by it. 

\item Added the \Condor{check\_userlogs} command, which checks user log
files for "illegal" events.

\item New settings
\Macro{SYSTEM\_PERIODIC\_HOLD},
\Macro{SYSTEM\_PERIODIC\_RELEASE}, and
\Macro{SYSTEM\_PERIODIC\_REMOVE}.
These expressions behave identically to the job expressions
\AdAttr{periodic\_hold},
\AdAttr{periodic\_release}, and
\AdAttr{periodic\_remove}, but are evaluated for all jobs in the
queue.  If not present, they default to FALSE.

\item An improved version of the DRMAA C library is available for download from
\URL{http://prdownloads.sourceforge.net/condor-ext/condor\_drmaa\_6\_7\_14\_src.tgz}

\item Added \Macro{CLAIM\_WORKLIFE} configuration option.  The startd
will not allow claims older than the specified number of seconds to
run more jobs.  Any existing job that is running when the worklife expires,
however, is allowed to continue to run as normal.

\end{itemize}

\noindent Bugs Fixed:

\begin{itemize}

\item Fixed the following problems with the Condor SOAP interface: a) placing a
job on hold now stops the job as expected, b) fixed potential schedd segfaults
when sending NULL buffers via SOAP, c) fixed compatibility problems with .NET
clients

\item Fixed a potential security problem where any machine in the pool
  could advertise an additional \Condor{negotiator} in the pool.
  Now, the \Condor{collector} will only accept negotiator classads
  from machines listed in the \Macro{HOSTALLOW\_NEGOTIATOR} variable.
  This bug has been in Condor since version 6.7.4.

\item Fixed bug in the dedicated scheduler where on busy pools
running mixed parallel and sequential jobs, it would incorrectly 
try to preempt dedicated jobs.

\item Fixed some problems when Microsoft .NET clients communicate with Condor
via SOAP. The issues were resolved by upgrading the version of gsoap included
inside of Condor to gsoap ver 2.7.6c.

\item Fixed the bug in the \Condor{ckpt\_server} from version 6.7.13
  where it would give clients the wrong IP address and no
  checkpointing was possible.
  This would result in the following sorts of errors in the log file
  generated by the \Condor{shadow} (by default, \File{ShadowLog}):
\footnotesize
\begin{verbatim}
Read: connect() failed - errno = 111
Read: open_tcp_stream() failed
Read: ERROR:open_ckpt_file failed, aborting ckpt
\end{verbatim}
\normalsize
  Version 6.7.14 of the \Condor{ckpt\_server} is working properly once
  again. 

\item Fixed bugs in Condor's Generic Connection Broker (GCB) support.
  Condor version 6.7.14 is linked with a new version of the GCB
  library (1.3.1) that fixes a major bug in how GCB handles UDP
  messages.
  Previous versions of GCB had a UDP receive buffer that was far too
  small, resulting in many dropped UDP packets.
  Now, GCB will dynamically allocate more buffer space as needed.
  The new version of GCB also adds support for comments (any line
  beginning with \verb@#@) in the GCB routing table.
  For more information about GCB, see section~\ref{sec:GCB} on
  page~\pageref{sec:GCB}. 

\item Update job information such as ImageSize, RemoteUserCpu and RemoteSysCpu 
at job completion.  Previously this was only done periodically.

\item Fixed a bug that could cause the \Condor{gridmanager} to crash when
a job using job leases left the queue.

\item Fixed a bug that could cause the \Condor{schedd} to repeatedly start
the \Condor{gridmanager} to manage jobs that were complete. This would
happen when LeaveJobInQueue evaluated to True.

\item When \Macro{GSI\_DAEMON\_TRUSTED\_CA\_DIR} is set, pass the setting
down to the gt4 gahp server.

% Gnats PR 518
\item Fixed a bug in \Condor{dagman} that caused the UNLESS-EXIT
feature to not work with POST scripts (the return value from a POST
script was not tested against the UNLESS-EXIT value).

% Gnats PR 469
\item Fixed a bug in \Condor{dagman} that caused POST scripts to work
incorrectly with node retries:  if the node job failed for a node
with retries, the POST script was only run on the last retry.

% Gnats PR 582
\item Fixed a bug in \Condor{dagman} that caused rescue DAGs to fail
if the original DAG was run with the \Opt{-UseDagDir} command-line flag.
(This bug was introduced at some point after version 6.7.10 and before
version 6.7.13.)

\item Improved usage of uid caching introduced in 6.6.0.  This will
further reduce load on NIS servers.  See the discussion of
\MacroNI{PASSWD\_CACHE\_REFRESH} in
section~\ref{param:PasswdCacheRefresh} on
page~\pageref{param:PasswdCacheRefresh} for more details.

\item Fixed a bug in 6.7.13 for Windows causing incorrect handling of
absolute paths in the job's output/error if the path began with a
forward slash rather than a backslash.

\item Fixed a bug in the \Condor{master} that caused ``condor\_off
-subsystem'' (and similar commands) to fail if the daemon name wasn't
hard-coded into the \Condor{master}.  The \Condor{master} now handle
any daemon listed in the \Macro{DAEMON\_LIST} for these commands.

\item Fixed a bug in the \condor{gridmanager} that caused it to undercount
already-submitted jobs at start-up for purposes of job throttling to Globus
grid resources.

\item Improved the handling of job leases for grid-type condor jobs. There
was a race condition between the lease expiring and the
\Condor{gridmanager} attempting to extend the lease. Also, a lease could
be set and not extended well before the job was actually submitted. In
certain cases, the forwarded lease could exceed \Attr{JobLeaseDuration}.

\item The startd no longer advertises itself as available to run jobs
when it is in shutdown mode (e.g. waiting for jobs to finish).  This
was a noticeable problem when using large values for
\Macro{MaxJobRetirementTime} on multi-VM startds; while waiting for
one of its VMs to finish running a job, the startd would be available
for matching to jobs, but it would reject them when the schedd tried
to start them, possibly causing an endless cycle of matching, attempting
to run, and failing.

\item Fixed some minor typos and formatting bugs in some of the log
  messages generated by the \Condor{ckpt\_server}.

\item For grid-type condor jobs, the \Condor{gridmanager} now notices
when a job disappears from the remote \Condor{schedd} unexpectedly.

\item When getting ``connection refused'', Condor command-line tools
and daemons no longer continuously retry the connection attempt until
timing out.  These retries were causing 10 second or longer delays
when trying to connect to Condor services which, for one reason or
another, were no longer listening on the expected TCP port.

\item Fixed a bug in the \Condor{c-gahp} that could cause it to crash
if it fails to connect to a remote \Condor{schedd} when submitting a
job.

\item Minor memory leaks have been fixed.

\item Fixed a bug in Quill that could result in an infinite loop in
\Condor{q} when querying Quill.

\end{itemize}

\noindent Changes:

\begin{itemize}

\item The \Condor{dagman} log file path is converted to an absolute
path inside \Condor{dagman} itself, so that the logging works for
multi-directory rescue DAGs (which it didn't before), but the
\File{.condor.sub} files are still portable.

\item Added the Stork log file (if any) to the list of log files that
\Condor{dagman} lists in the dagman.out file.

\item \Condor{dagman} now reports the node return value for all failed
nodes.

\item Attributes names forced into the job ad via '+' are no longer
converted to lower-case. This conversion was a side-effect of a bug-fix
in 6.7.11 and caused problems with code that assumed that Condor would
preserve the case of attribute names.

\item Job policy expressions are now evaluated on COMPLETED and REMOVED
jobs in the schedd.

\end{itemize}

\noindent Known Bugs:

\begin{itemize}

\item The \Macro{NEGOTIATOR\_MATCHLIST\_CACHING} setting is broken.
  It should not be used.
  This setting is \verb@FALSE@ by default, but if set to \verb@TRUE@,
  the \Condor{negotiator} will crash.  

\item Jobs that are are placed on held because of \AdAttr{on\_exit\_hold}
evaluated to TRUE or jobs that stay in the queue after finishing because
\AdAttr{on\_exit\_remove} evaluated to FALSE will erroneously report the reason
as "UNKNOWN (never set)".

\end{itemize}


%%%%%%%%%%%%%%%%%%%%%%%%%%%%%%%%%%%%%%%%%%%%%%%%%%%%%%%%%%%%%%%%%%%%%%
\subsection*{\label{sec:New-6-7.13}Version 6.7.13}
%%%%%%%%%%%%%%%%%%%%%%%%%%%%%%%%%%%%%%%%%%%%%%%%%%%%%%%%%%%%%%%%%%%%%%

\noindent Release Notes:

\begin{itemize}

\item Added a new natively compiled clipped port for the Red Hat
Enterprise Linux 3 IA64 distribution.

\end{itemize}

\noindent New Features:

\begin{itemize}

\item Added support complete support for Quill on Windows, so job queues can
now be accessed via a relation database.  Quill is now available on all Condor
supported platforms.  See page~\pageref{sec:Quill} for more information.

\item Added support in Condor for the Generic Connection Broker
  (GCB).
  This is a system for managing network connections across public and
  private networks.
  More information about GCB can be found in section~\ref{sec:GCB} on
  page~\pageref{sec:GCB}.

\item Added a new configuration option, \Macro{BIND\_ALL\_INTERFACES}
  This is a boolean value that controls if Condor should bind and
  listen to all the network interfaces on a multi-homed machine.
  If set to TRUE, the value of \Macro{NETWORK\_INTERFACE} will only
  control what IP address is published by Condor daemons, even though
  they will still be listening on all interfaces.
  The default is FALSE.

\item Added a \Opt{-pool} option to \Condor{submit}. It lets you submit
jobs to a \Condor{schedd} in a different pool. The other options to
\Condor{submit} now have long names, but the single-character versions
still work.

\item ``grid\_resource'' can now be used to directly set the new grid
universe job attribute ``GridResource.'' The old attributes still work,
but they will be ignored if ``grid\_resource'' is present. As a
side-effect, ``stream\_output'' and ``stream\_error'' will default to
``False'' for all jobs.

\item X509 user proxies are now updated for vanilla universe jobs.   If
a job specifically sets x509userproxy and is using file transfer, when
the proxy file is updated, it will be transfered to the running job.

\item If a  cycle is detected in the DAG while 
running, \Condor{dagman} now prints (in the \File{dagman.out} file)
the status of all DAG nodes.

\item BeginTransaction call in \Condor{schedd}'s SOAP interface now
  notifies the caller if too many transactions are currently running
  via an error code of FAIL. Previous behavior was to abort a running
  transaction in order to allow the BeginTransaction call to succeed.
  
\item \Macro{MAX\_SOAP\_TRANSACTION\_DURATION} config option added so that a
  single transaction cannot take up too many \Condor{schedd}
  resourced. This option specifies an optional maximum duration
  between SOAP calls in a single transaction.

\item If a machine is acting as both a submit and an execute node, and it
  cannot communicate with the central manager, it will attempt to run jobs
  locally.  If Condor specific terms, if the \Condor{schedd} fails to hear
  from the central manager, it will attempt to run jobs on a locally running
  \Condor{startd}.  The \Condor{SCHEDD\_ASSUME\_NEGOTIATOR\_GONE} config
  macro was added to support this feature; see
  page~\pageref{param:ScheddAssumeNegotiatorGone} for details.

\item You can now specify per-subsystem entries in your condor\_config file
by prepending the subsystem name and a period to the normal name.  The
per-subsystem settings take precedence over the regular settings.

\item \Condor{dagman} now recovers automatically after being abruptly
killed by something other than Condor itself (e.g., by Unix initd
during a ``fast'' system shutdown).  This is accomplished through the
use of a default \Attr{OnExitRemove} expression inserted by
\Condor{submit\_dag} which instructs the \Condor{schedd} not to treat
death by SIGKILL as a valid exit condition for \Condor{dagman}.

\item Added submit attribute \SubmitCmd{globus\_xml}, for use with grid-type
gt4 jobs. The given XML text will be inserted at the end of the XML job
description written by Condor for submission to the WS-GRAM server.

\item For grid-type gt4 jobs, if a URL scheme is missing from the resource
name, ``https://'' will be inserted automatically.

\item Added submit attribute \SubmitCmd{transfer\_output\_remaps}.
This specifies the name (and optionally path) to use when downloading output
files from the completed job.  Normally output files are transferred back
to the initial working directory with the same name they had in the execution
directory.  This gives you the option to save them with a different path
or name.

\end{itemize}

\noindent Bugs Fixed:

\begin{itemize}

\item Fixed a bug concerning backslash escaping in classad attribute values
	when \Condor{q} was using quill.

\item Fixed a bug where \Condor{q} could not accept multiple jobids
	on the command line.

\item Fixed parallel universe ssh script to now clean up all
temporary files it creates.

\item Fixed a bug in the dedicated scheduler that caused it to request
resources it could not use, resulting in longer job startup times.

\item Fixed a bug in the \Condor{schedd} that caused grid-type gt2 jobs
submitted by an older \Condor{submit} or in the queue during an upgrade
(version 6.7.10 or earlier) to go
on hold if the grid\_type was ``globus''.

\item Fixed a bug in \Condor{submit} that caused it to not set 
\Attr{JobGridType} in the job ad for grid universe jobs when submitting
to a \Condor{schedd} older than version 6.7.11.

\item When using file transfer, transferring the results back to
the submit machine could silently fail for Condor releases 6.7.0
though 6.7.12.  This was relatively rare through 6.7.10.  For
6.7.11 and 6.7.12, the bug would be easily triggered if a vanilla
job had an X509 user proxy associated with it.  This is now
fixed.

\item Fixed a logic bug in the \Condor{schedd}.
  Previously, if there was an error expanding any \verb@$$(attribute)@
  references in a job classad when trying to spawn a \Condor{shadow},
  the \Condor{schedd} would die with the fatal exception ``Impossible:
  GetJobAd() returned NULL for X.Y but that job is already known to
  exist''.
  Now, the \Condor{schedd} correctly distinguishes between a non-fatal
  error expanding \verb@$$(attribute)@ and the fatal error of the job
  already being gone (which is, in fact, impossible).
  This bug was first introduced in Condor version 6.7.1.

\item The reason strings generated when a user job policy expression fires
are now consistent for grid universe jobs.

\item The \Condor{gridmanager} now evaluates the periodic job policy
expressions at the interval set by \Macro{PERIODIC\_EXPR\_INTERVAL}.

\item Fixed a bug which prevented standard universe from working on a linux 
kernel post 2.6.12.2.


\item The \Condor{schedd} used to crash in certain cases if a given
  job was vacated using \Condor{vacate\_job}, then put on hold and
  released.
  The bug only appeared if a specific job id was given to
  \Condor{vacate\_job}, as opposed to specifying a username or another
  constraint.
  Now, the use of \Condor{vacate\_job} for individual job identifiers
  is safe and the \Condor{schedd} will not crash.
  This bug has been in Condor since support for \Condor{vacate\_job}
  was first added in version 6.7.0.

\item Fixed a bug that caused the \Condor{gridmanager} to crash if a
grid-type condor job ad contained the attribute \Attr{remote\_}.

\item Fixed a bug with the FS\_REMOTE authentication mechanism that caused
it to fail occasionally when using NFS.

% Gnats PR 553:
\item Fixed a bug in which a double terminated event in a DAG node with
a POST script could cause \Condor{dagman} to abort the DAG and claim
that a cycle exists in the DAG.

% Also related to Gnats PR 553:
\item In the DAG status messages in \File{dagman.out} files,
\Condor{dagman} now shows nodes with queued PRE or POST scripts
in the Pre or Post columns.  Previously, these nodes were shown
in the Un-Ready column.

\item Fixed the GetFile SOAP call on the \Condor{schedd} so that it
  behaves more like POSIX read() and does not report errors when
  trying to read more data than is available.

% Gnats PR 557:
\item Fixed a hash function bug that could cause \Condor{dagman}
to crash.

\item \Attr{JobCurrentStartDate} and \Attr{JobLastStartDate} are no longer
changed in the job ad when the \Condor{schedd} and \Condor{shadow} reconnect
to a running job after a crash.

% Gnats PR 554:
\item \Condor{dagman} now allows POST scripts to be used with DATA
nodes in a DAG (previously this caused the DAG to hang).

\item Using the new \SubmitCmd{Remote\_} simplified syntax no longer
generates unnecessary debug messages.

% Quill
\item Fixed a bug in estimating the size of attribute value buffers that
caused quill to crash.  This arose when job ads had variables with very
large values (more than 3KB).  

\item Fixed a bug in the \Condor{gridmanager} that could cause it to crash
when the \Attr{Rematch} attribute evaluates to True.

% http://bugzilla.globus.org/globus/show_bug.cgi?id=3802
% http://bugzilla.globus.org/globus/show_bug.cgi?id=3803
\item The default base scratch directory for WS-GRAM doesn't exist on most
server machines. Added a work-around to create the directory as part of the
job submission.

\item Starting in version 6.7.11, the execute host reported for grid jobs
in the user log execute event can contain spaces. The C++ user log reading
code now properly reads the entire string for these events.

\item Fixed a bug that caused the \Condor{gridmanager} to die when it 
tried to renew the job lease of a grid-type condor job.

\item Fixed a bug that was causing the \Condor{schedd} to crash on Solaris
if the cron macros aren't defined.

\item Fixed a bug where output may be lost when spooling (with the -s option to
\Condor{submit} or implicitly with Condor-C).  This bug could only happen if
the job terminated within one second of starting.

\item Fixed a bug affecting transferral of output and error files where the
file specified in the submit file contains path information.  The file
was being staged back into the initial working directory and then it was
copied to the final path specified.  The bug is that if there was an error
copying the file to the final location, the intermediate copy would not
be deleted and the job would still exit successfully, as though it had
succeeded.  Now, no intermediate copy of the file is made, and errors
in transferring the file will be treated as a failure to run the job,
which will typically cause the job to return to idle state and run again.

\end{itemize}

\noindent Changes:

\begin{itemize}

\item Added a couple missing parameters to the example configuration file
\File{condor\_config.generic}.

\item Slightly cleaned up event checking error messages in \Condor{dagman}.

\item Fixed a bug in the \Condor{c-gahp} that caused it to crash when
handling grid-type condor jobs with job leases.

\item Starting in 6.7.11, the ``JM-Contact'' field of the ``Job submitted
to Globus'' user log event was mis-printed. This has been corrected.

% Gnats 565
\item Fixed bug that prevented Stork detection of hung jobs.

% Gnats 562
\item Fixed an obscure bug that incorrectly quoted the status of completed
jobs, visible via \Stork{status}.

\end{itemize}

\noindent Known Bugs:

\begin{itemize}

\item The \Condor{ckpt\_server} is broken in version 6.7.13.
Please do not attempt to use it.
It is safe to use the 6.7.12 \Condor{ckpt\_server} in a pool running
6.7.13 until the 6.7.14 release is out.
Of course, the 6.7.12 \Condor{ckpt\_server} will not work with GCB, so
sites wishing to use both GCB and a \Condor{ckpt\_server} will have to
wait for 6.7.14.

% Gnats PR 582:
\item Rescue DAGs generated from DAGs run with the
\Opt{-UseDagDir} command-line flag no longer work.
(The original run with \Opt{-UseDagDir} should work,
but if it fails and generates a rescue DAG, the
rescue DAG will \emph{always} fail.)

\end{itemize}


%%%%%%%%%%%%%%%%%%%%%%%%%%%%%%%%%%%%%%%%%%%%%%%%%%%%%%%%%%%%%%%%%%%%%%
\subsection*{\label{sec:New-6-7.12}Version 6.7.12}
%%%%%%%%%%%%%%%%%%%%%%%%%%%%%%%%%%%%%%%%%%%%%%%%%%%%%%%%%%%%%%%%%%%%%%

\noindent Release Notes:

\begin{itemize}

\item 6.7.12 addresses several critical bugs in 6.7.11.  6.7.11 should 
not be used.

\end{itemize}

\noindent Bugs Fixed:

\begin{itemize}

\item Fixed a serious bug introduced in 6.7.11 which prevented \Condor 
{dagman} from successfully removing its own jobs from the Condor  
queue after receiving a \Condor{rm} request from the \Condor{schedd}.

\item Fixed a serious bug introduced in 6.7.11 where the \Condor{master} 
on Windows would not properly shut down.

\end{itemize}


%%%%%%%%%%%%%%%%%%%%%%%%%%%%%%%%%%%%%%%%%%%%%%%%%%%%%%%%%%%%%%%%%%%%%%
\subsection*{\label{sec:New-6-7.11}Version 6.7.11}
%%%%%%%%%%%%%%%%%%%%%%%%%%%%%%%%%%%%%%%%%%%%%%%%%%%%%%%%%%%%%%%%%%%%%%

\noindent Release Notes:

\begin{itemize}

\item Condor is now linked against GSI from Globus 4.0.1.

\item GSI security and the grid universe should now work in the Alpha 
Linux port.

\item All Condor release packages are now compressed with GNU's
  \Prog{gzip}.
  We no longer ship releases compressed with the vendor's
  \Prog{compress} utility.

\end{itemize}


\noindent New Features:

\begin{itemize}

\item Added a new feature called \Bold{Quill} to Condor which allows an
        SQL server to mirror the job queue in order to speed up queries about
        the job queue via \Condor{q} and \Condor{history}. Please see 
        page~\pageref{sec:Quill} for the description of this feature.

\item \Condor{dagman} has a new \Opt{-maxidle} command-line argument
that can be used to throttle DAG job submissions according to the number
of idle jobs in the DAG.

\item \Stork{submit} is now able to search for X.509 credentials in the
standard locations.

\item The \Condor{negotiator} can now limit how long it negotiates with a 
single submitter before moving on to the next one. 

\item On platforms and filesystems that support files larger than 2
GB, the history file can now be larger than 2 GB.

\item Added two options to \Condor{q}: \Opt{-jobads} and
  \Opt{-machineads}. They will take ads from files instead of the
  schedd and collector, respectively. These options are mostly useful
  for debugging. 

\item Added a new, hopefully less confusing, Cron (Hawkeye)
configuration syntax.  The old syntax is still supported, but should
be considered deprecated, and will eventually go away.  The new syntax
splits the old colon separated ``name:prefix:executable:period''
string into separate macros.

\item Improved support for job leases. ``job\_lease\_duration''  now works for
grid-type condor jobs. New job ad attribute ``TimerRemove'' specifies a
specific time at which a job should be removed. These attributes will be 
passed through multiple layers of grid-type condor jobs.

\item Grid universe jobs now use a unified pair of attributes 
(``GridResource'' and ``GridJobId'') to identify the remote resource. This
will make it possible to match jobs to multiple types of resources. The
submit file syntax remains the same for now, except that ``remote\_pool'' is
now required for grid-type condor jobs.

\item Significantly improved response time for \Condor{q} when job classads
are larger than 4 kbytes (by disabling TCP Nagle algorithm as appropriate).

\end{itemize}

\noindent Bugs Fixed:

\begin{itemize}

% posting that belongs in 6.7.12 manual
%\item Fixed a bug which prevented standard universe from working on a linux 
%kernel post 2.6.12.2.

\item Fixed bug in the dedicated scheduler where if the \Condor{startd} 
rejected a match, the \Condor{schedd} would never retry new matches 
for that machine.  This would result in MPI and parallel jobs sticking 
in the Idle state, and the message "DedicatedScheduler::negotiate sent 
match for machine, but we've already got it".

\item Fixed problem with the parallel universe to allow for LAM jobs
to get SIGTERM on exit so they can exit cleanly.

% condor-support #1419
\item Fixed a bug that was visible to the end user as file transfer
  failures on a busy system.
  The root problem was that if the \Condor{negotiator} gave out the
  same match twice (due to having stale info in the \Condor{collector}
  when trying to negotiate), the \Condor{schedd} would be confused,
  attempt to re-use the match, fail to do so, and then kill the
  previous (legitimate) use of the match.
  This bug was introduced in version 6.7.4.

\item Fixed bug in the parallel universe that caused the schedd to
crash when reconnecting to jobs that couldn't be reconnected to.

\item Fixed bug in parallel shadow which caused Shadow Exceptions
in parallel jobs when the components exited in the wrong order.

% Gnats PR 534.
\item Fixed a bug in \Condor{dagman} that caused it to fail on Windows
for DAGs with nodes having absolute paths to their log files.  (This bug
was introduced in version 6.7.10.)

% Gnats PR 492.
\item Fixed a bug whereby \Condor{dagman} could crash after executing
the POST script of a node whose Condor job had never been successfully
submitted due to repeated \Condor{submit} failures.  (This bug was
introduced in 6.7.7 or earlier.)

\item Fixed a bug in a debug message.  If an error occurred during file
transfer, Condor would print the wrong expected filesize in the error message
on some platforms.

\item Fixed bug where \Stork{submit} was corrupting log notes passed from the
command line.  This bug also had the effect of disabling Stork jobs running
from DAGMan versions v6.7.10, and later.

% Gnats PR 525.
\item If you have DATA nodes in your DAG but no Stork log specified
(with the \Opt{-Storklog} argument),
\Condor{dagman} now fails with an explanatory message when parsing
the DAG file(s). (Previously, it would just wait forever for the Stork
jobs to finish, because it wouldn't see the relevant events.)

\item In \Condor{dagman}, argument quoting for \Stork{submit} now matches
argument quoting for \Condor{submit}.

\item Corrected how \Condor{submit} handles attributes forced into the
job ad with '+'. Now, the attribute names are case-insensitive, they 
are not treated as normal submit attributes, and they always over-ride 
normal submit attributes.

\item Fixed bugs that would cause a segfault when reading a classad from
a file. Triggered by consecutive blank lines and lines containing only
white-space.

\item Fixed a bug that could cause duplicated output when a gt4 grid job 
is executed more than once.

\item Fixed a bug that could cause the \Condor{gridmanager} to assert if 
it tried to delegate credentials for gt4 grid jobs before the gahp server 
was started.

\item Fixed a race condition that could cause condor grid-type jobs to be
held with hold reason ``Spooling input data files''.

\item \Condor{glidein} now correctly handles extracting necessary
information from modern Condor configurations where
\MacroNI{NEGOTIATOR\_HOST} is not defined.

% Programmer explanation:
% MANAGED is no longer boolean in the job ad, it's
% a string of complex state.  This enables the gridmanager
% to no longer erroneously take control of jobs that are
% done but haven't left the queue.
\item Refinements in how grid universe components track jobs.  Grid universe
jobs are less likely to generate multiple terminate events in the job's user
log.  There will also be slight performance improvements are redundant work 
is no longer done.

% Gnats PR 542
\item Fixed a bug in \Condor{dagman} that caused it to core dump on
a 'job reconnected' event from a node job.

\item \Condor{submit} will now exit zero as long as the submission succeeds.  Debugging output will still be printed if the internal reschedule fails.

% RUST condor-support #1402
\item On Windows, exited child processes of the Condor services
will be handled in order of termination.  This fixes the problem where jobs
submitted from a Windows machine appear to run much longer than normal
because the \Condor{schedd} fails to notice that a \Condor{shadow} exits
when the system is very busy.

% Gnats PR 532
\item Fixed a bug that caused scheduler universe jobs to often wait five
minutes (or whatever \MacroNI{SCHEDD\_INTERVAL} is set to) before running.

\item Fixed a bug that prevented the \Condor{starter} from running on a
Win32 machine with a FAT32 filesystem.

\item A reschedule command will now be sent to the \Condor{schedd} whenever
a job is released from held state. This should make grid-type condor jobs
start much faster.

\item Config parameters \MacroNI{GAHP} and \MacroNI{GAHP\_ARGS} have been
deprecated. \MacroNI{GT2\_GAHP} should be used instead.

\end{itemize}

\noindent Changes:

\begin{itemize}

\item \Condor{configure} no longer creates a
\begin{verbatim} $(LOCAL_DIR)/ViewHist \end{verbatim}
directory, which was begun in version 6.7.10.  This directory was of limited
value for most users.

\end{itemize}

\noindent Known Bugs:

\begin{itemize}

\item None.

\end{itemize}



%%%%%%%%%%%%%%%%%%%%%%%%%%%%%%%%%%%%%%%%%%%%%%%%%%%%%%%%%%%%%%%%%%%%%%
\subsection*{\label{sec:New-6-7-10}Version 6.7.10}
%%%%%%%%%%%%%%%%%%%%%%%%%%%%%%%%%%%%%%%%%%%%%%%%%%%%%%%%%%%%%%%%%%%%%%

\noindent Release Notes:

\begin{itemize}

\item This release contains all of the bug fixes and improvements from
  the 6.6 stable series up to and including version 6.6.10.

\item The Mac OS X binaries shipped with this release were built on OS
  10.3.  Previous versions of Condor for OS X were built with version
  10.2.  Condor is officially dropping support for Mac OS 10.2 with
  this release (though it is possible the 10.3 binaries still work, we
  have not verified it either way).  These binaries are known to work
  with Mac OS 10.4 (``Tiger''), as well.

\item There is a minor bug in version 6.7.10's \Condor{configure}
  script.
  It will create a directory called \File{ViewHist} in the local
  directory (next to \File{log}, \File{spool}, etc).
  This directory is not used by Condor at all, except in the case of a
  \Condor{view} collector (which is optional, and not enabled by
  default). 
  This behavior will be removed in version 6.7.11, and
  \Condor{configure} will go back to not creating the \File{ViewHist}
  directory.

\end{itemize}

\noindent New Features:

\begin{itemize}

\item \Condor{dagman} can now run multiple DAGs in separate directories.


\item Added \Macro{DAGMAN\_CONDOR\_SUBMIT\_EXE},
\Macro{DAGMAN\_STORK\_SUBMIT\_EXE}, \Macro{DAGMAN\_CONDOR\_RM\_EXE},
and \Macro{DAGMAN\_STORK\_RM\_EXE} configuration settings to specify
the \Condor{submit}, \Stork{submit}, \Condor{rm}, and \Stork{rm}
executables used by \Condor{dagman}.  If unset (which they are by
default), \Condor{dagman} looks for each in the PATH.

\item For Condor-C jobs, the \Condor{gridmanager} will retry and delay
  failed connections to a remote \Condor{schedd} like it does for
  Condor-G jobs. The same configuration settings apply 
  (\Macro{GRIDMANAGER\_CONNECT\_FAILURE\_RETRY\_COUNT} and
  \Macro{GRIDMANAGER\_RESOURCE\_PROBE\_INTERVAL}).

\item \SubmitCmd{remote\_initialdir} is now supported in all universes except
for standard universe.  Previously, it was only supported in the grid universe.

\item \SubmitCmd{+Remote\_} syntax for Condor-C jobs has been
simplified for the specific commands of
\SubmitCmd{universe}, \SubmitCmd{remote\_schedd}, \SubmitCmd{remote\_pool}, \SubmitCmd{globus\_rsl}, and \SubmitCmd{globus\_scheduler}.

\item Added default user priority factors for accounting groups.  More on
        accounting groups will be available in future versions of the manual.

\item The \Condor{startd} can now be configured to write out the
  \Attr{ClaimId} of the next available claim for each virtual machine
  to separate files.
  This functionality will enable enhanced fault tolerance in future
  versions of Condor.
  For more information, see section~\ref{param:StartdClaimIdFile} for
  details on \Macro{STARTD\_SHOULD\_WRITE\_CLAIM\_ID\_FILE} and
  \Macro{STARTD\_CLAIM\_ID\_FILE}, the two configuration settings that
  control this behavior.

\end{itemize}

\noindent Bugs Fixed:

\begin{itemize}

\item Fixed bugs on the Win32 platform in the \Condor{schedd} that could
cause jobs to never complete when the \Condor{schedd} is busy with many jobs
running at once.

\item Fixed a bug on Windows where if lots of jobs submitted were from
  the same \Condor{schedd}, some of the \Condor{shadow} processes
  would block for an extremely long time trying to get a lock for
  writing to the \File{ShadowLog} file.
  Now, log writing happens more fairly, and no \Condor{shadow}
  processes can be delayed indefinitely.

\item \Condor{submit} \Opt{-name} formerly had no effect on Windows and
did not work properly. This is now fixed.

\item Significantly sped up the removal of large groups of jobs by
changing the default value of \MacroNI{JOB\_IS\_FINISHED\_INTERVAL} from 1
to 0 (see section~\ref{param:JobIsFinishedInterval} for details on this
setting).

\item Improved performance of the \Condor{schedd} when not running as
  root.
  In version 6.7.7, the new code to support the scheduler universe
  with Condor-C involved adding some additional overhead to the
  \Condor{schedd}.
  However, this overhead is not needed unless the \Condor{schedd} is
  running as root.
  In version 6.7.10, the \Condor{schedd} notices if it is not root and
  does an optimization to avoid the overhead.

\item Fixed a bug that caused the gridmanager to crash if a gt2, gt3, or
gt4 grid job had a proxy that couldn't be read properly. Now the job gets
put on hold.

\item The Condor-C GAHP now performs file staging in a 
separate process, allowing remote grid jobs to be started earlier.

\item When contacting the embedded web server on Condor daemons,
  authentication is no longer requested.
  The previous authentication requirement didn't provide any
  additional security, and could confuse users.

\item Fixed rare bug that could cause \Condor{submit} to crash when
  both getenv=true and environment=... were in a submit file and when
  very large variable names were in the environment. 

\item Fixed a rare bug where the \Condor{schedd} would die with a
  fatal exception under extremely heavy load on the machine.
  The error message was:
\begin{verbatim}  
  ERROR ``Impossible: Create_Thread child_errno (xxx) is not
  ERRNO_PID_COLLISION!'' at line 6181 in file daemon_core.C
\end{verbatim}  

\item Fixed a rare bug where certain attributes in a job description
  file could cause the \Condor{schedd} to crash when restarting and
  parsing the \File{job\_queue.log} file.

\item Improved performance of standard universe jobs when
  \Attr{WantRemoteIO} is set to false in the job ClassAd.  
  In this case, Condor's checkpointing libraries now avoid some
  additional communication with the \Condor{shadow} which are not
  required if there's no remote IO.

\item Fixed some messages in the Condor log files that were improperly
  formatted, or contained incomplete information.

\item Improved some user-log-reading error messages in \Condor{dagman}.

\item Removed support for deprecated \Opt{-NoPostFail} option from
\Condor{dagman}.  (The same functionality can be achieved through the
use of a simple POST script.)

\item Fixed bug in dedicated scheduler, where under heavy load, the
  schedd would occasionally try to start the same job twice, and
  subsequently exit with the message:
\begin{verbatim}  
  ERROR ``Trying to run job x.x, but already marked RUNNING!''
\end{verbatim}  

\item Fixed bug in dedicated scheduler, so that it now creates a 
  spool directory for each condor proc of a parallel or MPI job
  with multiple requirements.

\end{itemize}

\noindent Known Bugs:

\begin{itemize}

% Gnats PR 534.
\item On Windows only, \Condor{dagman} fails for DAGs with
nodes having absolute log file paths in their submit files.

% Gnats PR 492.
\item \Condor{dagman} does not correctly handle the case where all
submit attempts for a node job fail, and the node has a POST script.
If this happens for a single node in a DAG, it is usually okay,
but if it happens for a second node, \Condor{dagman} will crash.

\item The Condor-C GAHP now performs file staging in a 
separate process, allowing remote grid jobs to be started earlier.

\item Using the new \SubmitCmd{Remote\_} syntax simplification causes
\Condor{submit} to display debug messages to standard output, possibly
confusing programs that parse \Condor{submit}'s output.  Fixed in 6.7.13.

\end{itemize}


%%%%%%%%%%%%%%%%%%%%%%%%%%%%%%%%%%%%%%%%%%%%%%%%%%%%%%%%%%%%%%%%%%%%%%
\subsection*{\label{sec:New-6-7-9}Version 6.7.9}
%%%%%%%%%%%%%%%%%%%%%%%%%%%%%%%%%%%%%%%%%%%%%%%%%%%%%%%%%%%%%%%%%%%%%%

\noindent Release Notes:

\begin{itemize}

\item This release contains all of the bug fixes and improvements from
  the 6.6 stable series up to and including version 6.6.10.

\end{itemize}

\noindent New Features:

\begin{itemize}

\item The Parallel Universe has been added.
  For more information, see section~\ref{sec:Parallel} on
  page~\pageref{sec:Parallel}.


\item 
\index{environment variables!X509\_USER\_PROXY}
\index{X509\_USER\_PROXY}
The environment variable \Env{X509\_USER\_PROXY} is set to the
full path of the proxy if a proxy is associated with the job.
This is usually done using \SubmitCmd{x509userproxy} in the submit file.
This currently works in the local, java, and vanilla universes.

\item \Condor{submit} generates more precise error messages in 
some failure cases.

\item \Condor{hold}, \Condor{release} and \Condor{rm} now allow the user
to change the HoldReason, ReleaseReason or RemoveReason with the -reason
flag.

\item \Condor{dagman} no longer does a one-second sleep before each
submit if all node jobs have the same log file.  (The sleep is still
needed if there are multiple log files, for unambiguous ordering of
events during bootstrapping.)  Note that if \MacroNI{DAGMAN\_SUBMIT\_DELAY}
is specified, the specified delay takes effect whether or not all
jobs have the same log file.

\end{itemize}

\noindent Bugs Fixed:

\begin{itemize}

\item Many crashes related to running the Dedicated Scheduler have
been fixed.

\item Setting COLLECTOR\_HOST or NEGOTIATOR\_HOST with a port but without
a hostname no longer causes the \Condor{master} to crash.

\item The Condor-G Grid Monitor now works with Globus 4.0 pre-Web Services
GRAM.

\item Several deadlocks in the Condor-C GAHP server have been fixed.

\end{itemize}

%%%%%%%%%%%%%%%%%%%%%%%%%%%%%%%%%%%%%%%%%%%%%%%%%%%%%%%%%%%%%%%%%%%%%%
\subsection*{\label{sec:New-6-7-8}Version 6.7.8}
%%%%%%%%%%%%%%%%%%%%%%%%%%%%%%%%%%%%%%%%%%%%%%%%%%%%%%%%%%%%%%%%%%%%%%

\noindent Release Notes:

\begin{itemize}

\item This release contains all of the bug fixes and improvements from
  the 6.6 stable series up to and including version 6.6.9.

\end{itemize}

\noindent New Features:

\begin{itemize}

\item Controlling whether or not a standard universe job asks the
\Condor{shadow} about how/where to open every single file can be better
controlled with the \Attr{want\_remote\_io} attribute in the submit
description file.
This attribute can be set to true or false and it is true be default.
If set to false, then this attribute forces a standard universe job in 
Condor to always look to the local file system when opening files and not
to contact the shadow. 
This increases performance of user jobs where the jobs open a very large
amount of files in a small space of time.
However, the user jobs must be matched to machines that have the same
UID\_DOMAIN and FILESYSTEM\_DOMAIN, as per vanilla universe jobs with a 
homogeneous file system.

\item \Condor{dagman} now has the capability to run more than one
independent DAG in a single \Condor{dagman} process.

\item User policy expressions (on\_exit\_remove and on\_exit\_hold)
now work for scheduler universe jobs.

\item TotalCpus and TotalMemory are now set in machine ads.

\item \Condor{dagman} now tolerates the "two terminated events for
a single job" bug by default.  There is a new bit in
\MacroNI{DAGMAN\_ALLOW\_EVENTS} to control whether this bug is considered
a fatal error in a \Condor{dagman} run.

\item Added a new debug formatting flag, \Dflag{PID}, that prints out
  the process id (PID) of the process writing a given entry to a log
  file.
  This is useful in Condor daemons (such as the \Condor{schedd}) where 
  the daemon can fork() multiple processes to perform various tasks
  and it is helpful to see what log messages are coming from forked
  process versus the main thread of execution.
  The default \Macro{SCHEDD\_DEBUG} in the sample configuration files
  shipped with Condor now includes this flag.

\item When \Condor{dagman} writes rescue files, each node is now
specified with the same number of retries as was specified in the
original DAG, rather than with only the ``remaining'' number of
retries based on the failed run.  The latter behavior can be restored
by setting \Macro{DAGMAN\_RESET\_RETRIES\_UPON\_RESCUE} to false.

\item Added ``Hawkeye'' capabilities to \Condor{schedd}.  It's
configured identically to that of \Condor{startd}, but using 
``SCHEDD'' in place of ``STARTD'', in particular for the
``SCHEDD\_CRON\_NAME'' macro.

\end{itemize}

\noindent Bugs Fixed:

\begin{itemize}

% See Gnats PRs 297 and 430.
\item Fixed a bug in \Condor{dagman} that prevented POST scripts
from being used with jobs that write XML-format logs.

\item The event-checking code used by \Condor{dagman} now defaults
to allowing an execute event before the submit event for the same
job; if this happens, there will be a warning, but the DAG will
continue.  See section~\ref{param:DAGManAllowEvents} for more info.

\item \Condor{userprio} option \Opt{-pool} was failing with ``Can't
find address for negotiator'' since version 6.7.5.

\item Fixed a bug the prevented SOAP clients from being able to access
a job's spooled data files if the \Condor{schedd} restarted.

\item Fixed a bug that caused the \Condor{gridmanager} to panic when
trying to retire a job from the queue that was already gone. This
could cause multiple terminate events to be logged for some jobs.

\item Fixed a bug that caused match-making to not work for Condor-C
jobs.

\item Added workaround for a Globus bug that can cause re-execution of
a completed GT2 job in the correct failure case (Globus bugzilla ticket
3411).

\item Properly extend the lifetime of GT4 jobs and credentials on the
remote server.

\end{itemize}

%%%%%%%%%%%%%%%%%%%%%%%%%%%%%%%%%%%%%%%%%%%%%%%%%%%%%%%%%%%%%%%%%%%%%%
\subsection*{\label{sec:New-6-7-7}Version 6.7.7}
%%%%%%%%%%%%%%%%%%%%%%%%%%%%%%%%%%%%%%%%%%%%%%%%%%%%%%%%%%%%%%%%%%%%%%

\noindent Release Notes:

\begin{itemize}

\item This release contains all of the bug fixes and improvements from
  the 6.6 stable series up to and including version 6.6.9.

\end{itemize}

\noindent New Features:

\begin{itemize}

\item The \Expr{STARTD\_EXPRS} list can now be on a per-VM basis, and
entries on the list can also be specific to a VM. 
See ~\ref{sec:SMP-exprs} for more details.

\item The \Macro{LOCAL\_CONFIG\_FILE} can now be overridden. 
This now allows files to include other local config files. 
See ~\ref{param:LocalConfigFile} for more info.

\item Resources that are claimed but suspended can now optionally 
not be charged for at the accountant. 
When the resource is unsuspended, the accountant will resume charging
for usage. 
This is controlled by the \Expr{NEGOTIATOR\_DISCOUNT\_SUSPENDED\_RESOURCES}
config file entry, and it defaults to false.

\item The \Attr{DAGManJobID} attribute which \condor{dagman} inserts
into the classad of every job it submits now contains only its cluster
ID (instead of a cluster.proc ID pair), so that it may be referenced
as an integer in DAG job submit files.  This allows, for example, a
user to automatically set the relative local queue priority of jobs
based on the \condor{dagman} job that submitted them, so that jobs
submitted by ``older'' DAGs will start before jobs submitted by
``newer'' DAGs (assuming they are otherwise identical).

\item GSI authentication can now be used when Condor-C jobs are submitted
from one \condor{schedd} to another.

\item File permissions are now preserved when a job's data files are
transferred between unix machines. File transfers that involve a windows
machine or older version of Condor remain as before.

\item Condor-C now supports the scheduler remote universe.

\item \condor{advertise} now publishes a ``MyAddress'' if none is provided
in the source ClassAd.  This will prevent the collector from throwing out
ads with no address (see Bugs Fixed).

\item Added a new \Condor{dagman} parameter \MacroNI{DAGMAN\_ALLOW\_EVENTS}
controlling which ``bad'' events are not considered fatal errors;
the \Opt{-NoEventChecks} command-line argument is deprecated and has no effect.

\item \Condor{fetchlog} now takes an optional log file extension in order to
select logs such as ``StarterLog.vm2''.

\end{itemize}


\noindent Bugs Fixed:

\begin{itemize}

\item Fixed a throughput performance bottle neck when standard universe
        jobs vacate when the user has specified \Attr{WantCheckpoint} equal to
        False in the submit file.

\item Added initial support for the \Syscall{getdents},
        \Syscall{getdents64}, \Syscall{glob}, and the family of functions
        \Syscall{opendir}, \Syscall{readdir}, \Syscall{closedir} for the
        standard universe.  

        It is recommended that you do not directly invoke \Syscall{getdents} 
        or \Syscall{getdents64}, but instead use the other POSIX functions
        specified above.

        There are two caveats: these calls will not work in heterogeneous
        contexts, and you may not call \Syscall{getdents} directly when 
        \Condor{compile}ing a 32-bit program while specifying the 64-bit
        interfaces for the Unix API.

\item In versions 6.7.4 through 6.7.6, Computing On Demand (COD)
  support was broken due to a bug in how Condor daemons parsed their
  command line arguments.
  The bug was introduced with the changes to provide a web services
  (SOAP) interface to Condor.
  This bug has been fixed and COD support is now working again.

\item In version 6.7.6, the \MacroNI{DAGParentNodeNames} attribute
which \Condor{dagman} adds to all DAG job classads could grow too long
and cause job submission to fail.  Now, if the
\MacroNI{DAGParentNodeNames} value would be too long to add to the job
classad, the attribute is instead left undefined and a warning is
emitted in the DAGMan debugging log.  This behavior means that such a
node can be reliably distinguished from a node with no parents, as the
latter will have a \MacroNI{DAGParentNodeNames} attribute defined but
empty.

\item In version 6.7.3, the value of the X509UserProxySubject job attribute
was changed in such a way that Condor-G jobs submitted by a newer
\condor{submit} to an older \condor{schedd} could fail to run. Now,
\condor{submit} reverts to the old behavior when talking to an old
\condor{schedd}.

\item Bug-fixes and improvements to grid\_type gt4:

  \begin{itemize}

  \item Condor will now delegate a single proxy to the GT4 server for
  multiple. If the local proxy is refreshed, Condor will forward the
  refreshed copy to the server.

  \item Exit codes are now recorded properly.

  \item \Macro{JAVA\_EXTRA\_ARGUMENTS} now used when invoking the GT4 GAHP
  server (which is written in java).

  \item If \Macro{LOWPORT} and \Macro{HIGHPORT} are set in the config file,
  the GT4 GAHP server will now obey the port restriction.

  \item Fixed a bug that caused Condor not to notice when some GT4 jobs
  completed.

  \item Fixed a bug in handling the job's environment for GT4 jobs. Condor
  incorrectly used ``<name>=<value>'' for each variable's name.

  \item Improved hold reason in certain cases when a GT4 job goes on hold.

  \item \condor{q} -globus now works properly for GT4 jobs. Also, the resource
  name in the user log execute event is printed properly for GT4 jobs.

  \item Fixed a bug that could cause Condor to not detect when a GT4 job
  completes. This was triggered by Condor not properly recognizing the
  StageOut Globus job state.

  \end{itemize}

\item Fixed a bug that can cause the \condor{gridmanager} to abort if
\Attr{PeriodicRelease} evaluates to true while it's putting a job on hold.

% See Gnats PR 470
\item Fixed a bug in \Condor{dagman} that
caused the DAG to be aborted if a job generated an executable error
event.

\item Fixed a bug in \Condor{dagman} on Windows that would cause it to
hang or crash on exit.

\item MPI universe jobs now honor the \Attr{JOB\_START\_DELAY}
configuration setting.

\item The \Condor{collector} now throws out startd, schedd, and License
ClassAds that don't have a valid IP address (used in it's hashing).  The
collector now correctly will fall back to ``MyAddress'' if it's provided.

% See Gnats PR 479.
\item Fixed a bug in \Condor{dagman} that could cause \Condor{dagman}
to fail an assertion if PRE or POST scripts are throttled with the
\Opt{-maxpre} or \Opt{-maxpost} \Condor{submit\_dag} command line flags.

\end{itemize}


%%%%%%%%%%%%%%%%%%%%%%%%%%%%%%%%%%%%%%%%%%%%%%%%%%%%%%%%%%%%%%%%%%%%%%
\subsection*{\label{sec:New-6-7-6}Version 6.7.6}
%%%%%%%%%%%%%%%%%%%%%%%%%%%%%%%%%%%%%%%%%%%%%%%%%%%%%%%%%%%%%%%%%%%%%%

\noindent Release Notes:

\begin{itemize}

\item Version 6.7.6 contains all the bug fixes and improvements from
  the 6.6 stable series up to and including version 6.6.9.

\end{itemize}

\noindent New Features:

\begin{itemize}

\item Added support for libc's \Syscall(system) function for standard
        universe executables. This call is not checkpoint-safe in that
        the standard universe job could call it twice or more times
        in the event of a resumption from an earlier checkpoint. The
        invocation of this call by the shadow on behalf of the user
        job is controlled by a configuration file parameter called
        \Attr{SHADOW\_ALLOW\_UNSAFE\_REMOTE\_EXEC} and is off by default.
        The full environment of the user job is preserved during the
        invocation of \Syscall(system) and this might cause problems in 
        heterogeneous submission contexts of the user is not careful.

\item Added support for a web services (SOAP) interface to Condor.
  For more information, see and section~\ref{API-WebService} on
  page~\pageref{API-WebService}.

  \Note Due to a bug in gSOAP, the SOAP support in Condor 6.7.6 does
  not work with all SOAP toolkits.
  Some of the responses that gSOAP generates contain unqualified tags.
  Therefore, SOAP toolkits that are strict (such as gSOAP or .Net)
  will not accept these poorly formed responses.
  SOAP toolkits that are more lax in the responses they accept (such
  as Axis, SOAP::Lite, or ZSI) will work with version 6.7.6.
  This problem has already been fixed and the solution will be
  released in Condor version 6.7.7.

\item Added support for the GT4 grid\_type in Condor's grid universe.
  This new grid type supports jobs submitted to grid resources
  controlled by Globus Toolkit version 4 (GT4).

  New configuration settings are required to support jobs
  submitted for the GT4 grid type.
  These settings have been added to the default configuration files
  shipped with Condor, but sites that are upgrading an existing
  installation and choosing to keep their old configuration files must
  add these settings to allow GT4 jobs to work:
\begin{verbatim}
## The location of the wrapper for invoking GT4 GAHP server
GT4_GAHP = $(SBIN)/gt4_gahp
 
## The location of GT4 files. This should normally be lib/gt4
GT4_LOCATION = $(LIB)/gt4

## gt4-gahp requires gridftp server. This should be the address of gridftp
## server to use
GRIDFTP_URL_BASE = gsiftp://$(FULL_HOSTNAME)
\end{verbatim}

\item Condor version 6.7.6 includes the Stork data movement system, 
  the Condor Credential Daemon (\Condor{credd}), and support for using
  MyProxy for credential management.
  However, currently these are only supported in our release for Linux
  using the 2.4 kernel with glibc version 2.3 (RedHat 9, etc).
  All of these features require changes to the Condor configuration
  files to function properly.
  The default configuration files shipped with Condor already include
  all the new settings, but sites upgrading an existing installation
  must add these new settings to their Condor configuration.
  For a list of settings and more information, see
  section~\ref{sec:Stork-Config-File-Entries} on 
  page~\pageref{sec:Stork-Config-File-Entries} for Stork,
  section~\ref{sec:Credd-Config-File-Entries} on
  page~\pageref{sec:Credd-Config-File-Entries} for \Condor{credd},
  and section~\ref{sec:MyProxy-Config-File-Entries} on
  page~\pageref{sec:MyProxy-Config-File-Entries} for MyProxy.
  For more information about MyProxy, you can also see  
  \URL{http://grid.ncsa.uiuc.edu/myproxy}

\item Added preliminary support for the High Availability Daemon (HAD).

\item Added a new \MacroNI{SCHED\_UNIV\_RENICE\_INCREMENT}
configuration variable used by the \Condor{schedd} for scheduler
universe jobs, analogous to the existing
\MacroNI{JOB\_RENICE\_INCREMENT} variable used by the \Condor{startd}
for other job universes.  The \MacroNI{SCHED\_UNIV\_RENICE\_INCREMENT}
variable is undefined by default, and when undefined, defaults to 0
internally.

\item The relative priority of a user's own jobs in the local
\condor{schedd} queue is no longer limited to the range -20 to +20,
but can be any integer value.

\item DAGMan Improvements:

\begin{itemize}

  \item \Condor{dagman} now inserts a \MacroNI{DAGParentNodeNames}
  attribute into classad of all Condor jobs it submits, containing the
  names of the job's parents in the DAG.  The list is in the form of a
  comma-delimited string.

  \item Added the \Condor{dagman} arguments \Opt{-noeventchecks} and
  \Opt{-allowlogerror} to \Condor{submit\_dag}.

\end{itemize}

\item \Condor{glidein} Improvements:

\begin{itemize}

  \item Added \Condor{glidein} options for setting up GSI authentication.

  \item Added \Condor{glidein} option {-run\_here} for direct
  execution of Glidein, instead of submitting it for remote execution.
  You may also save a script for doing this and then run the script
  through whatever mechanism you want (like some batch system
  interface not supported by Condor-G).

\end{itemize}

\item Added support for the \Macro{NEGOTIATOR\_CYCLE\_DELAY}
  configuration setting, which is only intended for expert
  administrators.
  For more information, see section~\ref{param:NegotiatorCycleDelay}
  on page~\pageref{param:NegotiatorCycleDelay}.


\end{itemize}

\noindent Bugs Fixed:

\begin{itemize}

\item Previous versions of the \Condor{master} had a bug where if the
  administrator attempted to use \Macro{SUBSYS\_ARGS} to pass \Opt{-p}
  to any Condor daemon to have it listen on a specific, fixed port,
  the underlying daemon would not honor the flag.
  Now, the \Condor{master} correctly supports using
  \Macro{SUBSYS\_ARGS} to define a port using \Opt{-p}.
  For more information about \Macro{SUBSYS\_ARGS}, see
  section~\ref{param:SubsysArgs} on page~\pageref{param:SubsysArgs}.

\item Removed case-sensitivity of command-line argument names in
\Condor{submit\_dag}.

\item Fixed the {-r} (remote schedd) option in \Condor{submit\_dag}.

\item Condor versions 6.7.1 through 6.7.5 exhibit a bug in
  which the commands \Condor{off}, \Condor{restart}, and
  \Condor{vacate} did not handle the \Opt{-pool} command-line option
  correctly.
  The bug caused these commands to correctly query the central manager
  of the remote pool,
  and to incorrectly send the command to the central manager machine.
  This bug has now been fixed, and these tools no longer send
  the command to the central manager machine.

\end{itemize}

\noindent Known Bugs:

\begin{itemize}

\item None.

\end{itemize}


%%%%%%%%%%%%%%%%%%%%%%%%%%%%%%%%%%%%%%%%%%%%%%%%%%%%%%%%%%%%%%%%%%%%%%
\subsection*{\label{sec:New-6-7-5}Version 6.7.5}
%%%%%%%%%%%%%%%%%%%%%%%%%%%%%%%%%%%%%%%%%%%%%%%%%%%%%%%%%%%%%%%%%%%%%%
%  This was 6.7.4, but we took 6.7.4 away and replaced it quickly
%  with 6.7.5
\noindent Release Notes:

\begin{itemize}

\item None.

\end{itemize}


\noindent New Features:

\begin{itemize}

\item Added DAG aborting feature -- a DAG can be configured to
abort immediately if a node exits with a given exit value.

\item The dedicated scheduler can now preempt running MPI jobs from 
appropriately configured machines. See 
~\ref{sec:Configure-Dedicated-Preemption} for details.

\item The MPI universe now supports submit files with multiple procs (queue 
commands), each with distinct requirements.  This is useful for placing
the head node of an MPI job on a specific machine, and the rest of the 
nodes elsewhere. See ~\ref{sec:MPI} for details.

\item The \Condor{negotiator} now publishes its own ClassAd to the
  \Condor{collector} which includes the IP address and port where it
  is listening.
  This negotiator ClassAd can be viewed using the new
  \Opt{-negotiator} option with \Condor{status}.
  In addition to removing an unnecessary fixed port for the
  \Condor{negotiator}, this change corrects some problems with
  commands that attempted to communicate directly with the
  \Condor{negotiator}.
  These bugs were first listed in the Known Bugs section of the 6.6.0
  version history.

  To enable this feature and have the \Condor{negotiator} listen on a
  dynamic port, you must comment out the \Macro{NEGOTIATOR\_HOST}
  setting in your configuration file.
  The new example configuration files shipped with version 6.7.4 and
  later will already have this setting undefined.
  However, if you upgrade your binaries and retain an older copy of
  your configuration files, you should consider commenting out 
  \MacroNI{NEGOTIATOR\_HOST}.

  To disable this feature and have the \Condor{negotiator} still
  listen on a well-known port, you can uncomment the
  \MacroNI{NEGOTIATOR\_HOST} setting in the default configuration. 
  For example:
\begin{verbatim}
NEGOTIATOR_HOST = $(CONDOR_HOST)
\end{verbatim}

  Pools that are comprised of older versions of Condor and a 6.7.4 or
  later central manager machine should either continue to use their
  old \File{condor\_config} file (which will still have
  \MacroNI{NEGOTIATOR\_HOST} defined) or they should re-define the
  \MacroNI{NEGOTIATOR\_HOST} setting in the new example configuration
  files which are used during the installation process.

\item Added optional \Expr{DAGMAN\_RETRY\_SUBMIT\_FIRST} configuration
parameter that tells \Condor{dagman} whether to immediately retry
the submit if a node submit fails, or to put that job at the end of
the ready jobs queue.  The default is TRUE, which retries the failed
submit before trying to submit any other jobs.

\item The schedd now uses non-blocking connection attempts when contacting
startds.  This prevents the long (typically 40 second) hang of all schedd
operations when the connection attempt does not complete, due to
network problems.

\end{itemize}

\noindent Bugs Fixed:

\begin{itemize}

\item Fixed a performance problem with the standard universe when
\Syscall{gettimeofday} is called in a very tight loop by the application.

\item Fixed the default value of \Macro{OPSYS} in the MacOSX version
  of Condor.
  Once again, Condor reports \verb@OSX@ for all versions of MacOSX.
  This bug was introduced in version 6.7.3 of Condor.

\item Fixed a bug in \Condor{dagman} that caused it to be killed if
the \Expr{DAGMAN\_MAX\_SUBMIT\_ATTEMPTS} parameter was set to too
high a value.

\item Fixed a bug in \Condor{gridmanager} that caused it to crash if
the grid\_monitor was activated.

\item Fixed support for the getdents64() system call inside the
  standard universe on Linux and Solaris.

% Gnats PR 467
\item Fixed a bug in \Condor{dagman} that dealt
incorrectly with the problem of Condor sometimes writing both a
terminated and an aborted event for the same job. The spurious
aborted event is now ignored.

\end{itemize}

\noindent Known Bugs:

\begin{itemize}

\item None.

\end{itemize}


%%%%%%%%%%%%%%%%%%%%%%%%%%%%%%%%%%%%%%%%%%%%%%%%%%%%%%%%%%%%%%%%%%%%%%
\subsection*{\label{sec:New-6-7-3}Version 6.7.3}
%%%%%%%%%%%%%%%%%%%%%%%%%%%%%%%%%%%%%%%%%%%%%%%%%%%%%%%%%%%%%%%%%%%%%%

\noindent Release Notes:

\begin{itemize}

\item This release contains all the bug fixes from the 6.6 stable
  series up to and including version 6.6.7, and some of the fixes that
  will be included in version 6.6.8.
  The bug fixes in version 6.6.8 that were not included in version
  6.7.3 are listed in a separate section of the 6.6.8 version
  history. 

\end{itemize}


\noindent New Features:

\begin{itemize}

\item Added Full Ports of Condor to Redhat Fedora Core 1, 2 and 3 on
the 32-bit x86 architecture. 
Please read the Linux platform specific
section~\ref{sec:platform-linux-fed} in this manual for more information
on caveats with this port.

\item Added a feature to \Condor{dagman} that will allow VARS names to include
numerics and underscores.

\item Added optional \Expr{COLLECTOR\_HOST\_FOR\_NEGOTIATOR} configuration parameter to indicate which \Condor{collector} the  \Condor{negotiator} on this (local) host should query first. This is designed to improve negotiation performance.

\item Added a new \Condor{dagman} capability to allow the DAG to continue
if it encounters a double run of the same node job (set the
\Expr{DAGMAN\_IGNORE\_DUPLICATE\_JOB\_EXECUTION} parameter to true to do this).

\item Added Condor-C: the "condor" grid\_type.  Condor-C allows jobs to be handed from one \Condor{schedd} to another \Condor{schedd}.

\item Added \Opt{setup\_here} option to \Condor{glidein} for cases where
direct installation is desired instead of submitting a setup job to the
remote gatekeeper.  (For example, this is useful when doing an installation
onto AFS.)

\item If \Attr{RemoteOwner} is exported via \Expr{STARTER\_VM\_EXPRS} into the
ad of other virtual machines, the \Condor{negotiator} automatically inserts
\Attr{RemoteUserPrio} into the ad as well, so policy expressions can now take
into account the priority of jobs running on other virtual machines on the
same host.

\item Linux 2.6 kernels do not update the access time for console devices,
so Condor was unable to detect if there has been activity at the keyboard
or mouse. As a work-around, Condor now polls /proc/interrupts to detect
if the keyboard has requested attention. This does not work for USB keyboards
or pseudo TTYs, so \Attr{ConsoleIdle} on 2.6 kernels will be wrong for some
devices. Future versions of Condor or Linux may correct this.

\item \Condor{dagman} no longer removes the X509\_USER\_PROXY environment 
variable.
This should allow users to set the environment variable before invoking 
\Condor{submit\_dag} and have the jobs submitted by \Condor{dagman} correctly
find the proxy file.

\end{itemize}

\noindent Bugs Fixed:

\begin{itemize}

\item Fixed a \Condor{dagman} bug that could cause it to leave jobs running
when aborting a DAG.

\item Fixed a \Condor{dagman} bug which, if its debug level was set to
zero (silent), could cause it to to improperly recognize persistent
\Condor{submit} failures.

\item Fixed a bug in Condor's file transfer mechanism that showed up
  when users tried to use streaming output for either STDOUT or
  STDERR.
  There were situations where Condor would attempt to transfer back
  the STDOUT or STDERR file from the execution host, even though these
  files didn't exist and all the data was already streamed back to the
  submit host.
  Now, if either \Attr{stream\_output} or \Attr{stream\_error} are set
  to true in the job submit description file, Condor will transfer any
  other output but will not attempt to transfer back STDOUT or STDERR.

\item The Condor user log library (libcondorapi) now correctly handles
  execute events that lack a hostname.

\end{itemize}

\noindent Known Bugs:

\begin{itemize}

\item Unfortunately, the default \Macro{OPSYS} value for the MacOSX
  version of Condor was incorrectly changed in version 6.7.3.
  Condor used to always report \verb@OSX@, but in version 6.7.3 it
  will report either \verb@OSX10_2@, \verb@OSX10_3@, or
  \verb@OSX_UNK@.
  This is wrong, since Condor jobs submitted to any version of OSX
  should be able to run on any other version of OSX, and the above
  change needlessly partitions resources and complicates things for
  end-users.
  Therefore, anyone running version 6.7.3 on MacOSX is encouraged to
  add the following line to their global \File{condor\_config} file:
\begin{verbatim}
OPSYS = OSX
\end{verbatim}

  If your pool is already running the new release, you can cause the
  above change to take effect by running the following command on your
  pool's central manager machine (or any machine listed in the
  \MacroNI{HOSTALLOW\_ADMINISTRATOR} list) after you have changed the
  \MacroNI{OPSYS} value in your configuration:
\begin{verbatim}
condor_reconfig -all
\end{verbatim}

  However, if you have already submitted jobs to your pool with the
  old \MacroNI{OPSYS} value, the \Attr{Requirements} expression in
  those jobs will still refer to the incorrect value.
  In this case, you should either a) wait for the jobs to complete
  before making the above change, b) remove the jobs and resubmit
  them after you've made the change, or c) manually run \Condor{qedit}
  on the jobs to change their \Attr{Requirements} expressions.

\item When running in recovery mode on a DAG that has PRE scripts,
\Condor{dagman} may attempt more than the specified number of retries
of a node (counting retries attempted during the first run of the
DAG).  This is because if a node fails because of the PRE script
failing, that fact is not recorded in the log, so that retry is missed
in recovery mode.

\end{itemize}



%%%%%%%%%%%%%%%%%%%%%%%%%%%%%%%%%%%%%%%%%%%%%%%%%%%%%%%%%%%%%%%%%%%%%%
\subsection*{\label{sec:New-6-7-2}Version 6.7.2}
%%%%%%%%%%%%%%%%%%%%%%%%%%%%%%%%%%%%%%%%%%%%%%%%%%%%%%%%%%%%%%%%%%%%%%

\noindent Release Notes:

\begin{itemize}

\item Condor Version 6.7.2 includes some bug fixes from Version 6.6.7,
but none from Version 6.6.8.

\item MPI users who are upgrading from previous versions of Condor
to version 6.7.2 will need to modify the 
\Macro{MPI\_CONDOR\_RSH\_PATH} configuration macro of their dedicated
resource to be \MacroU{LIBEXEC} instead of \MacroUNI{SBIN}.
Users who are installing Condor version 6.7.2
for the first time will not need to make any changes.

\end{itemize}


\noindent New Features:

\begin{itemize}

\item Added an \Macro{INCLUDE} configuration file variable
   to define the location of header files shipped with Condor
   that are currently needed to be included when compiling
   Condor APIs.
   When \MacroNI{INCLUDE} is defined,
   \Condor{config\_val} can be used to list header files.


\item A Condor pool can now support multiple Collectors. This should
  improve stability due to automatic failover. All daemons will now
  send updates to ALL of the specified collectors. All daemons/tools
  will query the Collectors in sequence, until an appropriate 
  response is received. Thus if one (or more) of the Collectors are 
  down, the pool will continue to function normally, as long as 
  there is at least one functioning Collector. 
  You can specify multiple (comma-separated) collector host (and port) 
  addresses in the \Expr{COLLECTOR\_HOST} entry in the configuration
  file. A given \Condor{master} can only run one Collector.

\item When the \Condor{master} is started with the \Opt{-r} option to
  indicate that it should quite after a period of time, the
  \Condor{startd} will now indicate how much time is remaining before it
  exits. It does this by advertising TimeToLive in the machine
  ClassAd.

\item Added new macro \Macro{JOB\_START\_COUNT} that works in
conjunction with existing macro \Macro{JOB\_START\_DELAY} to 
throttle job starts.
Together, this macro pair provides greater flexibility
tuning job start rate given available \Condor{schedd} performance.

\item Added a \MacroNI{LIBEXEC} directory to the install process.
Support commands that
the Condor system needs will be added to this directory in future releases.
This directory should not be added to a user or system-wide path.  

\item Added the ability to decide for each file that condor transfers whether
it should be encrypted or not, using encrypt\_input\_files, 
dont\_encrypt\_input\_files, encrypt output files, and
        dont\_encrypt\_output\_files in the job's submit file.

\item Added DISABLE\_AUTHENTICATION\_IP\_CHECK which will work around problems
on dual-homed machines where the IP address is reported incorrectly to condor.
This is particularly a problem when using Kerberos on multi-homed machines.

\end{itemize}

\noindent Bugs Fixed:

\begin{itemize}

\item Fixed a bug on Linux systems caused by both 
      Condor and the Linux distribution having a library file 
      called \File{libc.a}.
      The problem caused the link step to fail on Condor API
      programs.
      The evaluation order to determine the location of library
      files caused use of the wrong file, given the duplicate naming.
      The bug is fixed by renaming the Condor library files.

\item When the \Condor{startd} is evaluating the state of each virtual
  machine (VM), it now refreshes any ClassAd attributes which are
  shared from other virtual machines (using \Expr{STARTD\_VM\_EXPRS})
  before it tries to evaluate.
  This way, if a given VM changes its state, all other VMs will
  immediately see this state change.

\item Fixed a bug where you couldn't transfer input files larger than 2 gigabytes.

\item Condor can now detect the size of memory on a Linux machine with the 2.6
kernel.

\item JAR files specified in the submit file were not being transfered
along with the job unless they were also explicitly placed in the list
of input files to transfer. Now, the JAR files are implicitly added to the
list of input files to transfer.

\end{itemize}

\noindent Known Bugs:

\begin{itemize}

\item None.

\end{itemize}




%%%%%%%%%%%%%%%%%%%%%%%%%%%%%%%%%%%%%%%%%%%%%%%%%%%%%%%%%%%%%%%%%%%%%%
\subsection*{\label{sec:New-6-7-1}Version 6.7.1}
%%%%%%%%%%%%%%%%%%%%%%%%%%%%%%%%%%%%%%%%%%%%%%%%%%%%%%%%%%%%%%%%%%%%%%

\noindent Release Notes:

\begin{itemize}

\item Version 6.7.1 contains all of the features, ports, and bug fixes
  from the previous stable series, up to and including version 6.6.6.
  There are a few additional bugs that have been fixed in the 6.6.x
  stable series which have not yet been released, but which will
  appear in version 6.6.7.
  These bug fixes have been included in version 6.7.1, and appear in
  the ``Bugs fixes included from version 6.6.7'' list below.
  In addition, a number of new features and some bug fixes have been
  made, which are described below in more detail.

\item None.

\end{itemize}


\noindent New Features:

\begin{itemize}

\item Added an option to DAGMan's retry ability. If a DAG specifies
  something like ``RETRY job 10 unless-exit 9'', then the retries will
  only happen if the node doesn't exit with a value of 9. 

\item Condor-G can now submit jobs to Globus 3.2 (WS) (for jobs with 
  \Expr{universe = grid}, \Expr{grid\_type = gt3}). Submitting to Globus 
  3.0 (as in Condor 6.7.0) is no longer supported. Submitting to pre-WS 
  Globus (2.x) is still supported (\Expr{grid\_type = gt2}).

\item Added new startd policy expression MaxJobRetirementTime.  This
specifies the maximum amount of time (in seconds) that the startd
is willing to wait for a job to finish on its own when the startd
needs to preempt the job (for owner preemption, negotiator preemption,
or graceful startd shutdown).

\item Added -peaceful shutdown/restart mode.  This will shut down the
startd without killing any jobs, effectively treating both
\Expr{MaxJobRetirementTime} and \Expr{GRACEFUL\_SHUTDOWN\_TIMEOUT} as
infinite.  The default shutdown/restart mode is still -graceful, which
behaves according to whatever \Expr{MaxJobRetirementTime} and
\Expr{GRACEFUL\_SHUTDOWN\_TIMEOUT} are.  The behavior of -fast mode
is unchanged; it kills jobs immediately, regardless of the other
timeout settings.

\item Jobs can now be submitted as ``noop'' jobs. Jobs submitted with
  \Expr{noop\_job = true} will not be executed by Condor, and instead will
   immediately have a terminate event written to the job log file and 
   removed from the queue. This is useful for DAGs where the pre-script
   determines the job should not run.

\item Added preliminary support for the Tool Daemon Protocol (TDP)
  into Condor.
  This protocol is still under development, but the goal is to provide
  a generic way for scheduling systems (daemons) to interact with
  monitoring tools.
  Assuming this protocol is adopted by other scheduling systems and by
  various monitoring tools, it would allow arbitrary combinations of
  tools and schedulers to co-exist, function properly, and provide
  monitoring services for jobs running under the schedulers.
  This initial support allows users to specify a ``tool'' that should
  be spawned along-side their regular Condor job.
  On Linux, the ability to have the batch Condor job suspend
  immediately upon start-up is also implemented, which allows a
  monitoring tool to attach with ptrace() before the job's main()
  function is called.

\end{itemize}

\noindent Bugs Fixed:

\begin{itemize}

\item Fixed a significant memory leak in the \Condor{schedd} that was
  introduced in version 6.7.0.
  In 6.7.0, the \Condor{schedd} would leak a copy of ClassAd for every
  job it tried to spawn (on average, around 2000 bytes per job).

\item Fixed the bugs in Condor's MPI support that were introduced in
  version 6.7.0.
  Condor now supports MPI jobs linked with MPICH 1.2.4 and older.
  Improved Condor's log messages and email notifications when MPI jobs
  run on multiple virtual machines (the messages now include the
  appropriate ``vmX'' identifier, not just the hostname).
  Unfortunately, due to changes in MPICH between version 1.2.4 and
  1.2.5, Condor's MPI support is not compatible with MPICH 1.2.5.
  We will be addressing this problem in a future release.

\end{itemize}

\noindent Bugs fixes included from version 6.6.7:

\begin{itemize}

\item Fixed an important bug in the low-level code that Condor uses to
  transfer files across a network.
  There were certain temporary failure cases that were being treated
  as permanent, fatal errors.
  This resulted in file transfers that aborted prematurely, causing
  jobs to needlessly re-run.
  The code now gracefully recovers from these temporary errors.
  This should significantly help throughput for some sites,
  particularly ones that transfer very large files as output from
  their jobs.

\item Fixed a number of bugs in the \Opt{-format} option to \Condor{q}
  and \Condor{status}.
  Now, these tools will properly handle printing boolean expressions
  in all cases.
  Previously, depending on how the boolean evaluated, either the
  expression was printed, or the tool could crash.
  Furthermore, the tools do a better job of handling the different 
  types of format conversion strings and printing out the appropriate
  value.
  For example, if a user tries to print out a boolean attribute with
  \verb@condor_status -format "%d\n" HasFileTransfer@, the
  \Condor{status} tool will evaluate \Attr{HasFiletransfer} and print
  either a 0 or a 1 (FALSE or TRUE).
  If, on the other hand, a user tries to print out a boolean attribute
  with \verb@condor_status -format "%s\n" HasFileTransfer@, the
  \Condor{status} tool will print out the string ``FALSE'' or ``TRUE''
  as appropriate.

\item The ClassAd attribute scope resolution prefixes, \texttt{MY.}
  and \texttt{TARGET.}, are no longer case sensitive.

\item \Condor{dagman} now does better checking for inconsistent events
(such as getting multiple terminate events for a single job).  This
checking can be disabled with the \Opt{-NoEventChecks} command-line
option.

\end{itemize}

\noindent Known Bugs:

\begin{itemize}

\item None.

\end{itemize}




%%%%%%%%%%%%%%%%%%%%%%%%%%%%%%%%%%%%%%%%%%%%%%%%%%%%%%%%%%%%%%%%%%%%%%
\subsection*{\label{sec:New-6-7-0}Version 6.7.0}
%%%%%%%%%%%%%%%%%%%%%%%%%%%%%%%%%%%%%%%%%%%%%%%%%%%%%%%%%%%%%%%%%%%%%%

\noindent Release Notes:

\begin{itemize}

\item Version 6.7.0 contains all of the features, ports, and bug fixes
  from the previous stable series, up to and including version 6.6.4.
  In addition, a number of new features and some bug fixes have been
  made, which are described below in more detail.

\end{itemize}


\noindent New Features:

\begin{itemize}

\item Added support for vanilla and Java jobs to reconnect when the
  connection between the submitting and execution nodes is lost for
  any reason.
  Possible reasons for this disconnect include: network outages,
  rebooting the submit machine, restarting the Condor daemons on the
  submit machine, etc.
  If the execution machine is rebooted or the Condor daemons are
  restarted, reconnection is not possible.
  To take advantage of this reconnect feature, jobs must be submitted
  with a \Attr{JobLeaseDuration}.
  There are new events in the UserLog related to disconnect and
  reconnect.

\item Added a new Condor tool, \Condor{vacate\_job}.
  This command is similar to \Condor{vacate}, except the kinds of
  arguments it takes define jobs in a job queue, not machines to
  vacate.
  For example, a user can vacate a specific job id, all the jobs in a
  given cluster, all the jobs matching a job queue constraint, or even
  all jobs owned by that user.
  The owner of a job can always vacate their own jobs, regardless of
  the pool security policy controlling \Condor{vacate} (which is an
  administrative command which acts directly on machines).
  See the new command reference, section~\ref{man-condor-vacate-job}
  on page~\pageref{man-condor-vacate-job} for details.
  
\item Added a new ``High Availability'' service to the \Condor{master}.
   You can now specify a daemon which can have ``fail over'' capabilities
   (i.e. the master on another machine can start a matching daemon if the
   first one fails).  Currently, this is only available over a shared
   file system (i.e. NFS), and has only been tested for the \Condor{schedd}.

\item Scheduler universe jobs on UNIX can now specify a
  \Attr{HoldKillSig}, the signal that should be sent when the job is
  put on hold.
  If not specified, the default is to use the \Attr{KillSig}, and if
  that is not defined, the job will be sent a SIGTERM.
  The submit file keyword to use for defining this signal is
  \AdAttr{hold\_kill\_sig}, for example,
  \verb@hold_kill_sig = SIGUSR1@.

\item The \Condor{startd} can now support policies on SMP machines
  where each virtual machine (VM) has knowledge of the other VMs on
  the same host.
  For example, if a job starts running on one of the VMs, a job
  running on another VM could immediately be suspended.
  This is accomplished by using the new configuration variable
  \Macro{STARTD\_VM\_EXPRS}, which is a list of ClassAd attribute
  names that should be shared across all VMs on the machine.
  For each VM on the machine, every attribute in this list is looked
  up in the VM-specific machine ClassAd, the attribute name is given a
  prefix indicating what VM it came from, and then inserted into the
  machine ClassAds of all the other VMs.

\item The \Condor{startd} publishes four new attributes into the
  machine ClassAds it generates when it is in the Claimed state:
  \Attr{TotalJobRunTime}, \Attr{TotalJobSuspendTime},
  \Attr{TotalClaimRunTime}, \Attr{TotalClaimSuspendTime}.
  These attributes keep track of the total time the resource was
  either running a job (in the Busy activity) or had a job suspended,
  regardless of how many suspend/resume cycles the job went through.
  The first two attributes (with ``Job'' in the name) keep track for a
  single job (i.e. since the last time the resource was
  Claimed/Idle). 
  The last two attributes (with ``Claim'' in the name) keep track of
  these totals across all jobs that ran under the same claim
  (i.e. since the last state change into the Claimed state).

\item Added a \Opt{-num} option to the \Condor{wait} tool to wait for
   a specified number of jobs to finish.

\item Added a configuration option \Macro{STARTER\_JOB\_ENVIRONMENT}
   so the admin can configure the default environment inherited by
   user jobs.

\item Added a (configurable, defaults to off) feature to the \Condor{schedd}
   to allow backup the spool file before doing anything else.

\item The "Continuous" option of the \Condor{startd} ``cron'' jobs is
being deprecated.   It's being replaced by two new options which
control separate aspects of it's behavior:
\begin{itemize}
\item "WaitForExit" specifies the "exit timing" mode
\item "ReConfig" specifies that the job can handle SIGHUPs, and it should 
be sent a SIGHUP when the \Condor{startd} is reconfigured.
\end{itemize}

\item A lot of the items logged by the \Condor{startd} ``cron'' logic,
changed to D\_FULLDEBUG (from D\_ALWAYS), etc.

\item Added \Macro{NEGOTIATOR\_PRE\_JOB\_RANK} and
\Macro{NEGOTIATOR\_POST\_JOB\_RANK}.  These expressions are applied
respectively before and after the user-supplied job rank when deciding
which of the possible matches to choose.  (The existing expression
\Macro{PREEMPTION\_RANK} is applied after
\Macro{NEGOTIATOR\_POST\_JOB\_RANK}.)  The pool administrator may use
these expressions to steer jobs in ways that improve the overall
performance of the pool.  For example, using the pre job rank,
preemption may be avoided as long as there are idle machines, even
when the user-supplied rank expression prefers a machine that happens
to be busy.  Using the post job rank, one could steer jobs towards
machines that are known to be dedicated to batch jobs, or one could
enforce breadth-first instead of depth-first filling of a cluster of
multi-processor machines.

\item Added the ability for Condor to transfer files larger than 2G on
platforms that support large files.  This works automatically for
transferred executables, input files and output files.

\item Added the ability for jobs to stream back standard input, output, and
error files while running.  This is activated by the \Opt{stream\_input},
\Opt{stream\_output}, and \Opt{stream\_error} options to \Condor{submit}.
Note that this feature is incompatible with the new feature described
above where the shadow and starter can reconnect in certain
circumstances. 

\item Added support for vanilla jobs to be mirrored on a second
  \Condor{schedd}. The jobs are submitted to the second \Condor{schedd}
  on hold and will be released if the second \Condor{schedd} hasn't
  heard from the first \Condor{schedd} (actually, a \Condor{gridmanager}
  running under the first \Condor{schedd}) for a configurable amount of
  time. Once the second \Condor{schedd} releases the jobs, the first
  \Condor{schedd} acts as a mirror, reflecting the state of the jobs on
  the second \Condor{schedd}.
  To use this mirroring feature, jobs must be submitted
  with a \Attr{mirror\_schedd} parameter in the submit file and require
  no file transfer.

\end{itemize}


\noindent Bugs Fixed:

\begin{itemize}

\item Fixed a bug in the \Condor{startd} ``cron'' logic which caused the
\Condor{startd} to except when trying to delete a job that could never
be run (i.e. invalid executable, etc).

\item Fixed a bug in \Condor{startd} ``cron'' logic which caused it to
not detect when the starting of a ``job'' failed.

\item Fixed several bugs in the reconfiguration handling of the
\Condor{startd} ``cron'' logic.  In particular, even if the job has
the "reconfig" option set (or "continuous"), the job(s) won't be sent
a SIGHUP when the startd first starts, or when the job itself is first
run (until it outputs its first output block, defined by the "-"
separator).

\end{itemize}


\noindent Known Bugs:

\begin{itemize}

\item Condor's MPI support (for MPICH 1.2.4) was broken by other
  changes in version 6.7.0.
  Support for MPI jobs will return in Condor version 6.7.1.

\end{itemize}
\begin{center}
\begin{table}[hbt]
\begin{tabular}{|ll|} \hline
\emph{Architecture} & \emph{Operating System} \\ \hline \hline
Hewlett Packard PA-RISC (both PA7000 and PA8000 series) & HPUX 10.20 \\ \hline
Sun SPARC Sun4m,Sun4c, Sun UltraSPARC & Solaris 2.6, 2.7, 8, 9 \\ \hline
Silicon Graphics MIPS (R5000, R8000, R10000) & IRIX 6.5 (clipped) \\ \hline
Intel x86 & Red Hat Linux 7.1, 7.2, 7.3, 8.0 \\
 & Red Hat Linux 9 \\
 & Windows 2000 Professional and Server, 2003 Server (clipped) \\
 & Windows XP Professional (clipped) \\ \hline
ALPHA & Digital Unix 4.0 \\
 & Red Hat Linux 7.1, 7.2, 7.3 (clipped) \\
 & Tru64 5.1 (clipped) \\ \hline
PowerPC & Macintosh OS X (clipped) \\
 & AIX 5.2L (clipped) \\ \hline
Itanium & Red Hat Linux 7.1, 7.2, 7.3 (clipped) \\
 & SuSE Linux Enterprise 8.1 (clipped) \\ \hline
\end{tabular}
\caption{\label{supported-platforms}Condor 6.7.0 supported platforms}
\end{table}
\end{center}

%This is a stable release series of Condor.
It is based on the 6.5 development series.
All new features added or bugs fixed in the 6.5 series are available
in the 6.6 series.
The details of each version are described below.

%%%%%%%%%%%%%%%%%%%%%%%%%%%%%%%%%%%%%%%%%%%%%%%%%%%%%%%%%%%%%%%%%%%%%
\subsection{\label{sec:New-6-6-0}Version 6.6.0}
%%%%%%%%%%%%%%%%%%%%%%%%%%%%%%%%%%%%%%%%%%%%%%%%%%%%%%%%%%%%%%%%%%%%%%

\noindent New Features:

\begin{itemize}

\item The \Condor{dagman} debugging log now reports the total number
      of ``Un-Ready'' Nodes (i.e. those waiting for unfinished
      dependencies) in its periodic summaries.  In the past, the
      omission of this state led to confusion because the total of all
      reported job states didn't always match the total number of jobs
      in the DAG.

\item Most Condor commands (\Condor{on}, \Condor{off},
  \Condor{restart}, \Condor{reconfig}, \Condor{vacate},
  \Condor{checkpoint}, \Condor{reschedule}) now support a \Opt{-all}
  command-line option to specify which daemons to act on.
  This is more efficient and much easier to use than previous methods
  for accomplishing the same effect.
  Using \Opt{-all} with \Condor{off} correctly leaves the existing
  \Condor{master} processes running on each host, so that a subsequent
  \Condor{on} would work.
  See section~\ref{sec:Pool-Shutdown-and-Restart} on
  page~\pageref{sec:Pool-Shutdown-and-Restart} for more details on
  proper use of \Opt{-all} with \Condor{off} and \Condor{on}

\end{itemize}

\noindent Bugs Fixed:

\begin{itemize}

\item Fixed a bug under Solaris 8 with Update 6+, and Solaris 9 where
Condor would incorrectly report the console and mouse idle times as zero.

\item The standard-universe fetch\_files feature was not cleaning up
temporary files on the execution machine.

\item In rare circumstances, a Linux kernel bug results in conflicting
information about system boot time (\File{/proc/stat} and
\File{/proc/uptime}). 
Specifically, the "btime" field in \File{/proc/stat} suddenly jumps to
the present moment and then stays at that value.  This
was resulting in incorrect estimation of process ages, which caused
Condor's estimation of CondorLoadAvg to be completely wrong.  A more
robust heuristic is now being used.

\item A long configuration line with with continuation lines can cause the
config file parser to not properly skip the leading whitespace from
the continued lines.  This has been corrected.

\item The Grid Monitor now will automatically probe for and work with
``unknown'' batch systems.

\item Fixed a bug where under certain circumstances \Condor{dagman}
      would fail to detect an unsuccessful invocation of
      \Condor{submit}, and would instead report the job as
      successfully submitted with job id 0.0.

\item Fixed a bug which was causing problems when a periodic\_remove
expression for a scheduler universe job evaluates to true.  Under
these conditions, the schedd did not log the job terminiation to the
job log.  Addtionally, the schedd would exit with an error status.

\item Fixed a recently-introduced \Condor{dagman} bug where the number
      of node retries (specified with the RETRY keyword) wasn't being
      updated after some failures; instead, the node would be allowed
      to retry indefinitely if it kept failing.

\item Fixed a recently-introduced bug where shutting down the
      \Condor{schedd} caused \Condor{dagman} to remove all its jobs
      from the queue and write a rescue file, rather than simply
      exiting so that it could recover automatically upon restart.

\item Changed the default ``Periodic Expression Interval'' parameter
(PERIODIC\_EXPR\_INTERVAL) from 60 seconds to 300 seconds.

\item Whenever \Condor{reconfig} was used to re-configure multiple
  daemons which included the \Condor{collector} for a pool, the
  command would start to fail after the \Condor{collector} was
  reconfigured due to problems with security sessions in Condor's
  strong authentication code.
  This situation no longer causes problems for the \Condor{reconfig}
  tool, and it can properly re-configure multiple daemons at once,
  even if one of them is the \Condor{collector} for a pool.

\item Most Condor commands (\Condor{on}, \Condor{off},
  \Condor{restart}, \Condor{reconfig}, \Condor{vacate},
  \Condor{checkpoint}, \Condor{reschedule}) now check to make sure
  they are not sending a duplicate command if the user specifies the
  same target machine or daemon twice.  For example:
\begin{verbatim}
     condor_reconfig hostname1 hostname2 hostname1
\end{verbatim}
  will only send a single reconfig command to \verb@hostname1@.

\end{itemize}

\noindent Known Bugs:

\begin{itemize}

\item None.

\end{itemize}



% Feb 2007 -- still in the manual source, just not incorporating
% these old histories into the finished product, thereby
% reducing the size of the manual by 200 pages
%%%%%%%%%%%%%%%%%%%%%%%%%%%%%%%%%%%%%%%%%%%%%%%%%%%%%%%%%%%%%%%%%%%%%%%
\section{\label{sec:History-6-5}Stable Release Series 6.5}
%%%%%%%%%%%%%%%%%%%%%%%%%%%%%%%%%%%%%%%%%%%%%%%%%%%%%%%%%%%%%%%%%%%%%%

This is the development release series of Condor,
The details of each version are described below.


%%%%%%%%%%%%%%%%%%%%%%%%%%%%%%%%%%%%%%%%%%%%%%%%%%%%%%%%%%%%%%%%%%%%%%
\subsection{\label{sec:New-6-5-0}Version 6.5.0}
%%%%%%%%%%%%%%%%%%%%%%%%%%%%%%%%%%%%%%%%%%%%%%%%%%%%%%%%%%%%%%%%%%%%%%
\noindent New Features:
\begin{itemize}

\item A new log\_xml option has been added to condor\_submit. It is
documented in the condor\_submit portion of the manual.

\item A new DAGMan option to produce dot files was added. Dot is a
program that creates visualizations of DAGs. This feature is
documented in Section~\ref{sec:DAGMan}.

\item The email report from condor\_preen is now less cryptic, and
more self-explanatory.

\item Specifying full device paths (e.g., ``/dev/mouse'') instead of bare
device names (e.g., ``mouse'') in CONSOLE\_DEVICES in the config file is no
longer an error.

\item The condor\_submit tool now prints a more helpful, specific error if
the specified job executable is not found, or can't be accessed.

\item The startd ``cron'' (Hawkeye) now permits zero length ``prefix''
strings.

\item A number of new Hawkeye modules have been added, and most have
various bug fixes and improvements.

\end{itemize}

\noindent Bugs Fixed:
\begin{itemize}

\item 

\item 

\end{itemize}

\noindent Known Bugs:
\begin{itemize}
\item 

\item 

\end{itemize}

%%%%%%%%%%%%%%%%%%%%%%%%%%%%%%%%%%%%%%%%%%%%%%%%%%%%%%%%%%%%%%%%%%%%%%%
\section{\label{sec:History-6-4}Stable Release Series 6.4}
%%%%%%%%%%%%%%%%%%%%%%%%%%%%%%%%%%%%%%%%%%%%%%%%%%%%%%%%%%%%%%%%%%%%%%

This is the stable release series of Condor.
New features will be added and tested in the 6.5 development series. 
The details of each version are described below.

%%%%%%%%%%%%%%%%%%%%%%%%%%%%%%%%%%%%%%%%%%%%%%%%%%%%%%%%%%%%%%%%%%%%%%
\subsection{\label{sec:New-6-4-6}Version 6.4.6}
%%%%%%%%%%%%%%%%%%%%%%%%%%%%%%%%%%%%%%%%%%%%%%%%%%%%%%%%%%%%%%%%%%%%%%

\noindent Bugs Fixed:
\begin{itemize}

\item When more than 512 distinct users submit to Condor or Condor-G,
the \Condor{schedd} no longer crashes. 

\end{itemize}

%%%%%%%%%%%%%%%%%%%%%%%%%%%%%%%%%%%%%%%%%%%%%%%%%%%%%%%%%%%%%%%%%%%%%%
\subsection{\label{sec:New-6-4-3}Version 6.4.3}
%%%%%%%%%%%%%%%%%%%%%%%%%%%%%%%%%%%%%%%%%%%%%%%%%%%%%%%%%%%%%%%%%%%%%%

\noindent New Features:
\begin{itemize}

\item Added a \Opt{-hold} and \Opt{-held} option to \Condor{q} which 
displays the reason that the job had been held.

\end{itemize}

\noindent Bugs Fixed:
\begin{itemize}

\item Fixed a bug where more than one space between arguments to a job
in the java universe would result in it being invoked with and incorrect
list arguments.

\item Removed renaming of the executable to \Prog{condor\_exec} in the java
universe. This fixes a bug where the JVM was looking at its path to determine
its installation directory.

\item Fixed a bug and resulting null pointer exception in the java universe
because under certain conditions, Condor would invoke the JVM incorrectly.

\item Fixed serveral error reporting messages to be more precise.

\item When the NIS environment was being used, the \Condor{starter} daemon
would produce heavy amounts of NIS traffic. This has been fixed.

\item Binary characters in the \File{StarterLog} file and a possible
segmentation fault have been fixed.

\item Fixed \Cmd{select}{2} in the standard universe on our Linux ports.

\item Fixed a small bug in \Condor{q} that was displaying the wrong
username for ``niceuser'' jobs.

\item Fixed a bug where, in the standard universe, you could not open a file
whose name had spaces in it.

\item Fixed a bug in DAGMan where pre and post scripts would fail to
run if the DAG description file had extra whitespace.
Also, reworded the error messages DAGMan produces when it fails to
parse the DAG description file to be more clear and helpful for
solving the problem.

\item Fixed some misleading error messages in the Condor log files
when there were permission problems trying to execute a program. 

\item Condor for Windows will now run on Windows XP.

\item Condor for Windows now supports the Java Universe.

\item Users logged into Windows Domain accounts rather than local accounts
can submit jobs.

\item Potential Windows registry bloating bug fixed. Condor for Windows no
longer creates and deletes an account on the execute machine each time a
job is run. Instead, a single account for each VM on the execute machine is
created once and enabled or disabled as needed.

\item Cross-submits from Windows to Unix and from Unix to Windows are now
supported, provided that both platforms are running Condor 6.4 series daemons.

\item Free disk space is now reported accurately on Windows.

\item A rare but serious bug that could allow non-Condor processes to be added
to the Condor process family on Windows has been fixed.

\item Condor for Windows will now also run 16-bit applications.

\item Fixed a minor bug where certain integer attributes in the
\File{condor\_config} file might not have been properly parsed if they
were defined in terms of other config file attributes, using the
\MacroUNI{attribute} notation.  

\end{itemize}

\noindent Known Bugs:
\begin{itemize}

\item You may not open a file in the standard universe whose name contains a
colon ``:''.

\end{itemize}

%%%%%%%%%%%%%%%%%%%%%%%%%%%%%%%%%%%%%%%%%%%%%%%%%%%%%%%%%%%%%%%%%%%%%%
\subsection{\label{sec:New-6-4-2}Version 6.4.2}
%%%%%%%%%%%%%%%%%%%%%%%%%%%%%%%%%%%%%%%%%%%%%%%%%%%%%%%%%%%%%%%%%%%%%%
\noindent New Features:
\begin{itemize}

\item. This release mirrored the Condor-G release, and has no new features.

\end{itemize}

\noindent Bugs Fixed:
\begin{itemize}
\item None.

\end{itemize}
\noindent Known Bugs:
\begin{itemize}

\item None.

\end{itemize}

%%%%%%%%%%%%%%%%%%%%%%%%%%%%%%%%%%%%%%%%%%%%%%%%%%%%%%%%%%%%%%%%%%%%%%
\subsection{\label{sec:New-6-4-1}Version 6.4.1}
%%%%%%%%%%%%%%%%%%%%%%%%%%%%%%%%%%%%%%%%%%%%%%%%%%%%%%%%%%%%%%%%%%%%%%
\noindent New Features:
\begin{itemize}

\item None.

\end{itemize}

\noindent Bugs Fixed:
\begin{itemize}

\item Users are now allowed to answer ``none'' when prompted by the
installer to provide a Java JVM path. This avoids an endless loop and
leaves the Java abilities of Condor unconfigured.

\end{itemize}

\noindent Known Bugs:
\begin{itemize}

\item None.

\end{itemize}

%%%%%%%%%%%%%%%%%%%%%%%%%%%%%%%%%%%%%%%%%%%%%%%%%%%%%%%%%%%%%%%%%%%%%%
\subsection{\label{sec:New-6-4-0}Version 6.4.0}
%%%%%%%%%%%%%%%%%%%%%%%%%%%%%%%%%%%%%%%%%%%%%%%%%%%%%%%%%%%%%%%%%%%%%%

\noindent New Features:

\begin{itemize}

\item If a job universe is not specified in a submit description file, 
\Condor{submit}  will check the config file for \Macro{DEFAULT\_UNIVERSE}
instead of always choosing the standard universe. 

\item The \Macro{D\_SECONDS} debug flag is deprecated. Seconds are now always
included in logfiles. 

\item For each daemon listed in \Macro{DAEMON\_LIST}, you can now control the
environment variables of the daemon with a config file setting of the form
\Macro{DAEMONNAME\_ENVIRONMENT}, where \MacroNI{DAEMONNAME} is the name of a
daemon listed in \Macro{DAEMON\_LIST}. For more information, see
section~\ref{sec:Master-Config-File-Entries}.

\end{itemize}

\noindent Bugs Fixed:

\begin{itemize}

\item Fixed a bug in the new starter where if the submit file set no
arguments, the job would receive one argument of zero length.

\end{itemize}

\noindent Known Bugs:

\begin{itemize}

\item None.

\end{itemize}



%%%%%%%%%%%%%%%%%%%%%%%%%%%%%%%%%%%%%%%%%%%%%%%%%%%%%%%%%%%%%%%%%%%%%%%
\section{\label{sec:History-6-3}Development Release Series 6.3}
%%%%%%%%%%%%%%%%%%%%%%%%%%%%%%%%%%%%%%%%%%%%%%%%%%%%%%%%%%%%%%%%%%%%%%

This is the second development release series of Condor.

It contains numerous enhancements over the 6.2 stable series.
For example:

\begin{itemize}

\item Support for Kerberos and X.509 authentication.

\item Support for transfering files needed by jobs (for all universes
except standard and PVM)

\item Support for MPICH jobs.

\item Support for JAVA jobs.

\item 
Condor DAGMan is dramatically more reliable and efficient, and offers
a number of new features.

\end{itemize}

The 6.3 series has many other improvements over the 6.2 series, and
may be available on newer platforms.  The new features, bugs fixed,
and known bugs of each version are described below in detail.


%%%%%%%%%%%%%%%%%%%%%%%%%%%%%%%%%%%%%%%%%%%%%%%%%%%%%%%%%%%%%%%%%%%%%%
\subsection{\label{sec:New-6-3-3}Version 6.3.3}
%%%%%%%%%%%%%%%%%%%%%%%%%%%%%%%%%%%%%%%%%%%%%%%%%%%%%%%%%%%%%%%%%%%%%%

\noindent New Features:

\begin{itemize}

\item Added support for Kerberos and X.509 authentication in Condor.  

\item Added the ability for vanilla jobs on Unix to use Condor's file
transfer mechanism so that you don't have to rely on a shared file
system.  

\item Added support for MPICH jobs on Windows NT and 2000.

\item Added support for the JAVA universe.

\item When you use \Condor{hold} and \Condor{release}, you now see an
entry about the event in the UserLog file for the job.

\item Whenever a job is removed, put on hold, or released (either by a
Condor user or by the Condor system itself), there is a ``reason''
attribute placed in the job ad and written to the UserLog file.  
If a job is held, \Attr{HoldReason} will be set.
If a job is released, \Attr{ReleaseReason} will be set.
If a job is removed, \Attr{RemoveReason} will be set.
In addition, whenever a job's status changes,
\Attr{EnteredCurrentStatus} will contain the epoch time when the
change took place.

\item The error messages you get from \Condor{rm}, \Condor{hold} and
\Condor{release} have all been updated to be more specific and
accurate. 

\item Condor users can now specify a policy for when their jobs should
leave the queue or be put on hold.
They can specify expressions that are evaluated periodically, and
whenever the job exits.
This policy can be used to ensure that the job remains in the queue
and is re-run until it exits with a certain exit code, that the job
should be put on hold if a certain condition is true, and so on. 
If any of these policy expressions result in the job being removed
from the queue or put on hold, the UserLog entry for the event
includes a string describing why the action was taken.

\item Changed the way Condor finds the various \Condor{shadow} and
\Condor{starter} binaries you have installed on your machine.
Now, you can specify a \Macro{SHADOW\_LIST} and a
\Macro{STARTER\_LIST}.
These are treated much like the \Macro{DAEMON\_LIST} setting, they
specify a list of attribute names, each of which point to the actual
binary you want to use.
On startup, Condor will check these lists, make sure all the binaries
specified exist, and find out what abilities each program provides.
This information is used during matchmaking to ensure that a job which
requires a certain ability (like having a new enough version of Condor
to support transfering files on Unix) can find a resource that
provides that ability.

\item Added new security feature to offer fine-grained control over
what configuration values can be modified by \Condor{config\_val}
using \Arg{-set} and related options.
Pool administrators can now define lists of attributes that can be set
by hosts that authenticate to the various permission levels of
Condor's host based security (for example, \DCPerm{WRITE},
\DCPerm{ADMINISTRATOR}, etc).
These lists are defined by attributes with names like
\Macro{SETTABLE\_ATTRS\_CONFIG} and
\Macro{STARTD\_SETTABLE\_ATTRS\_OWNER}. 
For more information about host-based security in Condor, see
section~\ref{sec:Host-Security} on page~\pageref{sec:Host-Security}.
For more information about how to configure the new settings, see the
same section of the manual.
In particular, see section~\ref{sec:Host-Security} on
page~\pageref{sec:Host-Security}. 

\item Greatly improved the handling of the ``soft kill signal'' you
can specify for your job.
This signal is now stored as a signal name, not an integer, so that it
works across different platforms.
Also, fixed some bugs where the signal numbers were getting translated
incorrectly in some circumstances.

\item Added the \Arg{-full} option to \Condor{reconfig}.
The \Arg{-full} option causes the Condor daemon to clear its cache of
DNS information and some other expensive operations.
So, the regular \Condor{reconfig} is now more light-weight, and can
be used more frequently without undue overhead on the Condor daemons. 
The default \Condor{reconfig} has also been changed so that it will
work from any host with \DCPerm{WRITE} permission in your pool,
instead of requiring \DCPerm{ADMINISTRATOR} access.

\item Added the \Macro{EMAIL\_DOMAIN} config file setting.
This allows Condor administrators to define a default domain where
Condor should send email if whatever \Macro{UID\_DOMAIN} is set to
would yield invalid email addresses.
For more information, see section~\ref{param:EmailDomain} on
page~\pageref{param:EmailDomain}.

\item
Added support for RedHat 7.2.

\item When printing out the UserLog, we now only log a new event for
``Image size of job updated'' when the new value is different than the
existing value.

\end{itemize}

\noindent Bugs Fixed:

\begin{itemize}

\item
Fixed a bug in Condor-PVM where it was possible that a machine would be 
placed into the virtual machine, but then ignored by Condor for the purposes
of scheduling tasks there.

\item
Under Solaris, the checkpointing libraries could segfault while determining
the page size of the machine. 
This has been fixed.

\item
In a heavily loaded submit machine, the \Condor{schedd} would time out
authentication checks with its shadows. 
This would cause the shadows to
exit believing the \Condor{schedd} had died placing jobs into the idle
state and the \Condor{schedd} to exhibit poor performance.
This timeout problem has been corrected.

\item
Removed use of the bfd libary in the Condor Linux distribution. 
This will make the dynamic versions of the Condor executables have a
higher chance of being usable when RedHat upgrades.

\item
When you specify ``STARTD\_HAS\_BAD\_UTMP = True'' in the config files
on a linux machine with a 2.4+ kernel, the \Condor{startd} would report
an error stating some of the tty entries in /dev. This would result
in incorrect tty activity sampling causing jobs to not be migrated or
incorrectly started on a resource. This has now been corrected.

\item 
When you specify ``GenEnv = True'' in a \Condor{submit} file,
your environment is no longer restricted to 10KB.

\item
The three-digit event numbers which begin each job event in the
userlog were incorrect for some events in Condor 6.3.0 and 6.3.1.
Specifically, ULOG\_JOB\_SUSPENDED, ULOG\_JOB\_UNSUSPENDED,
ULOG\_JOB\_HELD, ULOG\_JOB\_RELEASED, ULOG\_GENERIC, and
ULOG\_JOB\_ABORTED had incorrect event numbers.  This has now been
corrected.

\Note This means userlog-parsing code written for Condor 6.3.0 or
6.3.1 development releases may not work reliably with userlogs
generated by other versions of Condor, and visa-versa.  Userlog events
will remain compatible between all stable releases of Condor, however,
and with post-6.3.1 releases in this development series.

\item
The \Condor{run} script now correctly exits when it sees a job aborted
event, instead of hanging, waiting for a termination event.

\item
Until now, when a DAG node's Condor job failed, the node failed,
regardless of whether its POST script succeeded or failed.  This was a
bug, because it prevented users from using POST scripts to evaluate
jobs with non-zero exit codes and deem them successful anyway.  This
has now been fixed -- a node's success is equal to its POST script's
success -- but the change may affect existing DAGs which rely on the
old, broken behavior.  Users utilizing POST scripts must now be sure
to pass the POST script the job's return value, and return it again,
if they do not wish to alter it; otherwise failed jobs will be masked
by ignorant POST scripts which always succeed.

\end{itemize}

\noindent Known Bugs:

\begin{itemize}
\item The HP-UX Vendor C++ CFront compiler does not work with \Condor{compile}
if exception handling is enabled with +eh.

\item The HP-UX Vendor aCC compiler does not work at all with Condor.
\end{itemize}

%%%%%%%%%%%%%%%%%%%%%%%%%%%%%%%%%%%%%%%%%%%%%%%%%%%%%%%%%%%%%%%%%%%%%%
\subsection{\label{sec:New-6-3-2}Version 6.3.2}
%%%%%%%%%%%%%%%%%%%%%%%%%%%%%%%%%%%%%%%%%%%%%%%%%%%%%%%%%%%%%%%%%%%%%%

Version 6.3.2 of Condor was only released as a version of
``Condor-G''.
This version of Condor-G is not widely deployed.
However, to avoid confusion, the Condor developers did not want to
release a full Condor distribution with the same version number.


%%%%%%%%%%%%%%%%%%%%%%%%%%%%%%%%%%%%%%%%%%%%%%%%%%%%%%%%%%%%%%%%%%%%%%
\subsubsection{\label{sec:New-6-3-1}Version 6.3.1}
%%%%%%%%%%%%%%%%%%%%%%%%%%%%%%%%%%%%%%%%%%%%%%%%%%%%%%%%%%%%%%%%%%%%%%

\noindent New Features:
\begin{itemize}

\item
Added support for an \AdAttr{x509proxy} option in
\Condor{submit}. There is now a seperate \Condor{GridManager} for each
user and proxy pair. This will be detailed in a future release of
Condor.
 
\item
More Condor DAGMan improvements and bug fixes:

\begin{itemize}

\item 
Added a \oArgnm{-dag} flag to \Condor{q} to more succinctly display dags
and their ownership.

\item
Added a new event to the Condor userlog at the completion of a POST
script.  This allows DAGMan, during recovery, to know which POST
scripts have finished succesfully, so it no longer has to re-run them
all to make sure.

\item
Implemented separate \Arg{-MaxPre} and \Arg{-MaxPost} options to limit
the number of simultaneously running PRE and POST scripts.  The
\Arg{-MaxScripts} option is still available, and is equivalent to
setting both \Arg{-MaxPre} and \Arg{-MaxPost} to the same value.

\item
Added support for a new ``Retry'' parameter in the DAG file, which
instructs DAGMan to automatically retry a node a configurable number
of times if its PRE Script, Job, or POST Script fail for any reason.

\item
Added timestamps to all DAGMan log messages.

\item
Fixed a bug whereby DAGMan would clean up its lock file without
creating a rescue file when killed with SIGTERM.

\item
DAGMan no longer aborts the DAG if it encounters executable error or
job aborted events in the userlog, but rather marks the corresponding
DAG nodes as ``failed'' so the rest of the DAG can continue.

\item
Fixed a bug whereby DAGMan could crash if it saw userlog events for
jobs it didn't submit.

\end{itemize}

\item Added port restriction capabilities to Condor so you can specify a range
of ports to use for the communication between Condor Daemons.

\item To improve performance: if there's no \Macro{HISTORY} file
specified, don't connect back to the schedd to report your exit info on
successful compeletion, since the schedd is simply going to discard that
info anyway.

\item Added the macro \Macro{SECONDARY\_COLLECTOR\_LIST} to tell the
master to send classads to an additional list of collectors so you can
do administration commands when the primary collector is down.

\item When a job checkpoints it askes the shadow whether or not it
should and if so where. This fixes some flocking bugs and increases
performance of the pool.

\item Added match rejection diagnostics in \Condor{q} \oArgnm{-analyze} to
give more information on why a particular job hasn't started up yet.

\item Added \oArgnm{--vms} argument to \Condor{glidein} that enables the
control of how many virtual machines to start up on the target platform.

\item Added capability to the config file language to retrieve environment
variables while being processed.

\item Added capability to make default user user priority factor configurable
with the \Macro{DEFAULT\_PRIORITY\_FACTOR} macro in the config files.

\item Added full support for RedHat 7.1 and the gcc 2.96 compiler. However,
the standard universe binaries must still be statically linked.

\item When jobs are suspended or unsuspended, an event is now written into
the user job log.

\item Added \oArgnm{-a} flag to \Condor{submit} to add/override attributes
specified in the submit file.

\item Under Unix, added the ability for a submittor of a job to describe when
and how a job is allowed/not allowed to leave the queue. For example, if
a job has only run for 5 minutes, but it was supposed to have run an hour 
minimum, then do not let the job leave the queue but restart it instead.

\item New environment variable available CONDOR\_SCRATCH\_DIR available
in a standard or vanilla job's environment that denotes temporary space
the job can use that will be cleaned up automatically when the job leaves
from the machine.

\item Not exactly a new feature, but some internal parts of Condor had been
fixed up to try and improve the memory footprint of a few of our daemons.

\end{itemize}

\noindent Bugs Fixed:
\begin{itemize}

\item Fixed a bug where \Condor{q} would produce wildly inaccurate run time
reports of jobs in the queue.

\item Fixed it so that if the condor scheduler fails to notify the
administrator through email, just print a warning and do not except.

\item Fixed a bug where \Condor{submit} would incorrectly create the user
log file.

\item Fixed a bug where a job queue sorted by date with \Condor{q} would
be displayed in descending instead of ascending order.

\item Fixed and improved error handling when \Condor{submit} fails.

\item Numerous fixes in the Condor User Log System.

\item Fixed a bug where when Condor inspects its on disk job queue log,
it would do it with case sensitivity. Now there is no case sensitivity.

\item Fixed a bug in \Condor{glidein} where it have trouble figuring out
the architecture of a minimally installed HP-UX machine.

\item Fixed it so that email to the user has the word ``condor'' capitalized
in the subject.

\item Fixed a situation where when a user has multiple schedulers submitting
to the same pool, the Negotiator would starve some of the schedulers.

\item Added a feature whereby if a transfer of an executable
from a submission machine to an execute machine fails, Condor
will retry a configurable numbers of times denotated by the
\Macro{EXEC\_TRANSFER\_ATTEMPTS} macro. This macro defaults to three if
left undefined. This macro exists only for the Unix port of Condor.

\item Fixed a bug where if a schedd had too many rejected clusters during a
match phase, it would ``except'' and have to be restarted by the master.

\end{itemize}

\noindent Known Bugs:
\begin{itemize}
\item The HP-UX Vendor C++ CFront compiler does not work with \Condor{compile}
if exception handling is enabled with +eh.

\item The HP-UX Vendor aCC compiler does not work at all with Condor.
\end{itemize}

%%%%%%%%%%%%%%%%%%%%%%%%%%%%%%%%%%%%%%%%%%%%%%%%%%%%%%%%%%%%%%%%%%%%%%
\subsubsection{\label{sec:New-6-3-0}Version 6.3.0}
%%%%%%%%%%%%%%%%%%%%%%%%%%%%%%%%%%%%%%%%%%%%%%%%%%%%%%%%%%%%%%%%%%%%%%

\noindent New Features:
\begin{itemize}

\item Added support for running MPICH jobs under Condor.

\end{itemize}

\noindent
Many Condor DAGMan improvements and bug fixes:

\begin{itemize}

\item
PRE and POST scripts now run asynchronously, rather than synchronously
as in the past.  As a result, DAGMan now supports a \Arg{-MaxScripts}
option to limit the number of simultaneously running PRE and POST
scripts.

\item
Whether or not POST scripts are always executed after failed jobs is
now configurable with the \Arg{-NoPostFail} argument.

\item
Added a \Arg{-r} flag to \Condor{submit\_dag} to submit DAGMan to a
remote \Condor{schedd}.

\item
Made the arguments to \Condor{submit\_dag} case-insensitive.

\item
Fixed a variety of bugs in DAGMan's event handling, so DAGMan should
no longer hang indefinitely after failed jobs, or mistake one job's
userlog events for those of another.

\item
DAGMan's error handling and logging output have been substantially
clarified and improved.  For example, DAGMan now prints a list of
failed jobs when it exits, rather than just saying ``some jobs
failed''.

\item
Jobs submitted by a \Condor{dagman} job now have \AdAttr{DAGManJobId}
and \AdAttr{DAGNodeName} in the job classad.

\item
Fixed a \Condor{submit\_dag} bug preventing the submission of DAGMan
Rescue files.

\item
Improved the handling of userlog errors (less crashing, more coping).

\item
Fixed a bug when recovering from the userlog after a crash or reboot.

\item
Fixed bugs in the handling of \Arg{-MaxJobs}.

\item
Added a \Arg{-a line} argument to \Condor{submit} to add a line to the
submit file before processing (overriding the submit file).

\item
Added a \Arg{-dag} flag to \Condor{q} to format and sort DAG jobs
sensibly under their DAGMan master job.

\end{itemize}

\noindent Known Bugs:

\begin{itemize}

\item \Condor{kbdd} doesn't work properly under Compaq Tru64 5.1, and
as a result, resources may not leave the ``Unclaimed'' state
regardless of keyboard or pty activity.  Compaq Tru64 5.0a and earlier
do work properly.

\end{itemize}

%%%%%%%%%%%%%%%%%%%%%%%%%%%%%%%%%%%%%%%%%%%%%%%%%%%%%%%%%%%%%%%%%%%%%%%
\section{\label{sec:History-6-2}Stable Release Series 6.2}
%%%%%%%%%%%%%%%%%%%%%%%%%%%%%%%%%%%%%%%%%%%%%%%%%%%%%%%%%%%%%%%%%%%%%%

This is the second stable release series of Condor.
All of the new features developed in the 6.1 series are now considered
stable, supported features of Condor.
New releases of 6.2.0 should happen infrequently and will only include
bug fixes and support for new platforms.
New features will be added and tested in the 6.3 development series. 
The details of each version are described below.

%%%%%%%%%%%%%%%%%%%%%%%%%%%%%%%%%%%%%%%%%%%%%%%%%%%%%%%%%%%%%%%%%%%%%%
\subsection{\label{sec:New-6-2-1}Version 6.2.1}
%%%%%%%%%%%%%%%%%%%%%%%%%%%%%%%%%%%%%%%%%%%%%%%%%%%%%%%%%%%%%%%%%%%%%%

\noindent Bugs Fixed:

\begin{itemize}

\item Fixed a bug in the \Condor{startd} that would cause the daemon
to crash if you set the \Macro{POLLING\_INTERVAL} macro to a value
greater than 60.

\item In \Condor{q}, dash-arguments (e.g., -pool, -run, etc.) were being
parsed incorrectly such that the same arguments specified without a
dash would be interpreted as if the dash were present, making it
impossible to specify ``pool'' or ``globus'' or ``run'' as an owner
argument.

\item Fixed bug in \Condor{submit} that would cause certain submit
file directives to be silently ignored if you used the wrong attribute
name.  
Now, all submit file attributes can use the same names you see in the
job ClassAd (what you'd see with \begin{verbatim}condor\_q
-long\end{verbatim}).
For example, you can now use \begin{verbatim}CoreSize =
0\end{verbatim} or \begin{verbatim}core_size = 0\end{verbatim} in your 
submit file, and either one would be recognized.

\item Fixed some of the error messages in \Condor{submit} so that they
are all consistently formatted.

\end{itemize}

\noindent Known Bugs:

\begin{itemize}

\item None.

\end{itemize}


%%%%%%%%%%%%%%%%%%%%%%%%%%%%%%%%%%%%%%%%%%%%%%%%%%%%%%%%%%%%%%%%%%%%%%
\subsubsection{\label{sec:New-6-2-0}Version 6.2.0}
%%%%%%%%%%%%%%%%%%%%%%%%%%%%%%%%%%%%%%%%%%%%%%%%%%%%%%%%%%%%%%%%%%%%%%

\noindent New Features Over the 6.0 Release Series
\begin{itemize}

\item Support for running multiple jobs on SMP (Symmetric Mutli-Processor)
machines.

\end{itemize}

\noindent New Features Over the Last Development Series: 6.1.17
\begin{itemize}

\item If \Attr{CkptArch} isn't specified in the job submission file's
\Attr{Requirements} attribute, then automatically add this expression:

\begin{verbatim}
CkptRequirements = ((CkptArch == Arch) || (CkptArch =?= UNDEFINED)) &&
	((CkptOpSys == OpSys) || (CkptOpSys =?= UNDEFINED))
\end{verbatim}

to the \Attr{Requirements} expression. This allows for users who specify
a heterogeneous submission to not have to worry about having their checkpoints
incorrectly starting up on architectures for which they were not designed
to run.

\item The \Macro{APPEND\_REQ\_<universe>} config file entries now get
appended to the beginning of the expressions before Condor adds internal
default expressions.  This allows the sysadmin to override any default
policy that Condor enforces.

\item There is now a single \Macro{APPEND\_REQUIREMENTS} attribute
that will get appended to all universe's \Attr{Requirements}
expressions unless a specific \Macro{APPEND\_REQ\_STANDARD} or
\Macro{APPEND\_REQ\_VANILLA} expression is defined.

\item Increased certain networking parameters to help alleviate the 
\Condor{shadow}'s inability to contact the \Condor{schedd} during heavy load
of the system.

\item Added a \Condor{glidein} man page to the manual.

\item Some of the log messages in the \Condor{startd} were modified to
be more clear and to provide more information.

\item Added a new attribute to the \Condor{startd} ClassAd when the
machine is claimed, \AdAttr{RemoteOwner}.

\end{itemize}

\noindent Bugs fixed since 6.1.17
\begin{itemize}

\item On NT, the Registry would increase in size while Condor was
servicing jobs. This has been fixed.

\item Added \File{utmpx} support for Solaris 2.8 to fix a problem where
\AdAttr{KeyBoardIdle} wasn't being set correctly.

\item When doing a \Condor{hold} under NT, the job was removed instead of
held. This has been fixed.

\item When using the \Arg{-master} argument to\Condor{restart}, the
\Condor{master} used to exit instead of restarting.
Now, the \Condor{master} correctly restarts itself in this case.

\end{itemize}

\noindent Known Bugs:
\begin{itemize}

\item \Attr{STARTD\_HAS\_BAD\_UTMP} does not work if set to True on Solaris 
2.8.  However, since \File{utmpx} support is enabled, you shouldn't
need to do this normally.

\end{itemize}

%%%%%%%%%%%%%%%%%%%%%%%%%%%%%%%%%%%%%%%%%%%%%%%%%%%%%%%%%%%%%%%%%%%%%%%
\section{\label{sec:History-6-1}Development Release Series 6.1}
%%%%%%%%%%%%%%%%%%%%%%%%%%%%%%%%%%%%%%%%%%%%%%%%%%%%%%%%%%%%%%%%%%%%%%

This was the first development release series.
It contains numerous enhancements over the 6.0 stable series.
For example:

\begin{itemize}
\item Support for running multiple jobs on SMP machines
\item Enhanced functionality for pool administrators
\item Support for PVM, MPI and Globus jobs
\item Support for \Term{Flocking} jobs across different Condor pools
\end{itemize}

The 6.1 series has many other improvements over the 6.0 series, and  
is available on more platforms.  
The new features, bugs fixed, and known bugs of each version are
described below in detail.

%%%%%%%%%%%%%%%%%%%%%%%%%%%%%%%%%%%%%%%%%%%%%%%%%%%%%%%%%%%%%%%%%%%%%%
\subsection*{\label{sec:New-6-1-17}Version 6.1.17}
%%%%%%%%%%%%%%%%%%%%%%%%%%%%%%%%%%%%%%%%%%%%%%%%%%%%%%%%%%%%%%%%%%%%%%

This version is the 6.2.0 ``release candidate''.  
It was publically released in Feburary of 2001, and it will be released
as 6.2.0 once it is considered ``stable'' by heavy testing at the 
UW-Madison Computer Science Department Condor pool.

\noindent New Features:

\begin{itemize}

\item Hostnames in the HOSTALLOW and HOSTDENY entries are now case-insensitive.

\item It is now possible to submit NT jobs from a UNIX machine.

\item The NT release of Condor now supports a USE\_VISIBLE\_DESKTOP parameter. 
If true, Condor will allow the job to create windows on the desktop of the
execute machine and interact with the job. This is particularly useful for 
debugging why an application will not run under Condor.

\item The \Condor{startd} contains support for the new MPI dedicated 
scheduler that will appear in the 6.3 development series. This will allow
you to use your 6.2 Condor pool with the new scheduler.

\item Added a \Opt{mixedcase} option to \Condor{config\_val} to allow 
for overriding the default of lowercasing all the config names

\item Added a pid\_snapshot\_interval option to the config file to
control how often the \Condor{startd} should examine the running 
process family. It defaults to 50 seconds.

\end{itemize}

\noindent Bugs Fixed:

\begin{itemize}

\item Fixed a bug with the \Condor{schedd} reaching the MAX\_JOBS\_RUNNING
mark and properly calculating Scheduler Universe jobs for preemption.

\item Fixed a bug in the \Condor{schedd} loosing track of \Condor{startd}s 
in the initial claiming phase. This bug affected all platforms, but was most
likely to manifest on Solaris 2.6

\item CPU Time can be greater than wall clock time in Multi-threaded
apps, so do not consider it an error in the UserLog.

\item \Condor{restart} \Opt{-master} now works correctly.
 
\item Fixed a rare condition in the \Condor{startd} that could corrupt
memory and result in a signal 11 (SIGSEGV, or segmentation violation).

\item Fixed a bug that would cause the ``execute event'' to not be
logged to the UserLog if the binary for the job resided on AFS.

\item Fixed a race-condition in Condor's PVM support on SMP machines
(introduced in version 6.1.16) that caused PVM tasks to be associated
with the wrong daemon.

\item Better handling of checkpointing on large-memory Linux machines.

\item Fixed random occasions of job completion email not being sent.

\item It is no longer possible to use \Condor{user\_prio} to set a priority of less
than 1.

\item Fixed a bug in the job completion email statistics.
Run Time was being underreported when the job completed after doing a
periodic checkpoint.

\item Fixed a bug that caused CondorLoadAvg to get stuck at 0.0 on
Linux when the system clock was adjusted.

\item Fixed a \Condor{submit} bug that caused all machine\_count
commands after the first queue statement to be ignored for PVM jobs.

\item PVM tasks now run as the user when appropriate instead of always
running under the UNIX ``nobody'' account.

\item Fixed support for the PVM group server.

\item PVM uses an environment variable to communicate with it's children
instead of a file in /tmp. This file previously could become overwritten
by mulitple PVM jobs.

\item \Condor{stats} now lives in the ``bin'' directory instead of ``sbin''.

\end{itemize}

\noindent Known Bugs:

\begin{itemize}

\item The \Condor{negotiator} can crash if the Accountantnew.log file becomes
corrupted. This most often occurs if the Central Manager runs out of diskspace. 

\end{itemize}

%%%%%%%%%%%%%%%%%%%%%%%%%%%%%%%%%%%%%%%%%%%%%%%%%%%%%%%%%%%%%%%%%%%%%%
\subsection*{\label{sec:New-6-1-16}Version 6.1.16}
%%%%%%%%%%%%%%%%%%%%%%%%%%%%%%%%%%%%%%%%%%%%%%%%%%%%%%%%%%%%%%%%%%%%%%

\noindent New Features:

\begin{itemize}

\item Condor now supports multiple pvmds per user on a machine.  Users
can now submit more than one PVM job at a time, PVM tasks can now run
on the submission machine, and multiple PVM tasks can run on SMP
machines.  \Condor{submit} no longer inserts default job requirements
to restrict PVM jobs to one pvmd per user on a machine.  This new
functionality requires the \Condor{pvmd} included in this (and future)
Condor releases.  If you set ``PVM\_OLD\_PVMD = True'' in the Condor
configuration file, \Condor{submit} will insert the default PVM job
requirements as it did in previous releases.  You must set this if you
don't upgrade your \Condor{pvmd} binary or if your jobs flock with pools
that user an older \Condor{pvmd}.

\item The NT release of Condor no longer contains debugging
information.
This drastically reduces the size of the binaries you must install.  

\end{itemize}

\noindent Bugs Fixed:

\begin{itemize}

\item The configuration files shipped with version 6.1.15 contained a
number of errors relating to host-based security, the configuration of
the central manager, and a few other things.
These errors have all been corrected.

\item Fixed a memory management bug in the \Condor{schedd} that could
cause it to crash under certain circumstances when machines were taken
away from the schedd's control.

\item Fixed a potential memory leak in a library used by the
\Condor{startd} and \Condor{master} that could leak memory while
Condor jobs were executing.

\item Fixed a bug in the NT version of Condor that would result in
faulty reporting of the load average.

\item The \Condor{shadow.pvm} should now correctly return core files
when a task or \Condor{pvmd} crashes.

\item This release fixes a memory error introduced in version
6.1.15 that could crash the \Condor{shadow.pvm}.

\item Some \Condor{pvmd} binaries in previous releases included
debugging code we added that could cause the \Condor{pvmd} to crash.
This release includes new \Condor{pvmd} binaries for all platforms
with the problematic debugging code removed.

\item Fixed a bug in the \Opt{-unset} options to \Condor{config\_val}
that was introduced in version 6.1.15.
Both \Opt{-unset} and \Opt{-runset} work correctly, now.

\end{itemize}

\noindent Known Bugs:

\begin{itemize}

\item None.

\end{itemize}

%%%%%%%%%%%%%%%%%%%%%%%%%%%%%%%%%%%%%%%%%%%%%%%%%%%%%%%%%%%%%%%%%%%%%%
\subsection*{\label{sec:New-6-1-15}Version 6.1.15}
%%%%%%%%%%%%%%%%%%%%%%%%%%%%%%%%%%%%%%%%%%%%%%%%%%%%%%%%%%%%%%%%%%%%%%

\noindent New Features:

\begin{itemize}

\item In the job submit description file passed to \Condor{submit}, 
a new style of macro (with two dollar-signs) can reference attributes
from the machine ClassAd.  This new style macro can be used in the
job's \MacroNI{Executable}, \MacroNI{Arguments}, or \MacroNI{Environment}
settings in the submit description file.  For example, if you have both
Linux and Solaris machines in your pool, the following submit description
file will run either foo.INTEL.LINUX or foo.SUN4u.SOLARIS27 as appropiate,
and will pass in the amount of memory available on that machine on the
command line:
\begin{verbatim}
	executable = foo.$$(Arch).$$(Opsys)
	arguments = $$(Memory)
	queue
\end{verbatim}

\item The \DCPerm{CONFIG} security access level now controls the
modification of daemon configurations using \Condor{config\_val}.  For
more information about security access levels, see
section~\ref{sec:Host-Security} on
page~\pageref{sec:Host-Security}.

\item The \Macro{DC\_DAEMON\_LIST} macro now indicates to the
\Condor{master} which processes in the \Macro{DAEMON\_LIST} use
Condor's DaemonCore inter-process communication mechanisms.  This
allows the \Condor{master} to monitor both processes developed with or
without the Condor DaemonCore library.

\item The new \Macro{NEGOTIATE\_ALL\_JOBS\_IN\_CLUSTER} macro can be
use to configure the \Condor{schedd} to not assume (for efficiency)
that if one job in a cluster can't be scheduled, then no other jobs in
the cluster can be scheduled.
If \Macro{NEGOTIATE\_ALL\_JOBS\_IN\_CLUSTER} is set to True, the
\Condor{schedd} will now always try to schedule each individual job in
a cluster.

\item The \Condor{schedd} now automatically adds any machine it is
matched with to its HOSTALLOW\_WRITE list.
This simplifies setting up a machine for flocking, since the
submitting user doesn't have to know all the machines where the job
might execute, they only have to know what central manager they wish
to flock to.
Submitting users must trust a central manager they report to, so this
doesn't impact security in any way.

\item Some static limits relating to the number of jobs which can be 
simultaneously started by the \Condor{schedd} has been removed.

\item The default Condor config file(s) which are installed by
the installation program have been re-organized for greater 
clarity and simplicity.  

\end{itemize}

\noindent Bugs Fixed:

\begin{itemize}

\item In the STANDARD Universe, jobs submitted to Condor could segfault
if they opened multiple files with the same name.  Usually this bug
was exposed when users would submit jobs without specifying a file
for either stdout or stderr; in this case, both would default to 
\File{/dev/null}, and this could trigger the problem.

\item The Linux 2.2.14 kernel, which is used by default with Red Hat 6.2,
has a serious bug can cause the machine to lock up when 
the same socket is used for repeated connection attempts.   Thus, 
previous versions of Condor could cause the 2.2.14 kernel to hang
(lots of other applications could do this as well).  The Condor Team
recommends that you upgrade your kernel to 2.2.16 or later.  However,
in v6.1.15 of Condor, a patch was added to the Condor networking
layer so that Condor would not trigger this Linux kernel bug.

\item If no email address was specified when the job was submitted
with \Condor{submit}, completion email was being sent to 
user@submit-machine-hostname.  This is not the correct behavior.  Now 
email goes by default to user@uid-domain, where uid-domain is
defined by the \MacroNI{UID\_DOMAIN} setting in the config file.

\item The \Condor{master} can now correctly shutdown and restart the
\Condor{checkpoint\_server}.

\item Email sent when a SCHEDULER Universe job compeltes now has the
correct From: header.

\item In the STANDARD universe, jobs which call sigsuspend() will 
now receive the correct return value.

\item Abnormal error conditions, such as the hard disk on the submit
machine filling up, are much less likely to result in a job disappearing
from the queue.

\item The \Condor{checkpoint\_server} now correctly reconfigures when
a \Condor{reconfig} command is received by the \Condor{master}.

\item Fixed a bug with how the \Condor{schedd} associates jobs with
machines (claimed resources) which would, under certain circumstances,
cause some jobs to remain idle until other jobs in the queue complete
or are preempted.

\item A number of PVM universe bugs are fixed in this release.
Bugs in how the \Condor{shadow.pvm} exited, which caused jobs to hang
at exit or to run multiple times, have been fixed.
The \Condor{shadow.pvm} no longer exits if there is a problem starting
up PVM on one remote host.
The \Condor{starter.pvm} now ignores the periodic checkpoint command
from the startd.  Previously, it would vacate the job when it received
the periodic checkpoint command.
A number of bugs with how the \Condor{starter.pvm} handled
asynchronous events, which caused it to take a long time to clean up
an exited PVM task, have been fixed.
The \Condor{schedd} now sets the status correctly on multi-class PVM
jobs and removes them from the job queue correctly on exit.
\Condor{submit} no longer ignores the machine\_count command for PVM
jobs.
And, a problem which caused pvm\_exit() to hang was diagnosed:
PVM tasks which call pvm\_catchout() to catch the output of
child tasks should be sure to call it again with a NULL argument to
disable output collection before calling pvm\_exit().

\item The change introduced in 6.1.13 to the \Condor{shadow} regarding
when it logged the execute event to the user log produced situations
where the shadow could log other events (like the shadow exception
event) before the execute event was logged.
Now, the \Condor{shadow} will always log an execute event before it
logs any other events.
The timing is still improved over 6.1.12 and older versions, with the
execute event getting logged after the bulk of the job initialization
has finished, right before the job will actually start executing.
However, you will no longer see user logs that contain a ``shadow
exception'' or ``job evicted'' message without a ``job executing''
event, first.

\item \Syscall{stat} and varient calls now go through the file table to
get the correct logical size and access times of buffered files.
Before, \Syscall{stat} used to return zero size on a buffered file that had
not yet been synced to disk.

\end{itemize}

\noindent Known Bugs:

\begin{itemize}

\item On IRIX 6.2, C++ programs compiled with GNU C++ (g++) 2.7.2 and
linked with the Condor libraries (using \Condor{compile}) will not
execute the constructors for any global objects.
There is a work-around for this bug, so if this is a problem for you,
please send email to \Email{condor-admin@cs.wisc.edu}.

\item In HP-UX 10.20, \Condor{compile} will not work correctly with HP's
C++ compiler. 
The jobs might link, but they will produce incorrect output, or die with
a signal such as SIGSEGV during restart after a checkpoint/vacate cycle.
However, the GNU C/C++ and the HP C compilers work just fine.

\item The \Syscall{getrusage} call does not work always as expected in
STANDARD Universe jobs.  
If your program uses \Syscall{getrusage}, it 
could decrease incorrectly by a second
across a checkpoint and restart.  In addition, the time it takes
Condor to restart from a checkpoint is included in the usage times
reported by \Syscall{getrusage}, and it probably should not be.

\end{itemize}


%%%%%%%%%%%%%%%%%%%%%%%%%%%%%%%%%%%%%%%%%%%%%%%%%%%%%%%%%%%%%%%%%%%%%%
\subsection*{\label{sec:New-6-1-14}Version 6.1.14}
%%%%%%%%%%%%%%%%%%%%%%%%%%%%%%%%%%%%%%%%%%%%%%%%%%%%%%%%%%%%%%%%%%%%%%

\noindent New Features:

\begin{itemize}

\item Initial supported added for Red Hat Linux 6.2 (i.e. glibc 2.1.3).

\end{itemize}

\noindent Bugs Fixed:

\begin{itemize}

\item In version 6.1.13, periodic checkpoints would not occur (see the
Known Bugs section for v6.1.13 listed below).  This bug, which only
impacts v6.1.13, has been fixed.

\end{itemize}

\noindent Known Bugs:

\begin{itemize}

\item The \Syscall{getrusage} call does not work properly inside
``standard'' jobs.  
If your program uses \Syscall{getrusage}, it will not report correct values
across a checkpoint and restart.
If your program relies on proper reporting from \Syscall{getrusage}, you
should either use version 6.0.3 or 6.1.10.

\item While Condor now supports many networking calls such as
\Syscall{socket} and \Syscall{connect}, (see the description below of this
new feature added in 6.1.11), on Linux, we cannot at this time support
\Syscall{gethostbyname} and a number of other database lookup calls.
The reason is that on Linux, these calls are implemented by bringing in a
shared library that defines them, based on whether the machine is using
DNS, NIS, or some other database method.
Condor does not support the way in which the C library tries to explicitly
bring in these shared libraries and use them.
There are a number of possible solutions to this problem, but the Condor
developers are not yet agreed on the best one, so this limitation might not
be resolved by 6.1.14.

\item In HP-UX 10.20, \Condor{compile} will not work correctly with HP's
C++ compiler. 
The jobs might link, but they will produce incorrect output, or die with
a signal such as SIGSEGV during restart after a checkpoint/vacate cycle.
However, the GNU C/C++ and the HP C compilers work just fine.

\item When a program linked with the Condor libraries (using \Condor{compile})
is writing output to a file, \Syscall{stat}--and variant calls,
will return zero for the size of the file if the program has not yet
read from the file or flushed the file descriptors.
This is a side effect of the file buffering code in Condor and will be
corrected to the expected semantic.

\item On IRIX 6.2, C++ programs compiled with GNU C++ (g++) 2.7.2 and
linked with the Condor libraries (using \Condor{compile}) will not
execute the constructors for any global objects.
There is a work-around for this bug, so if this is a problem for you,
please send email to \Email{condor-admin@cs.wisc.edu}.

\end{itemize}
%%%%%%%%%%%%%%%%%%%%%%%%%%%%%%%%%%%%%%%%%%%%%%%%%%%%%%%%%%%%%%%%%%%%%%
\subsection*{\label{sec:New-6-1-13}Version 6.1.13}
%%%%%%%%%%%%%%%%%%%%%%%%%%%%%%%%%%%%%%%%%%%%%%%%%%%%%%%%%%%%%%%%%%%%%%

\noindent New Features:

\begin{itemize}

\item Added \Macro{DEFAULT\_IO\_BUFFER\_SIZE} and
\Macro{DEFAULT\_IO\_BUFFER\_BLOCK\_SIZE} to config parameters to allow
the administrator to set the default file buffer sizes for user jobs
in \Condor{submit}.

\item There is no longer any difference in the configuration file
syntax between ``macros'' (which were specified with an ``='' sign)
and ``expressions'' (which were specified with a ``:'' sign).  
Now, all config file entries are treated and referenced as macros. 
You can use either ``='' or ``:'' and they will work the same way. 
There is no longer any problem with forward-referencing macros
(referencing macros you haven't yet defined), so long as they are
eventually defined in your config files (even if the forward reference
is to a macro defined in another config file, like the local config
file, for example).

\item \Condor{vacate} now supports a \Opt{-fast} option that forces
Condor to hard-kill the job(s) immediately, instead of waiting for
them to checkpoint and gracefully shutdown.

\item \Condor{userlog} now displays times in days+hours:minutes format
instead of total hours or total minutes.

\item The \Condor{run} command provides a simple front-end to
\Condor{submit} for submitting a shell command-line as a vanilla
universe job.

\item Solaris 2.7 SPARC, 2.7 INTEL have been added to the
list of ports that now support remote system calls and checkpointing.

\item Any mail being sent from Condor now shows up as having been sent from
the designated Condor Account, instead of root or ``Super User''.

\item The \Condor{submit} ``hold'' command may be used to submit jobs
to the queue in the hold state.  Held jobs will not run until released
with \Condor{release}.

\item It is now possible to use checkpoint servers in remote pools
when flocking even if the local pool doesn't use a checkpoint server.
This is now the default behavior (see the next item).

\item \Macro{USE\_CKPT\_SERVER} now defaults to True if a checkpoint
server is available.  It is usually more efficient to use a checkpoint
server near the execution site instead of storing the checkpoint back
to the submission machine, especially when flocking.

\item All Condor tools that used to expect just a hostname or address 
(\Condor{checkpoint}, \Condor{off}, \Condor{on}, \Condor{restart},
\Condor{reconfig}, \Condor{reschedule}, \Condor{vacate}) to specify
what machine to effect, can now take an optional \Opt{-name} or
\Opt{-addr} in front of each target.
This provides consistancy with other Condor tools that require the
\Opt{-name} or \Opt{-addr} options.
For all of the above mentioned tools, you can still just provide
hostnames or addresses, the new flags are not required.

\item Added \Opt{-pool} and \Opt{-addr} options to \Condor{rm},
\Condor{hold} and \Condor{release}.

\item When you start up the \Condor{master} or \Condor{schedd} as any
user other than ``root'' or ``condor'' on Unix, or ``SYSTEM'' on NT,
the daemon will have a default \Attr{Name} attribute that includes
both the username of the user who the daemon is running as and the
full hostname of the machine where it is running.

\item Clarified our Linux platform support.  We now officially
support the Red Hat 5.2 and 6.x distributions, and although other Linux
distributions (especially those with similar libc versions) may work,
they are not tested or supported.

\item The schedd now periodically updates the run-time counters in the
job queue for running jobs, so if the schedd crashes, the counters
will remain relatively up-to-date.  This is controlled by the
\Macro{WALL\_CLOCK\_CKPT\_INTERVAL} parameter.

\item The \Condor{shadow} now logs the ``job executing'' event in the
user log after the binary has been successfully transfered, so that
the events appear closer to the actual time the job starts running.
This can create some somewhat unexpected log files.  
If something goes wrong with the job's initialization, you might see
an ``evicted'' event before you see an ``executing'' event.

\end{itemize}

\noindent Bugs Fixed:

\begin{itemize}

\item Fixed how we internally handle file names for user jobs. This
fixes a nasty bug due to changing directories between checkpoints.

\item Fixed a bug in our handling of the \Macro{Arguments} macro in
the command file for a job. If the arguments were extremely long, or
there were an extreme number of them, they would get corrupted when the
job was spawned.

\item Fixed DAGMan. It had not worked at all in the previous release.

\item Fixed a nasty bug under Linux where file seeks did not work
correctly when buffering was enabled.

\item Fixed a bug where \Condor{shadow} would crash while sending job
completion e-mail forcing a job to restart multiple times and the user
to get multiple completion messages.

\item Fixed a long standing bug where Fortran 90 would occasionally
truncate its output files to random sizes and fill them with zeros.

\item Fixed a bug where \Syscall{close} did not propogate its return
value back to the user job correctly.

\item If a SIGTERM was delivered to a \Condor{shadow}, it used to
remove the job it was running from the job queue, as if \Condor{rm}
had been used.
This could have caused jobs to leave the queue unexpectedly.
Now, the \Condor{shadow} ignores SIGTERM (since the \Condor{schedd}
knows how to gracefully shutdown all the shadows when it gets a
SIGTERM), so jobs should no longer leave the queue prematurely.
In addition, on a SIGQUIT, the shadow now does a fast shutdown, just
like the rest of the Condor daemons.

\item Fixed a number of bugs which caused checkpoint restarts
to fail on some releases of Irix 6.5 (for example, when migrating from
a mips4 to a mips3 CPU or when migrating between machines with
different pagesizes).

\item Fixed a bug in the implementation of the \Syscall{stat} family
of remote system calls on Irix 6.5 which caused file opens in Fortran
programs to sometimes fail.

\item Fixed a number of problems with the statistics reported in the
job completion email and by \Condor{q} \Opt{-goodput}, including the
number of checkpoints and total network usage.  Correct values will
now be computed for all new jobs.

\item Changes in \Macro{USE\_CKPT\_SERVER} and
\Macro{CKPT\_SERVER\_HOST} no longer cause problems for jobs in the
queue which have already checkpointed.

\item Many of the Condor administration tools had a bug where they
would suffer a segmentation violation if you specified a \Opt{-pool} 
option and did not specify a hostname.
This case now results in an error message instead.

\item Fixed a bug where the \Condor{schedd} could die with a
segmentation violation if there was an error mapping an IP address
into a hostname.

\item Fixed a bug where resetting the time in a large negative direction
caused the \Condor{negotiator} to have a floating point error on some
platforms.

\item Fixed \Condor{q}'s output so that certain arguments are not ignored.

\item Fixed a bug in \Condor{q} where issuing a \Opt{-global} with a
fairly restrictive \Opt{-constraint} argument would cause garbage to be
printed to the terminal sometimes.

\item Fixed a bug which caused jobs to exit without completing a
checkpoint when preempted in the middle of a periodic checkpoint.
Now, the jobs will complete their periodic checkpoint in this case
before exiting.
\end{itemize}

\noindent Known Bugs:

\begin{itemize}

\item Periodic checkpoints do not occur.  Normally, when the config
file attribute \Macro{PERIODIC\_CHECKPOINT} evaluates to True, 
Condor performs a periodic checkpoint of the running job.  This
bug has been fixed in v6.1.14.  \Note there is a work-around to permit
periodic checkpoints to occur in v6.1.13: include the attribute name
``PERIODIC\_CHECKPOINT'' to the attributes 
listed in the \Macro{STARTD\_EXPRS} entry in the config file.

\item The \Syscall{getrusage} call does not work properly inside
``standard'' jobs.  
If your program uses \Syscall{getrusage}, it will not report correct values
across a checkpoint and restart.
If your program relies on proper reporting from \Syscall{getrusage}, you
should either use version 6.0.3 or 6.1.10.

\item While Condor now supports many networking calls such as
\Syscall{socket} and \Syscall{connect}, (see the description below of this
new feature added in 6.1.11), on Linux, we cannot at this time support
\Syscall{gethostbyname} and a number of other database lookup calls.
The reason is that on Linux, these calls are implemented by bringing in a
shared library that defines them, based on whether the machine is using
DNS, NIS, or some other database method.
Condor does not support the way in which the C library tries to explicitly
bring in these shared libraries and use them.
There are a number of possible solutions to this problem, but the Condor
developers are not yet agreed on the best one, so this limitation might not
be resolved by 6.1.14.

\item In HP-UX 10.20, \Condor{compile} will not work correctly with HP's
C++ compiler. 
The jobs might link, but they will produce incorrect output, or die with
a signal such as SIGSEGV during restart after a checkpoint/vacate cycle.
However, the GNU C/C++ and the HP C compilers work just fine.

\item When writing output to a file, \Syscall{stat}--and variant calls,
will return zero for the size of the file if the program has not yet
read from the file or flushed the file descriptors,
This is a side effect of the file buffering code in Condor and will be
corrected to the expected semantic.

\item On IRIX 6.2, C++ programs compiled with GNU C++ (g++) 2.7.2 and
linked with the Condor libraries (using \Condor{compile}) will not
execute the constructors for any global objects.
There is a work-around for this bug, so if this is a problem for you,
please send email to \Email{condor-admin@cs.wisc.edu}.

\end{itemize}

%%%%%%%%%%%%%%%%%%%%%%%%%%%%%%%%%%%%%%%%%%%%%%%%%%%%%%%%%%%%%%%%%%%%%%
\subsection*{\label{sec:New-6-1-12}Version 6.1.12}
%%%%%%%%%%%%%%%%%%%%%%%%%%%%%%%%%%%%%%%%%%%%%%%%%%%%%%%%%%%%%%%%%%%%%%

Version 6.1.12 fixes a number of bugs from version 6.1.11.
If you linked your ``standard'' jobs with version 6.1.11, you should
upgrade to 6.1.12 and re-link your jobs (using \Condor{compile}) as soon as
possible.

\noindent New Features:

\begin{itemize}

\item None.

\end{itemize}

\noindent Bugs Fixed:

\begin{itemize}

\item A number of system calls that were not being trapped by the Condor
libraries in version 6.1.11 are now being caught and sent back to the
submit machine.
Not having these functions being executed as remote system calls prevented
a number of programs from working, in particular Fortran programs, and
many programs on IRIX and Solaris platforms.

\item Sometimes submitted jobs report back as having no owner and have
\Bold{-????-} in the status line for the job. This has been fixed.

\item \Condor{q} \Opt{-io} has been fixed in this release.

\end{itemize}

\noindent Known Bugs:

\begin{itemize}

\item The \Syscall{getrusage} call does not work properly inside
``standard'' jobs.  
If your program uses \Syscall{getrusage}, it will not report correct values
across a checkpoint and restart.
If your program relies on proper reporting from \Syscall{getrusage}, you
should either use version 6.0.3 or 6.1.10.

\item While Condor now supports many networking calls such as
\Syscall{socket} and \Syscall{connect}, (see the description below of this
new feature added in 6.1.11), on Linux, we cannot at this time support
\Syscall{gethostbyname} and a number of other database lookup calls.
The reason is that on Linux, these calls are implemented by bringing in a
shared library that defines them, based on whether the machine is using
DNS, NIS, or some other database method.
Condor does not support the way in which the C library tries to explicitly
bring in these shared libraries and use them.
There are a number of possible solutions to this problem, but the Condor
developers are not yet agreed on the best one, so this limitation might not
be resolved by 6.1.13.

\item In HP-UX 10.20, \Condor{compile} will not work correctly with HP's
C++ compiler. 
The jobs might link, but they will produce incorrect output, or die with
a signal such as SIGSEGV during restart after a checkpoint/vacate cycle.
However, the GNU C/C++ and the HP C compilers work just fine.

\item When writing output to a file, \Syscall{stat}--and variant calls,
will return zero for the size of the file if the program has not yet
read from the file or flushed the file descriptors,
This is a side effect of the file buffering code in Condor and will be
corrected to the expected semantic.

\item On IRIX 6.2, C++ programs compiled with GNU C++ (g++) 2.7.2 and
linked with the Condor libraries (using \Condor{compile}) will not
execute the constructors for any global objects.
There is a work-around for this bug, so if this is a problem for you,
please send email to \Email{condor-admin@cs.wisc.edu}.

\item The \Opt{-format} option in \Condor{q} has no effect when querying
remote machines with the \Opt{-n} option.

\item \Condor{dagman} does not work at all in this release. 
The behaviour of its failure is to exit immediately with a success and
to not perform any work. It will be fixed in the next release of Condor.

\end{itemize}


%%%%%%%%%%%%%%%%%%%%%%%%%%%%%%%%%%%%%%%%%%%%%%%%%%%%%%%%%%%%%%%%%%%%%%
\subsection*{\label{sec:New-6-1-11}Version 6.1.11}
%%%%%%%%%%%%%%%%%%%%%%%%%%%%%%%%%%%%%%%%%%%%%%%%%%%%%%%%%%%%%%%%%%%%%%

\noindent New Features:

\begin{itemize}

\item \Condor{status} outputs information for held jobs instead of
MaxRunningJobs when supplied with \Opt{-schedd} or \Opt{-submitter}.

\item \Condor{userprio} now prints 4 digit years (for Y2K compiance). 
If you give a two digit date, it also will assume that 1/1/00 is 1/1/2000
and not 1/1/1900.

\item IRIX 6.5 has been added to the list of ports that now support
remote system calls and checkpointing.

\item \Condor{q} has been fixed to be faster and much more memory
efficient.  This is much more obvious when getting the queue from
\Condor{schedd}'s that have more than 1000 jobs.

\item Added support for support for socket() and pipe() in standard
jobs.  Both sockets and pipes are created on the executing machine.
Checkpointing is deferred anytime a socket or pipe is open.

\item Added limited support for select() and poll() in standard jobs.
Both calls will work only on files opened locally.

\item Added limited support for fcntl() and ioctl() in standard jobs.
Both calls will be performed remotely if the control-number is understood
and the third argument is an integer.

\item Replaced buffer implementation in standard jobs.
The new buffer code reads and writes variable sized chunks.
It will never issue a read to satisfy a write.  Buffering is enabled
by default.

\item Added extensive feedback on I/O performance in the user's email.

\item Added \Opt{-io} option to \Condor{q} to show I/O statistics.

\item Removed libckpt.a and libzckpt.a.  To build for standalone
checkpointing, just do a regular \Condor{compile}.
No -standalone option is necessary.

\item The checkpointing library now only re-opens files when they are
actually used.  If files or other needed resources cannot be found
at restart time, the checkpointer will fail with a verbose error.

\item The \Attr{RemoteHost} and \Attr{LastRemoteHost} attributes in
the job classad now contain hostnames instead IP address and port
numbers.  The \Opt{-run} option of older versions of \Condor{q} is not
compatible with this change.

\item Condor will now automatically check for compatibility between
the version of the Condor libraries you have linked into a standard
job (using \Condor{compile}) and the version of the \Condor{shadow}
installed on your submit machine.
If they are incompatible, the \Condor{shadow} will now put your job on
hold.  
Unless you set ``Notification = Never'' in your submit file, Condor
will also send you email explaining what went wrong and what you can
do about it.

\item All Condor daemons and tools now have a \Attr{CondorPlatform}
string, which shows which platform a given set of Condor binaries was
built for.
In all places that you used to see \Attr{CondorVersion}, you will now
see both \Attr{CondorVersion} and \Attr{CondorPlatform}, such as in
each daemon's ClassAd, in the output to a \Opt{-version} option (if
supported), and when running \Prog{ident} on a given Condor binary. 
This string can help identify situations where you are running the 
wrong version of the Condor binaries for a given platform (for
example, running binaries built for Solaris 2.5.1 on a Solaris 2.6
machine).   

\item Added commented-out settings in the default
\File{condor\_config} file we ship for various SMP-specific settings
in the \Condor{startd}.
Be sure to read section~\ref{sec:Configuring-SMP} on ``Configuring the
Startd for SMP Machine'' on page~\pageref{sec:Configuring-SMP} for
details about using these settings. 

\item \Condor{rm}, \Condor{hold}, and \Condor{release} all support
\Opt{-help} and \Opt{-version} options now.

\end{itemize}

\noindent Bugs Fixed:

\begin{itemize}

\item A race condition which could cause the \Condor{shadow} to not
exit when its job was removed has been fixed.
This bug would cause jobs that had been removed with \Condor{rm} to
remain in the queue marked as status ``X'' for a long time.
In addition, Condor would not shutdown quickly on hosts that had hit
this race condition, since the \Condor{schedd} wouldn't exit until all
of its \Condor{shadow} children had exited.

\item A signal race condition during restart of a Condor job has
been fixed.

\item In a Condor linked job, \Syscall{getdomainname} is now
supported. 

\item IRIX 6.5 can give negative time reports for how long a process has been
running. We account for that now in our statistics about usage times.

\item The \Condor{status} memory error introduced in version 6.1.10
has been fixed.

\item The \Macro{DAEMON\_LIST} configuration setting is now case
insensitive.

\item Fixed a bug where the \Condor{schedd}, under rare circumstances,
cause another schedd's jobs not to be matched.

\item The free disk space is now properly computed on Digital Unix.
This fixed problems where the \Attr{Disk} attribute in the
\condor{startd} classad reported incorrect values.

\item The config file parser now detects incremental macro definitions
correctly (see section~\ref{sec:Config-File-Macros} on
page~\pageref{sec:Config-File-Macros}).  Previously, when a macro (or
expression) being defined was a substring of a macro (or expression)
being referenced in its definition, the reference would be erroneously
marked as an incremental definition and expanded immediately.  The
parser now verifies that the entire strings match.

\end{itemize}

\noindent Known Bugs:

\begin{itemize}

\item The output for \condor{q} \Opt{-io} is incorrect and will likely show
zeroes for all values.  A fixed version will appear in the next release.

\end{itemize}

%%%%%%%%%%%%%%%%%%%%%%%%%%%%%%%%%%%%%%%%%%%%%%%%%%%%%%%%%%%%%%%%%%%%%%
\subsection*{\label{sec:New-6-1-10}Version 6.1.10}
%%%%%%%%%%%%%%%%%%%%%%%%%%%%%%%%%%%%%%%%%%%%%%%%%%%%%%%%%%%%%%%%%%%%%%

\noindent New Features:

\begin{itemize}

\item \Condor{q} now accepts \texttt{-format} parameters like \Condor{status}

\item \Condor{rm}, \Condor{hold} and \Condor{release} accept
  \texttt{-constraint} parameters like \Condor{status}

\item \Condor{status} now sorts displayed totals by the first column.
(This feature introduced a bug in \Condor{status}.  See ``Known Bugs''
below.)

\item Condor version 6.1.10 introduces ``clipped'' support for Sparc
Solaris version 2.7.
This version does not support checkpointing or remote system calls.
Full support for Solaris 2.7 will be released soon.

\item Introduced code to enable Linux to use the standard C library's
I/O buffering again, instead of relying on the Condor I/O buffering
code (which is still in beta testing).  

\end{itemize}

\noindent Bugs Fixed:

\begin{itemize}

\item The bug in checkpointing introduced in version 6.1.9 has been
fixed.
Checkpointing will now work on all platforms, as it always used to.  
Any jobs linked with the 6.1.9 Condor libraries will need to be
relinked with \Condor{compile} once version 6.1.10 has been installed
at your site. 

\end{itemize}

\noindent Known Bugs:

\begin{itemize}

\item The \AdAttr{CondorLoadAvg} attribute in the \Condor{startd} has
some problems in the way it is computed.
The CondorLoadAvg is somewhat inaccurate for the first minute a job
starts running, and for the first minute after it completes.
Also, the computation of CondorLoadAvg is very wrong on NT.
All of this will be fixed in a future version.

\item A memory error may cause \Condor{status} to die with SIGSEGV
(segmentation violation) when displaying totals or cause incorrect
totals to be displayed.  This will be fixed in version 6.1.11.

\end{itemize}


%%%%%%%%%%%%%%%%%%%%%%%%%%%%%%%%%%%%%%%%%%%%%%%%%%%%%%%%%%%%%%%%%%%%%%
\subsection*{\label{sec:New-6-1-9}Version 6.1.9}
%%%%%%%%%%%%%%%%%%%%%%%%%%%%%%%%%%%%%%%%%%%%%%%%%%%%%%%%%%%%%%%%%%%%%%

\noindent New Features:

\begin{itemize}

\item Added full support for Linux 2.0.x and 2.2.x kernels using
libc5, glibc20 and glibc21.
This includes support for Red Hat 6.x, Debian 2.x and other popular
Linux distributions.
Whereas the Linux machines had once been fragmented across libc5 and
GNU libc, they have now been reunified.
This means there is no longer any need for the ``LINUX-GLIBC'' OpSys
setting in your pool: all machines will now show up as ``LINUX''.
Part of this reunification process was the removal of dynamically
linked user jobs on Linux.
\Condor{compile} now forces static linking of your Standard Universe
Condor jobs. 
Also, please use \Condor{compile} on the same machine on which you
compiled your object files.

\item Added \Condor{qedit} utility to allow users to modify job
attributes after submission.  See the new manual page on
page~\pageref{man-condor-qedit}.

\item Added \OptArg{{-runfor}{minutes}} option to daemonCore to have
the daemon gracefully shut down after the given number of minutes.

\item Added support for statfs(2) and fstatfs(2) in user jobs. We support 
only the fields
\textit{f\_bsize, f\_blocks, f\_bfree, f\_bavail, f\_files, f\_ffree} from
the structure statfs. This is still in the experimental stage.

\item Added the \Opt{-direct} option to \Condor{status}.
After you give \Opt{-direct}, you supply a hostname, and
\Condor{status} will query the \Condor{startd} on the specified host
and display information directly from there, instead of querying the
\Condor{collector}.
See the manual page on page~\pageref{man-condor-submit} for details. 

\item Users can now define \Macro{NUM\_CPUS} to override the automatic
computation of the number of CPUs in your machine.
Using this config setting can cause unexpected results, and is not
recommended. 
This feature is only provided for sites that specifically want this
behavior and know what they are doing.

\item The \Opt{-set} and \Opt{-rset} options to \Condor{config\_val}
have been changed to allow administrators to set both macros and
expressions.
Previously, \Condor{config\_val} assumed you wanted to set
expressions.
Now, these two options each take a single argument, the string
containing exactly what you would put into the config file, so you can
specify you want to create a macro by including an ``='' sign, or an
expression by including a ``:''.
See section~\ref{sec:Intro-to-Config-Files} on
page~\pageref{sec:Intro-to-Config-Files} for details on macros
vs. expressions.
See the \Condor{config\_val} man page on
page~\pageref{man-condor-config-val} for details on
\Condor{config\_val}.  

\item If the directory you specified for LOCK (which holds lock files
used by Condor) doesn't exist, Condor will now try to create that
directory for you instead of giving up right away.

\item If you change the \Attr{COLLECTOR\_HOST} setting and reconfig
the \Condor{startd}, the startd will ``invalidate'' its ClassAds at
the old collector before it starts reporting to the new one.

\end{itemize}

\noindent Bugs Fixed:

\begin{itemize}

\item Fixed a major bug dealing with the group access a Condor job is
started with.
Now, Condor jobs are started with all the groups the job's owner is
in, not just their default group.
This also fixes a security hole where user jobs could be started up in
access groups they didn't belong to.

\item Fixed a bug where there was a needless limitation on the number of open
file descriptors a user job could have.

\item Fixed a standalone checkpointing bug where we weren't blocking signals
in critical sections and causing file table corruption at checkpoint
time.

\item Fixed a linker bug on Digital Unix 4.0 concerning fortran where
the linker would fail on \_\_uname and \_\_sigsuspend.

\item Fixed a bug in \Condor{shadow} that would send incorrect job
completion email under Linux.

\item Fixed a bug in the remote system call of \Syscall{fchdir} that caused
a garbage file descriptor to be used in Standard Universe jobs.

\item Fixed a bug in the \Condor{shadow} which was causing \Condor{q}
\Opt{-goodput} to display incorrect values for some jobs.

\item Fixed some minor bugs and made some minor enhancements in the
\Condor{install} script.
The bugs included a typo in one of the questions asked, and incorrect
handling for the answers of a few different questions.
Also, if DNS is misconfigured on your system, \Condor{install} will
try a few ways to find your fully qualified hostname, and if it still
can't determine the correct hostname, it will prompt the user for it. 
In addition, we now avoid one installation step in cases were it is
not needed. 

\item Fixed a rare race condition that could delay the completion of
large clusters of short running jobs. 

\item Added more checking to the various arguments that might be
passed to \Condor{status}, so that in the case of bad input,
\Condor{status} will print an error message and exit, instead of
performing a segmentation fault.
Also, when you use the \Opt{-sort} option, \Condor{status} will only
display ClassAds where the attributes you use to sort are defined.

\item Fixed a bug in the handling of the config files created by
using the \Opt{-set} or \Opt{-rset} options to \Condor{config\_val}.
Previously, if you manually deleted the files that were created, you
could cause the affected Condor daemon to have a segmentation fault.
Now, the daemons simply exit with a fatal error but still have a
chance to clean up.

\item Fixed a bug in the \Opt{-negotiator} option for most Condor
tools that was causing it to get the wrong address.

\item Fixed a couple of bugs in the \Condor{master} that could cause
improper shutdowns. 
There were cases during shutdown where we would restart a daemon
(because we previously noticed a new executable, for example).
Now, once you begin a shutdown, the \Condor{master} will not restart
anything. 
Also, fixed a rare bug that could cause the \Condor{master} to stop
checking the timestamps on a daemon.

\item Fixed a minor bug in the \Opt{-owner} option to
\Condor{config\_val} that was causing \Condor{init} not to work.

\item Fixed a bug where the \Condor{startd}, while it was already
shutting down, was allowing certain actions to succeed that should
have failed.
For example, it allowed itself to be matched with a user looking for
available machines, or to begin a new PVM task.

\end{itemize}

\noindent Known Bugs:

\begin{itemize}

\item The \AdAttr{CondorLoadAvg} attribute in the \Condor{startd} has
some problems in the way it is computed.
The CondorLoadAvg is somewhat inaccurate for the first minute a job
starts running, and for the first minute after it completes.
Also, the computation of CondorLoadAvg is very wrong on NT.
All of this will be fixed in a future version.

\item There is a serious bug in checkpointing when using Condor's
I/O buffering for ``standard'' jobs.
By default, Linux uses Condor buffering in version 6.1.9 for all
standard jobs.
The bug prevents checkpointing from working more than once.
This renders the \Condor{vacate} and \Condor{checkpoint} commands
useless, and jobs will just be killed without a checkpoint when
machine owners come back to their machines.

\end{itemize}


%%%%%%%%%%%%%%%%%%%%%%%%%%%%%%%%%%%%%%%%%%%%%%%%%%%%%%%%%%%%%%%%%%%%%%
\subsection*{\label{sec:New-6-1-8}Version 6.1.8}
%%%%%%%%%%%%%%%%%%%%%%%%%%%%%%%%%%%%%%%%%%%%%%%%%%%%%%%%%%%%%%%%%%%%%%

\begin{itemize}

\item Added \Term{file\_remaps} as command in the job submit file given to
STANDARD universe jobs.
A Job can now specify that it would like to have files be remapped
from one file to another.
In addition you can specify that files should be read from the local machine
by specifing them.
See the \Condor{submit} manual page on page~\pageref{man-condor-submit} for
more details.

\item Added \Term{buffer\_size} and \Term{buffer\_block\_size} so that STANDARD
universe jobs can specify that they wish to have I/O buffering turned on.
Without buffering, all I/O requests in the STANDARD universe are sent back
over the network to be executed on the submit machine.  
With buffering, read ahead, write behind, and seek batch buffering is
performed to minimize network traffic and latency.
By default, jobs do not specify buffering, however, for many situations buffering
can drastically increase throughput.  See the \Condor{submit} manual page
on page~\pageref{man-condor-submit} for more details.

\item The \Condor{schedd} is much more memory efficient handling clusters
with hundreds/thousands of jobs.  
If you submit large clusters, your submit machine will only use a fraction
of the amount of RAM it used to require.  
\Note The memory savings will only be realized for new clusters submitted
after the upgrade to v6.1.8 -- clusters which previously existed in the
queue at upgrade time will still use the same amount of RAM in the
\Condor{schedd}.

\item Submitting jobs, especially submitting large clusters containing many
jobs, is much faster.

\item Added a \Opt{-goodput} option to \Condor{q}, which displays
statistics about the execution efficiency of STANDARD universe jobs.

\item Added FS\_REMOTE method of user authentication to possible values
of the configuration option \Macro{AUTHENTICATION\_METHODS} to fix problems
with using the \Opt{-r} remote scheduler option of \Condor{submit}.
Additionally, the user authentication protocol has changed, so previous
versions of Condor programs cannot co-exist with this new protocol.

\item Added a new utility and documentation for \Condor{glidein} which uses 
Globus resources to extend your local pool to use remote Globus machines as 
part of your Condor pool.

\item Fixed more bugs in the handling of the stat() system call
and its relatives on Linux with glibc.
This was causing problems mainly with Fortran I/O, though other I/O
related problems on glibc Linux will probably be solved now.

\item Fixed a bug in various Condor tools (\Condor{status},
\Condor{user\_prio}, \Condor{config\_val}, and \Condor{stats}) that
would cause them to seg fault on bad input to the \Opt{-pool} option. 

\item Fixed a bug with the \Opt{-rset} option to \Condor{config\_val} which
could crash the Condor daemon whose configuration was being changed.

\item Added \Term{allow\_startup\_script} command to the job submit
description file which is given to \Condor{submit}.  This allows the
submission of a startup script to the STANDARD universe.  See 

\item Fixed a bug in the \Condor{schedd} where it would get into an
infinite loop if the persistant log of the job queue got corrupted.  
The \Condor{schedd} now correctly handles corrupted log files.

\item The full release tar file now contains a \File{dagman}
subdirectory in the \File{examples} directory.
This subdirectory includes an example DAGMan job, including a README
(in both ASCII and HTML), a Makefile, and so on.

\item Condor will now insert an environment variable, \Env{CONDOR\_VM}, into
the environment of the user job.  
This variable specifies which SMP ``virtual machine'' the job was started on.
It will equal either vm1, vm2, vm3, \Dots , depending upon which virtual
machine was matched.
On a non-SMP machine, \Env{CONDOR\_VM} will always be set to vm1.

\item Fixed some timing bugs introduced in v6.1.6 which could occur when
Condor tries to simultaneously start a large number of jobs submitted from a
single machine.

\item Fixed bugs when Condor is told to gracefully shutdown; Condor no
longer starts up new jobs when shutting down.  Also, the \Condor{schedd}
progressively checkpoints running jobs during a graceful shutdown instead of
trying to vacate all the job simultaneously.  The rate at which the shutdown
occurs is controlled by the \Macro{JOB\_START\_DELAY} configuration
parameter (see page~\pageref{param:JobStartDelay}).

\item Fixed a bug which could cause the \Condor{master} process to exit if
the Condor daemons have been hung for a while by the operating system (if,
for instance, the LOG directory was placed on an NFS volume and the NFS
server is down for an extended period).

\item Previously, removing a large number of jobs with \Condor{rm} would
result in the \Condor{schedd} being unresponsive for a period of time
(perhaps leading to timeouts when running \Condor{q}).  The \Condor{schedd}
has been improved to multitask the removal of jobs while servicing new
requests.

\item Added new configuration parameter \Macro{COLLECTOR\_SOCKET\_BUFSIZE}
which controls the size of TCP/IP buffers used by the \Condor{collector}.
For more info, see section~ref{param:CollectorSocketBufsize} on
page~pageref{param:CollectorSocketBufsize}.

\item Fixed a bug with the \Opt{-analyze} option to \Condor{q}: in some
cases, the RANK expression would not be evaluated correctly.  This could
cause the output from \Opt{-analyze} to be in error.

\item When running on a multi-CPU (SMP) Hewlett-Packard machine, fixed bugs
computing the system load average.

\item Fixed bug in \Condor{q} which could cause the RUN\_TIME reported to
be temporarily incorrect when jobs first start running. 

\item The \Condor{startd} no longer rapidly sends multiple ClassAds one
right after another to the Central Manager when its state/activity is in
rapid transition.  Also, on SMP machines, the \Condor{startd} will only send
updates for 4 nodes per second (to avoid overflowing the central manager when
reporting the state of a very large SMP machine with dozens of CPUs).

\item Reading a parameter with \Condor{config\_val} is now allowed from any
machine with Host-IP READ permission.
Previsouly, you needed ADMINISTRATOR permission.  
Of course, setting a parameter still requires ADMINISTRATOR permission.

\item Worked around a bug in the StreamTokenizer Java class from Sun
that we use in the CondorView client Java applet.
The bug would cause errors if usernames or hostnames in your pool
contained ``-'' or ``\_'' characters.
The CondorView applet now gets around this and properly displays all
data, including entries with the ``bad'' characters.

\end{itemize}

%%%%%%%%%%%%%%%%%%%%%%%%%%%%%%%%%%%%%%%%%%%%%%%%%%%%%%%%%%%%%%%%%%%%%%
\subsection*{\label{sec:New-6-1-7}Version 6.1.7}
%%%%%%%%%%%%%%%%%%%%%%%%%%%%%%%%%%%%%%%%%%%%%%%%%%%%%%%%%%%%%%%%%%%%%%

\Note Version 6.1.7 only adds support for platforms not supported in
6.1.6.  
There are no bug fixes, so there are no binaries released for any
other platforms. 
You do not need 6.1.7 unless you are using one of the two platforms we
released binaries for.

\begin{itemize}

\item Added ``clipped'' support for Alpha Linux machines running the
2.0.X kernel and glibc 2.0.X (such as Red Hat 5.X).
We do not yet support checkpointing and remote system calls on this
platform, but we can start ``vanilla'' jobs.
See section~\ref{sec:Choosing-Universe} on
page~\pageref{sec:Choosing-Universe} for details on vanilla
vs. standard jobs.

\item Re-added support for Intel Linux machines running the 2.0.X
Linux kernel, glibc 2.0.X, using the GNU C compiler (gcc/g++ 2.7.X) or
the EGCS compilers (versions 1.0.X, 1.1.1 and 1.1.2).
This includes Red Hat 5.X, and Debian 2.0.
\Bold{Red Hat 6.0 and Debian 2.1 are not yet supported, since they use
glibc 2.1.X and the 2.2.X Linux kernel.}
Future versions of Condor will support all combinations of kernels,
compilers and versions of libc.

\end{itemize}


%%%%%%%%%%%%%%%%%%%%%%%%%%%%%%%%%%%%%%%%%%%%%%%%%%%%%%%%%%%%%%%%%%%%%%
\subsection*{\label{sec:New-6-1-6}Version 6.1.6}
%%%%%%%%%%%%%%%%%%%%%%%%%%%%%%%%%%%%%%%%%%%%%%%%%%%%%%%%%%%%%%%%%%%%%%

\begin{itemize}

\item Added \Term{file\_remaps} as command in the job submit file given to
\Condor{submit}.
This allows the user to explicitly specify where to find a given file (e.g.
either on the submit or execute machine), as well as remap file access to a
different filename altogether.

\item Changed the way that \Condor{master} spawns daemons and
\Condor{preen} which allows you to specify command line arguments for
any of them, though a \MacroNI{SUBSYS\_ARGS} setting.
Previously, when you specified \Macro{PREEN}, you added the command
line arguments directly to that setting, but that caused some
problems, and only worked for \Condor{preen}.
\Bold{Once you upgrade to version 6.1.6, if you continue to use your
old \File{condor\_config} files, you must change the \Macro{PREEN}
setting to remove any arguments you have defined and place those
arguments into a separate config setting, \Macro{PREEN\_ARGS}.}
See section~\ref{sec:Master-Config-File-Entries}, ``\condor{master}
Config File Entries'', on
page~\pageref{sec:Master-Config-File-Entries} for more details.

\item Fixed a very serious bug in the Condor library linked in with
\Condor{compile} to create standard jobs that was causing
checkpointing to fail in many cases.  
Any jobs that were linked with the 6.1.5 Condor libraries should
probably be removed, re-linked, and re-submitted. 

\item Fixed a bug in \Condor{userprio} that was introduced in version
6.1.5 that was preventing it from finding the address of the
\Condor{negotiator} for your pool.

\item Fixed a bug in \Condor{stats} that was introduced in version
6.1.5 that was preventing it from finding the address of the
\Condor{collector} for your pool.

\item Fixed a bug in the way the \Opt{-pool} option was handled by
many Condor tools that was introduced in version 6.1.5. 


\item \Condor{q} now displays job \emph{allocation time} by default, instead
of displaying CPU time.  
Job allocation time, or RUN\_TIME, is the amount of wall-clock time the job
has spent running.  
Unlike CPU time information which is only updated when a job is
checkpointed, the allocation time displayed by \Condor{q} is continuously
updated, even for vanilla universe jobs.  
By default, the allocation time displayed will be the total time across all
runs of the job.  
The new \Opt{-currentrun} option to \Condor{q} can be used to display the
allocation time for solely the current run of the job.
Additionally, the \Opt{-cputime} option can be used to view job CPU times as
in earlier versions of Condor.

\item \Condor{q} will display an error message if there is a timeout
fetching the job queue listing from a \condor{schedd}.  Previously,
\Condor{q} would simply list the queue as empty upon a communication error.

\item The \condor{schedd} daemon has been updated to verify all queue access
requests via Condor's IP/Host-Based Security mechanism (see
section~\ref{sec:Host-Security}).

\item Fixed a bug on platforms which require the \Condor{kbdd} (currently
Digital Unix and IRIX).  
This bug could have allowed Condor to start a job within the first five
minutes after the Condor daemons had been started, even if there is a user
typing on the keyboard.

\item \Condor{release} now gives an error message if the user tries to
release a job which either does not exist or is not in the hold state.

\item Added a new config file parameter, \Macro{USER\_JOB\_WRAPPER}, which
allows administrators to specify a file to act as a ``wrapper'' script
around all jobs started by Condor. 
See inside section~\ref{param:UserJobWrapper}, on 
page~\pageref{sec:Starter-Config-File-Entries}, for more details.

\item \Condor{dagman} now permits the backslash character (``\Bs'') to be used
as a line-continuation character for DAG Input Files, just like the
\condor{config} files.

\item The Condor version string is now included in all Condor
libraries.
You can now run \Prog{ident} on any program linked with
\Condor{compile} to view which version of the Condor libraries you
linked with.
In addition, the format of the version string changed in 6.1.6.
Now, the identifier used is ``CondorVersion'' instead of ``Version''
to prevent any potential ambiguity.
Also, the format of the date changed slightly.

\item The SMP startd can now handle dynamic reconfiguration of the
number of each type of virtual machine being reported.
This allows you, during the normal running of the startd, to increase
or decrease the number of CPUs that Condor is using.
If you reconfigure the startd to use less CPUs than it currently has
under its control, it will first remove CPUs that have no Condor jobs
running on them.
If more CPUs need to be evicted, the startd will checkpoint jobs and
evict them in reverse rank order (using the startd's \Macro{Rank}
expression).
So, the lower the value of the rank, the more likely a job will be
kicked off.

\item The SMP startd contrib module's \Condor{starter} no longer makes
a call that was causing warning messages about ``ERROR: Unknown System
Call (-58) - system call not supported by Condor'' when used with the
6.0.X \Condor{shadow}.
This was a harmless call, but removing the call prevents the error
message.

\item The SMP contrib module now includes the \Condor{checkpoint} and
\Condor{vacate} programs, which allow you to vacate or checkpoint jobs
on individual CPUs on the SMP, instead of checkpointing or vacating
everything.  
You can now use ``\condor{vacate} vm1@hostname'' to just vacate the
first virtual machine, or ``\condor{vacate} hostname'' to vacate all
virtual machines. 

\item Added support for SMP Digital Unix (Alpha) machines.

\item Fixed a bug that was causing an overflow in the computation of
free disk and swap space on Digital Unix (Alpha) machines.

\item The \Condor{startd} and \Condor{schedd} now can ``invalidate''
their classads from the collector.
So, when a daemon is shut down, or a machine is reconfigured to 
advertise fewer virtual machines, those changes will be instantly
visible with \Condor{status}, instead of having to wait 15 minutes for
the stale classads to time-out.

\item The \Condor{schedd} no longer forks a child process (a ``schedd
agent'') to claim available \Condor{startd}s.  
You should no longer see multiple \condor{schedd} processes running on
your machine after a negotiation cycle.
This is now accomplished in a non-blocking manner within the
\Condor{schedd} itself.

\item The startd now adds an \Attr{VirtualMachineID} attribute to
each virtual machine classad it advertises.
This is just an integer, starting at 1, and increasing for every
different virtual machine the startd is representing.
On regular hosts, this is the only ID you will ever see.
On SMP hosts, you will see the ID climb up to the number of different
virtual machines reported.
This ID can be used to help write more complex policy expressions on
SMP hosts, and to easily identify which hosts in your pool are in fact
SMP machines.

\item Modified the output for \Condor{q} -run for scheduler and PVM
universe jobs.  The host where the scheduler universe job is running
is now displayed correctly.  For PVM jobs, a count of the current
number of hosts where the job is running is displayed.

\item Fixed the \Condor{startd} so that it no longer prints lots of
ProcAPI errors to the log file when it is being run as non-root.

\item \Macro{FS\_PATHNAME} and \Macro{VOS\_PATHNAME} are no longer
used.  AFS support now works similar to NFS support, via the
\Macro{FILESYSTEM\_DOMAIN} macro.

\item Fixed a minor bug in the \File{Condor.pm} perl module that was
causing it to be case-sensitive when parsing the Condor submit file.
Now, the perl module is properly case-insensitive, as indicated in the
documentation.

\end{itemize}

%%%%%%%%%%%%%%%%%%%%%%%%%%%%%%%%%%%%%%%%%%%%%%%%%%%%%%%%%%%%%%%%%%%%%%
\subsection*{\label{sec:New-6-1-5}Version 6.1.5}
%%%%%%%%%%%%%%%%%%%%%%%%%%%%%%%%%%%%%%%%%%%%%%%%%%%%%%%%%%%%%%%%%%%%%%

\begin{itemize}

\item Fixed a nasty bug in \Condor{preen} that would cause it to
remove files it shouldn't remove if the \Condor{schedd} and/or
\Condor{startd} were down at the time \Condor{preen} ran.
This was causing jobs to mysteriously disappear from the job queue.

\item Added preliminary support to Condor for running on machines with
multiple network interfaces.
On such machines, users can specify the IP address Condor should use
in the \Macro{NETWORK\_INTERFACE} config file parameter on each host. 
In addition, if the pool's central manager is on such a machine, users
should set the \Macro{CM\_IP\_ADDR} parameter to the ip address you wish
to use on that machine.
See section~\ref{sec:Multiple-Interfaces} on
page~\pageref{sec:Multiple-Interfaces} for more details.

\item The support for multiple network interfaces introduced bugs in
\Condor{userprio}, \Condor{stats}, CondorPVM, and the \Opt{-pool}
option to many Condor tools.
All of these will be fixed in version 6.1.6.

\item Fixed a bug in the remote system call library that was
preventing certain Fortran operations from working correctly on
Linux.  

\item The Linux binaries for GLIBC we now distribute are compiled on a
Red Hat 5.2 machine.
If you're using this version of Red Hat, you might have better luck
with the dynamically linked version of Condor than previous releases
of Condor.
Sites using other GLIBC Linux distributions should continue to use the
statically linked version of Condor.

\item Fixed a bug in the \Condor{shadow} that could cause it to die
with signal 11 (segmentation violation) under certain rare
circumstances. 

\item Fixed a bug in the \Condor{schedd} that could cause it to die
with signal 11 (segmentation violation) under certain rare
circumstances. 

\item Fixed a bug in the \Condor{negotiator} that could cause it to
die with signal 8 (floating point exception) on Digital Unix
machines. 

\item The following shadow parameters have been added to control
checkpointing: \Macro{COMPRESS\_PERIODIC\_CKPT},
\Macro{COMPRESS\_VACATE\_CKPT}, \Macro{PERIODIC\_MEMORY\_SYNC},
\Macro{SLOW\_CKPT\_SPEED}.  See
section~\ref{sec:Shadow-Config-File-Entries} on
page~\pageref{sec:Shadow-Config-File-Entries} for more details.
In addition, the shadow now honors the \Attr{CkptWanted} flag in a job
classad, and if it is set to ``False'', the job will never
checkpoint.

\item Fixed a bug in the \Condor{startd} that could cause it to
report negative values for the CondorLoadAvg on rare occasions. 

\item Fixed a bug in the \Condor{startd} that could cause it to die
with a fatal exception in situations where the act of getting claimed
by a remote schedd failed for some reason.  
This resulted in the \Condor{startd} exiting on rare occasions with a
message in its log file to the effect of \texttt{ERROR ``Match timed
out but not in matched state''}.

\item Fixed a bug in the \Condor{schedd} that under rare circumstances
could cause a job to be left in the ``Running'' state even after the
\Condor{shadow} for that job had exited.

\item Fixed a bug in the \Condor{schedd} and various tools that
prevented remote read-only access to the job queue from working.
So, for example, \texttt{condor\_q -name foo}, if run on any machine
other than foo, wouldn't display any jobs from foo's queue. 
This fix re-enables the following options to \Condor{q} to work:
\Opt{submitter}, \Opt{name}, \Opt{global}, etc.

\item Changed the \Condor{schedd} so that when starting jobs, it
always sorts on the cluster number, in addition to the date the jobs
were enqueued and the process number within clusters, so that if many
clusters were submitted at the same time, the jobs are started in
order.

\item Fixed a bug in \Condor{compile} that was modifying the
\Env{PATH} environment variable by adding things to the front of it.
This would potentially cause jobs to be compiled and linked with a
different version of a compiler than they thought they were getting.  

\item Minor change in the way the \Condor{startd} handles the
\Dflag{LOAD} and \Dflag{KEYBOARD} debug flags.  
Now, each one, when set, will only display every
\Macro{UPDATE\_INTERVAL}, regardless of the startd state.
If you wish to see the values for keyboard activity or load average
every \Macro{POLLING\_INTERVAL}, you must enable \Dflag{FULLDEBUG}. 

\end{itemize}

%%%%%%%%%%%%%%%%%%%%%%%%%%%%%%%%%%%%%%%%%%%%%%%%%%%%%%%%%%%%%%%%%%%%%%
\subsection*{\label{sec:New-6-1-4}Version 6.1.4}
%%%%%%%%%%%%%%%%%%%%%%%%%%%%%%%%%%%%%%%%%%%%%%%%%%%%%%%%%%%%%%%%%%%%%%

\begin{itemize}

\item Fixed a bug in the socket communication library used by Condor
that was causing daemons and tools to die on some platforms (notably,
Digital Unix) with signal 8, SIGFPE (floating point exception).

\item Fixed a bug in the usage message of many Condor tools that
mentioned a \Opt{-all} option that isn't yet supported. 
This option will be supported in future versions of Condor.

\item Fixed a bug in the filesystem authentication code used to
authenticate operations on the job queue that left empty temporary
files in /tmp.  
These files are now properly removed after they are used.

\item Fixed a minor bug in the totals \Condor{status} displays when
you use the \Opt{ckptsrvr} option.

\item Fixed a minor syntax error in the \Condor{install} script that
would cause warnings.

\item the \File{Condor.pm} Perl module is now included in the
\File{lib} directory of the main release directory.

\end{itemize}

%%%%%%%%%%%%%%%%%%%%%%%%%%%%%%%%%%%%%%%%%%%%%%%%%%%%%%%%%%%%%%%%%%%%%%
\subsection*{\label{sec:New-6-1-3}Version 6.1.3}
%%%%%%%%%%%%%%%%%%%%%%%%%%%%%%%%%%%%%%%%%%%%%%%%%%%%%%%%%%%%%%%%%%%%%%

\Note There are a lot of new, unstable features in 6.1.3.  
PLEASE do not install all of 6.1.3 on a production pool.
Almost all of the bug fixes in 6.1.3 are in the \Condor{startd} or
\Condor{starter}, so, unless you really know what you're doing, we
recommend you just upgrade SMP-Startd contrib module, not the entire
6.1.3 release. 

\begin{itemize}

\item Owners can now specify how the SMP-Startd partitions the system
resources into the different types and numbers of virtual machines,
specifying the number of CPUs, megs of RAM, megs of swap space, etc.,
in each.
Previously, each virtual machine reported to Condor from an SMP
machine always had one CPU, and all shared system resources were
evenly divided among the virtual machines.

\item Fixed a bug in the reporting of virtual memory and disk space on
SMP machines where each virtual machine represented was advertising
the total in the system for itself, instead of its own share.
Now, both the totals, and the virtual machine-specific values are
advertised.  

\item Fixed a bug in the \Condor{starter} when it was trying to
suspend jobs.
While we always killed all of the processes when we were trying to
vacate, if a vanilla job forked, the starter would sometimes not
suspend some of the children processes.
In addition, we could sometimes miss a standard universe job for
suspending as well.
This is all fixed.

\item Fixed a bug in the SMP-Startd's load average computation that
could cause processes spawned by Condor to not be associated w/ the
Condor load average.
This would cause the startd to over-estimate the owner's load average,
and under-estimate the Condor load, which would cause a cycle of
suspending and resuming a Condor job, instead of just letting it run.

\item Fixed a bug in the SMP-Startd's load average computation that
could cause certain rare exceptions to be treated as fatal, when in
fact, the Startd could recover from them.

\item Fixed a bug in the computation of the total physical memory on
some platforms that was resulting in an overflow on machines with
lots of ram (over 1 gigabyte).

\item Fixed some bugs that could cause \Condor{starter} processes to
be left as zombies underneath the \Condor{startd} under very rare
conditions.  

\item For sites using AFS, if there are problems in the
\Condor{startd} computing the AFS cell of the machine it's running on,
the startd will exit with an error message at start-up time.

\item Fixed a minor bug in \Condor{install} that would lead to a
syntax error in your config file given a certain set of installation
options.  

\item Added the \Opt{-maxjobs} option to the \Condor{submit\_dag}
script that can be used to specify the maximum number of jobs Condor
will run from a DAG at any given time.
Also, \Condor{submit\_dag} automatically creates a ``rescue DAG''.
See section~\ref{sec:DAGMan} on page~\pageref{sec:DAGMan} for details
on DAGMan.

\item Fixed bug in ClassAd printing when you tried to display an
integer or float attribute that didn't exist in the given ClassAd. 
This could show up in \Condor{status}, \Condor{q}, \Condor{history},
etc. 

\item Various commands sent to the Condor daemons now have separate
debug levels associated with them.
For example, commands such as ``keep-alives'', and the command sent
from the \Condor{kbdd} to the \Condor{startd} are only seen in the
various log files if \Dflag{FULLDEBUG} is turned on, instead of
\Dflag{COMMAND}, which the default and now enabled for all daemons on
all platforms by default.
Administrators retaining their old configuration when upgrading to
this version are encouraged to enable \Dflag{COMMAND} in the
\Macro{SCHEDD\_DEBUG} setting.  
In addition, for IRIX and Digital Unix machines, it should be enabled
in the \Macro{STARTD\_DEBUG} setting as well.
See section~\ref{sec:Daemon-Logging-Config-File-Entries} on
page~\pageref{sec:Daemon-Logging-Config-File-Entries} for details on
debug levels in Condor.

\item New debug levels added to Condor: 
\begin{itemize}
\item \Dflag{NETWORK}, used by various daemons in Condor to report
various network statistics about the Condor daemons. 
\item \Dflag{PROCFAMILY}, used to report information about various
families of processes that are monitored by Condor.
For example, this is used in the \Condor{startd} when monitoring the
family of processes spawned by a given user job for the purposes of
computing the Condor load average.
\item \Dflag{KEYBOARD}, used by the \Condor{startd} to print out
statistics about remote tty and console idle times in the
\Condor{startd}.
This information used to be logged at \Dflag{FULLDEBUG}, along with
everything else, so now, you can see just the idle times, and/or have
the information stored to a separate file.
\end{itemize}

\item Added a \Opt{-run} option to \Condor{q}, which displays
information for running jobs, including the remote host where each job
is running.

\item Macros can now be incrementally defined.  See
section~\ref{sec:Config-File-Macros} on
page~\pageref{sec:Config-File-Macros} for more details.

\item \Condor{config\_val} can now be used to set configuration
variables.  See the man page on page~\pageref{man-condor-config-val}
for more details.

\item The job log file now contains a record of network activity.  The
evict, terminate, and shadow exception events indicate the number of
bytes sent and received by the job for the specific run.  
The terminate event additionally indicates totals for the life of the
job.

\item \Macro{STARTER\_CHOOSES\_CKPT\_SERVER} now defaults to true.
See section~\ref{param:StarterChoosesCkptServer} on
page~\pageref{param:StarterChoosesCkptServer} for more details.

\item The infrastructure for authentication within Condor has been
overhauled, allowing for much greater flexibility in supporting new
forms of authentication in the future.
This means that the 6.1.3 schedd and queue management tools (like
\Condor{q}, \Condor{submit}, \Condor{rm} and so on) are incompatible
with previous versions of Condor.

\item Many of the Condor administration tools have been improved to
allow you to specify the ``subsystem'' you want them to effect.  
For example, you can now use ``\condor{reconfig} -startd'' to just
have the startd reconfigure itself.
Similarly, \condor{off}, \condor{on} and \condor{restart} can now all 
work on a single daemon, instead of machine-wide.
See the man pages (section~\ref{sec:command-reference} on
page~\pageref{sec:command-reference}) or run any command with \Opt{-help}
for details. 
\Note The usage message in 6.1.3 incorrectly reports \Opt{-all} as a
valid option.

\item Fixed a bug in the Condor tools that could cause a segmentation
violation in certain rare error conditions.

\end{itemize}

%%%%%%%%%%%%%%%%%%%%%%%%%%%%%%%%%%%%%%%%%%%%%%%%%%%%%%%%%%%%%%%%%%%%%%
\subsection*{\label{sec:New-6-1-2}Version 6.1.2}
%%%%%%%%%%%%%%%%%%%%%%%%%%%%%%%%%%%%%%%%%%%%%%%%%%%%%%%%%%%%%%%%%%%%%%

\begin{itemize}

\item Fixed some bugs in the \Condor{install} script.
Also, enhanced \Condor{install} to customize the path to perl in
various perl scripts used by Condor.

\item Fixed a problem with our build environment that left some files
out of the \File{release.tar} files in the binary releases on some
platforms. 

\item \Condor{dagman}, ``DAGMan'' (see section~\ref{sec:DAGMan} on 
page~\pageref{sec:DAGMan} for details) is now included in the
development release by default.

\item Fixed a bug in the computation of the total physical memory in
HPUX machines that was resulting in an overflow on machines with
lots of ram (over 1 gigabyte).
Also, if you define ``MEMORY'' in your config file, that value will
override whatever value Condor computes for your machine.

\item Fixed a bug in \Condor{starter.pvm}, the PVM version of the
Condor starter (available as an optional ``Contrib module''), when you
disabled \Macro{STARTER\_LOCAL\_LOGGING}.
Now, having this set to ``False'' will properly place debug messages
from \Condor{starter.pvm} into the \File{ShadowLog} file of the
machine that submitted the job (as opposed to the \File{StarterLog}
file on the machine executing the job).  

\end{itemize}


%%%%%%%%%%%%%%%%%%%%%%%%%%%%%%%%%%%%%%%%%%%%%%%%%%%%%%%%%%%%%%%%%%%%%%
\subsection*{\label{sec:New-6-1-1}Version 6.1.1}
%%%%%%%%%%%%%%%%%%%%%%%%%%%%%%%%%%%%%%%%%%%%%%%%%%%%%%%%%%%%%%%%%%%%%%

\begin{itemize}

\item Fixed a bug in the \Condor{startd} where we compute the load
average caused by Condor that was causing us to get the wrong values.
This could cause a cycle of continuous job suspends and job resumes.

\item Beginning with this version, any jobs linked with the Condor
checkpoint libraries will use the zlib compression code (used by gzip
and others) to compress periodic checkpoints before they are written
to the network.  
These compressed checkpoints are uncompressed at startup time.  
This saves network bandwidth, disk space, as well as time (if the
network is the bottleneck to checkpointing, which it usually is). 
In future versions of Condor, all checkpoints will probably be
compressed, but at this time, it is only used for periodic
checkpoints.  
Note, you have to relink your jobs with the \Condor{compile} command
to have this feature enabled.
Old jobs (not relinked) will continue to run just fine, they just
won't be compressed.

\item \Condor{status} now has better support for displaying checkpoint
server ClassAds. 

\item More contrib modules from the development series are now
available, such as the checkpoint server, PVM support, and the
CondorView server.  

\item Fixed some minor bugs in the UserLog code that were causing
problems for DAGMan in exceptional error cases.

\item Fixed an obscure bug in the logging code when \Dflag{PRIV} was
enabled that could result in incorrect file permissions on log files. 

\end{itemize}

%%%%%%%%%%%%%%%%%%%%%%%%%%%%%%%%%%%%%%%%%%%%%%%%%%%%%%%%%%%%%%%%%%%%%%
\subsection*{\label{sec:New-6-1-0}Version 6.1.0}
%%%%%%%%%%%%%%%%%%%%%%%%%%%%%%%%%%%%%%%%%%%%%%%%%%%%%%%%%%%%%%%%%%%%%%

\begin{itemize}

\item Support has been added to the \condor{startd} to run multiple
jobs on SMP machines.
See section~\ref{sec:Configuring-SMP} on
page~\pageref{sec:Configuring-SMP} for details about setting up and
configuring SMP support.

\item The expressions that control the \condor{startd} policy for
vacating, jobs has been simplified.
See section~\ref{sec:Configuring-Policy} on
page~\pageref{sec:Configuring-Policy} for complete details on the new
policy expressions, and section~\ref{sec:V60-Policy-diffs} on
page~\pageref{sec:V60-Policy-diffs} for an explanation of what's
different from the version 6.0 expressions.

\item We now perform better tracking of processes spawned by Condor.
If children die and are inherited by init, we still know they belong
to Condor.
This allows us to better ensure we don't leave processes lying around
when we need to get off a machine, and enables us to have a much more
accurate computation of the load average generated by Condor (the
\Attr{CondorLoadAvg} as reported by the \Condor{startd}). 

\item The \condor{collector} now can store historical information
about your pool state.
This information can be queried with the \Condor{stats} program (see
the man page on page~\pageref{man-condor-stats}), which is used by the
\Condor{view} Java GUI, which is available as a separate contrib
module.

\item Condor jobs can now be put in a ``hold'' state with the
\Condor{hold} command.
Such jobs remain in the job queue (and can be viewed with \Condor{q}),
but there will not be any negotiation to find machines for them.
If a job is having a temporary problem (like the permissions are 
wrong on files it needs to access), the job can be put on hold until
the problem can be solved.
Jobs put on hold can be released with the \Condor{release} command.

\item \condor{userprio} now has the notion of \Term{user factors} as a
way to create different groups of users in different priority levels.
See section~\ref{sec:UserPrio} on page~\pageref{sec:UserPrio} for
details.
This includes the ability to specify a local priority domain, and all
users from other domains get a much worse priority.

\item Usage statistics by user is now available from
\condor{userprio}.
See the man page on page~\pageref{man-condor-userprio} for details. 

\item The \condor{schedd} has been enhanced to enable ``flocking'',
where it seeks matches with machines in multiple pools if its requests
cannot be serviced in the local pool.
See section~\ref{sec:Flocking} on page~\pageref{sec:Flocking} for more
details.

\item The \condor{schedd} has been enhanced to enable \condor{q} and
other interactive tools better response time.

\item The \condor{schedd} has also been enhanced to allow it to check
the permissions of the files you specify for input, output, error and
so on.  
If the schedd doesn't have the required access rights to the files,
the jobs will not be submitted, and \Condor{submit} will print an
error message.

\item When you perform a \Condor{rm} command, and the job you removed
was using a ``user log'', the remove event is now recorded into the
log. 

\item Two new attributes have been added to the job classad when it 
begins executing: \Attr{RemoteHost} and \Attr{LastRemoteHost}.
These attributes list the IP address and port of the startd that is
either currently running the job, or the last startd to run the job
(if it's run on more than one machine). 
This information helps users track their job's execution more closely,
and allows administrators to troubleshoot problems more effectively. 

\item The performance of checkpointing was increased by using larger
buffers for the network I/O used to get the checkpoint file on and off
the remote executing host (this helps for all pools, with or without
checkpoint servers). 

\end{itemize}


%%%%%%%%%%%%%%%%%%%%%%%%%%%%%%%%%%%%%%%%%%%%%%%%%%%%%%%%%%%%%%%%%%%%%%%
\section{\label{sec:History-6-0}Stable Release Series 6.0}
%%%%%%%%%%%%%%%%%%%%%%%%%%%%%%%%%%%%%%%%%%%%%%%%%%%%%%%%%%%%%%%%%%%%%%

6.0 is the first version of Condor with \Term{ClassAds}.
It contains many other fundamental enhancements over version 5.
It is also the first official stable release series, with a
development series (6.1) simultaneously available.


%%%%%%%%%%%%%%%%%%%%%%%%%%%%%%%%%%%%%%%%%%%%%%%%%%%%%%%%%%%%%%%%%%%%%%
\subsection*{\label{sec:New-6-0-3}Version 6.0.3}
%%%%%%%%%%%%%%%%%%%%%%%%%%%%%%%%%%%%%%%%%%%%%%%%%%%%%%%%%%%%%%%%%%%%%%

\begin{itemize}

\item Fixed a bug that was causing the hostname of the submit machine
that claimed a given execute machine to be incorrectly reported by the
\Condor{startd} at sites using NIS.

\item Fixed a bug in the \Condor{startd}'s benchmarking code that
could cause a floating point exception (SIGFPE, signal 8) on very,
very fast machines, such as newer Alphas.

\item Fixed an obscure bug in \Condor{submit} that could happen when
you set a requirements expression that references the ``Memory''
attribute.
The bug only showed up with certain formations of the requirement
expression.

\end{itemize}


%%%%%%%%%%%%%%%%%%%%%%%%%%%%%%%%%%%%%%%%%%%%%%%%%%%%%%%%%%%%%%%%%%%%%%
\subsection*{\label{sec:New-6-0-2}Version 6.0.2}
%%%%%%%%%%%%%%%%%%%%%%%%%%%%%%%%%%%%%%%%%%%%%%%%%%%%%%%%%%%%%%%%%%%%%%

\begin{itemize}

\item Fixed a bug in the \Syscall{fcntl} call for Solaris 2.6 that was
causing problems with file I/O inside Fortran jobs.

\item Fixed a bug in the way the \Macro{DEFAULT\_DOMAIN\_NAME}
parameter was handled so that this feature now works properly.  

\item Fixed a bug in how the \Macro{SOFT\_UID\_DOMAIN} config file
parameter was used in the \Condor{starter}.
This feature is also documented in the manual now (see
section~\ref{param:SoftUidDomain} on
page~\pageref{param:SoftUidDomain}).

\item You can now set the RunBenchmarks expression to ``False'' and
the \Condor{startd} will never run benchmarks, not even at startup
time. 

\item Fixed a bug in \Syscall{getwd} and \Syscall{getcwd} for sites
that use the NFS automounter.
his bug was only present if user programs tried to call
\Syscall{chdir} themselves.
Now, this is supported. 

\item Fixed a bug in the way we were computing the available virtual
memory on HPUX 10.20 machines.

\item Fixed a bug in \Condor{q} -analyze so it will correctly identify
more situations where a job won't run.

\item Fixed a bug in \Condor{status} -format so that if the requested 
attribute isn't available for a given machine, the format string
(including spaces, tabs, newlines, etc) is still printed, just the
value for the requested attribute will be an empty string. 

\item Fixed a bug in the \Condor{schedd} that was causing
\Condor{history} to not print out the first ClassAd attribute of all
jobs that have completed

\item Fixed a bug in \Condor{q} that would cause a segmentation fault
if the argument list was too long.

\end{itemize}

%%%%%%%%%%%%%%%%%%%%%%%%%%%%%%%%%%%%%%%%%%%%%%%%%%%%%%%%%%%%%%%%%%%%%%
\subsection*{\label{sec:New-6-0-1}Version 6.0.1}
%%%%%%%%%%%%%%%%%%%%%%%%%%%%%%%%%%%%%%%%%%%%%%%%%%%%%%%%%%%%%%%%%%%%%%

\begin{itemize}

\item Fixed bugs in the \Syscall{getuid}), \Syscall{getgid},
\Syscall{geteuid}, and \Syscall{getegid} system calls. 

\item Multiple config files are now supported as a list specified via
the \Macro{LOCAL\_CONFIG\_FILE} variable. 

\item \Macro{ARCH} and \Macro{OPSYS} are now automatically determined
on all machines (including HPUX 10 and Solaris). 

\item Machines running IRIX now correctly suspend vanilla jobs.

\item \Condor{submit} doesn't allow root to submit jobs.

\item The \Condor{startd} now notices if you have changed
\Macro{COLLECTOR\_HOST} on reconfig.

\item Physical memory is now correctly reported on Digital Unix when
daemons are not running as root. 

\item New \MacroU{SUBSYSTEM} macro in configuration files that changes
based on which daemon is reading the file (i.e. STARTD, SCHEDD, etc.) 
See section~\ref{sec:Condor-Subsystem-Names}, ``Condor Subsystem
Names'' on page~\pageref{sec:Condor-Subsystem-Names} for a complete
list of the subsystem names used in Condor.

\item Port to HP-UX 10.20.  

\item \Syscall{getrusage} is now a supported system call.  
This system call will allow you to get resource usage about the entire
history of your condor job.

\item Condor is now fully supported on Solaris 2.6 machines (both
Sparc and Intel). 

\item Condor now works on Linux machines with the GNU C library.  
This includes machines running Red Hat 5.x and Debian 2.0. 
In addition, there seems to be a bug in Red Hat that was causing the
output from \Condor{config\_val} to not appear when used in scripts
(like \Condor{compile}).
We put in explicit calls to flush the I/O buffers before
\Condor{config\_val} exits, which seems to solve the problem.

\item Hooks have been added to the checkpointing library to help
support the checkpointing of PVM jobs.

\item Condor jobs can now send signals to themselves when running in
the standard universe.
You do this just as you normally would:
\begin{verbatim}
        kill( getpid(), signal_number )
\end{verbatim}
Trying to send a signal to any other process will result in
\Syscall{kill} returning -1.

\item Support for NIS has been improved on Digital Unix and IRIX.

\item Fixed a bug that would cause the negotiator on IRIX machines to
never match jobs with available machines.  

\end{itemize}

%%%%%%%%%%%%%%%%%%%%%%%%%%%%%%%%%%%%%%%%%%%%%%%%%%%%%%%%%%%%%%%%%%%%%%
\subsection*{\label{sec:New-6-0-pl4}Version 6.0 pl4}
%%%%%%%%%%%%%%%%%%%%%%%%%%%%%%%%%%%%%%%%%%%%%%%%%%%%%%%%%%%%%%%%%%%%%%

\Note Back in the bad old days, we used this evil ``patch level''
version number scheme, with versions like ``6.0pl4''.
This has all gone away in the current versions of Condor. 

\begin{itemize}

\item Fixed a bug that could cause a segmentation violation in the 
\Condor{schedd} under rare conditions when a \Condor{shadow} exited.

\item Fixed a bug that was preventing any core files that user jobs
submitted to Condor might create from being transferred back to the
submit machine for inspection by the user who submitted them.

\item Fixed a bug that would cause some Condor daemons to go into an
infinite loop if the "ps" command output duplicate entries.
This only happens on certain platforms, and even then, only under rare
conditions.
However, the bug has been fixed and Condor now handles this case
properly.

\item Fixed a bug in the \Condor{shadow} that would cause a
segmentation violation if there was a problem writing to the user log
file specified by "log = filename" in the submit file used with
\Condor{submit}.

\item Added new command line arguments for the Condor daemons to support
saving the PID (process id) of the given daemon to a file, sending a
signal to the PID specified in a given file, and overriding what
directory is used for logging for a given daemon.
These are primarily for use with the \Condor{kbdd} when it needs to be
started by XDM for the user logged onto the console, instead of
running as root.
See section~\ref{sec:kbdd} on ``Installing the \Condor{kbdd}'' on
page~\pageref{sec:kbdd} for details.

\item Added support for the \Macro{CREATE\_CORE\_FILES} config file
parameter.  
If this setting is defined, Condor will override whatever limits you
have set and in the case of a fatal error, will either create core
files or not depending on the value you specify ("true" or "false").

\item Most Condor tools (\Condor{on}, \Condor{off},
\Condor{master\_off}, \Condor{restart}, \Condor{vacate},
\Condor{checkpoint}, \Condor{reconfig}, \Condor{reconfig\_schedd},
\Condor{reschedule}) can now take the IP address and port you want to
send the command to directly on the command line, instead of only
accepting hostnames. 
This IP/port must be passed in a special format used in Condor (which
you will see in the daemon's log files, etc).
It is of the form: \Sinful{ip.address:port}.  
For example: \Sinful{123.456.789.123:4567}.

\end{itemize}

%%%%%%%%%%%%%%%%%%%%%%%%%%%%%%%%%%%%%%%%%%%%%%%%%%%%%%%%%%%%%%%%%%%%%%
\subsection*{\label{sec:New-6-0-pl3}Version 6.0 pl3}
%%%%%%%%%%%%%%%%%%%%%%%%%%%%%%%%%%%%%%%%%%%%%%%%%%%%%%%%%%%%%%%%%%%%%%

\begin{itemize}

\item Fixed a bug that would cause a segmentation violation if a
machine was not configured with a full hostname as either the official
hostname or as any of the hostname aliases.

\item If your host information does not include a fully qualified
hostname anywhere, you can specify a domain in the
\Macro{DEFAULT\_DOMAIN\_NAME} parameter in your global config file
which will be appended to your hostname whenever Condor needs to use a
fully qualified name.

\item All Condor daemons and most tools now support a "-version"
option that displays the version information and exits.

\item The \Condor{install} script now prompts for a short description
of your pool, which it stores in your central manager's local config
file as \Macro{COLLECTOR\_NAME}.
This description is used to display the name of your pool when sending
information to the Condor developers.

\item When the \Condor{shadow} process starts up, if it is configured
to use a checkpoint server and it cannot connect to the server, the
shadow will check the \Macro{MAX\_DISCARDED\_RUN\_TIME} parameter.  
If the job in question has accumulated more CPU minutes than this
parameter, the \Condor{shadow} will keep trying to connect to the
checkpoint server until it is successful.
Otherwise, the \Condor{shadow} will just start the job over from
scratch immediately.

\item If Condor is configured to use a checkpoint server, it will only
use the checkpoint server.
Previously, if there was a problem connecting to the checkpoint
server, Condor would fall back to using the submit machine to store
checkpoints.
However, this caused problems with local disks filling up on machines
without much disk space.

\item Fixed a rare race condition that could cause a segmentation
violation if a Condor daemon or tool opened a socket to a daemon and
then closed it right away.

\item All TCP sockets in Condor now have the "keep alive" socket option
enabled.
This allows Condor daemons to notice if their peer goes away in a hard
crash.

\item Fixed a bug that could cause the \Condor{schedd} to kill jobs
without a checkpoint during its graceful shutdown method under certain
conditions.

\item The \Condor{schedd} now supports the
\Macro{MAX\_SHADOW\_EXCEPTIONS} parameter.
If the \Condor{shadow} processes for a given match die due to a fatal
error (an exception) more than this number of times, the
\Condor{schedd} will now relinquish that match and stop trying to
spawn \Condor{shadow} processes for it.

\item The "-master" option to \Condor{status} now displays the \Attr{Name}
attribute of all \Condor{master} daemons in your pool, as opposed
to the \Attr{Machine} attribute.
This helps for pools that have submit-only machines joining them, for
example.

\end{itemize}

%%%%%%%%%%%%%%%%%%%%%%%%%%%%%%%%%%%%%%%%%%%%%%%%%%%%%%%%%%%%%%%%%%%%%%
\subsection*{\label{sec:New-6-0-pl2}Version 6.0 pl2}
%%%%%%%%%%%%%%%%%%%%%%%%%%%%%%%%%%%%%%%%%%%%%%%%%%%%%%%%%%%%%%%%%%%%%%

\begin{itemize}

\item In patch level 1, code was added to more accurately find the
full hostname of the local machine.
Part of this code relied on the resolver, which on many platforms is a
dynamic library.
On Solaris, this library has needed many security patches and the
installation of Solaris on our development machines produced binaries
that are incompatible with sites that haven't applied all the security
patches.
So, the code in Condor that relies on this library was simply removed
for Solaris.

\item Version information is now built into Condor.
You can see the \Attr{CondorVersion} attribute in every daemon's
ClassAd. 
You can also run the UNIX command "ident" on any Condor binary to see
the version. 

\item Fixed a bug in the "remote submit" mode of \Condor{submit}.
The remote submit wasn't connecting to the specified schedd, but was
instead trying to connect to the local schedd.

\item Fixed a bug in the \Condor{schedd} that could cause it to exit
with an error due to its log file being locked improperly under
certain rare circumstances.

\end{itemize}

%%%%%%%%%%%%%%%%%%%%%%%%%%%%%%%%%%%%%%%%%%%%%%%%%%%%%%%%%%%%%%%%%%%%%%
\subsection*{\label{sec:New-6-0-pl1}Version 6.0 pl1}
%%%%%%%%%%%%%%%%%%%%%%%%%%%%%%%%%%%%%%%%%%%%%%%%%%%%%%%%%%%%%%%%%%%%%%

\begin{itemize}

\item \Condor{kbdd} bug patched: On Silicon Graphics and DEC Alpha
ports, if your X11 server is using Xauthority user authentication, and
the \Condor{kbdd} was unable to read the user's \File{.Xauthority}
file for some reason, the \Condor{kbdd} would fall into an infinite 
loop.

\item When using a Condor Checkpoint Server, the protocol between the
Checkpoint Server and the \Condor{schedd} has been made more robust
for a faulty network connection. Specifically, this improves
reliability when submitting jobs across the Internet and using a
remote Checkpoint Server.

\item Fixed a bug concerning \Macro{MAX\_JOBS\_RUNNING}: The parameter
\MacroNI{MAX\_JOBS\_RUNNING} in the config file controls the maximum
number of simultaneous \Condor{shadow} processes allowed on your
submission machine.
The bug was the number of shadow processes could, under certain
conditions, exceed the number specified by
\MacroNI{MAX\_JOBS\_RUNNING}. 

\item Added new parameter \Macro{JOB\_RENICE\_INCREMENT} that can be
specified in the config file.
This parameter specifies the UNIX nice level that the \Condor{starter}
will start the user job.
It works just like the \Cmd{renice}{1} command in UNIX. 
Can be any integer between 1 and 19; a value of 19 is the lowest
possible priority.

\item Improved response time for \Condor{userprio}.

\item Fixed a bug that caused periodic checkpoints to happen more
often than specified.

\item Fixed some bugs in the installation procedure for certain
environments that weren't handled properly, and made the documentation
for the installation procedure more clear.

\item Fixed a bug on IRIX that could allow vanilla jobs to be started
as root under certain conditions.
This was caused by the non-standard uid that user "nobody" has on
IRIX.
Thanks to Chris Lindsey at NCSA for help discovering this bug.

\item On machines where the \File{/etc/hosts} file is misconfigured to
list just the hostname first, then the full hostname as an alias,
Condor now correctly finds the full hostname anyway.

\item The local config file and local root config file are now only
found by the files listed in the \Macro{LOCAL\_CONFIG\_FILE} and
\Macro{LOCAL\_ROOT\_CONFIG\_FILE} parameters in the global config
files.
Previously, \File{/etc/condor} and user condor's home directory
(\~condor) were searched as well.
This could cause problems with submit-only installations of Condor at
a site that already had Condor installed.

\end{itemize}

%%%%%%%%%%%%%%%%%%%%%%%%%%%%%%%%%%%%%%%%%%%%%%%%%%%%%%%%%%%%%%%%%%%%%%
\subsection*{\label{sec:New-6-0-pl0}Version 6.0 pl0}
%%%%%%%%%%%%%%%%%%%%%%%%%%%%%%%%%%%%%%%%%%%%%%%%%%%%%%%%%%%%%%%%%%%%%%

\begin{itemize}

\item Initial Version 6.0 release.

\end{itemize}

