%%%%%%%%%%%%%%%%%%%%%%%%%%%%%%%%%%%%%%%%%%%%%%%%%%%%%%%%%%%%%%%%%%%%%%
\section{\label{sec:History-Intro}Introduction to Condor Versions}
%%%%%%%%%%%%%%%%%%%%%%%%%%%%%%%%%%%%%%%%%%%%%%%%%%%%%%%%%%%%%%%%%%%%%%

This chapter provides descriptions of what features have been added or
bugs fixed for each version of Condor.
The first section describes the Condor version numbering scheme, what
the numbers mean, and what the different \Term{release series} are.
The rest of the sections each describe a specific release series, and
all the Condor versions found in that series.

%%%%%%%%%%%%%%%%%%%%%%%%%%%%%%%%%%%%%%%%%%%%%%%%%%%%%%%%%%%%%%%%%%%%%%
\subsection{\label{sec:Version-Number-Scheme}
Condor Version Number Scheme}
%%%%%%%%%%%%%%%%%%%%%%%%%%%%%%%%%%%%%%%%%%%%%%%%%%%%%%%%%%%%%%%%%%%%%%

Starting with version 6.0.1, Condor adopted a new, hopefully easy to
understand version numbering scheme.
It reflects the fact that Condor is both a production system and a
research project.
The numbering scheme was primarily taken from the Linux kernel's
version numbering, so if you are familiar with that, it should seem
quite natural.

There will usually be two Condor versions available at any given time,
the \Term{stable} version, and the \Term{development} version.
Gone are the days of ``patch level 3'', ``beta2'', or any other random
words in the version string.
All versions of Condor now have exactly three numbers, seperated by
``.''   

\begin{itemize}

\item The first number represents the major version number, and will
change very infrequently.

\item \emph{The thing that determines whether a version of Condor is
``stable'' or ``development'' is the second digit.
Even numbers represent stable versions, while odd numbers represent
development versions.}

\item The final digit represents the minor version number, which
defines a particular version in a given release series.

\end{itemize}


%%%%%%%%%%%%%%%%%%%%%%%%%%%%%%%%%%%%%%%%%%%%%%%%%%%%%%%%%%%%%%%%%%%%%%
\subsection{\label{sec:Stable-Series}The Stable Release Series}
%%%%%%%%%%%%%%%%%%%%%%%%%%%%%%%%%%%%%%%%%%%%%%%%%%%%%%%%%%%%%%%%%%%%%%

People expecting the stable, production Condor system should download
the stable version, denoted with an even number in the second digit of
the version string.
Most people are encouraged to use this version.  
We will only offer our paid support for versions of Condor from the
stable release series.

\emph{On the stable series, new minor version releases will only
be made for bug fixes and to support new platforms.}
No new features will be added to the stable series.
People are encouraged to install new stable versions of Condor when
they appear, since they probably fix bugs you care about.
Hopefully, there won't be many minor version releases for any given
stable series.


%%%%%%%%%%%%%%%%%%%%%%%%%%%%%%%%%%%%%%%%%%%%%%%%%%%%%%%%%%%%%%%%%%%%%%
\subsection{\label{sec:Developement-Series}
The Development Release Series}
%%%%%%%%%%%%%%%%%%%%%%%%%%%%%%%%%%%%%%%%%%%%%%%%%%%%%%%%%%%%%%%%%%%%%%

Only people who are interested in the latest research, new features
that haven't been fully tested, etc, should download the development
version, denoted with an odd number in the second digit of the version
string.  
We will make a best effort to ensure that the development series will
work, but we make no guarantees.

On the development series, new minor version releases will probably
happen frequently.
People should not feel compelled to install new minor versions unless
they know they want features or bug fixes from the newer development
version.

\emph{Most sites will probably never want to install a development
version of Condor for any reason.}
Only if you know what you are doing (and like pain), or were
explicitly instructed to do so by someone on the Condor Team, should
you install a development version at your site.

\Note Different releases within a development series cannot be
installed side-by-side within the same pool. 
For example, the protocols used by version 6.1.6 are not compatible with the
protocols used in version 6.1.5.  
When you upgrade to a new development release, make certain you upgrade all
machines in your pool to the same version.

After the feature set of the development series is satisfactory to the
Condor Team, we will put a code freeze in place, and from that point
forward, only bug fixes will be made to that development series.
When we have fully tested this version, we will release a new stable
series, resetting the minor version number, and start work on a new
development release from there.

%%%%%%%%%%%%%%%%%%%%%%%%%%%%%%%%%%%%%%%%%%%%%%%%%%%%%%%%%%%%%%%%%%%%%%
% The rest of this file just inputs other files which contain sections
% describing each release series in detail.
%%%%%%%%%%%%%%%%%%%%%%%%%%%%%%%%%%%%%%%%%%%%%%%%%%%%%%%%%%%%%%%%%%%%%%

%%%%%%%%%%%%%%%%%%%%%%%%%%%%%%%%%%%%%%%%%%%%%%%%%%%%%%%%%%%%%%%%%%%%%%
\section{\label{sec:History-6-4}Stable Release Series 6.4}
%%%%%%%%%%%%%%%%%%%%%%%%%%%%%%%%%%%%%%%%%%%%%%%%%%%%%%%%%%%%%%%%%%%%%%

This is the stable release series of Condor.
New features will be added and tested in the 6.5 development series. 
The details of each version are described below.

%%%%%%%%%%%%%%%%%%%%%%%%%%%%%%%%%%%%%%%%%%%%%%%%%%%%%%%%%%%%%%%%%%%%%%
\subsection{\label{sec:New-6-4-6}Version 6.4.6}
%%%%%%%%%%%%%%%%%%%%%%%%%%%%%%%%%%%%%%%%%%%%%%%%%%%%%%%%%%%%%%%%%%%%%%

\noindent Bugs Fixed:
\begin{itemize}

\item When more than 512 distinct users submit to Condor or Condor-G,
the \Condor{schedd} no longer crashes. 

\end{itemize}

%%%%%%%%%%%%%%%%%%%%%%%%%%%%%%%%%%%%%%%%%%%%%%%%%%%%%%%%%%%%%%%%%%%%%%
\subsection{\label{sec:New-6-4-3}Version 6.4.3}
%%%%%%%%%%%%%%%%%%%%%%%%%%%%%%%%%%%%%%%%%%%%%%%%%%%%%%%%%%%%%%%%%%%%%%

\noindent New Features:
\begin{itemize}

\item Added a \Opt{-hold} and \Opt{-held} option to \Condor{q} which 
displays the reason that the job had been held.

\end{itemize}

\noindent Bugs Fixed:
\begin{itemize}

\item Fixed a bug where more than one space between arguments to a job
in the java universe would result in it being invoked with and incorrect
list arguments.

\item Removed renaming of the executable to \Prog{condor\_exec} in the java
universe. This fixes a bug where the JVM was looking at its path to determine
its installation directory.

\item Fixed a bug and resulting null pointer exception in the java universe
because under certain conditions, Condor would invoke the JVM incorrectly.

\item Fixed serveral error reporting messages to be more precise.

\item When the NIS environment was being used, the \Condor{starter} daemon
would produce heavy amounts of NIS traffic. This has been fixed.

\item Binary characters in the \File{StarterLog} file and a possible
segmentation fault have been fixed.

\item Fixed \Cmd{select}{2} in the standard universe on our Linux ports.

\item Fixed a small bug in \Condor{q} that was displaying the wrong
username for ``niceuser'' jobs.

\item Fixed a bug where, in the standard universe, you could not open a file
whose name had spaces in it.

\item Fixed a bug in DAGMan where pre and post scripts would fail to
run if the DAG description file had extra whitespace.
Also, reworded the error messages DAGMan produces when it fails to
parse the DAG description file to be more clear and helpful for
solving the problem.

\item Fixed some misleading error messages in the Condor log files
when there were permission problems trying to execute a program. 

\item Condor for Windows will now run on Windows XP.

\item Condor for Windows now supports the Java Universe.

\item Users logged into Windows Domain accounts rather than local accounts
can submit jobs.

\item Potential Windows registry bloating bug fixed. Condor for Windows no
longer creates and deletes an account on the execute machine each time a
job is run. Instead, a single account for each VM on the execute machine is
created once and enabled or disabled as needed.

\item Cross-submits from Windows to Unix and from Unix to Windows are now
supported, provided that both platforms are running Condor 6.4 series daemons.

\item Free disk space is now reported accurately on Windows.

\item A rare but serious bug that could allow non-Condor processes to be added
to the Condor process family on Windows has been fixed.

\item Condor for Windows will now also run 16-bit applications.

\item Fixed a minor bug where certain integer attributes in the
\File{condor\_config} file might not have been properly parsed if they
were defined in terms of other config file attributes, using the
\MacroUNI{attribute} notation.  

\end{itemize}

\noindent Known Bugs:
\begin{itemize}

\item You may not open a file in the standard universe whose name contains a
colon ``:''.

\end{itemize}

%%%%%%%%%%%%%%%%%%%%%%%%%%%%%%%%%%%%%%%%%%%%%%%%%%%%%%%%%%%%%%%%%%%%%%
\subsection{\label{sec:New-6-4-2}Version 6.4.2}
%%%%%%%%%%%%%%%%%%%%%%%%%%%%%%%%%%%%%%%%%%%%%%%%%%%%%%%%%%%%%%%%%%%%%%
\noindent New Features:
\begin{itemize}

\item. This release mirrored the Condor-G release, and has no new features.

\end{itemize}

\noindent Bugs Fixed:
\begin{itemize}
\item None.

\end{itemize}
\noindent Known Bugs:
\begin{itemize}

\item None.

\end{itemize}

%%%%%%%%%%%%%%%%%%%%%%%%%%%%%%%%%%%%%%%%%%%%%%%%%%%%%%%%%%%%%%%%%%%%%%
\subsection{\label{sec:New-6-4-1}Version 6.4.1}
%%%%%%%%%%%%%%%%%%%%%%%%%%%%%%%%%%%%%%%%%%%%%%%%%%%%%%%%%%%%%%%%%%%%%%
\noindent New Features:
\begin{itemize}

\item None.

\end{itemize}

\noindent Bugs Fixed:
\begin{itemize}

\item Users are now allowed to answer ``none'' when prompted by the
installer to provide a Java JVM path. This avoids an endless loop and
leaves the Java abilities of Condor unconfigured.

\end{itemize}

\noindent Known Bugs:
\begin{itemize}

\item None.

\end{itemize}

%%%%%%%%%%%%%%%%%%%%%%%%%%%%%%%%%%%%%%%%%%%%%%%%%%%%%%%%%%%%%%%%%%%%%%
\subsection{\label{sec:New-6-4-0}Version 6.4.0}
%%%%%%%%%%%%%%%%%%%%%%%%%%%%%%%%%%%%%%%%%%%%%%%%%%%%%%%%%%%%%%%%%%%%%%

\noindent New Features:

\begin{itemize}

\item If a job universe is not specified in a submit description file, 
\Condor{submit}  will check the config file for \Macro{DEFAULT\_UNIVERSE}
instead of always choosing the standard universe. 

\item The \Macro{D\_SECONDS} debug flag is deprecated. Seconds are now always
included in logfiles. 

\item For each daemon listed in \Macro{DAEMON\_LIST}, you can now control the
environment variables of the daemon with a config file setting of the form
\Macro{DAEMONNAME\_ENVIRONMENT}, where \MacroNI{DAEMONNAME} is the name of a
daemon listed in \Macro{DAEMON\_LIST}. For more information, see
section~\ref{sec:Master-Config-File-Entries}.

\end{itemize}

\noindent Bugs Fixed:

\begin{itemize}

\item Fixed a bug in the new starter where if the submit file set no
arguments, the job would receive one argument of zero length.

\end{itemize}

\noindent Known Bugs:

\begin{itemize}

\item None.

\end{itemize}



%%%%%%%%%%%%%%%%%%%%%%%%%%%%%%%%%%%%%%%%%%%%%%%%%%%%%%%%%%%%%%%%%%%%%%
\section{\label{sec:History-6-3}Development Release Series 6.3}
%%%%%%%%%%%%%%%%%%%%%%%%%%%%%%%%%%%%%%%%%%%%%%%%%%%%%%%%%%%%%%%%%%%%%%

This is the second development release series of Condor.

It contains numerous enhancements over the 6.2 stable series.
For example:

\begin{itemize}

\item Support for Kerberos and X.509 authentication.

\item Support for transfering files needed by jobs (for all universes
except standard and PVM)

\item Support for MPICH jobs.

\item Support for JAVA jobs.

\item 
Condor DAGMan is dramatically more reliable and efficient, and offers
a number of new features.

\end{itemize}

The 6.3 series has many other improvements over the 6.2 series, and
may be available on newer platforms.  The new features, bugs fixed,
and known bugs of each version are described below in detail.


%%%%%%%%%%%%%%%%%%%%%%%%%%%%%%%%%%%%%%%%%%%%%%%%%%%%%%%%%%%%%%%%%%%%%%
\subsection{\label{sec:New-6-3-3}Version 6.3.3}
%%%%%%%%%%%%%%%%%%%%%%%%%%%%%%%%%%%%%%%%%%%%%%%%%%%%%%%%%%%%%%%%%%%%%%

\noindent New Features:

\begin{itemize}

\item Added support for Kerberos and X.509 authentication in Condor.  

\item Added the ability for vanilla jobs on Unix to use Condor's file
transfer mechanism so that you don't have to rely on a shared file
system.  

\item Added support for MPICH jobs on Windows NT and 2000.

\item Added support for the JAVA universe.

\item When you use \Condor{hold} and \Condor{release}, you now see an
entry about the event in the UserLog file for the job.

\item Whenever a job is removed, put on hold, or released (either by a
Condor user or by the Condor system itself), there is a ``reason''
attribute placed in the job ad and written to the UserLog file.  
If a job is held, \Attr{HoldReason} will be set.
If a job is released, \Attr{ReleaseReason} will be set.
If a job is removed, \Attr{RemoveReason} will be set.
In addition, whenever a job's status changes,
\Attr{EnteredCurrentStatus} will contain the epoch time when the
change took place.

\item The error messages you get from \Condor{rm}, \Condor{hold} and
\Condor{release} have all been updated to be more specific and
accurate. 

\item Condor users can now specify a policy for when their jobs should
leave the queue or be put on hold.
They can specify expressions that are evaluated periodically, and
whenever the job exits.
This policy can be used to ensure that the job remains in the queue
and is re-run until it exits with a certain exit code, that the job
should be put on hold if a certain condition is true, and so on. 
If any of these policy expressions result in the job being removed
from the queue or put on hold, the UserLog entry for the event
includes a string describing why the action was taken.

\item Changed the way Condor finds the various \Condor{shadow} and
\Condor{starter} binaries you have installed on your machine.
Now, you can specify a \Macro{SHADOW\_LIST} and a
\Macro{STARTER\_LIST}.
These are treated much like the \Macro{DAEMON\_LIST} setting, they
specify a list of attribute names, each of which point to the actual
binary you want to use.
On startup, Condor will check these lists, make sure all the binaries
specified exist, and find out what abilities each program provides.
This information is used during matchmaking to ensure that a job which
requires a certain ability (like having a new enough version of Condor
to support transfering files on Unix) can find a resource that
provides that ability.

\item Added new security feature to offer fine-grained control over
what configuration values can be modified by \Condor{config\_val}
using \Arg{-set} and related options.
Pool administrators can now define lists of attributes that can be set
by hosts that authenticate to the various permission levels of
Condor's host based security (for example, \DCPerm{WRITE},
\DCPerm{ADMINISTRATOR}, etc).
These lists are defined by attributes with names like
\Macro{SETTABLE\_ATTRS\_CONFIG} and
\Macro{STARTD\_SETTABLE\_ATTRS\_OWNER}. 
For more information about host-based security in Condor, see
section~\ref{sec:Host-Security} on page~\pageref{sec:Host-Security}.
For more information about how to configure the new settings, see the
same section of the manual.
In particular, see section~\ref{sec:Host-Security} on
page~\pageref{sec:Host-Security}. 

\item Greatly improved the handling of the ``soft kill signal'' you
can specify for your job.
This signal is now stored as a signal name, not an integer, so that it
works across different platforms.
Also, fixed some bugs where the signal numbers were getting translated
incorrectly in some circumstances.

\item Added the \Arg{-full} option to \Condor{reconfig}.
The \Arg{-full} option causes the Condor daemon to clear its cache of
DNS information and some other expensive operations.
So, the regular \Condor{reconfig} is now more light-weight, and can
be used more frequently without undue overhead on the Condor daemons. 
The default \Condor{reconfig} has also been changed so that it will
work from any host with \DCPerm{WRITE} permission in your pool,
instead of requiring \DCPerm{ADMINISTRATOR} access.

\item Added the \Macro{EMAIL\_DOMAIN} config file setting.
This allows Condor administrators to define a default domain where
Condor should send email if whatever \Macro{UID\_DOMAIN} is set to
would yield invalid email addresses.
For more information, see section~\ref{param:EmailDomain} on
page~\pageref{param:EmailDomain}.

\item
Added support for RedHat 7.2.

\item When printing out the UserLog, we now only log a new event for
``Image size of job updated'' when the new value is different than the
existing value.

\end{itemize}

\noindent Bugs Fixed:

\begin{itemize}

\item
Fixed a bug in Condor-PVM where it was possible that a machine would be 
placed into the virtual machine, but then ignored by Condor for the purposes
of scheduling tasks there.

\item
Under Solaris, the checkpointing libraries could segfault while determining
the page size of the machine. 
This has been fixed.

\item
In a heavily loaded submit machine, the \Condor{schedd} would time out
authentication checks with its shadows. 
This would cause the shadows to
exit believing the \Condor{schedd} had died placing jobs into the idle
state and the \Condor{schedd} to exhibit poor performance.
This timeout problem has been corrected.

\item
Removed use of the bfd libary in the Condor Linux distribution. 
This will make the dynamic versions of the Condor executables have a
higher chance of being usable when RedHat upgrades.

\item
When you specify ``STARTD\_HAS\_BAD\_UTMP = True'' in the config files
on a linux machine with a 2.4+ kernel, the \Condor{startd} would report
an error stating some of the tty entries in /dev. This would result
in incorrect tty activity sampling causing jobs to not be migrated or
incorrectly started on a resource. This has now been corrected.

\item 
When you specify ``GenEnv = True'' in a \Condor{submit} file,
your environment is no longer restricted to 10KB.

\item
The three-digit event numbers which begin each job event in the
userlog were incorrect for some events in Condor 6.3.0 and 6.3.1.
Specifically, ULOG\_JOB\_SUSPENDED, ULOG\_JOB\_UNSUSPENDED,
ULOG\_JOB\_HELD, ULOG\_JOB\_RELEASED, ULOG\_GENERIC, and
ULOG\_JOB\_ABORTED had incorrect event numbers.  This has now been
corrected.

\Note This means userlog-parsing code written for Condor 6.3.0 or
6.3.1 development releases may not work reliably with userlogs
generated by other versions of Condor, and visa-versa.  Userlog events
will remain compatible between all stable releases of Condor, however,
and with post-6.3.1 releases in this development series.

\item
The \Condor{run} script now correctly exits when it sees a job aborted
event, instead of hanging, waiting for a termination event.

\item
Until now, when a DAG node's Condor job failed, the node failed,
regardless of whether its POST script succeeded or failed.  This was a
bug, because it prevented users from using POST scripts to evaluate
jobs with non-zero exit codes and deem them successful anyway.  This
has now been fixed -- a node's success is equal to its POST script's
success -- but the change may affect existing DAGs which rely on the
old, broken behavior.  Users utilizing POST scripts must now be sure
to pass the POST script the job's return value, and return it again,
if they do not wish to alter it; otherwise failed jobs will be masked
by ignorant POST scripts which always succeed.

\end{itemize}

\noindent Known Bugs:

\begin{itemize}
\item The HP-UX Vendor C++ CFront compiler does not work with \Condor{compile}
if exception handling is enabled with +eh.

\item The HP-UX Vendor aCC compiler does not work at all with Condor.
\end{itemize}

%%%%%%%%%%%%%%%%%%%%%%%%%%%%%%%%%%%%%%%%%%%%%%%%%%%%%%%%%%%%%%%%%%%%%%
\subsection{\label{sec:New-6-3-2}Version 6.3.2}
%%%%%%%%%%%%%%%%%%%%%%%%%%%%%%%%%%%%%%%%%%%%%%%%%%%%%%%%%%%%%%%%%%%%%%

Version 6.3.2 of Condor was only released as a version of
``Condor-G''.
This version of Condor-G is not widely deployed.
However, to avoid confusion, the Condor developers did not want to
release a full Condor distribution with the same version number.


%%%%%%%%%%%%%%%%%%%%%%%%%%%%%%%%%%%%%%%%%%%%%%%%%%%%%%%%%%%%%%%%%%%%%%
\subsubsection{\label{sec:New-6-3-1}Version 6.3.1}
%%%%%%%%%%%%%%%%%%%%%%%%%%%%%%%%%%%%%%%%%%%%%%%%%%%%%%%%%%%%%%%%%%%%%%

\noindent New Features:
\begin{itemize}

\item
Added support for an \AdAttr{x509proxy} option in
\Condor{submit}. There is now a seperate \Condor{GridManager} for each
user and proxy pair. This will be detailed in a future release of
Condor.
 
\item
More Condor DAGMan improvements and bug fixes:

\begin{itemize}

\item 
Added a \oArgnm{-dag} flag to \Condor{q} to more succinctly display dags
and their ownership.

\item
Added a new event to the Condor userlog at the completion of a POST
script.  This allows DAGMan, during recovery, to know which POST
scripts have finished succesfully, so it no longer has to re-run them
all to make sure.

\item
Implemented separate \Arg{-MaxPre} and \Arg{-MaxPost} options to limit
the number of simultaneously running PRE and POST scripts.  The
\Arg{-MaxScripts} option is still available, and is equivalent to
setting both \Arg{-MaxPre} and \Arg{-MaxPost} to the same value.

\item
Added support for a new ``Retry'' parameter in the DAG file, which
instructs DAGMan to automatically retry a node a configurable number
of times if its PRE Script, Job, or POST Script fail for any reason.

\item
Added timestamps to all DAGMan log messages.

\item
Fixed a bug whereby DAGMan would clean up its lock file without
creating a rescue file when killed with SIGTERM.

\item
DAGMan no longer aborts the DAG if it encounters executable error or
job aborted events in the userlog, but rather marks the corresponding
DAG nodes as ``failed'' so the rest of the DAG can continue.

\item
Fixed a bug whereby DAGMan could crash if it saw userlog events for
jobs it didn't submit.

\end{itemize}

\item Added port restriction capabilities to Condor so you can specify a range
of ports to use for the communication between Condor Daemons.

\item To improve performance: if there's no \Macro{HISTORY} file
specified, don't connect back to the schedd to report your exit info on
successful compeletion, since the schedd is simply going to discard that
info anyway.

\item Added the macro \Macro{SECONDARY\_COLLECTOR\_LIST} to tell the
master to send classads to an additional list of collectors so you can
do administration commands when the primary collector is down.

\item When a job checkpoints it askes the shadow whether or not it
should and if so where. This fixes some flocking bugs and increases
performance of the pool.

\item Added match rejection diagnostics in \Condor{q} \oArgnm{-analyze} to
give more information on why a particular job hasn't started up yet.

\item Added \oArgnm{--vms} argument to \Condor{glidein} that enables the
control of how many virtual machines to start up on the target platform.

\item Added capability to the config file language to retrieve environment
variables while being processed.

\item Added capability to make default user user priority factor configurable
with the \Macro{DEFAULT\_PRIORITY\_FACTOR} macro in the config files.

\item Added full support for RedHat 7.1 and the gcc 2.96 compiler. However,
the standard universe binaries must still be statically linked.

\item When jobs are suspended or unsuspended, an event is now written into
the user job log.

\item Added \oArgnm{-a} flag to \Condor{submit} to add/override attributes
specified in the submit file.

\item Under Unix, added the ability for a submittor of a job to describe when
and how a job is allowed/not allowed to leave the queue. For example, if
a job has only run for 5 minutes, but it was supposed to have run an hour 
minimum, then do not let the job leave the queue but restart it instead.

\item New environment variable available CONDOR\_SCRATCH\_DIR available
in a standard or vanilla job's environment that denotes temporary space
the job can use that will be cleaned up automatically when the job leaves
from the machine.

\item Not exactly a new feature, but some internal parts of Condor had been
fixed up to try and improve the memory footprint of a few of our daemons.

\end{itemize}

\noindent Bugs Fixed:
\begin{itemize}

\item Fixed a bug where \Condor{q} would produce wildly inaccurate run time
reports of jobs in the queue.

\item Fixed it so that if the condor scheduler fails to notify the
administrator through email, just print a warning and do not except.

\item Fixed a bug where \Condor{submit} would incorrectly create the user
log file.

\item Fixed a bug where a job queue sorted by date with \Condor{q} would
be displayed in descending instead of ascending order.

\item Fixed and improved error handling when \Condor{submit} fails.

\item Numerous fixes in the Condor User Log System.

\item Fixed a bug where when Condor inspects its on disk job queue log,
it would do it with case sensitivity. Now there is no case sensitivity.

\item Fixed a bug in \Condor{glidein} where it have trouble figuring out
the architecture of a minimally installed HP-UX machine.

\item Fixed it so that email to the user has the word ``condor'' capitalized
in the subject.

\item Fixed a situation where when a user has multiple schedulers submitting
to the same pool, the Negotiator would starve some of the schedulers.

\item Added a feature whereby if a transfer of an executable
from a submission machine to an execute machine fails, Condor
will retry a configurable numbers of times denotated by the
\Macro{EXEC\_TRANSFER\_ATTEMPTS} macro. This macro defaults to three if
left undefined. This macro exists only for the Unix port of Condor.

\item Fixed a bug where if a schedd had too many rejected clusters during a
match phase, it would ``except'' and have to be restarted by the master.

\end{itemize}

\noindent Known Bugs:
\begin{itemize}
\item The HP-UX Vendor C++ CFront compiler does not work with \Condor{compile}
if exception handling is enabled with +eh.

\item The HP-UX Vendor aCC compiler does not work at all with Condor.
\end{itemize}

%%%%%%%%%%%%%%%%%%%%%%%%%%%%%%%%%%%%%%%%%%%%%%%%%%%%%%%%%%%%%%%%%%%%%%
\subsubsection{\label{sec:New-6-3-0}Version 6.3.0}
%%%%%%%%%%%%%%%%%%%%%%%%%%%%%%%%%%%%%%%%%%%%%%%%%%%%%%%%%%%%%%%%%%%%%%

\noindent New Features:
\begin{itemize}

\item Added support for running MPICH jobs under Condor.

\end{itemize}

\noindent
Many Condor DAGMan improvements and bug fixes:

\begin{itemize}

\item
PRE and POST scripts now run asynchronously, rather than synchronously
as in the past.  As a result, DAGMan now supports a \Arg{-MaxScripts}
option to limit the number of simultaneously running PRE and POST
scripts.

\item
Whether or not POST scripts are always executed after failed jobs is
now configurable with the \Arg{-NoPostFail} argument.

\item
Added a \Arg{-r} flag to \Condor{submit\_dag} to submit DAGMan to a
remote \Condor{schedd}.

\item
Made the arguments to \Condor{submit\_dag} case-insensitive.

\item
Fixed a variety of bugs in DAGMan's event handling, so DAGMan should
no longer hang indefinitely after failed jobs, or mistake one job's
userlog events for those of another.

\item
DAGMan's error handling and logging output have been substantially
clarified and improved.  For example, DAGMan now prints a list of
failed jobs when it exits, rather than just saying ``some jobs
failed''.

\item
Jobs submitted by a \Condor{dagman} job now have \AdAttr{DAGManJobId}
and \AdAttr{DAGNodeName} in the job classad.

\item
Fixed a \Condor{submit\_dag} bug preventing the submission of DAGMan
Rescue files.

\item
Improved the handling of userlog errors (less crashing, more coping).

\item
Fixed a bug when recovering from the userlog after a crash or reboot.

\item
Fixed bugs in the handling of \Arg{-MaxJobs}.

\item
Added a \Arg{-a line} argument to \Condor{submit} to add a line to the
submit file before processing (overriding the submit file).

\item
Added a \Arg{-dag} flag to \Condor{q} to format and sort DAG jobs
sensibly under their DAGMan master job.

\end{itemize}

\noindent Known Bugs:

\begin{itemize}

\item \Condor{kbdd} doesn't work properly under Compaq Tru64 5.1, and
as a result, resources may not leave the ``Unclaimed'' state
regardless of keyboard or pty activity.  Compaq Tru64 5.0a and earlier
do work properly.

\end{itemize}

%%%%%%%%%%%%%%%%%%%%%%%%%%%%%%%%%%%%%%%%%%%%%%%%%%%%%%%%%%%%%%%%%%%%%%
\section{\label{sec:History-6-2}Stable Release Series 6.2}
%%%%%%%%%%%%%%%%%%%%%%%%%%%%%%%%%%%%%%%%%%%%%%%%%%%%%%%%%%%%%%%%%%%%%%

This is the second stable release series of Condor.
All of the new features developed in the 6.1 series are now considered
stable, supported features of Condor.
New releases of 6.2.0 should happen infrequently and will only include
bug fixes and support for new platforms.
New features will be added and tested in the 6.3 development series. 
The details of each version are described below.

%%%%%%%%%%%%%%%%%%%%%%%%%%%%%%%%%%%%%%%%%%%%%%%%%%%%%%%%%%%%%%%%%%%%%%
\subsection{\label{sec:New-6-2-1}Version 6.2.1}
%%%%%%%%%%%%%%%%%%%%%%%%%%%%%%%%%%%%%%%%%%%%%%%%%%%%%%%%%%%%%%%%%%%%%%

\noindent Bugs Fixed:

\begin{itemize}

\item Fixed a bug in the \Condor{startd} that would cause the daemon
to crash if you set the \Macro{POLLING\_INTERVAL} macro to a value
greater than 60.

\item In \Condor{q}, dash-arguments (e.g., -pool, -run, etc.) were being
parsed incorrectly such that the same arguments specified without a
dash would be interpreted as if the dash were present, making it
impossible to specify ``pool'' or ``globus'' or ``run'' as an owner
argument.

\item Fixed bug in \Condor{submit} that would cause certain submit
file directives to be silently ignored if you used the wrong attribute
name.  
Now, all submit file attributes can use the same names you see in the
job ClassAd (what you'd see with \begin{verbatim}condor\_q
-long\end{verbatim}).
For example, you can now use \begin{verbatim}CoreSize =
0\end{verbatim} or \begin{verbatim}core_size = 0\end{verbatim} in your 
submit file, and either one would be recognized.

\item Fixed some of the error messages in \Condor{submit} so that they
are all consistently formatted.

\end{itemize}

\noindent Known Bugs:

\begin{itemize}

\item None.

\end{itemize}


%%%%%%%%%%%%%%%%%%%%%%%%%%%%%%%%%%%%%%%%%%%%%%%%%%%%%%%%%%%%%%%%%%%%%%
\subsubsection{\label{sec:New-6-2-0}Version 6.2.0}
%%%%%%%%%%%%%%%%%%%%%%%%%%%%%%%%%%%%%%%%%%%%%%%%%%%%%%%%%%%%%%%%%%%%%%

\noindent New Features Over the 6.0 Release Series
\begin{itemize}

\item Support for running multiple jobs on SMP (Symmetric Mutli-Processor)
machines.

\end{itemize}

\noindent New Features Over the Last Development Series: 6.1.17
\begin{itemize}

\item If \Attr{CkptArch} isn't specified in the job submission file's
\Attr{Requirements} attribute, then automatically add this expression:

\begin{verbatim}
CkptRequirements = ((CkptArch == Arch) || (CkptArch =?= UNDEFINED)) &&
	((CkptOpSys == OpSys) || (CkptOpSys =?= UNDEFINED))
\end{verbatim}

to the \Attr{Requirements} expression. This allows for users who specify
a heterogeneous submission to not have to worry about having their checkpoints
incorrectly starting up on architectures for which they were not designed
to run.

\item The \Macro{APPEND\_REQ\_<universe>} config file entries now get
appended to the beginning of the expressions before Condor adds internal
default expressions.  This allows the sysadmin to override any default
policy that Condor enforces.

\item There is now a single \Macro{APPEND\_REQUIREMENTS} attribute
that will get appended to all universe's \Attr{Requirements}
expressions unless a specific \Macro{APPEND\_REQ\_STANDARD} or
\Macro{APPEND\_REQ\_VANILLA} expression is defined.

\item Increased certain networking parameters to help alleviate the 
\Condor{shadow}'s inability to contact the \Condor{schedd} during heavy load
of the system.

\item Added a \Condor{glidein} man page to the manual.

\item Some of the log messages in the \Condor{startd} were modified to
be more clear and to provide more information.

\item Added a new attribute to the \Condor{startd} ClassAd when the
machine is claimed, \AdAttr{RemoteOwner}.

\end{itemize}

\noindent Bugs fixed since 6.1.17
\begin{itemize}

\item On NT, the Registry would increase in size while Condor was
servicing jobs. This has been fixed.

\item Added \File{utmpx} support for Solaris 2.8 to fix a problem where
\AdAttr{KeyBoardIdle} wasn't being set correctly.

\item When doing a \Condor{hold} under NT, the job was removed instead of
held. This has been fixed.

\item When using the \Arg{-master} argument to\Condor{restart}, the
\Condor{master} used to exit instead of restarting.
Now, the \Condor{master} correctly restarts itself in this case.

\end{itemize}

\noindent Known Bugs:
\begin{itemize}

\item \Attr{STARTD\_HAS\_BAD\_UTMP} does not work if set to True on Solaris 
2.8.  However, since \File{utmpx} support is enabled, you shouldn't
need to do this normally.

\end{itemize}

%%%%%%%%%%%%%%%%%%%%%%%%%%%%%%%%%%%%%%%%%%%%%%%%%%%%%%%%%%%%%%%%%%%%%%
\section{\label{sec:History-6-1}Development Release Series 6.1}
%%%%%%%%%%%%%%%%%%%%%%%%%%%%%%%%%%%%%%%%%%%%%%%%%%%%%%%%%%%%%%%%%%%%%%

This was the first development release series.
It contains numerous enhancements over the 6.0 stable series.
For example:

\begin{itemize}
\item Support for running multiple jobs on SMP machines
\item Enhanced functionality for pool administrators
\item Support for PVM, MPI and Globus jobs
\item Support for \Term{Flocking} jobs across different Condor pools
\end{itemize}

The 6.1 series has many other improvements over the 6.0 series, and  
is available on more platforms.  
The new features, bugs fixed, and known bugs of each version are
described below in detail.

%%%%%%%%%%%%%%%%%%%%%%%%%%%%%%%%%%%%%%%%%%%%%%%%%%%%%%%%%%%%%%%%%%%%%%
\subsection*{\label{sec:New-6-1-17}Version 6.1.17}
%%%%%%%%%%%%%%%%%%%%%%%%%%%%%%%%%%%%%%%%%%%%%%%%%%%%%%%%%%%%%%%%%%%%%%

This version is the 6.2.0 ``release candidate''.  
It was publically released in Feburary of 2001, and it will be released
as 6.2.0 once it is considered ``stable'' by heavy testing at the 
UW-Madison Computer Science Department Condor pool.

\noindent New Features:

\begin{itemize}

\item Hostnames in the HOSTALLOW and HOSTDENY entries are now case-insensitive.

\item It is now possible to submit NT jobs from a UNIX machine.

\item The NT release of Condor now supports a USE\_VISIBLE\_DESKTOP parameter. 
If true, Condor will allow the job to create windows on the desktop of the
execute machine and interact with the job. This is particularly useful for 
debugging why an application will not run under Condor.

\item The \Condor{startd} contains support for the new MPI dedicated 
scheduler that will appear in the 6.3 development series. This will allow
you to use your 6.2 Condor pool with the new scheduler.

\item Added a \Opt{mixedcase} option to \Condor{config\_val} to allow 
for overriding the default of lowercasing all the config names

\item Added a pid\_snapshot\_interval option to the config file to
control how often the \Condor{startd} should examine the running 
process family. It defaults to 50 seconds.

\end{itemize}

\noindent Bugs Fixed:

\begin{itemize}

\item Fixed a bug with the \Condor{schedd} reaching the MAX\_JOBS\_RUNNING
mark and properly calculating Scheduler Universe jobs for preemption.

\item Fixed a bug in the \Condor{schedd} loosing track of \Condor{startd}s 
in the initial claiming phase. This bug affected all platforms, but was most
likely to manifest on Solaris 2.6

\item CPU Time can be greater than wall clock time in Multi-threaded
apps, so do not consider it an error in the UserLog.

\item \Condor{restart} \Opt{-master} now works correctly.
 
\item Fixed a rare condition in the \Condor{startd} that could corrupt
memory and result in a signal 11 (SIGSEGV, or segmentation violation).

\item Fixed a bug that would cause the ``execute event'' to not be
logged to the UserLog if the binary for the job resided on AFS.

\item Fixed a race-condition in Condor's PVM support on SMP machines
(introduced in version 6.1.16) that caused PVM tasks to be associated
with the wrong daemon.

\item Better handling of checkpointing on large-memory Linux machines.

\item Fixed random occasions of job completion email not being sent.

\item It is no longer possible to use \Condor{user\_prio} to set a priority of less
than 1.

\item Fixed a bug in the job completion email statistics.
Run Time was being underreported when the job completed after doing a
periodic checkpoint.

\item Fixed a bug that caused CondorLoadAvg to get stuck at 0.0 on
Linux when the system clock was adjusted.

\item Fixed a \Condor{submit} bug that caused all machine\_count
commands after the first queue statement to be ignored for PVM jobs.

\item PVM tasks now run as the user when appropriate instead of always
running under the UNIX ``nobody'' account.

\item Fixed support for the PVM group server.

\item PVM uses an environment variable to communicate with it's children
instead of a file in /tmp. This file previously could become overwritten
by mulitple PVM jobs.

\item \Condor{stats} now lives in the ``bin'' directory instead of ``sbin''.

\end{itemize}

\noindent Known Bugs:

\begin{itemize}

\item The \Condor{negotiator} can crash if the Accountantnew.log file becomes
corrupted. This most often occurs if the Central Manager runs out of diskspace. 

\end{itemize}

%%%%%%%%%%%%%%%%%%%%%%%%%%%%%%%%%%%%%%%%%%%%%%%%%%%%%%%%%%%%%%%%%%%%%%
\subsection*{\label{sec:New-6-1-16}Version 6.1.16}
%%%%%%%%%%%%%%%%%%%%%%%%%%%%%%%%%%%%%%%%%%%%%%%%%%%%%%%%%%%%%%%%%%%%%%

\noindent New Features:

\begin{itemize}

\item Condor now supports multiple pvmds per user on a machine.  Users
can now submit more than one PVM job at a time, PVM tasks can now run
on the submission machine, and multiple PVM tasks can run on SMP
machines.  \Condor{submit} no longer inserts default job requirements
to restrict PVM jobs to one pvmd per user on a machine.  This new
functionality requires the \Condor{pvmd} included in this (and future)
Condor releases.  If you set ``PVM\_OLD\_PVMD = True'' in the Condor
configuration file, \Condor{submit} will insert the default PVM job
requirements as it did in previous releases.  You must set this if you
don't upgrade your \Condor{pvmd} binary or if your jobs flock with pools
that user an older \Condor{pvmd}.

\item The NT release of Condor no longer contains debugging
information.
This drastically reduces the size of the binaries you must install.  

\end{itemize}

\noindent Bugs Fixed:

\begin{itemize}

\item The configuration files shipped with version 6.1.15 contained a
number of errors relating to host-based security, the configuration of
the central manager, and a few other things.
These errors have all been corrected.

\item Fixed a memory management bug in the \Condor{schedd} that could
cause it to crash under certain circumstances when machines were taken
away from the schedd's control.

\item Fixed a potential memory leak in a library used by the
\Condor{startd} and \Condor{master} that could leak memory while
Condor jobs were executing.

\item Fixed a bug in the NT version of Condor that would result in
faulty reporting of the load average.

\item The \Condor{shadow.pvm} should now correctly return core files
when a task or \Condor{pvmd} crashes.

\item This release fixes a memory error introduced in version
6.1.15 that could crash the \Condor{shadow.pvm}.

\item Some \Condor{pvmd} binaries in previous releases included
debugging code we added that could cause the \Condor{pvmd} to crash.
This release includes new \Condor{pvmd} binaries for all platforms
with the problematic debugging code removed.

\item Fixed a bug in the \Opt{-unset} options to \Condor{config\_val}
that was introduced in version 6.1.15.
Both \Opt{-unset} and \Opt{-runset} work correctly, now.

\end{itemize}

\noindent Known Bugs:

\begin{itemize}

\item None.

\end{itemize}

%%%%%%%%%%%%%%%%%%%%%%%%%%%%%%%%%%%%%%%%%%%%%%%%%%%%%%%%%%%%%%%%%%%%%%
\subsection*{\label{sec:New-6-1-15}Version 6.1.15}
%%%%%%%%%%%%%%%%%%%%%%%%%%%%%%%%%%%%%%%%%%%%%%%%%%%%%%%%%%%%%%%%%%%%%%

\noindent New Features:

\begin{itemize}

\item In the job submit description file passed to \Condor{submit}, 
a new style of macro (with two dollar-signs) can reference attributes
from the machine ClassAd.  This new style macro can be used in the
job's \MacroNI{Executable}, \MacroNI{Arguments}, or \MacroNI{Environment}
settings in the submit description file.  For example, if you have both
Linux and Solaris machines in your pool, the following submit description
file will run either foo.INTEL.LINUX or foo.SUN4u.SOLARIS27 as appropiate,
and will pass in the amount of memory available on that machine on the
command line:
\begin{verbatim}
	executable = foo.$$(Arch).$$(Opsys)
	arguments = $$(Memory)
	queue
\end{verbatim}

\item The \DCPerm{CONFIG} security access level now controls the
modification of daemon configurations using \Condor{config\_val}.  For
more information about security access levels, see
section~\ref{sec:Host-Security} on
page~\pageref{sec:Host-Security}.

\item The \Macro{DC\_DAEMON\_LIST} macro now indicates to the
\Condor{master} which processes in the \Macro{DAEMON\_LIST} use
Condor's DaemonCore inter-process communication mechanisms.  This
allows the \Condor{master} to monitor both processes developed with or
without the Condor DaemonCore library.

\item The new \Macro{NEGOTIATE\_ALL\_JOBS\_IN\_CLUSTER} macro can be
use to configure the \Condor{schedd} to not assume (for efficiency)
that if one job in a cluster can't be scheduled, then no other jobs in
the cluster can be scheduled.
If \Macro{NEGOTIATE\_ALL\_JOBS\_IN\_CLUSTER} is set to True, the
\Condor{schedd} will now always try to schedule each individual job in
a cluster.

\item The \Condor{schedd} now automatically adds any machine it is
matched with to its HOSTALLOW\_WRITE list.
This simplifies setting up a machine for flocking, since the
submitting user doesn't have to know all the machines where the job
might execute, they only have to know what central manager they wish
to flock to.
Submitting users must trust a central manager they report to, so this
doesn't impact security in any way.

\item Some static limits relating to the number of jobs which can be 
simultaneously started by the \Condor{schedd} has been removed.

\item The default Condor config file(s) which are installed by
the installation program have been re-organized for greater 
clarity and simplicity.  

\end{itemize}

\noindent Bugs Fixed:

\begin{itemize}

\item In the STANDARD Universe, jobs submitted to Condor could segfault
if they opened multiple files with the same name.  Usually this bug
was exposed when users would submit jobs without specifying a file
for either stdout or stderr; in this case, both would default to 
\File{/dev/null}, and this could trigger the problem.

\item The Linux 2.2.14 kernel, which is used by default with Red Hat 6.2,
has a serious bug can cause the machine to lock up when 
the same socket is used for repeated connection attempts.   Thus, 
previous versions of Condor could cause the 2.2.14 kernel to hang
(lots of other applications could do this as well).  The Condor Team
recommends that you upgrade your kernel to 2.2.16 or later.  However,
in v6.1.15 of Condor, a patch was added to the Condor networking
layer so that Condor would not trigger this Linux kernel bug.

\item If no email address was specified when the job was submitted
with \Condor{submit}, completion email was being sent to 
user@submit-machine-hostname.  This is not the correct behavior.  Now 
email goes by default to user@uid-domain, where uid-domain is
defined by the \MacroNI{UID\_DOMAIN} setting in the config file.

\item The \Condor{master} can now correctly shutdown and restart the
\Condor{checkpoint\_server}.

\item Email sent when a SCHEDULER Universe job compeltes now has the
correct From: header.

\item In the STANDARD universe, jobs which call sigsuspend() will 
now receive the correct return value.

\item Abnormal error conditions, such as the hard disk on the submit
machine filling up, are much less likely to result in a job disappearing
from the queue.

\item The \Condor{checkpoint\_server} now correctly reconfigures when
a \Condor{reconfig} command is received by the \Condor{master}.

\item Fixed a bug with how the \Condor{schedd} associates jobs with
machines (claimed resources) which would, under certain circumstances,
cause some jobs to remain idle until other jobs in the queue complete
or are preempted.

\item A number of PVM universe bugs are fixed in this release.
Bugs in how the \Condor{shadow.pvm} exited, which caused jobs to hang
at exit or to run multiple times, have been fixed.
The \Condor{shadow.pvm} no longer exits if there is a problem starting
up PVM on one remote host.
The \Condor{starter.pvm} now ignores the periodic checkpoint command
from the startd.  Previously, it would vacate the job when it received
the periodic checkpoint command.
A number of bugs with how the \Condor{starter.pvm} handled
asynchronous events, which caused it to take a long time to clean up
an exited PVM task, have been fixed.
The \Condor{schedd} now sets the status correctly on multi-class PVM
jobs and removes them from the job queue correctly on exit.
\Condor{submit} no longer ignores the machine\_count command for PVM
jobs.
And, a problem which caused pvm\_exit() to hang was diagnosed:
PVM tasks which call pvm\_catchout() to catch the output of
child tasks should be sure to call it again with a NULL argument to
disable output collection before calling pvm\_exit().

\item The change introduced in 6.1.13 to the \Condor{shadow} regarding
when it logged the execute event to the user log produced situations
where the shadow could log other events (like the shadow exception
event) before the execute event was logged.
Now, the \Condor{shadow} will always log an execute event before it
logs any other events.
The timing is still improved over 6.1.12 and older versions, with the
execute event getting logged after the bulk of the job initialization
has finished, right before the job will actually start executing.
However, you will no longer see user logs that contain a ``shadow
exception'' or ``job evicted'' message without a ``job executing''
event, first.

\item \Syscall{stat} and varient calls now go through the file table to
get the correct logical size and access times of buffered files.
Before, \Syscall{stat} used to return zero size on a buffered file that had
not yet been synced to disk.

\end{itemize}

\noindent Known Bugs:

\begin{itemize}

\item On IRIX 6.2, C++ programs compiled with GNU C++ (g++) 2.7.2 and
linked with the Condor libraries (using \Condor{compile}) will not
execute the constructors for any global objects.
There is a work-around for this bug, so if this is a problem for you,
please send email to \Email{condor-admin@cs.wisc.edu}.

\item In HP-UX 10.20, \Condor{compile} will not work correctly with HP's
C++ compiler. 
The jobs might link, but they will produce incorrect output, or die with
a signal such as SIGSEGV during restart after a checkpoint/vacate cycle.
However, the GNU C/C++ and the HP C compilers work just fine.

\item The \Syscall{getrusage} call does not work always as expected in
STANDARD Universe jobs.  
If your program uses \Syscall{getrusage}, it 
could decrease incorrectly by a second
across a checkpoint and restart.  In addition, the time it takes
Condor to restart from a checkpoint is included in the usage times
reported by \Syscall{getrusage}, and it probably should not be.

\end{itemize}


%%%%%%%%%%%%%%%%%%%%%%%%%%%%%%%%%%%%%%%%%%%%%%%%%%%%%%%%%%%%%%%%%%%%%%
\subsection*{\label{sec:New-6-1-14}Version 6.1.14}
%%%%%%%%%%%%%%%%%%%%%%%%%%%%%%%%%%%%%%%%%%%%%%%%%%%%%%%%%%%%%%%%%%%%%%

\noindent New Features:

\begin{itemize}

\item Initial supported added for Red Hat Linux 6.2 (i.e. glibc 2.1.3).

\end{itemize}

\noindent Bugs Fixed:

\begin{itemize}

\item In version 6.1.13, periodic checkpoints would not occur (see the
Known Bugs section for v6.1.13 listed below).  This bug, which only
impacts v6.1.13, has been fixed.

\end{itemize}

\noindent Known Bugs:

\begin{itemize}

\item The \Syscall{getrusage} call does not work properly inside
``standard'' jobs.  
If your program uses \Syscall{getrusage}, it will not report correct values
across a checkpoint and restart.
If your program relies on proper reporting from \Syscall{getrusage}, you
should either use version 6.0.3 or 6.1.10.

\item While Condor now supports many networking calls such as
\Syscall{socket} and \Syscall{connect}, (see the description below of this
new feature added in 6.1.11), on Linux, we cannot at this time support
\Syscall{gethostbyname} and a number of other database lookup calls.
The reason is that on Linux, these calls are implemented by bringing in a
shared library that defines them, based on whether the machine is using
DNS, NIS, or some other database method.
Condor does not support the way in which the C library tries to explicitly
bring in these shared libraries and use them.
There are a number of possible solutions to this problem, but the Condor
developers are not yet agreed on the best one, so this limitation might not
be resolved by 6.1.14.

\item In HP-UX 10.20, \Condor{compile} will not work correctly with HP's
C++ compiler. 
The jobs might link, but they will produce incorrect output, or die with
a signal such as SIGSEGV during restart after a checkpoint/vacate cycle.
However, the GNU C/C++ and the HP C compilers work just fine.

\item When a program linked with the Condor libraries (using \Condor{compile})
is writing output to a file, \Syscall{stat}--and variant calls,
will return zero for the size of the file if the program has not yet
read from the file or flushed the file descriptors.
This is a side effect of the file buffering code in Condor and will be
corrected to the expected semantic.

\item On IRIX 6.2, C++ programs compiled with GNU C++ (g++) 2.7.2 and
linked with the Condor libraries (using \Condor{compile}) will not
execute the constructors for any global objects.
There is a work-around for this bug, so if this is a problem for you,
please send email to \Email{condor-admin@cs.wisc.edu}.

\end{itemize}
%%%%%%%%%%%%%%%%%%%%%%%%%%%%%%%%%%%%%%%%%%%%%%%%%%%%%%%%%%%%%%%%%%%%%%
\subsection*{\label{sec:New-6-1-13}Version 6.1.13}
%%%%%%%%%%%%%%%%%%%%%%%%%%%%%%%%%%%%%%%%%%%%%%%%%%%%%%%%%%%%%%%%%%%%%%

\noindent New Features:

\begin{itemize}

\item Added \Macro{DEFAULT\_IO\_BUFFER\_SIZE} and
\Macro{DEFAULT\_IO\_BUFFER\_BLOCK\_SIZE} to config parameters to allow
the administrator to set the default file buffer sizes for user jobs
in \Condor{submit}.

\item There is no longer any difference in the configuration file
syntax between ``macros'' (which were specified with an ``='' sign)
and ``expressions'' (which were specified with a ``:'' sign).  
Now, all config file entries are treated and referenced as macros. 
You can use either ``='' or ``:'' and they will work the same way. 
There is no longer any problem with forward-referencing macros
(referencing macros you haven't yet defined), so long as they are
eventually defined in your config files (even if the forward reference
is to a macro defined in another config file, like the local config
file, for example).

\item \Condor{vacate} now supports a \Opt{-fast} option that forces
Condor to hard-kill the job(s) immediately, instead of waiting for
them to checkpoint and gracefully shutdown.

\item \Condor{userlog} now displays times in days+hours:minutes format
instead of total hours or total minutes.

\item The \Condor{run} command provides a simple front-end to
\Condor{submit} for submitting a shell command-line as a vanilla
universe job.

\item Solaris 2.7 SPARC, 2.7 INTEL have been added to the
list of ports that now support remote system calls and checkpointing.

\item Any mail being sent from Condor now shows up as having been sent from
the designated Condor Account, instead of root or ``Super User''.

\item The \Condor{submit} ``hold'' command may be used to submit jobs
to the queue in the hold state.  Held jobs will not run until released
with \Condor{release}.

\item It is now possible to use checkpoint servers in remote pools
when flocking even if the local pool doesn't use a checkpoint server.
This is now the default behavior (see the next item).

\item \Macro{USE\_CKPT\_SERVER} now defaults to True if a checkpoint
server is available.  It is usually more efficient to use a checkpoint
server near the execution site instead of storing the checkpoint back
to the submission machine, especially when flocking.

\item All Condor tools that used to expect just a hostname or address 
(\Condor{checkpoint}, \Condor{off}, \Condor{on}, \Condor{restart},
\Condor{reconfig}, \Condor{reschedule}, \Condor{vacate}) to specify
what machine to effect, can now take an optional \Opt{-name} or
\Opt{-addr} in front of each target.
This provides consistancy with other Condor tools that require the
\Opt{-name} or \Opt{-addr} options.
For all of the above mentioned tools, you can still just provide
hostnames or addresses, the new flags are not required.

\item Added \Opt{-pool} and \Opt{-addr} options to \Condor{rm},
\Condor{hold} and \Condor{release}.

\item When you start up the \Condor{master} or \Condor{schedd} as any
user other than ``root'' or ``condor'' on Unix, or ``SYSTEM'' on NT,
the daemon will have a default \Attr{Name} attribute that includes
both the username of the user who the daemon is running as and the
full hostname of the machine where it is running.

\item Clarified our Linux platform support.  We now officially
support the Red Hat 5.2 and 6.x distributions, and although other Linux
distributions (especially those with similar libc versions) may work,
they are not tested or supported.

\item The schedd now periodically updates the run-time counters in the
job queue for running jobs, so if the schedd crashes, the counters
will remain relatively up-to-date.  This is controlled by the
\Macro{WALL\_CLOCK\_CKPT\_INTERVAL} parameter.

\item The \Condor{shadow} now logs the ``job executing'' event in the
user log after the binary has been successfully transfered, so that
the events appear closer to the actual time the job starts running.
This can create some somewhat unexpected log files.  
If something goes wrong with the job's initialization, you might see
an ``evicted'' event before you see an ``executing'' event.

\end{itemize}

\noindent Bugs Fixed:

\begin{itemize}

\item Fixed how we internally handle file names for user jobs. This
fixes a nasty bug due to changing directories between checkpoints.

\item Fixed a bug in our handling of the \Macro{Arguments} macro in
the command file for a job. If the arguments were extremely long, or
there were an extreme number of them, they would get corrupted when the
job was spawned.

\item Fixed DAGMan. It had not worked at all in the previous release.

\item Fixed a nasty bug under Linux where file seeks did not work
correctly when buffering was enabled.

\item Fixed a bug where \Condor{shadow} would crash while sending job
completion e-mail forcing a job to restart multiple times and the user
to get multiple completion messages.

\item Fixed a long standing bug where Fortran 90 would occasionally
truncate its output files to random sizes and fill them with zeros.

\item Fixed a bug where \Syscall{close} did not propogate its return
value back to the user job correctly.

\item If a SIGTERM was delivered to a \Condor{shadow}, it used to
remove the job it was running from the job queue, as if \Condor{rm}
had been used.
This could have caused jobs to leave the queue unexpectedly.
Now, the \Condor{shadow} ignores SIGTERM (since the \Condor{schedd}
knows how to gracefully shutdown all the shadows when it gets a
SIGTERM), so jobs should no longer leave the queue prematurely.
In addition, on a SIGQUIT, the shadow now does a fast shutdown, just
like the rest of the Condor daemons.

\item Fixed a number of bugs which caused checkpoint restarts
to fail on some releases of Irix 6.5 (for example, when migrating from
a mips4 to a mips3 CPU or when migrating between machines with
different pagesizes).

\item Fixed a bug in the implementation of the \Syscall{stat} family
of remote system calls on Irix 6.5 which caused file opens in Fortran
programs to sometimes fail.

\item Fixed a number of problems with the statistics reported in the
job completion email and by \Condor{q} \Opt{-goodput}, including the
number of checkpoints and total network usage.  Correct values will
now be computed for all new jobs.

\item Changes in \Macro{USE\_CKPT\_SERVER} and
\Macro{CKPT\_SERVER\_HOST} no longer cause problems for jobs in the
queue which have already checkpointed.

\item Many of the Condor administration tools had a bug where they
would suffer a segmentation violation if you specified a \Opt{-pool} 
option and did not specify a hostname.
This case now results in an error message instead.

\item Fixed a bug where the \Condor{schedd} could die with a
segmentation violation if there was an error mapping an IP address
into a hostname.

\item Fixed a bug where resetting the time in a large negative direction
caused the \Condor{negotiator} to have a floating point error on some
platforms.

\item Fixed \Condor{q}'s output so that certain arguments are not ignored.

\item Fixed a bug in \Condor{q} where issuing a \Opt{-global} with a
fairly restrictive \Opt{-constraint} argument would cause garbage to be
printed to the terminal sometimes.

\item Fixed a bug which caused jobs to exit without completing a
checkpoint when preempted in the middle of a periodic checkpoint.
Now, the jobs will complete their periodic checkpoint in this case
before exiting.
\end{itemize}

\noindent Known Bugs:

\begin{itemize}

\item Periodic checkpoints do not occur.  Normally, when the config
file attribute \Macro{PERIODIC\_CHECKPOINT} evaluates to True, 
Condor performs a periodic checkpoint of the running job.  This
bug has been fixed in v6.1.14.  \Note there is a work-around to permit
periodic checkpoints to occur in v6.1.13: include the attribute name
``PERIODIC\_CHECKPOINT'' to the attributes 
listed in the \Macro{STARTD\_EXPRS} entry in the config file.

\item The \Syscall{getrusage} call does not work properly inside
``standard'' jobs.  
If your program uses \Syscall{getrusage}, it will not report correct values
across a checkpoint and restart.
If your program relies on proper reporting from \Syscall{getrusage}, you
should either use version 6.0.3 or 6.1.10.

\item While Condor now supports many networking calls such as
\Syscall{socket} and \Syscall{connect}, (see the description below of this
new feature added in 6.1.11), on Linux, we cannot at this time support
\Syscall{gethostbyname} and a number of other database lookup calls.
The reason is that on Linux, these calls are implemented by bringing in a
shared library that defines them, based on whether the machine is using
DNS, NIS, or some other database method.
Condor does not support the way in which the C library tries to explicitly
bring in these shared libraries and use them.
There are a number of possible solutions to this problem, but the Condor
developers are not yet agreed on the best one, so this limitation might not
be resolved by 6.1.14.

\item In HP-UX 10.20, \Condor{compile} will not work correctly with HP's
C++ compiler. 
The jobs might link, but they will produce incorrect output, or die with
a signal such as SIGSEGV during restart after a checkpoint/vacate cycle.
However, the GNU C/C++ and the HP C compilers work just fine.

\item When writing output to a file, \Syscall{stat}--and variant calls,
will return zero for the size of the file if the program has not yet
read from the file or flushed the file descriptors,
This is a side effect of the file buffering code in Condor and will be
corrected to the expected semantic.

\item On IRIX 6.2, C++ programs compiled with GNU C++ (g++) 2.7.2 and
linked with the Condor libraries (using \Condor{compile}) will not
execute the constructors for any global objects.
There is a work-around for this bug, so if this is a problem for you,
please send email to \Email{condor-admin@cs.wisc.edu}.

\end{itemize}

%%%%%%%%%%%%%%%%%%%%%%%%%%%%%%%%%%%%%%%%%%%%%%%%%%%%%%%%%%%%%%%%%%%%%%
\subsection*{\label{sec:New-6-1-12}Version 6.1.12}
%%%%%%%%%%%%%%%%%%%%%%%%%%%%%%%%%%%%%%%%%%%%%%%%%%%%%%%%%%%%%%%%%%%%%%

Version 6.1.12 fixes a number of bugs from version 6.1.11.
If you linked your ``standard'' jobs with version 6.1.11, you should
upgrade to 6.1.12 and re-link your jobs (using \Condor{compile}) as soon as
possible.

\noindent New Features:

\begin{itemize}

\item None.

\end{itemize}

\noindent Bugs Fixed:

\begin{itemize}

\item A number of system calls that were not being trapped by the Condor
libraries in version 6.1.11 are now being caught and sent back to the
submit machine.
Not having these functions being executed as remote system calls prevented
a number of programs from working, in particular Fortran programs, and
many programs on IRIX and Solaris platforms.

\item Sometimes submitted jobs report back as having no owner and have
\Bold{-????-} in the status line for the job. This has been fixed.

\item \Condor{q} \Opt{-io} has been fixed in this release.

\end{itemize}

\noindent Known Bugs:

\begin{itemize}

\item The \Syscall{getrusage} call does not work properly inside
``standard'' jobs.  
If your program uses \Syscall{getrusage}, it will not report correct values
across a checkpoint and restart.
If your program relies on proper reporting from \Syscall{getrusage}, you
should either use version 6.0.3 or 6.1.10.

\item While Condor now supports many networking calls such as
\Syscall{socket} and \Syscall{connect}, (see the description below of this
new feature added in 6.1.11), on Linux, we cannot at this time support
\Syscall{gethostbyname} and a number of other database lookup calls.
The reason is that on Linux, these calls are implemented by bringing in a
shared library that defines them, based on whether the machine is using
DNS, NIS, or some other database method.
Condor does not support the way in which the C library tries to explicitly
bring in these shared libraries and use them.
There are a number of possible solutions to this problem, but the Condor
developers are not yet agreed on the best one, so this limitation might not
be resolved by 6.1.13.

\item In HP-UX 10.20, \Condor{compile} will not work correctly with HP's
C++ compiler. 
The jobs might link, but they will produce incorrect output, or die with
a signal such as SIGSEGV during restart after a checkpoint/vacate cycle.
However, the GNU C/C++ and the HP C compilers work just fine.

\item When writing output to a file, \Syscall{stat}--and variant calls,
will return zero for the size of the file if the program has not yet
read from the file or flushed the file descriptors,
This is a side effect of the file buffering code in Condor and will be
corrected to the expected semantic.

\item On IRIX 6.2, C++ programs compiled with GNU C++ (g++) 2.7.2 and
linked with the Condor libraries (using \Condor{compile}) will not
execute the constructors for any global objects.
There is a work-around for this bug, so if this is a problem for you,
please send email to \Email{condor-admin@cs.wisc.edu}.

\item The \Opt{-format} option in \Condor{q} has no effect when querying
remote machines with the \Opt{-n} option.

\item \Condor{dagman} does not work at all in this release. 
The behaviour of its failure is to exit immediately with a success and
to not perform any work. It will be fixed in the next release of Condor.

\end{itemize}


%%%%%%%%%%%%%%%%%%%%%%%%%%%%%%%%%%%%%%%%%%%%%%%%%%%%%%%%%%%%%%%%%%%%%%
\subsection*{\label{sec:New-6-1-11}Version 6.1.11}
%%%%%%%%%%%%%%%%%%%%%%%%%%%%%%%%%%%%%%%%%%%%%%%%%%%%%%%%%%%%%%%%%%%%%%

\noindent New Features:

\begin{itemize}

\item \Condor{status} outputs information for held jobs instead of
MaxRunningJobs when supplied with \Opt{-schedd} or \Opt{-submitter}.

\item \Condor{userprio} now prints 4 digit years (for Y2K compiance). 
If you give a two digit date, it also will assume that 1/1/00 is 1/1/2000
and not 1/1/1900.

\item IRIX 6.5 has been added to the list of ports that now support
remote system calls and checkpointing.

\item \Condor{q} has been fixed to be faster and much more memory
efficient.  This is much more obvious when getting the queue from
\Condor{schedd}'s that have more than 1000 jobs.

\item Added support for support for socket() and pipe() in standard
jobs.  Both sockets and pipes are created on the executing machine.
Checkpointing is deferred anytime a socket or pipe is open.

\item Added limited support for select() and poll() in standard jobs.
Both calls will work only on files opened locally.

\item Added limited support for fcntl() and ioctl() in standard jobs.
Both calls will be performed remotely if the control-number is understood
and the third argument is an integer.

\item Replaced buffer implementation in standard jobs.
The new buffer code reads and writes variable sized chunks.
It will never issue a read to satisfy a write.  Buffering is enabled
by default.

\item Added extensive feedback on I/O performance in the user's email.

\item Added \Opt{-io} option to \Condor{q} to show I/O statistics.

\item Removed libckpt.a and libzckpt.a.  To build for standalone
checkpointing, just do a regular \Condor{compile}.
No -standalone option is necessary.

\item The checkpointing library now only re-opens files when they are
actually used.  If files or other needed resources cannot be found
at restart time, the checkpointer will fail with a verbose error.

\item The \Attr{RemoteHost} and \Attr{LastRemoteHost} attributes in
the job classad now contain hostnames instead IP address and port
numbers.  The \Opt{-run} option of older versions of \Condor{q} is not
compatible with this change.

\item Condor will now automatically check for compatibility between
the version of the Condor libraries you have linked into a standard
job (using \Condor{compile}) and the version of the \Condor{shadow}
installed on your submit machine.
If they are incompatible, the \Condor{shadow} will now put your job on
hold.  
Unless you set ``Notification = Never'' in your submit file, Condor
will also send you email explaining what went wrong and what you can
do about it.

\item All Condor daemons and tools now have a \Attr{CondorPlatform}
string, which shows which platform a given set of Condor binaries was
built for.
In all places that you used to see \Attr{CondorVersion}, you will now
see both \Attr{CondorVersion} and \Attr{CondorPlatform}, such as in
each daemon's ClassAd, in the output to a \Opt{-version} option (if
supported), and when running \Prog{ident} on a given Condor binary. 
This string can help identify situations where you are running the 
wrong version of the Condor binaries for a given platform (for
example, running binaries built for Solaris 2.5.1 on a Solaris 2.6
machine).   

\item Added commented-out settings in the default
\File{condor\_config} file we ship for various SMP-specific settings
in the \Condor{startd}.
Be sure to read section~\ref{sec:Configuring-SMP} on ``Configuring the
Startd for SMP Machine'' on page~\pageref{sec:Configuring-SMP} for
details about using these settings. 

\item \Condor{rm}, \Condor{hold}, and \Condor{release} all support
\Opt{-help} and \Opt{-version} options now.

\end{itemize}

\noindent Bugs Fixed:

\begin{itemize}

\item A race condition which could cause the \Condor{shadow} to not
exit when its job was removed has been fixed.
This bug would cause jobs that had been removed with \Condor{rm} to
remain in the queue marked as status ``X'' for a long time.
In addition, Condor would not shutdown quickly on hosts that had hit
this race condition, since the \Condor{schedd} wouldn't exit until all
of its \Condor{shadow} children had exited.

\item A signal race condition during restart of a Condor job has
been fixed.

\item In a Condor linked job, \Syscall{getdomainname} is now
supported. 

\item IRIX 6.5 can give negative time reports for how long a process has been
running. We account for that now in our statistics about usage times.

\item The \Condor{status} memory error introduced in version 6.1.10
has been fixed.

\item The \Macro{DAEMON\_LIST} configuration setting is now case
insensitive.

\item Fixed a bug where the \Condor{schedd}, under rare circumstances,
cause another schedd's jobs not to be matched.

\item The free disk space is now properly computed on Digital Unix.
This fixed problems where the \Attr{Disk} attribute in the
\condor{startd} classad reported incorrect values.

\item The config file parser now detects incremental macro definitions
correctly (see section~\ref{sec:Config-File-Macros} on
page~\pageref{sec:Config-File-Macros}).  Previously, when a macro (or
expression) being defined was a substring of a macro (or expression)
being referenced in its definition, the reference would be erroneously
marked as an incremental definition and expanded immediately.  The
parser now verifies that the entire strings match.

\end{itemize}

\noindent Known Bugs:

\begin{itemize}

\item The output for \condor{q} \Opt{-io} is incorrect and will likely show
zeroes for all values.  A fixed version will appear in the next release.

\end{itemize}

%%%%%%%%%%%%%%%%%%%%%%%%%%%%%%%%%%%%%%%%%%%%%%%%%%%%%%%%%%%%%%%%%%%%%%
\subsection*{\label{sec:New-6-1-10}Version 6.1.10}
%%%%%%%%%%%%%%%%%%%%%%%%%%%%%%%%%%%%%%%%%%%%%%%%%%%%%%%%%%%%%%%%%%%%%%

\noindent New Features:

\begin{itemize}

\item \Condor{q} now accepts \texttt{-format} parameters like \Condor{status}

\item \Condor{rm}, \Condor{hold} and \Condor{release} accept
  \texttt{-constraint} parameters like \Condor{status}

\item \Condor{status} now sorts displayed totals by the first column.
(This feature introduced a bug in \Condor{status}.  See ``Known Bugs''
below.)

\item Condor version 6.1.10 introduces ``clipped'' support for Sparc
Solaris version 2.7.
This version does not support checkpointing or remote system calls.
Full support for Solaris 2.7 will be released soon.

\item Introduced code to enable Linux to use the standard C library's
I/O buffering again, instead of relying on the Condor I/O buffering
code (which is still in beta testing).  

\end{itemize}

\noindent Bugs Fixed:

\begin{itemize}

\item The bug in checkpointing introduced in version 6.1.9 has been
fixed.
Checkpointing will now work on all platforms, as it always used to.  
Any jobs linked with the 6.1.9 Condor libraries will need to be
relinked with \Condor{compile} once version 6.1.10 has been installed
at your site. 

\end{itemize}

\noindent Known Bugs:

\begin{itemize}

\item The \AdAttr{CondorLoadAvg} attribute in the \Condor{startd} has
some problems in the way it is computed.
The CondorLoadAvg is somewhat inaccurate for the first minute a job
starts running, and for the first minute after it completes.
Also, the computation of CondorLoadAvg is very wrong on NT.
All of this will be fixed in a future version.

\item A memory error may cause \Condor{status} to die with SIGSEGV
(segmentation violation) when displaying totals or cause incorrect
totals to be displayed.  This will be fixed in version 6.1.11.

\end{itemize}


%%%%%%%%%%%%%%%%%%%%%%%%%%%%%%%%%%%%%%%%%%%%%%%%%%%%%%%%%%%%%%%%%%%%%%
\subsection*{\label{sec:New-6-1-9}Version 6.1.9}
%%%%%%%%%%%%%%%%%%%%%%%%%%%%%%%%%%%%%%%%%%%%%%%%%%%%%%%%%%%%%%%%%%%%%%

\noindent New Features:

\begin{itemize}

\item Added full support for Linux 2.0.x and 2.2.x kernels using
libc5, glibc20 and glibc21.
This includes support for Red Hat 6.x, Debian 2.x and other popular
Linux distributions.
Whereas the Linux machines had once been fragmented across libc5 and
GNU libc, they have now been reunified.
This means there is no longer any need for the ``LINUX-GLIBC'' OpSys
setting in your pool: all machines will now show up as ``LINUX''.
Part of this reunification process was the removal of dynamically
linked user jobs on Linux.
\Condor{compile} now forces static linking of your Standard Universe
Condor jobs. 
Also, please use \Condor{compile} on the same machine on which you
compiled your object files.

\item Added \Condor{qedit} utility to allow users to modify job
attributes after submission.  See the new manual page on
page~\pageref{man-condor-qedit}.

\item Added \OptArg{{-runfor}{minutes}} option to daemonCore to have
the daemon gracefully shut down after the given number of minutes.

\item Added support for statfs(2) and fstatfs(2) in user jobs. We support 
only the fields
\textit{f\_bsize, f\_blocks, f\_bfree, f\_bavail, f\_files, f\_ffree} from
the structure statfs. This is still in the experimental stage.

\item Added the \Opt{-direct} option to \Condor{status}.
After you give \Opt{-direct}, you supply a hostname, and
\Condor{status} will query the \Condor{startd} on the specified host
and display information directly from there, instead of querying the
\Condor{collector}.
See the manual page on page~\pageref{man-condor-submit} for details. 

\item Users can now define \Macro{NUM\_CPUS} to override the automatic
computation of the number of CPUs in your machine.
Using this config setting can cause unexpected results, and is not
recommended. 
This feature is only provided for sites that specifically want this
behavior and know what they are doing.

\item The \Opt{-set} and \Opt{-rset} options to \Condor{config\_val}
have been changed to allow administrators to set both macros and
expressions.
Previously, \Condor{config\_val} assumed you wanted to set
expressions.
Now, these two options each take a single argument, the string
containing exactly what you would put into the config file, so you can
specify you want to create a macro by including an ``='' sign, or an
expression by including a ``:''.
See section~\ref{sec:Intro-to-Config-Files} on
page~\pageref{sec:Intro-to-Config-Files} for details on macros
vs. expressions.
See the \Condor{config\_val} man page on
page~\pageref{man-condor-config-val} for details on
\Condor{config\_val}.  

\item If the directory you specified for LOCK (which holds lock files
used by Condor) doesn't exist, Condor will now try to create that
directory for you instead of giving up right away.

\item If you change the \Attr{COLLECTOR\_HOST} setting and reconfig
the \Condor{startd}, the startd will ``invalidate'' its ClassAds at
the old collector before it starts reporting to the new one.

\end{itemize}

\noindent Bugs Fixed:

\begin{itemize}

\item Fixed a major bug dealing with the group access a Condor job is
started with.
Now, Condor jobs are started with all the groups the job's owner is
in, not just their default group.
This also fixes a security hole where user jobs could be started up in
access groups they didn't belong to.

\item Fixed a bug where there was a needless limitation on the number of open
file descriptors a user job could have.

\item Fixed a standalone checkpointing bug where we weren't blocking signals
in critical sections and causing file table corruption at checkpoint
time.

\item Fixed a linker bug on Digital Unix 4.0 concerning fortran where
the linker would fail on \_\_uname and \_\_sigsuspend.

\item Fixed a bug in \Condor{shadow} that would send incorrect job
completion email under Linux.

\item Fixed a bug in the remote system call of \Syscall{fchdir} that caused
a garbage file descriptor to be used in Standard Universe jobs.

\item Fixed a bug in the \Condor{shadow} which was causing \Condor{q}
\Opt{-goodput} to display incorrect values for some jobs.

\item Fixed some minor bugs and made some minor enhancements in the
\Condor{install} script.
The bugs included a typo in one of the questions asked, and incorrect
handling for the answers of a few different questions.
Also, if DNS is misconfigured on your system, \Condor{install} will
try a few ways to find your fully qualified hostname, and if it still
can't determine the correct hostname, it will prompt the user for it. 
In addition, we now avoid one installation step in cases were it is
not needed. 

\item Fixed a rare race condition that could delay the completion of
large clusters of short running jobs. 

\item Added more checking to the various arguments that might be
passed to \Condor{status}, so that in the case of bad input,
\Condor{status} will print an error message and exit, instead of
performing a segmentation fault.
Also, when you use the \Opt{-sort} option, \Condor{status} will only
display ClassAds where the attributes you use to sort are defined.

\item Fixed a bug in the handling of the config files created by
using the \Opt{-set} or \Opt{-rset} options to \Condor{config\_val}.
Previously, if you manually deleted the files that were created, you
could cause the affected Condor daemon to have a segmentation fault.
Now, the daemons simply exit with a fatal error but still have a
chance to clean up.

\item Fixed a bug in the \Opt{-negotiator} option for most Condor
tools that was causing it to get the wrong address.

\item Fixed a couple of bugs in the \Condor{master} that could cause
improper shutdowns. 
There were cases during shutdown where we would restart a daemon
(because we previously noticed a new executable, for example).
Now, once you begin a shutdown, the \Condor{master} will not restart
anything. 
Also, fixed a rare bug that could cause the \Condor{master} to stop
checking the timestamps on a daemon.

\item Fixed a minor bug in the \Opt{-owner} option to
\Condor{config\_val} that was causing \Condor{init} not to work.

\item Fixed a bug where the \Condor{startd}, while it was already
shutting down, was allowing certain actions to succeed that should
have failed.
For example, it allowed itself to be matched with a user looking for
available machines, or to begin a new PVM task.

\end{itemize}

\noindent Known Bugs:

\begin{itemize}

\item The \AdAttr{CondorLoadAvg} attribute in the \Condor{startd} has
some problems in the way it is computed.
The CondorLoadAvg is somewhat inaccurate for the first minute a job
starts running, and for the first minute after it completes.
Also, the computation of CondorLoadAvg is very wrong on NT.
All of this will be fixed in a future version.

\item There is a serious bug in checkpointing when using Condor's
I/O buffering for ``standard'' jobs.
By default, Linux uses Condor buffering in version 6.1.9 for all
standard jobs.
The bug prevents checkpointing from working more than once.
This renders the \Condor{vacate} and \Condor{checkpoint} commands
useless, and jobs will just be killed without a checkpoint when
machine owners come back to their machines.

\end{itemize}


%%%%%%%%%%%%%%%%%%%%%%%%%%%%%%%%%%%%%%%%%%%%%%%%%%%%%%%%%%%%%%%%%%%%%%
\subsection*{\label{sec:New-6-1-8}Version 6.1.8}
%%%%%%%%%%%%%%%%%%%%%%%%%%%%%%%%%%%%%%%%%%%%%%%%%%%%%%%%%%%%%%%%%%%%%%

\begin{itemize}

\item Added \Term{file\_remaps} as command in the job submit file given to
STANDARD universe jobs.
A Job can now specify that it would like to have files be remapped
from one file to another.
In addition you can specify that files should be read from the local machine
by specifing them.
See the \Condor{submit} manual page on page~\pageref{man-condor-submit} for
more details.

\item Added \Term{buffer\_size} and \Term{buffer\_block\_size} so that STANDARD
universe jobs can specify that they wish to have I/O buffering turned on.
Without buffering, all I/O requests in the STANDARD universe are sent back
over the network to be executed on the submit machine.  
With buffering, read ahead, write behind, and seek batch buffering is
performed to minimize network traffic and latency.
By default, jobs do not specify buffering, however, for many situations buffering
can drastically increase throughput.  See the \Condor{submit} manual page
on page~\pageref{man-condor-submit} for more details.

\item The \Condor{schedd} is much more memory efficient handling clusters
with hundreds/thousands of jobs.  
If you submit large clusters, your submit machine will only use a fraction
of the amount of RAM it used to require.  
\Note The memory savings will only be realized for new clusters submitted
after the upgrade to v6.1.8 -- clusters which previously existed in the
queue at upgrade time will still use the same amount of RAM in the
\Condor{schedd}.

\item Submitting jobs, especially submitting large clusters containing many
jobs, is much faster.

\item Added a \Opt{-goodput} option to \Condor{q}, which displays
statistics about the execution efficiency of STANDARD universe jobs.

\item Added FS\_REMOTE method of user authentication to possible values
of the configuration option \Macro{AUTHENTICATION\_METHODS} to fix problems
with using the \Opt{-r} remote scheduler option of \Condor{submit}.
Additionally, the user authentication protocol has changed, so previous
versions of Condor programs cannot co-exist with this new protocol.

\item Added a new utility and documentation for \Condor{glidein} which uses 
Globus resources to extend your local pool to use remote Globus machines as 
part of your Condor pool.

\item Fixed more bugs in the handling of the stat() system call
and its relatives on Linux with glibc.
This was causing problems mainly with Fortran I/O, though other I/O
related problems on glibc Linux will probably be solved now.

\item Fixed a bug in various Condor tools (\Condor{status},
\Condor{user\_prio}, \Condor{config\_val}, and \Condor{stats}) that
would cause them to seg fault on bad input to the \Opt{-pool} option. 

\item Fixed a bug with the \Opt{-rset} option to \Condor{config\_val} which
could crash the Condor daemon whose configuration was being changed.

\item Added \Term{allow\_startup\_script} command to the job submit
description file which is given to \Condor{submit}.  This allows the
submission of a startup script to the STANDARD universe.  See 

\item Fixed a bug in the \Condor{schedd} where it would get into an
infinite loop if the persistant log of the job queue got corrupted.  
The \Condor{schedd} now correctly handles corrupted log files.

\item The full release tar file now contains a \File{dagman}
subdirectory in the \File{examples} directory.
This subdirectory includes an example DAGMan job, including a README
(in both ASCII and HTML), a Makefile, and so on.

\item Condor will now insert an environment variable, \Env{CONDOR\_VM}, into
the environment of the user job.  
This variable specifies which SMP ``virtual machine'' the job was started on.
It will equal either vm1, vm2, vm3, \Dots , depending upon which virtual
machine was matched.
On a non-SMP machine, \Env{CONDOR\_VM} will always be set to vm1.

\item Fixed some timing bugs introduced in v6.1.6 which could occur when
Condor tries to simultaneously start a large number of jobs submitted from a
single machine.

\item Fixed bugs when Condor is told to gracefully shutdown; Condor no
longer starts up new jobs when shutting down.  Also, the \Condor{schedd}
progressively checkpoints running jobs during a graceful shutdown instead of
trying to vacate all the job simultaneously.  The rate at which the shutdown
occurs is controlled by the \Macro{JOB\_START\_DELAY} configuration
parameter (see page~\pageref{param:JobStartDelay}).

\item Fixed a bug which could cause the \Condor{master} process to exit if
the Condor daemons have been hung for a while by the operating system (if,
for instance, the LOG directory was placed on an NFS volume and the NFS
server is down for an extended period).

\item Previously, removing a large number of jobs with \Condor{rm} would
result in the \Condor{schedd} being unresponsive for a period of time
(perhaps leading to timeouts when running \Condor{q}).  The \Condor{schedd}
has been improved to multitask the removal of jobs while servicing new
requests.

\item Added new configuration parameter \Macro{COLLECTOR\_SOCKET\_BUFSIZE}
which controls the size of TCP/IP buffers used by the \Condor{collector}.
For more info, see section~ref{param:CollectorSocketBufsize} on
page~pageref{param:CollectorSocketBufsize}.

\item Fixed a bug with the \Opt{-analyze} option to \Condor{q}: in some
cases, the RANK expression would not be evaluated correctly.  This could
cause the output from \Opt{-analyze} to be in error.

\item When running on a multi-CPU (SMP) Hewlett-Packard machine, fixed bugs
computing the system load average.

\item Fixed bug in \Condor{q} which could cause the RUN\_TIME reported to
be temporarily incorrect when jobs first start running. 

\item The \Condor{startd} no longer rapidly sends multiple ClassAds one
right after another to the Central Manager when its state/activity is in
rapid transition.  Also, on SMP machines, the \Condor{startd} will only send
updates for 4 nodes per second (to avoid overflowing the central manager when
reporting the state of a very large SMP machine with dozens of CPUs).

\item Reading a parameter with \Condor{config\_val} is now allowed from any
machine with Host-IP READ permission.
Previsouly, you needed ADMINISTRATOR permission.  
Of course, setting a parameter still requires ADMINISTRATOR permission.

\item Worked around a bug in the StreamTokenizer Java class from Sun
that we use in the CondorView client Java applet.
The bug would cause errors if usernames or hostnames in your pool
contained ``-'' or ``\_'' characters.
The CondorView applet now gets around this and properly displays all
data, including entries with the ``bad'' characters.

\end{itemize}

%%%%%%%%%%%%%%%%%%%%%%%%%%%%%%%%%%%%%%%%%%%%%%%%%%%%%%%%%%%%%%%%%%%%%%
\subsection*{\label{sec:New-6-1-7}Version 6.1.7}
%%%%%%%%%%%%%%%%%%%%%%%%%%%%%%%%%%%%%%%%%%%%%%%%%%%%%%%%%%%%%%%%%%%%%%

\Note Version 6.1.7 only adds support for platforms not supported in
6.1.6.  
There are no bug fixes, so there are no binaries released for any
other platforms. 
You do not need 6.1.7 unless you are using one of the two platforms we
released binaries for.

\begin{itemize}

\item Added ``clipped'' support for Alpha Linux machines running the
2.0.X kernel and glibc 2.0.X (such as Red Hat 5.X).
We do not yet support checkpointing and remote system calls on this
platform, but we can start ``vanilla'' jobs.
See section~\ref{sec:Choosing-Universe} on
page~\pageref{sec:Choosing-Universe} for details on vanilla
vs. standard jobs.

\item Re-added support for Intel Linux machines running the 2.0.X
Linux kernel, glibc 2.0.X, using the GNU C compiler (gcc/g++ 2.7.X) or
the EGCS compilers (versions 1.0.X, 1.1.1 and 1.1.2).
This includes Red Hat 5.X, and Debian 2.0.
\Bold{Red Hat 6.0 and Debian 2.1 are not yet supported, since they use
glibc 2.1.X and the 2.2.X Linux kernel.}
Future versions of Condor will support all combinations of kernels,
compilers and versions of libc.

\end{itemize}


%%%%%%%%%%%%%%%%%%%%%%%%%%%%%%%%%%%%%%%%%%%%%%%%%%%%%%%%%%%%%%%%%%%%%%
\subsection*{\label{sec:New-6-1-6}Version 6.1.6}
%%%%%%%%%%%%%%%%%%%%%%%%%%%%%%%%%%%%%%%%%%%%%%%%%%%%%%%%%%%%%%%%%%%%%%

\begin{itemize}

\item Added \Term{file\_remaps} as command in the job submit file given to
\Condor{submit}.
This allows the user to explicitly specify where to find a given file (e.g.
either on the submit or execute machine), as well as remap file access to a
different filename altogether.

\item Changed the way that \Condor{master} spawns daemons and
\Condor{preen} which allows you to specify command line arguments for
any of them, though a \MacroNI{SUBSYS\_ARGS} setting.
Previously, when you specified \Macro{PREEN}, you added the command
line arguments directly to that setting, but that caused some
problems, and only worked for \Condor{preen}.
\Bold{Once you upgrade to version 6.1.6, if you continue to use your
old \File{condor\_config} files, you must change the \Macro{PREEN}
setting to remove any arguments you have defined and place those
arguments into a separate config setting, \Macro{PREEN\_ARGS}.}
See section~\ref{sec:Master-Config-File-Entries}, ``\condor{master}
Config File Entries'', on
page~\pageref{sec:Master-Config-File-Entries} for more details.

\item Fixed a very serious bug in the Condor library linked in with
\Condor{compile} to create standard jobs that was causing
checkpointing to fail in many cases.  
Any jobs that were linked with the 6.1.5 Condor libraries should
probably be removed, re-linked, and re-submitted. 

\item Fixed a bug in \Condor{userprio} that was introduced in version
6.1.5 that was preventing it from finding the address of the
\Condor{negotiator} for your pool.

\item Fixed a bug in \Condor{stats} that was introduced in version
6.1.5 that was preventing it from finding the address of the
\Condor{collector} for your pool.

\item Fixed a bug in the way the \Opt{-pool} option was handled by
many Condor tools that was introduced in version 6.1.5. 


\item \Condor{q} now displays job \emph{allocation time} by default, instead
of displaying CPU time.  
Job allocation time, or RUN\_TIME, is the amount of wall-clock time the job
has spent running.  
Unlike CPU time information which is only updated when a job is
checkpointed, the allocation time displayed by \Condor{q} is continuously
updated, even for vanilla universe jobs.  
By default, the allocation time displayed will be the total time across all
runs of the job.  
The new \Opt{-currentrun} option to \Condor{q} can be used to display the
allocation time for solely the current run of the job.
Additionally, the \Opt{-cputime} option can be used to view job CPU times as
in earlier versions of Condor.

\item \Condor{q} will display an error message if there is a timeout
fetching the job queue listing from a \condor{schedd}.  Previously,
\Condor{q} would simply list the queue as empty upon a communication error.

\item The \condor{schedd} daemon has been updated to verify all queue access
requests via Condor's IP/Host-Based Security mechanism (see
section~\ref{sec:Host-Security}).

\item Fixed a bug on platforms which require the \Condor{kbdd} (currently
Digital Unix and IRIX).  
This bug could have allowed Condor to start a job within the first five
minutes after the Condor daemons had been started, even if there is a user
typing on the keyboard.

\item \Condor{release} now gives an error message if the user tries to
release a job which either does not exist or is not in the hold state.

\item Added a new config file parameter, \Macro{USER\_JOB\_WRAPPER}, which
allows administrators to specify a file to act as a ``wrapper'' script
around all jobs started by Condor. 
See inside section~\ref{param:UserJobWrapper}, on 
page~\pageref{sec:Starter-Config-File-Entries}, for more details.

\item \Condor{dagman} now permits the backslash character (``\Bs'') to be used
as a line-continuation character for DAG Input Files, just like the
\condor{config} files.

\item The Condor version string is now included in all Condor
libraries.
You can now run \Prog{ident} on any program linked with
\Condor{compile} to view which version of the Condor libraries you
linked with.
In addition, the format of the version string changed in 6.1.6.
Now, the identifier used is ``CondorVersion'' instead of ``Version''
to prevent any potential ambiguity.
Also, the format of the date changed slightly.

\item The SMP startd can now handle dynamic reconfiguration of the
number of each type of virtual machine being reported.
This allows you, during the normal running of the startd, to increase
or decrease the number of CPUs that Condor is using.
If you reconfigure the startd to use less CPUs than it currently has
under its control, it will first remove CPUs that have no Condor jobs
running on them.
If more CPUs need to be evicted, the startd will checkpoint jobs and
evict them in reverse rank order (using the startd's \Macro{Rank}
expression).
So, the lower the value of the rank, the more likely a job will be
kicked off.

\item The SMP startd contrib module's \Condor{starter} no longer makes
a call that was causing warning messages about ``ERROR: Unknown System
Call (-58) - system call not supported by Condor'' when used with the
6.0.X \Condor{shadow}.
This was a harmless call, but removing the call prevents the error
message.

\item The SMP contrib module now includes the \Condor{checkpoint} and
\Condor{vacate} programs, which allow you to vacate or checkpoint jobs
on individual CPUs on the SMP, instead of checkpointing or vacating
everything.  
You can now use ``\condor{vacate} vm1@hostname'' to just vacate the
first virtual machine, or ``\condor{vacate} hostname'' to vacate all
virtual machines. 

\item Added support for SMP Digital Unix (Alpha) machines.

\item Fixed a bug that was causing an overflow in the computation of
free disk and swap space on Digital Unix (Alpha) machines.

\item The \Condor{startd} and \Condor{schedd} now can ``invalidate''
their classads from the collector.
So, when a daemon is shut down, or a machine is reconfigured to 
advertise fewer virtual machines, those changes will be instantly
visible with \Condor{status}, instead of having to wait 15 minutes for
the stale classads to time-out.

\item The \Condor{schedd} no longer forks a child process (a ``schedd
agent'') to claim available \Condor{startd}s.  
You should no longer see multiple \condor{schedd} processes running on
your machine after a negotiation cycle.
This is now accomplished in a non-blocking manner within the
\Condor{schedd} itself.

\item The startd now adds an \Attr{VirtualMachineID} attribute to
each virtual machine classad it advertises.
This is just an integer, starting at 1, and increasing for every
different virtual machine the startd is representing.
On regular hosts, this is the only ID you will ever see.
On SMP hosts, you will see the ID climb up to the number of different
virtual machines reported.
This ID can be used to help write more complex policy expressions on
SMP hosts, and to easily identify which hosts in your pool are in fact
SMP machines.

\item Modified the output for \Condor{q} -run for scheduler and PVM
universe jobs.  The host where the scheduler universe job is running
is now displayed correctly.  For PVM jobs, a count of the current
number of hosts where the job is running is displayed.

\item Fixed the \Condor{startd} so that it no longer prints lots of
ProcAPI errors to the log file when it is being run as non-root.

\item \Macro{FS\_PATHNAME} and \Macro{VOS\_PATHNAME} are no longer
used.  AFS support now works similar to NFS support, via the
\Macro{FILESYSTEM\_DOMAIN} macro.

\item Fixed a minor bug in the \File{Condor.pm} perl module that was
causing it to be case-sensitive when parsing the Condor submit file.
Now, the perl module is properly case-insensitive, as indicated in the
documentation.

\end{itemize}

%%%%%%%%%%%%%%%%%%%%%%%%%%%%%%%%%%%%%%%%%%%%%%%%%%%%%%%%%%%%%%%%%%%%%%
\subsection*{\label{sec:New-6-1-5}Version 6.1.5}
%%%%%%%%%%%%%%%%%%%%%%%%%%%%%%%%%%%%%%%%%%%%%%%%%%%%%%%%%%%%%%%%%%%%%%

\begin{itemize}

\item Fixed a nasty bug in \Condor{preen} that would cause it to
remove files it shouldn't remove if the \Condor{schedd} and/or
\Condor{startd} were down at the time \Condor{preen} ran.
This was causing jobs to mysteriously disappear from the job queue.

\item Added preliminary support to Condor for running on machines with
multiple network interfaces.
On such machines, users can specify the IP address Condor should use
in the \Macro{NETWORK\_INTERFACE} config file parameter on each host. 
In addition, if the pool's central manager is on such a machine, users
should set the \Macro{CM\_IP\_ADDR} parameter to the ip address you wish
to use on that machine.
See section~\ref{sec:Multiple-Interfaces} on
page~\pageref{sec:Multiple-Interfaces} for more details.

\item The support for multiple network interfaces introduced bugs in
\Condor{userprio}, \Condor{stats}, CondorPVM, and the \Opt{-pool}
option to many Condor tools.
All of these will be fixed in version 6.1.6.

\item Fixed a bug in the remote system call library that was
preventing certain Fortran operations from working correctly on
Linux.  

\item The Linux binaries for GLIBC we now distribute are compiled on a
Red Hat 5.2 machine.
If you're using this version of Red Hat, you might have better luck
with the dynamically linked version of Condor than previous releases
of Condor.
Sites using other GLIBC Linux distributions should continue to use the
statically linked version of Condor.

\item Fixed a bug in the \Condor{shadow} that could cause it to die
with signal 11 (segmentation violation) under certain rare
circumstances. 

\item Fixed a bug in the \Condor{schedd} that could cause it to die
with signal 11 (segmentation violation) under certain rare
circumstances. 

\item Fixed a bug in the \Condor{negotiator} that could cause it to
die with signal 8 (floating point exception) on Digital Unix
machines. 

\item The following shadow parameters have been added to control
checkpointing: \Macro{COMPRESS\_PERIODIC\_CKPT},
\Macro{COMPRESS\_VACATE\_CKPT}, \Macro{PERIODIC\_MEMORY\_SYNC},
\Macro{SLOW\_CKPT\_SPEED}.  See
section~\ref{sec:Shadow-Config-File-Entries} on
page~\pageref{sec:Shadow-Config-File-Entries} for more details.
In addition, the shadow now honors the \Attr{CkptWanted} flag in a job
classad, and if it is set to ``False'', the job will never
checkpoint.

\item Fixed a bug in the \Condor{startd} that could cause it to
report negative values for the CondorLoadAvg on rare occasions. 

\item Fixed a bug in the \Condor{startd} that could cause it to die
with a fatal exception in situations where the act of getting claimed
by a remote schedd failed for some reason.  
This resulted in the \Condor{startd} exiting on rare occasions with a
message in its log file to the effect of \texttt{ERROR ``Match timed
out but not in matched state''}.

\item Fixed a bug in the \Condor{schedd} that under rare circumstances
could cause a job to be left in the ``Running'' state even after the
\Condor{shadow} for that job had exited.

\item Fixed a bug in the \Condor{schedd} and various tools that
prevented remote read-only access to the job queue from working.
So, for example, \texttt{condor\_q -name foo}, if run on any machine
other than foo, wouldn't display any jobs from foo's queue. 
This fix re-enables the following options to \Condor{q} to work:
\Opt{submitter}, \Opt{name}, \Opt{global}, etc.

\item Changed the \Condor{schedd} so that when starting jobs, it
always sorts on the cluster number, in addition to the date the jobs
were enqueued and the process number within clusters, so that if many
clusters were submitted at the same time, the jobs are started in
order.

\item Fixed a bug in \Condor{compile} that was modifying the
\Env{PATH} environment variable by adding things to the front of it.
This would potentially cause jobs to be compiled and linked with a
different version of a compiler than they thought they were getting.  

\item Minor change in the way the \Condor{startd} handles the
\Dflag{LOAD} and \Dflag{KEYBOARD} debug flags.  
Now, each one, when set, will only display every
\Macro{UPDATE\_INTERVAL}, regardless of the startd state.
If you wish to see the values for keyboard activity or load average
every \Macro{POLLING\_INTERVAL}, you must enable \Dflag{FULLDEBUG}. 

\end{itemize}

%%%%%%%%%%%%%%%%%%%%%%%%%%%%%%%%%%%%%%%%%%%%%%%%%%%%%%%%%%%%%%%%%%%%%%
\subsection*{\label{sec:New-6-1-4}Version 6.1.4}
%%%%%%%%%%%%%%%%%%%%%%%%%%%%%%%%%%%%%%%%%%%%%%%%%%%%%%%%%%%%%%%%%%%%%%

\begin{itemize}

\item Fixed a bug in the socket communication library used by Condor
that was causing daemons and tools to die on some platforms (notably,
Digital Unix) with signal 8, SIGFPE (floating point exception).

\item Fixed a bug in the usage message of many Condor tools that
mentioned a \Opt{-all} option that isn't yet supported. 
This option will be supported in future versions of Condor.

\item Fixed a bug in the filesystem authentication code used to
authenticate operations on the job queue that left empty temporary
files in /tmp.  
These files are now properly removed after they are used.

\item Fixed a minor bug in the totals \Condor{status} displays when
you use the \Opt{ckptsrvr} option.

\item Fixed a minor syntax error in the \Condor{install} script that
would cause warnings.

\item the \File{Condor.pm} Perl module is now included in the
\File{lib} directory of the main release directory.

\end{itemize}

%%%%%%%%%%%%%%%%%%%%%%%%%%%%%%%%%%%%%%%%%%%%%%%%%%%%%%%%%%%%%%%%%%%%%%
\subsection*{\label{sec:New-6-1-3}Version 6.1.3}
%%%%%%%%%%%%%%%%%%%%%%%%%%%%%%%%%%%%%%%%%%%%%%%%%%%%%%%%%%%%%%%%%%%%%%

\Note There are a lot of new, unstable features in 6.1.3.  
PLEASE do not install all of 6.1.3 on a production pool.
Almost all of the bug fixes in 6.1.3 are in the \Condor{startd} or
\Condor{starter}, so, unless you really know what you're doing, we
recommend you just upgrade SMP-Startd contrib module, not the entire
6.1.3 release. 

\begin{itemize}

\item Owners can now specify how the SMP-Startd partitions the system
resources into the different types and numbers of virtual machines,
specifying the number of CPUs, megs of RAM, megs of swap space, etc.,
in each.
Previously, each virtual machine reported to Condor from an SMP
machine always had one CPU, and all shared system resources were
evenly divided among the virtual machines.

\item Fixed a bug in the reporting of virtual memory and disk space on
SMP machines where each virtual machine represented was advertising
the total in the system for itself, instead of its own share.
Now, both the totals, and the virtual machine-specific values are
advertised.  

\item Fixed a bug in the \Condor{starter} when it was trying to
suspend jobs.
While we always killed all of the processes when we were trying to
vacate, if a vanilla job forked, the starter would sometimes not
suspend some of the children processes.
In addition, we could sometimes miss a standard universe job for
suspending as well.
This is all fixed.

\item Fixed a bug in the SMP-Startd's load average computation that
could cause processes spawned by Condor to not be associated w/ the
Condor load average.
This would cause the startd to over-estimate the owner's load average,
and under-estimate the Condor load, which would cause a cycle of
suspending and resuming a Condor job, instead of just letting it run.

\item Fixed a bug in the SMP-Startd's load average computation that
could cause certain rare exceptions to be treated as fatal, when in
fact, the Startd could recover from them.

\item Fixed a bug in the computation of the total physical memory on
some platforms that was resulting in an overflow on machines with
lots of ram (over 1 gigabyte).

\item Fixed some bugs that could cause \Condor{starter} processes to
be left as zombies underneath the \Condor{startd} under very rare
conditions.  

\item For sites using AFS, if there are problems in the
\Condor{startd} computing the AFS cell of the machine it's running on,
the startd will exit with an error message at start-up time.

\item Fixed a minor bug in \Condor{install} that would lead to a
syntax error in your config file given a certain set of installation
options.  

\item Added the \Opt{-maxjobs} option to the \Condor{submit\_dag}
script that can be used to specify the maximum number of jobs Condor
will run from a DAG at any given time.
Also, \Condor{submit\_dag} automatically creates a ``rescue DAG''.
See section~\ref{sec:DAGMan} on page~\pageref{sec:DAGMan} for details
on DAGMan.

\item Fixed bug in ClassAd printing when you tried to display an
integer or float attribute that didn't exist in the given ClassAd. 
This could show up in \Condor{status}, \Condor{q}, \Condor{history},
etc. 

\item Various commands sent to the Condor daemons now have separate
debug levels associated with them.
For example, commands such as ``keep-alives'', and the command sent
from the \Condor{kbdd} to the \Condor{startd} are only seen in the
various log files if \Dflag{FULLDEBUG} is turned on, instead of
\Dflag{COMMAND}, which the default and now enabled for all daemons on
all platforms by default.
Administrators retaining their old configuration when upgrading to
this version are encouraged to enable \Dflag{COMMAND} in the
\Macro{SCHEDD\_DEBUG} setting.  
In addition, for IRIX and Digital Unix machines, it should be enabled
in the \Macro{STARTD\_DEBUG} setting as well.
See section~\ref{sec:Daemon-Logging-Config-File-Entries} on
page~\pageref{sec:Daemon-Logging-Config-File-Entries} for details on
debug levels in Condor.

\item New debug levels added to Condor: 
\begin{itemize}
\item \Dflag{NETWORK}, used by various daemons in Condor to report
various network statistics about the Condor daemons. 
\item \Dflag{PROCFAMILY}, used to report information about various
families of processes that are monitored by Condor.
For example, this is used in the \Condor{startd} when monitoring the
family of processes spawned by a given user job for the purposes of
computing the Condor load average.
\item \Dflag{KEYBOARD}, used by the \Condor{startd} to print out
statistics about remote tty and console idle times in the
\Condor{startd}.
This information used to be logged at \Dflag{FULLDEBUG}, along with
everything else, so now, you can see just the idle times, and/or have
the information stored to a separate file.
\end{itemize}

\item Added a \Opt{-run} option to \Condor{q}, which displays
information for running jobs, including the remote host where each job
is running.

\item Macros can now be incrementally defined.  See
section~\ref{sec:Config-File-Macros} on
page~\pageref{sec:Config-File-Macros} for more details.

\item \Condor{config\_val} can now be used to set configuration
variables.  See the man page on page~\pageref{man-condor-config-val}
for more details.

\item The job log file now contains a record of network activity.  The
evict, terminate, and shadow exception events indicate the number of
bytes sent and received by the job for the specific run.  
The terminate event additionally indicates totals for the life of the
job.

\item \Macro{STARTER\_CHOOSES\_CKPT\_SERVER} now defaults to true.
See section~\ref{param:StarterChoosesCkptServer} on
page~\pageref{param:StarterChoosesCkptServer} for more details.

\item The infrastructure for authentication within Condor has been
overhauled, allowing for much greater flexibility in supporting new
forms of authentication in the future.
This means that the 6.1.3 schedd and queue management tools (like
\Condor{q}, \Condor{submit}, \Condor{rm} and so on) are incompatible
with previous versions of Condor.

\item Many of the Condor administration tools have been improved to
allow you to specify the ``subsystem'' you want them to effect.  
For example, you can now use ``\condor{reconfig} -startd'' to just
have the startd reconfigure itself.
Similarly, \condor{off}, \condor{on} and \condor{restart} can now all 
work on a single daemon, instead of machine-wide.
See the man pages (section~\ref{sec:command-reference} on
page~\pageref{sec:command-reference}) or run any command with \Opt{-help}
for details. 
\Note The usage message in 6.1.3 incorrectly reports \Opt{-all} as a
valid option.

\item Fixed a bug in the Condor tools that could cause a segmentation
violation in certain rare error conditions.

\end{itemize}

%%%%%%%%%%%%%%%%%%%%%%%%%%%%%%%%%%%%%%%%%%%%%%%%%%%%%%%%%%%%%%%%%%%%%%
\subsection*{\label{sec:New-6-1-2}Version 6.1.2}
%%%%%%%%%%%%%%%%%%%%%%%%%%%%%%%%%%%%%%%%%%%%%%%%%%%%%%%%%%%%%%%%%%%%%%

\begin{itemize}

\item Fixed some bugs in the \Condor{install} script.
Also, enhanced \Condor{install} to customize the path to perl in
various perl scripts used by Condor.

\item Fixed a problem with our build environment that left some files
out of the \File{release.tar} files in the binary releases on some
platforms. 

\item \Condor{dagman}, ``DAGMan'' (see section~\ref{sec:DAGMan} on 
page~\pageref{sec:DAGMan} for details) is now included in the
development release by default.

\item Fixed a bug in the computation of the total physical memory in
HPUX machines that was resulting in an overflow on machines with
lots of ram (over 1 gigabyte).
Also, if you define ``MEMORY'' in your config file, that value will
override whatever value Condor computes for your machine.

\item Fixed a bug in \Condor{starter.pvm}, the PVM version of the
Condor starter (available as an optional ``Contrib module''), when you
disabled \Macro{STARTER\_LOCAL\_LOGGING}.
Now, having this set to ``False'' will properly place debug messages
from \Condor{starter.pvm} into the \File{ShadowLog} file of the
machine that submitted the job (as opposed to the \File{StarterLog}
file on the machine executing the job).  

\end{itemize}


%%%%%%%%%%%%%%%%%%%%%%%%%%%%%%%%%%%%%%%%%%%%%%%%%%%%%%%%%%%%%%%%%%%%%%
\subsection*{\label{sec:New-6-1-1}Version 6.1.1}
%%%%%%%%%%%%%%%%%%%%%%%%%%%%%%%%%%%%%%%%%%%%%%%%%%%%%%%%%%%%%%%%%%%%%%

\begin{itemize}

\item Fixed a bug in the \Condor{startd} where we compute the load
average caused by Condor that was causing us to get the wrong values.
This could cause a cycle of continuous job suspends and job resumes.

\item Beginning with this version, any jobs linked with the Condor
checkpoint libraries will use the zlib compression code (used by gzip
and others) to compress periodic checkpoints before they are written
to the network.  
These compressed checkpoints are uncompressed at startup time.  
This saves network bandwidth, disk space, as well as time (if the
network is the bottleneck to checkpointing, which it usually is). 
In future versions of Condor, all checkpoints will probably be
compressed, but at this time, it is only used for periodic
checkpoints.  
Note, you have to relink your jobs with the \Condor{compile} command
to have this feature enabled.
Old jobs (not relinked) will continue to run just fine, they just
won't be compressed.

\item \Condor{status} now has better support for displaying checkpoint
server ClassAds. 

\item More contrib modules from the development series are now
available, such as the checkpoint server, PVM support, and the
CondorView server.  

\item Fixed some minor bugs in the UserLog code that were causing
problems for DAGMan in exceptional error cases.

\item Fixed an obscure bug in the logging code when \Dflag{PRIV} was
enabled that could result in incorrect file permissions on log files. 

\end{itemize}

%%%%%%%%%%%%%%%%%%%%%%%%%%%%%%%%%%%%%%%%%%%%%%%%%%%%%%%%%%%%%%%%%%%%%%
\subsection*{\label{sec:New-6-1-0}Version 6.1.0}
%%%%%%%%%%%%%%%%%%%%%%%%%%%%%%%%%%%%%%%%%%%%%%%%%%%%%%%%%%%%%%%%%%%%%%

\begin{itemize}

\item Support has been added to the \condor{startd} to run multiple
jobs on SMP machines.
See section~\ref{sec:Configuring-SMP} on
page~\pageref{sec:Configuring-SMP} for details about setting up and
configuring SMP support.

\item The expressions that control the \condor{startd} policy for
vacating, jobs has been simplified.
See section~\ref{sec:Configuring-Policy} on
page~\pageref{sec:Configuring-Policy} for complete details on the new
policy expressions, and section~\ref{sec:V60-Policy-diffs} on
page~\pageref{sec:V60-Policy-diffs} for an explanation of what's
different from the version 6.0 expressions.

\item We now perform better tracking of processes spawned by Condor.
If children die and are inherited by init, we still know they belong
to Condor.
This allows us to better ensure we don't leave processes lying around
when we need to get off a machine, and enables us to have a much more
accurate computation of the load average generated by Condor (the
\Attr{CondorLoadAvg} as reported by the \Condor{startd}). 

\item The \condor{collector} now can store historical information
about your pool state.
This information can be queried with the \Condor{stats} program (see
the man page on page~\pageref{man-condor-stats}), which is used by the
\Condor{view} Java GUI, which is available as a separate contrib
module.

\item Condor jobs can now be put in a ``hold'' state with the
\Condor{hold} command.
Such jobs remain in the job queue (and can be viewed with \Condor{q}),
but there will not be any negotiation to find machines for them.
If a job is having a temporary problem (like the permissions are 
wrong on files it needs to access), the job can be put on hold until
the problem can be solved.
Jobs put on hold can be released with the \Condor{release} command.

\item \condor{userprio} now has the notion of \Term{user factors} as a
way to create different groups of users in different priority levels.
See section~\ref{sec:UserPrio} on page~\pageref{sec:UserPrio} for
details.
This includes the ability to specify a local priority domain, and all
users from other domains get a much worse priority.

\item Usage statistics by user is now available from
\condor{userprio}.
See the man page on page~\pageref{man-condor-userprio} for details. 

\item The \condor{schedd} has been enhanced to enable ``flocking'',
where it seeks matches with machines in multiple pools if its requests
cannot be serviced in the local pool.
See section~\ref{sec:Flocking} on page~\pageref{sec:Flocking} for more
details.

\item The \condor{schedd} has been enhanced to enable \condor{q} and
other interactive tools better response time.

\item The \condor{schedd} has also been enhanced to allow it to check
the permissions of the files you specify for input, output, error and
so on.  
If the schedd doesn't have the required access rights to the files,
the jobs will not be submitted, and \Condor{submit} will print an
error message.

\item When you perform a \Condor{rm} command, and the job you removed
was using a ``user log'', the remove event is now recorded into the
log. 

\item Two new attributes have been added to the job classad when it 
begins executing: \Attr{RemoteHost} and \Attr{LastRemoteHost}.
These attributes list the IP address and port of the startd that is
either currently running the job, or the last startd to run the job
(if it's run on more than one machine). 
This information helps users track their job's execution more closely,
and allows administrators to troubleshoot problems more effectively. 

\item The performance of checkpointing was increased by using larger
buffers for the network I/O used to get the checkpoint file on and off
the remote executing host (this helps for all pools, with or without
checkpoint servers). 

\end{itemize}


%%%%%%%%%%%%%%%%%%%%%%%%%%%%%%%%%%%%%%%%%%%%%%%%%%%%%%%%%%%%%%%%%%%%%%
\section{\label{sec:History-6-0}Stable Release Series 6.0}
%%%%%%%%%%%%%%%%%%%%%%%%%%%%%%%%%%%%%%%%%%%%%%%%%%%%%%%%%%%%%%%%%%%%%%

6.0 is the first version of Condor with \Term{ClassAds}.
It contains many other fundamental enhancements over version 5.
It is also the first official stable release series, with a
development series (6.1) simultaneously available.


%%%%%%%%%%%%%%%%%%%%%%%%%%%%%%%%%%%%%%%%%%%%%%%%%%%%%%%%%%%%%%%%%%%%%%
\subsection*{\label{sec:New-6-0-3}Version 6.0.3}
%%%%%%%%%%%%%%%%%%%%%%%%%%%%%%%%%%%%%%%%%%%%%%%%%%%%%%%%%%%%%%%%%%%%%%

\begin{itemize}

\item Fixed a bug that was causing the hostname of the submit machine
that claimed a given execute machine to be incorrectly reported by the
\Condor{startd} at sites using NIS.

\item Fixed a bug in the \Condor{startd}'s benchmarking code that
could cause a floating point exception (SIGFPE, signal 8) on very,
very fast machines, such as newer Alphas.

\item Fixed an obscure bug in \Condor{submit} that could happen when
you set a requirements expression that references the ``Memory''
attribute.
The bug only showed up with certain formations of the requirement
expression.

\end{itemize}


%%%%%%%%%%%%%%%%%%%%%%%%%%%%%%%%%%%%%%%%%%%%%%%%%%%%%%%%%%%%%%%%%%%%%%
\subsection*{\label{sec:New-6-0-2}Version 6.0.2}
%%%%%%%%%%%%%%%%%%%%%%%%%%%%%%%%%%%%%%%%%%%%%%%%%%%%%%%%%%%%%%%%%%%%%%

\begin{itemize}

\item Fixed a bug in the \Syscall{fcntl} call for Solaris 2.6 that was
causing problems with file I/O inside Fortran jobs.

\item Fixed a bug in the way the \Macro{DEFAULT\_DOMAIN\_NAME}
parameter was handled so that this feature now works properly.  

\item Fixed a bug in how the \Macro{SOFT\_UID\_DOMAIN} config file
parameter was used in the \Condor{starter}.
This feature is also documented in the manual now (see
section~\ref{param:SoftUidDomain} on
page~\pageref{param:SoftUidDomain}).

\item You can now set the RunBenchmarks expression to ``False'' and
the \Condor{startd} will never run benchmarks, not even at startup
time. 

\item Fixed a bug in \Syscall{getwd} and \Syscall{getcwd} for sites
that use the NFS automounter.
his bug was only present if user programs tried to call
\Syscall{chdir} themselves.
Now, this is supported. 

\item Fixed a bug in the way we were computing the available virtual
memory on HPUX 10.20 machines.

\item Fixed a bug in \Condor{q} -analyze so it will correctly identify
more situations where a job won't run.

\item Fixed a bug in \Condor{status} -format so that if the requested 
attribute isn't available for a given machine, the format string
(including spaces, tabs, newlines, etc) is still printed, just the
value for the requested attribute will be an empty string. 

\item Fixed a bug in the \Condor{schedd} that was causing
\Condor{history} to not print out the first ClassAd attribute of all
jobs that have completed

\item Fixed a bug in \Condor{q} that would cause a segmentation fault
if the argument list was too long.

\end{itemize}

%%%%%%%%%%%%%%%%%%%%%%%%%%%%%%%%%%%%%%%%%%%%%%%%%%%%%%%%%%%%%%%%%%%%%%
\subsection*{\label{sec:New-6-0-1}Version 6.0.1}
%%%%%%%%%%%%%%%%%%%%%%%%%%%%%%%%%%%%%%%%%%%%%%%%%%%%%%%%%%%%%%%%%%%%%%

\begin{itemize}

\item Fixed bugs in the \Syscall{getuid}), \Syscall{getgid},
\Syscall{geteuid}, and \Syscall{getegid} system calls. 

\item Multiple config files are now supported as a list specified via
the \Macro{LOCAL\_CONFIG\_FILE} variable. 

\item \Macro{ARCH} and \Macro{OPSYS} are now automatically determined
on all machines (including HPUX 10 and Solaris). 

\item Machines running IRIX now correctly suspend vanilla jobs.

\item \Condor{submit} doesn't allow root to submit jobs.

\item The \Condor{startd} now notices if you have changed
\Macro{COLLECTOR\_HOST} on reconfig.

\item Physical memory is now correctly reported on Digital Unix when
daemons are not running as root. 

\item New \MacroU{SUBSYSTEM} macro in configuration files that changes
based on which daemon is reading the file (i.e. STARTD, SCHEDD, etc.) 
See section~\ref{sec:Condor-Subsystem-Names}, ``Condor Subsystem
Names'' on page~\pageref{sec:Condor-Subsystem-Names} for a complete
list of the subsystem names used in Condor.

\item Port to HP-UX 10.20.  

\item \Syscall{getrusage} is now a supported system call.  
This system call will allow you to get resource usage about the entire
history of your condor job.

\item Condor is now fully supported on Solaris 2.6 machines (both
Sparc and Intel). 

\item Condor now works on Linux machines with the GNU C library.  
This includes machines running Red Hat 5.x and Debian 2.0. 
In addition, there seems to be a bug in Red Hat that was causing the
output from \Condor{config\_val} to not appear when used in scripts
(like \Condor{compile}).
We put in explicit calls to flush the I/O buffers before
\Condor{config\_val} exits, which seems to solve the problem.

\item Hooks have been added to the checkpointing library to help
support the checkpointing of PVM jobs.

\item Condor jobs can now send signals to themselves when running in
the standard universe.
You do this just as you normally would:
\begin{verbatim}
        kill( getpid(), signal_number )
\end{verbatim}
Trying to send a signal to any other process will result in
\Syscall{kill} returning -1.

\item Support for NIS has been improved on Digital Unix and IRIX.

\item Fixed a bug that would cause the negotiator on IRIX machines to
never match jobs with available machines.  

\end{itemize}

%%%%%%%%%%%%%%%%%%%%%%%%%%%%%%%%%%%%%%%%%%%%%%%%%%%%%%%%%%%%%%%%%%%%%%
\subsection*{\label{sec:New-6-0-pl4}Version 6.0 pl4}
%%%%%%%%%%%%%%%%%%%%%%%%%%%%%%%%%%%%%%%%%%%%%%%%%%%%%%%%%%%%%%%%%%%%%%

\Note Back in the bad old days, we used this evil ``patch level''
version number scheme, with versions like ``6.0pl4''.
This has all gone away in the current versions of Condor. 

\begin{itemize}

\item Fixed a bug that could cause a segmentation violation in the 
\Condor{schedd} under rare conditions when a \Condor{shadow} exited.

\item Fixed a bug that was preventing any core files that user jobs
submitted to Condor might create from being transferred back to the
submit machine for inspection by the user who submitted them.

\item Fixed a bug that would cause some Condor daemons to go into an
infinite loop if the "ps" command output duplicate entries.
This only happens on certain platforms, and even then, only under rare
conditions.
However, the bug has been fixed and Condor now handles this case
properly.

\item Fixed a bug in the \Condor{shadow} that would cause a
segmentation violation if there was a problem writing to the user log
file specified by "log = filename" in the submit file used with
\Condor{submit}.

\item Added new command line arguments for the Condor daemons to support
saving the PID (process id) of the given daemon to a file, sending a
signal to the PID specified in a given file, and overriding what
directory is used for logging for a given daemon.
These are primarily for use with the \Condor{kbdd} when it needs to be
started by XDM for the user logged onto the console, instead of
running as root.
See section~\ref{sec:kbdd} on ``Installing the \Condor{kbdd}'' on
page~\pageref{sec:kbdd} for details.

\item Added support for the \Macro{CREATE\_CORE\_FILES} config file
parameter.  
If this setting is defined, Condor will override whatever limits you
have set and in the case of a fatal error, will either create core
files or not depending on the value you specify ("true" or "false").

\item Most Condor tools (\Condor{on}, \Condor{off},
\Condor{master\_off}, \Condor{restart}, \Condor{vacate},
\Condor{checkpoint}, \Condor{reconfig}, \Condor{reconfig\_schedd},
\Condor{reschedule}) can now take the IP address and port you want to
send the command to directly on the command line, instead of only
accepting hostnames. 
This IP/port must be passed in a special format used in Condor (which
you will see in the daemon's log files, etc).
It is of the form: \Sinful{ip.address:port}.  
For example: \Sinful{123.456.789.123:4567}.

\end{itemize}

%%%%%%%%%%%%%%%%%%%%%%%%%%%%%%%%%%%%%%%%%%%%%%%%%%%%%%%%%%%%%%%%%%%%%%
\subsection*{\label{sec:New-6-0-pl3}Version 6.0 pl3}
%%%%%%%%%%%%%%%%%%%%%%%%%%%%%%%%%%%%%%%%%%%%%%%%%%%%%%%%%%%%%%%%%%%%%%

\begin{itemize}

\item Fixed a bug that would cause a segmentation violation if a
machine was not configured with a full hostname as either the official
hostname or as any of the hostname aliases.

\item If your host information does not include a fully qualified
hostname anywhere, you can specify a domain in the
\Macro{DEFAULT\_DOMAIN\_NAME} parameter in your global config file
which will be appended to your hostname whenever Condor needs to use a
fully qualified name.

\item All Condor daemons and most tools now support a "-version"
option that displays the version information and exits.

\item The \Condor{install} script now prompts for a short description
of your pool, which it stores in your central manager's local config
file as \Macro{COLLECTOR\_NAME}.
This description is used to display the name of your pool when sending
information to the Condor developers.

\item When the \Condor{shadow} process starts up, if it is configured
to use a checkpoint server and it cannot connect to the server, the
shadow will check the \Macro{MAX\_DISCARDED\_RUN\_TIME} parameter.  
If the job in question has accumulated more CPU minutes than this
parameter, the \Condor{shadow} will keep trying to connect to the
checkpoint server until it is successful.
Otherwise, the \Condor{shadow} will just start the job over from
scratch immediately.

\item If Condor is configured to use a checkpoint server, it will only
use the checkpoint server.
Previously, if there was a problem connecting to the checkpoint
server, Condor would fall back to using the submit machine to store
checkpoints.
However, this caused problems with local disks filling up on machines
without much disk space.

\item Fixed a rare race condition that could cause a segmentation
violation if a Condor daemon or tool opened a socket to a daemon and
then closed it right away.

\item All TCP sockets in Condor now have the "keep alive" socket option
enabled.
This allows Condor daemons to notice if their peer goes away in a hard
crash.

\item Fixed a bug that could cause the \Condor{schedd} to kill jobs
without a checkpoint during its graceful shutdown method under certain
conditions.

\item The \Condor{schedd} now supports the
\Macro{MAX\_SHADOW\_EXCEPTIONS} parameter.
If the \Condor{shadow} processes for a given match die due to a fatal
error (an exception) more than this number of times, the
\Condor{schedd} will now relinquish that match and stop trying to
spawn \Condor{shadow} processes for it.

\item The "-master" option to \Condor{status} now displays the \Attr{Name}
attribute of all \Condor{master} daemons in your pool, as opposed
to the \Attr{Machine} attribute.
This helps for pools that have submit-only machines joining them, for
example.

\end{itemize}

%%%%%%%%%%%%%%%%%%%%%%%%%%%%%%%%%%%%%%%%%%%%%%%%%%%%%%%%%%%%%%%%%%%%%%
\subsection*{\label{sec:New-6-0-pl2}Version 6.0 pl2}
%%%%%%%%%%%%%%%%%%%%%%%%%%%%%%%%%%%%%%%%%%%%%%%%%%%%%%%%%%%%%%%%%%%%%%

\begin{itemize}

\item In patch level 1, code was added to more accurately find the
full hostname of the local machine.
Part of this code relied on the resolver, which on many platforms is a
dynamic library.
On Solaris, this library has needed many security patches and the
installation of Solaris on our development machines produced binaries
that are incompatible with sites that haven't applied all the security
patches.
So, the code in Condor that relies on this library was simply removed
for Solaris.

\item Version information is now built into Condor.
You can see the \Attr{CondorVersion} attribute in every daemon's
ClassAd. 
You can also run the UNIX command "ident" on any Condor binary to see
the version. 

\item Fixed a bug in the "remote submit" mode of \Condor{submit}.
The remote submit wasn't connecting to the specified schedd, but was
instead trying to connect to the local schedd.

\item Fixed a bug in the \Condor{schedd} that could cause it to exit
with an error due to its log file being locked improperly under
certain rare circumstances.

\end{itemize}

%%%%%%%%%%%%%%%%%%%%%%%%%%%%%%%%%%%%%%%%%%%%%%%%%%%%%%%%%%%%%%%%%%%%%%
\subsection*{\label{sec:New-6-0-pl1}Version 6.0 pl1}
%%%%%%%%%%%%%%%%%%%%%%%%%%%%%%%%%%%%%%%%%%%%%%%%%%%%%%%%%%%%%%%%%%%%%%

\begin{itemize}

\item \Condor{kbdd} bug patched: On Silicon Graphics and DEC Alpha
ports, if your X11 server is using Xauthority user authentication, and
the \Condor{kbdd} was unable to read the user's \File{.Xauthority}
file for some reason, the \Condor{kbdd} would fall into an infinite 
loop.

\item When using a Condor Checkpoint Server, the protocol between the
Checkpoint Server and the \Condor{schedd} has been made more robust
for a faulty network connection. Specifically, this improves
reliability when submitting jobs across the Internet and using a
remote Checkpoint Server.

\item Fixed a bug concerning \Macro{MAX\_JOBS\_RUNNING}: The parameter
\MacroNI{MAX\_JOBS\_RUNNING} in the config file controls the maximum
number of simultaneous \Condor{shadow} processes allowed on your
submission machine.
The bug was the number of shadow processes could, under certain
conditions, exceed the number specified by
\MacroNI{MAX\_JOBS\_RUNNING}. 

\item Added new parameter \Macro{JOB\_RENICE\_INCREMENT} that can be
specified in the config file.
This parameter specifies the UNIX nice level that the \Condor{starter}
will start the user job.
It works just like the \Cmd{renice}{1} command in UNIX. 
Can be any integer between 1 and 19; a value of 19 is the lowest
possible priority.

\item Improved response time for \Condor{userprio}.

\item Fixed a bug that caused periodic checkpoints to happen more
often than specified.

\item Fixed some bugs in the installation procedure for certain
environments that weren't handled properly, and made the documentation
for the installation procedure more clear.

\item Fixed a bug on IRIX that could allow vanilla jobs to be started
as root under certain conditions.
This was caused by the non-standard uid that user "nobody" has on
IRIX.
Thanks to Chris Lindsey at NCSA for help discovering this bug.

\item On machines where the \File{/etc/hosts} file is misconfigured to
list just the hostname first, then the full hostname as an alias,
Condor now correctly finds the full hostname anyway.

\item The local config file and local root config file are now only
found by the files listed in the \Macro{LOCAL\_CONFIG\_FILE} and
\Macro{LOCAL\_ROOT\_CONFIG\_FILE} parameters in the global config
files.
Previously, \File{/etc/condor} and user condor's home directory
(\~condor) were searched as well.
This could cause problems with submit-only installations of Condor at
a site that already had Condor installed.

\end{itemize}

%%%%%%%%%%%%%%%%%%%%%%%%%%%%%%%%%%%%%%%%%%%%%%%%%%%%%%%%%%%%%%%%%%%%%%
\subsection*{\label{sec:New-6-0-pl0}Version 6.0 pl0}
%%%%%%%%%%%%%%%%%%%%%%%%%%%%%%%%%%%%%%%%%%%%%%%%%%%%%%%%%%%%%%%%%%%%%%

\begin{itemize}

\item Initial Version 6.0 release.

\end{itemize}

