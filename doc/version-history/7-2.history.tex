%%%      PLEASE RUN A SPELL CHECKER BEFORE COMMITTING YOUR CHANGES!
%%%      PLEASE RUN A SPELL CHECKER BEFORE COMMITTING YOUR CHANGES!
%%%      PLEASE RUN A SPELL CHECKER BEFORE COMMITTING YOUR CHANGES!
%%%      PLEASE RUN A SPELL CHECKER BEFORE COMMITTING YOUR CHANGES!
%%%      PLEASE RUN A SPELL CHECKER BEFORE COMMITTING YOUR CHANGES!

%%%%%%%%%%%%%%%%%%%%%%%%%%%%%%%%%%%%%%%%%%%%%%%%%%%%%%%%%%%%%%%%%%%%%%
\section{\label{sec:History-7-2}Stable Release Series 7.2}
%%%%%%%%%%%%%%%%%%%%%%%%%%%%%%%%%%%%%%%%%%%%%%%%%%%%%%%%%%%%%%%%%%%%%%

This is a stable release series of Condor.
As usual, only bug fixes (and potentially, ports to new platforms)
will be provided in future 7.2.x releases.
New features will be added in the 7.3.x development series.

The details of each version are described below.


%%%%%%%%%%%%%%%%%%%%%%%%%%%%%%%%%%%%%%%%%%%%%%%%%%%%%%%%%%%%%%%%%%%%%%
\subsection*{\label{sec:New-7-2-1}Version 7.2.1}
%%%%%%%%%%%%%%%%%%%%%%%%%%%%%%%%%%%%%%%%%%%%%%%%%%%%%%%%%%%%%%%%%%%%%%

\noindent Release Notes:

\begin{itemize}

\item None.

\end{itemize}


\noindent New Features:

\begin{itemize}

\item None.

\end{itemize}

\noindent Configuration Variable Additions and Changes:

\begin{itemize}

\item The configuration variable \Macro{JAVA\_MAXHEAP\_ARGUMENT} 
  now defaults to the value \Opt{-Xmx1024m}.
  The installation process of Condor resets this value to \Expr{UNDEFINED} 
  in the local configuration file, if the detected JVM is not from 
  Sun Microsystems.

\item A new feature has been added to the \Condor{master} that makes
  it possible to append to the \MacroNI{DC\_DAEMON\_LIST} 
  configuration variable, instead of overwriting it. 
  To take advantage of this, place the plus character  ('\verb@+@') as
  the first character in the \MacroNI{DC\_DAEMON\_LIST} definition.  
  For example:
\begin{verbatim}
  DAEMON_LIST     = NEW_DAEMON
  DC_DAEMON_LIST  = +NEW_DAEMON
\end{verbatim}

% PR 959
\item The new configuration variable \Macro{DAGMAN\_COPY\_TO\_SPOOL}
  controls whether the \Condor{dagman} binary gets copied to 
  the spool directory when a DAG is submitted.
  See \ref{param:DAGManCopyToSpool} for details.

\end{itemize}

\noindent Bugs Fixed:

\begin{itemize}

\item Fixed a bug that could cause the \Condor{schedd} to never evaluate
periodic expressions.

\item Fixed a bug in \Condor{chirp} which was causing it to crash 
on invocation.

\item Fixed a Windows platform bug in the \Prog{condor\_mail.exe} program,
which was causing it to become unresponsive when run.
The application also increased its memory consumption.

\item Fixed a bug on Unix platforms where \Condor{configure} would provide
incorrect defaults for the \MacroNI{JAVA\_MAXHEAP\_ARGUMENT} attribute in
the installed configuration files. The new current default for Sun Java
JVMs is \Opt{-Xmx1024m}.

\item Fixed a bug on Unix platforms where \Condor{configure} would imply
  that using the Unix user \Login{root} or UID 0 for the \Opt{--owner} 
  option is a good thing--it is not,
and then complain that it could not find user \Login{root} in the password
file in this circumstance.

\item Fixed a bug on Unix platforms where \Condor{configure} would emit errors
about not being able to execute \Prog{ldd} when installing Condor on the
Mac OS X 10.5 platform.  \Condor{configure} now correctly detects shared
library requirements when installing the Condor binaries on the Mac OS
X 10.5 platform.

\item Fixed a bug where execute-side daemons started before the
\Condor{credd} would fail to match with Windows jobs with
\SubmitCmd{run\_as\_owner} set.
This condition persisted until the execute-side daemons were
either restarted or reconfigured.

\item Fixed a problem affecting JobRouter and Condor-C.  When jobs are
spooling input files, they enter a temporary hold state, which could
trigger actions by a naive periodic remove or release expression.
Periodic expressions are no longer evaluated when in this temporary
hold state, which has the hold reason \"Spooling input data files\".

\item The example init script condor.boot.generic erroneously claimed
that the \Condor{master} would begin sending SIGKILL to child processes
after 20 seconds if SIGQUIT (``fast shutdown'') failed.  The \Condor{master}
will actually wait \MacroNI{SHUTDOWN\_FAST\_TIMEOUT} seconds, a value that
currently defaults to 300 seconds.

\end{itemize}

\noindent Known Bugs:

\begin{itemize}

\item None.

\end{itemize}

\noindent Additions and Changes to the Manual:

\begin{itemize}

\item None.

\end{itemize}



%%%%%%%%%%%%%%%%%%%%%%%%%%%%%%%%%%%%%%%%%%%%%%%%%%%%%%%%%%%%%%%%%%%%%%
\subsection*{\label{sec:New-7-2-0}Version 7.2.0}
%%%%%%%%%%%%%%%%%%%%%%%%%%%%%%%%%%%%%%%%%%%%%%%%%%%%%%%%%%%%%%%%%%%%%%

\noindent Release Notes:

\begin{itemize}

\item A bug in some older Xen kernels can result in Condor errors
due to a broken assumption in the \Condor{procd} daemon.
See the FAQ entry at section~ \ref{sec:xen-jiffies-bug} for details.

\item A problem has been discovered when using snapshot disks with 
\SubmitCmd{vm} universe VMware jobs,
if the path that the \Condor{vm-gahp} uses to refer to the
virtual machine's VMX file contains a symbolic link.
See the FAQ entry at section~ \ref{sec:vmware-symlink-bug} for details.

\item The name of the Amazon EC2 GAHP binary has changed from
\Prog{amazon-gahp} to \Prog{amazon\_gahp}. This makes it consistent
with the naming of other Condor binaries.

\end{itemize}


\noindent New Features:

\begin{itemize}

\item The default \SubmitCmd{universe} for jobs is now 
\SubmitCmd{vanilla}, instead of \SubmitCmd{standard}.
The default can be changed using the configuration variable
\Macro{DEFAULT\_UNIVERSE}.

\item VMware \SubmitCmd{vm} universe jobs now have any BIOS settings saved in
an \File{nvram} file in the \SubmitCmd{vmware\_dir} given in the
job's submit file transferred to the execute machine, so that they
apply to the job's execution.

\item Daemons that become unresponsive are now killed using the
SIGABRT signal, which causes a core file to be dropped.
Setting the configuration variable \Macro{NOT\_RESPONDING\_WANT\_CORE}
to \Expr{False} will revert to the previous behavior that used
the SIGKILL signal.

\item \Condor{job\_router} and \Condor{q} \Opt{-better-analyze} now
support more classad functions than they previously did.  They now
support all classad functions except for those with names beginning
with \Code{stringList}.

\end{itemize}

\noindent Configuration Variable Additions and Changes:

\begin{itemize}

\item The HAD configuration variable \MacroNI{NEGOTIATOR\_STATE\_FILE}
has changed its name to \MacroNI{STATE\_FILE}.

\end{itemize}

\noindent Bugs Fixed:

\begin{itemize}

\item \Security A flaw was found and fixed that could allow an unauthenticated
user to cause Condor daemons to shut down,
and could allow running jobs to be removed from the queue.

% PR 952
\item Fixed a bug that caused \Condor{dagman} to stay in the Condor queue,
if \Condor{dagman} was accidentally submitted with an empty DAG input file.

% PR 959
\item \Condor{submit\_dag} now generates a \File{.condor.sub} file with
the submit description file command \SubmitCmd{copy\_to\_spool}
set to \Expr{True}, to ease version upgrades while
large DAGs are running.

\item Fixed a problem in the \Condor{startd} when using
\MacroNI{STARTD\_SLOT\_EXPRS} for attributes that are sometimes
present and sometimes absent from the machine ClassAd.  This is most
typical of attributes that enter the machine ClassAd from the job, via
\MacroNI{STARTD\_JOB\_EXPRS}.  When the attribute went away from slot X
(for example, because the job on slot X finished), the corresponding
\MacroNI{SlotX\_<AttributeName>} attribute was not reliably removed from
all of the other slots.

\item Removed some redundant information from the \Condor{startd} 
advertisements to the \Condor{collector}, 
from within the private ClassAd that is not user-visible.
This fix reduces UDP traffic and memory usage generated by
the \Condor{startd} by about 20\Percent\
in the \Condor{collector} and \Condor{negotiator} daemons.

\item Fixed the \Condor{master} daemon to correctly preserve all command-line
arguments when restarting itself.  In some cases, not preserving \Code{argv[0]}
confused external utilities that monitor the \Condor{master} process by looking
at the output of \Prog{ps} or similar programs.  Also, not preserving
\Opt{-pid} and \Opt{-runfor} could cause unexpected behavior.

\item Fixed a bug that exhibited itself when
the configuration variable \MacroNI{NEGOTIATOR\_CONSIDER\_PREEMPTION}
was set to \Expr{False}, in which jobs
would not be matched to slots in the backfill state.  Corrected, slots doing
backfill are included in the matchmaking process.

\item The \Condor{job\_router} did not work while managing jobs from
multiple users when read access to the \Condor{schedd} required
authentication.  The \Condor{job\_router} was also not able to use
authentication methods other than FS.  Now it can use any
authentication method, as long as the resulting identity is listed in
the configuration variable
\MacroNI{QUEUE\_SUPER\_USERS} or the \Condor{job\_router} and
\Condor{schedd} are running as a Personal Condor in non-root mode.

% Commented out by Karen, as it provides no relevant information
% in the given form.
% \item Fixed a number of memory leaks.

\item Fixed a bug in the \Condor{schedd} daemon that could cause it to write
  an incorrect Unique ID to the event log's header.

\item Fixed a bug in the user log reader API that could cause it to
  incorrectly return a ULOG\_NO\_EVENT in rare cases.

\item Fixed a bug in the user log reader API that could cause it to
  crash if the application attempted to re-initialize the ReadUserLog
  object.  The code now detects this condition, and returns an error
  when the application attempts to re-initialization an already
  initialized ReadUserLog object.

\item Fixed a bug that limited the size of \File{stdin}, \File{stdout},
and \File{stderr} files in the vanilla universe to 2GBytes.

\item Fixed a bug that could cause the \Condor{starter} to EXCEPT upon 
completion or eviction of a \SubmitCmd{vm} universe job.
The error message that appeared in the \File{StarterLog} file was
\begin{verbatim}
  Write_Pipe: invalid pipe end
\end{verbatim}

\item When a held job is removed, the values of the attributes
\Attr{HoldReason}, \Attr{HoldReasonCode} and \Attr{HoldReasonSubCode}
are moved to \Attr{LastHoldReason}, \Attr{LastHoldReasonCode} and
\Attr{LastHoldReasonSubCode}. Before, a hold reason could be lost if a
removed job was subsequently held.

\item The executable attribute for amazon grid universe jobs no longer
needs to be a valid file path.

\item Improved error reporting when a Xen or VMware command fails in the
\SubmitCmd{vm} universe.

\item For \SubmitCmd{vm} universe jobs,
virtual floppy disks are no longer disabled.

\item Fixed a bug introduced in Condor 7.1.4 that caused Condor to
ignore the virtual machine status reported by Xen in the \SubmitCmd{vm} universe.

\item Fixed a 20-second delay in the start up of the \Condor{c-gahp} and
the \Condor{vm-gahp}.

\item Fixed a bug which caused the net mask to be published
  into the machine ClassAd incorrectly.

\item Fixed a bug introduced in Condor 7.1.4 which could cause any
  Condor daemon to crash if the level of debugging output \MacroNI{D\_ALL}
  is enabled when a \Condor{reconfig} command is issued.

\item Fixed a bug introduced in Condor 7.1.4 which caused standard universe
jobs to fail to start up, if security authentication, but not encryption was
enabled between the submit side and the execute side.

% Commented out by Karen, as it gives no relevant information to any
% reader of this version history.  
%\item Many bugs fixed in the \Condor{job\_router} hooks.

\item Fixed a bug with streaming \File{stdin}, \File{stdout}, and
\File{stderr} when using \Prog{glexec}.

% Commented out by Karen, as it gives no relevant information to any
% reader of this version history, and has nothing to do with bugs fixed.
% \item Many improvements in error propagation and debugging output.

\end{itemize}

\noindent Known Bugs:

\begin{itemize}

\item None.

\end{itemize}

\noindent Additions and Changes to the Manual:

\begin{itemize}

\item Initial documentation for dynamic provisioning is available
in section~ \ref{sec:SMP-dynamicprovisioning}.

\end{itemize}

