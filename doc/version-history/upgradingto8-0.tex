%%%%%%%%%%%%%%%%%%%%%%%%%%%%%%%%%%%%%%%%%%%%%%%%%%%%%%%%%%%%%%%%%%%%%%
\section{\label{sec:to-8.0}Upgrading from the 7.8 series to the 8.0 series of HTCondor}
%%%%%%%%%%%%%%%%%%%%%%%%%%%%%%%%%%%%%%%%%%%%%%%%%%%%%%%%%%%%%%%%%%%%%%

\index{upgrading!items to be aware of}
While upgrading from the 7.8 series of HTCondor to the 8.0 series 
will bring many
new features and improvements introduced in the 7.9 series of HTCondor,
it will
also introduce changes that administrators of sites running from an older
HTCondor version should be aware of when planning an upgrade.  
Here is a list of items that administrators should be aware of.

\begin{itemize}

\item There is an issue with DAGMan jobs upon upgrade
from HTCondor version 7.8.x or an earlier version
to version 8.0.0.
Without administrative intervention,
queued DAGMan jobs will restart from the beginning of the DAG
after the upgrade.
There will be no issue if the upgrade is 
from HTCondor version 7.8.x or an earlier version to HTCondor version 8.0.1
or later versions.

To avoid starting DAGMan jobs from the beginning after the upgrade,
the administrator should ensure that no \Condor{dagman} jobs are queued.
Do a \Condor{rm} on all \Condor{dagman} jobs and wait for Rescue DAGs
to be written before shutting down HTCondor to perform the upgrade.
Any \Condor{dagman} jobs that are on hold should be released before
being removed.
After the upgrade is complete and HTCondor has restarted,
all of these DAGMan jobs should be re-submitted.
This will cause them to read the appropriate Rescue DAGs and 
continue on.

To avoid losing work within partially-completed node jobs,
an alternative is to use the halt file feature,
as described in section~\ref{sec:DagSuspend}.
This will cause all
\Condor{dagman} jobs to eventually drain from the queue(s).
This will take longer than doing a \Condor{rm} on those jobs.
\Condor{dagman} jobs drained via the halt file method will also
have to be re-submitted after the upgrade.

\item The upgrade will change the machine ClassAd attribute
\AdAttr{CheckpointPlatform} for all machines.
This implies that any standard universe job with a checkpoint 
from before the upgrade will not resume after the upgrade.
To work around this potential difficulty, either change the 
attribute \AdAttr{CheckpointPlatform} on all machines to their previous value 
by setting the \Macro{CHECKPOINT\_PLATFORM} configuration variable,
or change the \AdAttr{LastCheckpointPlatform} attribute for all jobs
that have produced a checkpoint.
Make the change by using \Condor{qedit}.

For example, if machine attribute \AdAttr{CheckpointPlatform} changed 
from \verb;LINUX INTEL 2.6.x normal N/A; to 
\verb;LINUX INTEL 2.6.x normal N/A ssse3 sse4_1 sse4_2;,
use the following command:

\footnotesize
\begin{verbatim}
condor_qedit -constraint 'LastCheckpointPlatform == "LINUX INTEL 2.6.x normal N/A"'
    LastCheckpointPlatform "LINUX INTEL 2.6.x normal N/A ssse3 sse4_1 sse4_2"
\end{verbatim}
\normalsize

\end{itemize}

