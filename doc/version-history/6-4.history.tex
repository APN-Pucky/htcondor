%%%%%%%%%%%%%%%%%%%%%%%%%%%%%%%%%%%%%%%%%%%%%%%%%%%%%%%%%%%%%%%%%%%%%%
\section{Stable Release Series 6.4}\label{sec:History-6-4}
%%%%%%%%%%%%%%%%%%%%%%%%%%%%%%%%%%%%%%%%%%%%%%%%%%%%%%%%%%%%%%%%%%%%%%

This is the stable release series of Condor.
New features will be added and tested in the 6.5 development series. 
The details of each version are described below.

%%%%%%%%%%%%%%%%%%%%%%%%%%%%%%%%%%%%%%%%%%%%%%%%%%%%%%%%%%%%%%%%%%%%%%
\subsection*{\label{sec:New-6-4-8}Version 6.4.8}
%%%%%%%%%%%%%%%%%%%%%%%%%%%%%%%%%%%%%%%%%%%%%%%%%%%%%%%%%%%%%%%%%%%%%%

\noindent New Features:

\begin{itemize}

\item None.

\end{itemize}

\noindent Bugs Fixed:

\begin{itemize}

\item The starter would crash if the execute directory contained
symlinks from child to parent directories.

\item condor\_submit\_dag allowed you to submit a DAG which contained
jobs with no log file (as long as ALL jobs had no logfile)

\item Dagman would RETRY your jobs as specified, but did properly
note that the failed job was no longer in the queue.  The result is
that Dagman thought there were more jobs yet to finish when in fact
all had actually finished, which caused it exit with failure.

\end{itemize}

\noindent Known Bugs:

\begin{itemize}

\item None.

\end{itemize}

%%%%%%%%%%%%%%%%%%%%%%%%%%%%%%%%%%%%%%%%%%%%%%%%%%%%%%%%%%%%%%%%%%%%%%
\subsection*{\label{sec:New-6-4-7}Version 6.4.7}
%%%%%%%%%%%%%%%%%%%%%%%%%%%%%%%%%%%%%%%%%%%%%%%%%%%%%%%%%%%%%%%%%%%%%%
\noindent New Features:
\begin{itemize}

\item. None

\end{itemize}

\noindent Bugs Fixed:
\begin{itemize}

\item Fixed major problem with inflated job counts in \Condor{status} -submitter 
\item Fixed a problem with the UserLog parser if the Aborted, Held, or Removed
event did not have the optional reason string that caused those events to be
ignored.

\item Fixed a problem with Condor-G on some older Linux distributions that
would cause it to crash on startup because of an invalid file descriptor for
stderr

\end{itemize}
\noindent Known Bugs:
\begin{itemize}

\item None.

\end{itemize}

%%%%%%%%%%%%%%%%%%%%%%%%%%%%%%%%%%%%%%%%%%%%%%%%%%%%%%%%%%%%%%%%%%%%%%
\subsection*{\label{sec:New-6-4-6}Version 6.4.6}
%%%%%%%%%%%%%%%%%%%%%%%%%%%%%%%%%%%%%%%%%%%%%%%%%%%%%%%%%%%%%%%%%%%%%%

\noindent New Features:
\begin{itemize}

\item Clarified the output of \Condor{q} -analyze.

\end{itemize}

\noindent Bugs Fixed:
\begin{itemize}

\item Fixed a major bug that caused standard universe Condor jobs to
crash whenever they ran in a pool with the \Macro{LOWPORT} and
\Macro{HIGHPORT} config file settings enabled.
These settings restrict what ports Condor will use in your pool.
To allow standard universe jobs to run in such a pool, you must relink
your executable with \Condor{compile} after the 6.4.6 condor libraries
are installed in your Condor \File{lib} directory.

\item When more than 512 distinct users submit to Condor or Condor-G,
the \Condor{schedd} no longer crashes. 

\item DAGMan now correctly reports an error and rejects DAGs
which contain two nodes with the same name; in the past DAGMan
did not detect this, and would not exit even when the DAG was
complete.

\item DAGMan now correctly reports an error when the null string
is passed to any of its arguments which require a filename.

\item The \Opt{-format} option to \Condor{q} and \Condor{status} can
now be used to print out boolean expressions in ClassAds, not just
strings, integers and floating point numbers.
Any boolean expression will now be treated like a string, so be sure
to use \verb@%s@ as the conversion character in the formatting string
you pass to \Opt{-format}.

\item If no \Attr{Rank} is specified in the job description file, or
in the \Macro{DEFAULT\_RANK} config file variable, the default value
is now ``0.0'' not just ``0''.
Since \Attr{Rank} is supposed to be a floating point value, the
``0.0'' ensures that the \Opt{-format} option to tools like \Condor{q}
and \Condor{status} will always treat this variable as a float.

\item Fixed the help message for the \Condor{rm}, \Condor{hold} and
\Condor{release} tools to be more helpful, and to mention the
\Opt{-constraint} option.

\item When you ran \Condor{q} in a way that it needed to query the
\Condor{collector} and that query failed, the error message used to be
very short and cryptic.
Now, there is a verbose message that explains what went wrong and
possible ways you can fix the problem.
Also fixed a misleading error message in \Condor{q} that was
displayed under very rare circumstances.

\end{itemize}

\noindent Packaging Changes:

\begin{itemize}

\item Begining with Condor version 6.4.6, all of the Condor-G related
binaries (\Condor{gridmanager}, \Condor{glidein}, and
\Prog{gahp\_server}) are also included in the full release of Condor.
So, if you are using the full Condor system and want to use Condor's
grid-enabled functionality, you no longer need to download and install
a separate ``contrib module''.  
However, if all you want is Condor-G, the contrib module is still
available.
For more information, please the chapter on ``Grid Computing'' on
page~\pageref{grid-computing}.

\item Fixed a process family managment bug in the \Condor{startd} on 
Windows. This bug was preventing the \Condor{startd} from detecting
the \Condor{starter} as a member of its process family.

\end{itemize}

%%%%%%%%%%%%%%%%%%%%%%%%%%%%%%%%%%%%%%%%%%%%%%%%%%%%%%%%%%%%%%%%%%%%%%
\subsection*{\label{sec:New-6-4-3}Version 6.4.3}
%%%%%%%%%%%%%%%%%%%%%%%%%%%%%%%%%%%%%%%%%%%%%%%%%%%%%%%%%%%%%%%%%%%%%%

\noindent New Features:
\begin{itemize}

\item Added a \Opt{-hold} and \Opt{-held} option to \Condor{q} which 
displays the reason that the job had been held.

\end{itemize}

\noindent Bugs Fixed:
\begin{itemize}

\item Fixed a bug where more than one space between arguments to a job
in the java universe would result in it being invoked with and incorrect
list arguments.

\item Removed renaming of the executable to \Prog{condor\_exec} in the java
universe. This fixes a bug where the JVM was looking at its path to determine
its installation directory.

\item Fixed a bug and resulting null pointer exception in the java universe
because under certain conditions, Condor would invoke the JVM incorrectly.

\item Fixed serveral error reporting messages to be more precise.

\item When the NIS environment was being used, the \Condor{starter} daemon
would produce heavy amounts of NIS traffic. This has been fixed.

\item Binary characters in the \File{StarterLog} file and a possible
segmentation fault have been fixed.

\item Fixed \Cmd{select}{2} in the standard universe on our Linux ports.

\item Fixed a small bug in \Condor{q} that was displaying the wrong
username for ``niceuser'' jobs.

\item Fixed a bug where, in the standard universe, you could not open a file
whose name had spaces in it.

\item Fixed a bug in DAGMan where pre and post scripts would fail to
run if the DAG description file had extra whitespace.
Also, reworded the error messages DAGMan produces when it fails to
parse the DAG description file to be more clear and helpful for
solving the problem.

\item Fixed some misleading error messages in the Condor log files
when there were permission problems trying to execute a program. 

\item Condor for Windows will now run on Windows XP.

\item Condor for Windows now supports the Java Universe.

\item Users logged into Windows Domain accounts rather than local accounts
can submit jobs.

\item Potential Windows registry bloating bug fixed. Condor for Windows no
longer creates and deletes an account on the execute machine each time a
job is run. Instead, a single account for each VM on the execute machine is
created once and enabled or disabled as needed.

\item Cross-submits from Windows to Unix and from Unix to Windows are now
supported, provided that both platforms are running Condor 6.4 series daemons.

\item Free disk space is now reported accurately on Windows.

\item A rare but serious bug that could allow non-Condor processes to be added
to the Condor process family on Windows has been fixed.

\item Condor for Windows will now also run 16-bit applications.

\item Fixed a minor bug where certain integer attributes in the
\File{condor\_config} file might not have been properly parsed if they
were defined in terms of other config file attributes, using the
\MacroUNI{attribute} notation.  

\end{itemize}

\noindent Known Bugs:
\begin{itemize}

\item You may not open a file in the standard universe whose name contains a
colon ``:''.

\end{itemize}

%%%%%%%%%%%%%%%%%%%%%%%%%%%%%%%%%%%%%%%%%%%%%%%%%%%%%%%%%%%%%%%%%%%%%%
\subsection*{\label{sec:New-6-4-2}Version 6.4.2}
%%%%%%%%%%%%%%%%%%%%%%%%%%%%%%%%%%%%%%%%%%%%%%%%%%%%%%%%%%%%%%%%%%%%%%
\noindent New Features:
\begin{itemize}

\item. This release mirrored the Condor-G release, and has no new features.

\end{itemize}

\noindent Bugs Fixed:
\begin{itemize}
\item None.

\end{itemize}
\noindent Known Bugs:
\begin{itemize}

\item None.

\end{itemize}

%%%%%%%%%%%%%%%%%%%%%%%%%%%%%%%%%%%%%%%%%%%%%%%%%%%%%%%%%%%%%%%%%%%%%%
\subsection*{\label{sec:New-6-4-1}Version 6.4.1}
%%%%%%%%%%%%%%%%%%%%%%%%%%%%%%%%%%%%%%%%%%%%%%%%%%%%%%%%%%%%%%%%%%%%%%
\noindent New Features:
\begin{itemize}

\item None.

\end{itemize}

\noindent Bugs Fixed:
\begin{itemize}

\item Users are now allowed to answer ``none'' when prompted by the
installer to provide a Java JVM path. This avoids an endless loop and
leaves the Java abilities of Condor unconfigured.

\end{itemize}

\noindent Known Bugs:
\begin{itemize}

\item None.

\end{itemize}

%%%%%%%%%%%%%%%%%%%%%%%%%%%%%%%%%%%%%%%%%%%%%%%%%%%%%%%%%%%%%%%%%%%%%%
\subsection*{\label{sec:New-6-4-0}Version 6.4.0}
%%%%%%%%%%%%%%%%%%%%%%%%%%%%%%%%%%%%%%%%%%%%%%%%%%%%%%%%%%%%%%%%%%%%%%

\noindent New Features:

\begin{itemize}

\item If a job universe is not specified in a submit description file, 
\Condor{submit}  will check the config file for \Macro{DEFAULT\_UNIVERSE}
instead of always choosing the standard universe. 

\item The \MacroNI{D\_SECONDS} debug flag is deprecated. Seconds are now always
included in logfiles. 

\item For each daemon listed in \MacroNI{DAEMON\_LIST}, you can now control the
environment variables of the daemon with a config file setting of the form
\MacroNI{DAEMONNAME\_ENVIRONMENT}, where \MacroNI{DAEMONNAME} is the name of a
daemon listed in \MacroNI{DAEMON\_LIST}. For more information, see
section~\ref{sec:Master-Config-File-Entries}.

\end{itemize}

\noindent Bugs Fixed:

\begin{itemize}

\item Fixed a bug in the new starter where if the submit file set no
arguments, the job would receive one argument of zero length.

\end{itemize}

\noindent Known Bugs:

\begin{itemize}

\item None.

\end{itemize}
