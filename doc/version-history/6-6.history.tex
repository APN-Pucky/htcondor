%%%%%%%%%%%%%%%%%%%%%%%%%%%%%%%%%%%%%%%%%%%%%%%%%%%%%%%%%%%%%%%%%%%%%%
\section{\label{sec:History-6-6}Stable Release Series 6.6}
%%%%%%%%%%%%%%%%%%%%%%%%%%%%%%%%%%%%%%%%%%%%%%%%%%%%%%%%%%%%%%%%%%%%%%

This is a stable release series of Condor.
It is based on the 6.5 development series.
All new features added or bugs fixed in the 6.5 series are available
in the 6.6 series.
The details of each version are described below.

%%%%%%%%%%%%%%%%%%%%%%%%%%%%%%%%%%%%%%%%%%%%%%%%%%%%%%%%%%%%%%%%%%%%%%
\subsection{\label{sec:New-6-6-1}Version 6.6.1}
%%%%%%%%%%%%%%%%%%%%%%%%%%%%%%%%%%%%%%%%%%%%%%%%%%%%%%%%%%%%%%%%%%%%%%

\noindent New Features:

\begin{itemize}

\item None.

\end{itemize}

\noindent Bugs Fixed:

\begin{itemize}

\item None.

\end{itemize}

\noindent Known Bugs:

\begin{itemize}

\item None.

\end{itemize}


%%%%%%%%%%%%%%%%%%%%%%%%%%%%%%%%%%%%%%%%%%%%%%%%%%%%%%%%%%%%%%%%%%%%%%
\subsection{\label{sec:New-6-6-0}Version 6.6.0}
%%%%%%%%%%%%%%%%%%%%%%%%%%%%%%%%%%%%%%%%%%%%%%%%%%%%%%%%%%%%%%%%%%%%%%

\noindent New Features:

\begin{itemize}

\item The \Condor{dagman} debugging log now reports the total number
      of ``Un-Ready'' Nodes (i.e. those waiting for unfinished
      dependencies) in its periodic summaries.  In the past, the
      omission of this state led to confusion because the total of all
      reported job states didn't always match the total number of jobs
      in the DAG.

\item Most Condor commands (\Condor{on}, \Condor{off},
  \Condor{restart}, \Condor{reconfig}, \Condor{vacate},
  \Condor{checkpoint}, \Condor{reschedule}) now support a \Opt{-all}
  command-line option to specify which daemons to act on.
  This is more efficient and much easier to use than previous methods
  for accomplishing the same effect.
  Using \Opt{-all} with \Condor{off} correctly leaves the existing
  \Condor{master} processes running on each host, so that a subsequent
  \Condor{on} would work.
  See section~\ref{sec:Pool-Shutdown-and-Restart} on
  page~\pageref{sec:Pool-Shutdown-and-Restart} for more details on
  proper use of \Opt{-all} with \Condor{off} and \Condor{on}

\end{itemize}

\noindent Bugs Fixed:

\begin{itemize}

\item Fixed a bug under Solaris 8 with Update 6+, and Solaris 9 where
Condor would incorrectly report the console and mouse idle times as zero.

\item The standard-universe fetch\_files feature was not cleaning up
temporary files on the execution machine.

\item In rare circumstances, a Linux kernel bug results in conflicting
information about system boot time (\File{/proc/stat} and
\File{/proc/uptime}). 
Specifically, the "btime" field in \File{/proc/stat} suddenly jumps to
the present moment and then stays at that value.  This
was resulting in incorrect estimation of process ages, which caused
Condor's estimation of CondorLoadAvg to be completely wrong.  A more
robust heuristic is now being used.

\item A long configuration line with with continuation lines can cause the
config file parser to not properly skip the leading whitespace from
the continued lines.  This has been corrected.

\item The Grid Monitor now will automatically probe for and work with
``unknown'' batch systems.

\item Fixed a bug where under certain circumstances \Condor{dagman}
      would fail to detect an unsuccessful invocation of
      \Condor{submit}, and would instead report the job as
      successfully submitted with job id 0.0.

\item Fixed a bug which was causing problems when a periodic\_remove
expression for a scheduler universe job evaluates to true.  Under
these conditions, the schedd did not log the job terminiation to the
job log.  Addtionally, the schedd would exit with an error status.

\item Fixed a recently-introduced \Condor{dagman} bug where the number
      of node retries (specified with the RETRY keyword) wasn't being
      updated after some failures; instead, the node would be allowed
      to retry indefinitely if it kept failing.

\item Fixed a recently-introduced bug where shutting down the
      \Condor{schedd} caused \Condor{dagman} to remove all its jobs
      from the queue and write a rescue file, rather than simply
      exiting so that it could recover automatically upon restart.

\item Changed the default ``Periodic Expression Interval'' parameter
(PERIODIC\_EXPR\_INTERVAL) from 60 seconds to 300 seconds.

\item Whenever \Condor{reconfig} was used to re-configure multiple
  daemons which included the \Condor{collector} for a pool, the
  command would start to fail after the \Condor{collector} was
  reconfigured due to problems with security sessions in Condor's
  strong authentication code.
  This situation no longer causes problems for the \Condor{reconfig}
  tool, and it can properly re-configure multiple daemons at once,
  even if one of them is the \Condor{collector} for a pool.

\item Most Condor commands (\Condor{on}, \Condor{off},
  \Condor{restart}, \Condor{reconfig}, \Condor{vacate},
  \Condor{checkpoint}, \Condor{reschedule}) now check to make sure
  they are not sending a duplicate command if the user specifies the
  same target machine or daemon twice.  For example:
\begin{verbatim}
     condor_reconfig hostname1 hostname2 hostname1
\end{verbatim}
  will only send a single reconfig command to \verb@hostname1@.

\item Fixed a bug in the HPUX version of Condor which was causing the
startd to occasionally abort operation.  This has been in Condor since
version 6.1.1.

\item The Condor daemons will no longer overwhelm NIS servers
when large numbers of daemons are running. Condor now caches
uid and group information internally, and refreshes the
cache entries on a specified interval (which defaults to 5
minutes). See section~\ref{param:PasswdCacheRefresh} on
page~\pageref{param:PasswdCacheRefresh} for more details.

\end{itemize}

\noindent Known Bugs:

\begin{itemize}

\item None.

\end{itemize}



\begin{center}
\begin{table}[hbt]
\begin{tabular}{|ll|} \hline
\emph{Architecture} & \emph{Operating System} \\ \hline \hline
Hewlett Packard PA-RISC (both PA7000 and PA8000 series) & HPUX 10.20 \\ \hline
Sun SPARC Sun4m,Sun4c, Sun UltraSPARC & Solaris 2.6, 2.7, 8, 9 \\ \hline
Silicon Graphics MIPS (R5000, R8000, R10000) & IRIX 6.5 \\ \hline
Intel x86 & Red Hat Linux 7.1, 7.2, 7.3 \\
 & Red Hat Linux 8 (clipped) \\ \hline
 & Red Hat Linux 9 (clipped) \\ \hline
 & Windows NT 4.0 Workstation and Server (clipped) \\ \hline
 & Windows 2000 Professional and Server, 2003 Server (clipped) \\ \hline
 & Windows XP Professional (clipped) \\ \hline
ALPHA & Digital Unix 4.0 \\
 & Red Hat Linux 7.1, 7.2, 7.3 (clipped) \\ \hline
 & Tru64 5.1 (clipped) \\ \hline
PowerPC & Macintosh OS X (clipped) \\
Itanium & Red Hat Linux 7.1, 7.2, 7.3 (clipped) \\
\end{tabular}
\caption{\label{6.6.0-supported-platforms}Condor version 6.6.0 supported platforms}
\end{table}
\end{center}

