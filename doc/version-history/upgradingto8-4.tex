%%%%%%%%%%%%%%%%%%%%%%%%%%%%%%%%%%%%%%%%%%%%%%%%%%%%%%%%%%%%%%%%%%%%%%
\section{Upgrading from the 8.2 series to the 8.4 series of HTCondor}\label{sec:to-8.4}
%%%%%%%%%%%%%%%%%%%%%%%%%%%%%%%%%%%%%%%%%%%%%%%%%%%%%%%%%%%%%%%%%%%%%%

\index{upgrading!items to be aware of}
Upgrading from the 8.2 series of HTCondor to the 8.4 series
will bring new features introduced in the 8.3 series of HTCondor.
These new features include:
\begin{itemize}

\item Implemented numerous scalability changes (such as reducing memory
footprint, using fewer system resources, and streamlined algorithms) to
handle 200,000 simultaneously running HTCondor jobs in a single pool.

\item Added a Docker Universe to run a Docker container as an HTCondor job.

\item New features increase the power of job specification
in the submit description file.

  \begin{itemize}
  \item Submit description files are now parsed the same as configuration files.

  \item The \SubmitCmd{queue} submit command may be used in
flexible and powerful new ways to specify job submissions.
See section~\ref{sec:user-man-queue} for details.
\Ticket{4819}

  \item New macro functions are supported,
and may be used in submit description files as well as in configuration.
\Ticket{4944}

  \item \Condor{submit} has new command line options \Opt{-queue}
and \Opt{-dry-run},
to provide flexible and powerful new ways to specify job submissions,
as well as to test what job would be submitted without submitting.
\Ticket{4933}

  \item \Condor{submit} now supports assignment of ClassAd attributes
on the command line.
\Ticket{4983}

  \item \Condor{submit} accepts \SubmitCmdNI{if} and \SubmitCmdNI{include}
statements in the same way that configuration files do.
\Ticket{4913}

  \end{itemize}

\item HTCondor pools can use IPv4 and IPv6 simultaneously.

\item Execute Directories can be encrypted upon user or administrator request.

\item Vanilla Universe jobs can utilize periodic application-level checkpoints.

\item The administrator can establish requirements that must be satisfied in
order for a job to be queued.

\end{itemize}

Upgrading from the 8.2 series of HTCondor to the 8.4 series will
also introduce changes that administrators of sites running from an older
HTCondor version should be aware of when planning an upgrade.
Here is a list of items that administrators should be aware of.

\begin{itemize}

\item New configuration and changes to existing configuration:
  \begin{itemize}

\item The RPM packages have been restructured to allow running a 32-bit
static shadow on Red Hat Enterprise Linux 6. The new \texttt{condor-all}
RPM is used to install all of the RPMs for a typical HTCondor installation.
Since the binary distribution of HTCondor for Red Hat Enterprise Linux 6 and 7
consists of more that a handful of RPMs, the RPMs are only available from our
YUM repository.
\Ticket{4621}

\item If enabled, the \Condor{shared\_port} daemon will now use port 9618
instead of the previous default, which was to randomly select a port from
the allowed range (from \MacroNI{LOWPORT} to \MacroNI{HIGHPORT}; see section
see section \ref{sec:Ports-Firewalls}).  To restore \Condor{shared\_port}'s
previous behavior, set \MacroNI{SHARED\_PORT\_PORT} to \Expr{0}.
\Ticket{4752}

\item Configuration variable \Macro{MAX\_JOBS\_RUNNING} has been
modified such that it only applies to job universes that require a
\Condor{shadow} process.
Scheduler and local universe jobs are no longer affected by this
variable.
The number of running scheduler and local universe jobs can be controlled
with configuration variables \MacroNI{START\_SCHEDULER\_UNIVERSE} and
\MacroNI{START\_LOCAL\_UNIVERSE}, respectively.
\Ticket{4589}

\item ClassAd attributes written by the \Condor{schedd} that
count the number of jobs in various states now include all jobs,
not only jobs that need to be matched by the \Condor{negotiator} daemon.
These attributes include \Attr{TotalRunningJobs}, \Attr{TotalIdleJobs},
\Attr{TotalHeldJobs}, and \Attr{TotalRemovedJobs}.
\Ticket{4683}

\item On Linux platforms, the \Condor{master} daemon now runs a script when it
starts up.
This script tunes several Linux kernel parameters to the values
we suggest for better scalability.
New configuration variables \Macro{ENABLE\_KERNEL\_TUNING},
\Macro{KERNEL\_TUNING\_LOG}, and \Macro{LINUX\_KERNEL\_TUNING\_SCRIPT}
enable the use of the script and specify file locations.
\Ticket{5126}

\item The default values of the configuration variables and ClassAd attributes
listed in Table~\ref{table:new-knob-defaults} have changed,
such that the default now represents the commonly configured value.

% This table must be formatted oddly, to make the pdf version look OK.
\begin{center}
\begin{table}[hbt]
\begin{tabular}{|l|c|c|} \hline
\textbf{Variable Name} & \textbf{Previous Default} & \textbf{New Default}\\ \hline \hline
\MacroNI{NEGOTIATOR\_INFORM\_STARTD} & \Expr{True} & \Expr{False}  \\ \hline
\MacroNI{MAX\_JOBS\_PER\_SUBMISSION} & largest positive integer & 20000  \\ \hline
\MacroNI{MAX\_JOBS\_PER\_OWNER} & largest positive integer & 100000  \\ \hline
\MacroNI{MAX\_JOBS\_RUNNING} (on Windows) & 200 & 2000  \\ \hline
\MacroNI{PRIVATE\_NETWORK\_NAME} & no default & \MacroUNI{FULL\_HOSTNAME}  \\ \hline
\MacroNI{JobLeaseDuration} & 20 minutes & 40 minutes \\ \hline
\MacroNI{WANT\_VACATE} & \Expr{False} & \Expr{True} \\ \hline
\MacroNI{SCHEDD\_SEND\_VACATE\_VIA\_TCP} & \Expr{False} & \Expr{True} \\ \hline
\end{tabular}
\caption{\label{table:new-knob-defaults}Changes to defaults in HTCondor 8.3.7}
\end{table}
\end{center}

  \end{itemize}

\item None.

\end{itemize}

