%%%      PLEASE RUN A SPELL CHECKER BEFORE COMMITTING YOUR CHANGES!
%%%      PLEASE RUN A SPELL CHECKER BEFORE COMMITTING YOUR CHANGES!
%%%      PLEASE RUN A SPELL CHECKER BEFORE COMMITTING YOUR CHANGES!
%%%      PLEASE RUN A SPELL CHECKER BEFORE COMMITTING YOUR CHANGES!
%%%      PLEASE RUN A SPELL CHECKER BEFORE COMMITTING YOUR CHANGES!

%%%%%%%%%%%%%%%%%%%%%%%%%%%%%%%%%%%%%%%%%%%%%%%%%%%%%%%%%%%%%%%%%%%%%%
\section{\label{sec:History-7-9}Development Release Series 7.9}
%%%%%%%%%%%%%%%%%%%%%%%%%%%%%%%%%%%%%%%%%%%%%%%%%%%%%%%%%%%%%%%%%%%%%%

This is the development release series of Condor.
The details of each version are described below.

%%%%%%%%%%%%%%%%%%%%%%%%%%%%%%%%%%%%%%%%%%%%%%%%%%%%%%%%%%%%%%%%%%%%%%
\subsection*{\label{sec:New-7-9-1}Version 7.9.1}
%%%%%%%%%%%%%%%%%%%%%%%%%%%%%%%%%%%%%%%%%%%%%%%%%%%%%%%%%%%%%%%%%%%%%%

\noindent Release Notes:

\begin{itemize}

\item Condor version 7.9.1 not yet released.
%\item Condor version 7.9.1 released on Month Date, 2012.

\item Condor no longer looks for its main configuration file in the
location \File{\MacroUNI{GLOBUS\_LOCATION}/etc/condor\_config}.
\Ticket{2830}

\end{itemize}


\noindent New Features:

\begin{itemize}

\item None.

\end{itemize}

\noindent Configuration Variable and ClassAd Attribute Additions and Changes:

\begin{itemize}

\item Added the attribute \Attr{DAGManNodesMask} to control the verboseness of
the log referred to by \Attr{DAGManNodesLog}.

\end{itemize}

\noindent Bugs Fixed:

\begin{itemize}

\item None.

\end{itemize}

\noindent Known Bugs:

\begin{itemize}

\item None.

\end{itemize}

\noindent Additions and Changes to the Manual:

\begin{itemize}

\item None.

\end{itemize}


%%%%%%%%%%%%%%%%%%%%%%%%%%%%%%%%%%%%%%%%%%%%%%%%%%%%%%%%%%%%%%%%%%%%%%
\subsection*{\label{sec:New-7-9-0}Version 7.9.0}
%%%%%%%%%%%%%%%%%%%%%%%%%%%%%%%%%%%%%%%%%%%%%%%%%%%%%%%%%%%%%%%%%%%%%%

\noindent Release Notes:

\begin{itemize}

\item Condor version 7.9.0 not yet released.
%\item Condor version 7.9.0 released on Month Date, 2012.

\end{itemize}


\noindent New Features:

\begin{itemize}

\item Added support for classad caching to reduce memory footprint. 
\Ticket{2541}

\item Added configuration variable CLASSAD\_LOG\_STRICT\_PARSING, which defaults to 
\Expr{True}, and can be used to control whether ClassAd log files such as the job queue
log are read in with strict parse checking on ClassAd expressions.
\Ticket{3069}

\item \Condor{status} now returns the schedd ad directly from the schedd daemon
if both -direct and -schedd are given on the command line.
\Ticket{2492}

\item Added status and echo command line options to \Condor{wait} command.
\Ticket{2926}

\item Added a \Expr{DEBUG} logging level output flag \Dflag{CATEGORY},
which causes Condor to include the logging level
flags in effect for each line of logged output.
\Ticket{2712}

\item \Condor{status} and \Condor{q} each have a new \Opt{-autoformat} option
to make some output format specifications easier than the existing
\Opt{-format} option.
See the \Condor{status} manual page located on page~\pageref{man-condor-status}
and the \Condor{q} manual page located on page~\pageref{man-condor-q} 
for details.
\Ticket{2941}

\item Enhanced the classad log system (e.g. the job que log) to report the record number on parse failures, 
and improved its ability to detect parse failures closer to the point of corruption.
\Ticket{2934}

\item Added an \Opt{-evaluate} option to \Condor{config\_val}, which causes the configured value queried from
a given daemon to be evaluated with respect to that daemon's ClassAd.
\Ticket{856}

\item Added code to \Condor{dagman},
such that a \Expr{VARS} assignment in a top-level DAG is applied to splices.
\Ticket{1780}

\item Condor now uses libraries from Globus 5.2.1.
\Ticket{2838}

\item When authenticating Condor daemons with GSI and
\MacroNI{GSI\_DAEMON\_NAME} is undefined, Condor checks
that the server name in the certificate matches the hostname that
the client is connecting to.  When \MacroNI{GSI\_DAEMON\_NAME} is
defined, the old behavior is preserved: only certificates matching
\MacroNI{GSI\_DAEMON\_NAME} pass the authentication step, and no
hostname check is performed.  The behavior of the hostname check
may be further controlled with the new configuration settings
\MacroNI{GSI\_SKIP\_HOST\_CHECK} and
\MacroNI{GSI\_SKIP\_HOST\_CHECK\_CERT\_REGEX}.

\item Added new capability to \Condor{submit} to allow recursive macros in
submit files. This facility allows one to update variables recursively.  An
example of a construction that works is the following: \texttt{foo = bar},
followed by \texttt{foo =  snap \$(foo)}.  Then the value of foo will be
\texttt{foo = snap bar}.  Previously, this would send \Condor{submit} into an
infinite loop of expanding the macro so that the expansion would be \texttt{foo
= snap bar bar ...}, filling up memory with copies of ``\texttt{bar}''. Note
that both left- and right- recursion works.  The construction \texttt{foo =
\$(foo) bar} by itself will not work, as it does not have an initial base case.
Mutually recursive constructions such as: \texttt{B = bar}, \texttt{C = \$(B)},
and \texttt{B= \$(C) boo} will also fill memory with expansions.
\Ticket{406}

\item Removed restriction on having macros in the "log" directive in submit
files for \Condor{dagman}
\Ticket{2428}

\item \Condor{dagman} now uses and watches, by default, only its default node
userlog for events.  DAGman now requests the schedd and shadow to write each
event to the default log, i.e., $\langle$DAG file$\rangle$.nodes.log, along
with whatever log is specified by the user.  \Condor{dagman} only writes POST
Script TERMINATE events to its default log, not to the user log. This can be
turned off by setting the configuration variable
\Macro{DAGMAN\_ALWAYS\_USE\_NODE\_LOG} to False.  In particular,
\Macro{DAGMAN\_ALWAYS\_USE\_NODE\_LOG} should be set to false if this newer
version of \Condor{dagman} is submitting jobs to an older version of
\Condor{schedd} or using an older version of \Condor{submit}
\Ticket{2807}

\end{itemize}

\noindent Configuration Variable and ClassAd Attribute Additions and Changes:

\begin{itemize}

\item The default value for configuration variable \Macro{USE\_PROCD}
is now \Expr{True} for the \Condor{master} daemon.  
This means that by
default the \Condor{master} will start a \Condor{procd} daemon to be used 
by it and all other daemons on that machine.
\Ticket{2911}

\item There is a new configuration variable used by the \Condor{starter}.
If \Macro{STARTER\_RLIMIT\_AS} is set to an integer value, 
the \Condor{starter}
will use the \Procedure{setrlimit} system call with the 
\Code{RLIMIT\_AS} parameter to
limit the virtual memory size of each process in the user job.  
The value of this configuration variable is a ClassAd expression, 
evaluated in the context of both the machine and job ClassAds, 
where the machine ClassAd is the \Expr{my} ClassAd, 
and the job ClassAd is the \Expr{target} ClassAd.
\Ticket{1663}

\item New configuration variables were added to to the \Condor{schedd} to
define statistics that count subsets of jobs. 
These variables have the form \Macro{SCHEDD\_COLLECT\_STATS\_BY\_<Name>},
and should be defined by a ClassAd expression that evaluates to a string.
See section~\ref{param:ScheddCollectStatsByName}
for the complete definition.
The optional configuration variable of the form
\Macro{SCHEDD\_EXPIRE\_STATS\_BY\_<Name>} can be used to set an expiration time,
in seconds, for each set of statistics.
\Ticket{2862}

\item The new \SubmitCmd{batch\_queue} submit description file command
and new job ClassAd attribute \Attr{BatchQueue} specify which job
queue to use for grid universe jobs of type
\SubmitCmd{pbs}, \SubmitCmd{lsf}, and \SubmitCmd{sge}.
\Ticket{2996}

\item The new configuration macro \MacroNI{GSI\_SKIP\_HOST\_CHECK} is
a boolean setting that controls whether a check is performed during
GSI authentication of a Condor daemon.  When false (the default),
the check is not skipped, so the daemon hostname must match the
hostname in the daemon's certificate, unless otherwise exempted
by \MacroNI{GSI\_DAEMON\_NAME} or
\MacroNI{GSI\_SKIP\_HOST\_CHECK\_CERT\_REGEX}.
When true, this check is skipped, and hosts will not be rejected
due to a mismatch of certificate and hostname.

\item The new configuration macro
\MacroNI{GSI\_SKIP\_HOST\_CHECK\_CERT\_REGEX} may be set to a
regular expression.  GSI certificates of Condor daemons with a
subject name that are matched in full by this regular expression
are not required to have a matching daemon hostname and certificate
hostname.  The default is an empty regular expression, which will
not match any certificates (even if they have an empty subject name).

\end{itemize}

\noindent Bugs Fixed:

\begin{itemize}

\item The EC2 GAHP will now respect the value of the environment variable
\Env{X509\_CERT\_DIR} and the configuration variable
\Macro{GSI\_DAEMON\_TRUSTED\_CA\_DIR} for \emph{all} secure connections.
\Ticket{2823}

\item Condor will avoid selecting down (disabled) network interfaces.  Previously Condor could select a down interface over an up (active) interface.
\Ticket{2893}

\item Made logic in the \Condor{negotiator} that computes submitter limits 
properly aware of the configuration variable
\Macro{NEGOTIATOR\_CONSIDER\_PREEMPTION}.
\Ticket{2952}


\item Condor no longer back-dates file modification times by 3 minutes
when transferring job input files into the job spool directory or the job
execute directory.
\Ticket{2423}

\end{itemize}

\noindent Known Bugs:

\begin{itemize}

\item None.

\end{itemize}

\noindent Additions and Changes to the Manual:

\begin{itemize}

\item None.

\end{itemize}


