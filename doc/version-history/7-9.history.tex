%%%      PLEASE RUN A SPELL CHECKER BEFORE COMMITTING YOUR CHANGES!
%%%      PLEASE RUN A SPELL CHECKER BEFORE COMMITTING YOUR CHANGES!
%%%      PLEASE RUN A SPELL CHECKER BEFORE COMMITTING YOUR CHANGES!
%%%      PLEASE RUN A SPELL CHECKER BEFORE COMMITTING YOUR CHANGES!
%%%      PLEASE RUN A SPELL CHECKER BEFORE COMMITTING YOUR CHANGES!

%%%%%%%%%%%%%%%%%%%%%%%%%%%%%%%%%%%%%%%%%%%%%%%%%%%%%%%%%%%%%%%%%%%%%%
\section{\label{sec:History-7-9}Development Release Series 7.9}
%%%%%%%%%%%%%%%%%%%%%%%%%%%%%%%%%%%%%%%%%%%%%%%%%%%%%%%%%%%%%%%%%%%%%%

This is the development release series of Condor.
The details of each version are described below.

%%%%%%%%%%%%%%%%%%%%%%%%%%%%%%%%%%%%%%%%%%%%%%%%%%%%%%%%%%%%%%%%%%%%%%
\subsection*{\label{sec:New-7-9-0}Version 7.9.0}
%%%%%%%%%%%%%%%%%%%%%%%%%%%%%%%%%%%%%%%%%%%%%%%%%%%%%%%%%%%%%%%%%%%%%%

\noindent Release Notes:

\begin{itemize}

\item Condor version 7.9.0 not yet released.
%\item Condor version 7.9.0 released on Month Date, 2012.

\end{itemize}


\noindent New Features:

\begin{itemize}

\item Condor slots can now be configured with customized local 
machine resources via \Macro{MACHINE\_RESOURCE\_NAMES} and 
\Macro{MACHINE\_RESOURCE\_<name>}, then including local resource
names on \Macro{SLOT\_TYPE\_<N>}.  Submitted jobs may request
these resources using the standard request\_<name> syntax.
See section~\ref{sec:SMP-Local-Machine-Resources}.
\Ticket{2905}

\item Condor now supports and implements the caching of ClassAds 
to reduce memory footprints. Caching may be disabled by setting
the new configuration variable \Macro{ENABLE\_CLASSAD\_CACHING}
to \Expr{False}.
Third party use of the ClassAd library does not enable the
caching of ClassAds by default; 
Condor does enable the caching of ClassAds by default.
\Ticket{2541}
\Ticket{3127}

\item \Condor{status} now returns the \Condor{schedd} ClassAd directly 
from the \Condor{schedd} daemon,
if both options \Opt{-direct} and \Opt{-schedd} are given on the command line.
\Ticket{2492}

\item The new \Opt{-status} and \Opt{-echo} command line options to 
\Condor{wait} command cause it to show job start and terminate information,
and to print events to \Code{stdout}.
\Ticket{2926}

\item Added a \Expr{DEBUG} logging level output flag \Dflag{CATEGORY},
which causes Condor to include the logging level
flags in effect for each line of logged output.
\Ticket{2712}

\item \Condor{status} and \Condor{q} each have a new \Opt{-autoformat} option
to make some output format specifications easier than the existing
\Opt{-format} option.
See the \Condor{status} manual page located on page~\pageref{man-condor-status}
and the \Condor{q} manual page located on page~\pageref{man-condor-q} 
for details.
\Ticket{2941}

\item Enhanced the ClassAd log system to report the log line number 
on parse failures, 
and improved the ability to detect parse failures closer to 
the point of corruption.
\Ticket{2934}

\item Added an \Opt{-evaluate} option to \Condor{config\_val}, which causes the configured value queried from
a given daemon to be evaluated with respect to that daemon's ClassAd.
\Ticket{856}

\item Added code to \Condor{dagman},
such that a \Expr{VARS} assignment in a top-level DAG is applied to splices.
\Ticket{1780}

\item Condor now uses libraries from Globus 5.2.1.
\Ticket{2838}

\item When authenticating Condor daemons with GSI and
configuration variable \MacroNI{GSI\_DAEMON\_NAME} is undefined, 
Condor checks that the server name in the certificate matches the 
host name that the client is connecting to. 
When \MacroNI{GSI\_DAEMON\_NAME} is defined,
the old behavior is preserved: only certificates matching
\MacroNI{GSI\_DAEMON\_NAME} pass the authentication step, 
and no host name check is performed.  
The behavior of the host name check
may be further controlled with the new configuration variables
\MacroNI{GSI\_SKIP\_HOST\_CHECK} and
\MacroNI{GSI\_SKIP\_HOST\_CHECK\_CERT\_REGEX}.
\Ticket{1605}

\item Added new capability to \Condor{submit} to allow recursive macros in
submit description files. 
This facility allows one to update variables recursively. 
Before this new capability was added,
recursive definition would send \Condor{submit} into an
infinite loop of expanding the macro,
such that the expansion would fill up memory.
See section~\ref{macro-in-submit-description-file} for details.
\Ticket{406}

\item A DAGMan limitation and restriction has been removed.  
It is now permitted to define a \SubmitCmd{log} command using a macro,
within a node job's submit description file.
\Ticket{2428}

\item To enforce the dependencies of a DAG,
DAGMan now uses and watches only the default node
user log of the \Condor{dagman} job for events.  
DAGMan requests the \Condor{schedd} and \Condor{shadow} daemons to write each
event to this default log, 
in addition to writing to a log specified by the node job.
\Condor{dagman} writes POST script terminate events only to its default log;
these terminate events are not written to the user log.
This behavior can be turned off by setting the configuration variable
\Macro{DAGMAN\_ALWAYS\_USE\_NODE\_LOG} to \Expr{False}.
For correct behavior,
\MacroNI{DAGMAN\_ALWAYS\_USE\_NODE\_LOG} should be set to \Expr{False}
if \Condor{dagman} version 7.9.0 or later is submitting jobs 
to an older version of
a \Condor{schedd} daemon or of a \Condor{submit} executable.
\Ticket{2807}

\end{itemize}

\noindent Configuration Variable and ClassAd Attribute Additions and Changes:

\begin{itemize}

\item The new configuration variable \MacroNI{ENABLE\_CLASSAD\_CACHING}
controls whether the new caching feature of ClassAds is used.
The default value is \Expr{True}.
\Ticket{3127}

\item The new configuration variable \Macro{CLASSAD\_LOG\_STRICT\_PARSING}
controls whether ClassAd log files such as the job queue
log are read with strict parse checking on ClassAd expressions.
\Ticket{3069}

\item The default value for configuration variable \Macro{USE\_PROCD}
is now \Expr{True} for the \Condor{master} daemon.  
This means that by
default the \Condor{master} will start a \Condor{procd} daemon to be used 
by it and all other daemons on that machine.
\Ticket{2911}

\item There is a new configuration variable used by the \Condor{starter}.
If \Macro{STARTER\_RLIMIT\_AS} is set to an integer value, 
the \Condor{starter}
will use the \Procedure{setrlimit} system call with the 
\Code{RLIMIT\_AS} parameter to
limit the virtual memory size of each process in the user job.  
The value of this configuration variable is a ClassAd expression, 
evaluated in the context of both the machine and job ClassAds, 
where the machine ClassAd is the \Expr{my} ClassAd, 
and the job ClassAd is the \Expr{target} ClassAd.
\Ticket{1663}

\item New configuration variables were added to to the \Condor{schedd} to
define statistics that count subsets of jobs. 
These variables have the form \Macro{SCHEDD\_COLLECT\_STATS\_BY\_<Name>},
and should be defined by a ClassAd expression that evaluates to a string.
See section~\ref{param:ScheddCollectStatsByName}
for the complete definition.
The optional configuration variable of the form
\Macro{SCHEDD\_EXPIRE\_STATS\_BY\_<Name>} can be used to set an expiration time,
in seconds, for each set of statistics.
\Ticket{2862}

\item The new \SubmitCmd{batch\_queue} submit description file command
and new job ClassAd attribute \Attr{BatchQueue} specify which job
queue to use for grid universe jobs of type
\SubmitCmd{pbs}, \SubmitCmd{lsf}, and \SubmitCmd{sge}.
\Ticket{2996}

\item The new configuration variable \Macro{GSI\_SKIP\_HOST\_CHECK} is
a boolean that controls whether a check is performed during
GSI authentication of a Condor daemon.  
When the default value \Expr{False},
the check is not skipped, so the daemon host name must match the
host name in the daemon's certificate, unless otherwise exempted
by values of \MacroNI{GSI\_DAEMON\_NAME} or
\MacroNI{GSI\_SKIP\_HOST\_CHECK\_CERT\_REGEX}.
When \Expr{True}, this check is skipped, and hosts will not be rejected
due to a mismatch of certificate and host name.
\Ticket{1605}

\item The new configuration variable
\MacroNI{GSI\_SKIP\_HOST\_CHECK\_CERT\_REGEX} may be set to a
regular expression.  GSI certificates of Condor daemons with a
subject name that are matched in full by this regular expression
are not required to have a matching daemon host name and certificate
host name.  The default is an empty regular expression, which will
not match any certificates, even if they have an empty subject name.
\Ticket{1605}

\end{itemize}

\noindent Bugs Fixed:

\begin{itemize}

\item The EC2 GAHP will now respect the value of the environment variable
\Env{X509\_CERT\_DIR} and the configuration variable
\Macro{GSI\_DAEMON\_TRUSTED\_CA\_DIR} for \emph{all} secure connections.
\Ticket{2823}

\item Condor will avoid selecting down (disabled) network interfaces.  Previously Condor could select a down interface over an up (active) interface.
\Ticket{2893}

\item Made logic in the \Condor{negotiator} that computes submitter limits 
properly aware of the configuration variable
\Macro{NEGOTIATOR\_CONSIDER\_PREEMPTION}.
\Ticket{2952}


\item Condor no longer back-dates file modification times by 3 minutes
when transferring job input files into the job spool directory or the job
execute directory.
\Ticket{2423}

\end{itemize}

\noindent Known Bugs:

\begin{itemize}

\item None.

\end{itemize}

\noindent Additions and Changes to the Manual:

\begin{itemize}

\item Machine ClassAd attribute string values relating to \Attr{OpSys} have
been documented for Scientific Linux platforms.
\Ticket{2366}

\end{itemize}


