%%%      PLEASE RUN A SPELL CHECKER BEFORE COMMITTING YOUR CHANGES!
%%%      PLEASE RUN A SPELL CHECKER BEFORE COMMITTING YOUR CHANGES!
%%%      PLEASE RUN A SPELL CHECKER BEFORE COMMITTING YOUR CHANGES!
%%%      PLEASE RUN A SPELL CHECKER BEFORE COMMITTING YOUR CHANGES!
%%%      PLEASE RUN A SPELL CHECKER BEFORE COMMITTING YOUR CHANGES!

%%%%%%%%%%%%%%%%%%%%%%%%%%%%%%%%%%%%%%%%%%%%%%%%%%%%%%%%%%%%%%%%%%%%%%
\section{\label{sec:History-7-5}Development Release Series 7.5}
%%%%%%%%%%%%%%%%%%%%%%%%%%%%%%%%%%%%%%%%%%%%%%%%%%%%%%%%%%%%%%%%%%%%%%

This is the development release series of Condor.
The details of each version are described below.

%%%%%%%%%%%%%%%%%%%%%%%%%%%%%%%%%%%%%%%%%%%%%%%%%%%%%%%%%%%%%%%%%%%%%%
\subsection*{\label{sec:New-7-5-0}Version 7.5.0}
%%%%%%%%%%%%%%%%%%%%%%%%%%%%%%%%%%%%%%%%%%%%%%%%%%%%%%%%%%%%%%%%%%%%%%

\noindent Release Notes:

\begin{itemize}

\item None.

\end{itemize}


\noindent New Features:

\begin{itemize}

% gittrack #892
\item Added the new daemon \Condor{shared\_port} for Unix platforms 
  (except for HPUX).
  It allows Condor daemons to share a
  single network port.  This makes opening access to Condor through a
  firewall easier and safer.  It also increases the scalability of a
  submit node by decreasing port usage. See
  section~\ref{sec:Config-shared-port} for more information.

% gittrack #960
\item Improved CCB's handling of rude NAT/firewalls that silently drop
TCP connections.

% gittrack #968
\item Simplified the publication of daemon addresses.
  \Attr{PublicNetworkIpAddr} and \Attr{PrivateNetworkIpAddr} have been removed.
  \Attr{MyAddress} contains both public and private addresses.  For now,
  \Attr{<Subsys>IpAddr} contains the same information.  In a future release,
  the latter may be removed.

% gittrack #975
\item Changes to \MacroNI{TCP\_FORWARDING\_HOST},
  \MacroNI{PRIVATE\_NETWORK\_ADDRESS}, and
  \MacroNI{PRIVATE\_NETWORK\_NAME} can now be made without requiring a
  full restart.  It may take up to one \Condor{collector} update interval 
  for the changes to become visible.

% gittrack #1002
\item Network compatibility with Condor prior to 6.3.3 is no longer
  supported unless \MacroNI{SEC\_CLIENT\_NEGOTIATION} is set to
  \Expr{NEVER}.  This change removes the risk of communication errors
  causing performance problems resulting from automatic fall-back to the
  old protocol.

\end{itemize}

\noindent Configuration Variable and ClassAd Attribute Additions and Changes:

\begin{itemize}

% gittrack #997
\item Removed the configuration variable 
  \MacroNI{COLLECTOR\_SOCKET\_CACHE\_SIZE}.
  Configuration of this parameter used to be mandatory to enable TCP updates
  to the \Condor{collector}.  Now no special configuration of the
  \Condor{collector} is required to allow TCP updates, but it is
  important to ensure that there are sufficient file descriptors for
  efficient operation.  See section~\ref{sec:tcp-collector-update} for
  more information.

% gittrack #892
\item The new configuration variable \MacroNI{USE\_SHARED\_PORT} 
  is a boolean value that specifies
  whether a Condor process should rely on the \Condor{shared\_port} daemon for
  receiving incoming connections.  Write access to
  \Macro{DAEMON\_SOCKET\_DIR} is required for this to take effect.
  The default is \Expr{False}.  If set to \Expr{True}, \MacroNI{SHARED\_PORT}
  should be added to \MacroNI{DAEMON\_LIST}.  See
  section~\ref{sec:Config-shared-port} for more information.

% gittrack #960
\item Added the new configuration variable \MacroNI{CCB\_HEARTBEAT\_INTERVAL}.
  It is the maximum
  number of seconds of silence on a daemon's connection to the CCB server
  after which it will ping the server to verify that the connection still
  works.  
  The default value is 1200 (20 minutes).
  This feature serves to both speed
  up detection of dead connections and to generate a guaranteed minimum
  frequency of activity to attempt to prevent the connection from being
  dropped.

\end{itemize}

\noindent Bugs Fixed:

\begin{itemize}

\item None.

\end{itemize}

\noindent Known Bugs:

\begin{itemize}

\item None.

\end{itemize}

\noindent Additions and Changes to the Manual:

\begin{itemize}

\item None.

\end{itemize}
