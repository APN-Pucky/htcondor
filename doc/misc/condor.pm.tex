The Condor perl module facilitates automatic submitting and monitoring of
condor jobs, along with automated administration of condor.  The most common
use of the perl module is the monitoring of condor jobs.  The condor perl
module uses the user log of a condor job for monitoring.

The Condor perl module is made up of several subroutines.  Many subroutines 
take other subroutines as arguments.  These subroutines are used as callbacks
which are called when interesting events happen.

\subsection{Subroutines}
\begin{enumerate}
	\item Submit(command\_file) \\
	The submit subroutine takes a command file name as an argument and
	submits it to condor.  The \Condor{submit} program should be in the
	path of the user.  If the user wishes to monitor the job with condor
	they must specify a log file in the command file.  The cluster
	submitted is returned. For more information
	see the \Condor{submit} man page.
	
	\item Vacate(machine) \\
	Vacate the machine specified.  The machine may be specified
	either by hostname, or by \Term{sinful string}.  For more information
	see the \Condor{vacate} man page.

	\item Reschedule(machine) \\
	Reschedule the machine specified.  The machine may be specified either
 	by hostname, or by \Term{sinful string}.  For more information see
	the \Condor{reschedule} man page.

	\item RegisterEvicted(cluster, sub) \\
	Register an eviction handler that will be called anytime a job from
	the specified cluster is evicted.  The eviction handler will be
	called with three arguments: cluster, proc, and ckpt.  The cluster
	and proc are the cluster number and process number of the job that
	was evicted.  The ckpt argument denotes whether the job was 
	checkpointed when it was evicted.

	\item RegisterNormalTerm( cluster, sub ) \\
	Register a normal termination handler that is called when a job exits
	normally.  The normal termination handler will be called with five
	arguments: cluster, proc, return\_val, output and error.  The cluster
	and proc are the cluster and process numbers of the exiting job, 
	their return\_val is the exit code of the job, and the output and error
	arguments are arrays containing the output and error produced by the
	job, one line per element.

	\item RegisterAbnormalTerm( cluster, sub ) \\
	Register an abnormal termination handler.  Works similar to 
	RegisterNormalTerm except that it is called when a job abnormally 
	exits (segmentation fault, bus error, ...) and the return\_val
	argument is replaced with two arguments: signal\_val and core.
	Signal\_val indicates the signal that the job died with and
	core indicates whether a core file was created and if so, what
	the full path to the core file is.

	\item RegisterExecute(cluster, sub) \\
	Register an execution handler that is called whenever a job starts
	running on a given host.  The handler is called with four arguments:
	cluster, proc, ip, and port.  Cluster and proc are the cluster and
	process numbers for the job, ip is the Internet address of the
	machine running the job, and port is the command port of the 
	\Condor{starter} supervising the job.

	\item RegisterSubmit(cluster, sub) \\
	Register a submit handler that is called whenever a job is submitted
	with the given cluster.  The handler is called with cluster, proc,
	ip, and port.  Cluster and proc are the cluster and
	process numbers for the job, ip is the Internet address of the
	machine running the job, and port is the command port of the 
	\Condor{schedd} responsible for the job.

	\item Monitor(cluster) \\
	Begin monitoring this cluster.  This process starts a sub process
	in order to monitor the child, so other actions may proceed in the
	main loop of the perl script.  However, handlers cannot rely on
	being able to communicate back to the main script by simply changing
	variables latter on.
	
	\item Wait() \\
	Wait until all monitors finish and exit.

	\item DebugOn() \\
	Turn debug messages on.  This may be useful if you don't understand
	what your script is doing.	
\end{enumerate}

\subsection{An Example}
The following is a simple example of using the condor perl module.
\begin{verbatim}
#!/usr/bin/perl
use Condor;

$CMD_FILE = 'mycmdfile.cmd';
$checkpoints = 0;
$evicts = 0;
$vacates = 0;

# A subroutine that will be used as the normal execution callback
$normal = sub
{
    ($cluster, $proc, $return_val, @output, @error) = @_;
    print "Job $cluster.$proc exited normally with status $return_val.\n";
    print "Job was vacated $vacates times and evicted $evicts times\n";
    print "Job checkpointed $checkpoints times\n";
    exit(0);
};	

$evicted = sub
{
    ($cluster, $proc, $ckpt) = @_;
    print "Job $cluster, $proc was evicted.\n";
    $checkpoints += $ckpt;
    $evicts++;
    &Condor::Reschedule();	
};

$execute = sub
{
    ($cluster, $proc, $ip, $port) = @_;
    print "Job running on $ip:$port, vacating...\n";
    &Condor::Vacate("\"<$ip:$port>\"");
    $vacates++;
};

$cluster = Condor::Submit($CMD_FILE);
&Condor::RegisterNormalTerm($cluster, $normal);
&Condor::RegisterEvicted($cluster, $evicted);
&Condor::RegisterExecute($cluster, $execute);
&Condor::Monitor($cluster);
&Condor::Wait();
\end{verbatim}

This example program will submit the command file 'mycmdfile.cmd' and attempt
to vacate any machine that the job runs on.  The eviction handler will
then keep track of how many times the job has checkpointed.  The termination
handler then prints out a summary of what has happened.

    

