The Condor perl module facilitates automatic submitting and monitoring of
condor jobs, along with automated administration of condor.  The most common
use of the perl module is the monitoring of condor jobs.  The condor perl
module uses the user log of a condor job for monitoring.

The Condor perl module is made up of several subroutines.  Many subroutines 
take other subroutines as arguments.  These subroutines are used as callbacks
which are called when interesting events happen.

\subsection{Subroutines}
\begin{enumerate}
	\item Submit(command\_file) \\
	The submit subroutine takes a command file name as an argument and
	submits it to condor.  The \Condor{submit} program should be in the
	path of the user.  If the user wishes to monitor the job with condor
	they must specify a log file in the command file.  The cluster
	submitted is returned. For more information
	see the \Condor{submit} man page.
	
	\item Vacate(machine) \\
	Vacate the machine specified.  The machine may be specified
	either by hostname, or by \Term{sinful string}.  For more information
	see the \Condor{vacate} man page.

	\item Reschedule(machine) \\
	Reschedule the machine specified.  The machine may be specified either
 	by hostname, or by \Term{sinful string}.  For more information see
	the \Condor{reschedule} man page.

	\item RegisterEvicted(sub) \\
	Register an eviction handler that will be called anytime a job from
	the specified cluster is evicted.  The eviction handler will be
	called with two arguments: cluster and job. The cluster
	and job are the cluster number and process number of the job that
	was evicted.
	
	\item RegisterEvictedWithCheckpoint(sub) \\
	Same as RegisterEvicted except that the handler is called when the 
	evicted job was checkpointed.

	\item RegisterEvictedWithoutCheckpoint(sub) \\
	Same as RegisterEvicted except that the handler is called when the
	evicted job was not checkpointed.

	\item RegisterExit(sub) \\
	Register a termination handler that is called when a job exits.
	The termination handler will be called with two arguments: cluster and
	job. The cluster and job are the cluster and process numbers of the
	existing job. 
	
	\item RegisterExitSuccess(sub) \\
	Register a termination handler that is called when a job exits without
	errors. The termination handler will be called with two arguments: 
	cluster and job  The cluster and job are the cluster and process
	numbers of the existing job. 

	\item RegisterExitFailure(sub) \\
	Register a termination handler that is called when a job exits with 
	errors. The termination handler will be called with three arguments:
	cluster, job and retval. The cluster and job are the cluster 
	and process numbers of the existing job and the retval is the exit
	code of the job.

	\item RegisterExitAbnormal(sub) \\
	Register an termination handler that is called when a job abnormally
	exits (segmentation fault, bus error, ...). The termination handler
	will be called with four arguments: cluster, job  signal and
	core. The cluster and job are the cluster and process numbers of 
	the existing job. The signal indicates the signal that the job
	died with and core indicates whether a core file was created and if 
	so, what the full path to the core file is.

	\item RegisterAbort(sub) \\
	Register a handler that is called when a job is aborted by a user.

	\item RegisterJobErr(sub) \\
	Register a handler that is called when a job is not executable.

	\item RegisterExecute(sub) \\
	Register an execution handler that is called whenever a job starts
	running on a given host.  The handler is called with four arguments:
	cluster, job  host, and sinful.  Cluster and job are the cluster and
	process numbers for the job, host is the Internet address of the
	machine running the job, and sinful is the Internet address and 
	command port of the \Condor{starter} supervising the job.

	\item RegisterSubmit(sub) \\
	Register a submit handler that is called whenever a job is submitted
	with the given cluster.  The handler is called with cluster, job 
	host, and sinful. Cluster and job are the cluster and
	process numbers for the job, host is the Internet address of the
	machine running the job, and sinful is the Internet address and
	command port of the \Condor{schedd} responsible for the job.

	\item Monitor(cluster) \\
	Begin monitoring this cluster. Returns when all jobs in cluster
	terminate.  \\
	
	\item Wait() \\
	Wait until all monitors finish and exit.

	\item DebugOn() \\
	Turn debug messages on.  This may be useful if you don't understand
	what your script is doing.	

	\item DebugOff() \\
	Turn debug messages off.

\end{enumerate}

\subsection{An Example}
The following is a simple example of using the condor perl module.
\begin{verbatim}
#!/usr/bin/perl
use Condor;

$CMD_FILE = 'mycmdfile.cmd';
$evicts = 0;
$vacates = 0;

# A subroutine that will be used as the normal execution callback
$normal = sub
{
    %parameters = @_;
    $cluster = $parameters{'cluster'};
    $job = $parameters{'job'};

    print "Job $cluster.$job exited normally without errors.\n";
    print "Job was vacated $vacates times and evicted $evicts times\n";
    exit(0);
};	

$evicted = sub
{
    %parameters = @_;
    $cluster = $parameters{'cluster'};
    $job = $parameters{'job'};

    print "Job $cluster, $job was evicted.\n";
    $evicts++;
    &Condor::Reschedule();	
};

$execute = sub
{
    %parameters = @_;
    $cluster = $parameters{'cluster'};
    $job = $parameters{'job'};
    $host = $parameters{'host'};
    $sinful = $parameters{'sinful'};

    print "Job running on $sinful, vacating...\n";
    &Condor::Vacate($sinful);
    $vacates++;
};

$cluster = Condor::Submit($CMD_FILE);
printf("Could not open. Access Denied\n");
			break;
&Condor::RegisterExitSuccess($normal);
&Condor::RegisterEvicted($evicted);
&Condor::RegisterExecute($execute);
&Condor::Monitor($cluster);
&Condor::Wait();
\end{verbatim}

This example program will submit the command file 'mycmdfile.cmd' and attempt
to vacate any machine that the job runs on. The termination
handler then prints out a summary of what has happened.

    

