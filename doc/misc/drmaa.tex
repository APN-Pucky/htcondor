%%%%%%%%%%%%%%%%%%%%%%%%%%%%%%%%%%%%
\subsection{\label{API-DRMAA} The DRMAA API}
%%%%%%%%%%%%%%%%%%%%%%%%%%%%%%%%%%%%
\index{DRMAA (Distributed Resource Management Application API)}
\index{API!DRMAA}
\index{Distributed Resource Management Application API (DRMAA)}

The following quote from the DRMAA Specification 1.0 abstract
nicely describes the purpose of the API:

The Distributed Resource Management Application API (DRMAA),
developed by a working group of the Global Grid Forum (GGF),
\begin{quote}
provides a generalized API to distributed resource management systems
(DRMSs) in order to facilitate integration of application programs.
The scope of DRMAA is limited to job submission,
job monitoring and control,
and the retrieval of the finished job status.
DRMAA provides application developers and
distributed resource management builders
with a programming model that enables
the development of distributed applications
tightly coupled to an underlying DRMS.
For deployers of such distributed applications,
DRMAA preserves flexibility and choice in system design.
\end{quote}

The API allows users who write programs using DRMAA functions
and link to a DRMAA library to submit,
control, and retrieve information about jobs to a Grid system.
The Condor implementation of a portion of the API
allows programs (applications) to use the library
functions provided to submit, monitor and control
Condor jobs.

See the DRMAA site 
(\URL{http://www.drmaa.org}) to find the
API specification for DRMA 1.0 for further details on the API.

%%%%%%%%%%%%%%%%%%%%%%%%%%%%%%%%%%%%
\subsubsection{\label{DRMAA-Implementation} Implementation Details}
%%%%%%%%%%%%%%%%%%%%%%%%%%%%%%%%%%%%


The library was developed from the DRMA API Specification 1.0 of January 2004
and the DRMAA C Bindings v0.9 of September 2003.
It is a static C library that expects a POSIX thread model
on Unix systems and a Windows thread model on Windows systems.
Unix systems that do not support POSIX threads
are not guaranteed thread safety when calling the library's functions.

The object library file is called \File{libcondordrmaa.a},
and it is located within
the \File{<release>/lib} directory in the Condor download.
Its header file  is called \File{lib\_condor\_drmaa.h}, and it is located within
the \File{<release>/include} directory in the Condor download.
Also within \File{<release>/include} is the file
\File{lib\_condor\_drmaa.README},
which gives further details on the implementation.

Use of the library requires that a
local \Condor{schedd} daemon  must be running,
and the program linked to the library must have
sufficient spool space.
This space should be in \File{/tmp}
or specified by the environment variables
\Env{TEMP}, \Env{TMP}, or \Env{SPOOL}.
The program linked to the library and the local \Condor{schedd} daemon
must have read, write, and traverse rights to the spool space.

The library currently supports the following specification-defined
job attributes:
\begin{description}
\item{DRMAA\_REMOTE\_COMMAND}
\item{DRMAA\_JS\_STATE}
\item{DRMAA\_NATIVE\_SPECIFICATION}
\item{DRMAA\_BLOCK\_EMAIL}
\item{DRMAA\_INPUT\_PATH}
\item{DRMAA\_OUTPUT\_PATH}
\item{DRMAA\_ERROR\_PATH}
\item{DRMAA\_V\_ARGV}
\item{DRMAA\_V\_ENV}
\item{DRMAA\_V\_EMAIL}
\end{description}

The attribute \Attr{DRMAA\_NATIVE\_SPECIFICATION} can be used
to direct all commands supported within
submit description files.  
See the \Condor{submit} manual page at
section~\ref{man-condor-submit} for a complete list.
Multiple commands can be specified if separated by newlines.  

As in the normal submit file,
arbitrary attributes can be added to the job's ClassAd
by prefixing the attribute with +.  In this case, you will need to put
string values in quotation marks, the same as in a submit file.

Thus to tell Condor that the job will likely use 64 megabytes of memory (65536
kilobytes), to more highly rank machines with more memory, and to add the
arbitrary attribute of department set to chemistry, you would set
Attr{DRMAA\_NATIVE\_SPECIFICATION} to the C string:

\begin{verbatim}
  drmaa_set_attribute(jobtemplate, DRMAA_NATIVE_SPECIFICATION,
      "image_size=65536\nrank=Memory\n+department=\"chemistry\"",
      err_buf, sizeof(err_buf)-1);

\end{verbatim}

