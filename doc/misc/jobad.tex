\begin{description}

\index{ClassAd!job attributes}

%%% ClassAd attribute: Args
\index{ClassAd job attribute!Args}
\index{Args!job ClassAd attribute}
\item[\AdAttr{Args}:]  String representing the arguments passed to the job.

%%% ClassAd attribute: CkptArch
\index{ClassAd job attribute!CkptArch}
\index{CkptArch!job ClassAd attribute}
\item[\AdAttr{CkptArch}:]  String describing the architecture of the machine
this job executed on at the time it last produced a checkpoint.
If the job has never produced a checkpoint,
this attribute is \Expr{undefined}.

%%% ClassAd attribute: CkptOpSys
\index{ClassAd job attribute!CkptOpSys}
\index{CkptOpSys!job ClassAd attribute}
\item[\AdAttr{CkptOpSys}:]  String describing the operating system of
the machine
this job executed on at the time it last produced a checkpoint.
If the job has never produced a checkpoint,
this attribute is \Expr{undefined}.

%%% ClassAd attribute: ClusterId
\index{ClassAd job attribute!ClusterId}
\index{ClusterId!job ClassAd attribute}
\index{cluster!definition}
\index{job ID!cluster identifier}
\item[\AdAttr{ClusterId}:]  Integer cluster identifier for this job.
A cluster is a group of jobs that were submitted together.  Each
job has its own unique job identifier within the cluster, but shares a
common cluster identifier.
The value changes each time a job or set of jobs are queued for
execution under Condor.

%%% ClassAd attribute: Cmd
\index{ClassAd job attribute!Cmd}
\index{Cmd!job ClassAd attribute}
\item[\AdAttr{Cmd}:]  The path to and the file name of the job to be executed.

%%% ClassAd attribute: CompletionDate
\index{ClassAd job attribute!CompletionDate}
\index{CompletionDate!job ClassAd attribute}
\item[\AdAttr{CompletionDate}:]  The time when the job completed,
or the value 0 if the job has not yet completed.
Measured in the
number of seconds since the epoch (00:00:00 UTC, Jan 1, 1970).

%%% ClassAd attribute: CumulativeSuspensionTime
\index{ClassAd job attribute!CumulativeSuspensionTime}
\index{CumulativeSuspensionTime!job ClassAd attribute}
\item[\AdAttr{CumulativeSuspensionTime}:]  A running total of the number of
seconds the job has spent in suspension for the life of the job.

%%% ClassAd attribute: CurrentHosts
\index{ClassAd job attribute!CurrentHosts}
\index{CurrentHosts!job ClassAd attribute}
\item[\AdAttr{CurrentHosts}:]  The number of hosts in the claimed state,
due to this job.

%%% ClassAd attribute: EnteredCurrentStatus
\index{ClassAd job attribute!EnteredCurrentStatus}
\index{EnteredCurrentStatus!job ClassAd attribute}
\item[\AdAttr{EnteredCurrentStatus}:]  An integer containing the
epoch time of when the job entered into its current status
So for example, if the job is on hold, the ClassAd expression
\begin{verbatim}
    CurrentTime - EnteredCurrentStatus
\end{verbatim}
will equal the number of seconds that the job has been on hold.

%%% ClassAd attribute: ExecutableSize
\index{ClassAd job attribute!ExecutableSize}
\index{ExecutableSize!job ClassAd attribute}
\item[\AdAttr{ExecutableSize}:]  Size of the executable in Kbytes.

%%% ClassAd attribute: ExitBySignal
\index{ClassAd job attribute!ExitBySignal}
\index{ExitBySignal!job ClassAd attribute}
\item[\AdAttr{ExitBySignal}:]  An attribute that is \Expr{True}
when a user job exits via a signal and \Expr{False} otherwise.
For some grid universe jobs, how the job exited is
unavailable. In this case, \AdAttr{ExitBySignal} is set to  \Expr{False}.

%%% ClassAd attribute: ExitCode
\index{ClassAd job attribute!ExitCode}
\index{ExitCode!job ClassAd attribute}
\item[\AdAttr{ExitCode}:]  When a user job exits by means other than a signal,
this is the exit return code of the user job.
For some grid universe jobs, how the job exited is
unavailable. In this case, \AdAttr{ExitCode} is set to  0.

%%% ClassAd attribute: ExitSignal
\index{ClassAd job attribute!ExitSignal}
\index{ExitSignal!job ClassAd attribute}
\item[\AdAttr{ExitSignal}:]  When a user job exits by means of an unhandled 
signal, this attribute takes on the numeric value of the signal.
For some grid universe jobs, how the job exited is
unavailable. In this case, \AdAttr{ExitSignal} will be undefined.

%% Karen added proper index entries up to this point in the file

%%% ClassAd attribute: ExitStatus
\index{ClassAd job attribute!ExitStatus}
\index{ExitStatus!job ClassAd attribute}
\item[\AdAttr{ExitStatus}:]  The way that Condor previously dealt with
a job's exit status.
This attribute should no longer be used.
It is not always accurate in
heterogeneous pools, or if the job exited with a signal.
Instead, see the attributes: \AdAttr{ExitBySignal},
\AdAttr{ExitCode}, and
\AdAttr{ExitSignal}.

%%% ClassAd attribute: HoldReasonCode
\index{ClassAd job attribute!HoldReasonCode}
\index{HoldReasonCode!job ClassAd attribute}
\item[\AdAttr{HoldReasonCode}:]    An integer value that represents the
reason that a job was put on hold.

\begin{center}
\begin{table}[hbt]
\begin{tabular}{|p{2cm}p{18cm}|} \hline
\emph{Integer Code} & \emph{Reason for Hold} \\ \hline \hline
1 & The user put the job on hold with \Condor{hold}.  \\ \hline
2 & Globus middleware reported an error.  
  \Attr{HoldReasonSubCode} is the GRAM error number. \\ \hline
3 & The \MacroNI{PERIODIC\_HOLD} expression evaluated to \Expr{True}.  \\ \hline
4 & The credentials for the job are invalid. \\ \hline
5 & A job policy expression evaluated to \Expr{Undefined}. \\ \hline
6 & The \Condor{starter} failed to start the executable.
  \Attr{HoldReasonSubCode} is the Unix  error number. \\ \hline
7 & The standard output file for the job could not be opened.
  \Attr{HoldReasonSubCode} is the Unix  error number. \\ \hline
8 & The standard input file for the job could not be opened.
  \Attr{HoldReasonSubCode} is the Unix  error number. \\ \hline
9 & The standard output stream for the job could not be opened.
  \Attr{HoldReasonSubCode} is the Unix  error number. \\ \hline
10 & The standard input stream for the job could not be opened.
  \Attr{HoldReasonSubCode} is the Unix  error number. \\ \hline
11 & An internal Condor protocol error was encountered when transferring files. \\ \hline
12 & The \Condor{starter} failed to download input files.
  \Attr{HoldReasonSubCode} is the Unix  error number. \\ \hline
13 & The \Condor{starter} failed to upload output files.
  \Attr{HoldReasonSubCode} is the Unix  error number. \\ \hline
14 & The initial working directory of the job cannot be accessed.
  \Attr{HoldReasonSubCode} is the Unix  error number. \\ \hline
\end{tabular}
\end{table}
\end{center}

%%% ClassAd attribute: HoldReasonSubCode
\index{ClassAd job attribute!HoldReasonSubCode}
\index{HoldReasonSubCode!job ClassAd attribute}
\item[\AdAttr{HoldReasonSubCode}:]    An integer value that represents further
information to go along with the \Attr{HoldReasonCode}, for
some values of \Attr{HoldReasonCode}.
See \Attr{HoldReasonCode} for the values.

%%% ClassAd attribute: HoldKillSig
\index{ClassAd job attribute!HoldKillSig}
\index{HoldKillSig!job ClassAd attribute}
\item[\AdAttr{HoldKillSig}:]    Currently only for scheduler and local
universe jobs,
a string containing a name of
a signal to be sent to the job if the job is put on hold.

%%% ClassAd attribute: HoldReason
\index{ClassAd job attribute!HoldReason}
\index{HoldReason!job ClassAd attribute}
\item[\AdAttr{HoldReason}:]    A string containing a human-readable
message about why a job is on hold.
This is the message that will be displayed in response to
the command \verb@condor_q -hold@.
It can be used to determine if a job should be released or not.

%%% ClassAd attribute: ImageSize
\index{ClassAd job attribute!ImageSize}
\index{ImageSize!job ClassAd attribute}
\item[\AdAttr{ImageSize}:]  Estimate of the memory image size of the
job in Kbytes.  The initial estimate may be specified in the job
submit file.  Otherwise, the initial value is equal to the size of the
executable.  When the job checkpoints, the \AdAttr{ImageSize}
attribute is set to the size of the checkpoint file (since the
checkpoint file contains the job's memory image).
A vanilla universe job's \AdAttr{ImageSize} is recomputed
internally every 15 seconds.

%%% ClassAd attribute: JobLeaseDuration
\index{ClassAd job attribute!JobLeaseDuration}
\index{JobLeaseDuration!job ClassAd attribute}
\item[\AdAttr{JobLeaseDuration}:]  The number of seconds set for
a job lease, the amount of time that a job may continue running
on a remote resource,
despite its submitting machine's lack of response.
See section~\ref{sec:Job-Lease} for details on job leases.

%%% ClassAd attribute: JobPrio
\index{ClassAd job attribute!JobPrio}
\index{JobPrio!job ClassAd attribute}
\item[\AdAttr{JobPrio}:]  Integer priority for this job, set by
\Condor{submit} or \Condor{prio}.  The default value is 0.  The higher
the number, the greater (better) the priority.

%%% ClassAd attribute: JobStartDate
\index{ClassAd job attribute!JobStartDate}
\index{JobStartDate!job ClassAd attribute}
\item[\AdAttr{JobStartDate}:]  Time at which the job first began
running.  Measured in the
number of seconds since the epoch (00:00:00 UTC, Jan 1, 1970).

%%% ClassAd attribute: JobStatus
\index{ClassAd job attribute!JobStatus}
\index{JobStatus!job ClassAd attribute}
\index{job!state}
\item[\AdAttr{JobStatus}:]  Integer which indicates the current
status of the job.
\begin{center}
\begin{table}[hbt]
\begin{tabular}{|p{2cm}p{10cm}|} \hline
\emph{Value} & \emph{Status} \\ \hline \hline
0 & Unexpanded (the job has never run) \\ \hline
1 & Idle \\ \hline
2 & Running \\ \hline
3 & Removed \\ \hline
4 & Completed \\ \hline
5 & Held \\ \hline
\end{tabular}
\end{table}
\end{center}

%%% ClassAd attribute: JobUniverse
\index{ClassAd job attribute!JobUniverse}
\index{JobUniverse!job ClassAd attribute}
\index{job!universe}
\index{universe!job attribute definitions}
\item[\AdAttr{JobUniverse}:]  Integer which indicates the job
universe.

\begin{center}
\begin{table}[hbt]
\begin{tabular}{|p{2cm}p{3cm}|} \hline
\emph{Value} & \emph{Universe} \\ \hline \hline
1 & standard \\ \hline
4 & PVM \\ \hline
5 & vanilla \\ \hline
7 & scheduler \\ \hline
8 & MPI \\ \hline
9 & grid \\ \hline
10 & java \\ \hline
11 & parallel \\ \hline
12 & local \\ \hline
13 & vm \\ \hline
\end{tabular}
\end{table}
\end{center}

%%% ClassAd attribute: LastCheckpointPlatform
\index{ClassAd job attribute!LastCheckpointPlatform}
\index{LastCheckpointPlatform!job ClassAd attribute}
\item[\AdAttr{LastCheckpointPlatform}:]  An opaque string which is the
\AdAttr{CheckpointPlatform} identifier from the last machine where this
standard universe job had successfully produced a checkpoint.

%%% ClassAd attribute: LastCkptServer
\index{ClassAd job attribute!LastCkptServer}
\index{LastCkptServer!job ClassAd attribute}
\item[\AdAttr{LastCkptServer}:]  Host name of the last checkpoint
server used by this job.  When a pool is using multiple checkpoint
servers, this tells the job where to find its checkpoint file.

%%% ClassAd attribute: LastCkptTime
\index{ClassAd job attribute!LastCkptTime}
\index{LastCkptTime!job ClassAd attribute}
\item[\AdAttr{LastCkptTime}:]  Time at which the job last performed a
successful checkpoint.  Measured in the number of seconds since the
epoch (00:00:00 UTC, Jan 1, 1970).

%%% ClassAd attribute: LastMatchTime
\index{ClassAd job attribute!LastMatchTime}
\index{LastMatchTime!job ClassAd attribute}
\item[\AdAttr{LastMatchTime}:]  An integer containing the epoch time
when the job was last successfully matched with a resource (gatekeeper) Ad.

%%% ClassAd attribute: LastRejMatchReason
\index{ClassAd job attribute!LastRejMatchReason}
\index{LastRejMatchReason!job ClassAd attribute}
\item[\AdAttr{LastRejMatchReason}:]   If, at any point in the past,
this job failed to match with a resource ad,
this attribute will contain a string with a
human-readable message about why the match failed.

%%% ClassAd attribute: LastRejMatchTime
\index{ClassAd job attribute!LastRejMatchTime}
\index{LastRejMatchTime!job ClassAd attribute}
\item[\AdAttr{LastRejMatchTime}:]   An integer containing the epoch
time when Condor-G last tried to find a match for the job,
but failed to do so.

%%% ClassAd attribute: LastSuspensionTime
\index{ClassAd job attribute!LastSuspensionTime}
\index{LastSuspensionTime!job ClassAd attribute}
\item[\AdAttr{LastSuspensionTime}:]  Time at which the job last performed a
successful suspension.  Measured in the number of seconds since the
epoch (00:00:00 UTC, Jan 1, 1970).

%%% ClassAd attribute: LastVacateTime
\index{ClassAd job attribute!LastVacateTime}
\index{LastVacateTime!job ClassAd attribute}
\item[\AdAttr{LastVacateTime}:]  Time at which the job was last
evicted from a remote workstation.  Measured in the number of seconds
since the epoch (00:00:00 UTC, Jan 1, 1970).

%%% ClassAd attribute: LocalSysCpu
\index{ClassAd job attribute!LocalSysCpu}
\index{LocalSysCpu!job ClassAd attribute}
\item[\AdAttr{LocalSysCpu}:]  An accumulated number of seconds of 
system CPU time that the job caused to the machine upon which
the job was submitted.

%%% ClassAd attribute: LocalUserCpu
\index{ClassAd job attribute!LocalUserCpu}
\index{LocalUserCpu!job ClassAd attribute}
\item[\AdAttr{LocalUserCpu}:]  An accumulated number of seconds of 
user CPU time that the job caused to the machine upon which
the job was submitted.

%%% ClassAd attribute: MaxHosts
\index{ClassAd job attribute!MaxHosts}
\index{MaxHosts!job ClassAd attribute}
\item[\AdAttr{MaxHosts}:]  The maximum number of hosts that this job would
like to claim. As long as \AdAttr{CurrentHosts} is the same as
\AdAttr{MaxHosts}, no more hosts are negotiated for.

%%% ClassAd attribute: MaxJobRetirementTime
\index{ClassAd job attribute!MaxJobRetirementTime}
\index{MaxJobRetirementTime!job ClassAd attribute}
\item[\AdAttr{MaxJobRetirementTime}:]  Maximum time in seconds to let this
job run uninterrupted before kicking it off when it is being preempted.
This can only decrease the amount of time from what the corresponding
startd expression allows.

%%% ClassAd attribute: MinHosts
\index{ClassAd job attribute!MinHosts}
\index{MinHosts!job ClassAd attribute}
\item[\AdAttr{MinHosts}:]  The minimum number of hosts that must be in
the claimed state for this job, before the job may enter the running state.

%%% ClassAd attribute: NiceUser
\index{ClassAd job attribute!NiceUser}
\index{NiceUser!job ClassAd attribute}
\item[\AdAttr{NiceUser}:]  Boolean value which indicates whether
this is a nice-user job.

%%% ClassAd attribute: NumCkpts
\index{ClassAd job attribute!NumCkpts}
\index{NumCkpts!job ClassAd attribute}
\item[\AdAttr{NumCkpts}:]  A count of the number of checkpoints
written by this job during its lifetime.

%%% ClassAd attribute: NumGlobusSubmits
\index{ClassAd job attribute!NumGlobusSubmits}
\index{NumGlobusSubmits!job ClassAd attribute}
\item[\AdAttr{NumGlobusSubmits}:]   An integer that is incremented each
time the \Condor{gridmanager} receives confirmation of a successful job
submission into Globus.

%%% ClassAd attribute: NumJobMatches
\index{ClassAd job attribute!NumJobMatches}
\index{NumJobMatches!job ClassAd attribute}
\item[\AdAttr{NumJobMatches}:]  An integer that is incremented by the
\Condor{schedd} each time the job is matched with a resource ad by the
negotiator.

%%% ClassAd attribute: NumRestarts
\index{ClassAd job attribute!NumRestarts}
\index{NumRestarts!job ClassAd attribute}
\item[\AdAttr{NumRestarts}:]  A count of the number of restarts from a
checkpoint attempted by this job during its lifetime.

%%% ClassAd attribute: NumSystemHolds
\index{ClassAd job attribute!NumSystemHolds}
\index{NumSystemHolds!job ClassAd attribute}
\item[\AdAttr{NumSystemHolds}:]   An integer that is incremented each time
Condor-G places a job on hold due to some sort of error condition.  This
counter is useful, since Condor-G will always place a job on hold when it
gives up on some error condition.  Note that if the user places the job
on hold using the \Condor{hold} command, this attribute is not incremented.

%%% ClassAd attribute: Owner
\index{ClassAd job attribute!Owner}
\index{Owner!job ClassAd attribute}
\item[\AdAttr{Owner}:]  String describing the user who submitted this
job.

%%% ClassAd attribute: ProcId
\index{ClassAd job attribute!ProcId}
\index{ProcId!job ClassAd attribute}
\index{process!definition for a submitted job}
\index{job ID!process identifier}
\item[\AdAttr{ProcId}:]  Integer process identifier for this job.
Within a cluster of many jobs,
each job has the same \Attr{ClusterId}, but will have a unique \Attr{ProcId}.
Within a cluster, assignment of a \Attr{ProcId} value will start
with the value 0.
The job (process) identifier described here is unrelated to operating
system PIDs.


%%% ClassAd attribute: QDate
\index{ClassAd job attribute!QDate}
\index{QDate!job ClassAd attribute}
\item[\AdAttr{QDate}:]  Time at which the job was submitted to the job
queue.  Measured in the
number of seconds since the epoch (00:00:00 UTC, Jan 1, 1970).

%%% ClassAd attribute: ReleaseReason
\index{ClassAd job attribute!ReleaseReason}
\index{ReleaseReason!job ClassAd attribute}
\item[\AdAttr{ReleaseReason}:]     A string containing a human-readable
message about why the job was released from hold.

%%% ClassAd attribute: RemoteIwd
\index{ClassAd job attribute!RemoteIwd}
\index{RemoteIwd!job ClassAd attribute}
\item[\AdAttr{RemoteIwd}:]  The path to the directory in which
a job is to be executed on a remote machine.

%%% ClassAd attribute: RemoteSysCpu
\index{ClassAd job attribute!RemoteSysCpu}
\index{RemoteSysCpu!job ClassAd attribute}
\item[\AdAttr{RemoteSysCpu}:]  The total number of seconds
of system CPU time (the time spent at system calls) the job used
on remote machines.

%%% ClassAd attribute: RemoteUserCpu
\index{ClassAd job attribute!RemoteUserCpu}
\index{RemoteUserCpu!job ClassAd attribute}
\item[\AdAttr{RemoteUserCpu}:]  The total number of seconds
of user CPU time the job used on remote machines.

%%% ClassAd attribute: RemoteWallClockTime
\index{ClassAd job attribute!RemoteWallClockTime}
\index{RemoteWallClockTime!job ClassAd attribute}
\item[\AdAttr{RemoteWallClockTime}:]  Cumulative number of seconds
the job has been allocated a machine.
This also includes time spent in suspension (if any),
so the total real time spent running is 
\begin{verbatim}
RemoteWallClockTime - CumulativeSuspensionTime
\end{verbatim}
Note that this number does not get reset to
zero when a job is forced to migrate from one machine to another.

%%% ClassAd attribute: RemoveKillSig
\index{ClassAd job attribute!RemoveKillSig}
\index{RemoveKillSig!job ClassAd attribute}
\item[\AdAttr{RemoveKillSig}:]    Currently only for scheduler universe jobs,
a string containing a name of
a signal to be sent to the job if the job is removed.

%%% ClassAd attribute: StreamErr
\index{ClassAd job attribute!StreamErr}
\index{StreamErr!job ClassAd attribute}
\item[\AdAttr{StreamErr}:]   
An attribute utilized only for grid universe jobs.
The default value is \Expr{True}.
If \Expr{True}, and \Attr{TransferErr} is \Expr{True}, then 
standard error is streamed back to the submit machine, instead
of doing the transfer (as a whole) after the job completes.
If \Expr{False}, then
standard error is transfered back to the submit machine
(as a whole) after the job completes.
If \Attr{TransferErr} is \Expr{False}, then this job attribute is ignored.

%%% ClassAd attribute: StreamOut
\index{ClassAd job attribute!StreamOut}
\index{StreamOut!job ClassAd attribute}
\item[\AdAttr{StreamOut}:]   
An attribute utilized only for grid universe jobs.
The default value is \Expr{True}.
If \Expr{True}, and \Attr{TransferOut} is \Expr{True}, then 
job output is streamed back to the submit machine, instead
of doing the transfer (as a whole) after the job completes.
If \Expr{False}, then
job output is transferred back to the submit machine
(as a whole) after the job completes.
If \Attr{TransferOut} is \Expr{False}, then this job attribute is ignored.

%%% ClassAd attribute: TotalSuspensions
\index{ClassAd job attribute!TotalSuspensions}
\index{TotalSuspensions!job ClassAd attribute}
\item[\AdAttr{TotalSuspensions}:]  A count of the number of times this job
has been suspended during its lifetime.

%%% ClassAd attribute: TransferErr
\index{ClassAd job attribute!TransferErr}
\index{TransferErr!job ClassAd attribute}
\item[\AdAttr{TransferErr}:]   
An attribute utilized only for grid universe jobs.
The default value is \Expr{True}.
If \Expr{True}, then the error output from the job
is transferred from the remote machine back to the submit machine.
The name of the file after transfer is the file referred to
by job attribute \Attr{Err}.
If \Expr{False}, no transfer takes place (remote to submit machine),
and the name of the file is the file referred to
by job attribute \Attr{Err}.

%%% ClassAd attribute: TransferExecutable
\index{ClassAd job attribute!TransferExecutable}
\index{TransferExecutable!job ClassAd attribute}
\item[\AdAttr{TransferExecutable}:]   
An attribute utilized only for grid universe jobs.
The default value is \Expr{True}.
If \Expr{True}, then the job executable is transferred from the submit
machine to the remote machine.
The name of the file (on the submit machine)
that is transferred is given by the
job attribute \Attr{Cmd}.
If \Expr{False}, no transfer takes place, and
the name of the file used (on the remote machine) will be as
given in the job attribute \Attr{Cmd}.

%%% ClassAd attribute: TransferIn
\index{ClassAd job attribute!TransferIn}
\index{TransferIn!job ClassAd attribute}
\item[\AdAttr{TransferIn}:]   
An attribute utilized only for grid universe jobs.
The default value is \Expr{True}.
If \Expr{True}, then the job input is transferred from the submit
machine to the remote machine.
The name of the file that is transferred is given by the
job attribute \Attr{In}.
If \Expr{False}, then the job's input is taken from a file on the
remote machine (pre-staged), and 
the name of the file is given by the job attribute \Attr{In}.

%%% ClassAd attribute: TransferOut
\index{ClassAd job attribute!TransferOut}
\index{TransferOut!job ClassAd attribute}
\item[\AdAttr{TransferOut}:]   
An attribute utilized only for grid universe jobs.
The default value is \Expr{True}.
If \Expr{True}, then the output from the job
is transferred from the remote machine back to the submit machine.
The name of the file after transfer is the file referred to
by job attribute \Attr{Out}.
If \Expr{False}, no transfer takes place (remote to submit machine),
and the name of the file is the file referred to
by job attribute \Attr{Out}.

\end{description}
