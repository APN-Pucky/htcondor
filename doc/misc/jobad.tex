\begin{description}

\index{ClassAd!job attributes}

%%% ClassAd attribute: Absent
\index{ClassAd job attribute!Absent}
\index{Absent!job ClassAd attribute}
\item[\AdAttr{Absent}:] Boolean set to true \Expr{True} if the ad is absent.

%%% ClassAd attribute: AcctGroup
\index{ClassAd job attribute!AcctGroup}
\index{AcctGroup!job ClassAd attribute}
\item[\AdAttr{AcctGroup}:] The accounting group name, as set in the
submit description file via the \SubmitCmd{accounting\_group} command.
This attribute is only present if an accounting group was requested by the
submission. See section~\ref{sec:group-accounting} for more 
information about accounting groups.

%%% ClassAd attribute: AcctGroupUser
\index{ClassAd job attribute!AcctGroupUser}
\index{AcctGroupUser!job ClassAd attribute}
\item[\AdAttr{AcctGroupUser}:] The user name associated with the
accounting group.  This attribute is only present if an
accounting group was requested by the submission.

%%% ClassAd attribute: AllRemoteHosts
\index{ClassAd job attribute!AllRemoteHosts}
\index{AllRemoteHosts!job ClassAd attribute}
\item[\AdAttr{AllRemoteHosts}:]  String containing a comma-separated list
of all the remote machines running a parallel or mpi universe job.

%%% ClassAd attribute: Args
\index{ClassAd job attribute!Args}
\index{Args!job ClassAd attribute}
\item[\AdAttr{Args}:]  A string representing the command line arguments 
passed to the job, when those arguments are specified using the
\emph{old} syntax, as specified in section~\ref{man-condor-submit-arguments}.

%%% ClassAd attribute: Arguments
\index{ClassAd job attribute!Arguments}
\index{Arguments!job ClassAd attribute}
\item[\AdAttr{Arguments}:]  A string representing the command line arguments 
passed to the job, when those arguments are specified using the
\emph{new} syntax, as specified in section~\ref{man-condor-submit-arguments}.

%%% ClassAd attribute: BatchQueue
\index{ClassAd job attribute!BatchQueue}
\index{BatchQueue!job ClassAd attribute}
\item[\AdAttr{BatchQueue}:]  For grid universe jobs destined for 
PBS, LSF or SGE, the name of the queue in the remote batch system.

%%% ClassAd attribute: BlockReadKbytes
\index{ClassAd job attribute!BlockReadKbytes}
\index{BlockReadKbytes!job ClassAd attribute}
\item[\AdAttr{BlockReadKbytes}:] The integer number of KiB
read from disk for this job.

%%% ClassAd attribute: BlockReads
\index{ClassAd job attribute!BlockReads}
\index{BlockReads!job ClassAd attribute}
\item[\AdAttr{BlockReads}:] The integer number of disk blocks
read for this job.

%%% ClassAd attribute: BlockWriteKbytes
\index{ClassAd job attribute!BlockWriteKbytes}
\index{BlockWriteKbytes!job ClassAd attribute}
\item[\AdAttr{BlockWriteKbytes}:] The integer number of KiB
written to disk for this job.

%%% ClassAd attribute: BlockWrites
\index{ClassAd job attribute!BlockWrites}
\index{BlockWrites!job ClassAd attribute}
\item[\AdAttr{BlockWrites}:] The integer number of blocks
written to disk for this job.

%%% ClassAd attribute: BoincAuthenticatorFile
\index{ClassAd job attribute!BoincAuthenticatorFile}
\index{BoincAuthenticatorFile!job ClassAd attribute}
\item[\AdAttr{BoincAuthenticatorFile}:] 
Used for grid type boinc jobs;
a string taken from the definition of the submit description file command
\SubmitCmd{boinc\_authenticator\_file}.
Defines the path and file name of the file containing the authenticator
string to use to authenticate to the BOINC service.

%%% ClassAd attribute: CkptArch
\index{ClassAd job attribute!CkptArch}
\index{CkptArch!job ClassAd attribute}
\item[\AdAttr{CkptArch}:]  String describing the architecture of the machine
this job executed on at the time it last produced a checkpoint.
If the job has never produced a checkpoint,
this attribute is \Expr{undefined}.

%%% ClassAd attribute: CkptOpSys
\index{ClassAd job attribute!CkptOpSys}
\index{CkptOpSys!job ClassAd attribute}
\item[\AdAttr{CkptOpSys}:]  String describing the operating system of
the machine
this job executed on at the time it last produced a checkpoint.
If the job has never produced a checkpoint,
this attribute is \Expr{undefined}.

%%% ClassAd attribute: ClusterId
\index{ClassAd job attribute!ClusterId}
\index{ClusterId!job ClassAd attribute}
\index{cluster!definition}
\index{job ID!cluster identifier}
\item[\AdAttr{ClusterId}:]  Integer cluster identifier for this job.
A cluster is a group of jobs that were submitted together.  Each
job has its own unique job identifier within the cluster, but shares a
common cluster identifier.
The value changes each time a job or set of jobs are queued for
execution under HTCondor.

%%% ClassAd attribute: Cmd
\index{ClassAd job attribute!Cmd}
\index{Cmd!job ClassAd attribute}
\item[\AdAttr{Cmd}:]  The path to and the file name of the job to be executed.

%%% ClassAd attribute: CommittedTime
\index{ClassAd job attribute!CommittedTime}
\index{CommittedTime!job ClassAd attribute}
\label{CommittedTime}
\item[\AdAttr{CommittedTime}:] The number of seconds of wall clock time
that the job has been allocated a machine,
excluding the time spent on run attempts that
were evicted without a checkpoint.  
Like \AdAttr{RemoteWallClockTime},
this includes time the job spent in a suspended state,
so the total committed wall time spent running is 
\begin{verbatim}
CommittedTime - CommittedSuspensionTime
\end{verbatim}

%%% ClassAd attribute: CommittedSlotTime
\index{ClassAd job attribute!CommittedSlotTime}
\index{CommittedSlotTime!job ClassAd attribute}
\label{CommittedSlotTime}
\item[\AdAttr{CommittedSlotTime}:] This attribute is identical to
\AdAttr{CommittedTime} except that the time is multiplied by the
\AdAttr{SlotWeight} of the machine(s) that ran the job.  This relies
on \AdAttr{SlotWeight} being listed in \Macro{SYSTEM\_JOB\_MACHINE\_ATTRS}.

%%% ClassAd attribute: CommittedSuspensionTime
\index{ClassAd job attribute!CommittedSuspensionTime}
\index{CommittedSuspensionTime!job ClassAd attribute}
\item[\AdAttr{CommittedSuspensionTime}:]  A running total of the number of
seconds the job has spent in suspension during time in which the job was
not evicted without a checkpoint.  This number is updated when the job is
checkpointed and when it exits.

%%% ClassAd attribute: CompletionDate
\index{ClassAd job attribute!CompletionDate}
\index{CompletionDate!job ClassAd attribute}
\item[\AdAttr{CompletionDate}:]  The time when the job completed,
or the value 0 if the job has not yet completed.
Measured in the
number of seconds since the epoch (00:00:00 UTC, Jan 1, 1970).

%%% ClassAd attribute: ConcurrencyLimits
\index{ClassAd job attribute!ConcurrencyLimits}
\index{ConcurrencyLimits!job ClassAd attribute}
\item[\AdAttr{ConcurrencyLimits}:]  A string list,
delimited by commas and space characters.
The items in the list
identify named resources that the job requires.
The value can be a ClassAd expression which, when evaluated in the context
of the job ClassAd and a matching machine ClassAd, results in a string list.

%%% ClassAd attribute: CumulativeSlotTime
\index{ClassAd job attribute!CumulativeSlotTime}
\index{CumulativeSlotTime!job ClassAd attribute}
\label{CumulativeSlotTime}
\item[\AdAttr{CumulativeSlotTime}:] This attribute is identical to
\AdAttr{RemoteWallClockTime} except that the time is multiplied by the
\AdAttr{SlotWeight} of the machine(s) that ran the job.  This relies
on \AdAttr{SlotWeight} being listed in \Macro{SYSTEM\_JOB\_MACHINE\_ATTRS}.

%%% ClassAd attribute: CumulativeSuspensionTime
\index{ClassAd job attribute!CumulativeSuspensionTime}
\index{CumulativeSuspensionTime!job ClassAd attribute}
\item[\AdAttr{CumulativeSuspensionTime}:]  A running total of the number of
seconds the job has spent in suspension for the life of the job.

%%% ClassAd attribute: CumulativeTransferTime
\index{ClassAd job attribute!CumulativeTransferTime}
\index{CumulativeTransferTime!job ClassAd attribute}
\label{CumulativeTransferTime}
\item[\AdAttr{CumulativeTransferTime}:] The total time, in seconds, that 
condor has spent transferring the input and output sandboxes for the life of the job.

%%% ClassAd attribute: CurrentHosts
\index{ClassAd job attribute!CurrentHosts}
\index{CurrentHosts!job ClassAd attribute}
\item[\AdAttr{CurrentHosts}:]  The number of hosts in the claimed state,
due to this job.

%%% ClassAd attribute: DAGManJobId
\index{ClassAd job attribute!DAGManJobId}
\index{DAGManJobId!job ClassAd attribute}
\item[\AdAttr{DAGManJobId}:] For a DAGMan node job only,
the \Attr{ClusterId} job ClassAd attribute
of the \Condor{dagman} job which is the parent of this node job.
For nested DAGs, this attribute holds only the \Attr{ClusterId} of
the job's immediate parent.

%%% ClassAd attribute: DAGParentNodeNames
\index{ClassAd job attribute!DAGParentNodeNames}
\index{DAGParentNodeNames!job ClassAd attribute}
\item[\AdAttr{DAGParentNodeNames}:] For a DAGMan node job only,
a comma separated list of each \Arg{JobName} which is a parent node of
this job's node.
This attribute is passed through to the job via the \Condor{submit}
command line, if it does not exceed the line length defined with
\Code{\_POSIX\_ARG\_MAX}. For example, if a node job has two parents
with \Arg{JobName}s B and C, the \Condor{submit} command line will 
contain
\begin{verbatim}
  -append +DAGParentNodeNames=B,C
\end{verbatim}

%%% ClassAd attribute: DAGManNodesLog
\index{ClassAd job attribute!DAGManNodesLog}
\index{DAGManNodesLog!job ClassAd attribute}
\item[\AdAttr{DAGManNodesLog}:] For a DAGMan node job only, gives the path to
an event log used exclusively by DAGMan to monitor the state of the DAG's jobs.
Events are written to this log file in addition to any log file
specified in the job's submit description file.

%%% ClassAd attribute: DAGManNodesLogMask
\index{ClassAd job attribute!DAGManNodesMask}
\index{DAGManNodesMask!job ClassAd attribute}
\item[\AdAttr{DAGManNodesMask}:] For a DAGMan node job only, 
a comma-separated list of the event codes that should be written
to the log specified by \Attr{DAGManNodesLog}, 
known as the auxiliary log.
All events not specified in the
\Attr{DAGManNodesMask} string are not written to the auxiliary event log.
The value of this attribute is determined
by DAGMan, and it is passed to the job via the \Condor{submit} command line.
By default, the following events are written to the
auxiliary job log:  
\begin{itemize}
 \item \Expr{Submit}, event code is 0
 \item \Expr{Execute}, event code is 1
 \item \Expr{Executable error}, event code is 2
 \item \Expr{Job evicted}, event code is 4
 \item \Expr{Job terminated}, event code is 5
 \item \Expr{Shadow exception}, event code is 7
 \item \Expr{Job aborted}, event code is 9
 \item \Expr{Job suspended}, event code is 10
 \item \Expr{Job unsuspended}, event code is 11
 \item \Expr{Job held}, event code is 12
 \item \Expr{Job released}, event code is 13
 \item \Expr{Post script terminated}, event code is 16
 \item \Expr{Globus submit}, event code is 17
 \item \Expr{Grid submit}, event code is 27
\end{itemize}
If \Attr{DAGManNodesLog} is
not defined, it has no effect. The value of \Attr{DAGManNodesMask} does not
affect events recorded in the job event log file referred to by \Attr{UserLog}.

%%% ClassAd attribute: DelegateJobGSICredentialsLifetime
\index{ClassAd job attribute!DelegateJobGSICredentialsLifetime}
\index{DelegateJobGSICredentialsLifetime!job ClassAd attribute}
\item[\AdAttr{DelegateJobGSICredentialsLifetime}:]   
An integer that specifies the maximum number of seconds for which
delegated proxies should be valid.  The default behavior is determined
by the configuration
setting \Macro{DELEGATE\_JOB\_GSI\_CREDENTIALS\_LIFETIME}, which
defaults to one day.  A value of 0 indicates that the delegated proxy
should be valid for as long as allowed by the credential used to
create the proxy.  This setting currently only applies to proxies
delegated for non-grid jobs and HTCondor-C jobs.  It does not currently
apply to globus grid jobs, which always behave as though this setting
were 0.  This setting has no effect if the configuration
setting \Macro{DELEGATE\_JOB\_GSI\_CREDENTIALS} is false, because in
that case the job proxy is copied rather than delegated.

%%% ClassAd attribute: DiskUsage
\index{ClassAd job attribute!DiskUsage}
\index{DiskUsage!job ClassAd attribute}
\item[\AdAttr{DiskUsage}:] Amount of disk space (KiB) in the HTCondor
execute directory on the execute machine that this job has used.
An initial value may be set at the job's request, placing into the
job's submit description file a setting such as
\begin{verbatim}
  # 1 megabyte initial value
  +DiskUsage = 1024
\end{verbatim}
\SubmitCmdNI{vm} universe jobs will default to an initial value of the disk
image size. 
If not initialized by the job,
non-\SubmitCmdNI{vm} universe jobs will default to an initial value of the 
sum of the job's executable and all input files.

%%% ClassAd attribute: EC2AccessKeyId
\index{ClassAd job attribute!EC2AccessKeyId}
\index{EC2AccessKeyId!job ClassAd attribute}
\item[\AdAttr{EC2AccessKeyId}:] 
Used for grid type ec2 jobs;
a string taken from the definition of the submit description file command
\SubmitCmd{ec2\_access\_key\_id}.
Defines the path and file name of the file containing the EC2 Query API's
access key.

%%% ClassAd attribute: EC2AmiID
\index{ClassAd job attribute!EC2AmiID}
\index{EC2AmiID!job ClassAd attribute}
\item[\AdAttr{EC2AmiID}:] 
Used for grid type ec2 jobs;
a string taken from the definition of the submit description file command
\SubmitCmd{ec2\_ami\_id}.
Identifies the machine image of the instance.

%%% ClassAd attribute: EC2BlockDeviceMapping
\index{ClassAd job attribute!EC2BlockDeviceMapping}
\index{EC2BlockDeviceMapping!job ClassAd attribute}
\item[\AdAttr{EC2BlockDeviceMapping}:] 
Used for grid type ec2 jobs;
a string taken from the definition of the submit description file command
\SubmitCmd{ec2\_block\_device\_mapping}.
Defines the map from block device names to kernel device names
for the instance.

%%% ClassAd attribute: EC2ElasticIp
\index{ClassAd job attribute!EC2ElasticIp}
\index{EC2ElasticIp!job ClassAd attribute}
\item[\AdAttr{EC2ElasticIp}:] 
Used for grid type ec2 jobs;
a string taken from the definition of the submit description file command
\SubmitCmd{ec2\_elastic\_ip}.
Specifies an Elastic IP address to associate with the instance.

%%% ClassAd attribute: EC2IamProfileArn
\index{ClassAd job attribute!EC2IamProfileArn}
\index{EC2IamProfileArn!job ClassAd attribute}
\item[\AdAttr{EC2IamProfileArn}:]
Used for grid type ec2 jobs;
a string taken from the definition of the submit description file command
\SubmitCmd{ec2\_iam\_profile\_arn}.
Specifies the IAM (instance) profile to associate with this instance.

%%% ClassAd attribute: EC2IamProfileName
\index{ClassAd job attribute!EC2IamProfileName}
\index{EC2IamProfileName!job ClassAd attribute}
\item[\AdAttr{EC2IamProfileName}:]
Used for grid type ec2 jobs;
a string taken from the definition of the submit description file command
\SubmitCmd{ec2\_iam\_profile\_name}.
Specifies the IAM (instance) profile to associate with this instance.

%%% ClassAd attribute: EC2InstanceName
\index{ClassAd job attribute!EC2InstanceName}
\index{EC2InstanceName!job ClassAd attribute}
\item[\AdAttr{EC2InstanceName}:] 
Used for grid type ec2 jobs;
a string set for the job once the instance starts running, 
as assigned by the EC2 service, 
that represents the unique ID assigned to the instance by the EC2 service.


%%% ClassAd attribute: EC2InstanceName
\index{ClassAd job attribute!EC2InstanceName}
\index{EC2InstanceName!job ClassAd attribute}
\item[\AdAttr{EC2InstanceName}:] 
Used for grid type ec2 jobs;
a string set for the job once the instance starts running, 
as assigned by the EC2 service, 
that represents the unique ID assigned to the instance by the EC2 service.

%%% ClassAd attribute: EC2InstanceType
\index{ClassAd job attribute!EC2InstanceType}
\index{EC2InstanceType!job ClassAd attribute}
\item[\AdAttr{EC2InstanceType}:] 
Used for grid type ec2 jobs;
a string taken from the definition of the submit description file command
\SubmitCmd{ec2\_instance\_type}.
Specifies a service-specific instance type.

%%% ClassAd attribute: EC2KeyPair
\index{ClassAd job attribute!EC2KeyPair}
\index{EC2KeyPair!job ClassAd attribute}
\item[\AdAttr{EC2KeyPair}:] 
Used for grid type ec2 jobs;
a string taken from the definition of the submit description file command
\SubmitCmd{ec2\_keypair}.
Defines the key pair associated with the EC2 instance. 

%%% ClassAd attribute: EC2ParameterNames
\index{ClassAd job attribute!EC2ParameterNames}
\index{EC2ParameterNames!job ClassAd attribute}
\item[\AdAttr{EC2ParameterNames}:]
Used for grid type ec2 jobs;
a string taken from the definition of the submit description file command
\SubmitCmd{ec2\_parameter\_names}.
Contains a space or comma separated list of the names of additional
parameters to pass when instantiating an instance.

%%% ClassAd attribute: EC2SpotPrice
\index{ClassAd job attribute!EC2SpotPrice}
\index{EC2SpotPrice!job ClassAd attribute}
\item[\AdAttr{EC2SpotPrice}:] 
Used for grid type ec2 jobs;
a string taken from the definition of the submit description file command
\SubmitCmd{ec2\_spot\_price}.
Defines the maximum amount per hour a job submitter is willing to 
pay to run this job.

%%% ClassAd attribute: EC2SpotRequestID
\index{ClassAd job attribute!EC2SpotRequestID}
\index{EC2SpotRequestID!job ClassAd attribute}
\item[\AdAttr{EC2SpotRequestID}:] 
Used for grid type ec2 jobs;
identifies the spot request HTCondor made on behalf of this job.

%%% ClassAd attribute: EC2StatusReasonCode
\index{ClassAd job attribute!EC2StatusReasonCode}
\index{EC2StatusReasonCode!job ClassAd attribute}
\item[\AdAttr{EC2StatusReasonCode}:] 
Used for grid type ec2 jobs;
reports the reason for the most recent EC2-level state transition.
Can be used to determine if a spot request was terminated
due to a rise in the spot price.

%%% ClassAd attribute: EC2TagNames
\index{ClassAd job attribute!EC2TagNames}
\index{EC2TagNames!job ClassAd attribute}
\item[\AdAttr{EC2TagNames}:] 
Used for grid type ec2 jobs;
a string taken from the definition of the submit description file command
\SubmitCmd{ec2\_tag\_names}.
Defines the set, and case, of tags associated with the EC2 instance. 

%%% ClassAd attribute: EC2KeyPairFile
\index{ClassAd job attribute!EC2KeyPairFile}
\index{EC2KeyPairFile!job ClassAd attribute}
\item[\AdAttr{EC2KeyPairFile}:] 
Used for grid type ec2 jobs;
a string taken from the definition of the submit description file command
\SubmitCmd{ec2\_keypair\_file}.
Defines the path and file name of the file 
into which to write the SSH key used to access the image, once it is running. 

%%% ClassAd attribute: EC2RemoteVirtualMachineName
\index{ClassAd job attribute!EC2RemoteVirtualMachineName}
\index{EC2RemoteVirtualMachineName!job ClassAd attribute}
\item[\AdAttr{EC2RemoteVirtualMachineName}:] 
Used for grid type ec2 jobs;
a string set for the job once the instance starts running, 
as assigned by the EC2 service, that represents
the host name upon which the instance runs, such that the
user can communicate with the running instance.

%%% ClassAd attribute: EC2SecretAccessKey
\index{ClassAd job attribute!EC2SecretAccessKey}
\index{EC2SecretAccessKey!job ClassAd attribute}
\item[\AdAttr{EC2SecretAccessKey}:] 
Used for grid type ec2 jobs;
a string taken from the definition of the submit description file command
\SubmitCmd{ec2\_secret\_access\_key}.
Defines that path and file name of the file 
containing the EC2 Query API's secret access key.

%%% ClassAd attribute: EC2SecurityGroups
\index{ClassAd job attribute!EC2SecurityGroups}
\index{EC2SecurityGroups!job ClassAd attribute}
\item[\AdAttr{EC2SecurityGroups}:] 
Used for grid type ec2 jobs;
a string taken from the definition of the submit description file command
\SubmitCmd{ec2\_security\_groups}.
Defines the list of EC2 security groups which should be associated with the job.

%%% ClassAd attribute: EC2SecurityIDs
\index{ClassAd job attribute!EC2SecurityIDs}
\index{EC2SecurityIDs!job ClassAd attribute}
\item[\AdAttr{EC2SecurityIDs}:] 
Used for grid type ec2 jobs;
a string taken from the definition of the submit description file command
\SubmitCmd{ec2\_security\_ids}.
Defines the list of EC2 security group IDs which should be associated with the job.

%%% ClassAd attribute: EC2UserData
\index{ClassAd job attribute!EC2UserData}
\index{EC2UserData!job ClassAd attribute}
\item[\AdAttr{EC2UserData}:] 
Used for grid type ec2 jobs;
a string taken from the definition of the submit description file command
\SubmitCmd{ec2\_user\_data}.
Defines a block of data that can be accessed by the virtual machine.

%%% ClassAd attribute: EC2UserDataFile
\index{ClassAd job attribute!EC2UserDataFile}
\index{EC2UserDataFile!job ClassAd attribute}
\item[\AdAttr{EC2UserDataFile}:] 
Used for grid type ec2 jobs;
a string taken from the definition of the submit description file command
\SubmitCmd{ec2\_user\_data\_file}.
Specifies a path and file name of a file containing 
data that can be accessed by the virtual machine.

%%% ClassAd attribute: EmailAttributes
\index{ClassAd job attribute!EmailAttributes}
\index{EmailAttributes!job ClassAd attribute}
\item[\AdAttr{EmailAttributes}:]  A string containing a comma-separated
list of job ClassAd attributes. For each attribute name in the list,
its value will be included in the e-mail notification upon job completion.

%%% ClassAd attribute: EncryptExecuteDirectory
\index{ClassAd job attribute!EncryptExecuteDirectory}
\index{EncryptExecuteDirectory!job ClassAd attribute}
\item[\AdAttr{EncryptExecuteDirectory}:]
A boolean value taken from the submit description file command
\SubmitCmd{encrypt\_execute\_directory}. 
It specifies if HTCondor should encrypt the remote scratch directory 
on the machine where the job executes.

%%% ClassAd attribute: EnteredCurrentStatus
\index{ClassAd job attribute!EnteredCurrentStatus}
\index{EnteredCurrentStatus!job ClassAd attribute}
\item[\AdAttr{EnteredCurrentStatus}:]  An integer containing the
epoch time of when the job entered into its current status
So for example, if the job is on hold, the ClassAd expression
\begin{verbatim}
    time() - EnteredCurrentStatus
\end{verbatim}
will equal the number of seconds that the job has been on hold.

%%% ClassAd attribute: Env
\index{ClassAd job attribute!Env}
\index{Env!job ClassAd attribute}
\item[\AdAttr{Env}:]  A string representing the environment variables 
passed to the job, when those arguments are specified using the
\emph{old} syntax, as specified in section~\ref{man-condor-submit-environment}.

%%% ClassAd attribute: Environment
\index{ClassAd job attribute!Environment}
\index{Environment!job ClassAd attribute}
\item[\AdAttr{Environment}:]  A string representing the environment variables
passed to the job, when those arguments are specified using the
\emph{new} syntax, as specified in section~\ref{man-condor-submit-environment}.

%%% ClassAd attribute: ExecutableSize
\index{ClassAd job attribute!ExecutableSize}
\index{ExecutableSize!job ClassAd attribute}
\item[\AdAttr{ExecutableSize}:]  Size of the executable in KiB.

%%% ClassAd attribute: ExitBySignal
\index{ClassAd job attribute!ExitBySignal}
\index{ExitBySignal!job ClassAd attribute}
\item[\AdAttr{ExitBySignal}:]  An attribute that is \Expr{True}
when a user job exits via a signal and \Expr{False} otherwise.
For some grid universe jobs, how the job exited is
unavailable. In this case, \AdAttr{ExitBySignal} is set to  \Expr{False}.

%%% ClassAd attribute: ExitCode
\index{ClassAd job attribute!ExitCode}
\index{ExitCode!job ClassAd attribute}
\item[\AdAttr{ExitCode}:]  When a user job exits by means other than a signal,
this is the exit return code of the user job.
For some grid universe jobs, how the job exited is
unavailable. In this case, \AdAttr{ExitCode} is set to  0.

%%% ClassAd attribute: ExitSignal
\index{ClassAd job attribute!ExitSignal}
\index{ExitSignal!job ClassAd attribute}
\item[\AdAttr{ExitSignal}:]  When a user job exits by means of an unhandled 
signal, this attribute takes on the numeric value of the signal.
For some grid universe jobs, how the job exited is
unavailable. In this case, \AdAttr{ExitSignal} will be undefined.

%% Karen added proper index entries up to this point in the file

%%% ClassAd attribute: ExitStatus
\index{ClassAd job attribute!ExitStatus}
\index{ExitStatus!job ClassAd attribute}
\item[\AdAttr{ExitStatus}:]  The way that HTCondor previously dealt with
a job's exit status.
This attribute should no longer be used.
It is not always accurate in
heterogeneous pools, or if the job exited with a signal.
Instead, see the attributes: \AdAttr{ExitBySignal},
\AdAttr{ExitCode}, and
\AdAttr{ExitSignal}.

%%% ClassAd attribute: GceAuthFile
\index{ClassAd job attribute!GceAuthFile}
\index{GceAuthFile!job ClassAd attribute}
\item[\AdAttr{GceAuthFile}:] 
Used for grid type gce jobs;
a string taken from the definition of the submit description file command
\SubmitCmd{gce\_auth\_file}.
Defines the path and file name of the file containing authorization
credentials to use the GCE service.

%%% ClassAd attribute: GceImage
\index{ClassAd job attribute!GceImage}
\index{GceImage!job ClassAd attribute}
\item[\AdAttr{GceImage}:] 
Used for grid type gce jobs;
a string taken from the definition of the submit description file command
\SubmitCmd{gce\_image}.
Identifies the machine image of the instance.

%%% ClassAd attribute: GceJsonFile
\index{ClassAd job attribute!GceJsonFile}
\index{GceJsonFile!job ClassAd attribute}
\item[\AdAttr{GceJsonFile}:]
Used for grid type gce jobs;
a string taken from the definition of the submit description file command
\SubmitCmd{gce\_json\_file}.
Specifies the path and file name of a file containing
a set of JSON object members that should be added to the instance
description submitted to the GCE service.

%%% ClassAd attribute: GceMachineType
\index{ClassAd job attribute!GceMachineType}
\index{GceMachineType!job ClassAd attribute}
\item[\AdAttr{GceMachineType}:] 
Used for grid type gce jobs;
a string taken from the definition of the submit description file command
\SubmitCmd{gce\_machine\_type}.
Specifies the hardware profile that should be used for a GCE instance.

%%% ClassAd attribute: GceMetadata
\index{ClassAd job attribute!GceMetadata}
\index{GceMetadata!job ClassAd attribute}
\item[\AdAttr{GceMetadata}:] 
Used for grid type gce jobs;
a string taken from the definition of the submit description file command
\SubmitCmd{gce\_metadata}.
Defines a set of name/value pairs that can be accessed by the virtual machine.

%%% ClassAd attribute: GceMetadataFile
\index{ClassAd job attribute!GceMetadataFile}
\index{GceMetadataFile!job ClassAd attribute}
\item[\AdAttr{GceMetadataFile}:] 
Used for grid type gce jobs;
a string taken from the definition of the submit description file command
\SubmitCmd{gce\_metadata\_file}.
Specifies a path and file name of a file containing 
a set of name/value pairs that can be accessed by the virtual machine.

%%% ClassAd attribute: GcePreemptible
\index{ClassAd job attribute!GcePreemptible}
\index{GcePreemptible!job ClassAd attribute}
\item[\AdAttr{GcePreemptible}:]
Used for grid type gce jobs;
a boolean taken from the definition of the submit description file command
\SubmitCmd{gce\_preemptible}.
Specifies whether the virtual machine instance created in GCE should
be preemptible.

%%% ClassAd attribute: GlobalJobId
\index{ClassAd job attribute!GlobalJobId}
\index{GlobalJobId!job ClassAd attribute}
\item[\AdAttr{GlobalJobId}:] A string intended to be a unique
job identifier within a pool.
It currently contains the \Condor{schedd} daemon \Attr{Name} attribute, 
a job identifier composed of attributes \Attr{ClusterId} and 
\Attr{ProcId} separated by a period, 
and the job's submission time in seconds since 1970-01-01 00:00:00 UTC,
separated by \verb@#@ characters.
The value \verb@submit.example.com#152.3#1358363336@ is an example.

%%% ClassAd attribute: GridJobStatus
\index{ClassAd job attribute!GridJobStatus}
\index{GridJobStatus!job ClassAd attribute}
\item[\AdAttr{GridJobStatus}:] A string containing the job's status as
reported by the remote job management system.

%%% ClassAd attribute: GridResource
\index{ClassAd job attribute!GridResource}
\index{GridResource!job ClassAd attribute}
\item[\AdAttr{GridResource}:] A string defined by the right hand side
of the the submit description file command \SubmitCmd{grid\_resource}.
It specifies the target grid type, plus additional parameters
specific to the grid type.

\index{ClassAd job attribute!HoldKillSig}
\index{HoldKillSig!job ClassAd attribute}
\item[\AdAttr{HoldKillSig}:]    Currently only for scheduler and local
universe jobs,
a string containing a name of
a signal to be sent to the job if the job is put on hold.

%%% ClassAd attribute: HoldReason
\index{ClassAd job attribute!HoldReason}
\index{HoldReason!job ClassAd attribute}
\item[\AdAttr{HoldReason}:]    A string containing a human-readable
message about why a job is on hold.
This is the message that will be displayed in response to
the command \verb@condor_q -hold@.
It can be used to determine if a job should be released or not.

%%% ClassAd attribute: HoldReasonCode
\index{ClassAd job attribute!HoldReasonCode}
\index{HoldReasonCode!job ClassAd attribute}
\label{HoldReasonCode-job-attribute}
\item[\AdAttr{HoldReasonCode}:]    An integer value that represents the
reason that a job was put on hold.

%% Don't mess with the column widths, as they currently fit within the pdf,
%% even if they look stupid in the html.

\setlength{\LTpre}{\smallskipamount}
\setlength{\LTpost}{\smallskipamount}

\begin{center}
\begin{longtable}{|p{2cm}p{9cm}|p{4cm}|} \hline
\emph{Integer Code} & \emph{Reason for Hold} & \emph{HoldReasonSubCode} \\ \hline \hline \endhead
\caption*{\emph{continued}\ldots} \\ \endfoot
\endlastfoot
1 & The user put the job on hold with \Condor{hold}. & \\ \hline
2 & Globus middleware reported an error. &
  The GRAM error number. \\ \hline
3 & The \MacroNI{PERIODIC\_HOLD} expression evaluated to \Expr{True}. & \\ \hline
4 & The credentials for the job are invalid. & \\ \hline
5 & A job policy expression evaluated to \Expr{Undefined}. & \\ \hline
6 & The \Condor{starter} failed to start the executable. &
  The Unix  errno number. \\ \hline
7 & The standard output file for the job could not be opened. &
  The Unix  errno number. \\ \hline
8 & The standard input file for the job could not be opened. &
  The Unix  errno number. \\ \hline
9 & The standard output stream for the job could not be opened. &
  The Unix  errno number. \\ \hline
10 & The standard input stream for the job could not be opened. &
  The Unix  errno number. \\ \hline
11 & An internal HTCondor protocol error was encountered when transferring files. & \\ \hline
12 & The \Condor{starter} or \Condor{shadow} failed to receive or write job files. &
  The Unix  errno number. \\ \hline
13 & The \Condor{starter} or \Condor{shadow} failed to read or send job files. &
  The Unix  errno number. \\ \hline
14 & The initial working directory of the job cannot be accessed. &
  The Unix  errno number. \\ \hline
15 & The user requested the job be submitted on hold. & \\ \hline
16 & Input files are being spooled. & \\ \hline
17 & A standard universe job is not compatible with the
  \Condor{shadow} version available on the submitting machine. & \\ \hline
18 & An internal HTCondor protocol error was encountered when transferring
  files. & \\ \hline
19 & \Macro{<Keyword>\_HOOK\_PREPARE\_JOB} was defined but could not be executed or returned failure. & \\ \hline
20 & The job missed its deferred execution time and therefore failed to run. & \\ \hline
21 & The job was put on hold because \Macro{WANT\_HOLD} in the machine policy was true. & \\ \hline
22 & Unable to initialize job event log. & \\ \hline
23 & Failed to access user account. & \\ \hline
24 & No compatible shadow. & \\ \hline
25 & Invalid cron settings. & \\ \hline
26 & \Macro{SYSTEM\_PERIODIC\_HOLD} evaluated to true. & \\ \hline
27 & The system periodic job policy evaluated to undefined. & \\ \hline
28 & Failed while using glexec to set up the job's working directory (chown sandbox to the user). & \\ \hline
30 & Failed while using glexec to prepare output for transfer (chown sandbox to condor). & \\ \hline
32 & The maximum total input file transfer size was exceeded.  (See \Macro{MAX\_TRANSFER\_INPUT\_MB}.) & \\ \hline
33 & The maximum total output file transfer size was exceeded. (See \Macro{MAX\_TRANSFER\_OUTPUT\_MB}.) & \\ \hline
34 & Memory usage exceeds a memory limit. & \\ \hline
35 & Specified Docker image was invalid. & \\ \hline
36 & Job failed when sent the checkpoint signal it requested. & \\ \hline
37 & User error in the EC2 universe: & \\
   & \hfill Public key file not defined. & 1 \\
   & \hfill Private key file not defined. & 2 \\
   & \hfill Grid resource string missing EC2 service URL. & 4 \\
   & \hfill Failed to authenticate. & 9 \\
   & \hfill Can't use existing SSH keypair with the given server's type. & 10 \\
   & \hfill You, or somebody like you, cancelled this request. & 20 \\
   \hline
38 & Internal error in the EC2 universe: & \\
   & \hfill Grid resource type not EC2. & 3 \\
   & \hfill Grid resource type not set. & 5 \\
   & \hfill Grid job ID is not for EC2. & 7 \\
   & \hfill Unexpected remote job status. & 21 \\
   \hline
39 & Adminstrator error in the EC2 universe: & \\
   & \hfill EC2\_GAHP not defined. & 6 \\
   \hline
40 & Connection problem in the EC2 universe & \\
   & \hfill \ldots while creating an SSH keypair. & 11 \\
   & \hfill \ldots while starting an on-demand instance. & 12 \\
   & \hfill \ldots while requesting a spot instance. & 17 \\
   \hline
41 & Server error in the EC2 universe: & \\
   & \hfill Abnormal instance termination reason. & 13 \\
   & \hfill Unrecognized instance termination reason. & 14 \\
   & \hfill Resource was down for too long. & 22 \\
   \hline
42 & Instance potentially lost due to an error in the EC2 universe: & \\
   & \hfill Connection error while terminating an instance. & 15 \\
   & \hfill Failed to terminate instance too many times. & 16 \\
   & \hfill Connection error while terminating a spot request. & 17 \\
   & \hfill Failed to terminated a spot request too many times. & 18 \\
   & \hfill Spot instance request purged before instance ID acquired. & 19 \\
   \hline
43 & Pre script failed. & \\ \hline
44 & Post script failed. & \\ \hline
\end{longtable}
\end{center}

%%% ClassAd attribute: HoldReasonSubCode
\index{ClassAd job attribute!HoldReasonSubCode}
\index{HoldReasonSubCode!job ClassAd attribute}
\item[\AdAttr{HoldReasonSubCode}:]    An integer value that represents further
information to go along with the \Attr{HoldReasonCode}, for
some values of \Attr{HoldReasonCode}.
See \Attr{HoldReasonCode} for the values.

%%% ClassAd attribute: HookKeyword
\index{ClassAd machine attribute!HookKeyword}
\item[\AdAttr{HookKeyword}:] A string that uniquely identifies
a set of job hooks, and added to the ClassAd once a job is fetched.

%%% ClassAd attribute: ImageSize
\index{ClassAd job attribute!ImageSize}
\index{ImageSize!job ClassAd attribute}
\item[\AdAttr{ImageSize}:]  Maximum observed memory image size
(i.e. virtual memory) of the
job in KiB.  The initial value is equal to the size of the
executable for non-vm universe jobs, and 0 for vm universe jobs.
When the job writes a checkpoint, the \AdAttr{ImageSize}
attribute is set to the size of the checkpoint file (since the
checkpoint file contains the job's memory image).
A vanilla universe job's \AdAttr{ImageSize} is recomputed
internally every 15 seconds.
How quickly this updated information becomes visible to \Condor{q} is
controlled by \MacroNI{SHADOW\_QUEUE\_UPDATE\_INTERVAL} and
\MacroNI{STARTER\_UPDATE\_INTERVAL}.

Under Linux, \AdAttr{ProportionalSetSize} is a better indicator of
memory usage for jobs with significant sharing of memory between
processes, because \AdAttr{ImageSize} is simply the sum of virtual
memory sizes across all of the processes in the job, which may count
the same memory pages more than once.

%%% ClassAd attribute: IwdFlushNFSCache
\index{ClassAd job attribute!IwdFlushNFSCache}
\index{IwdFlushNFSCache!job ClassAd attribute}
\label{IwdFlushNFSCache-job-attribute}
\item[\AdAttr{IwdFlushNFSCache}:]  A boolean expression that controls
whether or not HTCondor attempts to flush a submit machine's NFS cache,
in order to refresh an HTCondor job's initial working directory.
The value will be \Expr{True}, unless a job explicitly adds this
attribute, setting it to \Expr{False}.

%%% ClassAd attribute: JobAdInformationAttrs
\index{ClassAd job attribute!JobAdInformationAttrs}
\index{JobAdInformationAttrs!job ClassAd attribute}
\label{JobAdInformationAttrs-job-attribute}
\item[\AdAttr{JobAdInformationAttrs}:] A comma-separated list
of attribute names.  The named attributes and their values are written
in the job event log whenever any event is being written to the log.
This is the same as the configuration setting
\MacroNI{EVENT\_LOG\_INFORMATION\_ATTRS} (see
page~\pageref{param:EventLogJobAdInformationAttrs}) but it applies to
the job event log instead of the system event log.

%%% ClassAd attribute: JobDescription
\index{ClassAd job attribute!JobDescription}
\index{JobDescription!job ClassAd attribute}
\label{JobDescription-job-attribute}
\item[\AdAttr{JobDescription}:]  A string that may be defined for
a job by setting \SubmitCmd{description} in the submit description file.
When set, tools which display the executable such as \Condor{q}
will instead use this string.
For interactive jobs that do not have a submit description file,
this string will default to \Expr{"Interactive job"}.

%%% ClassAd attribute: JobCurrentStartDate
\index{ClassAd job attribute!JobCurrentStartDate}
\index{JobCurrentStartDate!job ClassAd attribute}
\label{JobCurrentStartDate-job-attribute}
\item[\AdAttr{JobCurrentStartDate}:]  Time at which the job most recently began
running.  Measured in the
number of seconds since the epoch (00:00:00 UTC, Jan 1, 1970).  

%%% ClassAd attribute: JobCurrentStartExecutingDate
\index{ClassAd job attribute!JobCurrentStartExecutingDate}
\index{JobCurrentStartExecutingDate!job ClassAd attribute}
\label{JobCurrentStartExecutingDate-job-attribute}
\item[\AdAttr{JobCurrentStartExecutingDate}:]  Time at which the job most recently finished
transferring its input sandbox and began executing.  Measured in the
number of seconds since the epoch (00:00:00 UTC, Jan 1, 1970)

%%% ClassAd attribute: JobCurrentStartTransferOutputDate
\index{ClassAd job attribute!JobCurrentStartTransferOutputDate}
\index{JobCurrentStartTransferOutputDate!job ClassAd attribute}
\label{JobCurrentStartTransferOutputDate-job-attribute}
\item[\AdAttr{JobCurrentStartTransferOutputDate}:]  Time at which the job most recently finished
executing and began transferring its output sandbox.  Measured in the
number of seconds since the epoch (00:00:00 UTC, Jan 1, 1970)


%%% ClassAd attribute: JobLeaseDuration
\index{ClassAd job attribute!JobLeaseDuration}
\index{JobLeaseDuration!job ClassAd attribute}
\item[\AdAttr{JobLeaseDuration}:]  The number of seconds set for
a job lease, the amount of time that a job may continue running
on a remote resource,
despite its submitting machine's lack of response.
See section~\ref{sec:Job-Lease} for details on job leases.

%%% ClassAd attribute: JobMaxVacateTime
\index{ClassAd job attribute!JobMaxVacateTime}
\index{JobMaxVacateTime!job ClassAd attribute}
\item[\AdAttr{JobMaxVacateTime}:] An integer expression that specifies
the time in seconds requested by the job for being allowed to
gracefully shut down.

%%% ClassAd attribute: JobNotification
\index{ClassAd job attribute!JobNotification}
\index{JobNotification!job ClassAd attribute}
\item[\AdAttr{JobNotification}:] An integer indicating what events should
be emailed to the user. The integer values correspond to the user 
choices for the submit command \SubmitCmd{notification}.
\begin{center}
\begin{table}[hbt]
\begin{tabular}{|p{2cm}p{10cm}|} \hline
\emph{Value} & \emph{Notification value} \\ \hline \hline
0 & Never \\ \hline
1 & Always \\ \hline
2 & Complete \\ \hline
3 & Error \\ \hline
\end{tabular}
\end{table}
\end{center}

%%% ClassAd attribute: JobPrio
\index{ClassAd job attribute!JobPrio}
\index{JobPrio!job ClassAd attribute}
\item[\AdAttr{JobPrio}:]  Integer priority for this job, set by
\Condor{submit} or \Condor{prio}.  The default value is 0.  The higher
the number, the greater (better) the priority.

%%% ClassAd attribute: JobRunCount
\index{ClassAd job attribute!JobRunCount}
\index{JobRunCount!job ClassAd attribute}
\item[\AdAttr{JobRunCount}:]  This attribute is retained for backwards
  compatibility.  It may go away in the future.  It is equivalent to
  \AdAttr{NumShadowStarts} for all universes except \SubmitCmdNI{scheduler}.
  For the \SubmitCmdNI{scheduler} universe, this attribute is equivalent to
  \AdAttr{NumJobStarts}.

%%% ClassAd attribute: JobStartDate
\index{ClassAd job attribute!JobStartDate}
\index{JobStartDate!job ClassAd attribute}
\item[\AdAttr{JobStartDate}:]  Time at which the job first began
running.  Measured in the
number of seconds since the epoch (00:00:00 UTC, Jan 1, 1970).
Due to a long standing bug in the 8.6 series and earlier, the job classad that is internal to the
\Condor{startd} and \Condor{starter} sets this to the time that the job most
recently began executing.  This bug is scheduled to be fixed in the 8.7 series.

%%% ClassAd attribute: JobStatus
\index{ClassAd job attribute!JobStatus}
\index{JobStatus!job ClassAd attribute}
\index{job!state}
\item[\AdAttr{JobStatus}:]  Integer which indicates the current
status of the job.
\begin{center}
\begin{table}[hbt]
\begin{tabular}{|p{2cm}p{10cm}|} \hline
\emph{Value} & \emph{Status} \\ \hline \hline
1 & Idle \\ \hline
2 & Running \\ \hline
3 & Removed \\ \hline
4 & Completed \\ \hline
5 & Held \\ \hline
6 & Transferring Output \\ \hline
7 & Suspended \\ \hline
\end{tabular}
\end{table}
\end{center}

%%% ClassAd attribute: JobUniverse
\index{ClassAd job attribute!JobUniverse}
\index{JobUniverse!job ClassAd attribute}
\index{job!universe}
\index{universe!job ClassAd attribute definitions!standard = 1}
\index{universe!job ClassAd attribute definitions!pipe = 2 (no longer used)}
\index{universe!job ClassAd attribute definitions!linda = 3 (no longer used)}
\index{universe!job ClassAd attribute definitions!pvm = 4 (no longer used)}
\index{universe!job ClassAd attribute definitions!vanilla = 5, docker = 5}
%\index{universe!job ClassAd attribute definitions!vanilla = 5}
\index{universe!job ClassAd attribute definitions!pvmd = 6 (no longer used)}
\index{universe!job ClassAd attribute definitions!scheduler = 7}
\index{universe!job ClassAd attribute definitions!mpi = 8}
\index{universe!job ClassAd attribute definitions!grid = 9}
\index{universe!job ClassAd attribute definitions!parallel = 10}
\index{universe!job ClassAd attribute definitions!java = 11}
\index{universe!job ClassAd attribute definitions!local = 12}
\index{universe!job ClassAd attribute definitions!vm = 13}
\item[\AdAttr{JobUniverse}:]  Integer which indicates the job
universe.

\begin{center}
\begin{table}[hbt]
\begin{tabular}{|p{2cm}p{3cm}|} \hline
\emph{Value} & \emph{Universe} \\ \hline \hline
1 & standard \\ \hline
5 & vanilla, docker \\ \hline
%5 & vanilla \\ \hline
7 & scheduler \\ \hline
8 & MPI \\ \hline
9 & grid \\ \hline
10 & java \\ \hline
11 & parallel \\ \hline
12 & local \\ \hline
13 & vm \\ \hline
\end{tabular}
\end{table}
\end{center}

%%% ClassAd attribute: KeepClaimIdle
\index{ClassAd job attribute!KeepClaimIdle}
\index{KeepClaimIdle!job ClassAd attribute}
\item[\AdAttr{KeepClaimIdle}:]  An integer value that represents the number
of seconds that the \Condor{schedd} will continue to keep a claim,
in the Claimed Idle state,
after the job with this attribute defined completes, 
and there are no other jobs ready to run from this user.
This attribute may improve the performance of linear DAGs,
in the case when a dependent job can not be scheduled until its
parent has completed.
Extending the claim on the machine may permit the dependent job to be
scheduled with less delay than with waiting for the \Condor{negotiator}
to match with a new machine. 

%%% ClassAd attribute: KillSig
\index{ClassAd job attribute!KillSig}
\index{KillSig!job ClassAd attribute}
\item[\AdAttr{KillSig}:]  The Unix signal number that the job wishes to be
sent before being forcibly killed.
It is relevant only for jobs running on Unix machines.

%%% ClassAd attribute: KillSigTimeout
\index{ClassAd job attribute!KillSigTimeout}
\index{KillSigTimeout!job ClassAd attribute}
\item[\AdAttr{KillSigTimeout}:]  This attribute is replaced by the
functionality in \AdAttr{JobMaxVacateTime} as of HTCondor version 7.7.3.
The number of seconds that the job
(other than the standard universe) requests the \Condor{starter} wait
after sending the signal defined as \Attr{KillSig} and before forcibly
removing the job.
The actual amount of time will be the minimum of this value
and the execute machine's configuration variable \Macro{KILLING\_TIMEOUT}.

%%% ClassAd attribute: LastCheckpointPlatform
\index{ClassAd job attribute!LastCheckpointPlatform}
\index{LastCheckpointPlatform!job ClassAd attribute}
\item[\AdAttr{LastCheckpointPlatform}:]  An opaque string which is the
\AdAttr{CheckpointPlatform} identifier from the last machine where this
standard universe job had successfully produced a checkpoint.

%%% ClassAd attribute: LastCkptServer
\index{ClassAd job attribute!LastCkptServer}
\index{LastCkptServer!job ClassAd attribute}
\item[\AdAttr{LastCkptServer}:]  Host name of the last checkpoint
server used by this job.  When a pool is using multiple checkpoint
servers, this tells the job where to find its checkpoint file.

%%% ClassAd attribute: LastCkptTime
\index{ClassAd job attribute!LastCkptTime}
\index{LastCkptTime!job ClassAd attribute}
\item[\AdAttr{LastCkptTime}:]  Time at which the job last performed a
successful checkpoint.  Measured in the number of seconds since the
epoch (00:00:00 UTC, Jan 1, 1970).

%%% ClassAd attribute: LastMatchTime
\index{ClassAd job attribute!LastMatchTime}
\index{LastMatchTime!job ClassAd attribute}
\item[\AdAttr{LastMatchTime}:]  An integer containing the epoch time
when the job was last successfully matched with a resource (gatekeeper) Ad.

%%% ClassAd attribute: LastRejMatchReason
\index{ClassAd job attribute!LastRejMatchReason}
\index{LastRejMatchReason!job ClassAd attribute}
\item[\AdAttr{LastRejMatchReason}:]   If, at any point in the past,
this job failed to match with a resource ad,
this attribute will contain a string with a
human-readable message about why the match failed.

%%% ClassAd attribute: LastRejMatchTime
\index{ClassAd job attribute!LastRejMatchTime}
\index{LastRejMatchTime!job ClassAd attribute}
\item[\AdAttr{LastRejMatchTime}:]   An integer containing the epoch
time when HTCondor-G last tried to find a match for the job,
but failed to do so.

%%% ClassAd attribute: LastRemotePool
\index{ClassAd job attribute!LastRemotePool}
\index{LastRemotePool!job ClassAd attribute}
\item[\AdAttr{LastRemotePool}:]  The name of the \Condor{collector} of the pool
in which a job ran via flocking in the most recent run attempt.  
This attribute is not defined if the job did not run via flocking.

%%% ClassAd attribute: LastSuspensionTime
\index{ClassAd job attribute!LastSuspensionTime}
\index{LastSuspensionTime!job ClassAd attribute}
\item[\AdAttr{LastSuspensionTime}:]  Time at which the job last performed a
successful suspension.  Measured in the number of seconds since the
epoch (00:00:00 UTC, Jan 1, 1970).

%%% ClassAd attribute: LastVacateTime
\index{ClassAd job attribute!LastVacateTime}
\index{LastVacateTime!job ClassAd attribute}
\item[\AdAttr{LastVacateTime}:]  Time at which the job was last
evicted from a remote workstation.  Measured in the number of seconds
since the epoch (00:00:00 UTC, Jan 1, 1970).

%%% ClassAd attribute: LeaveJobInQueue
\index{ClassAd job attribute!LeaveJobInQueue}
\index{LeaveJobInQueue!job ClassAd attribute}
\item[\AdAttr{LeaveJobInQueue}:]  A boolean expression that defaults to
\Expr{False}, causing the job to be removed from the queue upon completion.
An exception is if the job is submitted using \Expr{condor\_submit -spool}.
For this case, the default expression causes the job to be kept in the queue
for 10 days after completion.

%%% ClassAd attribute: LocalSysCpu
\index{ClassAd job attribute!LocalSysCpu}
\index{LocalSysCpu!job ClassAd attribute}
\item[\AdAttr{LocalSysCpu}:]  An accumulated number of seconds of 
system CPU time that the job caused to the machine upon which
the job was submitted.

%%% ClassAd attribute: LocalUserCpu
\index{ClassAd job attribute!LocalUserCpu}
\index{LocalUserCpu!job ClassAd attribute}
\item[\AdAttr{LocalUserCpu}:]  An accumulated number of seconds of 
user CPU time that the job caused to the machine upon which
the job was submitted.

%%% ClassAd attribute: MachineAttr<X><N>
\index{ClassAd job attribute!MachineAttr<X><N>}
\index{MachineAttr<X><N>!job ClassAd attribute}
\item[\AdAttr{MachineAttr<X><N>}:] 
Machine attribute of name \Attr{<X>} that is placed into this job ClassAd,
as specified by the configuration variable
\MacroNI{SYSTEM\_JOB\_MACHINE\_ATTRS}.
With the potential for multiple run attempts, \Attr{<N>} represents
an integer value providing historical values of this machine attribute
for multiple runs.
The most recent run will have a value of \Attr{<N>} equal to \Expr{0}.
The next most recent run will have a value of \Attr{<N>} equal to \Expr{1}.

%%% ClassAd attribute: MaxHosts
\index{ClassAd job attribute!MaxHosts}
\index{MaxHosts!job ClassAd attribute}
\item[\AdAttr{MaxHosts}:]  The maximum number of hosts that this job would
like to claim. As long as \AdAttr{CurrentHosts} is the same as
\AdAttr{MaxHosts}, no more hosts are negotiated for.

%%% ClassAd attribute: MaxJobRetirementTime
\index{ClassAd job attribute!MaxJobRetirementTime}
\index{MaxJobRetirementTime!job ClassAd attribute}
\item[\AdAttr{MaxJobRetirementTime}:]  Maximum time in seconds to let this
job run uninterrupted before kicking it off when it is being preempted.
This can only decrease the amount of time from what the corresponding
startd expression allows.

%%% ClassAd attribute: MaxTransferInputMB
\index{ClassAd job attribute!MaxTransferInputMB}
\index{MaxTransferInputMB!job ClassAd attribute}
\item[\AdAttr{MaxTransferInputMB}:]
This integer expression specifies the maximum allowed total size in
Mbytes of the input files that are transferred for a job.  This
expression does \emph{not} apply to grid universe, standard universe, or
files transferred via file transfer plug-ins.  The expression may refer
to attributes of the job.  The special value -1 indicates no limit.
If not set, the system setting \Macro{MAX\_TRANSFER\_INPUT\_MB} is
used.  If the observed size of all input files at submit time is
larger than the limit, the job will be immediately placed on hold with
a \Attr{HoldReasonCode} value of 32.
If the job passes this initial test, but the size of
the input files increases or the limit decreases so that the limit is
violated, the job will be placed on hold at the time when the file
transfer is attempted.

%%% ClassAd attribute: MaxTransferOutputMB
\index{ClassAd job attribute!MaxTransferOutputMB}
\index{MaxTransferOutputMB!job ClassAd attribute}
\item[\AdAttr{MaxTransferOutputMB}:]
This integer expression specifies the maximum allowed total size in
Mbytes of the output files that are transferred for a job.  This
expression does \emph{not} apply to grid universe, standard universe, or
files transferred via file transfer plug-ins.  The expression may refer
to attributes of the job.  The special value -1 indicates no limit.
If not set, the system setting \Macro{MAX\_TRANSFER\_OUTPUT\_MB} is
used.  If the total size of the job's output files to be transferred
is larger than the limit, the job will be placed on hold with 
a \Attr{HoldReasonCode} value of 33.
The output will be transferred up to the point when the
limit is hit, so some files may be fully transferred, some partially,
and some not at all.

%%% ClassAd attribute: MemoryUsage
\index{ClassAd job attribute!MemoryUsage}
\index{MemoryUsage!job ClassAd attribute}
\item[\AdAttr{MemoryUsage}:]  An integer expression in units of Mbytes that
represents the peak memory usage for the job.
Its purpose is to be compared with the value defined by a job with the
\SubmitCmd{request\_memory} submit command,
for purposes of policy evaluation.

%%% ClassAd attribute: MinHosts
\index{ClassAd job attribute!MinHosts}
\index{MinHosts!job ClassAd attribute}
\item[\AdAttr{MinHosts}:]  The minimum number of hosts that must be in
the claimed state for this job, before the job may enter the running state.

%%% ClassAd attribute: NextJobStartDelay
\index{ClassAd job attribute!NextJobStartDelay}
\index{NextJobStartDelay!job ClassAd attribute}
\item[\AdAttr{NextJobStartDelay}:]  An integer number of seconds delay time
after this job starts until the next job is started. The value is limited
by the configuration variable \Macro{MAX\_NEXT\_JOB\_START\_DELAY}.

%%% ClassAd attribute: NiceUser
\index{ClassAd job attribute!NiceUser}
\index{NiceUser!job ClassAd attribute}
\item[\AdAttr{NiceUser}:]  Boolean value which when \Expr{True} indicates
that this job is a \emph{nice} job, raising its user priority value, 
thus causing it to run on a machine only when no other HTCondor jobs want 
the machine.

%%% ClassAd attribute: Nonessential
\index{ClassAd job attribute!Nonessential}
\index{Nonessential!job ClassAd attribute}
\item[\AdAttr{Nonessential}:]  A boolean value only relevant to grid universe
jobs, which when \Expr{True} tells HTCondor to simply abort (remove) 
any problematic job, instead of putting the job on hold.
It is the equivalent of doing \Condor{rm} followed by 
\Condor{rm} \Opt{-forcex} any time the job would have otherwise gone on hold. 
If not explicitly set to \Expr{True}, the default value will be \Expr{False}.

%%% ClassAd attribute: NTDomain
\index{ClassAd job attribute!NTDomain}
\index{NTDomain!job ClassAd attribute}
\item[\AdAttr{NTDomain}:]  A string that identifies the NT domain under
which a job's owner authenticates on a platform running Windows.

%%% ClassAd attribute: NumCkpts
\index{ClassAd job attribute!NumCkpts}
\index{NumCkpts!job ClassAd attribute}
\item[\AdAttr{NumCkpts}:]  A count of the number of checkpoints
written by this job during its lifetime.

%%% ClassAd attribute: NumGlobusSubmits
\index{ClassAd job attribute!NumGlobusSubmits}
\index{NumGlobusSubmits!job ClassAd attribute}
\item[\AdAttr{NumGlobusSubmits}:]   An integer that is incremented each
time the \Condor{gridmanager} receives confirmation of a successful job
submission into Globus.

%%% ClassAd attribute: NumJobCompletions
\index{ClassAd job attribute!NumJobCompletions}
\index{NumJobCompletions!job ClassAd attribute}
\item[\AdAttr{NumJobCompletions}:]  An integer, initialized to zero,
that is incremented by the \Condor{shadow} each time the job's
executable exits of its own accord, with or without errors, and
successfully completes file transfer (if requested).  Jobs which have
done so normally enter the completed state; this attribute is therefore
normally only of use when, for example, \Attr{on\_exit\_remove} or
\Attr{on\_exit\_hold} is set.

%%% ClassAd attribute: NumJobMatches
\index{ClassAd job attribute!NumJobMatches}
\index{NumJobMatches!job ClassAd attribute}
\item[\AdAttr{NumJobMatches}:]  An integer that is incremented by the
\Condor{schedd} each time the job is matched with a resource ad by the
negotiator.

%%% ClassAd attribute: NumJobReconnects
\index{ClassAd job attribute!NumJobReconnects}
\index{NumJobReconnects!job ClassAd attribute}
\item[\AdAttr{NumJobReconnects}:]  An integer count of the number of times a
  job successfully reconnected after being disconnected.
  This occurs when the
  \Condor{shadow} and \Condor{starter} lose contact,
  for example because of
  transient network failures or a \Condor{shadow} or \Condor{schedd}
  restart.
  This attribute is only defined for jobs that can reconnected:
  those in the \SubmitCmdNI{vanilla} and \SubmitCmdNI{java} universes.

%%% ClassAd attribute: NumJobStarts
\index{ClassAd job attribute!NumJobStarts}
\index{NumJobStarts!job ClassAd attribute}
\item[\AdAttr{NumJobStarts}:]  An integer count of the number of times the
  job started executing.
  This is not (yet) defined for \SubmitCmdNI{standard} universe jobs.

%%% ClassAd attribute: NumPids
\index{ClassAd job attribute!NumPids}
\index{NumPids!job ClassAd attribute}
\item[\AdAttr{NumPids}:]  A count of the number of child processes that
this job has.

%%% ClassAd attribute: NumRestarts
\index{ClassAd job attribute!NumRestarts}
\index{NumRestarts!job ClassAd attribute}
\item[\AdAttr{NumRestarts}:]  A count of the number of restarts from a
checkpoint attempted by this job during its lifetime.

%%% ClassAd attribute: NumShadowExceptions
\index{ClassAd job attribute!NumShadowExceptions}
\index{NumShadowExceptions!job ClassAd attribute}
\item[\AdAttr{NumShadowExceptions}:]  An integer count of the number of
  times the \Condor{shadow} daemon had a fatal error for a given job.

%%% ClassAd attribute: NumShadowStarts
\index{ClassAd job attribute!NumShadowStarts}
\index{NumShadowStarts!job ClassAd attribute}
\item[\AdAttr{NumShadowStarts}:]  An integer count of the number of
  times a \Condor{shadow} daemon was started for a given job.
  This attribute is not defined for
  \SubmitCmdNI{scheduler} universe jobs, since
  they do not have a \Condor{shadow} daemon associated with them.
  For \SubmitCmdNI{local} universe jobs, this attribute \emph{is}
  defined, even though the process that manages the job is technically
  a \Condor{starter} rather than a \Condor{shadow}.  
  This keeps the management of the
  local universe and other universes as similar as possible.
  \Bold{Note that this attribute is incremented every time the job
  is matched, even if the match is rejected by the execute machine;
  in other words, the value of this attribute may be greater than
  the number of times the job actually ran.}

%%% ClassAd attribute: NumSystemHolds
\index{ClassAd job attribute!NumSystemHolds}
\index{NumSystemHolds!job ClassAd attribute}
\item[\AdAttr{NumSystemHolds}:]   An integer that is incremented each time
HTCondor-G places a job on hold due to some sort of error condition.  This
counter is useful, since HTCondor-G will always place a job on hold when it
gives up on some error condition.  Note that if the user places the job
on hold using the \Condor{hold} command, this attribute is not incremented.

%%% ClassAd attribute: OtherJobRemoveRequirements
\label{attribute-OtherJobRemoveRequirements}
\index{ClassAd job attribute!OtherJobRemoveRequirements}
\index{OtherJobRemoveRequirements!job ClassAd attribute}
\item[\AdAttr{OtherJobRemoveRequirements}:]
A string that defines a list of jobs.
When the job with this attribute defined is removed,
all other jobs defined by the list are also removed.
The string is an expression that defines a constraint equivalent to
the one implied by the command 
\begin{verbatim}
  condor_rm -constraint <constraint>
\end{verbatim}
This attribute is used for jobs managed with \Condor{dagman} to ensure
that node jobs of the DAG are removed when the \Condor{dagman} job
itself is removed.  Note that the list of jobs defined by this attribute
must not form a cyclic removal of jobs,
or the \Condor{schedd} will go into an infinite loop 
when any of the jobs is removed.

%%% ClassAd attribute: OutputDestination
\index{ClassAd job attribute!OutputDestination}
\index{OutputDestination!job ClassAd attribute}
\item[\AdAttr{OutputDestination}:]  A URL, as defined by submit command
\SubmitCmdNI{output\_destination}.

%%% ClassAd attribute: Owner
\index{ClassAd job attribute!Owner}
\index{Owner!job ClassAd attribute}
\item[\AdAttr{Owner}:]  String describing the user who submitted this
job.

%%% ClassAd attribute: ParallelShutdownPolicy
\index{ClassAd job attribute!ParallelShutdownPolicy}
\index{ParallelShutdownPolicy!job ClassAd attribute}
\item[\AdAttr{ParallelShutdownPolicy}:]  A string that is only relevant
to parallel universe jobs.  Without this attribute defined, the default
policy applied to parallel universe jobs is to consider the whole job
completed when the first node exits, killing processes running on
all remaining nodes.  If defined to the following strings, HTCondor's
behavior changes:
  \begin{description}
  \item[\AdStr{WAIT\_FOR\_ALL}] HTCondor will wait until every node in 
  the parallel job has completed to consider the job finished.
  \end{description}

%%% This is what the {Post,Pre}{Arg[ument]s,Env[ironment],Cmd}
%%% attributes were called when they were documented in the condor_submit
%%% man page, which was not the right place for them.
\index{Starter pre and post scripts}

%%% ClassAd attribute: PostArgs
\index{ClassAd job attribute!PostArgs}
\index{PostArgs!ClassAd job attribute}
\item[\AdAttr{PostArgs}:] Defines the command-line arguments for the
post command using the \emph{old} argument syntax, as specified in
section~\ref{man-condor-submit-arguments}. If both \AdAttr{PostArgs}
and \AdAttr{PostArguments} exists, the former is ignored.

%%% ClassAd attribute: PostArguments
\index{ClassAd job attribute!PostArguments}
\index{PostArguments!ClassAd job attribute}
\item[\AdAttr{PostArguments}:] Defines the command-line arguments for the
post command using the \emph{new} argument syntax, as specified in
section~\ref{man-condor-submit-arguments}, excepting that double quotes must
be escaped with a backslash instead of another double quote. If both
\AdAttr{PostArgs} and \AdAttr{PostArguments} exists, the former is ignored.

%%% ClassAd attribute: PostCmd
\index{ClassAd job attribute!PostCmd}
\index{PostCmd!ClassAd job attribute}
\label{PostCmd}
\item[\AdAttr{PostCmd}:] A job in the vanilla, Docker, Java, or virtual machine
universes may specify a command to run after the \SubmitCmd{Executable} has
exited, but before file transfer is started.  Unlike a DAGMan POST script
command, this command \emph{is} run on the execute machine; however, it
is not run in the same environment as the \SubmitCmd{Executable}.  Instead,
its environment is set by \AdAttr{PostEnv} or \AdAttr{PostEnvironment}.  Like
the DAGMan POST script command, this command is not run in the same universe
as the \SubmitCmd{Executable}; in particular, this command is not run in a
Docker container, nor in a virtual machine, nor in Java.  This command is
also not run with any of vanilla universe's features active, including (but
not limited to): cgroups, PID namespaces, bind mounts, CPU affinity,
Singularity, or job wrappers. This command is \emph{not}
automatically transferred with the job, so if you're using file transfer,
you must add it to the \SubmitCmd{transfer\_input\_files} list.

If the specified command is in the job's execute directory, or any
sub-directory, you should not set \SubmitCmd{vm\_no\_output\_vm}, as that
will delete all the files in the job's execute directory before this
command has a chance to run.  If you don't want any output back from
your VM universe job, but you do want to run a post command, do not
set \SubmitCmd{vm\_no\_output\_vm} and instead delete the job's execute
directory in your post command.

%%% ClassAd attribute: PostCmdExitBySignal
\index{ClassAd job attribute!PostCmdExitBySignal}
\index{PostCmdExitBySignal!ClassAd job attribute}
\item[\AdAttr{PostCmdExitBySignal}:] If \AdAttr{SuccessPostExitCode} or
\AdAttr{SuccessPostExitSignal} were set, and the post command has run,
this attribute will true if the the post command exited on a signal and
false if it did not.  It is otherwise unset.

%%% ClassAd attribute: PostCmdExitCode
\index{ClassAd job attribute!PostCmdExitCode}
\index{PostCmdExitCode!ClassAd job attribute}
\item[\AdAttr{PostCmdExitCode}:] If \AdAttr{SuccessPostExitCode} or
\AdAttr{SuccessPostExitSignal} were set, the post command has run,
and the post command did not exit on a signal, then this attribute will
be set to the exit code.  It is otherwise unset.

%%% ClassAd attribute: PostCmdExitSignal
\index{ClassAd job attribute!PostCmdExitSignal}
\index{PostCmdExitSignal!ClassAd job attribute}
\item[\AdAttr{PostCmdExitSignal}:] If \AdAttr{SuccessPostExitCode} or
\AdAttr{SuccessPostExitSignal} were set, the post command has run,
and the post command exited on a signal, then this attribute will
be set to that signal.  It is otherwise unset.

%%% ClassAd attribute: PostEnv
\index{ClassAd job attribute!PostEnv}
\index{PostEnv!ClassAd job attribute}
\item[\AdAttr{PostEnv}:] Defines the environment for the Postscript
using the Old environment syntax. If both \AdAttr{PostEnv} and
\AdAttr{PostEnvironment} exist, the former is ignored.

%%% ClassAd attribute: PostEnvironment
\index{ClassAd job attribute!PostEnvironment}
\index{PostEnvironment!ClassAd job attribute}
\item[\AdAttr{PostEnvironment}:] Defines the environment for the Postscript
using the New environment syntax. If both \AdAttr{PostEnv} and
\AdAttr{PostEnvironment} exist, the former is ignored.

%%% ClassAd attribute: PreArgs
\index{ClassAd job attribute!PreArgs}
\index{PreArgs!ClassAd job attribute}
\item[\AdAttr{PreArgs}:] Defines the command-line arguments for the
pre command using the \emph{old} argument syntax, as specified in
section~\ref{man-condor-submit-arguments}.  If both \AdAttr{PreArgs}
and \AdAttr{PreArguments} exists, the former is ignored.

%%% ClassAd attribute: PreArguments
\index{ClassAd job attribute!PreArguments}
\index{PreArguments!ClassAd job attribute}
\item[\AdAttr{PreArguments}:] Defines the command-line arguments for the
pre command using the \emph{new} argument syntax, as specified in
section~\ref{man-condor-submit-arguments}, excepting that double quotes must
be escape with a backslash instead of another double quote. If both
\AdAttr{PreArgs} and \AdAttr{PreArguments} exists, the former is ignored.

%%% ClassAd attribute: PreCmd
\index{ClassAd job attribute!PreCmd}
\index{PreCmd!ClassAd job attribute}
\label{PreCmd}
\item[\AdAttr{PreCmd}:] A job in the vanilla, Docker, Java, or virtual machine
universes may specify a command to run after file transfer (if any) completes
but before the \SubmitCmd{Executable} is started. Unlike a DAGMan PRE script
command, this command \emph{is} run on the execute machine; however, it
is not run in the same environment as the \SubmitCmd{Executable}.  Instead,
its environment is set by \AdAttr{PreEnv} or \AdAttr{PreEnvironment}.  Like
the DAGMan POST script command, this command is not run in the same universe
as the \SubmitCmd{Executable}; in particular, this command is not run in a
Docker container, nor in a virtual machine, nor in Java.  This command is
also not run with any of vanilla universe's features active, including (but
not limited to): cgroups, PID namespaces, bind mounts, CPU affinity,
Singularity, or job wrappers. This command is \emph{not}
automatically transferred with the job, so if you're using file transfer,
you must add it to the \SubmitCmd{transfer\_input\_files} list.

%%% ClassAd attribute: PreCmdExitBySignal
\index{ClassAd job attribute!PreCmdExitBySignal}
\index{PreCmdExitBySignal!ClassAd job attribute}
\item[\AdAttr{PreCmdExitBySignal}:] If \AdAttr{SuccessPreExitCode} or
\AdAttr{SuccessPreExitSignal} were set, and the pre command has run,
this attribute will true if the the pre command exited on a signal and
false if it did not.  It is otherwise unset.

%%% ClassAd attribute: PreCmdExitCode
\index{ClassAd job attribute!PreCmdExitCode}
\index{PreCmdExitCode!ClassAd job attribute}
\item[\AdAttr{PreCmdExitCode}:] If \AdAttr{SuccessPreExitCode} or
\AdAttr{SuccessPreExitSignal} were set, the pre command has run,
and the pre command did not exit on a signal, then this attribute will
be set to the exit code.  It is otherwise unset.

%%% ClassAd attribute: PreCmdExitSignal
\index{ClassAd job attribute!PreCmdExitSignal}
\index{PreCmdExitSignal!ClassAd job attribute}
\item[\AdAttr{PreCmdExitSignal}:] If \AdAttr{SuccessPreExitCode} or
\AdAttr{SuccessPreExitSignal} were set, the pre command has run,
and the pre command exited on a signal, then this attribute will
be set to that signal.  It is otherwise unset.

%%% ClassAd attribute: PreEnv
\index{ClassAd job attribute!PreEnv}
\index{PreEnv!ClassAd job attribute}
\item[\AdAttr{PreEnv}:] Defines the environment for the prescript
using the Old environment syntax. If both \AdAttr{PreEnv} and
\AdAttr{PreEnvironment} exist, the former is ignored.

%%% ClassAd attribute: PreEnvironment
\index{ClassAd job attribute!PreEnvironment}
\index{PreEnvironment!ClassAd job attribute}
\item[\AdAttr{PreEnvironment}:] Defines the environment for the prescript
using the New environment syntax. If both \AdAttr{PreEnv} and
\AdAttr{PreEnvironment} exist, the former is ignored.

%%% ClassAd attribute: PreJobPrio1
\index{ClassAd job attribute!PreJobPrio1}
\index{PreJobPrio1!job ClassAd attribute}
\item[\AdAttr{PreJobPrio1}:]  An integer value representing a user's
priority to affect of choice of jobs to run.
A larger value gives higher priority. 
The range of valid values is  \Expr{INT\_MIN + 1} to \Expr{INT\_MAX}.
When not explicitly set for a job,
\Expr{INT\_MIN}, the lowest possible priority, is used for comparison purposes.
This attribute, when set, is considered first:
before \AdAttr{PreJobPrio2}, before \AdAttr{JobPrio},
before \AdAttr{PostJobPrio1}, before \AdAttr{PostJobPrio2},
and before \AdAttr{QDate}.

%%% ClassAd attribute: PreJobPrio2
\index{ClassAd job attribute!PreJobPrio2}
\index{PreJobPrio2!job ClassAd attribute}
\item[\AdAttr{PreJobPrio2}:]  An integer value representing a user's
priority to affect of choice of jobs to run.
A larger value gives higher priority. 
The range of valid values is  \Expr{INT\_MIN + 1} to \Expr{INT\_MAX}.
When not explicitly set for a job,
\Expr{INT\_MIN}, the lowest possible priority, is used for comparison purposes.
This attribute, when set, is considered after \AdAttr{PreJobPrio1}, 
but before \AdAttr{JobPrio},
before \AdAttr{PostJobPrio1}, before \AdAttr{PostJobPrio2},
and before \AdAttr{QDate}.

%%% ClassAd attribute: PostJobPrio1
\index{ClassAd job attribute!PostJobPrio1}
\index{PostJobPrio1!job ClassAd attribute}
\item[\AdAttr{PostJobPrio1}:]  An integer value representing a user's
priority to affect of choice of jobs to run.
A larger value gives higher priority. 
The range of valid values is  \Expr{INT\_MIN + 1} to \Expr{INT\_MAX}.
When not explicitly set for a job,
\Expr{INT\_MIN}, the lowest possible priority, is used for comparison purposes.
This attribute, when set, is considered after \AdAttr{PreJobPrio1}, 
after \AdAttr{PreJobPrio1},
and after \AdAttr{JobPrio},
but before \AdAttr{PostJobPrio2},
and before \AdAttr{QDate}.

%%% ClassAd attribute: PostJobPrio2
\index{ClassAd job attribute!PostJobPrio2}
\index{PostJobPrio2!job ClassAd attribute}
\item[\AdAttr{PostJobPrio2}:]  An integer value representing a user's
priority to affect of choice of jobs to run.
A larger value gives higher priority. 
The range of valid values is  \Expr{INT\_MIN + 1} to \Expr{INT\_MAX}.
When not explicitly set for a job,
\Expr{INT\_MIN}, the lowest possible priority, is used for comparison purposes.
This attribute, when set, is considered after \AdAttr{PreJobPrio1}, 
after \AdAttr{PreJobPrio1},
after \AdAttr{JobPrio},
and after \AdAttr{PostJobPrio1},
but before \AdAttr{QDate}.

%%% ClassAd attribute: PreserveRelativeExecutable
\index{ClassAd job attribute!PreserveRelativeExecutable}
\index{PreserveRelativeExecutable!job ClassAd attribute}
\item[\AdAttr{PreserveRelativeExecutable}:]  When \Expr{True}, 
the \Condor{starter} will not prepend \Attr{Iwd}
to \Attr{Cmd}, when \Attr{Cmd} is a relative path name
and \Attr{TransferExecutable} is \Expr{False}.  
The default value is \Expr{False}.
This attribute is primarily of interest for users of 
\MacroNI{USER\_JOB\_WRAPPER}
for the purpose of allowing an executable's location to be resolved 
by the user's path in the job wrapper.

%%% ClassAd attribute: ProcId
\index{ClassAd job attribute!ProcId}
\index{ProcId!job ClassAd attribute}
\index{process!definition for a submitted job}
\index{job ID!process identifier}
\item[\AdAttr{ProcId}:]  Integer process identifier for this job.
Within a cluster of many jobs,
each job has the same \Attr{ClusterId}, but will have a unique \Attr{ProcId}.
Within a cluster, assignment of a \Attr{ProcId} value will start
with the value 0.
The job (process) identifier described here is unrelated to operating
system PIDs.

%%% ClassAd attribute: ProportionalSetSizeKb
\index{ClassAd job attribute!ProportionalSetSizeKb}
\index{ProportionalSetSizeKb!job ClassAd attribute}
\item[\AdAttr{ProportionalSetSizeKb}:]
On Linux execute machines with kernel version more recent than 2.6.27,
this is the maximum observed proportional set size (PSS) in KiB,
summed across all processes in the job.
If the execute machine does not
support monitoring of PSS or PSS has not yet been measured,
this attribute will be undefined.  
PSS differs from \AdAttr{ImageSize} in how memory shared
between processes is accounted.
The PSS for one process is the sum of that process' memory pages 
divided by the number of processes sharing each of the pages.
\Attr{ImageSize} is the same,
except there is no division by the number of processes sharing the pages.

%%% ClassAd attribute: QDate
\index{ClassAd job attribute!QDate}
\index{QDate!job ClassAd attribute}
\item[\AdAttr{QDate}:]  Time at which the job was submitted to the job
queue.  Measured in the
number of seconds since the epoch (00:00:00 UTC, Jan 1, 1970).

%%% ClassAd attribute: RecentBlockReadKbytes
\index{ClassAd job attribute!RecentBlockReadKbytes}
\index{RecentBlockReadKbytes!job ClassAd attribute}
\item[\AdAttr{RecentBlockReadKbytes}:] The integer number of KiB
read from disk for this job over the previous time interval defined
by configuration variable \MacroNI{STATISTICS\_WINDOW\_SECONDS}.

%%% ClassAd attribute: RecentBlockReads
\index{ClassAd job attribute!RecentBlockReads}
\index{RecentBlockReads!job ClassAd attribute}
\item[\AdAttr{RecentBlockReads}:] The integer number of disk blocks
read for this job over the previous time interval defined
by configuration variable \MacroNI{STATISTICS\_WINDOW\_SECONDS}.

%%% ClassAd attribute: RecentBlockWriteKbytes
\index{ClassAd job attribute!RecentBlockWriteKbytes}
\index{RecentBlockWriteKbytes!job ClassAd attribute}
\item[\AdAttr{RecentBlockWriteKbytes}:] The integer number of KiB
written to disk for this job over the previous time interval defined
by configuration variable \MacroNI{STATISTICS\_WINDOW\_SECONDS}.

%%% ClassAd attribute: RecentBlockWrites
\index{ClassAd job attribute!RecentBlockWrites}
\index{RecentBlockWrites!job ClassAd attribute}
\item[\AdAttr{RecentBlockWrites}:] The integer number of blocks
written to disk for this job over the previous time interval defined
by configuration variable \MacroNI{STATISTICS\_WINDOW\_SECONDS}.

%%% ClassAd attribute: ReleaseReason
\index{ClassAd job attribute!ReleaseReason}
\index{ReleaseReason!job ClassAd attribute}
\item[\AdAttr{ReleaseReason}:]     A string containing a human-readable
message about why the job was released from hold.

%%% ClassAd attribute: RemoteIwd
\index{ClassAd job attribute!RemoteIwd}
\index{RemoteIwd!job ClassAd attribute}
\item[\AdAttr{RemoteIwd}:]  The path to the directory in which
a job is to be executed on a remote machine.

%%% ClassAd attribute: RemotePool
\index{ClassAd job attribute!RemotePool}
\index{RemotePool!job ClassAd attribute}
\item[\AdAttr{RemotePool}:]  The name of the \Condor{collector} of the pool 
in which a job is running via flocking.  
This attribute is not defined if the job is not running via flocking.

%%% ClassAd attribute: RemoteSysCpu
\index{ClassAd job attribute!RemoteSysCpu}
\index{RemoteSysCpu!job ClassAd attribute}
\item[\AdAttr{RemoteSysCpu}:]  The total number of seconds
of system CPU time (the time spent at system calls) the job used
on remote machines.  This does not count time spent on run attempts that
were evicted without a checkpoint.

%%% ClassAd attribute: CumulativeRemoteSysCpu
\index{ClassAd job attribute!CumulativeRemoteSysCpu}
\index{CumulativeRemoteSysCpu!job ClassAd attribute}
\item[\AdAttr{CumulativeRemoteSysCpu}:]  The total number of seconds
of system CPU time the job used on remote machines, summed over all
execution attempts.

%%% ClassAd attribute: RemoteUserCpu
\index{ClassAd job attribute!RemoteUserCpu}
\index{RemoteUserCpu!job ClassAd attribute}
\item[\AdAttr{RemoteUserCpu}:]  The total number of seconds
of user CPU time the job used on remote machines.  This does not
count time spent on run attempts that were evicted without a checkpoint.
A job in the virtual machine universe will only report this attribute if run
on a KVM hypervisor.

%%% ClassAd attribute: CumulativeRemoteUserCpu
\index{ClassAd job attribute!CumulativeRemoteUserCpu}
\index{CumulativeRemoteUserCpu!job ClassAd attribute}
\item[\AdAttr{CumulativeRemoteUserCpu}:]  The total number of seconds
of user CPU time the job used on remote machines, summed over all
execution attempts.

%%% ClassAd attribute: RemoteWallClockTime
\index{ClassAd job attribute!RemoteWallClockTime}
\index{RemoteWallClockTime!job ClassAd attribute}
\label{RemoteWallClockTime}
\item[\AdAttr{RemoteWallClockTime}:]  Cumulative number of seconds
the job has been allocated a machine.
This also includes time spent in suspension (if any),
so the total real time spent running is 
\begin{verbatim}
RemoteWallClockTime - CumulativeSuspensionTime
\end{verbatim}
Note that this number does not get reset to
zero when a job is forced to migrate from one machine to another.
\AdAttr{CommittedTime}, on the other hand, is just like
\AdAttr{RemoteWallClockTime} except it does get reset to 0 whenever
the job is evicted without a checkpoint.

%%% ClassAd attribute: RemoveKillSig
\index{ClassAd job attribute!RemoveKillSig}
\index{RemoveKillSig!job ClassAd attribute}
\item[\AdAttr{RemoveKillSig}:]    Currently only for scheduler universe jobs,
a string containing a name of
a signal to be sent to the job if the job is removed.

%%% ClassAd attribute: RequestCpus
\index{ClassAd job attribute!RequestCpus}
\index{RequestCpus!job ClassAd attribute}
\item[\AdAttr{RequestCpus}:]  The number of CPUs requested for this job.
If dynamic \Condor{startd} provisioning is enabled,
it is the minimum number of CPUs that are needed in the created dynamic slot.

%%% ClassAd attribute: RequestDisk
\index{ClassAd job attribute!RequestDisk}
\index{RequestDisk!job ClassAd attribute}
\item[\AdAttr{RequestDisk}:]  The amount of disk space in KiB requested 
for this job.
If dynamic \Condor{startd} provisioning is enabled,
it is the minimum amount of disk space needed in the created dynamic slot.

%%% ClassAd attribute: RequestedChroot
\index{ClassAd job attribute!RequestedChroot}
\index{RequestedChroot!job ClassAd attribute}
\item[\AdAttr{RequestedChroot}:]  A full path to the directory that the job
requests the \Condor{starter} use as an argument to \Procedure{chroot}.

%%% ClassAd attribute: RequestMemory
\index{ClassAd job attribute!RequestMemory}
\index{RequestMemory!job ClassAd attribute}
\item[\AdAttr{RequestMemory}:]  The amount of memory space in MiB 
requested for this job.
If dynamic \Condor{startd} provisioning is enabled,
it is the minimum amount of memory needed in the created dynamic slot.
If not set by the job, its definition is specified by 
configuration variable \Macro{JOB\_DEFAULT\_REQUESTMEMORY}.

%%% ClassAd attribute: ResidentSetSize
\index{ClassAd job attribute!ResidentSetSize}
\index{ResidentSetSize!job ClassAd attribute}
\item[\AdAttr{ResidentSetSize}:]  Maximum observed
physical memory in use by the job in KiB while running.

%%% ClassAd attribute: StackSize
\index{ClassAd job attribute!StackSize}
\index{StackSize!job ClassAd attribute}
\item[\AdAttr{StackSize}:]   
Utilized for Linux jobs only, 
the number of bytes allocated for stack space for this job.
This number of bytes replaces the default allocation of 512 Mbytes.

%%% ClassAd attribute: StageOutFinish
\index{ClassAd job attribute!StageOutFinish}
\index{StageOutFinish!job ClassAd attribute}
\item[\AdAttr{StageOutFinish}:]   
An attribute representing a Unix epoch time that is defined for a job that is
spooled to a remote site using \Expr{condor\_submit -spool} or HTCondor-C
and causes HTCondor to hold the output in the spool while the job waits 
in the queue in the \Expr{Completed} state.
This attribute is defined when retrieval of the output finishes.

%%% ClassAd attribute: StageOutStart
\index{ClassAd job attribute!StageOutStart}
\index{StageOutStart!job ClassAd attribute}
\item[\AdAttr{StageOutStart}:]   
An attribute representing a Unix epoch time that is defined for a job that is
spooled to a remote site using \Expr{condor\_submit -spool} or HTCondor-C
and causes HTCondor to hold the output in the spool while the job waits 
in the queue in the \Expr{Completed} state.
This attribute is defined when retrieval of the output begins.

%%% ClassAd attribute: StreamErr
\index{ClassAd job attribute!StreamErr}
\index{StreamErr!job ClassAd attribute}
\item[\AdAttr{StreamErr}:]   
An attribute utilized only for grid universe jobs.
The default value is \Expr{True}.
If \Expr{True}, and \Attr{TransferErr} is \Expr{True}, then 
standard error is streamed back to the submit machine, instead
of doing the transfer (as a whole) after the job completes.
If \Expr{False}, then
standard error is transferred back to the submit machine
(as a whole) after the job completes.
If \Attr{TransferErr} is \Expr{False}, then this job attribute is ignored.

%%% ClassAd attribute: StreamOut
\index{ClassAd job attribute!StreamOut}
\index{StreamOut!job ClassAd attribute}
\item[\AdAttr{StreamOut}:]   
An attribute utilized only for grid universe jobs.
The default value is \Expr{True}.
If \Expr{True}, and \Attr{TransferOut} is \Expr{True}, then 
job output is streamed back to the submit machine, instead
of doing the transfer (as a whole) after the job completes.
If \Expr{False}, then
job output is transferred back to the submit machine
(as a whole) after the job completes.
If \Attr{TransferOut} is \Expr{False}, then this job attribute is ignored.

%%% ClassAd attribute: SubmitterAutoregroup
\index{ClassAd job attribute!SubmitterAutoregroup}
\index{SubmitterAutoregroup!job ClassAd attribute}
\item[\AdAttr{SubmitterAutoregroup}:]  A boolean attribute defined
by the \Condor{negotiator} when it makes a match. 
It will be \Expr{True} if the resource was claimed via negotiation
when the configuration variable \Macro{GROUP\_AUTOREGROUP} was \Expr{True}.
It will be \Expr{False} otherwise.

%%% ClassAd attribute: SubmitterGlobalJobId
\index{ClassAd job attribute!SubmitterGlobalJobId}
\index{SubmitterGlobalJobId!job ClassAd attribute}
\item[\AdAttr{SubmitterGlobalJobId}:]
When HTCondor-C submits a job to a remote \Condor{schedd}, it sets this
attribute in the remote job ad to match the \AdAttr{GlobalJobId} attribute
of the original, local job.

%%% ClassAd attribute: SubmitterGroup 
\index{ClassAd job attribute!SubmitterGroup}
\index{SubmitterGroup!job ClassAd attribute}
\item[\AdAttr{SubmitterGroup}:]  The accounting group name defined
by the \Condor{negotiator} when it makes a match. 

%%% ClassAd attribute: SubmitterNegotiatingGroup 
\index{ClassAd job attribute!SubmitterNegotiatingGroup}
\index{SubmitterNegotiatingGroup!job ClassAd attribute}
\item[\AdAttr{SubmitterNegotiatingGroup}:]  The accounting group name under
which the resource negotiated when it was claimed, 
as set by the \Condor{negotiator}. 

%%% ClassAd attribute: SuccessPreExitBySignal
\index{ClassAd job attribute!SuccessPreExitBySignal}
\index{SuccessPreExitBySignal!ClassAd job attribute}
\item[\AdAttr{SuccessPreExitBySignal}:] Specifies if a succesful pre command
must exit with a signal.

%%% ClassAd attribute: SuccessPreExitCode
\index{ClassAd job attribute!SuccessPreExitCode}
\index{SuccessPreExitCode!ClassAd job attribute}
\item[\AdAttr{SuccessPreExitCode}:] Specifies the code with which the pre
command must exit to be considered successful. Pre commands which are
not successful cause the job to go on hold with \AdAttr{ExitCode} set to
\Attr{PreCmdExitCode}. The exit status of a pre command without one of
\AdAttr{SuccessPreExitCode} or \AdAttr{SuccessPreExitSignal} defined is
ignored.

%%% ClassAd attribute: SuccessPreExitSignal
\index{ClassAd job attribute!SuccessPreExitSignal}
\index{SuccessPreExitSignal!ClassAd job attribute}
\item[\AdAttr{SuccessPreExitSignal}:] Specifies the signal on which the
pre command must exit be considered successful. Pre commands which are not
successful cause the job to go on hold with \AdAttr{ExitSignal} set to
\AdAttr{PreCmdExitSignal}. The exit status of a pre command without one of
\AdAttr{SuccessPreExitCode} or \AdAttr{SuccessPreExitSignal} defined is
ignored.

%%% ClassAd attribute: SuccessPostExitBySignal
\index{ClassAd job attribute!SuccessPostExitBySignal}
\index{SuccessPostExitBySignal!ClassAd job attribute}
\item[\AdAttr{SuccessPostExitBySignal}:] Specifies if a succesful post command
must exit with a signal.

%%% ClassAd attribute: SuccessPostExitCode
\index{ClassAd job attribute!SuccessPostExitCode}
\index{SuccessPostExitCode!ClassAd job attribute}
\item[\AdAttr{SuccessPostExitCode}:] Specifies the code with which the post
command must exit to be considered successful. Post commands which are
not successful cause the job to go on hold with \AdAttr{ExitCode} set to
\Attr{PostCmdExitCode}. The exit status of a post command without one of
\AdAttr{SuccessPostExitCode} or \AdAttr{SuccessPostExitSignal} defined is
ignored.

%%% ClassAd attribute: SuccessPostExitSignal
\index{ClassAd job attribute!SuccessPostExitSignal}
\index{SuccessPostExitSignal!ClassAd job attribute}
\item[\AdAttr{SuccessPostExitSignal}:] Specifies the signal on which the
post command must exit be considered successful. Post commands which are not
successful cause the job to go on hold with \AdAttr{ExitSignal} set to
\AdAttr{PostCmdExitSignal}. The exit status of a post command without one of
\AdAttr{SuccessPostExitCode} or \AdAttr{SuccessPostExitSignal} defined is
ignored.

%%% ClassAd attribute: TotalSuspensions
\index{ClassAd job attribute!TotalSuspensions}
\index{TotalSuspensions!job ClassAd attribute}
\item[\AdAttr{TotalSuspensions}:]  A count of the number of times this job
has been suspended during its lifetime.

%%% ClassAd attribute: TransferErr
\index{ClassAd job attribute!TransferErr}
\index{TransferErr!job ClassAd attribute}
\item[\AdAttr{TransferErr}:]   
An attribute utilized only for grid universe jobs.
The default value is \Expr{True}.
If \Expr{True}, then the error output from the job
is transferred from the remote machine back to the submit machine.
The name of the file after transfer is the file referred to
by job attribute \Attr{Err}.
If \Expr{False}, no transfer takes place (remote to submit machine),
and the name of the file is the file referred to
by job attribute \Attr{Err}.

%%% ClassAd attribute: TransferExecutable
\index{ClassAd job attribute!TransferExecutable}
\index{TransferExecutable!job ClassAd attribute}
\item[\AdAttr{TransferExecutable}:]   
An attribute utilized only for grid universe jobs.
The default value is \Expr{True}.
If \Expr{True}, then the job executable is transferred from the submit
machine to the remote machine.
The name of the file (on the submit machine)
that is transferred is given by the
job attribute \Attr{Cmd}.
If \Expr{False}, no transfer takes place, and
the name of the file used (on the remote machine) will be as
given in the job attribute \Attr{Cmd}.

%%% ClassAd attribute: TransferIn
\index{ClassAd job attribute!TransferIn}
\index{TransferIn!job ClassAd attribute}
\item[\AdAttr{TransferIn}:]   
An attribute utilized only for grid universe jobs.
The default value is \Expr{True}.
If \Expr{True}, then the job input is transferred from the submit
machine to the remote machine.
The name of the file that is transferred is given by the
job attribute \Attr{In}.
If \Expr{False}, then the job's input is taken from a file on the
remote machine (pre-staged), and 
the name of the file is given by the job attribute \Attr{In}.

%%% ClassAd attribute: TransferInputSizeMB
\index{ClassAd job attribute!TransferInputSizeMB}
\index{TransferInputSizeMB!job ClassAd attribute}
\item[\AdAttr{TransferInputSizeMB}:]
The total size in Mbytes of input files to be transferred for the
job.  Files transferred via file transfer plug-ins are not included.
This attribute is automatically set by \Condor{submit}; jobs submitted
via other submission methods, such as SOAP, may not define this
attribute.

%%% ClassAd attribute: TransferOut
\index{ClassAd job attribute!TransferOut}
\index{TransferOut!job ClassAd attribute}
\item[\AdAttr{TransferOut}:]   
An attribute utilized only for grid universe jobs.
The default value is \Expr{True}.
If \Expr{True}, then the output from the job
is transferred from the remote machine back to the submit machine.
The name of the file after transfer is the file referred to
by job attribute \Attr{Out}.
If \Expr{False}, no transfer takes place (remote to submit machine),
and the name of the file is the file referred to
by job attribute \Attr{Out}.

%%% ClassAd attribute: TransferringInput
\index{ClassAd job attribute!TransferringInput}
\index{TransferringInput!job ClassAd attribute}
\item[\AdAttr{TransferringInput}:]
A boolean value that indicates whether the job is currently
transferring input files.  The value is \Expr{Undefined} if the job is
not scheduled to run or has not yet attempted to start transferring
input.  When this value is \Expr{True}, to see whether the transfer is
active or queued, check \AdAttr{TransferQueued}.

%%% ClassAd attribute: TransferringOutput
\index{ClassAd job attribute!TransferringOutput}
\index{TransferringOutput!job ClassAd attribute}
\item[\AdAttr{TransferringOutput}:]
A boolean value that indicates whether the job is currently
transferring output files.  The value is \Expr{Undefined} if the job
is not scheduled to run or has not yet attempted to start transferring
output.  When this value is \Expr{True}, to see whether the transfer
is active or queued, check \AdAttr{TransferQueued}.

%%% ClassAd attribute: TransferQueued
\index{ClassAd job attribute!TransferQueued}
\index{TransferQueued!job ClassAd attribute}
\item[\AdAttr{TransferQueued}:]

A boolean value that indicates whether the job is currently waiting to
transfer files because of limits placed by
\Macro{MAX\_CONCURRENT\_DOWNLOADS} or
\Macro{MAX\_CONCURRENT\_UPLOADS}.

%%% ClassAd attribute: UserLog
\index{ClassAd job attribute!UserLog}
\index{UserLog!job ClassAd attribute}
\item[\AdAttr{UserLog}:] The full path and file name on the submit machine
of the log file of job events.   

\index{ClassAd job attribute!WantGracefulRemoval}
\item[\AdAttr{WantGracefulRemoval}:] A boolean expression that,
when \Expr{True}, specifies that a graceful shutdown of the job
should be done when the job is removed or put on hold.

\index{ClassAd job attribute!WindowsBuildNumber}
\item[\AdAttr{WindowsBuildNumber}:] An integer, extracted from the
platform type of the machine upon which this job is submitted,
representing a build number for a Windows operating system.
This attribute only exists for jobs submitted from Windows machines.

\index{ClassAd job attribute!WindowsMajorVersion}
\item[\AdAttr{WindowsMajorVersion}:] An integer, extracted from the
platform type of the machine upon which this job is submitted,
representing a major version number (currently 5 or 6)
for a Windows operating system.
This attribute only exists for jobs submitted from Windows machines.

\index{ClassAd job attribute!WindowsMinorVersion}
\item[\AdAttr{WindowsMinorVersion}:] An integer, extracted from the
platform type of the machine upon which this job is submitted, 
representing a minor version number (currently 0, 1, or 2)
for a Windows operating system.
This attribute only exists for jobs submitted from Windows machines.

%%% ClassAd attribute: X509UserProxy
\index{ClassAd job attribute!X509UserProxy}
\index{X509UserProxy!job ClassAd attribute}
\item[\AdAttr{X509UserProxy}:]   
The full path and file name of the file containing the X.509 user proxy.

%%% ClassAd attribute: X509UserProxyEmail
\index{ClassAd job attribute!X509UserProxyEmail}
\index{X509UserProxyEmail!job ClassAd attribute}
\item[\AdAttr{X509UserProxyEmail}:]   
\item For a job with an X.509 proxy credential, this is the email
address extracted from the proxy.

%%% ClassAd attribute: X509UserProxyExpiration
\index{ClassAd job attribute!X509UserProxyExpiration}
\index{X509UserProxyExpiration!job ClassAd attribute}
\item[\AdAttr{X509UserProxyExpiration}:]   
For a job that defines the submit description file command
\SubmitCmd{x509userproxy}, this is the time at which the indicated
X.509 proxy credential will expire, measured in the
number of seconds since the epoch (00:00:00 UTC, Jan 1, 1970).

%%% ClassAd attribute: X509UserProxyFirstFQAN
\index{ClassAd job attribute!X509UserProxyFirstFQAN}
\index{X509UserProxyFirstFQAN!job ClassAd attribute}
\item[\AdAttr{X509UserProxyFirstFQAN}:]   
For a vanilla or grid universe job that defines the submit description
file command \SubmitCmd{x509userproxy}, 
this is the VOMS Fully Qualified Attribute Name (FQAN) of
the primary role of the credential. 
A credential may have multiple roles defined, 
but by convention the one listed first is the primary role. 

%%% ClassAd attribute: X509UserProxyFQAN
\index{ClassAd job attribute!X509UserProxyFQAN}
\index{X509UserProxyFQAN!job ClassAd attribute}
\item[\AdAttr{X509UserProxyFQAN}:]   
For a vanilla or grid universe job that defines the submit description
file command \SubmitCmd{x509userproxy}, 
this is a serialized list of the DN and all FQAN.
A comma is used as a separator,
and any existing commas in the DN or FQAN are replaced with the string
\Expr{\&comma;}.
Likewise, any ampersands in the DN or FQAN are replaced with
\Expr{\&amp;}.

%%% ClassAd attribute: X509UserProxySubject
\index{ClassAd job attribute!X509UserProxySubject}
\index{X509UserProxySubject!job ClassAd attribute}
\item[\AdAttr{X509UserProxySubject}:]   
For a vanilla or grid universe job that defines the submit description
file command \SubmitCmd{x509userproxy}, 
this attribute contains the Distinguished Name (DN) of the credential
used to submit the job.

%%% ClassAd attribute: X509UserProxyVOName
\index{ClassAd job attribute!X509UserProxyVOName}
\index{X509UserProxyVOName!job ClassAd attribute}
\item[\AdAttr{X509UserProxyVOName}:]   
For a vanilla or grid universe job that defines the submit description
file command \SubmitCmd{x509userproxy}, 
this is the name of the VOMS virtual organization (VO) that 
the user's credential is part of. 

\end{description}

The following job ClassAd attributes are relevant only for
\SubmitCmdNI{vm} universe jobs.

\begin {description}
%%% ClassAd attribute: VM_Checkpoint
%\index{ClassAd job attribute!VM\_Checkpoint}
%\item[\AdAttr{VM\_Checkpoint}:] Definition here.

%%% ClassAd attribute: VM_MACAddr
\index{ClassAd job attribute!VM\_MACAddr}
\item[\AdAttr{VM\_MACAddr}:] The MAC address of the virtual
machine's network interface,
in the standard format of six groups of
two hexadecimal digits separated by colons.
This attribute is currently limited to apply only to Xen virtual machines.

%%% ClassAd attribute: VM_Memory
%\index{ClassAd job attribute!VM\_Memory}
%\item[\AdAttr{VM\_Memory}:] Definition here.

%%% ClassAd attribute: VM_Networking
%\index{ClassAd job attribute!VM\_Networking}
%\item[\AdAttr{VM\_Networking}:] Definition here.

%%% ClassAd attribute: VM_Networking_Type
%\index{ClassAd job attribute!VM\_Networking\_Type}
%\item[\AdAttr{VM\_Networking\_Type}:] Definition here.

%%% ClassAd attribute: VM_Type
%\index{ClassAd job attribute!VM\_Type}
%\item[\AdAttr{VM\_Type}:] Definition here.

%%% ClassAd attribute: VM_VCPUS
%\index{ClassAd job attribute!VM\_VCPUS}
%\item[\AdAttr{VM\_VCPUS}:] Definition here.

\end{description}

\label{Job-ClassAd-DAGManAttributes}
The following job ClassAd attributes appear in the job ClassAd
only for the \Condor{dagman}
job submitted under DAGMan.
They represent status information for the DAG.

\begin {description}

%%% ClassAd attribute: DAG_InRecovery
\index{ClassAd job attribute!DAG\_InRecovery}
\index{DAG\_InRecovery!job ClassAd attribute}
\label{AttrDAGInRecovery}
\item[\AdAttr{DAG\_InRecovery}:]   
The value 1 if the DAG is in recovery mode, and
The value 0 otherwise. 

%%% ClassAd attribute: DAG_NodesDone
\index{ClassAd job attribute!DAG\_NodesDone}
\index{DAG\_NodesDone!job ClassAd attribute}
\label{AttrDAGNodesDone}
\item[\AdAttr{DAG\_NodesDone}:]   
The number of DAG nodes that have finished successfully.
This means that the entire node has finished, 
not only an actual HTCondor job or jobs.

%%% ClassAd attribute: DAG_NodesFailed
\index{ClassAd job attribute!DAG\_NodesFailed}
\index{DAG\_NodesFailed!job ClassAd attribute}
\label{AttrDAGNodesFailed}
\item[\AdAttr{DAG\_NodesFailed}:]
The number of DAG nodes that have failed.
This value includes all retries, if there are any.

%%% ClassAd attribute: DAG_NodesPostrun
\index{ClassAd job attribute!DAG\_NodesPostrun}
\index{DAG\_NodesPostrun!job ClassAd attribute}
\label{AttrDAGNodesPostrun}
\item[\AdAttr{DAG\_NodesPostrun}:]
The number of DAG nodes for which a POST script is running 
or has been deferred because of a POST script throttle setting. 

%%% ClassAd attribute: DAG_NodesPrerun
\index{ClassAd job attribute!DAG\_NodesPrerun}
\index{DAG\_NodesPrerun!job ClassAd attribute}
\label{AttrDAGNodesPrerun}
\item[\AdAttr{DAG\_NodesPrerun}:]
The number of DAG nodes for which a PRE script is running 
or has been deferred because of a PRE script throttle setting. 

%%% ClassAd attribute: DAG_NodesQueued
\index{ClassAd job attribute!DAG\_NodesQueued}
\index{DAG\_NodesQueued!job ClassAd attribute}
\label{AttrDAGNodesQueued}
\item[\AdAttr{DAG\_NodesQueued}:]
The number of DAG nodes for which the actual HTCondor job or jobs 
are queued.
The queued jobs may be in any state.

%%% ClassAd attribute: DAG_NodesReady
\index{ClassAd job attribute!DAG\_NodesReady}
\index{DAG\_NodesReady!job ClassAd attribute}
\label{AttrDAGNodesReady}
\item[\AdAttr{DAG\_NodesReady}:]
The number of DAG nodes that are ready to run,
but which have not yet started running.

%%% ClassAd attribute: DAG_NodesTotal
\index{ClassAd job attribute!DAG\_NodesTotal}
\index{DAG\_NodesTotal!job ClassAd attribute}
\label{AttrDAGNodesTotal}
\item[\AdAttr{DAG\_NodesTotal}:]   
The total number of nodes in the DAG, including the FINAL node, if there
is a FINAL node.

%%% ClassAd attribute: DAG_NodesUnready
\index{ClassAd job attribute!DAG\_NodesUnready}
\index{DAG\_NodesUnready!job ClassAd attribute}
\label{AttrDAGNodesUnready}
\item[\AdAttr{DAG\_NodesUnready}:]   
The number of DAG nodes that are not ready to run.
This is a node in which one or more of the parent nodes has not yet finished. 

%%% ClassAd attribute: DAG_Status
\index{ClassAd job attribute!DAG\_Status}
\index{DAG\_Status!job ClassAd attribute}
\label{AttrDAGStatus}
\item[\AdAttr{DAG\_Status}:]   
The overall status of the DAG, with the same values as the macro
\Env{\$DAG\_STATUS} used in DAGMan FINAL nodes.
\begin{center}
\begin{table}[hbt]
\begin{tabular}{|p{2cm}p{10cm}|} \hline
\emph{Value} & \emph{Status} \\ \hline \hline
0 & OK \\ \hline
1 & error; an error condition different than those listed here \\ \hline
2 & one or more nodes in the DAG have failed \\ \hline
3 & the DAG has been aborted by an ABORT-DAG-ON specification  \\ \hline
4 & removed; the DAG has been removed by \Condor{rm} \\ \hline
5 & a cycle was found in the DAG \\ \hline
6 & the DAG has been suspended (see section~\ref{sec:DagSuspend}) \\ \hline
\end{tabular}
\end{table}
\end{center}
\end{description}

The following job ClassAd attributes do \emph{not} appear in the 
job ClassAd as kept by the \Condor{schedd} daemon.
They appear in the job ClassAd written to the job's execute directory
while the job is running.

\begin{description}

%%% ClassAd attribute: CpusProvisioned
\index{ClassAd job attribute!CpusProvisioned}
\index{CpusProvisioned!job ClassAd attribute}
\label{CpusProvisioned}
\item[\AdAttr{CpusProvisioned}:]   
The number of Cpus allocated to the job.
With statically-allocated slots, it is the number of Cpus allocated to
the slot.
With dynamically-allocated slots, it is based upon the job attribute
\Attr{RequestCpus}, but may be larger due to the minimum given to
a dynamic slot.

%%% ClassAd attribute: DiskProvisioned
\index{ClassAd job attribute!DiskProvisioned}
\index{DiskProvisioned!job ClassAd attribute}
\label{DiskProvisioned}
\item[\AdAttr{DiskProvisioned}:]   
The amount of disk space in KiB allocated to the job.
With statically-allocated slots, it is the amount of disk space allocated to
the slot.
With dynamically-allocated slots, it is based upon the job attribute
\Attr{RequestDisk}, but may be larger due to the minimum given to
a dynamic slot.

%%% ClassAd attribute: MemoryProvisioned
\index{ClassAd job attribute!MemoryProvisioned}
\index{MemoryProvisioned!job ClassAd attribute}
\label{MemoryProvisioned}
\item[\AdAttr{MemoryProvisioned}:]   
The amount of memory in MiB allocated to the job.
With statically-allocated slots, it is the amount of memory space allocated to
the slot.
With dynamically-allocated slots, it is based upon the job attribute
\Attr{RequestMemory}, but may be larger due to the minimum given to
a dynamic slot.

%%% ClassAd attribute: <Name>Provisioned
\index{ClassAd job attribute!<Name>Provisioned}
\index{<Name>Provisioned!job ClassAd attribute}
\label{<Name>Provisioned}
\item[\AdAttr{<Name>Provisioned}:]   
The amount of the custom resource identified by \Attr{<Name>} allocated to 
the job.  
For jobs using GPUs, \Attr{<Name>} will be \Expr{GPUs}.
With statically-allocated slots, 
it is the amount of the resource allocated to the slot.
With dynamically-allocated slots, it is based upon the job attribute
\Attr{Request<Name>}, but may be larger due to the minimum given to
a dynamic slot.

\end{description}
