\begin{description}
%
\index{ClassAd!machine attributes}
\index{ClassAd machine attribute!Activity}
\item[\AdAttr{Activity}:] String which describes HTCondor job activity on the machine.
Can have one of the following values:
	\begin{description}
	\item[\AdStr{Idle}:] There is no job activity
	\item[\AdStr{Busy}:] A job is busy running
	\item[\AdStr{Suspended}:] A job is currently suspended
	\item[\AdStr{Vacating}:] A job is currently checkpointing
	\item[\AdStr{Killing}:] A job is currently being killed
	\item[\AdStr{Benchmarking}:] The startd is running benchmarks
	\item[\AdStr{Retiring}:] Waiting for a job to finish or for the maximum retirement time to expire
	\end{description}
%
\index{ClassAd machine attribute!Arch}
\label{Arch-machine-attribute}
\item[\AdAttr{Arch}:] String with the architecture of the machine.  
Currently supported architectures have the following string
definitions:
	\begin{description}
	\item[\AdStr{INTEL}:] Intel x86 CPU (Pentium, Xeon, etc).
	\item[\AdStr{X86\_64}:] AMD/Intel 64-bit X86
	\end{description}
These strings show definitions for architectures no longer supported:
	\begin{description}
	\item[\AdStr{IA64}:] Intel Itanium
	\item[\AdStr{SUN4u}:] Sun UltraSparc CPU
	\item[\AdStr{SUN4x}:] A Sun Sparc CPU other than an UltraSparc, i.e.
sun4m or sun4c CPU found in older Sparc workstations such as the Sparc~10, 
Sparc~20, IPC, IPX, etc.
	\item[\AdStr{PPC}:] 32-bit PowerPC
	\item[\AdStr{PPC64}:] 64-bit PowerPC
	\end{description}
%
\index{ClassAd machine attribute!CanHibernate}
\item[\AdAttr{CanHibernate}:] The \Condor{startd} has the capability to 
shut down or hibernate a machine when certain configurable criteria are met.
However, before the \Condor{startd} can shut down a machine, 
the hardware itself must support hibernation, as must the operating system. 
When the \Condor{startd} initializes, 
it checks for this support.
If the machine has the ability to hibernate, 
then this boolean ClassAd attribute will be \Expr{True}.
By default, it is \Expr{False}.
%
\index{ClassAd machine attribute!CheckpointPlatform}
\label{CheckpointPlatform-machine-attribute}
\item[\AdAttr{CheckpointPlatform}:] A string which opaquely encodes various
aspects about a machine's operating system, hardware, and kernel
attributes.
It is used to identify systems where previously taken checkpoints for
the standard universe may resume.
%
\index{ClassAd machine attribute!ClockDay}
\item[\AdAttr{ClockDay}:] The day of the week, 
where 0 = Sunday, 1 = Monday, \Dots, and 6 = Saturday. 
%
\index{ClassAd machine attribute!ClockMin}
\item[\AdAttr{ClockMin}:] The number of minutes passed since midnight.
%
\index{ClassAd machine attribute!CondorLoadAvg}
\item[\AdAttr{CondorLoadAvg}:] The load average contributed  
by HTCondor, either from remote jobs or running benchmarks.
%
\index{ClassAd machine attribute!CondorVersion}
\item[\AdAttr{CondorVersion}:] A string containing the HTCondor version
number for the \Condor{startd} daemon, the release date, and the build
identification number.
%
\index{ClassAd machine attribute!ConsoleIdle}
\item[\AdAttr{ConsoleIdle}:] The number of seconds since activity on the system
console keyboard or console mouse has last been detected.
The value can be modified with \Macro{SLOTS\_CONNECTED\_TO\_CONSOLE}
as defined at ~\ref{param:SlotsConnectedToConsole}.
%
\index{ClassAd machine attribute!Cpus}
\item[\AdAttr{Cpus}:]  The number of CPUs in this slot.
It is 1 for a single CPU slot, 2 for a dual CPU slot, etc.
%
\index{ClassAd machine attribute!CurrentRank}
\item[\AdAttr{CurrentRank}:] A float which represents this machine
owner's affinity
for running the HTCondor job which it is currently hosting.  If not
currently hosting an HTCondor job, \AdAttr{CurrentRank} is 0.0.
When a machine is claimed,
the attribute's value is computed by evaluating the machine's
\AdAttr{Rank} expression with respect to the current job's ClassAd.
%
\index{ClassAd machine attribute!Disk}
\item[\AdAttr{Disk}:] The amount of disk space on this machine available for
the job in Kbytes ( e.g. 23000 = 23 megabytes ).  Specifically, this
is the amount of disk space available in the directory specified in
the HTCondor configuration files by the \Macro{EXECUTE} macro, minus any
space reserved with the \Macro{RESERVED\_DISK} macro.
%
\index{ClassAd machine attribute!Draining}
\item[\AdAttr{Draining}:] This attribute is \Expr{True} when the slot
is draining and undefined if not.
%
\index{ClassAd machine attribute!DrainingRequestId}
\item[\AdAttr{DrainingRequestId}:] This attribute contains a string that
is the request id of the draining request that put this slot in a draining
state.  It is undefined if the slot is not draining.
%
\index{ClassAd machine attribute!DotNetVersions}
\item[\AdAttr{DotNetVersions}:] The .NET framework versions
currently installed on this computer. 
Default format is a comma delimited list. 
Current definitions:
  \begin{description}
  \item[\AdStr{1.1}:] for .Net Framework 1.1
  \item[\AdStr{2.0}:] for .Net Framework 2.0
  \item[\AdStr{3.0}:] for .Net Framework 3.0
  \item[\AdStr{3.5}:] for .Net Framework 3.5
  \item[\AdStr{4.0Client}:] for .Net Framework 4.0 Client install
  \item[\AdStr{4.0Full}:] for .Net Framework 4.0 Full install
  \end{description}
%
\index{ClassAd machine attribute!DynamicSlot}
\label{DynamicSlot-machine-attribute} 
\item[\AdAttr{DynamicSlot}:] For SMP machines that allow dynamic
partitioning of a slot,
this boolean value identifies that this dynamic slot may be partitioned.
%
\index{ClassAd machine attribute!EnteredCurrentActivity}
\item[\AdAttr{EnteredCurrentActivity}:] Time at which the machine
entered the current Activity (see \AdAttr{Activity} entry above).  On
all platforms (including NT), this is measured in the number of
integer seconds since the Unix epoch (00:00:00 UTC, Jan 1, 1970).
%
\index{ClassAd machine attribute!ExpectedMachineGracefulDrainingBadput}
\item[\AdAttr{ExpectedMachineGracefulDrainingBadput}:] The
job run time in cpu-seconds that would be lost if graceful draining
were initiated at the time this ClassAd was published.  
This calculation assumes
that jobs will run for the full retirement time and then be evicted
without saving a checkpoint.
%
\index{ClassAd machine attribute!ExpectedMachineGracefulDrainingCompletion}
\item[\AdAttr{ExpectedMachineGracefulDrainingCompletion}:] The
estimated time at
which graceful draining of the machine could complete if it were
initiated at the time this ClassAd was published
and there are no active claims.  This is measured in the
number of integer seconds since the Unix epoch (00:00:00 UTC, Jan 1,
1970).  This value is computed with the assumption that the machine
policy will not suspend jobs during draining while the machine is
waiting for the job to use up its retirement time.  If suspension
happens, the upper bound on how long draining could take is
unlimited.  To avoid suspension during draining, the \MacroNI{SUSPEND}
and \MacroNI{CONTINUE} expressions could be configured to pay
attention to the \AdAttr{Draining} attribute.
%
\index{ClassAd machine attribute!ExpectedMachineQuickDrainingBadput}
\item[\AdAttr{ExpectedMachineGracefulQuickBadput}:] The
job run time in cpu-seconds that would be lost if quick or fast draining
were initiated at the time this ClassAd was published.  This calculation assumes
that all evicted jobs will not save a checkpoint.
%
\index{ClassAd machine attribute!ExpectedMachineQuickDrainingCompletion}
\item[\AdAttr{ExpectedMachineQuickDrainingCompletion}:] Time at
which quick or fast draining of the machine could complete if it were
initiated at the time this ClassAd was published 
and there are no active claims.  This is measured in the
number of integer seconds since the Unix epoch (00:00:00 UTC, Jan 1,
1970).
%
\index{ClassAd machine attribute!FileSystemDomain}
\item[\AdAttr{FileSystemDomain}:] A domain name configured by the
HTCondor administrator which describes a cluster of machines which all
access the same, uniformly-mounted, networked file systems usually via
NFS or AFS.  This is useful for Vanilla universe jobs which require
remote file access.
%
\index{ClassAd machine attribute!Has\_sse4\_1}
\item[\AdAttr{Has\_sse4\_1}:] A boolean value set to \Expr{True}
 if the machine being advertised supports
the SSE 4.1 instructions, and \Expr{Undefined} otherwise.
%
\index{ClassAd machine attribute!Has\_sse4\_2}
\item[\AdAttr{Has\_sse4\_2}:] A boolean value set to \Expr{True}
if the machine being advertised supports
the SSE 4.2 instructions, and \Expr{Undefined} otherwise.
%
\index{ClassAd machine attribute!has\_ssse3}
\item[\AdAttr{has\_ssse3}:] A boolean value set to \Expr{True}
if the machine being advertised supports
the SSSE 3 instructions, and \Expr{Undefined} otherwise.
%
\index{ClassAd machine attribute!HasVM}
\item[\AdAttr{HasVM}:] A boolean value added to the machine ClassAd
when the configuration triggers the detection of virtual machine
software.
%
\index{ClassAd machine attribute!IsWakeAble}
\item[\AdAttr{IsWakeAble}:] A boolean value that when \Expr{True} identifies
that the machine has the capability to be woken into a 
fully powered and running state by receiving a Wake On LAN (WOL) packet.
This ability is a function of the operating system, 
the network adapter in the machine 
(notably, wireless network adapters usually do not have this function),
and BIOS settings. 
When the \Condor{startd} initializes, 
it tries to detect if the operating system and network adapter both support 
waking from hibernation by receipt of a WOL packet.
The default value is \Expr{False}.
%
\index{ClassAd machine attribute!IsWakeEnabled}
\item[\AdAttr{IsWakeEnabled}:] If the hardware and software have the capacity 
to be woken into a fully powered and running state by receiving 
a Wake On LAN (WOL) packet,
this feature can still be disabled via the BIOS or software.
If BIOS or the operating system have disabled this feature, 
the \Condor{startd} sets this boolean attribute to \Expr{False}.
%
\index{ClassAd machine attribute!JobPreemptions}
\item[\AdAttr{JobPreemptions}:] The total number of times
a running job has been preempted on this machine.  
%
\index{ClassAd machine attribute!JobRankPreemptions}
\item[\AdAttr{JobRankPreemptions}:] The total number of times
a running job has been preempted on this machine due to the machine's
rank of jobs since the \Condor{startd} started running.  
%
\index{ClassAd machine attribute!JobStarts}
\item[\AdAttr{JobStarts}:] The total number of jobs which
have been started on this machine since the \Condor{startd} started running.
%
\index{ClassAd machine attribute!JobUserPrioPreemptions}
\item[\AdAttr{JobUserPrioPreemptions}:] The total number of times
a running job has been preempted on this machine based on a fair share
allocation of the pool 
since the \Condor{startd} started running.  
%
\index{ClassAd machine attribute!JobVM\_VCPUS}
\item[\AdAttr{JobVM\_VCPUS}:] An attribute defined if a vm universe job
is running on this slot.  Defined by the number of virtualized CPUs
in the virtual machine.
%
\index{ClassAd machine attribute!KeyboardIdle}
\item[\AdAttr{KeyboardIdle}:] The number of seconds since activity on any
keyboard or mouse associated with this machine has last been detected.
Unlike \AdAttr{ConsoleIdle}, \AdAttr{KeyboardIdle} also takes activity 
on pseudo-terminals into
account.
Pseudo-terminals have virtual keyboard activity from telnet and rlogin
sessions.  Note that \AdAttr{KeyboardIdle} will always be equal to or
less than \AdAttr{ConsoleIdle}.
The value can be modified with \Macro{SLOTS\_CONNECTED\_TO\_KEYBOARD}
as defined at ~\ref{param:SlotsConnectedToKeyboard}.
%
\index{ClassAd machine attribute!KFlops}
\item[\AdAttr{KFlops}:] Relative floating point performance as determined via a
Linpack benchmark.
%
\index{ClassAd machine attribute!LastDrainStartTime}
\item[\AdAttr{LastDrainStartTime}:] Time when draining of this
\Condor{startd} was last initiated (e.g. due to \Condor{defrag} or
\Condor{drain}).
%
\index{ClassAd machine attribute!LastHeardFrom}
\item[\AdAttr{LastHeardFrom}:] Time when the HTCondor central manager last
received a status update from this machine.  
Expressed as 
the number of integer seconds since the Unix epoch (00:00:00 UTC, Jan 1, 1970).
Note: This attribute is only inserted by the central manager once it
receives the ClassAd.
It is not present in the \Condor{startd} copy of the ClassAd.
Therefore, you could not use this attribute in defining \Condor{startd}
expressions (and you would not want to).
%
\index{ClassAd machine attribute!LoadAvg}
\item[\AdAttr{LoadAvg}:] A floating point number representing the 
current load average.
%
\index{ClassAd machine attribute!Machine}
\item[\AdAttr{Machine}:] A string with the machine's fully qualified host name.
%
\index{ClassAd machine attribute!MachineMaxVacateTime}
\item[\AdAttr{MachineMaxVacateTime}:] An integer expression that specifies
the time in seconds the machine will allow the job to gracefully shut
down.
%
\index{ClassAd machine attribute!MaxJobRetirementTime}
\item[\AdAttr{MaxJobRetirementTime}:] When the \Condor{startd} wants
to kick the job off, a job which has run for less than this number
of seconds will not be hard-killed.  The \Condor{startd} will wait
for the job to finish or to exceed this amount of time, whichever
comes sooner.  If the job vacating policy grants the job X seconds
of vacating time, a preempted job will be soft-killed X seconds
before the end of its retirement time, so that hard-killing of the
job will not happen until the end of the retirement time if the job
does not finish shutting down before then.  This is an expression
evaluated in the context of the job ClassAd, so it may refer to job
attributes as well as machine attributes.
%
\index{ClassAd machine attribute!Memory}
\item[\AdAttr{Memory}:] The amount of RAM in megabytes.
%
\index{ClassAd machine attribute!Mips}
\item[\AdAttr{Mips}:] Relative integer performance as determined via a Dhrystone
benchmark.

\index{ClassAd machine attribute!MonitorSelfAge}
\item[\AdAttr{MonitorSelfAge}:] The number of seconds that this daemon
  has been running.

\index{ClassAd machine attribute!MonitorSelfCPUUsage}
\item[\AdAttr{MonitorSelfCPUUsage}:] The fraction of recent CPU time utilized
  by this daemon. 

\index{ClassAd machine attribute!MonitorSelfImageSize}
\item[\AdAttr{MonitorSelfImageSize}:] The amount of virtual memory consumed by
  this daemon in Kbytes.

\index{ClassAd machine attribute!MonitorSelfRegisteredSocketCount}
\item[\AdAttr{MonitorSelfRegisteredSocketCount}:] The current number of sockets
  registered by this daemon.

\index{ClassAd machine attribute!MonitorSelfResidentSetSize}
\item[\AdAttr{MonitorSelfResidentSetSize}:] The amount of resident memory
  used by this daemon in Kbytes.

\index{ClassAd machine attribute!MonitorSelfSecuritySessions}
\item[\AdAttr{MonitorSelfSecuritySessions}:] The number of open (cached)
  security sessions for this daemon.

\index{ClassAd machine attribute!MonitorSelfTime}
\item[\AdAttr{MonitorSelfTime}:] The  time, represented as the number of
  second elapsed since the Unix epoch (00:00:00 UTC, Jan 1, 1970),
  at which this daemon last checked and set the attributes with names that
  begin with the string \Attr{MonitorSelf}.
  
\index{ClassAd machine attribute!MyAddress}
\item[\AdAttr{MyAddress}:] String with the IP and port address of the
\Condor{startd} daemon which is publishing this machine ClassAd.
When using CCB, \Condor{shared\_port}, and/or an additional private
network interface, that information will be included here as well.

\index{ClassAd machine attribute!MyType}
\item[\AdAttr{MyType}:] The ClassAd type; always set to the literal string \AdStr{Machine}.
%
\index{ClassAd machine attribute!Name}
\item[\AdAttr{Name}:] The name of this resource; typically the same value as
the \AdAttr{Machine} attribute, but could be customized by the site
administrator.
On SMP machines, the \Condor{startd} will divide the CPUs up into separate
slots, each with with a unique name.
These names will be of the form ``slot\#@full.hostname'', for example,
``slot1@vulture.cs.wisc.edu'', which signifies slot number 1 from
vulture.cs.wisc.edu.
%
\index{ClassAd machine attribute!OpSys}
\label{OpSys-machine-attribute}
\item[\AdAttr{OpSys}:] String describing the operating system running on this
machine.  
Currently supported operating systems have the following string
definitions:
	\begin{description}
	\item[\AdStr{LINUX}:] for LINUX 2.0.x, LINUX 2.2.x,
	LINUX 2.4.x, or LINUX 2.6.x kernel systems, as well as Scientific Linux 
        and Ubuntu 12.04
	\item[\AdStr{OSX}:] for Darwin
	\item[\AdStr{FREEBSD7}:] for FreeBSD 7
	\item[\AdStr{FREEBSD8}:] for FreeBSD 8
	\item[\AdStr{WINDOWS}:] for all versions of Windows
	\item[\AdStr{SOLARIS5.10}:] for Solaris 2.10 or 5.10
	\item[\AdStr{SOLARIS5.11}:] for Solaris 2.11 or 5.11
	\end{description}
These strings show definitions for operating systems no longer supported:
	\begin{description}
	\item[\AdStr{SOLARIS28}:] for Solaris 2.8 or 5.8
	\item[\AdStr{SOLARIS29}:] for Solaris 2.9 or 5.9
	\end{description}
%
\index{ClassAd machine attribute!OpSysAndVer}
\item[\AdAttr{OpSysAndVer}:] A string indicating an operating system and
a version number.

For Linux operating systems, it is the value of the \Attr{OpSysName} attribute 
concatenated with the string version of the \Attr{OpSysMajorVersion} attribute:
	\begin{description}
	\item[\AdStr{RedHat5}:] for RedHat Linux version 5
	\item[\AdStr{RedHat6}:] for RedHat Linux version 6
	\item[\AdStr{Fedora16}:] for Fedora Linux version 16
	\item[\AdStr{Debian5}:] for Debian Linux version 5
	\item[\AdStr{Debian6}:] for Debian Linux version 6
	\item[\AdStr{Ubuntu12}:] for Ubuntu 12.04
	\item[\AdStr{SL5}:] for Scientific Linux version 5
	\item[\AdStr{SL6}:] for Scientific Linux version 6
	\item[\AdStr{SLFermi5}:] for Fermi's Scientific Linux version 5
	\item[\AdStr{SLFermi6}:] for Fermi's Scientific Linux version 6
	\item[\AdStr{SLCern5}:] for CERN's Scientific Linux version 5
	\item[\AdStr{SLCern6}:] for CERN's Scientific Linux version 6
	\end{description}
For MacOS operating systems, it is the value of the \Attr{OpSysShortName} 
attribute concatenated with the string version of the \Attr{OpSysVer} attribute: 
	\begin{description}
	\item[\AdStr{MacOSX605}:] for MacOS version 10.6.5 (Snow Leopard)
	\item[\AdStr{MacOSX703}:] for MacOS version 10.7.3 (Lion)
	\end{description}
For BSD operating systems, it is the value of the \Attr{OpSysName} attribute 
concatenated with the string version of the \Attr{OpSysMajorVersion} attribute:
	\begin{description}
	\item[\AdStr{FREEBSD7}:] for FreeBSD version 7
	\item[\AdStr{FREEBSD8}:] for FreeBSD version 8
	\end{description}
For Solaris Unix operating systems, 
it is the same value as the \Attr{OpSys} attribute: 
	\begin{description}
	\item[\AdStr{SOLARIS5.10}:] for Solaris 2.10 or 5.10
	\item[\AdStr{SOLARIS5.11}:] for Solaris 2.11 or 5.11
	\end{description}
For Windows operating systems, it is the value of the \Attr{OpSys} attribute 
concatenated with the string version of the \Attr{OpSysMajorVersion} attribute:
	\begin{description}
	\item[\AdStr{WINDOWS500}:] for Windows 2000
	\item[\AdStr{WINDOWS501}:] for Windows XP
	\item[\AdStr{WINDOWS502}:] for Windows Server 2003
	\item[\AdStr{WINDOWS600}:] for Windows Vista
	\item[\AdStr{WINDOWS601}:] for Windows 7
	\end{description}
%
\index{ClassAd machine attribute!OpSysLegacy}
\item[\AdAttr{OpSysLegacy}:] A string that holds the long-standing values for the \Attr{OpSys} attribute.
Currently supported operating systems have the following string
definitions:
	\begin{description}
	\item[\AdStr{LINUX}:] for LINUX 2.0.x, LINUX 2.2.x, LINUX 2.4.x, or LINUX 2.6.x kernel systems, as well as Scientific Linux and  Ubuntu 12.04 versions
	\item[\AdStr{OSX}:] for Darwin
	\item[\AdStr{FREEBSD7}:] for FreeBSD version 7
	\item[\AdStr{FREEBSD8}:] for FreeBSD version 8
	\item[\AdStr{SOLARIS5.10}:] for Solaris 2.10 or 5.10
	\item[\AdStr{SOLARIS5.11}:] for Solaris 2.11 or 5.11
	\item[\AdStr{WINDOWS}:] for all versions of Windows
	\end{description}
%
\index{ClassAd machine attribute!OpSysLongName}
\item[\AdAttr{OpSysLongName}:] A string giving a full description of 
the operating system.
For Linux platforms, this is generally the string taken from \File{/etc/hosts},
with extra characters stripped off Debian versions.
	\begin{description}
	\item[\AdStr{Red Hat Enterprise Linux Server release 5.7 (Tikanga)}:] for RedHat Linux version 5
	\item[\AdStr{Red Hat Enterprise Linux Server release 6.2 (Santiago)}:] for RedHat Linux version 6
	\item[\AdStr{Ubuntu 12.04.3 LTS}:] for Ubuntu 12.04 point release 3 
	\item[\AdStr{Fedora release 16 (Verne)}:] for Fedora Linux version 16
	\item[\AdStr{MacOSX 6.5}:] for MacOS version 10.6.5 (Snow Leopard)
	\item[\AdStr{MacOSX 7.3}:] for MacOS version 10.7.3 (Lion)
	\item[\AdStr{FreeBSD8.2-RELEASE-p3}:] for FreeBSD version 8
	\item[\AdStr{SOLARIS5.10}:] for Solaris 2.10 or 5.10
	\item[\AdStr{SOLARIS5.11}:] for Solaris 2.11 or 5.11
	\item[\AdStr{Windows XP SP3}:] for Windows XP
	\item[\AdStr{Windows 7 SP2}:] for Windows 7
	\end{description}
%
\index{ClassAd machine attribute!OpSysMajorVersion}
\item[\AdAttr{OpSysMajorVersion}:] An integer value representing the major version of the operating system.
	\begin{description}
	\item[\Expr{5}:] for RedHat Linux version 5 
and derived platforms such as Scientific Linux
	\item[\Expr{6}:] for RedHat Linux version 6
and derived platforms such as Scientific Linux
	\item[\Expr{12}:] for Ubuntu 12.04
	\item[\Expr{16}:] for Fedora Linux version 16
	\item[\Expr{6}:] for MacOS version 10.6.5 (Snow Leopard)
	\item[\Expr{7}:] for MacOS version 10.7.3 (Lion)
	\item[\Expr{7}:] for FreeBSD version 7
	\item[\Expr{8}:] for FreeBSD version 8
	\item[\Expr{5}:] for Solaris 2.10, 5.10, 2.11, or 5.11
	\item[\Expr{501}:] for Windows XP
	\item[\Expr{600}:] for Windows Vista
	\item[\Expr{601}:] for Windows 7
	\end{description}
%
\index{ClassAd machine attribute!OpSysName}
\item[\AdAttr{OpSysName}:] A string containing a terse description of the operating system.
	\begin{description}
	\item[\AdStr{RedHat}:] for RedHat Linux version 6
	\item[\AdStr{Fedora}:] for Fedora Linux version 16
	\item[\AdStr{Ubuntu}:] for Ubuntu 12.04
	\item[\AdStr{SnowLeopard}:] for MacOS version 10.6.5 (Snow Leopard)
	\item[\AdStr{Lion}:] for MacOS version 10.7.3 (Lion)
	\item[\AdStr{FREEBSD}:] for FreeBSD version 7 or 8
	\item[\AdStr{SOLARIS5.10}:] for Solaris 2.10 or 5.10
	\item[\AdStr{SOLARIS5.11}:] for Solaris 2.11 or 5.11
	\item[\AdStr{WindowsXP}:] for Windows XP
	\item[\AdStr{WindowsVista}:] for Windows Vista
	\item[\AdStr{Windows7}:] for Windows 7
	\item[\AdStr{SL}:] for Scientific Linux
	\item[\AdStr{SLFermi}:] for Fermi's Scientific Linux
	\item[\AdStr{SLCern}:] for CERN's Scientific Linux
	\end{description}
%
\index{ClassAd machine attribute!OpSysShortName}
\item[\AdAttr{OpSysShortName}:] A string containing a short name for
the operating system.
	\begin{description}
	\item[\AdStr{RedHat}:] for RedHat Linux version 5 or 6
	\item[\AdStr{Fedora}:] for Fedora Linux version 16
	\item[\AdStr{Debian}:] for Debian Linux version 5 or 6
	\item[\AdStr{Ubuntu}:] for Ubuntu 12.04
	\item[\AdStr{MacOSX}:] for MacOS version 10.6.5 (Snow Leopard) or 
for MacOS version 10.7.3 (Lion)
	\item[\AdStr{FreeBSD}:] for FreeBSD version 7 or 8
	\item[\AdStr{SOLARIS5.10}:] for Solaris 2.10 or 5.10
	\item[\AdStr{SOLARIS5.11}:] for Solaris 2.11 or 5.11
	\item[\AdStr{XP}:] for Windows XP
	\item[\AdStr{Vista}:] for Windows Vista
	\item[\AdStr{7}:] for Windows 7
	\item[\AdStr{SL}:] for Scientific Linux
	\item[\AdStr{SLFermi}:] for Fermi's Scientific Linux
	\item[\AdStr{SLCern}:] for CERN's Scientific Linux
	\end{description}
%
\index{ClassAd machine attribute!OpSysVer}
\item[\AdAttr{OpSysVer}:] An integer value representing the operating system
version number.
	\begin{description}
	\item[\Expr{602}:] for RedHat Linux version 6.2
	\item[\Expr{1600}:] for Fedora Linux version 16.0
	\item[\Expr{1204}:] for Ubuntu 12.04
	\item[\Expr{704}:] for FreeBSD version 7.4
	\item[\Expr{802}:] for FreeBSD version 8.2
	\item[\Expr{605}:] for MacOS version 10.6.5 (Snow Leopard)
	\item[\Expr{703}:] for MacOS version 10.7.3 (Lion)
	\item[\Expr{500}:] for Windows 2000
	\item[\Expr{501}:] for Windows XP
	\item[\Expr{502}:] for Windows Server 2003
	\item[\Expr{600}:] for Windows Vista or Windows Server 2008
	\item[\Expr{601}:] for Windows 7 or Windows Server 2008
	\end{description}
%
\index{ClassAd machine attribute!RecentJobPreemptions}
\item[\AdAttr{RecentJobPreemptions}:] The total number of jobs which
have been preempted from this machine in the last twenty minutes.
%
\index{ClassAd machine attribute!RecentJobRankPreemptions}
\item[\AdAttr{RecentJobRankPreemptions}:] The total number of times
a running job has been preempted on this machine due to the machine's
rank of jobs in the last twenty minutes.  
%
\index{ClassAd machine attribute!RecentJobStarts}
\item[\AdAttr{RecentJobStarts}:] The total number of jobs which
have been started on this machine in the last twenty minutes.
%
\index{ClassAd machine attribute!RecentJobUserPrioPreemptions}
\item[\AdAttr{RecentJobUserPrio}:] The total number of times
a running job has been preempted on this machine based on a fair share
allocation of the pool 
in the last twenty minutes.
%
\index{ClassAd machine attribute!Requirements}
\item[\AdAttr{Requirements}:] A boolean, which when evaluated within the context
of the machine ClassAd and a job ClassAd, must evaluate to
TRUE before HTCondor will allow the job to use this machine.
%
\index{ClassAd machine attribute!RetirementTimeRemaining}
\item[\AdAttr{RetirementTimeRemaining}:] An integer number of seconds
after \AdAttr{MyCurrentTime} when the running job can be evicted.
\AdAttr{MaxJobRetirementTime} is the expression of how much retirement
time the machine offers to new jobs, whereas \AdAttr{RetirementTimeRemaining}
is the negotiated amount of time remaining for the current running
job.  This may be less than the amount offered by the machine's
\AdAttr{MaxJobRetirementTime} expression, because the job may
ask for less.
%
\index{ClassAd machine attribute!PartitionableSlot}
\label{PartitionableSlot-machine-attribute} 
\item[\AdAttr{PartitionableSlot}:] For SMP machines,
a boolean value identifying that this slot may be partitioned.
%
\index{ClassAd machine attribute!SlotID}
\item[\AdAttr{SlotID}:] For SMP machines, the integer
that identifies the slot.
The value will be \verb@X@ for the slot with 
\begin{verbatim}
name="slotX@full.hostname"
\end{verbatim}
For non-SMP machines with one slot, the value will be 1.
\Note This attribute was added in HTCondor version 6.9.3.
For older versions of HTCondor, see \AdAttr{VirtualMachineID} below.
%
\index{ClassAd machine attribute!SlotWeight}
\item[\AdAttr{SlotWeight}:]
  This specifies the weight of the slot when
  calculating usage, computing fair shares, and enforcing group
  quotas.  For example, claiming a slot with \Expr{SlotWeight = 2} is
  equivalent to claiming two \Expr{SlotWeight = 1} slots.
  See the description of \MacroNI{SlotWeight} on
  page~\pageref{param:SlotWeight}.

%
\index{ClassAd machine attribute!StartdIpAddr}
\item[\AdAttr{StartdIpAddr}:] String with the IP and port address of the
\Condor{startd} daemon which is publishing this machine ClassAd.
When using CCB, \Condor{shared\_port}, and/or an additional private
network interface, that information will be included here as well.

%
\index{ClassAd machine attribute!State}
\item[\AdAttr{State}:] String which publishes the machine's HTCondor state.
Can be:
	\begin{description}
	\item[\AdStr{Owner}:] The machine owner is using the machine, and
it is unavailable to HTCondor.
	\item[\AdStr{Unclaimed}:] The machine is available to run HTCondor jobs,
but a good match is either not available or not 
yet found.
	\item[\AdStr{Matched}:] The HTCondor central manager has found a good
match for this resource, but an HTCondor scheduler has not yet claimed it.
	\item[\AdStr{Claimed}:] The machine is claimed by a remote
\Condor{schedd} and is probably running a job.
	\item[\AdStr{Preempting}:] An HTCondor job is being preempted (possibly
via checkpointing) in order to clear the machine for either a higher
priority job or because the machine owner wants the machine back.
	\item[\AdStr{Drained}:] This slot is not accepting jobs,
because the machine is being drained.
	\end{description}   % of State
%
\index{ClassAd machine attribute!TargetType}
\item[\AdAttr{TargetType}:] Describes what type of ClassAd to match with.
Always set to the string literal \AdStr{Job}, because machine ClassAds
always want to be matched with jobs, and vice-versa.
%
\index{ClassAd machine attribute!TotalCondorLoadAvg}
\item[\AdAttr{TotalCondorLoadAvg}:] The load average contributed  
by HTCondor summed across all slots on the machine, 
either from remote jobs or running benchmarks.
%
\index{ClassAd machine attribute!TotalCpus}
\item[\AdAttr{TotalCpus}:] The number of CPUs that are on the machine.
This is in contrast with \Attr{Cpus},
which is the number of CPUs in the slot.
%
\index{ClassAd machine attribute!TotalLoadAvg}
\item[\AdAttr{TotalLoadAvg}:] A floating point number representing the 
current load average summed across all slots on the machine.
%
\index{ClassAd machine attribute!TotalMachineDrainingBadput}
\item[\AdAttr{TotalMachineDrainingBadput}:] The
total job runtime in cpu-seconds that has been lost due to job evictions
caused by draining since this \Condor{startd} began executing.  In
this calculation, it is assumed that jobs are evicted without
checkpointing.
%
\index{ClassAd machine attribute!TotalMachineDrainingUnclaimedTime}
\item[\AdAttr{TotalMachineDrainingUnclaimedTime}:] The
total machine-wide time in cpu-seconds that has not been used
(i.e. not matched to a job submitter) due to draining since this
\Condor{startd} began executing.
%
\index{ClassAd machine attribute!TotalTimeBackfillBusy}
\item[\AdAttr{TotalTimeBackfillBusy}:] The number of seconds
that this machine (slot) has accumulated within the
backfill busy state and activity pair since the \Condor{startd}
began executing.
This attribute will only be defined if it has a value greater than 0.
%
\index{ClassAd machine attribute!TotalTimeBackfillIdle}
\item[\AdAttr{TotalTimeBackfillIdle}:] The number of seconds
that this machine (slot) has accumulated within the
backfill idle state and activity pair since the \Condor{startd}
began executing.
This attribute will only be defined if it has a value greater than 0.
%
\index{ClassAd machine attribute!TotalTimeBackfillKilling}
\item[\AdAttr{TotalTimeBackfillKilling}:] The number of seconds
that this machine (slot) has accumulated within the
backfill killing state and activity pair since the \Condor{startd}
began executing.
This attribute will only be defined if it has a value greater than 0.
%
\index{ClassAd machine attribute!TotalTimeClaimedBusy}
\item[\AdAttr{TotalTimeClaimedBusy}:] The number of seconds
that this machine (slot) has accumulated within the
claimed busy state and activity pair since the \Condor{startd}
began executing.
This attribute will only be defined if it has a value greater than 0.
%
\index{ClassAd machine attribute!TotalTimeClaimedIdle}
\item[\AdAttr{TotalTimeClaimedIdle}:] The number of seconds
that this machine (slot) has accumulated within the
claimed idle state and activity pair since the \Condor{startd}
began executing.
This attribute will only be defined if it has a value greater than 0.
%
\index{ClassAd machine attribute!TotalTimeClaimedRetiring}
\item[\AdAttr{TotalTimeClaimedRetiring}:] The number of seconds
that this machine (slot) has accumulated within the
claimed retiring state and activity pair since the \Condor{startd}
began executing.
This attribute will only be defined if it has a value greater than 0.
%
\index{ClassAd machine attribute!TotalTimeClaimedSuspended}
\item[\AdAttr{TotalTimeClaimedSuspended}:] The number of seconds
that this machine (slot) has accumulated within the
claimed suspended state and activity pair since the \Condor{startd}
began executing.
This attribute will only be defined if it has a value greater than 0.
%
\index{ClassAd machine attribute!TotalTimeMatchedIdle}
\item[\AdAttr{TotalTimeMatchedIdle}:] The number of seconds
that this machine (slot) has accumulated within the
matched idle state and activity pair since the \Condor{startd}
began executing.
This attribute will only be defined if it has a value greater than 0.
%
\index{ClassAd machine attribute!TotalTimeOwnerIdle}
\item[\AdAttr{TotalTimeOwnerIdle}:] The number of seconds
that this machine (slot) has accumulated within the
owner idle state and activity pair since the \Condor{startd}
began executing.
This attribute will only be defined if it has a value greater than 0.
%
\index{ClassAd machine attribute!TotalTimePreemptingKilling}
\item[\AdAttr{TotalTimePreemptingKilling}:] The number of seconds
that this machine (slot) has accumulated within the
preempting killing state and activity pair since the \Condor{startd}
began executing.
This attribute will only be defined if it has a value greater than 0.
%
\index{ClassAd machine attribute!TotalTimePreemptingVacating}
\item[\AdAttr{TotalTimePreemptingVacating}:] The number of seconds
that this machine (slot) has accumulated within the
preempting vacating state and activity pair since the \Condor{startd}
began executing.
This attribute will only be defined if it has a value greater than 0.
%
\index{ClassAd machine attribute!TotalTimeUnclaimedBenchmarking}
\item[\AdAttr{TotalTimeUnclaimedBenchmarking}:] The number of seconds
that this machine (slot) has accumulated within the
unclaimed benchmarking state and activity pair since the \Condor{startd}
began executing.
This attribute will only be defined if it has a value greater than 0.
%
\index{ClassAd machine attribute!TotalTimeUnclaimedIdle}
\item[\AdAttr{TotalTimeUnclaimedIdle}:] The number of seconds
that this machine (slot) has accumulated within the
unclaimed idle state and activity pair since the \Condor{startd}
began executing.
This attribute will only be defined if it has a value greater than 0.
%
\index{ClassAd machine attribute!UidDomain}
\item[\AdAttr{UidDomain}:] a domain name configured by the HTCondor 
administrator which describes a cluster of machines which all have 
the same \File{passwd} file entries, and therefore all have the same logins.
%
\index{ClassAd machine attribute!VirtualMachineID}
\item[\AdAttr{VirtualMachineID}:] 
Starting with HTCondor version 6.9.3, this attribute is now longer used.
Instead, use \AdAttr{SlotID}, as described above.
This will only be present if \Macro{ALLOW\_VM\_CRUFT} is TRUE.
%
\index{ClassAd machine attribute!VirtualMemory}
\item[\AdAttr{VirtualMemory}:] The amount of currently available virtual memory 
(swap space) expressed in Kbytes.
On Linux platforms, it is the sum of paging space and physical memory, 
which more accurately represents the virtual memory size of the machine. 
%
\index{ClassAd machine attribute!VM\_AvailNum}
\item[\AdAttr{VM\_AvailNum}:] The maximum number of vm universe jobs that
can be started on this machine. This maximum is set by the configuration
variable \Macro{VM\_MAX\_NUMBER}. 
%
\index{ClassAd machine attribute!VM\_Guest\_Mem}
\item[\AdAttr{VM\_Guest\_Mem}:] An attribute defined if a vm universe job
is running on this slot.  Defined by the amount of memory in use by the 
virtual machine, given in Mbytes.
%
\index{ClassAd machine attribute!VM\_Memory}
\item[\AdAttr{VM\_Memory}:] Gives the amount of memory available for starting 
additional VM jobs on this machine, given in Mbytes.
The maximum value is set by the configuration variable \Macro{VM\_MEMORY}.
%
\index{ClassAd machine attribute!VM\_Networking}
\item[\AdAttr{VM\_Networking}:] A boolean value indicating whether networking 
is allowed for virtual machines on this machine.
%
\index{ClassAd machine attribute!VM\_Type}
\item[\AdAttr{VM\_Type}:] The type of virtual machine software that can run
on this machine.  The value is set by the configuration variable
\Macro{VM\_TYPE}.
%
\index{ClassAd machine attribute!WindowsBuildNumber}
\item[\AdAttr{WindowsBuildNumber}:] An integer, extracted from the
platform type, representing a build number 
for a Windows operating system.
This attribute only exists on Windows machines.
%
\index{ClassAd machine attribute!WindowsMajorVersion}
\item[\AdAttr{WindowsMajorVersion}:] An integer, extracted from the
platform type, representing a major version number (currently 5 or 6)
for a Windows operating system.
This attribute only exists on Windows machines.
%
\index{ClassAd machine attribute!WindowsMinorVersion}
\item[\AdAttr{WindowsMinorVersion}:] An integer, extracted from the
platform type, representing a minor version number (currently 0, 1, or 2)
for a Windows operating system.
This attribute only exists on Windows machines.

\end{description}

In addition, there are a few attributes that are automatically
inserted into the machine ClassAd whenever a resource is in the
Claimed state:

\begin{description}

\index{ClassAd machine attribute (in Claimed State)!ClientMachine}
\item[\AdAttr{ClientMachine}:] The host name of the machine that has
claimed this resource

\index{ClassAd machine attribute (in Claimed State)!RemoteAutoregroup}
\item[\AdAttr{RemoteAutoregroup}:]  A boolean attribute which is \Expr{True}
if this resource was claimed via negotiation 
when the configuration variable \Macro{GROUP\_AUTOREGROUP} is \Expr{True}.
It is \Expr{False} otherwise.

\index{ClassAd machine attribute (in Claimed State)!RemoteGroup}
\item[\AdAttr{RemoteGroup}:]  The accounting group name corresponding to
the submitter that claimed this resource.

\index{ClassAd machine attribute (in Claimed State)!RemoteNegotiatingGroup}
\item[\AdAttr{RemoteNegotiatingGroup}:]  The accounting group name under
which this resource negotiated when it was claimed.  This attribute will
frequently be the same as attribute \Attr{RemoteGroup},
but it may differ in cases such
as when configuration variable \Macro{GROUP\_AUTOREGROUP} is \Expr{True},
in which case it will have the name of the root group, 
identified as \Expr{<none>}.

\index{ClassAd machine attribute (in Claimed State)!RemoteOwner}
\item[\AdAttr{RemoteOwner}:] The name of the user who originally
claimed this resource.

\index{ClassAd machine attribute (in Claimed State)!RemoteUser}
\item[\AdAttr{RemoteUser}:] The name of the user who is currently
using this resource.
In general, this will always be the same as the \AdAttr{RemoteOwner},
but in some cases, a resource can be claimed by one entity that hands
off the resource to another entity which uses it.
In that case, \AdAttr{RemoteUser} would hold the name of the entity
currently using the resource, while \AdAttr{RemoteOwner} would hold
the name of the entity that claimed the resource.

\index{ClassAd machine attribute (in Claimed State)!PreemptingOwner}
\item[\AdAttr{PreemptingOwner}:] The name of the user who is preempting
the job that is currently running on this resource.

\index{ClassAd machine attribute (in Claimed State)!PreemptingUser}
\item[\AdAttr{PreemptingUser}:] The name of the user who is preempting
the job that is currently running on this resource.  The relationship
between \AdAttr{PreemptingUser} and \AdAttr{PreemptingOwner} is the same
as the relationship between \AdAttr{RemoteUser} and \AdAttr{RemoteOwner}.

\index{ClassAd machine attribute (in Claimed State)!PreemptingRank}
\item[\AdAttr{PreemptingRank}:] A float which represents this machine
owner's affinity for running the HTCondor job which is waiting for the
current job to finish or be preempted.  If not currently hosting an
HTCondor job, \AdAttr{PreemptingRank} is undefined.  When a machine is
claimed and there is already a job running, the attribute's value is
computed by evaluating the machine's \AdAttr{Rank} expression with
respect to the preempting job's ClassAd.

\index{ClassAd machine attribute (in Claimed State)!TotalClaimRunTime}
\item[\AdAttr{TotalClaimRunTime}:] A running total of the amount of
time (in seconds) that all jobs (under the same claim) ran
(have spent in the Claimed/Busy state).


\index{ClassAd machine attribute (in Claimed State)!TotalClaimSuspendTime}
\item[\AdAttr{TotalClaimSuspendTime}:] A running total of the amount of
time (in seconds) that all jobs (under the same claim) have been
suspended (in the Claimed/Suspended state).

\index{ClassAd machine attribute (in Claimed State)!TotalJobRunTime}
\item[\AdAttr{TotalJobRunTime}:] A running total of the amount of
time (in seconds) that a single job ran
(has spent in the Claimed/Busy state).

\index{ClassAd machine attribute (in Claimed State)!TotalJobSuspendTime}
\item[\AdAttr{TotalJobSuspendTime}:] A running total of the amount of
time (in seconds) that a single job has been suspended
(in the Claimed/Suspended state).

\end{description}

There are a few attributes that are only inserted into the
machine ClassAd if a job is currently executing.  
If the resource is claimed but no job are running, none of these
attributes will be defined.

\begin{description}

\index{ClassAd machine attribute (when running)!JobId}
\item[\AdAttr{JobId}:] The job's identifier (for example,
\verb@152.3@), as seen from \Condor{q}
on the submitting machine.

\index{ClassAd machine attribute (when running)!JobStart}
\item[\AdAttr{JobStart}:] The time stamp in integer seconds of when the job began
executing, since the Unix epoch (00:00:00 UTC, Jan 1, 1970).  For idle
machines, the value is \Expr{UNDEFINED}.

\index{ClassAd machine attribute (when running)!LastPeriodicCheckpoint}
\item[\AdAttr{LastPeriodicCheckpoint}:] If the job has performed a
periodic checkpoint, this attribute will be defined and will hold the
time stamp of when the last periodic checkpoint was begun.
If the job has yet to perform a periodic checkpoint, or cannot
checkpoint at all, the \AdAttr{LastPeriodicCheckpoint} attribute will
not be defined.

\end{description}

\index{offline ClassAd}
There are a few attributes that are applicable to machines that
are offline, that is, hibernating.

\begin{description}

\index{ClassAd machine attribute (when offline)!MachineLastMatchTime}
\item[\AdAttr{MachineLastMatchTime}:] The Unix epoch time when this offline 
ClassAd
would have been matched to a job, if the machine were online.  
In addition,
the slot1 ClassAd of a multi-slot machine will have 
\AdAttr{slot<X>\_MachineLastMatchTime} defined,
where \Expr{<X>} is replaced by the slot id of each of the slots
with \AdAttr{MachineLastMatchTime} defined.

\index{ClassAd machine attribute (when offline)!Offline}
\item[\AdAttr{Offline}:] A boolean value, that when \Expr{True},
indicates this machine is in an offline state in the \Condor{collector}.
Such ClassAds are stored persistently, 
such that they will continue to exist after the \Condor{collector} restarts.

\index{ClassAd machine attribute (when offline)!Unhibernate}
\item[\AdAttr{Unhibernate}:] A boolean expression that specifies when
a hibernating machine should be woken up, for example, by \Condor{rooster}.

\end{description}

Finally, the single attribute, 
\Attr{CurrentTime}, is defined by the ClassAd
environment.
\begin{description}
\index{ClassAd attribute!CurrentTime}
\item[\AdAttr{CurrentTime}:] Evaluates to the 
the number of integer seconds since the Unix epoch (00:00:00 UTC, Jan 1, 1970).
\end{description}
