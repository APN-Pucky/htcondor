%%%%%%%%%%%%%%%%%%%%%%%%%%%%%%%%%%%%%%%%%%%%%%%%%%%%%%%%%%%%%%%%%%%%%%
\section{\label{sec:cod}Computing On Demand (COD)}
%%%%%%%%%%%%%%%%%%%%%%%%%%%%%%%%%%%%%%%%%%%%%%%%%%%%%%%%%%%%%%%%%%%%%%
\index{COD (Computing on Demand)|(}
\index{Computing on Demand (see COD)}

\index{COD!introduction} 
Computing On Demand (COD) extends Condor's high throughput
computing abilities to include
a method for running short-term jobs on instantly-available resources.
Support for COD was added to Condor in version 6.5.2.   

%This section describes how to execute short-running interactive
%applications on batch resources controlled by Condor.
%This functionality is known as \Term{Computing on Demand} or
%\Term{COD}.
%Support for Computing On Demand was added to Condor in version 6.5.2.   

%In the following sections, we will describe the concept and what kinds
%of applications it is meant to be used for, give an overview of how it
%works and interacts with Condor, describe how to configure Condor
%resources to provide this service, explain how to use various tools
%included with Condor to manage COD applications, and explore some
%current limitations in Condor's support for COD.

The motivation for COD extends Condor's job management to
include interactive, compute-intensive jobs,
giving these jobs immediate access to the
compute power they need over a relatively short period of time.
COD provides
computing power \emph{on demand}, 
switching predefined resources from working on Condor jobs
to working on the COD jobs. 
These COD jobs (applications) cannot use the batch
scheduling functionality of Condor, since the COD jobs require
interactive response-time.
Many of the applications that are well-suited to Condor's
COD capabilities involve a cycle:
application blocked on user input, computation burst to compute
results, block again on user input, computation burst, etc.
When the resources are not being used for the bursts of computation to
service the application, they should continue to execute
long-running batch jobs.

Here are examples of applications that may benefit from COD capability:

\begin{itemize}

\item A giant spreadsheet with a large number of highly complex
  formulas which take a lot of compute power to recalculate.
  The spreadsheet application (as a COD application) predefines
  a claim on resources within the Condor pool.
  When the user presses a \verb@recalculate@ button, 
  the predefined Condor resources (nodes) 
  work on the computation and send the results
  back to the master application providing the user interface and
  displaying the data.
  Ideally, while the user is entering new data or modifying formulas,
  these nodes work on non-COD jobs.
  %should not tie up the batch resources where they are running.

\item A graphics rendering application that waits for user
  input to select an image to render.
  The rendering requires a huge burst
  of computation to produce the image.
  Examples are various Computer-Aided Design (CAD) tools, fractal
  rendering programs, and ray-tracing tools.
 
\item Visualization tools for data mining.

\end{itemize}

The way Condor helps these kinds of applications is to provide an
infrastructure to use Condor batch resources for the types of compute
nodes described above.
Condor does \emph{NOT} provide tools to parallelize existing GUI
applications.
The COD functionality is an interface to allow these compute nodes to
interact with long-running Condor batch jobs.
The user provides both the compute node applications and the
interactive master application that controls them.
Condor only provides a mechanism to allow these interactive (and often
parallelized) applications to seamlessly interact with the Condor
batch system.


%%%%%%%%%%%%%%%%%%%%%%%%%%%%%%%%%%%%%%%%%%%%%%%%%%%%%%%%%%%%%%%%%%%%%%
\subsection{\label{sec:cod-overview}
Overview of How COD Works}
%%%%%%%%%%%%%%%%%%%%%%%%%%%%%%%%%%%%%%%%%%%%%%%%%%%%%%%%%%%%%%%%%%%%%%
\index{COD!overview} 

The resources of a Condor pool (nodes)
run jobs.
When a high-priority COD job appears at a node,
the lower-priority (currently running) batch job is
suspended.
The COD job runs immediately, while the batch job remains suspended.
When the COD job completes, the batch job instantly resumes execution.

Administratively,
an interactive COD application puts claims on nodes.
While the COD application does not need the nodes (to run the
COD jobs),  the claims are suspended, allowing batch jobs to run.


%%%%%%%%%%%%%%%%%%%%%%%%%%%%%%%%%%%%%%%%%%%%%%%%%%%%%%%%%%%%%%%%%%%%%%
\subsection{\label{sec:cod-authorizing}
Authorizing Users to Create and Manage COD Claims}
%%%%%%%%%%%%%%%%%%%%%%%%%%%%%%%%%%%%%%%%%%%%%%%%%%%%%%%%%%%%%%%%%%%%%%
\index{COD!authorizing users} 

Claims on nodes are assigned to users.
A user with a claim on a resource can then suspend and
resume a COD job at will.
This gives the user a great deal of power on the claimed resource,
even if it is owned by another user.
Because of this, it is essential that users allowed to claim
COD resources can be
trusted not to abuse this power.
Users are authorized to have access to the privilege
of creating and using a COD claim on a machine.
This privilege is granted when the Condor administrator places a given
username in the \Macro{VALID\_COD\_USERS} list in the Condor
configuration for the machine (usually in a local configuration file).

In addition, the tools to request and manage COD claims
require that the user issuing the commands be authenticated. 
Use one of the strong authentication methods described
in section~\ref{sec:Config-Security} ``Security Configuration'' on
page~\pageref{sec:Config-Security}.
If one of these methods cannot be used,
then file system authentication may be used
when directly logging in to that machine (to be claimed)
and issuing the command locally.
%This is analogous to commands that submit or remove jobs from the job
%queue managed by a given \Condor{schedd}.
%Unless some form of strong authentication such as GSI or Kerberos is
%enabled at your Condor pool, you must issue \Condor{submit} and
%\Condor{rm} commands from the same machine where the \Condor{schedd}
%is running.


%%%%%%%%%%%%%%%%%%%%%%%%%%%%%%%%%%%%%%%%%%%%%%%%%%%%%%%%%%%%%%%%%%%%%%
\subsection{\label{sec:cod-setup}
Defining a COD Application}
%%%%%%%%%%%%%%%%%%%%%%%%%%%%%%%%%%%%%%%%%%%%%%%%%%%%%%%%%%%%%%%%%%%%%%
\index{COD!defining an application} 

To run an application on a claimed COD resource,
an authorized user defines 
characteristics of the application.
Examples of characteristics are the executable or script to use,
the directory to run the application in,
command-line arguments, and
files to use for standard input and output.
COD users specify a ClassAd that describes these characteristics
for their application.  
There are two ways for a user to define a COD application's ClassAd:

\begin{enumerate}
\item in the Condor configuration files of the COD resources
\item when they use the \Condor{cod} command-line tool to launch the
application itself
\end{enumerate}

These two methods for defining the ClassAd can be used together.
For example, the user can define some attributes
in the configuration file, and only provide a few dynamically defined
attributes with the \Condor{cod} tool.

Regardless of how the COD application's ClassAd is defined,
the application's executable and input
data must be pre-staged at the node.
This is a current limitation of Condor's support for COD that will
eventually go away.
For now, there is no mechanism to transfer files for a COD
application, and all I/O must be performed locally or onto a network
file system that is accessible by a node.

The following three sections detail defining the attributes.
The first lists the attributes that can be used to define a COD
application.
The second describes how to define these attributes in a Condor
configuration file.
The third explains how to define these attributes using the \Condor{cod} tool.


%%%%%%%%%%%%%%%%%%%%%%%%%%%%%%%%%%%%%%%%%%%%%%%%%%%%%%%%%%%%%%%%%%%%%%
\subsubsection{\label{sec:cod-application-attributes}
COD Application Attributes}
%%%%%%%%%%%%%%%%%%%%%%%%%%%%%%%%%%%%%%%%%%%%%%%%%%%%%%%%%%%%%%%%%%%%%%

\index{COD!attributes} 
\index{Computing On Demand!Defining Applications!Required attributes}

Attributes are for a COD application are either required
or optional.
The following attributes are \emph{required}:
\index{COD!required attributes} 

\begin{description}

 \item[\Attr{Cmd}] This attribute 
\index{COD!required attributes!Cmd} 
   defines the full path to the executable program to be run as a
   COD application.
   Since Condor does not currently provide any mechanism to transfer
   files on behalf of COD applications, this path should be a valid
   path on the machine where the application will be run.
   It is a string attribute, and must therefore be enclosed in
   quotation marks (\verb@"@).
   There is no default.

 \item[\Attr{IWD}] IWD is an acronym for Initial Working Directory.
\index{COD!required attributes!IWD} 
   It defines the full path to the directory where a given COD
   application are to be run.
   Unless the application changes its current working directory, any
   relative path names used by the application will be relative to
   the IWD.
   If any other attributes that define file names (for example,
   \Attr{In}, \Attr{Out}, and so on) do not contain a full path, the
   \Attr{IWD} will automatically be prepended to those filenames.
   It is a string attribute, and must therefore be enclosed in 
   quotation marks (\verb@"@).
   There is no default.

 \item[\Attr{Owner}] If the \Condor{startd} daemon is executing as root on
\index{COD!required attributes!Owner} 
   the resource where a COD application will run, the user must also
   define \Attr{Owner} to specify what user name the application will
   run as.
   (On Windows, the \Condor{startd} daemon always runs as an Administrator
   service, which is equivalent to running as root on UNIX platforms).
   If the user specifies any COD application attributes with the
   \Condor{cod\_activate} command-line tool, the \Attr{Owner}
   attribute will be defined as the user name that ran
   \Condor{cod\_activate}.
   However, if the user defines all attributes of their COD
   application in the Condor configuration files, and does not define
   any attributes with the \Condor{cod\_activate} command-line tool
   (both methods are described below in more detail), there is no
   default and \Attr{Owner} must be specified in the configuration
   file.
   \Attr{Owner} must contain a valid user
   name on the given COD resource. 
   It is a string attribute, and must therefore be enclosed in 
   quotation marks (\verb@"@).

\end{description}

\index{COD!optional attributes} 
\index{Computing On Demand!Defining Applications!Optional attributes}

The following list of attributes are \emph{optional}:

\begin{description}

 \item[\Attr{In}] This string defines the path to the file on the
\index{COD!optional attributes!In} 
   COD resource that should be used as standard input (\File{stdin}) for 
   the COD application.
   This file (and all parent directories) must be readable by whatever
   user the COD application will run as.
   If not specified, the default is \File{/dev/null}.
 
 \item[\Attr{Out}] This string defines the path to the file on the
\index{COD!optional attributes!Out} 
   COD resource that should be used as standard output (\File{stdout})
   for the COD application.
   This file must be writable (and all parent directories readable) by
   whatever user the COD application will run as.
   If not specified, the default is \File{/dev/null}.
   It is a string attribute, and must therefore be enclosed in 
   quotation marks (\verb@"@).
 
 \item[\Attr{Err}] This string defines the path to the file on the
\index{COD!optional attributes!Err} 
   COD resource that should be used as standard error (\File{stderr})
   for the COD application.
   This file must be writable (and all parent directories readable) by
   whatever user the COD application will run as.
   If not specified, the default is \File{/dev/null}.
   It is a string attribute, and must therefore be enclosed in 
   quotation marks (\verb@"@).

 \item[\Attr{Env}] This string defines environment variables to
\index{COD!optional attributes!Env} 
   set for a given COD application.
   Each environment variable has the form \verb@NAME=value@.
   Multiple variables are delimited with a semicolon.
   An example: \verb@Env = "PATH=/usr/local/bin:/usr/bin;TERM=vt100"@ 
   It is a string attribute, and must therefore be enclosed in 
   quotation marks (\verb@"@).

 \item[\Attr{Args}] This string attribute defines the list of
\index{COD!optional attributes!Args} 
   arguments to be supplied to the program on the command-line.
   The arguments are delimited (separated) by space characters. 
   There is no default. 
   If the \Attr{JobUniverse} corresponds to the Java
   universe, the first argument must be the name of the class
   containing \Code{main}.
   It is a string attribute, and must therefore be enclosed in 
   quotation marks (\verb@"@).

 \item[\Attr{JobUniverse}] This attribute defines what Condor job
\index{COD!optional attributes!JobUniverse} 
   universe to use for the given COD application.
   At this point, the only supported universes are vanilla and Java.
   This attribute must be an integer, with vanilla using the value 5,
   and Java the value 10.
   If \Attr{JobUniverse} is not specified, the vanilla universe is
   used by default.
   For more information about the Condor job universes, see
   section~\ref{sec:Choosing-Universe} on
   page~\pageref{sec:Choosing-Universe}. 

 \item[\Attr{JarFiles}] This string attribute is only used if
\index{COD!optional attributes!JarFiles} 
   \Attr{JobUniverse} is 10 (the Java universe).
   If a given COD application is a Java program, specify the
   JAR files that the program requires with this attribute.
   There is no default.
   It is a string attribute, and must therefore be enclosed in 
   quotation marks (\verb@"@).
   Multiple file names may be delimited with either commas or whitespace
   characters, and
   therefore, file names can not contain spaces.

 \item[\Attr{KillSig}] This attribute specifies what signal should be
\index{COD!optional attributes!KillSig} 
   sent whenever the Condor system needs to gracefully shutdown the
   COD application.
   It can either be specified as a string containing the signal name
   (for example \verb@KillSig = "SIGQUIT"@), or as an integer
   (\verb@KillSig = 3@)
   The default is to use SIGTERM.

 \item[\Attr{StarterUserLog}] This string specifies a file name for a
\index{COD!optional attributes!StarterUserLog} 
   log file that the \Condor{starter} daemon can write with entries
   for relevant 
   events in the life of a given COD application.
   It is similar to the UserLog file specified for regular Condor
   jobs with the \Attr{Log} setting in a submit description
   file.
   However, certain attributes that are placed in the regular UserLog
   file do not make sense in the COD environment, and are therefore
   omitted.
   The default is not to write this log file.
   It is a string attribute, and must therefore be enclosed in 
   quotation marks (\verb@"@).

 \item[\Attr{StarterUserLogUseXML}] If the \Attr{StarterUserLog}
\index{COD!optional attributes!StarterUserLogUseXML} 
   attribute is defined, the default format is a
   human-readable format.
   However, Condor can write out this log in an XML representation,
   instead.
   To enable the XML format for this UserLog, the
   \Attr{StarterUserLogUseXML} boolean is set to \verb@TRUE@.
   The default if not specified is \verb@FALSE@.

\end{description}

\Note If any path attribute (\Attr{Cmd}, \Attr{In},
\Attr{Out},\Attr{Err}, \Attr{StarterUserLog}) is not a full path name,
Condor automatically prepends the value of \Attr{IWD}.


\index{Computing On Demand!Defining Applications!Job ID}
\index{COD!defining applications!Job ID}

The final set of attributes define an identification for a COD application.
The job ID is made up of both the \Attr{ClusterId} and \Attr{ProcId}
attributes (as described below).
This job ID is similar to the job ID that is created whenever a
regular Condor batch job is submitted.
When using COD, the job ID is only used to identify the job in various
log messages and in the COD-specific output of \Condor{status}.

The COD job ID is part of the information included in all
events written to the \Attr{StarterUserLog}
regarding a given job.
The COD job ID is also used in the Condor debugging logs described in
section~\ref{sec:Daemon-Logging-Config-File-Entries} on
page~\pageref{sec:Daemon-Logging-Config-File-Entries}
For example, in the \Condor{starter} daemon's log file for COD jobs
(called \File{StarterLog.cod} by default) or in the \Condor{startd}
daemon's log
file (called \File{StartLog} by default).

These COD IDs are optional.
The job ID is useful to define where it helps a
user with accounting or debugging of their own application.
  
\begin{description}
 \item[\Attr{ClusterId}] This integer defines the
\index{COD!attributes!ClusterId} 
   cluster identifier for a COD job.
   The default value is 1.
   The \Attr{ClusterId} can also be defined with the
\index{COD!\Condor{cod\_activate} command} 
   \Condor{cod\_activate} command-line tool using the \Opt{-cluster}
   option.

 \item[\Attr{ProcId}]  This integer defines the
\index{COD!attributes!ProcID} 
   process identifier (within a cluster) for a COD job.
   The default value is 0.
   The \Attr{ProcId} can also be defined with the
   \Condor{cod\_activate} command-line tool using the \Opt{-cluster}
   option.

\end{description}


%%%%%%%%%%%%%%%%%%%%%%%%%%%%%%%%%%%%%%%%%%%%%%%%%%%%%%%%%%%%%%%%%%%%%%
\subsubsection{\label{sec:cod-config-attrs}
Defining Attributes in the Condor Configuration Files}
%%%%%%%%%%%%%%%%%%%%%%%%%%%%%%%%%%%%%%%%%%%%%%%%%%%%%%%%%%%%%%%%%%%%%%
\index{COD!defining attributes by configuration} 


To define COD attributes in the Condor configuration file for a given
application, the user selects a keyword to uniquely name 
ClassAd attributes of the application.
This case-insensitive keyword is used as a prefix for the various
configuration file attribute names.
When a user wishes to spawn a given application, the
keyword is given as an argument to the \Condor{cod} tool and the keyword
is used at the remote COD resource to find attributes which define the
application.

Any of the ClassAd attributes described in the previous section can be
specified in the configuration file with the keyword prefix followed
by an underscore character (\verb@"_"@).

For example, if the user's keyword for a given fractal generation
application is ``FractGen'', the resulting entries in the Condor
configuration file may appear as:

\begin{verbatim}
FractGen_Cmd = "/usr/local/bin/fractgen"
FractGen_Iwd = "/tmp/cod-fractgen"
FractGen_Out = "/tmp/cod-fractgen/output"
FractGen_Err = "/tmp/cod-fractgen/error"
FractGen_Args = "mandelbrot -0.65865,-0.56254 -0.45865,-0.71254"
\end{verbatim}

In this example, the executable may create other files.
The \Attr{Out} and \Attr{Err} attributes specified in the
configuration file are only for standard output and standard error
redirection.

When the user wishes to spawn an instance of this application,
they use the \Opt{-keyword} option of \OptArg{FractGen}
in the command-line of the \Condor{cod\_activate} command.

\Note If a user is defining all attributes of their COD application in
the Condor configuration files, and the \Condor{startd} daemon on the COD
resource they are using is running as root, the user must also define
\Attr{Owner} to be the user that the COD application should run as
(see section~\ref{sec:cod-application-attributes} above). 


%%%%%%%%%%%%%%%%%%%%%%%%%%%%%%%%%%%%%%%%%%%%%%%%%%%%%%%%%%%%%%%%%%%%%%
\subsubsection{\label{sec:cod-command-line-attrs}
Defining Attributes with the \Condor{cod} Tool} 
%%%%%%%%%%%%%%%%%%%%%%%%%%%%%%%%%%%%%%%%%%%%%%%%%%%%%%%%%%%%%%%%%%%%%%
\index{COD!\Condor{cod} tool} 

COD users may define attributes dynamically (at the time they spawn a
COD application).
In this case, the user writes the ClassAd attributes into a file, and the
file name is passed to the \Condor{cod\_activate} tool using the
\Opt{-jobad} command-line option.
These attributes are read by the \Condor{cod} tool and passed through
the system onto the \Condor{starter} daemon which spawns the COD application. 
If the file name given is \File{-}, the \Condor{cod} tool will read
from standard input (\File{stdin}).

% TODO
% For more information about using \Condor{cod\_activate}, see the
% command reference on page~\pageref{man-condor-submit}

Users should not add a keyword prefix 
when defining attributes with the \Condor{cod\_activate} tool.
The attribute names can be used in the file directly.

\Warn The current syntax for this file is not the same as the syntax
in the file used with \Condor{submit}.

\Note Users should not define the \Attr{Owner} attribute
when using \Condor{cod\_activate} on the command line, since Condor
will automatically insert the correct value based on what user runs the
\Condor{cod\_activate} command and how that user authenticates to the
COD resource.
If a user defines an attribute that does not match the authenticated
identity, Condor treats this case as an error, and it will fail to launch the
application.


%%%%%%%%%%%%%%%%%%%%%%%%%%%%%%%%%%%%%%%%%%%%%%%%%%%%%%%%%%%%%%%%%%%%%%
\subsection{\label{sec:cod-managing-claims}
Managing COD Resource Claims}
%%%%%%%%%%%%%%%%%%%%%%%%%%%%%%%%%%%%%%%%%%%%%%%%%%%%%%%%%%%%%%%%%%%%%%
\index{COD!managing claims} 

Separate commands are provided by Condor to manage COD
claims on batch resources.
Once created, each COD claim has a unique identifying string, called the
claim ID.
Most commands require a claim ID to specify which claim you wish to
act on. 
These commands are the means by which COD applications interact with
the rest of the Condor system.
They should be issued by the controller application to manage its
compute nodes.
Here is a list of the commands:

\begin{description}

\item [Request] Create a new COD claim on a given resource.

\item [Activate] Spawn a specific application on a specific COD claim.

\item [Suspend] Suspend a running application within a specific COD claim.

\item [Resume] Resume a suspended application on a specific COD claim.

\item [Deactivate] Shut down an application, but hold onto the COD claim
  for future use.

\item [Release] Destroy a specific COD claim, and shut down any job that is
  currently running on it.

\end{description}

To issue these commands, a user or application invokes the 
\Condor{cod} tool.
A command may be specified as the first argument to this tool, 
as
\begin{verbatim}
condor_cod -request -name c02.cs.wisc.edu
\end{verbatim}
or
the \Condor{cod} tool can be installed in such a way that the same
binary is used for a set of names, as
\begin{verbatim}
condor_cod_request -name c02.cs.wisc.edu
\end{verbatim}

In addition, there is now a \Opt{-cod} option to \Condor{status}.

The following sections describe each option in greater detail.

%%%%%%%%%%%%%%%%%%%%%%%%%%%%%%%%%%%%%%%%%%%%%%%%%%%%%%%%%%%%
\subsubsection{\label{sec:cod-claim-request}Request}
%%%%%%%%%%%%%%%%%%%%%%%%%%%%%%%%%%%%%%%%%%%%%%%%%%%%%%%%%%%%
\index{COD!\Condor{cod\_request} command} 

A user must be granted authorization to a create COD
claims on a specific machine.
In addition, when the user uses these COD claims, the
application they wish to run must be pre-staged on the machine.
% Karen says that previous stuff in this file makes this
% statement not true, since it can be done with config or
% command-line arguments
%and
%defined in the \File{condor\_config} file (or a local config file). 
Therefore, a user cannot simply request a COD claim at random.

The user specifies the resource on which to make a COD claim.
This is accomplished by specifying the name of the
\Condor{startd} daemon desired by invoking
\Condor{cod\_request} with the \Opt{-name} option and the
host name.  For example:
\begin{verbatim}
condor_cod_request -name c02.cs.wisc.edu
\end{verbatim}
If the \Condor{startd} daemon desired belongs to a different Condor
pool than the one where executing the COD commands,
use the \Opt{-pool} option to provide the name of the central manager
machine of the other pool.  For example:
\begin{verbatim}
condor_cod_request -name c02.cs.wisc.edu -pool condor.cs.wisc.edu
\end{verbatim}

An alternative provides the IP address and port number
where the \Condor{startd} daemon is listening
with the \Opt{-addr} option.
This information can be found in the \Condor{startd} ClassAd as the
attribute \Attr{StartdIpAddr} or by reading the log file when the
\Condor{startd} first starts up.
For example:
\begin{verbatim}
condor_cod_request -addr "<128.105.146.102:40967>"
\end{verbatim}
  
If neither \Opt{-name} or \Opt{-addr} are specified,
\Condor{cod\_request} attempts to connect to the \Condor{startd}
daemon running
on the local machine (where the request command was issued).

If the \Condor{startd} daemon to be used for the COD claim is an SMP
machine and has multiple virtual machines, specify which
resource on the machine to use for COD by providing a
\Opt{-requirements} option.  For example:

\begin{verbatim}
condor_cod_request -requirements 'VirtualMachineId==3'
\end{verbatim}
or
\begin{verbatim}
condor_cod_request -requirements 'State!="Claimed"'
\end{verbatim}

In general, be careful with shell quoting issues, so that
your shell is not confused by the ClassAd expression syntax (in
particular if the expression includes a string).
The safest method is to enclose any requirement expression you provide
within single quote marks (as shown above).
 
Once a given \Condor{startd} daemon has been contacted to request a new COD
claim, the \Condor{startd} daemon checks for proper
authorization of the user
issuing the command.
If the user has the authority, and the \Condor{startd} daemon
finds a
resource that matches any given requirements,
the \Condor{startd} daemon
creates a new COD claim and gives it a unique identifier,
the claim ID.
This ID is used to identify COD claims when using other commands.
If \Condor{cod\_request} succeeds, the claim ID for the new claim is printed
out to the screen.
All other commands to manage this claim require the claim ID to be
provided as a command-line option.

When the \Condor{startd} daemon assigns a COD claim,
the ClassAd describing the resource is returned to the user that
requested the claim. 
This ClassAd is a snap-shot of 
the output of \verb@condor_status -long@ for the given machine.
If \Condor{cod\_request} is invoked with the \Opt{-classad} option
(which takes a file name as an argument), this ClassAd will be written
out to the given file.
Otherwise, the ClassAd is printed to the screen.
The only essential piece of information in this ClassAd is the Claim
ID, so that is printed to the screen, even if the whole ClassAd is
also being written to a file.

\Note Once a COD claim is created, there is no persistent record of it
kept by the \Condor{startd} daemon.
So, if the \Condor{startd} daemon is restarted for any reason, all existing
COD claims will be destroyed and the new \Condor{startd} daemon will not
recognize any attempts to use the previous claims.

% Karen has editted to this point in the file


%%%%%%%%%%%%%%%%%%%%%%%%%%%%%%%%%%%%%%%%%%%%%%%%%%%%%%%%%%%%
\subsubsection{\label{sec:cod-claim-activate}Activate}
%%%%%%%%%%%%%%%%%%%%%%%%%%%%%%%%%%%%%%%%%%%%%%%%%%%%%%%%%%%%
\index{COD!\Condor{cod\_activate} command} 

\Todo


%%%%%%%%%%%%%%%%%%%%%%%%%%%%%%%%%%%%%%%%%%%%%%%%%%%%%%%%%%%%
\subsubsection{\label{sec:cod-claim-suspend}Suspend}
%%%%%%%%%%%%%%%%%%%%%%%%%%%%%%%%%%%%%%%%%%%%%%%%%%%%%%%%%%%%
\index{COD!\Condor{cod\_suspend} command} 

\Todo


%%%%%%%%%%%%%%%%%%%%%%%%%%%%%%%%%%%%%%%%%%%%%%%%%%%%%%%%%%%%
\subsubsection{\label{sec:cod-claim-resume}Resume}
%%%%%%%%%%%%%%%%%%%%%%%%%%%%%%%%%%%%%%%%%%%%%%%%%%%%%%%%%%%%
\index{COD!\Condor{cod\_resume} command} 

\Todo



%%%%%%%%%%%%%%%%%%%%%%%%%%%%%%%%%%%%%%%%%%%%%%%%%%%%%%%%%%%%
\subsubsection{\label{sec:cod-claim-deactivate}Deactivate}
%%%%%%%%%%%%%%%%%%%%%%%%%%%%%%%%%%%%%%%%%%%%%%%%%%%%%%%%%%%%
\index{COD!\Condor{cod\_deactivate} command} 

\Todo


%%%%%%%%%%%%%%%%%%%%%%%%%%%%%%%%%%%%%%%%%%%%%%%%%%%%%%%%%%%%
\subsubsection{\label{sec:cod-claim-release}Release}
%%%%%%%%%%%%%%%%%%%%%%%%%%%%%%%%%%%%%%%%%%%%%%%%%%%%%%%%%%%%
\index{COD!\Condor{cod\_release} command} 

\Todo


%%%%%%%%%%%%%%%%%%%%%%%%%%%%%%%%%%%%%%%%%%%%%%%%%%%%%%%%%%%%%%%%%%%%%%
\subsection{\label{sec:cod-limitations}Limitations of COD Support in Condor}
%%%%%%%%%%%%%%%%%%%%%%%%%%%%%%%%%%%%%%%%%%%%%%%%%%%%%%%%%%%%%%%%%%%%%%
\index{COD!limitations} 

Condor's support for COD has a few limitations.

The following items are all limitations we plan to remove in future
releases of Condor:

\begin{itemize}

\item Applications must be pre-staged at a given machine. 

\item There is no way to define limits for how long a given COD claim
  can be active, how often it is run, and so on.

\item There is no accounting done for applications run under COD
  claims.
  Therefore, if you use a lot of COD resources in a given Condor pool,
  it does not adversely affect your user priority.

\end{itemize}

None of the above items are fundamentally difficult to add and we hope
to address them relatively quickly.
If you run into one of these limitations and it is a barrier to you
using COD, please contact \Email{condor-admin@cs.wisc.edu} with the
subject ``COD limitation'' to gain quick help.

The following list are more fundamental limitations that we do not
plan to address:

\begin{itemize}

\item COD claims are not persistent on a given \Condor{startd} daemon.

\item Condor does not provide a mechanism to parallelize a graphic
  application to take advantage of COD.  
  The Condor Team is not in the business of developing applications,
  we just provide mechanisms to execute them.

\end{itemize}
\index{COD (Computing on Demand)|)}
