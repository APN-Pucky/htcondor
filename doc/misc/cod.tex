%%%%%%%%%%%%%%%%%%%%%%%%%%%%%%%%%%%%%%%%%%%%%%%%%%%%%%%%%%%%%%%%%%%%%%
\section{\label{sec:cod}Computing On Demand (COD)}
%%%%%%%%%%%%%%%%%%%%%%%%%%%%%%%%%%%%%%%%%%%%%%%%%%%%%%%%%%%%%%%%%%%%%%
\index{Computing On Demand}
\index{COD}

This section describes how to execute short-running interactive
applications on batch resources controlled by Condor.
This functionality is known as \Term{Computing on Demand} or
\Term{COD}.
Support for Computing On Demand was added to Condor in version 6.5.2.   

In the following sections, we will describe the concept and what kinds
of applications it is meant to be used for, give an overview of how it
works and interacts with Condor, describe how to configure Condor
resources to provide this service, explain how to use various tools
included with Condor to manage COD applications, and explore some
current limitations in Condor's support for COD.

%%%%%%%%%%%%%%%%%%%%%%%%%%%%%%%%%%%%%%%%%%%%%%%%%%%%%%%%%%%%%%%%%%%%%%
\subsection{\label{sec:cod-intro}
Introduction to COD}
%%%%%%%%%%%%%%%%%%%%%%%%%%%%%%%%%%%%%%%%%%%%%%%%%%%%%%%%%%%%%%%%%%%%%%
\index{Computing On Demand!Introduction} 

The basic motivation behind COD is to enable people to use Condor to
manage interactive, yet compute-intensive jobs.
These are jobs that take lots of compute power over a relatively short
period of time.
They want computing power \Term{on demand}, but they want to use their
existing batch computing resources to provide the cycles.
The applications they are trying to run cannot use the batch
scheduling functionality of Condor, since they require
\Term{interactive} response-time.
Many of the applications that are well-suited for COD involve a cycle:
application blocked on user input, computation burst to compute
results, block for more user input, computation burst, etc.
When the resources are not being used for the bursts of computation to
service the interactive application, they should continue to execute
long-running batch jobs.

Here are some example applications to consider:

\begin{itemize}

\item A giant spreadsheet with a large number of highly complex
  formulas which take a lot of compute power to recalculate.
  The user's spreadsheet application would claim a number of
  compute-server helper nodes.
  When the user presses a \verb@recalculate@ button, these worker
  nodes would work on a subset of the computation and send the results
  back to the master application providing the user interface and
  displaying the data.
  Ideally, while the user is entering new data or modifying formulas,
  these worker nodes should not tie up the batch resources where they
  are running.

\item Some kind of graphics rendering application that waits for user
  input to select an image to render, that then requires a huge burst
  of computation to produce the image.
  For example: various Computer-Aided Design (CAD) tools, fractal
  rendering programs, ray-tracing tools, etc.
 
\item Visualization tools for data mining.

\end{itemize}

The way Condor helps these kinds of applications is to provide an
infrastructure to use Condor batch resources for the types of compute
nodes described above.
Condor does \emph{NOT} provide tools to parallelize existing GUI
applications.
The COD functionality is an interface to allow these compute nodes to
interact with long-running Condor batch jobs.
The user must provide both the compute node applications, and the
interactive master application that controls them.
Condor only provides a mechanism to allow these interactive (and often
parallelized) applications to seamlessly interact with the Condor
batch system.


%%%%%%%%%%%%%%%%%%%%%%%%%%%%%%%%%%%%%%%%%%%%%%%%%%%%%%%%%%%%%%%%%%%%%%
\subsection{\label{sec:cod-overview}
Overview of How COD Works}
%%%%%%%%%%%%%%%%%%%%%%%%%%%%%%%%%%%%%%%%%%%%%%%%%%%%%%%%%%%%%%%%%%%%%%
\index{Computing On Demand!Overview of how it works}

\Todo

% (Text from condor-week 2003 presentation)
% When a high-priority COD job appears, the lower-priority batch job is
% suspended
% The COD job can run right away, while the batch job is suspended
% Batch jobs (even those that can not checkpoint) can resume instantly
% once there are no more active COD jobs
%
% The interactive COD application starts up, and goes out to claim some
% compute nodes
% Once the helper applications are in place and ready, these COD claims
% are suspended, allowing batch jobs to run
% When the interactive application has work, it can instantly suspend
% the batch jobs and resume the COD applications to perform the
% computations


%%%%%%%%%%%%%%%%%%%%%%%%%%%%%%%%%%%%%%%%%%%%%%%%%%%%%%%%%%%%%%%%%%%%%%
\subsection{\label{sec:cod-authorizing}
Authorizing Users to Create and Manage COD Claims}
%%%%%%%%%%%%%%%%%%%%%%%%%%%%%%%%%%%%%%%%%%%%%%%%%%%%%%%%%%%%%%%%%%%%%%
\index{Computing On Demand!Authorizing Users}

Once a user has a COD claim on a resource that they can suspend and
resume at will, that user therefore has the power to suspend and
resume whatever batch job might be running on that resource, even if
it is owned by another user.
Because of this, it is essential that users of COD resources can be
trusted not to abuse this power.
As a result, users must be authorized to have access to the privilege
of creating and using a COD claim on a machine.
This privilege is granted when the Condor administrator places a given
username in the \Macro{VALID\_COD\_USERS} list in the condor
configuration for the machine (usually in a local config file).

In addition, the tools to request and manage COD claims (described
below) all require that the user issuing the commands be
authenticated. 
This requires more than host-based authorization in Condor described
in section~\ref{sec:Host-Security} ``Setting Up IP/Host-Based
Security in Condor'' on page~\pageref{sec:Host-Security}.
If your site is not using the strong authentication methods described
in section~\ref{sec:Config-Security} ``Security Configuration'' on
page~\pageref{sec:Config-Security}, then to request a COD claim on a
given machine, you must log in directly to that machine and issue the
command locally.
This way, Condor can rely on the local file system to authenticate the
user.
This is analogous to commands that submit or remove jobs from the job
queue managed by a given \Condor{schedd}.
Unless some form of strong authentication such as GSI or Kerberos is
enabled at your Condor pool, you must issue \Condor{submit} and
\Condor{rm} commands from the same machine where the \Condor{schedd}
is running.


%%%%%%%%%%%%%%%%%%%%%%%%%%%%%%%%%%%%%%%%%%%%%%%%%%%%%%%%%%%%%%%%%%%%%%
\subsection{\label{sec:cod-setup}
Defining a COD Application}
%%%%%%%%%%%%%%%%%%%%%%%%%%%%%%%%%%%%%%%%%%%%%%%%%%%%%%%%%%%%%%%%%%%%%%
\index{Computing On Demand!Defining an application}

To run an application under a COD resource, a user must define certain
characteristics about the application.
For example, they must specify what executable or script to use, what
directory to run the application in, optionally define command-line
arguments, files to use for standard input and output, and so on.

COD users must specify a ClassAd that describes these characteristics
of their application.  
There are two ways for a user to define a COD application's ClassAd:

\begin{enumerate}
\item In the Condor configuration files of the COD resources
\item When they use the \Condor{cod} command-line tool to launch the
application itself
\end{enumerate}

These two methods for defining the ClassAd can be used together.
For example, a user can define most attributes for a given application
in the configuration file, and only provide a few dynamically defined
attributes with the \Condor{cod} tool.
(NOTE: detailed instructions on using the \Condor{cod} tool will be
provided below in another section).

Regardless of how the COD application's ClassAd is defined (locally in
the Condor configuration files, remotely with the \Condor{cod} tool,
or some combination of both) the application's executable and input
data must be pre-installed on the COD resource.
This is a current limitation of Condor's support for COD that will
eventually go away.
For now, there is no mechanism to transfer files for a COD
application, and all I/O must be performed locally or onto a network
filesystem that is accessible via a given COD resource.

The following three sections describe this process in more detail.
The first lists all the attributes that can be used to define a COD
application.
The second describes how to define these attributes in the Condor
configuration files.
The third explains how to define them using the \Condor{cod} tool.


%%%%%%%%%%%%%%%%%%%%%%%%%%%%%%%%%%%%%%%%%%%%%%%%%%%%%%%%%%%%%%%%%%%%%%
\subsubsection{\label{sec:cod-application-attributes}
Attributes That Describe a COD Application}
%%%%%%%%%%%%%%%%%%%%%%%%%%%%%%%%%%%%%%%%%%%%%%%%%%%%%%%%%%%%%%%%%%%%%%

\index{Computing On Demand!Defining Applications!Required attributes}

The following list of attributes are required for a COD application:

\begin{description}

 \item[\Attr{Cmd}] This attribute name is short for \Term{Command}.
   It defines the full path to the executable program to be run as a
   COD application.
   Since Condor does not currently provide any mechanism to transfer
   files on behalf of COD applications, this path should be a valid
   path on the machine where the application will be run.
   This is a string attribute, and must therefore be enclosed in
   quotation marks (\verb@"@).
   There is no default.

 \item[\Attr{IWD}] This setting is an acronym for \Term{Initial
   Working Directory}.
   It defines the full path to the directory where a given COD
   application should be run.
   Unless the application changes its current working directory, all
   relative path names used by the application will be relative to
   this path.
   If any other attributes that define filenames (for example,
   \Attr{In}, \Attr{Out}, and so on) do not contain a full path, the
   \Attr{IWD} will automatically be prepended to those filenames.
   This is a string attribute, and must therefore be enclosed in 
   quotation marks (\verb@"@).
   There is no default.

 \item[\Attr{Owner}] If the \Condor{startd} is executing as root on
   the resource where a COD application will run, the user must also
   define \Attr{Owner} to specify what user name the application will
   run as.
   (On Windows, the \Condor{startd} always runs as an Administrator
   service, which is equivalent to running as root on UNIX platforms).
   If the user specifies any COD application attributes with the
   \Condor{cod\_activate} command-line tool, the \Attr{Owner}
   attribute will be filled in with the user name that ran
   \Condor{cod\_activate}.
   However, if the user defines all attributes of their COD
   application in the Condor configuration files, and does not define
   any attributes with the \Condor{cod\_activate} command-line tool
   (both methods are described below in more detail), there is no
   default and \Attr{Owner} must be specified in the configuration
   file.
   \Attr{Owner} is a string attribute, and must contain a valid user
   name on the given COD resource. 

\end{description}


\index{Computing On Demand!Defining Applications!Optional attributes}

The following list of attributes are optional:

\begin{description}

 \item[\Attr{In}] This string defines the path to the file on the
   execution machine that should be used as standard input (STDIN) for 
   the COD application.
   This file (and all parent directories) must be readable by whatever
   user the COD application will run as.
   If this attribute is not specified, the default is to use
   \File{/dev/null}.
 
 \item[\Attr{Out}] This string defines the path to the file on the
   execution machine that should be used as standard output (STDOUT)
   for the COD application.
   This file must be writable (and all parent directories readable) by
   whatever user the COD application will run as.
   If this attribute is not specified, the default is to use
   \File{/dev/null}.
 
 \item[\Attr{Err}] This string defines the path to the file on the
   execution machine that should be used as standard error (STDERR)
   for the COD application.
   This file must be writable (and all parent directories readable) by
   whatever user the COD application will run as.
   If this attribute is not specified, the default is to use
   \File{/dev/null}.

 \item[\Attr{Env}] This string defines what environment variables will
   be set for a given COD application.
   Each environment variable has the form \verb@NAME=value@.
   Multiple variables are delimited with a semicolon \verb@;@.
   For example: \verb@Env = "PATH=/usr/local/bin:/usr/bin;TERM=vt100"@ 

 \item[\Attr{Args}] This string attribute defines the list of
   arguments to be supplied to the program on the command-line.
   The arguments are delimited (separated) by space characters. 
   There is no default. 
   If the \Attr{JobUniverse} (described below) corresponds to the Java
   universe, the first argument must be the name of the class
   containing \Code{main}.

 \item[\Attr{JobUniverse}] This attribute defines what Condor job
   universe to use for the given COD application.
   At this point, the only supported universes are vanilla and Java.
   This attribute must be an integer, with vanilla using the value 5,
   and Java the value 10.
   If \Attr{JobUniverse} is not specified, the vanilla universe is
   used by default.
   For more information about the Condor job universes, see
   section~\ref{sec:Choosing-Universe} on
   page~\pageref{sec:Choosing-Universe}. 

 \item[\Attr{JarFiles}] This string attribute is only used if
   \Attr{JobUniverse} (described above) is set to \verb@10@ for the
   Java universe.
   If a given COD application is a Java program, you can specify what
   JAR files that program requires with this attribute.
   There is no default.

 \item[\Attr{KillSig}] This attribute specifies what signal should be
   sent whenever the Condor system needs to gracefully shutdown the
   COD application.
   It can either be specified as a string containing the signal name
   (for example \verb@KillSig = "SIGQUIT"@), or as an integer
   (\verb@KillSig = 3@)
   The default is to use SIGTERM.

 \item[\Attr{StarterUserLog}] This string specifies a filename for a
   log file the \Condor{starter} can write with entries for relevant 
   events in the life of a given COD application.
   It is very similar to the UserLog file specified for regular Condor
   jobs with the \Attr{Log} setting in the \Condor{submit} description
   file.
   However, certain attributes that are placed in the regular UserLog
   file do not make sense in the COD environment, and are therefore
   left out.
   The default is not to write this log file.

 \item[\Attr{StarterUserLogUseXML}] If the \Attr{StarterUserLog}
   attribute (described above) is defined, the default format is a
   human-readable format.
   However, Condor can write out this log in an XML representation,
   instead.
   To enable the XML format for this UserLog, the
   \Attr{StarterUserLogUseXML} boolean can be set to \verb@TRUE@.
   The default if not specified is \verb@FALSE@.

\end{description}

\Note If any path attribute (\Attr{Cmd}, \Attr{In},
\Attr{Out},\Attr{Err}, \Attr{StarterUserLog}) is not a full path name,
Condor automatically prepends the value of \Attr{IWD}.


\index{Computing On Demand!Defining Applications!Job ID}

The final set of attributes define a COD application's \Term{job ID}.
The job ID is made up of both the \Attr{ClusterId} and \Attr{ProcId}
attributes (described below).
This job ID is similar to the job ID that is created whenever a
regular batch job is submitted to the Condor queue.
When using COD, the job ID is only used to identify the job in various
log messages and in the COD-specific output of \Condor{status}
(described below).

The COD job ID is part of the information included in all
events written to the \Attr{StarterUserLog} (described above)
regarding a given job.
The COD job ID is also used in the Condor debugging logs described in
section~\ref{sec:Daemon-Logging-Config-File-Entries} on
page~\pageref{sec:Daemon-Logging-Config-File-Entries}
For example, in the \Condor{starter}'s log file for COD jobs
(\File{StarterLog.cod} by default) or in the \Condor{startd}'s log
file (\File{StartLog} by default).

All COD command-line tools use the COD \Term{claim ID} described below
in section~\ref{sec:cod-managing-claims} on
page~\pageref{sec:cod-managing-claims} to uniquely identify a given
instance of a COD application.
Therefore, under COD the job ID is only useful to define if it helps a
user with accounting or debugging of their own application.
There is no special significance to COD for these attributes, and
therefore, they are optional.
  
\begin{description}
 \item[\Attr{ClusterId}] This integer defines the
   \Term{cluster identifier} for a COD job.
   The default value is 1.
   The \Attr{ClusterId} can also be defined with the
   \Condor{cod\_activate} command-line tool using the \Opt{-cluster}
   option.

 \item[\Attr{ProcId}]  This integer defines the
   \Term{process identifier} for a COD job.
   The default value is 0.
   The \Attr{ProcId} can also be defined with the
   \Condor{cod\_activate} command-line tool using the \Opt{-cluster}
   option.

\end{description}


%%%%%%%%%%%%%%%%%%%%%%%%%%%%%%%%%%%%%%%%%%%%%%%%%%%%%%%%%%%%%%%%%%%%%%
\subsubsection{\label{sec:cod-config-attrs}
Defining Attributes in the Condor Configuration Files}
\index{Computing On Demand!Defining Applications!In Condor
configuration files}
%%%%%%%%%%%%%%%%%%%%%%%%%%%%%%%%%%%%%%%%%%%%%%%%%%%%%%%%%%%%%%%%%%%%%%


To define COD attributes in the Condor configuration file for a given
application, a user must select an arbitrary \Term{COD keyword} for
the application.
This case-insensitive keyword is used as a prefix for the various
configuration file settings to define the application.
When a user wishes to spawn a given application, they simply provide
this keyword as an argument to the \Condor{cod} tool and the keyword
is used at the remote COD resource to find attributes which define the
application.

Any of the ClassAd attributes described in the previous section can be
specified in the configuration file with the keyword prefix and an
underscore character (\verb@"_"@).

For example, if the user's keyword for a given fractal generation
application is ``FractGen'', the resulting entries in the Condor
configuration file might look like this:

\begin{verbatim}
FractGen_Cmd = "/usr/local/bin/fractgen"
FractGen_Iwd = "/tmp/cod-fractgen"
FractGen_Out = "/tmp/cod-fractgen/output"
FractGen_Err = "/tmp/cod-fractgen/error"
FractGen_Args = "mandelbrot -0.65865,-0.56254 -0.45865,-0.71254"
\end{verbatim}

In this example, the ``fractgen'' binary might create other files.
The \Attr{Out} and \Attr{Err} attributes specified in the
configuration file are only for standard output and standard error
redirection.

When the user wanted to spawn an instance of this application, they
would specify a \verb@-keyword FractGen@ command-line argument when
they invoked \Condor{cod\_activate} (described below).

\Note If a user is defining all attributes of their COD application in
the Condor configuration files, and the \Condor{startd} on the COD
resource they are using is running as root, the user must also define
\Attr{Owner} to be the user that the COD application should run as
(see section~\ref{sec:cod-application-attributes} above). 


%%%%%%%%%%%%%%%%%%%%%%%%%%%%%%%%%%%%%%%%%%%%%%%%%%%%%%%%%%%%%%%%%%%%%%
\subsubsection{\label{sec:cod-command-line-attrs}
Defining Attributes with the \Condor{cod} Tool} 
\index{Computing On Demand!Defining Applications!With the \Condor{cod}
tool}
%%%%%%%%%%%%%%%%%%%%%%%%%%%%%%%%%%%%%%%%%%%%%%%%%%%%%%%%%%%%%%%%%%%%%%

COD users can define attributes dynamically at the time they spawn a
given COD application.
In this case, the ClassAd attributes are written into a file and the
filename is passed to the \Condor{cod\_activate} tool using the
\Opt{-jobad} command-line option.
These attributes are read by the \Condor{cod} tool and passed through
the system onto the Condor{starter} which spawns the COD application. 
If the filename given is \File{-}, the \Condor{cod} tool will read
from standard input (STDIN).

% TODO
% For more information about using \Condor{cod\_activate}, see the
% command reference on page~\pageref{man-condor-submit}

Users should not add a keyword prefix (as described above in the
section on using the configuration file) when defining attributes with
the \Condor{cod\_activate} tool.
The attribute names can be used in the file directly.

\Warn The current syntax for this file is not the same as the syntax
in the file used with \Condor{submit}.

\Note Users should not define \Attr{Owner} in this case, since Condor
will automatically insert the right value based on what user runs the
\Condor{cod\_activate} command and how that user authenticates to the
COD resource.
If a user defines something that does not match their authenticated
identity, Condor will treat it as an error and fail to launch the
application.


%%%%%%%%%%%%%%%%%%%%%%%%%%%%%%%%%%%%%%%%%%%%%%%%%%%%%%%%%%%%%%%%%%%%%%
\subsection{\label{sec:cod-managing-claims}
Managing COD Resource Claims}
%%%%%%%%%%%%%%%%%%%%%%%%%%%%%%%%%%%%%%%%%%%%%%%%%%%%%%%%%%%%%%%%%%%%%%
\index{Computing On Demand!Managing claims}

There are a number of commands provided by Condor which manage COD
claims on batch resources.
Once created, each COD claim has a unique identifying string, the
\Term{claim ID}.
Most commands require a claim ID to specify which claim you wish to
act on. 
These commands are the means by which COD applications interact with
the rest of the Condor system
They should be issued by the controller application to manage its
compute nodes.
Briefly, the commands are:

\begin{description}

\item [Request] Create a new COD claim on a given resource.

\item [Activate] Spawn a specific application on a given COD claim.

\item [Suspend] Suspend a running application on a given COD claim.

\item [Resume] Resume a suspended application on a given COD claim.

\item [Deactivate] Shutdown an application but hold onto the COD claim
  for future use.

\item [Release] Destroy a given COD claim and shutdown any job that is
  currently running on it.

\end{description}

To issue these commands, a user or application would invoke the 
\Condor{cod} tool.
Each command can be specified as the first argument to this tool, or
the \Condor{cod} tool can be installed in such a way that the same
binary is used for a set of names, for example: \Condor{cod\_request},
\Condor{cod\_release} and so on.

In addition, there is now a \Opt{-cod} option to \Condor{status}.

The following sections describe each command in greater detail.

%%%%%%%%%%%%%%%%%%%%%%%%%%%%%%%%%%%%%%%%%%%%%%%%%%%%%%%%%%%%
\subsubsection{\label{sec:cod-claim-request}Request}
\index{Computing On Demand!Managing claims!Request}
%%%%%%%%%%%%%%%%%%%%%%%%%%%%%%%%%%%%%%%%%%%%%%%%%%%%%%%%%%%%

As described above, a given user must be granted access to create COD
claims at a certain machine.
In addition, if the user wishes to use these COD claims, the
application they wish to run must be pre-installed on the machine and
defined in the \File{condor\_config} file (or a local config file). 
Therefore, a user cannot simply request a COD claim at random.

When a user wants to create a new COD claim, they must specify exactly
what resource they want to create it on.
The user normally does this by specifying the name of the
\Condor{startd} that they wish to use by invoking
\Condor{cod\_request} with the \Opt{-name} option and supplying a
hostname.  For example:
\begin{verbatim}
condor_cod_request -name c02.cs.wisc.edu
\end{verbatim}
If the \Condor{startd} you wish to use belongs to a different Condor
pool than the one where you are executing the COD commands, you can
use the \Opt{-pool} option to provide the name of the central manager
machine of the other pool.  For example:
\begin{verbatim}
condor_cod_request -name c02.cs.wisc.edu -pool condor.cs.wisc.edu
\end{verbatim}

Another option is use \Opt{-addr} to provide the IP address and port
number where the \Condor{startd} is listening.
This information can be found in the \Condor{startd} ClassAd as the
attribute \Attr{StartdIpAddr} or by reading the log file when the
\Condor{startd} first starts up.
For example:
\begin{verbatim}
condor_cod_request -addr "<128.105.146.102:40967>"
\end{verbatim}
  
If neither \Opt{-name} or \Opt{-addr} are specified,
\Condor{cod\_request} tries to connect to the \Condor{startd} running
on the local machine where you executed the command to request the COD
claim.

If the \Condor{startd} you want to use for your COD claim is an SMP
machine and has multiple virtual machines, you can specify which
resource on the machine you wish to use for COD by providing a
\Opt{-requirements} option.  For example:

\begin{verbatim}
condor_cod_request -requirements 'VirtualMachineId==3'
\end{verbatim}
or
\begin{verbatim}
condor_cod_request -requirements 'State!="Claimed"'
\end{verbatim}

In general, you should be careful with shell quoting issues, so that
your shell is not confused by the ClassAd expression syntax (in
particular if the expression includes a string).
The safest method is to enclose any requirement expression you provide
within single quote marks (as shown above).
 
Once a given \Condor{startd} has been contacted to request a new COD
claim, the startd authorizes the user issuing the command as described
above in section~\ref{sec:cod-authorization} to ensure that the user
should be allowed to have a COD claim.
If the user is authorized and the \Condor{startd} could find a
resource that matches any given requirements, the \Condor{startd}
creates a new COD claim gives it a unique identifier, the \Term{claim
ID}.
This ID is used to identify COD claims for all the other commands.
If \Condor{cod\_request} succeeds, the ID for the new claim is printed
out to the screen.
All other commands to manage this claim require the Claim ID to be
provided as a command-line option.

When the \Condor{startd} gives out a COD claim on a given resource,
the ClassAd describing the resource is returned to the user that
requested the claim. 
This ClassAd is basically a snap-shot of what you would see by looking
at the output from \verb@condor_status -long@ for the given machine.
If \Condor{cod\_request} is invoked with the \Opt{-classad} option
(which takes a filename as an argument), this ClassAd will be printed
out to the given file.
Otherwise, the ClassAd will be printed to the screen.
The only essential piece of information in this ClassAd is the Claim
ID, so that is printed to the screen, even if the whole ClassAd is
also being written to a file.

\Note Once a COD claim is created, there is no persistent record of it
kept at the \Condor{startd}.
So, if the \Condor{startd} is restarted for any reason, all existing
COD claims will be destroyed and the new \Condor{startd} will not
recognize any attempts to use the old claims.


%%%%%%%%%%%%%%%%%%%%%%%%%%%%%%%%%%%%%%%%%%%%%%%%%%%%%%%%%%%%
\subsubsection{\label{sec:cod-claim-activate}Activate}
\index{Computing On Demand!Managing claims!Activate}
%%%%%%%%%%%%%%%%%%%%%%%%%%%%%%%%%%%%%%%%%%%%%%%%%%%%%%%%%%%%

\Todo


%%%%%%%%%%%%%%%%%%%%%%%%%%%%%%%%%%%%%%%%%%%%%%%%%%%%%%%%%%%%
\subsubsection{\label{sec:cod-claim-suspend}Suspend}
\index{Computing On Demand!Managing claims!Suspend}
%%%%%%%%%%%%%%%%%%%%%%%%%%%%%%%%%%%%%%%%%%%%%%%%%%%%%%%%%%%%

\Todo


%%%%%%%%%%%%%%%%%%%%%%%%%%%%%%%%%%%%%%%%%%%%%%%%%%%%%%%%%%%%
\subsubsection{\label{sec:cod-claim-resume}Resume}
\index{Computing On Demand!Managing claims!Resume}
%%%%%%%%%%%%%%%%%%%%%%%%%%%%%%%%%%%%%%%%%%%%%%%%%%%%%%%%%%%%

\Todo



%%%%%%%%%%%%%%%%%%%%%%%%%%%%%%%%%%%%%%%%%%%%%%%%%%%%%%%%%%%%
\subsubsection{\label{sec:cod-claim-deactivate}Deactivate}
\index{Computing On Demand!Managing claims!Deactivate}
%%%%%%%%%%%%%%%%%%%%%%%%%%%%%%%%%%%%%%%%%%%%%%%%%%%%%%%%%%%%

\Todo


%%%%%%%%%%%%%%%%%%%%%%%%%%%%%%%%%%%%%%%%%%%%%%%%%%%%%%%%%%%%
\subsubsection{\label{sec:cod-claim-release}Release}
\index{Computing On Demand!Managing claims!Release}
%%%%%%%%%%%%%%%%%%%%%%%%%%%%%%%%%%%%%%%%%%%%%%%%%%%%%%%%%%%%

\Todo


%%%%%%%%%%%%%%%%%%%%%%%%%%%%%%%%%%%%%%%%%%%%%%%%%%%%%%%%%%%%
\subsubsection{\label{sec:cod-claim-release}condor\_status -cod}
\index{Computing On Demand!condor\_status -cod}
%%%%%%%%%%%%%%%%%%%%%%%%%%%%%%%%%%%%%%%%%%%%%%%%%%%%%%%%%%%%

The \Opt{-cod} option to \Condor{status} can be used to view any COD
claims in a given Condor pool.  

\begin{verbatim}
Name        ID   ClaimState TimeInState RemoteUser JobId Keyword 
astro.cs.wi COD1 Idle        0+00:00:04 wright                   
chopin.cs.w COD1 Running     0+00:02:05 wright     3.0   fractgen
chopin.cs.w COD2 Suspended   0+00:10:21 wright     4.0   fractgen

               Total  Idle  Running  Suspended  Vacating  Killing
 INTEL/LINUX       3     1        1          1         0        0
       Total       3     1        1          1         0        0
\end{verbatim}

%
% TODO: describe what this information means
%

Please see the man page for \Condor{status} on
page~\pageref{man-condor-status} for more details about using
\Condor{status} and its other options.


%%%%%%%%%%%%%%%%%%%%%%%%%%%%%%%%%%%%%%%%%%%%%%%%%%%%%%%%%%%%%%%%%%%%%%
\subsection{\label{sec:cod-limitations}
Limitations of COD Support in Condor}
%%%%%%%%%%%%%%%%%%%%%%%%%%%%%%%%%%%%%%%%%%%%%%%%%%%%%%%%%%%%%%%%%%%%%%
\index{Computing On Demand!Limitations}

In this section we describe some of the current limitations of
Condor's support for COD.

The following items are all limitations we plan to remove in future
releases of Condor:

%
% TODO: Complete this list and give a little more detail in the
% explanations.  
%

\begin{itemize}

\item Applications have to be pre-installed and pre-staged at a given
  machine. 

\item There is no way to define limits for how long a given COD claim
  can be active, how often it is run, and so on.

\item There is no accounting done for applications run under COD
  claims.
  Therefore, if you use a lot of COD resources in a given Condor pool,
  it does not aversely effect your user priority.

\end{itemize}

None of the above items are fundamentally difficult to add and we hope
to address them relatively quickly.
If you run into one of these limitations and it is a barrier to you
using COD, please contact \Email{condor-admin@cs.wisc.edu} with the
subject ``COD limitation'' and we will help you as quickly as we can.

The following items are more fundamental limitations that we do not
plan to address:

\begin{itemize}

\item COD claims are not persistent on a given \Condor{startd}

\item Condor does not provide a mechanism to parallelize a graphic
  application to take advantage of COD.  
  The Condor Team is not in the business of developing applications,
  we just provide mechanisms to execute them.

\end{itemize}
