%%%%%%%%%%%%%%%%%%%%%%%%%%%%%%%%%%%%%%%%%%%%%%%%%%%5
\section{\label{classad-reference}
Condor's ClassAd Mechanism}
%%%%%%%%%%%%%%%%%%%%%%%%%%%%%%%%%%%%%%%%%%%%%%%%%%%5

\label{sec:classadref}
ClassAds are a flexible mechanism for representing the characteristics and
constraints of machines and jobs in the Condor system.  ClassAds are used
extensively in the Condor system to represent jobs, resources, submitters
and other Condor daemons.  An understanding of this mechanism is required
to harness the full flexibility of the Condor system.

A ClassAd is is a set of uniquely named expressions.  Each named expression
is called an \Term{attribute}.  Figure~\ref{ClassAd:example} shows an example 
of a ClassAd with ten attributes.

\begin{figure}[hbt]
\footnotesize
\begin{verbatim}
MyType       = "Machine"
TargetType   = "Job"
Machine      = "froth.cs.wisc.edu"
Arch         = "INTEL"
OpSys        = "SOLARIS251"
Disk         = 35882
Memory       = 128
KeyboardIdle = 173
LoadAvg      = 0.1000
Requirements = TARGET.Owner=="smith" || LoadAvg<=0.3 && KeyboardIdle>15*60
\end{verbatim}
\normalsize
\caption{\label{ClassAd:example}An example ClassAd}
\end{figure}

ClassAd expressions look very much like expressions in C, and are composed
of literals and attribute references composed with operators.  The difference
between ClassAd expressions and C expressions arise from the fact that ClassAd
expressions operate in a much more dynamic environment.  For example, an
expression from a machine's ClassAd may refer to an attribute in a job's 
ClassAd, such as \verb+TARGET.Owner+ in the above example.  The value and type 
of the attribute is not known until the expression is evaluated in an 
environment which pairs a specific job ClassAd with the machine ClassAd.

ClassAd expressions handle these uncertainties by defining all operators
to be \Term{total} operators, which means that they have well defined
behavior regardless of supplied operands.  This functionality is provided
through two distinguished values, \texttt{UNDEFINED} and \texttt{ERROR},
and defining all operators so that they can operate on all possible values
in the ClassAd system.  For example, the multiplication operator which usually
only operates on numbers, has a well defined behavior if supplied with values
which are not meaningful to multiply.  Thus, the expression 
\verb+10 * "A string"+ evaluates to the value \texttt{ERROR}.  Most operators
are \Term{strict} with respect to \texttt{ERROR}, which means that they evaluate
to \texttt{ERROR} if any of their operands are \texttt{ERROR}.  Similarly,
most operators are strict with respect to \texttt{UNDEFINED}.

\subsection{Syntax}
ClassAd expressions are formed by composing literals, attribute references and 
other sub-expressions with operators. 
\subsubsection{Literals}
Literals in the ClassAd language may be of integer, real, string, undefined or 
error types.  The syntax of these literals is as follows:
\begin{description}
	\item[Integer]  A sequence of continuous digits (i.e., \verb@[0-9]@).
		Additionally, the keywords \verb+TRUE+ and \verb+FALSE+ (case
		insensitive) are syntactic representations of the integers 1 and 0 
		respectively.

	\item[Real] Two sequences of continuous digits separated by a period
		(i.e., \verb@[0-9]+.[0-9]+@).

	\item[String] A double quote character, followed by an list of characters
		terminated by a double quote character.  A backslash character inside
		the string causes the following character to be considered as part of
		the string, irrespective of what that character is.

	\item[Undefined] The keyword \texttt{UNDEFINED} (case insensitive)
		represents the \texttt{UNDEFINED} value.

	\item[Error] The keyword \texttt{ERROR} (case insensitive)
		represents the \texttt{ERROR} value.
\end{description}

\subsubsection{Attributes}
Every expression in a ClassAd is named by an \Term{attribute name}.  Together,
the (name,expression) pair is called an \Term{attribute}.  An attributes may be
referred to in other expressions through its attribute name.

Attribute names are sequences of alphabetic characters, digits and underscores,
and may not begin with a digit.  All characters in the name are significant,
but case is \emph{not} significant.  Thus, \verb+Memory+, \verb+memory+ and 
\verb+MeMoRy+ all refer to the same attribute.

An \Term{attribute reference} consists of the name of the attribute being 
referenced, and an optional \Term{scope resolution prefix}.  The three 
prefixes that may be used are \texttt{MY.}, \texttt{TARGET.} and \texttt{ENV.}.
The semantics of supplying a prefix are discussed in 
Section~\ref{ClassAd:evaluation}.

\subsubsection{Operators}
The operators that may be used in ClassAd expressions are similar to those
available in C.  The available operators and their relative precedence is shown 
in figure~\ref{ClassAd:operator-fig}.
\begin{figure}[h]
\begin{verbatim}
  -                    (high precedence)
  *   / 
  +   -
  <   <=   >=   >
  ==  !=  =?=  =!=
  &&
  ||                   (low precedence) 
\end{verbatim}
\caption{\label{ClassAd:operator-fig}Relative precedence of ClassAd expression
operators}
\end{figure}
The operator with the highest precedence is the unary minus operator.  The
only operators which are unfamiliar are the \verb+=?=+ and \verb+=!=+
operators, which are discussed in Section~\ref{ClassAd:evaluation-meta}.

\subsection{Evaluation Semantics}
\label{ClassAd:evaluation}
The ClassAd mechanism's primary purpose is for matching entities who supply
constraints on candidate matches.  The mechanism is therefore defined to
carry out expression evaluations in the context of two ClassAds which are
testing each other for a potential match.  For example, the \Condor{negotiator}
evaluates the \Attr{Requirements} expressions of machine and job ClassAds to
test if they can be matched.  The semantics of evaluating such constraints
is defined below.

\subsubsection{Literals}
Literals are self-evaluating, Thus, integer, string, real, undefined and
error values evaluate to themselves.

\subsubsection{Attribute References}
Since the expression evaluation is being carried out in the context of two
ClassAds, there is a potential for namespace ambiguities.  The following
rules define the semantics of attribute references made by ad $A$ that is being 
evaluated in a context with another ad $B$:
\begin{enumerate}
	\item If the reference is prefixed by a scope resolution prefix, 
	\begin{itemize}
		\item If the prefix is \texttt{MY.}, the attribute is looked up in 
			ClassAd $A$.  If the named attribute does not exist in $A$, the
			value of the reference is \texttt{UNDEFINED}.  Otherwise, the
			value of the reference is the value of the expression bound to
			the attribute name.

		\item Similarly, if the prefix is \texttt{TARGET.}, the attribute is 
			looked up in ClassAd $B$.  If the named attribute does not exist in 
			$B$, the value of the reference is \texttt{UNDEFINED}.  Otherwise, 
			the value of the reference is the value of the expression bound to
			the attribute name.

		\item Finally, if the prefix is \texttt{ENV.}, the attribute is
			evaluated in the ``environment.''  Currently, the only attribute
			of the environment is \Attr{CurrentTime}, which evaluates to the
			integer value returned by the system call \texttt{time(2)}.
	\end{itemize}

	\item If the reference is not prefixed by a scope resolution prefix,
	\begin{itemize}
		\item If the attribute is defined in $A$, the value of the reference
			is the value of the expression bound to the attribute name in $A$.
		\item Otherwise, if the attribute is defined in $B$, the value of the
			reference is the value of the expression bound to the attribute
			name in $B$.
		\item Otherwise, if the attribute is defined in the environment, the
			value of the reference is the evaluated value in the environment.
		\item Otherwise, the value of the reference is \texttt{UNDEFINED}.
	\end{itemize}

	\item Finally, if the reference refers to an expression that is itself in 
	the process of being evaluated, there is a circular dependency in the 
	evaluation.  The value of the reference is \texttt{ERROR}.
\end{enumerate}

\subsubsection{Operators}
\label{ClassAd:evaluation-meta}
All operators in the ClassAd language are \Term{total}, and thus have well
defined behavior regardless of the supplied operands.  Furthermore, most
operators are \Term{strict} with respect to \texttt{ERROR} and 
\texttt{UNDEFINED}, and thus evaluate to \texttt{ERROR} (or \texttt{UNDEFINED})
if either of their operands have these exceptional values.

\begin{itemize}
	\item\textbf{Arithmetic operators:}  
	\begin{enumerate}
		\item The operators \verb@*@, \verb@/@, \verb@+@ and \verb@-@ operate 
		arithmetically only on integers and reals.

		\item Arithmetic is carried out in the same type as both operands,
		and type promotions from integers to reals are performed if one operand 
		is an integer and the other real.

		\item The operators are strict with respect to both \texttt{UNDEFINED} 
		and \texttt{ERROR}.  

		\item If either operand is not a numerical type, the value of the
		operation is \texttt{ERROR}.
	\end{enumerate}

	\item\textbf{Comparison operators:}
	\begin{enumerate}
		\item The comparison operators \verb@==@, \verb@!=@, \verb@<=@, 
		\verb@<@, \verb@>=@ and \verb@>@ operate on integers, reals and strings.

		\item Comparisons are carried out in the same type as both operands,
		and type promotions from integers to reals are performed if one operand
		is a real, and the other an integer.  Strings may not be converted to
		any other type, so comparing a string and an integer results in 
		\texttt{ERROR}.

		\item The operators \verb@==@, \verb@!=@, \verb@<=@, \verb@<@ and 
		\verb@>=@ \verb@>@ are strict with respect to both \texttt{UNDEFINED} 
		and \texttt{ERROR}.

		\item In addition, the operators \verb@=?=@ and \verb@=!=@ behave
		similar to \verb@==@ and \verb@!=@, but are not strict.  Semantically,
		the \verb@=?=@ tests if its operands are ``identical,'' i.e., have
		the same type and the same value.  For example, \verb@10 == UNDEFINED@ 
		and \verb@UNDEFINED == UNDEFINED@ both evaluate to \texttt{UNDEFINED},
		but \verb@10 =?= UNDEFINED@ and \verb@UNDEFINED =?= UNDEFINED@ 
		evaluate to \texttt{FALSE} and \texttt{TRUE} respectively.  The
		\verb@=!=@ operator test for the ``is not identical to'' condition.
	\end{enumerate}

	\item\textbf{Logical operators:}
	\begin{enumerate}
		\item The logical operators \verb@&&@ and \verb@||@ operate on 
		integers and reals.  The zero value of these types are considered 
		\texttt{FALSE} and non-zero values \texttt{TRUE}.

		\item The operators are \emph{not} strict, and exploit the 
		``don't care'' properties of the operators to squash \texttt{UNDEFINED}
		and \texttt{ERROR} values when possible.  For example,
		\verb@UNDEFINED && FALSE@ evaluates to \texttt{FALSE}, but	
		\verb@UNDEFINED || FALSE@ evaluates to \texttt{UNDEFINED}.

		\item Any string operand is equivalent to an \texttt{ERROR} operand.
	\end{enumerate}
\end{itemize}

\subsection{ClassAds in the Condor System}
The simplicity and flexibility of ClassAds is heavily exploited in the Condor
system.  ClassAds are not only used to represent machines and jobs in the 
Condor pool, but also other entities that exist in the pool such as 
checkpoint servers, submitters of jobs and master daemons.  Since arbitrary
expressions may be supplied and evaluated over these ads, users have a uniform
and powerful mechanism to specify constraints over these ads.  These constraints
may take the form of \Attr{Requirements} expressions in resource and job ads,
or queries over other ads.

\subsubsection{Requirements and Ranks}
This is the mechanism by which users specify the constraints over machines
and jobs respectively.  Requirements for machines are specified through
configuration files, while requirements for jobs are specified through the
submit command file.

In both cases, the \Attr{Requirements} expression specifies the correctness
criterion that the match must meet, and the \Attr{Rank} expression specifies
the desirability of the match (where higher numbers mean better matches).
For example, a job ad may contain the following expressions:
\footnotesize
\begin{verbatim}
Requirements = Arch=="SUN4u" && OpSys == "SOLARIS251"
Rank         = TARGET.Memory + TARGET.Mips
\end{verbatim}
\normalsize
In this case, the customer requires an UltraSparc computer running the Solaris 
2.5.1 operating system.  Among all such computers, the customer prefers those
with large physical memories and high MIPS ratings.  Since the \Attr{Rank} is
a user specified metric, \emph{any} expression may be used to specify the
perceived desirability of the match.  The \Condor{negotiator} runs algorithms
to deliver the ``best'' resource (as defined by the \Attr{Rank} expression)
while satisfying other criteria.

Similarly, owners of resources may place constraints and preferences on 
their machines.  For example,
\footnotesize
\begin{verbatim}
    Friend        = Owner == "tannenba" || Owner == "wright"
    ResearchGroup = Owner == "jbasney" || Owner == "raman"
    Trusted       = Owner != "rival" && Owner != "riffraff"
    Requirements  = Trusted && ( ResearchGroup || LoadAvg < 0.3 &&
                         KeyboardIdle > 15*60 )
    Rank          = Friend + ResearchGroup*10
\end{verbatim}
\normalsize
The above policy states that the computer will never run jobs owned by
users ``rival'' and ``riffraff,'' while the computer will always run a 
job submitted by members of the research group.  Furthermore, jobs
submitted by friends are preferred to other foreign jobs, and jobs submitted
by the research group are preferred to jobs submitted by friends. 

\textbf{Note:}  Because of the dynamic nature of ClassAd expressions, there
is no \emph{a priori} notion of an integer valued expression, a real valued
expression, etc.  However, it is intuitive to think of the \Attr{Requirements}
and \Attr{Rank} expressions as integer valued and real valued expressions
respectively.  If the actual type of the expression is not of the expected 
type, the value is assumed to be zero.

\subsubsection{Querying with ClassAd Expressions}
The flexibility of this system may also be used when querying ClassAds
through the \Condor{status} and \Condor{q} tools which allow users to
supply ClassAd constraint expressions from the command line.

For example, to find all computers which have had their keyboards idle for 
more than 20 minutes and have more than 100 MB of memory:
\footnotesize
\begin{verbatim}
% condor_status -const 'KeyboardIdle > 20*60 && Memory > 100'

Name       Arch     OpSys        State      Activity   LoadAv Mem  ActvtyTime

amul.cs.wi SUN4u    SOLARIS251   Claimed    Busy       1.000  128   0+03:45:01
aura.cs.wi SUN4u    SOLARIS251   Claimed    Busy       1.000  128   0+00:15:01
balder.cs. INTEL    SOLARIS251   Claimed    Busy       1.000  1024  0+01:05:00
beatrice.c INTEL    SOLARIS251   Claimed    Busy       1.000  128   0+01:30:02
...
...
                     Machines Owner Claimed Unclaimed Matched Preempting

    SUN4u/SOLARIS251        3     0       3         0       0          0
    INTEL/SOLARIS251       21     0      21         0       0          0
    SUN4x/SOLARIS251        3     0       3         0       0          0
           SGI/IRIX6        1     0       0         1       0          0
         INTEL/LINUX        1     0       1         0       0          0

               Total       29     0      28         1       0          0
\end{verbatim}
\normalsize

The similar flexibility exists in querying job queues in the Condor system.

