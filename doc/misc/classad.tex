%%%%%%%%%%%%%%%%%%%%%%%%%%%%%%%%%%%%%%%%%%%%%%%%%%%
\section{HTCondor's ClassAd Mechanism}\label{sec:classad-reference}
%%%%%%%%%%%%%%%%%%%%%%%%%%%%%%%%%%%%%%%%%%%%%%%%%%%
\index{ClassAd|(}

ClassAds are a flexible mechanism for representing the characteristics and
constraints of machines and jobs in the HTCondor system.  ClassAds are used
extensively in the HTCondor system to represent jobs, resources, submitters
and other HTCondor daemons.  An understanding of this mechanism is required
to harness the full flexibility of the HTCondor system.

A ClassAd is is a set of uniquely named expressions.  Each named expression
is called an \Term{attribute}.  Figure~\ref{ClassAd:example} 
shows ten attributes,
a portion of an example ClassAd.

\begin{figure}[hbt]
\footnotesize
\begin{verbatim}
MyType       = "Machine"
TargetType   = "Job"
Machine      = "froth.cs.wisc.edu"
Arch         = "INTEL"
OpSys        = "LINUX"
Disk         = 35882
Memory       = 128
KeyboardIdle = 173
LoadAvg      = 0.1000
Requirements = TARGET.Owner=="smith" || LoadAvg<=0.3 && KeyboardIdle>15*60
\end{verbatim}
\normalsize
\caption{\label{ClassAd:example}An example ClassAd}
\end{figure}

ClassAd expressions look very much like expressions in C, and are composed
of literals and attribute references composed with operators 
and functions.
The difference
between ClassAd expressions and C expressions arise from the fact that ClassAd
expressions operate in a much more dynamic environment.  For example, an
expression from a machine's ClassAd may refer to an attribute in a job's 
ClassAd, such as \verb+TARGET.Owner+ in the above example.  The value and type 
of the attribute is not known until the expression is evaluated in an 
environment which pairs a specific job ClassAd with the machine ClassAd.

ClassAd expressions handle these uncertainties by defining all operators
to be \Term{total} operators, which means that they have well defined
behavior regardless of supplied operands.  This functionality is provided
through two distinguished values, \texttt{UNDEFINED} and \texttt{ERROR},
and defining all operators so that they can operate on all possible values
in the ClassAd system.  For example, the multiplication operator which usually
only operates on numbers, has a well defined behavior if supplied with values
which are not meaningful to multiply.  Thus, the expression 
\verb+10 * "A string"+ evaluates to the value \texttt{ERROR}.  Most operators
are \Term{strict} with respect to \texttt{ERROR}, which means that they evaluate
to \texttt{ERROR} if any of their operands are \texttt{ERROR}.  Similarly,
most operators are strict with respect to \texttt{UNDEFINED}.

%%%%%%%%%%%%%%%%%%%%%%%%%%%%%%%%%%%%%%%%%%%%%%%%%%%
\subsection{\label{sec:classad-newandold}
ClassAds: Old and New}
%%%%%%%%%%%%%%%%%%%%%%%%%%%%%%%%%%%%%%%%%%%%%%%%%%%

ClassAds have existed for quite some time in two forms:
Old and New.
Old ClassAds were the original form and were used in HTCondor
until HTCondor version 7.5.0.
They were heavily tied to the HTCondor development libraries.
New ClassAds added new features
and were designed as a stand-alone library that could be used apart
from HTCondor.

In HTCondor version 7.5.1, HTCondor switched to using the New ClassAd library
for all use of ClassAds within HTCondor. The library is
placed into a compatibility mode so that HTCondor 7.5.1 is still able to exchange ClassAds with
older versions of HTCondor.

All user interaction with tools (such as \Condor{q}) as well as
output of tools is still compatible with Old ClassAds.
Before HTCondor version 7.5.1, New ClassAds were used only in the Job Router.
There are some syntax and behavior differences between Old and New
 ClassAds, all of which should remain invisible to users of HTCondor.

A complete description of New ClassAds can be found at
\URL{http://htcondor.org/classad/classad.html},
and in the ClassAd Language Reference Manual found on that web page.

Some of the features of New ClassAds that are \emph{not} in Old ClassAds are
lists, nested ClassAds, time values, and matching groups of ClassAds.
HTCondor has avoided using these features,
as using them makes it difficult to interact with older versions of HTCondor.
But, users can start using them if they do not need to interact with
versions of HTCondor older than 7.5.1.

The syntax varies slightly between Old and New ClassAds.
Here is an example ClassAd presented in both forms.
The Old form:

\begin{verbatim}
Foo = 3
Bar = "ab\"cd\ef"
Moo = Foo =!= Undefined
\end{verbatim}

The New form:

\begin{verbatim}
[
Foo = 3;
Bar = "ab\"cd\\ef";
Moo = Foo isnt Undefined;
]
\end{verbatim}

HTCondor will convert to and from Old ClassAd syntax as needed.

%%%%%%%%%%%%%%%%%%%%%%%%%%%%%%%%%%%%%%%%%%%%%%%%%%%
\subsubsection{New ClassAd Attribute References}
%%%%%%%%%%%%%%%%%%%%%%%%%%%%%%%%%%%%%%%%%%%%%%%%%%%

Expressions often refer to ClassAd attributes.
These attribute references work differently in Old ClassAds as compared
with New ClassAds.
In New ClassAds, 
an unscoped reference is looked for only in the local ClassAd. 
An \Term{unscoped reference} is an attribute that does not have a
\Attr{MY.} or \Attr{TARGET.} prefix.
The \Term{local ClassAd} may be described by an example.
Matchmaking uses two ClassAds: the job ClassAd and the machine ClassAd. 
The job ClassAd is evaluated to see if it is a match for the machine ClassAd.
The job ClassAd is the local ClassAd.
Therefore, in the \Attr{Requirements} attribute of the job ClassAd,
any attribute without the prefix \Attr{TARGET.} is looked up only
in the job ClassAd.
With New ClassAd evaluation, the use of the prefix \Attr{MY.} is eliminated,
as an unscoped reference can only refer to the local ClassAd.

The \Attr{MY.} and \Attr{TARGET.} scoping prefixes
only apply when evaluating an expression within the context of two ClassAds.
Two examples that exemplify this are matchmaking and machine policy evaluation.
When evaluating an expression within the context of a single ClassAd,
\Attr{MY.} and \Attr{TARGET.} are not defined.
Using them within the context of a single ClassAd will result
in a value of \Expr{Undefined}.
Two examples that exemplify evaluating an expression 
within the context of a single ClassAd are 
during user job policy evaluation,
and with the \Opt{-constraint} option to command-line tools.

New ClassAds have no \Attr{CurrentTime} attribute.
If needed, use the \Procedure{time} function instead.
In order to mimic Old ClassAd semantics in current versions of HTCondor,
all ClassAds have an implicit \Attr{CurrentTime} attribute,
with a value of \Procedure{time}.

In current versions of HTCondor, 
New ClassAds will mimic the evaluation behavior of Old ClassAds.
No configuration variables or submit description file contents should
need to be changed.
To eliminate this behavior and use only the semantics of New ClassAds, 
set the configuration variable
\Macro{STRICT\_CLASSAD\_EVALUATION} to \Expr{True}.
This permits testing expressions to see if any adjustment is required,
before a future version of HTCondor potentially makes New ClassAds
evaluation behavior the default or the only option. 

%%%%%%%%%%%%%%%%%%%%%%%%%%%%%%%%%%%%%%%%%%%%%%%%%%%
\subsection{Old ClassAd Syntax}
%%%%%%%%%%%%%%%%%%%%%%%%%%%%%%%%%%%%%%%%%%%%%%%%%%%
\index{ClassAd!expression syntax of Old ClassAds}
ClassAd expressions are formed by composing literals, attribute references and 
other sub-expressions with operators and functions. 
\subsubsection{Literals}
\label{ClassAd:literals}
Literals in the ClassAd language may be of integer, real, string, undefined or 
error types.  The syntax of these literals is as follows:
\begin{description}
	\item[Integer]  A sequence of continuous digits (i.e., \verb@[0-9]@).
		Additionally, the keywords \verb+TRUE+ and \verb+FALSE+ (case
		insensitive) are syntactic representations of the integers 1 and 0 
		respectively.

	\item[Real] Two sequences of continuous digits separated by a period
		(i.e., \verb@[0-9]+.[0-9]+@).

	\item[String] A double quote character, followed by an list of characters
		terminated by a double quote character.  A backslash character inside
		the string causes the following character to be considered as part of
		the string, irrespective of what that character is.

	\item[Undefined] The keyword \texttt{UNDEFINED} (case insensitive)
		represents the \texttt{UNDEFINED} value.

	\item[Error] The keyword \texttt{ERROR} (case insensitive)
		represents the \texttt{ERROR} value.
\end{description}

%%%%%%%%%%%%%%%%%%%%%%%%%%%%%%%%%%%%%%%%%%%%%%%%%%%
\subsubsection{Attributes}
%%%%%%%%%%%%%%%%%%%%%%%%%%%%%%%%%%%%%%%%%%%%%%%%%%%
\index{ClassAd!attributes}
Every expression in a ClassAd is named by an \Term{attribute name}.  Together,
the (name,expression) pair is called an \Term{attribute}.  An attribute may be
referred to in other expressions through its attribute name.

Attribute names are sequences of alphabetic characters, digits and underscores,
and may not begin with a digit.  All characters in the name are significant,
but case is \emph{not} significant.  Thus, \verb+Memory+, \verb+memory+ and 
\verb+MeMoRy+ all refer to the same attribute.

An \Term{attribute reference} consists of the name of the attribute being 
referenced, and an optional \Term{scope resolution prefix}.  The 
prefixes that may be used are \Attr{MY.} and \Attr{TARGET.}.
The case used for these prefixes is \emph{not} significant.
The semantics of supplying a prefix are discussed in 
Section~\ref{ClassAd:evaluation}.

%%%%%%%%%%%%%%%%%%%%%%%%%%%%%%%%%%%%%%%%%%%%%%%%%%%
\subsubsection{Operators}
%%%%%%%%%%%%%%%%%%%%%%%%%%%%%%%%%%%%%%%%%%%%%%%%%%%
\index{ClassAd!expression operators}
The operators that may be used in ClassAd expressions are similar to those
available in C.  The available operators and their relative precedence is shown 
in figure~\ref{ClassAd:operator-fig}.
\begin{figure}[ht]
\begin{verbatim}
  - (unary negation)   (high precedence)
  *   / 
  +   - (addition, subtraction)
  <   <=   >=   >
  ==  !=  =?=  is  =!=  isnt
  &&
  ||                   (low precedence) 
\end{verbatim}
\caption{\label{ClassAd:operator-fig}Relative precedence of ClassAd expression
operators}
\end{figure}
The operator with the highest precedence is the unary minus operator.  The
only operators which are unfamiliar are the \verb+=?=+, \verb+is+, \verb+=!=+ and \verb+isnt+
operators, which are discussed in Section~\ref{ClassAd:evaluation-meta}.


%%%%%%%%%%%%%%%%%%%%%%%%%%%%%%%%%%%%%%%%%%%%%%%%%%%
\subsubsection{\label{sec:classadFunctions}Predefined Functions}
%%%%%%%%%%%%%%%%%%%%%%%%%%%%%%%%%%%%%%%%%%%%%%%%%%%
\index{ClassAd!expression functions}
\index{ClassAd functions}

Any ClassAd expression may utilize predefined functions.
Function names are case insensitive.
Parameters to functions 
and a return value from a function 
may be typed (as given) or not.
Nested or recursive function calls are allowed.

Here are descriptions of each of these predefined functions.
The possible types are the same as itemized
in Section~\ref{ClassAd:literals}.
Where the type may be any of these literal types, it is
called out as \verb@AnyType@.
Where the type is \verb@Integer@, but only returns
the value 1 or 0 (implying \Expr{True} or \Expr{False}),
it is called out as \verb@Boolean@.
The format of each function is given as

\footnotesize
\begin{verbatim}
ReturnType FunctionName(ParameterType parameter1, ParameterType parameter2, ...)
\end{verbatim}
\normalsize
Optional parameters are given within square brackets.

\begin{description}
  \index{ClassAd functions!eval()}
  \item[\Code{AnyType eval(AnyType Expr)}]
  Evaluates \Expr{Expr} as a string and then returns the result of
  evaluating the \emph{contents} of the string as a ClassAd expression.
  This is useful when referring to an attribute such as \Expr{slotX\_State}
  where \Expr{X}, the desired slot number is an expression, such as
  \Expr{SlotID+10}.  
  In such a case, if attribute \Attr{SlotID} is 5,
  the value of the attribute \Attr{slot15\_State} can
  be referenced using the expression
  \Expr{eval(strcat("slot", SlotID+10,"\_State"))}.
  Function \Procedure{strcat} calls function \Procedure{string}
  on the second parameter, which evaluates the expression,
  and then converts the integer result 15 to the string \AdStr{15}.
  The concatenated string returned by \Procedure{strcat} is
  \AdStr{slot15\_State}, and this string is then evaluated.

  Note that referring to attributes of a job from within the string
  passed to \Procedure{eval} in the \Attr{Requirements} or
  \Attr{Rank} expressions could cause inaccuracies in HTCondor's
  automatic auto-clustering of jobs into equivalent groups for
  matchmaking purposes.  This is because HTCondor needs to determine
  which ClassAd attributes are significant for matchmaking purposes,
  and indirect references from within the string passed to \Procedure{eval}
  will not be counted.

  \index{ClassAd functions!unparse()}
  \item[\Code{String unparse(Attribute attr)}]
  This function looks up the value of the provided attribute and returns
  the unparsed version as a string. The attribute's value is not evaluated.
  If the attribute's value is \Expr{x + 3}, then the function would return
  the string \AdStr{x + 3}.
  If the provided attribute cannot be found, an empty string is returned.

  This function returns \Expr{ERROR} if other than exactly 1 argument is
  given or the argument is not an attribute reference.

  \index{ClassAd functions!ifThenElse()}
  \item[\Code{AnyType ifThenElse(AnyType IfExpr,AnyType ThenExpr, AnyType ElseExpr)}]
    A conditional expression is described by \Expr{IfExpr}.
    The following defines return values, when \Expr{IfExpr}
    evaluates to
    \begin{itemize}
    \item{\Expr{True}. Evaluate and return the value as given
      by \Expr{ThenExpr}.}
    \item{\Expr{False}. Evaluate and return the value as given
      by \Expr{ElseExpr}.}
    \item{\Expr{UNDEFINED}. Return the value \Expr{UNDEFINED}.}
    \item{\Expr{ERROR}. Return the value \Expr{ERROR}.}
    \item{\Expr{0.0}. Evaluate, and return the value as given
      by \Expr{ElseExpr}.}
    \item{non-\Expr{0.0} Real values. Evaluate, and return the value as given
      by \Expr{ThenExpr}.}
    \end{itemize}
    Where \Expr{IfExpr} evaluates to give a value of type \Expr{String},
    the function returns the value \Expr{ERROR}.
    The implementation uses lazy evaluation, so expressions
    are only evaluated as defined.

    This function returns \Expr{ERROR} if other than exactly 3
    arguments are given.


  \index{ClassAd functions!isUndefined()}
  \item[\Code{Boolean isUndefined(AnyType Expr)}]
    Returns \Expr{True}, if \Expr{Expr} evaluates to \Expr{UNDEFINED}.
    Returns \Expr{False} in all other cases.

    This function returns \Expr{ERROR} if other than exactly 1
    argument is given.

  \index{ClassAd functions!isError()}
  \item[\Code{Boolean isError(AnyType Expr)}]
    Returns \Expr{True}, if \Expr{Expr} evaluates to \Expr{ERROR}.
    Returns \Expr{False} in all other cases.

    This function returns \Expr{ERROR} if other than exactly 1
    argument is given.

  \index{ClassAd functions!isString()}
  \item[\Code{Boolean isString(AnyType Expr)}]
    Returns \Expr{True}, if the evaluation of \Expr{Expr}
    gives a value of type \Expr{String}.
    Returns \Expr{False} in all other cases.

    This function returns \Expr{ERROR} if other than exactly 1
    argument is given.

  \index{ClassAd functions!isInteger()}
  \item[\Code{Boolean isInteger(AnyType Expr)}]
    Returns \Expr{True}, if the evaluation of \Expr{Expr}
    gives a value of type \Expr{Integer}.
    Returns \Expr{False} in all other cases.

    This function returns \Expr{ERROR} if other than exactly 1
    argument is given.

  \index{ClassAd functions!isReal()}
  \item[\Code{Boolean isReal(AnyType Expr)}]
    Returns \Expr{True}, if the evaluation of \Expr{Expr}
    gives a value of type \Expr{Real}.
    Returns \Expr{False} in all other cases.

    This function returns \Expr{ERROR} if other than exactly 1
    argument is given.

  \index{ClassAd functions!isBoolean()}
  \item[\Code{Boolean isBoolean(AnyType Expr)}]
    Returns \Expr{True}, if the evaluation of \Expr{Expr}
    gives the integer value 0 or 1.
    Returns \Expr{False} in all other cases.

    This function returns \Expr{ERROR} if other than exactly 1
    argument is given.

  \index{ClassAd functions!int()}
  \item[\Code{Integer int(AnyType Expr)}]
    Returns the integer value as defined by \Expr{Expr}.
    Where the type of the evaluated \Expr{Expr} is \Expr{Real},
    the value is truncated (round towards zero) to an integer.
    Where the type of the evaluated \Expr{Expr} is \Expr{String},
    the string is converted to an integer using a C-like
    \Procedure{atoi} function. When this result is not an integer,
    \Expr{ERROR} is returned.
    Where the evaluated \Expr{Expr} is \Expr{ERROR} or \Expr{UNDEFINED},
    \Expr{ERROR} is returned.

    This function returns \Expr{ERROR} if other than exactly 1
    argument is given.

  \index{ClassAd functions!real()}
  \item[\Code{Real real(AnyType Expr)}]
    Returns the real value as defined by \Expr{Expr}.
    Where the type of the evaluated \Expr{Expr} is \Expr{Integer},
    the return value is the converted integer.
    Where the type of the evaluated \Expr{Expr} is \Expr{String},
    the string is converted to a real value using a C-like
    \Procedure{atof} function. When this result is not a real,
    \Expr{ERROR} is returned.
    Where the evaluated \Expr{Expr} is \Expr{ERROR} or \Expr{UNDEFINED},
    \Expr{ERROR} is returned.

    This function returns \Expr{ERROR} if other than exactly 1
    argument is given.

  \index{ClassAd functions!string()}
  \item[\Code{String string(AnyType Expr)}]
    Returns the string that results from the evaluation of \Expr{Expr}.
    Converts a non-string value to a string.
    Where the evaluated \Expr{Expr} is \Expr{ERROR} or \Expr{UNDEFINED},
    \Expr{ERROR} is returned.

    This function returns \Expr{ERROR} if other than exactly 1
    argument is given.

  \index{ClassAd functions!floor()}
  \item[\Code{Integer floor(AnyType Expr)}]
    Returns the integer that results from the evaluation of \Expr{Expr},
    where the type of the evaluated \Expr{Expr} is \Expr{Integer}.
    Where the type of the evaluated \Expr{Expr} is \emph{not} \Expr{Integer},
    function \Expr{real(Expr)} is called.
    Its return value is then used to return the largest magnitude
    integer that is not larger than the returned value. 
    Where \Expr{real(Expr)} returns \Expr{ERROR} or \Expr{UNDEFINED},
    \Expr{ERROR} is returned.

    This function returns \Expr{ERROR} if other than exactly 1
    argument is given.

  \index{ClassAd functions!ceiling()}
  \item[\Code{Integer ceiling(AnyType Expr)}]
    Returns the integer that results from the evaluation of \Expr{Expr},
    where the type of the evaluated \Expr{Expr} is \Expr{Integer}.
    Where the type of the evaluated \Expr{Expr} is \emph{not} \Expr{Integer},
    function \Expr{real(Expr)} is called.
    Its return value is then used to return the smallest magnitude
    integer that is not less than the returned value. 
    Where \Expr{real(Expr)} returns \Expr{ERROR} or \Expr{UNDEFINED},
    \Expr{ERROR} is returned.

    This function returns \Expr{ERROR} if other than exactly 1
    argument is given.

  \index{ClassAd functions!pow()}
  \item[\Code{Integer pow(Integer base, Integer exponent)}]
  \item[OR \Code{Real pow(Integer base, Integer exponent)}]
  \item[OR \Code{Real pow(Real base, Real exponent)}]
    Calculates \Code{base} raised to the power of \Code{exponent}.
    If \Code{exponent} is an integer value greater than or equal to 0,
    and \Code{base} is an integer, then an integer value is returned.
    If \Code{exponent} is an integer value less than 0, or if either
    \Code{base} or \Code{exponent} is a real, then a real value is
    returned.
    An invocation with \Expr{exponent=0} or \Expr{exponent=0.0},
    for any value of \Code{base}, including 0 or 0.0, returns the value 1
    or 1.0, type appropriate. 

  \index{ClassAd functions!quantize()}
  \item[\Code{Integer quantize(AnyType a, Integer b)}]
  \item[OR \Code{Real quantize(AnyType a, Real b)}]
  \item[OR \Code{AnyType quantize(AnyType a, AnyType list b)}]
    \Code{quantize()} computes the quotient of \Expr{a/b}, 
    in order to further compute \Expr{ ceiling(quotient) * b}.
    This computes and returns an integral multiple of \Code{b} that is 
    at least as large as \Code{a}.
    So, when \Expr{b >= a}, the return value will be \Code{b}.
    The return type is the same as that of \Code{b}, where \Code{b} 
    is an Integer or Real.

    When \Code{b} is a list, \Code{quantize()} returns the first value
    in the list that is greater than or equal to \Code{a}.
    When no value in the list is greater than or equal to \Code{a},
    this computes and returns an integral multiple of 
    the last member in the list that is at least as large as \Code{a}.

    This function returns \Expr{ERROR} if \Code{a} or \Code{b}, 
    or a member of the list that must be considered is not an Integer or Real.

    Here are examples:
\begin{verbatim}
     8     = quantize(3, 8)
     4     = quantize(3, 2)
     0     = quantize(0, 4)
     6.8   = quantize(1.5, 6.8)
     7.2   = quantize(6.8, 1.2)
     10.2  = quantize(10, 5.1)

     4     = quantize(0, {4})
     2     = quantize(2, {1, 2, "A"})
     3.0   = quantize(3, {1, 2, 0.5})
     3.0   = quantize(2.7, {1, 2, 0.5})
     ERROR = quantize(3, {1, 2, "A"})
\end{verbatim}

  \index{ClassAd functions!round()}
  \item[\Code{Integer round(AnyType Expr)}]
    Returns the integer that results from the evaluation of \Expr{Expr},
    where the type of the evaluated \Expr{Expr} is \Expr{Integer}.
    Where the type of the evaluated \Expr{Expr} is \emph{not} \Expr{Integer},
    function \Expr{real(Expr)} is called.
    Its return value is then used to return the 
    integer that results from a round-to-nearest rounding method. 
    The nearest integer value to the return value is returned,
    except in the case of the value at the exact midpoint between
    two integer values.  
    In this case, the even valued integer is returned.
    Where \Expr{real(Expr)} returns \Expr{ERROR} or \Expr{UNDEFINED},
    or the integer value does not fit into 32 bits,
    \Expr{ERROR} is returned.

    This function returns \Expr{ERROR} if other than exactly 1
    argument is given.

  \index{ClassAd functions!random()}
  \item[\Code{Integer random(\Lbr\ AnyType Expr \Rbr)}]
    Where the optional argument \Expr{Expr} evaluates to type \Expr{Integer}
    or type \Expr{Real}
    (and called \Expr{x}),
    the return value is the integer or real \Expr{r} randomly chosen
    from the interval \Expr{0 <= r < x}.
    With no argument, the return value is chosen with \Expr{random(1.0)}.
    Returns \Expr{ERROR} in all other cases.

    This function returns \Expr{ERROR} if greater than 1
    argument is given.

  \index{ClassAd functions!strcat()}
  \item[\Code{String strcat(AnyType Expr1 \Lbr\ , AnyType Expr2 \Dots \Rbr)}]
    Returns the string which is the concatenation of all arguments, where all arguments are 
    converted to type \Expr{String} by function \Expr{string(Expr)}.
    Returns \Expr{ERROR} if any argument evaluates to \Expr{UNDEFINED} or \Expr{ERROR}.

  \index{ClassAd functions!join()}
  \item[\Code{String join(String sep, AnyType Expr1 \Lbr\ , AnyType Expr2 \Dots \Rbr)}]
  \item[OR \Code{String join(String sep, List list}]
  \item[OR \Code{String join(List list}]
    Returns the string which is the concatenation of all arguments after the first one.
    The first argument is the separator, and it is inserted between each of the other arguments
    during concatenation.
    All arguments are converted to type \Expr{String} by function \Expr{string(Expr)} before
    concatenation. When there are exactly two arguments, If the second argument is a List,
    all members of the list are converted to strings and then joined using the separator.
    When there is only one argument, and the argument is a List, all members of the list
    are converted to strings and then concatenated.

    Returns \Expr{ERROR} if any argument evaluates to \Expr{UNDEFINED} or \Expr{ERROR}.

    For example:
\begin{verbatim}
    "a, b, c" = join(", ", "a", "b", "c")
    "abc"   = join(split("a b c"))
    "a;b;c" = join(";", split("a b c"))
\end{verbatim}

  \index{ClassAd functions!substr()}
  \item[\Code{String substr(String s, Integer offset \Lbr\ , Integer length  \Rbr)}]
    Returns the substring of \Expr{s}, from the position indicated by \Expr{offset},
    with (optional) \Expr{length} characters.
    The first character within \Expr{s} is at offset 0.
    If the optional \Expr{length} argument is not present, the substring extends to the
    end of the string.
    If \Expr{offset} is negative, the value \Expr{(length - offset)} is used for the offset.
    If \Expr{length} is negative, an initial substring is computed, from the offset
    to the end of the string.
    Then, the absolute value of \Expr{length} characters are deleted from the
    right end of the initial substring.
    Further, where characters of this resulting substring lie outside the original
    string, the part that lies within the original string is returned.
    If the substring lies completely outside of the original string, the null string
    is returned.

    This function returns \Expr{ERROR} if greater than 3 or less than 2
    arguments are given.
    
  \index{ClassAd functions!strcmp()}
  \item[\Code{Integer strcmp(AnyType Expr1, AnyType Expr2)}]
    Both arguments are converted to type \Expr{String} by function \Expr{string(Expr)}.
    The return value is an integer that will be
    \begin{itemize}
      \item{less than 0},
      if \Expr{Expr1} is lexicographically less than \Expr{Expr2}
      \item{equal to 0},
      if \Expr{Expr1} is lexicographically equal to \Expr{Expr2}
      \item{greater than 0},
      if \Expr{Expr1} is lexicographically greater than \Expr{Expr2}
    \end{itemize}
    Case is significant in the comparison.
    Where either argument evaluates to \Expr{ERROR} or \Expr{UNDEFINED},
    \Expr{ERROR} is returned.

    This function returns \Expr{ERROR} if other than 2 arguments are given.

  \index{ClassAd functions!stricmp()}
  \item[\Code{Integer stricmp(AnyType Expr1, AnyType Expr2)}]
    This function is the same as \Expr{strcmp}, except that letter case is
    \emph{not} significant.

  \index{ClassAd functions!toUpper()}
  \item[\Code{String toUpper(AnyType Expr)}]
    The single argument is converted to type \Expr{String} by function \Expr{string(Expr)}.
    The return value is this string, with all lower case letters converted to
    upper case.
    If the argument evaluates to \Expr{ERROR} or \Expr{UNDEFINED},
    \Expr{ERROR} is returned.

    This function returns \Expr{ERROR} if other than exactly 1
    argument is given.

  \index{ClassAd functions!toLower()}
  \item[\Code{String toLower(AnyType Expr)}]
    The single argument is converted to type \Expr{String} by function \Expr{string(Expr)}.
    The return value is this string, with all upper case letters converted to
    lower case.
    If the argument evaluates to \Expr{ERROR} or \Expr{UNDEFINED},
    \Expr{ERROR} is returned.

    This function returns \Expr{ERROR} if other than exactly 1
    argument is given.

  \index{ClassAd functions!size()}
  \item[\Code{Integer size(AnyType Expr)}]
    Returns the number of characters in the string, after calling function
    \Expr{string(Expr)}.
    If the argument evaluates to \Expr{ERROR} or \Expr{UNDEFINED},
    \Expr{ERROR} is returned.

    This function returns \Expr{ERROR} if other than exactly 1
    argument is given.

  \index{ClassAd functions!split()}
  \item[\Code{List split(String s \Lbr\ , String tokens \Rbr\ )}]
    Returns a list of the substrings of \Expr{s} that have been split up
    by using any of the characters within string \Expr{tokens}.
    If \Expr{tokens} is not specified, then all white space characters are
    used to delimit the string.

  \index{ClassAd functions!splitUserName()}
  \item[\Code{List splitUserName(String Name)}]
    Returns a list of two strings.
    Where \Code{Name} includes an \Expr{@} character,
    the first string in the list will be the substring that comes before
    the \Expr{@} character, 
    and the second string in the list will be the substring that comes after.
    Thus, if \Code{Name} is \AdStr{user@domain}, 
    then the returned list will be \verb|{"user", "domain"}|.
    If there is no \Expr{@} character in \Code{Name},
    then the first string in the list will be \Code{Name}, and the
    second string in the list will be the empty string.
    Thus, if \Code{Name} is \AdStr{username}, then the returned list
    will be \verb|{"username", ""}|.

  \index{ClassAd functions!splitSlotName()}
  \item[\Code{List splitSlotName(String Name)}]
    Returns a list of two strings.
    Where \Code{Name} includes an \Expr{@} character,
    the first string in the list will be the substring that comes before
    the \Expr{@} character, 
    and the second string in the list will be the substring that comes after.
    Thus, if \Code{Name} is \AdStr{slot1@machine}, then the returned list
    will be \verb|{"slot1", "machine"}|.
    If there is no \Expr{@} character in \Code{Name},
    then the first string in the list will be the empty string,
    and the second string in the list will be \Code{Name}, 
    Thus, if \Code{Name} is \AdStr{machinename}, then the returned list
    will be \verb|{"", "machinename"}|.

  \index{ClassAd functions!time()}
  \item[\Code{Integer time()}]
    Returns the current coordinated universal time.
    This is the time, in seconds, since midnight of January 1, 1970.

  \index{ClassAd functions!formatTime()}
  \item[\Code{String formatTime(\Lbr\ Integer time \Rbr\ \Lbr\ , String format \Rbr)}]

    Returns a formatted string that is a representation of \Expr{time}.
    The argument \Expr{time} is interpreted as coordinated universal time in
    seconds, since midnight of January 1, 1970. If not specified,
    \Expr{time} will default to the current time.
		
    The argument \Expr{format} is interpreted similarly to the format
    argument of the ANSI C strftime function. It consists of arbitrary text
    plus placeholders for elements of the time. These placeholders are
    percent signs (\Percent) followed by a single letter.
    To have a percent sign in
    the output, use a double percent sign (\Percent\Percent).  If
    \Expr{format} is not specified, it defaults to \Expr{\%c}.

    Because the implementation uses \Procedure{strftime} to implement this,
    and some
    versions implement extra, non-ANSI C options, the exact options
    available to an implementation may vary. An implementation is only
    required to implement the ANSI C options, which are: 
    \begin{description}
    \item[\Expr{\%a}] abbreviated weekday name
    \item[\Expr{\%A}] full weekday name
    \item[\Expr{\%b}] abbreviated month name
    \item[\Expr{\%B}] full month name
    \item[\Expr{\%c}] local date and time representation
    \item[\Expr{\%d}] day of the month (01-31)
    \item[\Expr{\%H}] hour in the 24-hour clock (0-23)
    \item[\Expr{\%I}] hour in the 12-hour clock (01-12)
    \item[\Expr{\%j}] day of the year (001-366)
    \item[\Expr{\%m}] month (01-12)
    \item[\Expr{\%M}] minute (00-59)
    \item[\Expr{\%p}] local equivalent of AM or PM
    \item[\Expr{\%S}] second (00-59)
    \item[\Expr{\%U}] week number of the year (Sunday as first day of week) (00-53)
    \item[\Expr{\%w}] weekday (0-6, Sunday is 0)
    \item[\Expr{\%W}] week number of the year (Monday as first day of week) (00-53)
    \item[\Expr{\%x}] local date representation
    \item[\Expr{\%X}] local time representation
    \item[\Expr{\%y}] year without century (00-99)
    \item[\Expr{\%Y}] year with century
    \item[\Expr{\%Z}] time zone name, if any
    \end{description}

  \index{ClassAd functions!interval()}
  \item[\Code{String interval(Integer seconds)}]
    Uses \Expr{seconds} to return a string of the form
    \Expr{days+hh:mm:ss}.
    This represents an interval of time.
    Leading values that are zero are omitted from the string.
    For example, \Expr{seconds} of 67 becomes "1:07".
    A second example, \Expr{seconds} of 
    1472523 = 17*24*60*60 + 1*60*60 + 2*60 + 3, results in the
    string "17+1:02:03".

  \index{ClassAd functions!debug()}
  \item[\Code{AnyType debug(AnyType expression)}]
     This function evaluates its argument, and it returns the result.
     Thus, it is a no-operation.
     However, a side-effect of the function is that information about
     the evaluation is logged to the evaluating program's log file,
     at the \Expr{D\_FULLDEBUG} debug level.
     This is useful for determining why a given ClassAd expression
     is evaluating the way it does.  
     For example, if a \Condor{startd} \MacroNI{START} expression
     is unexpectedly evaluating to \Expr{UNDEFINED},
     then wrapping the expression in this \Procedure{debug} function will
     log information about each component of the expression to the log file,
     making it easier to understand the expression.

  \index{ClassAd functions!envV1ToV2()}
  \item[\Code{String envV1ToV2(String old\_env)}]
    This function converts a set of environment variables from the old
    HTCondor syntax to the new syntax.
    The single argument should evaluate to a string that represents a
    set of environment variables using the old HTCondor syntax
    (usually stored in the job ClassAd attribute \Attr{Env}).
    The result is the same set of environment variables using the new
    HTCondor syntax (usually stored in the job ClassAd attribute
    \Attr{Environment}).
    If the argument evaluates to \Expr{UNDEFINED}, then the result is
    also \Expr{UNDEFINED}.

  \index{ClassAd functions!mergeEnvironment()}
  \item[\Code{String mergeEnvironment(String env1 [ , String env2, ... ])}]
    This function merges multiple sets of environment variables into
    a single set.
    If multiple arguments include the same variable, the one that appears
    last in the argument list is used.
    Each argument should evaluate to a string which represents a set of
    environment variables using the new HTCondor syntax or \Expr{UNDEFINED},
    which is treated like an empty string.
    The result is a string that represents the merged set of environment
    variables using the new HTCondor syntax (suitable for use as the value
    of the job ClassAd attribute \Attr{Environment}).

\end{description}

For the following functions, a delimiter is represented by a string.
Each character within the delimiter string
delimits individual strings within a list of strings 
that is given by a single string.
The default delimiter contains the comma and space characters.
A string within the list is ended (delimited) by one or more
characters within the delimiter string.

\begin{description}
  \index{ClassAd functions!stringListSize()}
  \item[\Code{Integer stringListSize(String list \Lbr\ , String delimiter \Rbr)}]
    Returns the number of elements in the string \Expr{list},
    as delimited by the optional \Expr{delimiter} string.
    Returns \Expr{ERROR} if either argument is not a string.

    This function returns \Expr{ERROR} if other than 1 or 2 arguments are given.

  \index{ClassAd functions!stringListSum()}
  \item[\Code{Integer stringListSum(String list \Lbr\ , String delimiter \Rbr)}]
  \item[OR \Code{Real stringListSum(String list \Lbr\ , String delimiter \Rbr)}]
    Sums and returns the sum of all items in the string \Expr{list},
    as delimited by the optional \Expr{delimiter} string.
    If all items in the list are integers, the return value is also
    an integer.
    If any item in the list is a real value (noninteger),
    the return value is a real.
    If any item does not represent an integer or real value,
    the return value is \Expr{ERROR}.


  \index{ClassAd functions!stringListAvg()}
  \item[\Code{Real stringListAvg(String list \Lbr\ , String delimiter \Rbr)}]
    Sums and returns the real-valued average of all items in the 
    string \Expr{list},
    as delimited by the optional \Expr{delimiter} string.
    If any item does not represent an integer or real value,
    the return value is \Expr{ERROR}.
    A list with 0 items (the empty list) returns the value 0.0.

  \index{ClassAd functions!stringListMin()}
  \item[\Code{Integer stringListMin(String list \Lbr\ , String delimiter \Rbr)}]
  \item[OR \Code{Real stringListMin(String list \Lbr\ , String delimiter \Rbr)}]
    Finds and returns the minimum value from all items in the
    string \Expr{list},
    as delimited by the optional \Expr{delimiter} string.
    If all items in the list are integers, the return value is also
    an integer.
    If any item in the list is a real value (noninteger),
    the return value is a real.
    If any item does not represent an integer or real value,
    the return value is \Expr{ERROR}.
    A list with 0 items (the empty list) returns the value \Expr{UNDEFINED}.

  \index{ClassAd functions!stringListMax()}
  \item[\Code{Integer stringListMax(String list \Lbr\ , String delimiter \Rbr)}]
  \item[OR \Code{Real stringListMax(String list \Lbr\ , String delimiter \Rbr)}]
    Finds and returns the maximum value from all items in the
    string \Expr{list},
    as delimited by the optional \Expr{delimiter} string.
    If all items in the list are integers, the return value is also
    an integer.
    If any item in the list is a real value (noninteger),
    the return value is a real.
    If any item does not represent an integer or real value,
    the return value is \Expr{ERROR}.
    A list with 0 items (the empty list) returns the value \Expr{UNDEFINED}.

  \index{ClassAd functions!stringListMember()}
  \item[\Code{Boolean stringListMember(String x, String list \Lbr\ , String delimiter \Rbr)}]
    Returns \Expr{TRUE} if item \Expr{x} is in the string \Expr{list},
    as delimited by the optional \Expr{delimiter} string.
    Returns \Expr{FALSE} if item \Expr{x} is not in the string \Expr{list}.
    Comparison is done with \Expr{strcmp()}.
    The return value is \Expr{ERROR}, if any of the arguments
    are not strings.

  \index{ClassAd functions!stringListIMember()}
  \item[\Code{Boolean stringListIMember(String x, String list \Lbr\ , String delimiter \Rbr)}]
    Same as \Code{stringListMember()}, but comparison is done
    with \Expr{stricmp()}, so letter case is not relevant.

  \index{ClassAd functions!stringListsIntersect()}
  \item[\Code{Integer stringListsIntersect(String list1, String list2 \Lbr\ , String delimiter \Rbr)}]
    Returns \Expr{TRUE} if the lists contain any matching elements,
    and returns \Expr{FALSE} if the lists do not contain any matching elements.
    Returns \Expr{ERROR} if either argument is not a string or if an
    incorrect number of arguments are given.

\end{description}

The following three functions utilize regular expressions as defined
and supported by the PCRE library.
See \URL{http://www.pcre.org} for complete documentation of
regular expressions.

The \Expr{options} argument to these functions is a string of 
special characters that modify the use of the regular expressions.
Inclusion of characters other than these as options are ignored.
\begin{description}
  \item[\Expr{I} or \Expr{i}]
    Ignore letter case.
  \item[\Expr{M} or \Expr{m}]
    Modifies the interpretation of the caret (\verb@^@) and dollar sign
    (\verb@$@) characters.
    The caret character matches the start of a string, as well as
    after each newline character.
    The dollar sign character matches before a newline character.
  \item[\Expr{S} or \Expr{s}]
    The period matches any character, including the newline character. 
%%% The X flag is not documented as it's basically useless without
%%% the ability to have newlines in the regex.
%  \item[\Expr{X} or \Expr{x}]
%    Ignore both white space and comments within the pattern.
%    A comment is defined by starting with the pound sign (\verb@#@)
%    character, and continuing until the newline character.  Note that
%    ClassAd values generally cannot contain newlines, limiting the
%    usefulness of this option.
\end{description}

\begin{description}
  \index{ClassAd functions!regexp()}
  \item[\Code{Boolean regexp(String pattern, String target \Lbr\ , String options \Rbr)}]
    Uses the description of a regular expression 
    given by string \Expr{pattern}
    to scan through the string \Expr{target}.
    Returns \Expr{TRUE} when \Expr{target} is 
    a regular expression as described by \Expr{pattern}.
    Returns \Expr{FALSE} otherwise.
    If any argument is not a string, or if \Expr{pattern} does not describe
    a valid regular expression, returns \Expr{ERROR}.

% LaTeX can't get the right stuff on a single line, so break the item[]
% early. Stupid solution, but if it works. . .
  \index{ClassAd functions!regexps()}
  \item[\Code{String regexps}] \Code{(String pattern, String target, 
   String substitute \Lbr\ , String options \Rbr) } 
    Uses the description of a regular expression 
    given by string \Expr{pattern}
    to scan through the string \Expr{target}.
    When \Expr{target} is a regular expression
    as described by \Expr{pattern},
    the string \Expr{substitute} is returned,
    with backslash expansion performed.
    If any argument is not a string, returns \Expr{ERROR}.

% LaTeX can't get the right stuff on a single line, so break the item[]
% early. Stupid solution, but if it works. . .
  \index{ClassAd functions!stringList\_regexpMember()}
  \item[\Code{Boolean stringList\_regexpMember}]
    \Code{(String pattern, String list \Lbr\ , String delimiter 
    \Rbr\ \Lbr\ , String options \Rbr) } 
    Uses the description of a regular expression 
    given by string \Expr{pattern}
    to scan through the list of strings in \Expr{list}.
    Returns \Expr{TRUE} when one of the strings in \Expr{list} is 
    a regular expression as described by \Expr{pattern}.
    The optional \Expr{delimiter} describes how the list is delimited,
    and string \Expr{options} modifies how the match is performed.
    Returns \Expr{FALSE} if \Expr{pattern} does not match any entries in
    \Expr{list}.
    The return value is \Expr{ERROR}, if any of the arguments
    are not strings, or if \Expr{pattern} is not a valid regular expression.

  \index{ClassAd functions!userHome()}
  \item[\Code{String userHome(String userName \Lbr\ , String default \Rbr)}]
    Returns the home directory of the given user as configured on the
    current system (determined using the getpwdnam() call).  (Returns
    \Expr{default} if the \Expr{default} argument is passed and the
    home directory of the user is not defined.)

  \index{ClassAd functions!userMap()}
  \item[\Code{List userMap(String mapSetName, String userName)}]
    Map an input string using the given mapping set.
    Returns a list of groups to which the user belongs.

  \index{ClassAd functions!userMap()}
  \item[\Code{String userMap(String mapSetName, String userName,
  String preferredGroup)}]
    Map an input string using the given mapping set.
    Returns a string, which is the preferred group if the user is in
    that group; otherwise it is the first group to which the user
    belongs, or undefined if the user belongs to no groups.

  \index{ClassAd functions!userMap()}
  \item[\Code{String userMap(String mapSetName, String userName,
  String preferredGroup, String defaultGroup)}]
    Map an input string using the given mapping set.
    Returns a string, which is the preferred group if the user is
    in that group; the first group to which the user belongs, if any;
    and the default group if the user belongs to no groups.

    The maps for the \Code{userMap()} function are defined by the
    following configuration macros:
    \MacroNI{<SUBSYS>\_CLASSAD\_USER\_MAP\_NAMES}
    (see ~\ref{param:ClassAdUserMapNames}),
    \MacroNI{CLASSAD\_USER\_MAPFILE\_<name>
    (see ~\ref{param:ClassAdUserMapFile})}
    and \MacroNI{CLASSAD\_USER\_MAPDATA\_<name>
    (see ~\ref{param:ClassAdUserMapData})}.

\end{description}


%%%%%%%%%%%%%%%%%%%%%%%%%%%%%%%%%%%%%%%%%%%%%%%%%%%
\subsection{Old ClassAd Evaluation Semantics}
%%%%%%%%%%%%%%%%%%%%%%%%%%%%%%%%%%%%%%%%%%%%%%%%%%%
\label{ClassAd:evaluation}
The ClassAd mechanism's primary purpose is for matching entities that supply
constraints on candidate matches.  The mechanism is therefore defined to
carry out expression evaluations in the context of two ClassAds that are
testing each other for a potential match.  For example, the \Condor{negotiator}
evaluates the \Attr{Requirements} expressions of machine and job ClassAds to
test if they can be matched.  The semantics of evaluating such constraints
is defined below.

%%%%%%%%%%%%%%%%%%%%%%%%%%%%%%%%%%%%%%%%%%%%%%%%%%%
\subsubsection{Literals}
%%%%%%%%%%%%%%%%%%%%%%%%%%%%%%%%%%%%%%%%%%%%%%%%%%%
Literals are self-evaluating, Thus, integer, string, real, undefined and
error values evaluate to themselves.

%%%%%%%%%%%%%%%%%%%%%%%%%%%%%%%%%%%%%%%%%%%%%%%%%%%
\subsubsection{Attribute References}
%%%%%%%%%%%%%%%%%%%%%%%%%%%%%%%%%%%%%%%%%%%%%%%%%%%
\index{ClassAd!scope of evaluation, MY.}
\index{ClassAd!scope of evaluation, TARGET.}
\index{TARGET., ClassAd scope resolution prefix}
\index{MY., ClassAd scope resolution prefix}
Since the expression evaluation is being carried out in the context of two
ClassAds, there is a potential for name space ambiguities.  
The following rules define the semantics of attribute references made 
by ClassAd $A$ that is being 
evaluated in a context with another ClassAd $B$:
\begin{enumerate}
    \item If the reference is prefixed by a scope resolution prefix, 
    \begin{itemize}
        \item If the prefix is \texttt{MY.}, the attribute is looked up in 
        ClassAd $A$.  If the named attribute does not exist in $A$, the
        value of the reference is \texttt{UNDEFINED}.  Otherwise, the
        value of the reference is the value of the expression bound to
        the attribute name.

        \item Similarly, if the prefix is \texttt{TARGET.}, the attribute is 
        looked up in ClassAd $B$.  If the named attribute does not exist in 
        $B$, the value of the reference is \texttt{UNDEFINED}.  Otherwise, 
        the value of the reference is the value of the expression bound to
        the attribute name.

    \end{itemize}

    \item If the reference is not prefixed by a scope resolution prefix,
    \begin{itemize}
        \item If the attribute is defined in $A$, the value of the reference
        is the value of the expression bound to the attribute name in $A$.
        \item Otherwise, if the attribute is defined in $B$, the value of the
        reference is the value of the expression bound to the attribute
        name in $B$.
        \item Otherwise, if the attribute is defined in the ClassAd environment, the
        value from the environment is returned.
        This is a special environment, to be
        distinguished from the Unix environment.
        Currently, the only attribute
        of the environment is \Attr{CurrentTime}, which evaluates to the
        integer value returned by the system call \texttt{time(2)}.
        \item Otherwise, the value of the reference is \texttt{UNDEFINED}.
    \end{itemize}

    \item Finally, if the reference refers to an expression that is itself in 
    the process of being evaluated, there is a circular dependency in the 
    evaluation.  The value of the reference is \texttt{ERROR}.
\end{enumerate}

%%%%%%%%%%%%%%%%%%%%%%%%%%%%%%%%%%%%%%%%%%%%%%%%%%%
\subsubsection{Operators}
%%%%%%%%%%%%%%%%%%%%%%%%%%%%%%%%%%%%%%%%%%%%%%%%%%%
\label{ClassAd:evaluation-meta}
\index{ClassAd!expression operators}
All operators in the ClassAd language are \Term{total}, and thus have well
defined behavior regardless of the supplied operands.  Furthermore, most
operators are \Term{strict} with respect to \Expr{ERROR} and 
\Expr{UNDEFINED}, and thus evaluate to \Expr{ERROR} or \Expr{UNDEFINED}
if either of their operands have these exceptional values.

\begin{itemize}
	\item\textbf{Arithmetic operators:}  
	\begin{enumerate}
		\item The operators \verb@*@, \verb@/@, \verb@+@ and \verb@-@ operate 
		arithmetically only on integers and reals.

		\item Arithmetic is carried out in the same type as both operands,
		and type promotions from integers to reals are performed if one operand 
		is an integer and the other real.

		\item The operators are strict with respect to both \Expr{UNDEFINED} 
		and \Expr{ERROR}.  

		\item If either operand is not a numerical type, the value of the
		operation is \Expr{ERROR}.
	\end{enumerate}

	\item\textbf{Comparison operators:}
	\begin{enumerate}
		\item The comparison operators \verb@==@, \verb@!=@, \verb@<=@, 
		\verb@<@, \verb@>=@ and \verb@>@ operate on integers, reals and strings.

		\item String comparisons are case insensitive for most operators.  The only
		exceptions are the operators \verb@=?=@ and \verb@=!=@, which do case sensitive
		comparisons assuming both sides are strings. 

		\item Comparisons are carried out in the same type as both operands,
		and type promotions from integers to reals are performed if one operand
		is a real, and the other an integer.  Strings may not be converted to
		any other type, so comparing a string and an integer or a
		string and a real results in \Expr{ERROR}.

		\item The operators \verb@==@, \verb@!=@, \verb@<=@, \verb@<@ and 
		\verb@>=@ \verb@>@ are strict with respect to both \Expr{UNDEFINED} 
		and \Expr{ERROR}.

		\item In addition, the operators \verb@=?=@, \verb@is@, \verb@=!=@ and \verb@isnt@ behave
		similar to \verb@==@ and \verb@!=@, but are not strict.  Semantically,
		the \verb@=?=@ and \verb@is@ test if their operands are ``identical,'' i.e., have
		the same type and the same value.  For example, \verb@10 == UNDEFINED@ 
		and \verb@UNDEFINED == UNDEFINED@ both evaluate to \Expr{UNDEFINED},
		but \verb@10 =?= UNDEFINED@ and \verb@UNDEFINED is UNDEFINED@ 
		evaluate to \Expr{FALSE} and \Expr{TRUE} respectively.  The
		\verb@=!=@ and \verb@isnt@ operators test for the ``is not identical to'' condition.

		\verb@=?=@ and \verb@is@ have the same behavior as each other. And \verb@isnt@ and \verb@=!=@
		behave the same as each other.  The ClassAd unparser will always use \verb@=?=@ in preference to \verb@is@
		and \verb@=!=@ in preference to \verb@isnt@ when printing out ClassAds.
	\end{enumerate}

	\item\textbf{Logical operators:}
	\begin{enumerate}
		\item The logical operators \verb@&&@ and \verb@||@ operate on 
		integers and reals.  The zero value of these types are considered 
		\Expr{FALSE} and non-zero values \Expr{TRUE}.

		\item The operators are \emph{not} strict, and exploit the 
		"don't care" properties of the operators to squash \Expr{UNDEFINED}
		and \Expr{ERROR} values when possible.  For example,
		\verb@UNDEFINED && FALSE@ evaluates to \Expr{FALSE}, but	
		\verb@UNDEFINED || FALSE@ evaluates to \Expr{UNDEFINED}.

		\item Any string operand is equivalent to an \Expr{ERROR} operand
		for a logical operator.  In other words,
		\verb@TRUE && "foobar"@ evaluates to \Expr{ERROR}.
	\end{enumerate}
	\item\textbf{The Ternary operator:}
	\begin{enumerate}
		\item The Ternary operator (\Expr{expr1} ? \Expr{expr2} : \Expr{expr3})
		operate with expressions.  If all three expressions are given, the operation
		is strict.

		\item However, if the middle expression is missing, eg. \Expr{expr1} ?: \Expr{expr3},
		then, when expr1 is defined, that defined value is returned.  Otherwise, when expr1
		evaluated to \Expr{UNDEFINED}, the value of expr3 is evaluated and returned.  This
		can be a convenient shortcut for writing what would otherwise be a much longer
		classad expression.
	\end{enumerate}

\end{itemize}

%%%%%%%%%%%%%%%%%%%%%%%%%%%%%%%%%%%%%%%%%%%%%%%%%%%
\subsubsection{Expression Examples}
\label{ClassAd:examples}
%%%%%%%%%%%%%%%%%%%%%%%%%%%%%%%%%%%%%%%%%%%%%%%%%%%
\index{ClassAd!expression examples}

The \Expr{=?=} operator is similar to the \Expr{==} operator.
It checks if the left hand side operand is identical in both type and value
to the the right hand side operand, returning \Expr{TRUE} when they
are identical. \Bold{For strings, the comparison is case-insensitive with the
\Expr{==} operator and case-sensitive with the \Expr{=?=} operator.}
A key point in understanding is that
the \Expr{=?=} operator only produces evaluation results of \Expr{TRUE}
and \Expr{FALSE},
where the \Expr{==} operator may produce evaluation results \Expr{TRUE},
\Expr{FALSE}, \Expr{UNDEFINED}, or \Expr{ERROR}.
Table~\ref{expr-examples-1} presents examples that define the
outcome of the \Expr{==} operator.
Table~\ref{expr-examples-2} presents examples that define the
outcome of the \Expr{=?=} operator.

% Karen's table of == expression examples
\begin{center}
\begin{table}[hbt]
\begin{tabular}{|p{8cm}p{4cm}|} \hline
\Bold{expression} & \Bold{evaluated result} \\ \hline \hline
\Expr{(10 == 10)}                       & \Expr{TRUE}  \\
\Expr{(10 == 5)}                        & \Expr{FALSE} \\
\Expr{(10 == "ABC")}                    & \Expr{ERROR} \\
\Expr{{"ABC" == "abc"}}                 & \Expr{TRUE} \\
\Expr{(10 == UNDEFINED)}                & \Expr{UNDEFINED} \\
\Expr{(UNDEFINED == UNDEFINED)}         & \Expr{UNDEFINED}  \\ \hline
\end{tabular}
\caption{\label{expr-examples-1}Evaluation examples for the \Expr{==} operator}
\end{table}
\end{center}

% Karen's table of =?= expression examples
\begin{center}
\begin{table}[hbt]
\begin{tabular}{|p{8cm}p{4cm}|} \hline
\Bold{expression} & \Bold{evaluated result} \\ \hline \hline
\Expr{(10 =?= 10)}                       & \Expr{TRUE}  \\
\Expr{(10 =?= 5)}                        & \Expr{FALSE} \\
\Expr{(10 =?= "ABC")}                    & \Expr{FALSE} \\
\Expr{{"ABC" =?= "abc"}}                 & \Expr{FALSE} \\
\Expr{(10 =?= UNDEFINED)}                & \Expr{FALSE} \\
\Expr{(UNDEFINED =?= UNDEFINED)}         & \Expr{TRUE}  \\ \hline
\end{tabular}
\caption{\label{expr-examples-2}Evaluation examples for the \Expr{=?=} operator}
\end{table}
\end{center}

The \Expr{=!=} operator is similar to the \Expr{!=} operator.
It checks if the left hand side operand is \emph{not} identical 
in both type and value to the the right hand side operand,
returning \Expr{FALSE} when they are identical. \Bold{For strings, the comparison is
case-insensitive with the \Expr{!=} operator and case-sensitive with the
\Expr{=!=} operator.}
A key point in understanding is that
the \Expr{=!=} operator only produces evaluation results of \Expr{TRUE}
and \Expr{FALSE},
where the \Expr{!=} operator may produce evaluation results \Expr{TRUE},
\Expr{FALSE}, \Expr{UNDEFINED}, or \Expr{ERROR}.
Table~\ref{expr-examples-3} presents examples that define the
outcome of the \Expr{!=} operator.
Table~\ref{expr-examples-4} presents examples that define the
outcome of the \Expr{=!=} operator.

% Karen's table of != expression examples
\begin{center}
\begin{table}[hbt]
\begin{tabular}{|p{8cm}p{4cm}|} \hline
\Bold{expression} & \Bold{evaluated result} \\ \hline \hline
\Expr{(10 != 10)}                       & \Expr{FALSE}  \\
\Expr{(10 != 5)}                        & \Expr{TRUE} \\
\Expr{(10 != "ABC")}                    & \Expr{ERROR} \\
\Expr{{"ABC" != "abc"}}                 & \Expr{FALSE} \\
\Expr{(10 != UNDEFINED)}                & \Expr{UNDEFINED} \\
\Expr{(UNDEFINED != UNDEFINED)}         & \Expr{UNDEFINED}  \\ \hline
\end{tabular}
\caption{\label{expr-examples-3}Evaluation examples for the \Expr{!=} operator}
\end{table}
\end{center}

% Karen's table of =!= expression examples
\begin{center}
\begin{table}[hbt]
\begin{tabular}{|p{8cm}p{4cm}|} \hline
\Bold{expression} & \Bold{evaluated result} \\ \hline \hline
\Expr{(10 =!= 10)}                       & \Expr{FALSE}  \\
\Expr{(10 =!= 5)}                        & \Expr{TRUE} \\
\Expr{(10 =!= "ABC")}                    & \Expr{TRUE} \\
\Expr{{"ABC" =!= "abc"}}                 & \Expr{TRUE} \\
\Expr{(10 =!= UNDEFINED)}                & \Expr{TRUE} \\
\Expr{(UNDEFINED =!= UNDEFINED)}         & \Expr{FALSE}  \\ \hline
\end{tabular}
\caption{\label{expr-examples-4}Evaluation examples for the \Expr{=!=} operator}
\end{table}
\end{center}

%%%%%%%%%%%%%%%%%%%%%%%%%%%%%%%%%%%%%%%%%%%%%%%%%%%
\subsection{Old ClassAds in the HTCondor System}
%%%%%%%%%%%%%%%%%%%%%%%%%%%%%%%%%%%%%%%%%%%%%%%%%%%
The simplicity and flexibility of ClassAds is heavily exploited in the HTCondor
system.  ClassAds are not only used to represent machines and jobs in the 
HTCondor pool, but also other entities that exist in the pool such as 
checkpoint servers, submitters of jobs and master daemons.  Since arbitrary
expressions may be supplied and evaluated over these ClassAds, 
users have a uniform
and powerful mechanism to specify constraints over these ClassAds.
These constraints
can take the form of \Attr{Requirements} expressions in resource 
and job ClassAds,
or queries over other ClassAds.

%%%%%%%%%%%%%%%%%%%%%%%%%%%%%%%%%%%%%%%%%%%%%%%%%%%
\subsubsection{Constraints and Preferences}
%%%%%%%%%%%%%%%%%%%%%%%%%%%%%%%%%%%%%%%%%%%%%%%%%%%
\index{ClassAd attribute!requirements}
\index{ClassAd attribute!rank}
The \AdAttr{requirements} and \AdAttr{rank} expressions
within the submit description file
are the mechanism
by which users specify the constraints and preferences of jobs.
For machines, the configuration determines both 
constraints and preferences of the machines.

\index{rank attribute!examples}
\index{requirements attribute}
For both machine and job, 
the \Attr{rank} expression specifies
the desirability of the match (where higher numbers mean better matches).
For example, a job ClassAd may contain the following expressions:
\footnotesize
\begin{verbatim}
Requirements = (Arch == "INTEL") && (OpSys == "LINUX")
Rank         = TARGET.Memory + TARGET.Mips
\end{verbatim}
\normalsize
In this case, the job requires a 32-bit Intel processor running a Linux
operating system.
Among all such computers,
the customer prefers those with large physical memories and high MIPS ratings.  
Since the \Attr{Rank} is a user-specified metric,
\emph{any} expression may be used to specify the
perceived desirability of the match.
The \Condor{negotiator} daemon runs algorithms
to deliver the best resource (as defined by the \Attr{rank} expression),
while satisfying other required criteria.

Similarly, the machine may place constraints and preferences on 
the jobs that it will run by setting the machine's configuration.
For example,
\footnotesize
\begin{verbatim}
    Friend        = Owner == "tannenba" || Owner == "wright"
    ResearchGroup = Owner == "jbasney" || Owner == "raman"
    Trusted       = Owner != "rival" && Owner != "riffraff"
    START         = Trusted && ( ResearchGroup || LoadAvg < 0.3 &&
                         KeyboardIdle > 15*60 )
    RANK          = Friend + ResearchGroup*10
\end{verbatim}
\normalsize

The above policy states that the computer will never run jobs owned by
users rival and riffraff, while the computer will always run a 
job submitted by members of the research group.
Furthermore,
jobs submitted by friends are preferred to other foreign jobs,
and jobs submitted
by the research group are preferred to jobs submitted by friends. 

\textbf{Note:}  Because of the dynamic nature of ClassAd expressions, there
is no \emph{a priori} notion of an integer-valued expression, a real-valued
expression, etc.  However, it is intuitive to think of the \Attr{Requirements}
and \Attr{Rank} expressions as integer-valued and real-valued expressions,
respectively.  If the actual type of the expression is not of the expected 
type, the value is assumed to be zero.

%%%%%%%%%%%%%%%%%%%%%%%%%%%%%%%%%%%%%%%%%%%%%%%%%%%
\subsubsection{\label{sec:classad-query-examples}Querying with ClassAd Expressions}
%%%%%%%%%%%%%%%%%%%%%%%%%%%%%%%%%%%%%%%%%%%%%%%%%%%
The flexibility of this system may also be used when querying ClassAds
through the \Condor{status} and \Condor{q} tools which allow users to
supply ClassAd constraint expressions from the command line.

Needed syntax is different on Unix and Windows platforms, 
due to the interpretation of characters in forming command-line arguments.
The expression must be a single command-line argument,
and the resulting examples differ for the platforms.
For Unix shells,
single quote marks are used to delimit a single argument.
For a Windows command window,
double quote marks are used to delimit a single argument.
Within the argument,
Unix escapes the double quote mark by prepending a backslash to the double
quote mark.
Windows escapes the double quote mark by prepending another double
quote mark. There may not be spaces in between.

Here are several examples.
To find all computers which have had their keyboards idle for 
more than 60 minutes and have more than 4000 MB of memory,
the desired ClassAd expression is
\footnotesize
\begin{verbatim}
KeyboardIdle > 60*60 && Memory > 4000
\end{verbatim}
\normalsize
On a Unix platform, the command appears as
\footnotesize
\begin{verbatim}
% condor_status -const 'KeyboardIdle > 60*60 && Memory > 4000'

Name               OpSys   Arch   State     Activity LoadAv Mem  ActvtyTime
100
slot1@altair.cs.wi LINUX   X86_64 Owner     Idle     0.000 8018 13+00:31:46
slot2@altair.cs.wi LINUX   X86_64 Owner     Idle     0.000 8018 13+00:31:47
...
...
slot1@athena.stat. LINUX   X86_64 Unclaimed Idle     0.000 7946  0+00:25:04
slot2@athena.stat. LINUX   X86_64 Unclaimed Idle     0.000 7946  0+00:25:05
...
...
\end{verbatim}
\normalsize

The Windows equivalent command is
\footnotesize
\begin{verbatim}
>condor_status -const "KeyboardIdle > 60*60 && Memory > 4000"
\end{verbatim}
\normalsize

Here is an example for a Unix platform that utilizes a regular expression
ClassAd function to list specific information.
A file contains ClassAd information.
\Condor{advertise} is used to inject this information,
and \Condor{status} constrains the search with an expression
that contains a ClassAd function.

\footnotesize
\begin{verbatim}
% cat ad
MyType = "Generic"
FauxType = "DBMS"
Name = "random-test"
Machine = "f05.cs.wisc.edu"
MyAddress = "<128.105.149.105:34000>"
DaemonStartTime = 1153192799
UpdateSequenceNumber = 1

% condor_advertise UPDATE_AD_GENERIC ad

% condor_status -any -constraint 'FauxType=="DBMS" && 
  regexp("random.*", Name, "i")'

MyType               TargetType           Name                          

Generic              None                 random-test                   

\end{verbatim}
\normalsize

The ClassAd expression describing a machine that
advertises a Windows operating system:
\footnotesize
\begin{verbatim}
OpSys == "WINDOWS"
\end{verbatim}
\normalsize
Here are three equivalent ways on a Unix platform to list all machines
advertising a Windows operating system.
Spaces appear in these examples to show where they are permitted.
\footnotesize
\begin{verbatim}
% condor_status -constraint ' OpSys == "WINDOWS"  '
\end{verbatim}
\normalsize
\footnotesize
\begin{verbatim}
% condor_status -constraint OpSys==\"WINDOWS\"
\end{verbatim}
\normalsize
\footnotesize
\begin{verbatim}
% condor_status -constraint "OpSys==\"WINDOWS\""
\end{verbatim}
\normalsize

The equivalent command on a Windows platform to list all machines
advertising a Windows operating system must delimit the single
argument with double quote marks, and then escape the needed
double quote marks that identify the string within the expression. 
Spaces appear in this example where they are permitted.
\footnotesize
\begin{verbatim}
>condor_status -constraint " OpSys == ""WINDOWS"" "
\end{verbatim}
\normalsize

%%%%%%%%%%%%%%%%%%%%%%%%%%%%%%%%%%%%%%%%%%%%%%%%%%%
\subsection{\label{sec:classad-userfunctions}
Extending ClassAds with User-written Functions}
%%%%%%%%%%%%%%%%%%%%%%%%%%%%%%%%%%%%%%%%%%%%%%%%%%%

The ClassAd language provides a rich set of functions.  It is possible
to add new functions to the ClassAd language without recompiling
the HTCondor system or the ClassAd library.  
This requires implementing the
new function in the C++ programming language, 
compiling the code into a shared library, 
and telling HTCondor where in the file system the shared library lives.

While the details of the ClassAd implementation are beyond the scope of 
this document,
the ClassAd source distribution ships with an example source file that 
extends ClassAds by adding two new functions,
named \Procedure{todays\_date} and \Procedure{double}.  
This can be used as a model for users to implement
their own functions. 
To deploy this example extension, follow the following steps on Linux:

\begin{itemize}
\item{Download the ClassAd source distribution from 
\URL{http://www.cs.wisc.edu/condor/classad}.}
\item{Unpack the tarball.}
\item{Inspect the source file \File{shared.cpp}.
This one file contains the whole extension.}
\item{Build \File{shared.cpp} into a shared library.  
On Linux, the command line to do so is
\begin{verbatim}
$ g++ -DWANT_CLASSAD_NAMESPACE -I. -shared -o shared.so \
  -Wl,-soname,shared.so -o shared.so -fPIC shared.cpp
\end{verbatim}
}
\item{Copy the file \File{shared.so} to a location that all of 
the HTCondor tools and daemons can read.
\begin{verbatim}
$ cp shared.so `condor_config_val LIBEXEC`
\end{verbatim}
}
\item{Tell HTCondor to load the shared library into all tools and daemons, 
by setting the \Macro{CLASSAD\_USER\_LIBS} configuration variable
 to the full name of the shared library.  In this case,
\begin{verbatim}
CLASSAD_USER_LIBS = $(LIBEXEC)/shared.so
\end{verbatim}
}
\item{Restart HTCondor.}
\item{Test the new functions by running
\begin{verbatim}
$ condor_status -format "%s\n" todays_date()
\end{verbatim}
}

\end{itemize}

\index{ClassAd|)}
