%%%%%%%%%%%%%%%%%%%%%%%%%%%%%%%%%%%%%%%%%%%%%%%%%%%
\section{\label{classad-reference}
Condor's ClassAd Mechanism}
%%%%%%%%%%%%%%%%%%%%%%%%%%%%%%%%%%%%%%%%%%%%%%%%%%%
\label{sec:classadref}

\index{ClassAd|(}
ClassAds are a flexible mechanism for representing the characteristics and
constraints of machines and jobs in the Condor system.  ClassAds are used
extensively in the Condor system to represent jobs, resources, submitters
and other Condor daemons.  An understanding of this mechanism is required
to harness the full flexibility of the Condor system.

A ClassAd is is a set of uniquely named expressions.  Each named expression
is called an \Term{attribute}.  Figure~\ref{ClassAd:example} shows an example 
of a ClassAd with ten attributes.

\begin{figure}[hbt]
\footnotesize
\begin{verbatim}
MyType       = "Machine"
TargetType   = "Job"
Machine      = "froth.cs.wisc.edu"
Arch         = "INTEL"
OpSys        = "SOLARIS251"
Disk         = 35882
Memory       = 128
KeyboardIdle = 173
LoadAvg      = 0.1000
Requirements = TARGET.Owner=="smith" || LoadAvg<=0.3 && KeyboardIdle>15*60
\end{verbatim}
\normalsize
\caption{\label{ClassAd:example}An example ClassAd}
\end{figure}

ClassAd expressions look very much like expressions in C, and are composed
of literals and attribute references composed with operators 
and functions.
The difference
between ClassAd expressions and C expressions arise from the fact that ClassAd
expressions operate in a much more dynamic environment.  For example, an
expression from a machine's ClassAd may refer to an attribute in a job's 
ClassAd, such as \verb+TARGET.Owner+ in the above example.  The value and type 
of the attribute is not known until the expression is evaluated in an 
environment which pairs a specific job ClassAd with the machine ClassAd.

ClassAd expressions handle these uncertainties by defining all operators
to be \Term{total} operators, which means that they have well defined
behavior regardless of supplied operands.  This functionality is provided
through two distinguished values, \texttt{UNDEFINED} and \texttt{ERROR},
and defining all operators so that they can operate on all possible values
in the ClassAd system.  For example, the multiplication operator which usually
only operates on numbers, has a well defined behavior if supplied with values
which are not meaningful to multiply.  Thus, the expression 
\verb+10 * "A string"+ evaluates to the value \texttt{ERROR}.  Most operators
are \Term{strict} with respect to \texttt{ERROR}, which means that they evaluate
to \texttt{ERROR} if any of their operands are \texttt{ERROR}.  Similarly,
most operators are strict with respect to \texttt{UNDEFINED}.

%%%%%%%%%%%%%%%%%%%%%%%%%%%%%%%%%%%%%%%%%%%%%%%%%%%
\subsection{Syntax}
%%%%%%%%%%%%%%%%%%%%%%%%%%%%%%%%%%%%%%%%%%%%%%%%%%%
\index{ClassAd!expression syntax}
ClassAd expressions are formed by composing literals, attribute references and 
other sub-expressions with operators and functions. 
\subsubsection{Literals}
\label{ClassAd:literals}
Literals in the ClassAd language may be of integer, real, string, undefined or 
error types.  The syntax of these literals is as follows:
\begin{description}
	\item[Integer]  A sequence of continuous digits (i.e., \verb@[0-9]@).
		Additionally, the keywords \verb+TRUE+ and \verb+FALSE+ (case
		insensitive) are syntactic representations of the integers 1 and 0 
		respectively.

	\item[Real] Two sequences of continuous digits separated by a period
		(i.e., \verb@[0-9]+.[0-9]+@).

	\item[String] A double quote character, followed by an list of characters
		terminated by a double quote character.  A backslash character inside
		the string causes the following character to be considered as part of
		the string, irrespective of what that character is.

	\item[Undefined] The keyword \texttt{UNDEFINED} (case insensitive)
		represents the \texttt{UNDEFINED} value.

	\item[Error] The keyword \texttt{ERROR} (case insensitive)
		represents the \texttt{ERROR} value.
\end{description}

%%%%%%%%%%%%%%%%%%%%%%%%%%%%%%%%%%%%%%%%%%%%%%%%%%%
\subsubsection{Attributes}
%%%%%%%%%%%%%%%%%%%%%%%%%%%%%%%%%%%%%%%%%%%%%%%%%%%
\index{ClassAd!attributes}
Every expression in a ClassAd is named by an \Term{attribute name}.  Together,
the (name,expression) pair is called an \Term{attribute}.  An attributes may be
referred to in other expressions through its attribute name.

Attribute names are sequences of alphabetic characters, digits and underscores,
and may not begin with a digit.  All characters in the name are significant,
but case is \emph{not} significant.  Thus, \verb+Memory+, \verb+memory+ and 
\verb+MeMoRy+ all refer to the same attribute.

An \Term{attribute reference} consists of the name of the attribute being 
referenced, and an optional \Term{scope resolution prefix}.  The 
prefixes that may be used are \Attr{MY.} and \Attr{TARGET.}.
The case used for these prefixes is \emph{not} significant.
The semantics of supplying a prefix are discussed in 
Section~\ref{ClassAd:evaluation}.

%%%%%%%%%%%%%%%%%%%%%%%%%%%%%%%%%%%%%%%%%%%%%%%%%%%
\subsubsection{Operators}
%%%%%%%%%%%%%%%%%%%%%%%%%%%%%%%%%%%%%%%%%%%%%%%%%%%
\index{ClassAd!expression operators}
The operators that may be used in ClassAd expressions are similar to those
available in C.  The available operators and their relative precedence is shown 
in figure~\ref{ClassAd:operator-fig}.
\begin{figure}[h]
\begin{verbatim}
  - (unary negation)   (high precedence)
  *   / 
  +   - (addition, subtraction)
  <   <=   >=   >
  ==  !=  =?=  =!=
  &&
  ||                   (low precedence) 
\end{verbatim}
\caption{\label{ClassAd:operator-fig}Relative precedence of ClassAd expression
operators}
\end{figure}
The operator with the highest precedence is the unary minus operator.  The
only operators which are unfamiliar are the \verb+=?=+ and \verb+=!=+
operators, which are discussed in Section~\ref{ClassAd:evaluation-meta}.


%%%%%%%%%%%%%%%%%%%%%%%%%%%%%%%%%%%%%%%%%%%%%%%%%%%
\subsubsection{Predefined Functions}
%%%%%%%%%%%%%%%%%%%%%%%%%%%%%%%%%%%%%%%%%%%%%%%%%%%
\index{ClassAd!expression functions}
Any ClassAd expression may utilize predefined functions.
Function names are case insensitive.
Parameters to functions 
and a return value from a function 
may be typed (as given) or not.
Nested or recursive function calls are allowed.

Here are descriptions of each of these predefined functions.
The possible types are the same as itemized in 
in Section~\ref{ClassAd:literals}.
Where the type may be any of these literal types, it is
called out as \verb@AnyType@.
Where the type is \verb@Integer@, but only returns
the value 1 or 0 (implying \Expr{True} or \Expr{False}),
it is called out as \verb@Boolean@.
The format of each function is given as

\footnotesize
\begin{verbatim}
ReturnType FunctionName(ParameterType parameter1, ParameterType parameter2, ...)
\end{verbatim}
\normalsize
Optional parameters are given within square brackets.

\begin{description}
  \item[\Expr{AnyType ifThenElse(AnyType IfExpr,AnyType ThenExpr, AnyType ElseExpr)}]
    A conditional expression is described by \Expr{IfExpr}.
    The following defines return values, when \Expr{IfExpr}
    evaluates to
    \begin{itemize}
    \item{\Expr{True}. Evaluate and return the value as given
      by \Expr{ThenExpr}.}
    \item{\Expr{False}. Evaluate and return the value as given
      by \Expr{ElseExpr}.}
    \item{\Expr{UNDEFINED}. Return the value \Expr{UNDEFINED}.}
    \item{\Expr{ERROR}. Return the value \Expr{ERROR}.}
    \item{\Expr{0.0}. Evaluate, and return the value as given
      by \Expr{ElseExpr}.}
    \item{non-\Expr{0.0} Real values. Evaluate, and return the value as given
      by \Expr{ThenExpr}.}
    \end{itemize}
    Where \Expr{IfExpr} evaluates to give a value of type \Expr{String},
    the function returns the value \Expr{ERROR}.
    The implementation uses lazy evaluation, so expressions
    are only evaluated as defined.

    This function returns \Expr{ERROR} if other than exactly 3
    arguments are given.


  \item[\Expr{Boolean isUndefined(AnyType Expr)}]
    Returns \Expr{True}, if \Expr{Expr} evaluates to \Expr{UNDEFINED}.
    Returns \Expr{False} in all other cases.

    This function returns \Expr{ERROR} if other than exactly 1
    argument is given.

  \item[\Expr{Boolean isError(AnyType Expr)}]
    Returns \Expr{True}, if \Expr{Expr} evaluates to \Expr{ERROR}.
    Returns \Expr{False} in all other cases.

    This function returns \Expr{ERROR} if other than exactly 1
    argument is given.

  \item[\Expr{Boolean isString(AnyType Expr)}]
    Returns \Expr{True}, if the evaluation of \Expr{Expr}
    gives a value of type \Expr{String}.
    Returns \Expr{False} in all other cases.

    This function returns \Expr{ERROR} if other than exactly 1
    argument is given.

  \item[\Expr{Boolean isInteger(AnyType Expr)}]
    Returns \Expr{True}, if the evaluation of \Expr{Expr}
    gives a value of type \Expr{Integer}.
    Returns \Expr{False} in all other cases.

    This function returns \Expr{ERROR} if other than exactly 1
    argument is given.

  \item[\Expr{Boolean isReal(AnyType Expr)}]
    Returns \Expr{True}, if the evaluation of \Expr{Expr}
    gives a value of type \Expr{Real}.
    Returns \Expr{False} in all other cases.

    This function returns \Expr{ERROR} if other than exactly 1
    argument is given.

  \item[\Expr{Boolean isBoolean(AnyType Expr)}]
    Returns \Expr{True}, if the evaluation of \Expr{Expr}
    gives the integer value 0 or 1.
    Returns \Expr{False} in all other cases.

    This function returns \Expr{ERROR} if other than exactly 1
    argument is given.

  \item[\Expr{Integer int(AnyType Expr)}]
    Returns the integer value as defined by \Expr{Expr}.
    Where the type of the evaluated \Expr{Expr} is \Expr{Real},
    the value is truncated (round towards zero) to an integer.
    Where the type of the evaluated \Expr{Expr} is \Expr{String},
    the string is converted to an integer using a C-like
    \Procedure{atoi} function. When this result is not an integer,
    \Expr{ERROR} is returned.
    Where the evaluated \Expr{Expr} is \Expr{ERROR} or \Expr{UNDEFINED},
    \Expr{ERROR} is returned.

    This function returns \Expr{ERROR} if other than exactly 1
    argument is given.

  \item[\Expr{Real real(AnyType Expr)}]
    Returns the real value as defined by \Expr{Expr}.
    Where the type of the evaluated \Expr{Expr} is \Expr{Integer},
    the return value is the converted integer.
    Where the type of the evaluated \Expr{Expr} is \Expr{String},
    the string is converted to a real value using a C-like
    \Procedure{atof} function. When this result is not a real,
    \Expr{ERROR} is returned.
    Where the evaluated \Expr{Expr} is \Expr{ERROR} or \Expr{UNDEFINED},
    \Expr{ERROR} is returned.

    This function returns \Expr{ERROR} if other than exactly 1
    argument is given.

  \item[\Expr{String string(AnyType Expr)}]
    Returns the string that results from the evaluation of \Expr{Expr}.
    Converts a non-string value to a string.
    Where the evaluated \Expr{Expr} is \Expr{ERROR} or \Expr{UNDEFINED},
    \Expr{ERROR} is returned.

    This function returns \Expr{ERROR} if other than exactly 1
    argument is given.

  \item[\Expr{Integer floor(AnyType Expr)}]
    Returns the integer that results from the evaluation of \Expr{Expr},
    where the type of the evaluated \Expr{Expr} is \Expr{Integer}.
    Where the type of the evaluated \Expr{Expr} is \emph{not} \Expr{Integer},
    function \Expr{real(Expr)} is called.
    Its return value is then used to return the largest magnitude
    integer that is not larger than the returned value. 
    Where \Expr{real(Expr)} returns \Expr{ERROR} or \Expr{UNDEFINED},
    \Expr{ERROR} is returned.

    This function returns \Expr{ERROR} if other than exactly 1
    argument is given.

  \item[\Expr{Integer ceiling(AnyType Expr)}]
    Returns the integer that results from the evaluation of \Expr{Expr},
    where the type of the evaluated \Expr{Expr} is \Expr{Integer}.
    Where the type of the evaluated \Expr{Expr} is \emph{not} \Expr{Integer},
    function \Expr{real(Expr)} is called.
    Its return value is then used to return the smallest magnitude
    integer that is not less than the returned value. 
    Where \Expr{real(Expr)} returns \Expr{ERROR} or \Expr{UNDEFINED},
    \Expr{ERROR} is returned.

    This function returns \Expr{ERROR} if other than exactly 1
    argument is given.

  \item[\Expr{Integer round(AnyType Expr)}]
    Returns the integer that results from the evaluation of \Expr{Expr},
    where the type of the evaluated \Expr{Expr} is \Expr{Integer}.
    Where the type of the evaluated \Expr{Expr} is \emph{not} \Expr{Integer},
    function \Expr{real(Expr)} is called.
    Its return value is then used to return the 
    integer that results from a round-to-nearest rounding method. 
    The nearest integer value to the return value is returned,
    except in the case of the value at the exact midpoint between
    two integer values.  
    In this case, the even valued integer is returned.
    Where \Expr{real(Expr)} returns \Expr{ERROR} or \Expr{UNDEFINED},
    or the integer value does not fit into 32 bits,
    \Expr{ERROR} is returned.

    This function returns \Expr{ERROR} if other than exactly 1
    argument is given.

  \item[\Expr{Integer random(\Lbr\ AnyType Expr \Rbr)}]
    Where the optional argument \Expr{Expr} evaluates to type \Expr{Integer}
    or type \Expr{Real}
    (and called \Expr{x}),
    the return value is the integer or real \Expr{r} randomly chosen
    from the interval \Expr{0 <= r < x}.
    With no argument, the return value is chosen with \Expr{random(1.0)}.
    Returns \Expr{ERROR} in all other cases.

    This function returns \Expr{ERROR} if greater than 1
    argument is given.

  \item[\Expr{String strcat(AnyType Expr1 \Lbr\ , AnyType Expr2 \Dots \Rbr)}]
    Returns the string which is the concatenation of all arguments, where all arguments are 
    converted to type \Expr{String} by function \Expr{string(Expr)}.
    Returns \Expr{ERROR} if any argument evaluates to \Expr{UNDEFINED} or \Expr{ERROR}.

  \item[\Expr{String substr(String s, Integer offset \Lbr\ , Integer length  \Rbr)}]
    Returns the substring of \Expr{s}, from the position indicated by \Expr{offset},
    with (optional) \Expr{length} characters.
    The first character within \Expr{s} is at offset 0.
    If the optional \Expr{length} argument is not present, the substring extends to the
    end of the string.
    If \Expr{offset} is negative, the value \Expr{(length - offset)} is used for the offset.
    If \Expr{length} is negative, an initial substring is computed, from the offset
    to the end of the string.
    Then, the absolute value of \Expr{length} characters are deleted from the
    right end of the initial substring.
    Further, where characters of this resulting substring lie outside the original
    string, the part that lies within the original string is returned.
    If the substring lies completely outside of the original string, the null string
    is returned.

    This function returns \Expr{ERROR} if greater than 3 or less than 2
    arguments are given.
    
  \item[\Expr{Integer strcmp(AnyType Expr1, AnyType Expr2)}]
    Both arguments are converted to type \Expr{String} by function \Expr{string(Expr)}.
    The return value is an integer that will be
    \begin{itemize}
      \item{less than 0},
      if \Expr{Expr1} is lexicographically less than \Expr{Expr2}
      \item{equal to 0},
      if \Expr{Expr1} is lexicographically equal to \Expr{Expr2}
      \item{greater than 0},
      if \Expr{Expr1} is lexicographically greater than \Expr{Expr2}
    \end{itemize}
    Case is significant in the comparison.
    Where either argument evaluates to \Expr{ERROR} or \Expr{UNDEFINED},
    \Expr{ERROR} is returned.

    This function returns \Expr{ERROR} if other than 2 arguments are given.

  \item[\Expr{Integer stricmp(AnyType Expr1, AnyType Expr2)}]
    This function is the same as \Expr{strcmp}, except that letter case is
    \emph{not} significant.

  \item[\Expr{String toUpper(AnyType Expr)}]
    The single argument is converted to type \Expr{String} by function \Expr{string(Expr)}.
    The return value is this string, with all lower case letters converted to
    upper case.
    If the argument evaluates to \Expr{ERROR} or \Expr{UNDEFINED},
    \Expr{ERROR} is returned.

    This function returns \Expr{ERROR} if greater than 1
    argument is given.

  \item[\Expr{String toLower(AnyType Expr)}]
    The single argument is converted to type \Expr{String} by function \Expr{string(Expr)}.
    The return value is this string, with all upper case letters converted to
    lower case.
    If the argument evaluates to \Expr{ERROR} or \Expr{UNDEFINED},
    \Expr{ERROR} is returned.

    This function returns \Expr{ERROR} if other than exactly 1
    argument is given.

  \item[\Expr{Integer size(AnyType Expr)}]
    Returns the number of characters in the string, after calling function
    \Expr{string(Expr)}.
    If the argument evaluates to \Expr{ERROR} or \Expr{UNDEFINED},
    \Expr{ERROR} is returned.

    This function returns \Expr{ERROR} if other than exactly 1
    argument is given.

\end{description}

For the following functions, a delimiter is represented by a string.
Each character within the delimiter string
delimits individual strings within a list of strings 
that is given by a single string.
The default delimiter contains the comma and space characters.
A string within the list is ended (delimited) by one or more
characters within the delimiter string.

\begin{description}
  \item[\Expr{Integer stringListSize(String list \Lbr\ , String delimiter \Rbr)}]
    Returns the number of elements in the string \Expr{list},
    as delimited by the optional \Expr{delimiter} string.
    Returns \Expr{ERROR} if either argument is not a string.

    This function returns \Expr{ERROR} if other than 1 or 2 arguments are given.

  \item[\Expr{Integer stringListSum(String list \Lbr\ , String delimiter \Rbr)} \\
  OR
  \Expr{Real stringListSum(String list \Lbr\ , String delimiter \Rbr)}]
    Sums and returns the sum of all items in the string \Expr{list},
    as delimited by the optional \Expr{delimiter} string.
    If all items in the list are integers, the return value is also
    an integer.
    If any item in the list is a real value (noninteger),
    the return value is a real.
    If any item does not represent an integer or real value,
    the return value is \Expr{ERROR}.


  \item[\Expr{Real stringListAve(String list \Lbr\ , String delimiter \Rbr)}]
    Sums and returns the real-valued average of all items in the 
    string \Expr{list},
    as delimited by the optional \Expr{delimiter} string.
    If any item does not represent an integer or real value,
    the return value is \Expr{ERROR}.
    A list with 0 items (the empty list) returns the value 0.0.

  \item[\Expr{Integer stringListMin(String list \Lbr\ , String delimiter \Rbr)} \\
  OR
  \Expr{Real stringListMin(String list \Lbr\ , String delimiter \Rbr)}]
    Finds and returns the minimum value from all items in the
    string \Expr{list},
    as delimited by the optional \Expr{delimiter} string.
    If all items in the list are integers, the return value is also
    an integer.
    If any item in the list is a real value (noninteger),
    the return value is a real.
    If any item does not represent an integer or real value,
    the return value is \Expr{ERROR}.
    A list with 0 items (the empty list) returns the value \Expr{UNDEFINED}.

  \item[\Expr{Integer stringListMax(String list \Lbr\ , String delimiter \Rbr)} \\
  OR
  \Expr{Real stringListMax(String list \Lbr\ , String delimiter \Rbr)}]
    Finds and returns the maximum value from all items in the
    string \Expr{list},
    as delimited by the optional \Expr{delimiter} string.
    If all items in the list are integers, the return value is also
    an integer.
    If any item in the list is a real value (noninteger),
    the return value is a real.
    If any item does not represent an integer or real value,
    the return value is \Expr{ERROR}.
    A list with 0 items (the empty list) returns the value \Expr{UNDEFINED}.

  \item[\Expr{Boolean stringListMember(String x, String list \Lbr\ , String delimiter \Rbr)}]
    Returns \Expr{TRUE} if item \Expr{x} is in the string \Expr{list},
    as delimited by the optional \Expr{delimiter} string.
    Returns \Expr{FALSE} if item \Expr{x} is not in the string \Expr{list}.
    Comparison is done with \Expr{strcmp()}.
    The return value is \Expr{ERROR}, if any of the arguments
    are not strings.

  \item[\Expr{Boolean stringListIMember(String x, String list \Lbr\ , String delimiter \Rbr)}]
    Same as \Expr{stringlistmember()}, but comparison is done
    with \Expr{stricmp()}, so letter case is not relevant.

\end{description}

The following two functions utilize regular expressions as defined
and supported by the PCRE library.
See \URL{http://www.pcre.org} for complete documentation of
regular expressions.

The \Expr{options} argument to these functions is a string of 
special characters that modify the use of the regular expressions.
Inclusion of characters other than these as options are ignored.
\begin{description}
  \item[\Expr{I} or \Expr{i}]
    Ignore letter case.
  \item[\Expr{M} or \Expr{m}]
    Modifies the interpretation of the carat (\verb@^@) and dollar sign
    (\verb@$@) characters.
    The carat character matches the start of a string, as well as
    after each newline character.
    The dollar sign character matches before a newline character.
  \item[\Expr{S} or \Expr{s}]
    The period matches any character, including the newline character. 
  \item[\Expr{X} or \Expr{x}]
    Ignore both white space and comments within the pattern.
    A comment is defined by starting with the pound sign (\verb@#@)
    character, and continuing until the newline character.
\end{description}

\begin{description}
  \item[\Expr{Boolean regexp(String pattern, String target \Lbr\ , String options \Rbr)}]
    Returns \Expr{TRUE} if the string \Expr{target} is 
    a regular expression as described by \Expr{pattern}.
    Returns \Expr{FALSE} otherwise.
    If any argument is not a string, or if \Expr{pattern} does not describe
    a valid regular expression, returns \Expr{ERROR}.

  \item[\Expr{String regexps(String pattern, String target, String substitute, \Lbr\ String options \Rbr)}]
 	The regexp pattern is applied to target.  If the pattern matches,
	the string substitute is returned, with backslash expansion performed.
    Options are the same as above.  If any of the arguments are not strings
	it returns \Expr{ERROR}.

  \item[\Expr{Boolean stringListRegexpMember(String pattern, String list \Lbr\ , String delimiter \Rbr\ \Lbr\ , String options \Rbr)}]
    Returns \Expr{TRUE} if any of the strings within the
    \Expr{list} is a regular expression as described by \Expr{pattern}.
    Returns \Expr{FALSE} otherwise.
    If any argument is not a string, or if \Expr{pattern} does not describe
    a valid regular expression, returns \Expr{ERROR}.
    To include the fourth (optional) argument \Expr{options}, a third argument
    of \Expr{delimiter} is required.
    A default value for a delimiter is " ,".

\end{description}


\begin{description}
  \item[\Expr{Integer time()}]
    Returns the current coordinated universal time, which is the same
    as the ClassAd attribute \Attr{CurrentTime}.
    This is the time, in seconds, since midnight of January 1, 1970.


  \item[\Expr{String interval(Integer seconds)}]
    Uses \Expr{seconds} to return a string of the form
    \Expr{days+hh:mm:ss}.
    This represents an interval of time.
    Leading values that are zero are omitted from the string.
    For example, \Expr{seconds} of 67 becomes "1:07".
    A second example, \Expr{seconds} of 
    1472523 = 17*24*60*60 + 1*60*60 + 2*60 + 3, results in the
    string "17+1:02:03".

\end{description}


%%%%%%%%%%%%%%%%%%%%%%%%%%%%%%%%%%%%%%%%%%%%%%%%%%%
\subsection{Evaluation Semantics}
%%%%%%%%%%%%%%%%%%%%%%%%%%%%%%%%%%%%%%%%%%%%%%%%%%%
\label{ClassAd:evaluation}
The ClassAd mechanism's primary purpose is for matching entities that supply
constraints on candidate matches.  The mechanism is therefore defined to
carry out expression evaluations in the context of two ClassAds that are
testing each other for a potential match.  For example, the \Condor{negotiator}
evaluates the \Attr{Requirements} expressions of machine and job ClassAds to
test if they can be matched.  The semantics of evaluating such constraints
is defined below.

%%%%%%%%%%%%%%%%%%%%%%%%%%%%%%%%%%%%%%%%%%%%%%%%%%%
\subsubsection{Literals}
%%%%%%%%%%%%%%%%%%%%%%%%%%%%%%%%%%%%%%%%%%%%%%%%%%%
Literals are self-evaluating, Thus, integer, string, real, undefined and
error values evaluate to themselves.

%%%%%%%%%%%%%%%%%%%%%%%%%%%%%%%%%%%%%%%%%%%%%%%%%%%
\subsubsection{Attribute References}
%%%%%%%%%%%%%%%%%%%%%%%%%%%%%%%%%%%%%%%%%%%%%%%%%%%
\index{ClassAd!scope of evaluation, MY.}
\index{ClassAd!scope of evaluation, TARGET.}
\index{TARGET., ClassAd scope resolution prefix}
\index{MY., ClassAd scope resolution prefix}
Since the expression evaluation is being carried out in the context of two
ClassAds, there is a potential for name space ambiguities.  The following
rules define the semantics of attribute references made by ad $A$ that is being 
evaluated in a context with another ad $B$:
\begin{enumerate}
    \item If the reference is prefixed by a scope resolution prefix, 
    \begin{itemize}
        \item If the prefix is \texttt{MY.}, the attribute is looked up in 
        ClassAd $A$.  If the named attribute does not exist in $A$, the
        value of the reference is \texttt{UNDEFINED}.  Otherwise, the
        value of the reference is the value of the expression bound to
        the attribute name.

        \item Similarly, if the prefix is \texttt{TARGET.}, the attribute is 
        looked up in ClassAd $B$.  If the named attribute does not exist in 
        $B$, the value of the reference is \texttt{UNDEFINED}.  Otherwise, 
        the value of the reference is the value of the expression bound to
        the attribute name.

    \end{itemize}

    \item If the reference is not prefixed by a scope resolution prefix,
    \begin{itemize}
        \item If the attribute is defined in $A$, the value of the reference
        is the value of the expression bound to the attribute name in $A$.
        \item Otherwise, if the attribute is defined in $B$, the value of the
        reference is the value of the expression bound to the attribute
        name in $B$.
        \item Otherwise, if the attribute is defined in the ClassAd environment, the
        value from the environment is returned.
        This is a special environment, to be
        distinguished from the Unix environment.
        Currently, the only attribute
        of the environment is \Attr{CurrentTime}, which evaluates to the
        integer value returned by the system call \texttt{time(2)}.
        \item Otherwise, the value of the reference is \texttt{UNDEFINED}.
    \end{itemize}

    \item Finally, if the reference refers to an expression that is itself in 
    the process of being evaluated, there is a circular dependency in the 
    evaluation.  The value of the reference is \texttt{ERROR}.
\end{enumerate}

%%%%%%%%%%%%%%%%%%%%%%%%%%%%%%%%%%%%%%%%%%%%%%%%%%%
\subsubsection{Operators}
%%%%%%%%%%%%%%%%%%%%%%%%%%%%%%%%%%%%%%%%%%%%%%%%%%%
\label{ClassAd:evaluation-meta}
\index{ClassAd!expression operators}
All operators in the ClassAd language are \Term{total}, and thus have well
defined behavior regardless of the supplied operands.  Furthermore, most
operators are \Term{strict} with respect to \texttt{ERROR} and 
\texttt{UNDEFINED}, and thus evaluate to \texttt{ERROR} (or \texttt{UNDEFINED})
if either of their operands have these exceptional values.

\begin{itemize}
	\item\textbf{Arithmetic operators:}  
	\begin{enumerate}
		\item The operators \verb@*@, \verb@/@, \verb@+@ and \verb@-@ operate 
		arithmetically only on integers and reals.

		\item Arithmetic is carried out in the same type as both operands,
		and type promotions from integers to reals are performed if one operand 
		is an integer and the other real.

		\item The operators are strict with respect to both \texttt{UNDEFINED} 
		and \texttt{ERROR}.  

		\item If either operand is not a numerical type, the value of the
		operation is \texttt{ERROR}.
	\end{enumerate}

	\item\textbf{Comparison operators:}
	\begin{enumerate}
		\item The comparison operators \verb@==@, \verb@!=@, \verb@<=@, 
		\verb@<@, \verb@>=@ and \verb@>@ operate on integers, reals and strings.

		\item String comparisons are case insensitive for most operators.  The only
		exceptions are the operators \verb@=?=@ and \verb@=!=@, which do case sensitive
		comparisons assuming both sides are strings. 

		\item Comparisons are carried out in the same type as both operands,
		and type promotions from integers to reals are performed if one operand
		is a real, and the other an integer.  Strings may not be converted to
		any other type, so comparing a string and an integer or a
		string and a real results in \texttt{ERROR}.

		\item The operators \verb@==@, \verb@!=@, \verb@<=@, \verb@<@ and 
		\verb@>=@ \verb@>@ are strict with respect to both \texttt{UNDEFINED} 
		and \texttt{ERROR}.

		\item In addition, the operators \verb@=?=@ and \verb@=!=@ behave
		similar to \verb@==@ and \verb@!=@, but are not strict.  Semantically,
		the \verb@=?=@ tests if its operands are ``identical,'' i.e., have
		the same type and the same value.  For example, \verb@10 == UNDEFINED@ 
		and \verb@UNDEFINED == UNDEFINED@ both evaluate to \texttt{UNDEFINED},
		but \verb@10 =?= UNDEFINED@ and \verb@UNDEFINED =?= UNDEFINED@ 
		evaluate to \texttt{FALSE} and \texttt{TRUE} respectively.  The
		\verb@=!=@ operator test for the ``is not identical to'' condition.
	\end{enumerate}

	\item\textbf{Logical operators:}
	\begin{enumerate}
		\item The logical operators \verb@&&@ and \verb@||@ operate on 
		integers and reals.  The zero value of these types are considered 
		\texttt{FALSE} and non-zero values \texttt{TRUE}.

		\item The operators are \emph{not} strict, and exploit the 
		``don't care'' properties of the operators to squash \texttt{UNDEFINED}
		and \texttt{ERROR} values when possible.  For example,
		\verb@UNDEFINED && FALSE@ evaluates to \texttt{FALSE}, but	
		\verb@UNDEFINED || FALSE@ evaluates to \texttt{UNDEFINED}.

		\item Any string operand is equivalent to an \texttt{ERROR} operand
		for a logical operator.  In other words,
		\verb@TRUE && "foobar"@ evaluates to \texttt{ERROR}.
	\end{enumerate}
\end{itemize}

%%%%%%%%%%%%%%%%%%%%%%%%%%%%%%%%%%%%%%%%%%%%%%%%%%%
\subsection{ClassAds in the Condor System}
%%%%%%%%%%%%%%%%%%%%%%%%%%%%%%%%%%%%%%%%%%%%%%%%%%%
The simplicity and flexibility of ClassAds is heavily exploited in the Condor
system.  ClassAds are not only used to represent machines and jobs in the 
Condor pool, but also other entities that exist in the pool such as 
checkpoint servers, submitters of jobs and master daemons.  Since arbitrary
expressions may be supplied and evaluated over these ads, users have a uniform
and powerful mechanism to specify constraints over these ads.  These constraints
can take the form of \Attr{Requirements} expressions in resource and job ads,
or queries over other ads.

%%%%%%%%%%%%%%%%%%%%%%%%%%%%%%%%%%%%%%%%%%%%%%%%%%%
\subsubsection{Constraints and Preferences}
%%%%%%%%%%%%%%%%%%%%%%%%%%%%%%%%%%%%%%%%%%%%%%%%%%%
\index{ClassAd attribute!requirements}
\index{ClassAd attribute!rank}
The \AdAttr{requirements} and \AdAttr{rank} expressions
within the submit description file
are the mechanism
by which users specify the constraints and preferences of jobs.
For machines, the configuration determines both 
constraints and preferences of the machines.

\index{rank attribute!examples}
\index{requirements attribute}
For both machine and job, 
the \Attr{rank} expression specifies
the desirability of the match (where higher numbers mean better matches).
For example, a job ad may contain the following expressions:
\footnotesize
\begin{verbatim}
Requirements = Arch=="SUN4u" && OpSys == "SOLARIS251"
Rank         = TARGET.Memory + TARGET.Mips
\end{verbatim}
\normalsize
In this case, the job requires an UltraSparc computer running the Solaris 
2.5.1 operating system.
Among all such computers,
the customer prefers those with large physical memories and high MIPS ratings.  
Since the \Attr{Rank} is a user-specified metric,
\emph{any} expression may be used to specify the
perceived desirability of the match.
The \Condor{negotiator} daemon runs algorithms
to deliver the best resource (as defined by the \Attr{rank} expression)
while satisfying other required criteria.

Similarly, the machine may place constraints and preferences on 
the jobs that it will run by setting the machine's configuration.
For example,
\footnotesize
\begin{verbatim}
    Friend        = Owner == "tannenba" || Owner == "wright"
    ResearchGroup = Owner == "jbasney" || Owner == "raman"
    Trusted       = Owner != "rival" && Owner != "riffraff"
    START         = Trusted && ( ResearchGroup || LoadAvg < 0.3 &&
                         KeyboardIdle > 15*60 )
    RANK          = Friend + ResearchGroup*10
\end{verbatim}
\normalsize

The above policy states that the computer will never run jobs owned by
users rival and riffraff, while the computer will always run a 
job submitted by members of the research group.
Furthermore,
jobs submitted by friends are preferred to other foreign jobs,
and jobs submitted
by the research group are preferred to jobs submitted by friends. 

\textbf{Note:}  Because of the dynamic nature of ClassAd expressions, there
is no \emph{a priori} notion of an integer-valued expression, a real-valued
expression, etc.  However, it is intuitive to think of the \Attr{Requirements}
and \Attr{Rank} expressions as integer-valued and real-valued expressions,
respectively.  If the actual type of the expression is not of the expected 
type, the value is assumed to be zero.

%%%%%%%%%%%%%%%%%%%%%%%%%%%%%%%%%%%%%%%%%%%%%%%%%%%
\subsubsection{Querying with ClassAd Expressions}
%%%%%%%%%%%%%%%%%%%%%%%%%%%%%%%%%%%%%%%%%%%%%%%%%%%
The flexibility of this system may also be used when querying ClassAds
through the \Condor{status} and \Condor{q} tools which allow users to
supply ClassAd constraint expressions from the command line.

For example, to find all computers which have had their keyboards idle for 
more than 20 minutes and have more than 100 MB of memory:
\footnotesize
\begin{verbatim}
% condor_status -const 'KeyboardIdle > 20*60 && Memory > 100'

Name       Arch     OpSys        State      Activity   LoadAv Mem  ActvtyTime

amul.cs.wi SUN4u    SOLARIS251   Claimed    Busy       1.000  128   0+03:45:01
aura.cs.wi SUN4u    SOLARIS251   Claimed    Busy       1.000  128   0+00:15:01
balder.cs. INTEL    SOLARIS251   Claimed    Busy       1.000  1024  0+01:05:00
beatrice.c INTEL    SOLARIS251   Claimed    Busy       1.000  128   0+01:30:02
...
...
                     Machines Owner Claimed Unclaimed Matched Preempting

    SUN4u/SOLARIS251        3     0       3         0       0          0
    INTEL/SOLARIS251       21     0      21         0       0          0
    SUN4x/SOLARIS251        3     0       3         0       0          0
           SGI/IRIX6        1     0       0         1       0          0
         INTEL/LINUX        1     0       1         0       0          0

               Total       29     0      28         1       0          0
\end{verbatim}
\normalsize

Here is an example that utilizes a regular expression
ClassAd function to list specific information.
A file contains ClassAd information.
\Condor{advertise} is used to inject this information,
and \Condor{status} constrains the search with an expression
that contains a ClassAd function.

\footnotesize
\begin{verbatim}
% cat ad
MyType = "Generic"
FauxType = "DBMS"
Name = "random-test"
Machine = "f05.cs.wisc.edu"
MyAddress = "<128.105.149.105:34000>"
DaemonStartTime = 1153192799
UpdateSequenceNumber = 1

% condor_advertise UPDATE_AD_GENERIC ad

% condor_status -any -constraint 'FauxType=="DBMS" && regexp("random.*", Name, "i")'

MyType               TargetType           Name                          

Generic              None                 random-test                   

\end{verbatim}
\normalsize

Similar flexibility exists in querying job queues in the Condor system.

\index{ClassAd|)}
