%%%%%%%%%%%%%%%%%%%%%%%%%%%%%%%%%%%%%%%%%%%%%%%%%%%%%%%%%%%%%%%%%%%%%%
\section{\label{sec:logging}Logging in HTCondor}
%%%%%%%%%%%%%%%%%%%%%%%%%%%%%%%%%%%%%%%%%%%%%%%%%%%%%%%%%%%%%%%%%%%%%%
\index{logging|(}
HTCondor records many types of information in a variety of logs.
Administration may require locating and using the
contents of a log to debug issues.
Listed here are details of the logs, to aid in identification.

%%%%%%%%%%%%%%%%%%%%%%%%%%%%%%%%%%%%%%%%%%%%%%%%%%%%%%%%%%%%%%%%%%%%%%
\subsection{\label{sec:job-systemlogs}Job and Daemon Logs}
%%%%%%%%%%%%%%%%%%%%%%%%%%%%%%%%%%%%%%%%%%%%%%%%%%%%%%%%%%%%%%%%%%%%%%
\begin{description}
\item[job event log]
  The job event log is an optional, chronological list of events that occur
  as a job runs.
  The submit description file for the job requests a job event log with
  the submit command \SubmitCmd{log}.
  The log is created and remains on the submit machine.
  Contents of the log are detailed in section~\ref{sec:job-log-events}.
  Examples of events are that the job is running,
  that the job  is placed on hold, or that the job completed.
\item[daemon logs]
  Each daemon configured to have a log writes events relevant to that daemon.
  Each event written consists of a timestamp and message.
  The name of the file is set by the value of configuration variable
  \Macro{<SUBSYS>\_LOG}, where \MacroNI{<SUBSYS>} is replaced by the name
  of the daemon.
  The log is not permitted to grow without bound;
  log rotation takes place after a configurable maximum size or length of time
  is encountered.
  This maximum is specified by configuration variable 
  \Macro{MAX\_<SUBSYS>\_LOG}.

  Which events are logged for a particular daemon are determined by 
  the value of configuration variable \Macro{<SUBSYS>\_DEBUG}.
  The possible values for \MacroNI{<SUBSYS>\_DEBUG} categorize events,
  such that it is possible to control the level and quantity of events
  written to the daemon's log.

  Configuration variables that affect daemon logs are 
  \begin{description}
  \item [\Macro{MAX\_NUM\_<SUBSYS>\_LOG}] 
  \item [\Macro{TRUNC\_<SUBSYS>\_LOG\_ON\_OPEN}] 
  \item [\Macro{<SUBSYS>\_LOG\_KEEP\_OPEN}] 
  \item [\Macro{<SUBSYS>\_LOCK}] 
  \item [\Macro{FILE\_LOCK\_VIA\_MUTEX}] 
  \item [\Macro{TOUCH\_LOG\_INTERVAL}] 
  \item [\Macro{LOGS\_USE\_TIMESTAMP}] 
  \end{description}

  Daemon logs are often investigated to accomplish administrative debugging.
  \Condor{config\_val} can be used to determine the location and file name
  of the daemon log.
  For example, to display the location of the log for the \Condor{collector} 
  daemon, use
\begin{verbatim}
  condor_config_val COLLECTOR_LOG
\end{verbatim}

\item[job queue log]
  The job queue log is a transactional representation of the current job queue. 
  If the \Condor{schedd} crashes, the job queue can be rebuilt using
  this log.
  The file name is set by configuration variable \Macro{JOB\_QUEUE\_LOG},
  and defaults to \File{\$(SPOOL)/job\_queue.log}.

  Within the log,
  each transaction is identified with an integer value and followed where
  appropriate with other values relevant to the transaction.
  To reduce the size of the log and remove any transactions that are 
  no longer relevant,
  a copy of the log is kept
  by renaming the log at each time interval defined by configuration variable
  \MacroNI{QUEUE\_CLEAN\_INTERVAL}, 
  and then a new log is written with only current and relevant transactions. 

  Configuration variables that affect the job queue log are 
  \begin{description}
  \item [\Macro{SCHEDD\_BACKUP\_SPOOL}] 
  \item [\Macro{ROTATE\_HISTORY\_DAILY}] 
  \item [\Macro{ROTATE\_HISTORY\_MONTHLY}] 
  \item [\Macro{QUEUE\_CLEAN\_INTERVAL}] 
  \item [\Macro{MAX\_JOB\_QUEUE\_LOG\_ROTATIONS}] 
  \end{description}
\item[\Condor{schedd} audit log]
  Configuration variables that affect the audit log are 
  \begin{description}
  \item [\Macro{SCHEDD\_AUDIT\_LOG}] 
  \item [\Macro{MAX\_SCHEDD\_AUDIT\_LOG}] 
  \item [\Macro{MAX\_NUM\_SCHEDD\_AUDIT\_LOG}] 
  \end{description}
\item[event log]
  The event log is an optional, chronological list of events that occur
  for all jobs and all users.
  The events logged are the same as those that would go into a job event
  log.
  The file name is set by configuration variable \Macro{EVENT\_LOG}.
  The log is created only if this configuration variable is set.

  Configuration variables that affect the event log, 
  setting details such as
  the maximum size to which this log may grow and details of file rotation
  and locking are
  \begin{description}
  \item [\Macro{EVENT\_LOG\_MAX\_SIZE}] 
  \item [\Macro{EVENT\_LOG\_MAX\_ROTATIONS}]
  \item [\Macro{EVENT\_LOG\_LOCKING}]
  \item [\Macro{EVENT\_LOG\_FSYNC}]
  \item [\Macro{EVENT\_LOG\_ROTATION\_LOCK}]
  \item [\Macro{EVENT\_LOG\_JOB\_AD\_INFORMATION\_ATTRS}]
  \item [\Macro{EVENT\_LOG\_USE\_XML}]
  \end{description}

  
\item[accountant log]
\item[negotiator match log]
\end{description}
\MoreTodo

%%%%%%%%%%%%%%%%%%%%%%%%%%%%%%%%%%%%%%%%%%%%%%%%%%%%%%%%%%%%%%%%%%%%%%
\subsection{\label{sec:DAGMan-logs}DAGMan Logs}
%%%%%%%%%%%%%%%%%%%%%%%%%%%%%%%%%%%%%%%%%%%%%%%%%%%%%%%%%%%%%%%%%%%%%%
\MoreTodo

\index{logging|)}
