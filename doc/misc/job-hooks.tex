%%%%%%%%%%%%%%%%%%%%%%%%%%%%%%%%%%%%%%%%%%%%%%%%%%%%%%%%%%%%%%%%%%%%%%
\section{\label{sec:job-hooks}Job Hooks}
%%%%%%%%%%%%%%%%%%%%%%%%%%%%%%%%%%%%%%%%%%%%%%%%%%%%%%%%%%%%%%%%%%%%%%
\index{Job hooks|(}

\index{Job hooks!introduction} 
In the past, Condor has always sent work to the execute machines by
pushing jobs to the \Condor{startd} daemon, either from the \Condor{schedd}
daemon or via \Condor{cod}.
Beginning with the Condor 7.1.0, the \Condor{startd} daemon now has the
ability to pull work by fetching jobs via a system of plug-ins or
hooks.
Any site can configure a set of hooks to fetch work completely
outside of the usual Condor matchmaking system.

A \Term{hook} is an external program or script invoked by Condor at various
points during the life cycle of a job.
Instead of putting all the code and logic directly into the Condor
daemons to handle the variety of external systems from which it
might fetch work,
sites can write their own programs or scripts and allow
Condor to invoke these hooks at the right moments to accomplish the
desired outcome.
This eliminates the expense of the matchmaking and 
scheduling provided by the the \Condor{schedd} and
the \Condor{negotiator}, although at the price of the flexibility
they offer.
Therefore, the hooks allow Condor to more easily and directly interface with
external scheduling systems.

A projected use of the hook mechanism implements what might
be termed a \Term{glide-in factory}, especially where the
factory is behind a firewall.
Without using the hook mechanism to fetch work,
a glide-in \Condor{startd} daemon behind a firewall
depends on GCB to help it listen and eventually receive
work pushed from elsewhere.
With the hook mechanism, a glide-in \Condor{startd} daemon
behind a firewall uses the hook to pull work.
The hook needs only an outbound network connection to complete
its task,
thereby being able to operate from behind the firewall,
without the intervention of GCB.

The following sections describe how this system of hooks works,
the semantics of fetched jobs, the interaction between fetched work
and regular Condor jobs, what hooks are invoked by Condor at various
stages of the job work flow, how to configure a machine to fetch jobs,
and how to write your own hooks.


%%%%%%%%%%%%%%%%%%%%%%%%%%%%%%%%%%%%%%%%%%%%%%%%%%%%%%%%%%%%%%%%%%%%%%
\subsection{\label{sec:job-hooks-overview}
Overview of Fetching Work}
%%%%%%%%%%%%%%%%%%%%%%%%%%%%%%%%%%%%%%%%%%%%%%%%%%%%%%%%%%%%%%%%%%%%%%
\index{Job hooks!overview} 

Periodically, each execution slot managed by a \Condor{startd} will
invoke a hook to see if there is any work that can be fetched.
Whenever this hook returns a valid job, the \Condor{startd} will
evaluate the current state of the slot and decide if it should start
executing the fetched work.
If the slot is unclaimed and the \Attr{Start} expression evaluates to
TRUE, a new claim will be created for the fetched job.
If the slot is claimed, the \Condor{startd} will evaluate the
\Attr{Rank} expression relative to the fetched job and compare it to
the value of the \Attr{Rank} for the currently running job and decide
if the existing job should be preempted due to the fetched job having
a higher rank.
If the slot is unavailable for whatever reason, the \Condor{startd}
will refuse the fetched job and ignore it.
Either way, once the \Condor{startd} decides what it should do with
the fetched job, it will invoke another hook to reply to the claim
request, so that the external system knows what happened to that work
unit.

If the job is accepted, a claim is created for it and the slot moves
into the Claimed state.
As soon as this happens, the \Condor{startd} will spawn a
\Condor{starter} to manage the execution of the job.
At this point, from the perspective of the \Condor{startd}, this claim
is just like any other.
The usual policy expressions are evaluated, and if the job needs to be
suspended or evicted, it will be.
If a higher-ranked job being managed by a \Condor{schedd} is matched
with the slot, that job will preempt the fetched work.

The \Condor{starter} itself can optionally invoke additional hooks to
help manage the execution of the specific job.
There are hooks to prepare the execution environment for the job,
periodically update information about the job as it runs, notify when
the job exits, and to take special actions when the job is being evicted.

Assuming there are no interruptions, the job completes, and the
\Condor{starter} exits, the \Condor{startd} will invoke the hook to
fetch work again.
If another job is available, the existing claim will be reused and a
new \Condor{starter} is spawned.
If the hook returns that there is no more work to perform, the claim
will be evicted, and the slot will return to the Owner state.


%%%%%%%%%%%%%%%%%%%%%%%%%%%%%%%%%%%%%%%%%%%%%%%%%%%%%%%%%%%%%%%%%%%%%%
\subsection{\label{sec:job-hooks-hooks}
Hooks Invoked by Condor}
%%%%%%%%%%%%%%%%%%%%%%%%%%%%%%%%%%%%%%%%%%%%%%%%%%%%%%%%%%%%%%%%%%%%%%
\index{Job hooks!Hooks invoked by Condor} 
\index{Job hooks!Hooks}

There are a handful of hooks invoked by Condor related to fetching
work, some of which are called by the \Condor{startd} and others by
the \Condor{starter}.
Each hook will be described below, including when it is invoked, what
task it is supposed to accomplish, what data is passed to the hook,
and what output (and, when relevant) exit status is expected.


%%%%%%%%%%%%%%%%%%%%%%%%%%%%%%%%%%%%%%%%%%%%%%%%%%%%%%%%%%%%%%%%%%%%%%
\subsubsection{\label{sec:job-hooks-fetch-work}
Hook: Fetch Work}
%%%%%%%%%%%%%%%%%%%%%%%%%%%%%%%%%%%%%%%%%%%%%%%%%%%%%%%%%%%%%%%%%%%%%%
\index{Job hooks!Hooks!Fetch work}

\Macro{HOOK\_FETCH\_WORK} is invoked whenever the \Condor{startd}
wants to see if there is any work to fetch.
This is the most important hook in the whole system, and is the only
hook that must be defined for any of the other \Condor{startd} hooks
to operate.

\begin{description}
\item[Arguments]
  None.

\item[Standard input]
  ClassAd of the slot that is looking for work.

\item[Expected output]
  ClassAd of a job that can be run.
  If there is no work, the hook should return no output.

\item[Exit status]
  Ignored.
\end{description}

The job ClassAd returned by the hook needs to contain enough
information for the \Condor{starter} to eventually spawn the work.
The required and optional attributes in this ClassAd are identical to
the ones described for Computing on Demand (COD) jobs in
section~\ref{sec:cod-application-attributes} 
``COD Application Attributes'' on
page~\pageref{sec:cod-application-attributes}.


%%%%%%%%%%%%%%%%%%%%%%%%%%%%%%%%%%%%%%%%%%%%%%%%%%%%%%%%%%%%%%%%%%%%%%
\subsubsection{\label{sec:job-hooks-reply-claim}
Hook: Reply Claim}
%%%%%%%%%%%%%%%%%%%%%%%%%%%%%%%%%%%%%%%%%%%%%%%%%%%%%%%%%%%%%%%%%%%%%%
\index{Job hooks!Hooks!Reply to a claim}

\Macro{HOOK\_REPLY\_CLAIM} is invoked whenever
\Macro{HOOK\_FETCH\_WORK} returns data and the the \Condor{startd}
decides if it's going to accept the job or not.

\begin{description}
\item[Arguments]
  Either the string \verb@accept@ or \verb@reject@.

\item[Standard input]
  A copy of the job ClassAd and the slot ClassAd
  (separated by the string \verb@-----@ and a new line).

\item[Expected output]
  None.

\item[Exit status]
  Ignored.
\end{description}

The \Condor{startd} will not wait for this hook to return before
taking other actions, and ignores all output.
The hook is simply advisory, and has no impact on the behavior of the
\Condor{startd}.


%%%%%%%%%%%%%%%%%%%%%%%%%%%%%%%%%%%%%%%%%%%%%%%%%%%%%%%%%%%%%%%%%%%%%%
\subsubsection{\label{sec:job-hooks-evict-claim}
Hook: Evict Claim}
%%%%%%%%%%%%%%%%%%%%%%%%%%%%%%%%%%%%%%%%%%%%%%%%%%%%%%%%%%%%%%%%%%%%%%
\index{Job hooks!Hooks!Evict a claim}

\Macro{HOOK\_EVICT\_CLAIM} is invoked whenever the \Condor{startd}
needs to evict a claim representing fetched work.

\begin{description}
\item[Arguments]
  None.

\item[Standard input]
  A copy of the job ClassAd and the slot ClassAd
  (separated by the string \verb@-----@ and a new line).

\item[Expected output]
  None.

\item[Exit status]
  Ignored.
\end{description}

The \Condor{startd} will not wait for this hook to return before
taking other actions, and ignores all output.
The hook is simply advisory, and has no impact on the behavior of the
\Condor{startd}.


%%%%%%%%%%%%%%%%%%%%%%%%%%%%%%%%%%%%%%%%%%%%%%%%%%%%%%%%%%%%%%%%%%%%%%
\subsubsection{\label{sec:job-hooks-prepare-job}
Hook: Prepare Job}
%%%%%%%%%%%%%%%%%%%%%%%%%%%%%%%%%%%%%%%%%%%%%%%%%%%%%%%%%%%%%%%%%%%%%%
\index{Job hooks!Hooks!Prepare job}

\Macro{HOOK\_PREPARE\_JOB} is invoked by the \Condor{starter} before
a job is going to be run.
This hook provides a chance to execute commands to setup the job
environment, for example to transfer input files.

\begin{description}
\item[Arguments]
  None.

\item[Standard input]
  A copy of the job ClassAd and the slot ClassAd
  (separated by the string \verb@-----@ and a new line).

\item[Expected output]
  None.

\item[Exit status]
  0 for success preparing the job, any non-zero value on failure.
\end{description}

The \Condor{starter} waits until this hook returns before
attempting to execute the job.
If the hook returns a non-zero exit status, the \Condor{starter} will
assume an error was reached while attempting to setup the job
environment and abort the job.


%%%%%%%%%%%%%%%%%%%%%%%%%%%%%%%%%%%%%%%%%%%%%%%%%%%%%%%%%%%%%%%%%%%%%%
\subsubsection{\label{sec:job-hooks-update-job-info}
Hook: Update Job Info}
%%%%%%%%%%%%%%%%%%%%%%%%%%%%%%%%%%%%%%%%%%%%%%%%%%%%%%%%%%%%%%%%%%%%%%
\index{Job hooks!Hooks!Update job info}

\Macro{HOOK\_UPDATE\_JOB\_INFO} is invoked periodically during the
life of the job to update information about the status of the job.
When the job is first spawned, the \Condor{starter} will invoke this
hook after \Macro{STARTER\_INITIAL\_UPDATE\_INTERVAL} seconds
(defaults to 8).
Thereafter, the \Condor{starter} will invoke the hook every 
\Macro{STARTER\_INITIAL\_UPDATE\_INTERVAL} seconds (defaults to 300,
in other words, every 5 minutes).

\begin{description}
\item[Arguments]
  None.

\item[Standard input]
  A copy of the job ClassAd that has been augmented with additional
  attributes describing the current status and execution behavior of
  the job.

\item[Expected output]
  None.

\item[Exit status]
  Ignored.
\end{description}

The \Condor{starter} will not wait for this hook to return before
taking other actions, and ignores all output.
The hook is simply advisory, and has no impact on the behavior of the
\Condor{starter}.

The additional attributes included inside the job ClassAd are:
\begin{description}
\item[\AdAttr{JobState}]
  The current state of the job.
  Can be either ``Running'' or ``Suspended''.

\item[\AdAttr{JobPid}]
  The process identifier for the initial job directly spawned by the
  \Condor{starter}.

\item[\AdAttr{NumPids}]
  The number of processes that the job has currently spawned.

\item[\AdAttr{JobStartDate}]
  The epoch time when the job was first spawned by the \Condor{starter}.

\item[\AdAttr{RemoteSysCpu}]
  The total number of seconds of system CPU time (the time spent at
  system calls) the job has used.

\item[\AdAttr{RemoteUserCpu}]
  The total number of seconds of user CPU time the job has used.

\item[\AdAttr{ImageSize}]
  The memory image size of the job in Kbytes.
\end{description}


%%%%%%%%%%%%%%%%%%%%%%%%%%%%%%%%%%%%%%%%%%%%%%%%%%%%%%%%%%%%%%%%%%%%%%
\subsubsection{\label{sec:job-hooks-job-exit}
Hook: Job Exit}
%%%%%%%%%%%%%%%%%%%%%%%%%%%%%%%%%%%%%%%%%%%%%%%%%%%%%%%%%%%%%%%%%%%%%%
\index{Job hooks!Hooks!Job Exit}

\Macro{HOOK\_JOB\_EXIT} is invoked whenever a job exits, either on its
own or when being evicted from an execution slot.

\begin{description}
\item[Arguments]
  A string describing how the job exited:
  \begin{itemize}
    \item \verb@exit@ The job exited or died with a signal on its own.
    \item \verb@remove@ The job was removed with \Condor{rm} or the
    user job policy expressions (for example, \Attr{PeriodicRemove}).
    \item \verb@hold@ The job was held with \Condor{hold} or the
    user job policy expressions (for example, \Attr{PeriodicHold}).
    \item \verb@evict@ The job was evicted from the execution slot for
    any other reason (\Macro{PREEMPT} evaluated to TRUE in the
    \Condor{startd}, \Condor{vacate}, \Condor{off}, etc).
  \end{itemize}

\item[Standard input]
  A copy of the job ClassAd that has been augmented with additional
  attributes describing the execution behavior of the job and its
  final results.

\item[Expected output]
  None.

\item[Exit status]
  Ignored.
\end{description}

The \Condor{starter} will wait for this hook to return before
taking any other actions.
In the case of jobs that are being managed by a \Condor{shadow}, this
hook is invoked before the \Condor{starter} does its own optional file
transfer back to the submission machine, writes to the local user log
file, or notifies the \Condor{shadow} that the job has exited.

The job ClassAd passed to this hook contains all of the extra
attributes described above for \Macro{HOOK\_UPDATE\_JOB\_INFO}, and
the following additional attributes that are only present once a job
exits:
\begin{description}
\item[\AdAttr{ExitReason}]
  A human-readable string describing why the job exited.

\item[\AdAttr{ExitBySignal}]
  A boolean indicating if the job exited due to being killed by a
  signal, or if it exited with an exit status.

\item[\AdAttr{ExitSignal}]
  If \AdAttr{ExitBySignal} is true, the signal number that killed the job.

\item[\AdAttr{ExitCode}]
  If \AdAttr{ExitBySignal} is false, the integer exit code of the job.

\item[\AdAttr{JobDuration}]
  The number of seconds that the job ran during this invocation.
\end{description}

% \Todo
% If the job was running in the Java universe and died with a Java
% exception, the following attributes can be present:
% ATTR_EXCEPTION_HIERARCHY
% ATTR_EXCEPTION_NAME
% ATTR_EXCEPTION_TYPE

% \Todo
% Is this relevant?
% If the job is running in the Virtual Machine (VM) universe and 
% requested checkpointing by setting \AdAttr{JobVMCheckpoint} to true in
% its ClassAd, the following additional attributes might be present:
% \begin{description}
% \item[\AdAttr{VM\_CkptMac}]
%   The MAC address of the virtual machine.
% 
% \item[\AdAttr{VM\_CkptIP}]
%   The IP address of the virtual machine.
% \end{description}


%%%%%%%%%%%%%%%%%%%%%%%%%%%%%%%%%%%%%%%%%%%%%%%%%%%%%%%%%%%%%%%%%%%%%%
\subsection{\label{sec:job-hooks-keywords}
Keywords to Define Hooks in the Condor Configuration files }
%%%%%%%%%%%%%%%%%%%%%%%%%%%%%%%%%%%%%%%%%%%%%%%%%%%%%%%%%%%%%%%%%%%%%%
\index{Job hooks!keywords} 

Hooks are defined in the Condor configuration files by prefixing
the name of the hook with a keyword.
This way, a given machine can have multiple sets of hooks, each set
identified by a specific keyword.

Each slot on the machine can define a separate keyword for the set
of hooks that should be used (\Macro{[SLOTN\_JOB\_HOOK\_KEYWORD}).
Note that the ``N'' in ``SLOTN'' should be replaced with the slot
identification number, for example, on slot1, the setting would be
called \MacroNI{[SLOT1\_JOB\_HOOK\_KEYWORD}.
If the slot-specific keyword is not defined, the \Condor{startd} will
use a global keyword (\Macro{STARTD\_JOB\_HOOK\_KEYWORD}).

Once a job is fetched via \Macro{HOOK\_FETCH\_WORK}, the
\Condor{startd} will insert the keyword used to fetch that job into
the job ClassAd as \AdAttr{HookKeyword}.
This way, the same keyword will be used to select the hooks invoked by
the \Condor{starter} during the actual execution of the job.
However, the \Macro{STARTER\_JOB\_HOOK\_KEYWORD} can be defined to
force the \Condor{starter} to always use a given keyword for its own
hooks, instead of looking the job ClassAd for a \AdAttr{HookKeyword}
attribute.

For example, the following configuration defines two sets of hooks,
and on a machine with 4 slots, 3 of the slots use the global keyword
for running work from a database-driven system, and one of the slots
uses a custom keyword to handle work fetched from a web service.
\begin{verbatim}
  # Most slots fetch and run work from the database system.
  STARTD_JOB_HOOK_KEYWORD = DATABASE

  # Slot4 fetches and runs work from a web service.
  SLOT4_JOB_HOOK_KEYWORD = WEB

  # The database system needs to both provide work and know the reply
  # for each attempted claim.
  DATABASE_HOOK_DIR = /usr/local/condor/fetch/database
  DATABASE_HOOK_FETCH_WORK = $(DATABASE_HOOK_DIR)/fetch_work.php
  DATABASE_HOOK_REPLY_CLAIM = $(DATABASE_HOOK_DIR)/reply_claim.php

  # The web system only needs to fetch work.
  WEB_HOOK_DIR = /usr/local/condor/fetch/web
  WEB_HOOK_FETCH_WORK = $(WEB_HOOK_DIR)/fetch_work.php
\end{verbatim}

The keywords ``DATABASE'' and ``WEB'' are completely arbitrary, so
each site is encouraged to use different (more specific) names as
appropriate for their own needs.

\index{Job hooks|)}
