%%%%%%%%%%%%%%%%%%%%%%%%%%%%%%%%%%%%%%%%%%%%%%%%%%%%%%%%%%%%%%%%%%%%%%
\subsection{\label{sec:job-hooks}Job Hooks That Fetch Work}
%%%%%%%%%%%%%%%%%%%%%%%%%%%%%%%%%%%%%%%%%%%%%%%%%%%%%%%%%%%%%%%%%%%%%%
\index{Job hooks}
\index{Hooks!job hooks that fetch work}

In the past, Condor has always sent work to the execute machines by
pushing jobs to the \Condor{startd} daemon, either from the \Condor{schedd}
daemon or via \Condor{cod}.
Beginning with the Condor version 7.1.0, the \Condor{startd} daemon now has the
ability to pull work by fetching jobs via a system of plug-ins or
hooks.
Any site can configure a set of hooks to fetch work, completely
outside of the usual Condor matchmaking system.

A projected use of the hook mechanism implements what might
be termed a \Term{glide-in factory}, especially where the
factory is behind a firewall.
Without using the hook mechanism to fetch work,
a glide-in \Condor{startd} daemon behind a firewall
depends on CCB or GCB to help it listen and eventually receive
work pushed from elsewhere.
With the hook mechanism, a glide-in \Condor{startd} daemon
behind a firewall uses the hook to pull work.
The hook needs only an outbound network connection to complete
its task,
thereby being able to operate from behind the firewall,
without the intervention of CCB or GCB.

Periodically, each execution slot managed by a \Condor{startd} will
invoke a hook to see if there is any work that can be fetched.
Whenever this hook returns a valid job, the \Condor{startd} will
evaluate the current state of the slot and decide if it should start
executing the fetched work.
If the slot is unclaimed and the \Attr{Start} expression evaluates to
\Expr{True}, a new claim will be created for the fetched job.
If the slot is claimed, the \Condor{startd} will evaluate the
\Attr{Rank} expression relative to the fetched job, compare it to
the value of \Attr{Rank} for the currently running job, and decide
if the existing job should be preempted due to the fetched job having
a higher rank.
If the slot is unavailable for whatever reason, the \Condor{startd}
will refuse the fetched job and ignore it.
Either way, once the \Condor{startd} decides what it should do with
the fetched job, it will invoke another hook to reply to the attempt
to fetch work, so that the external system knows what happened to that
work unit.

If the job is accepted, a claim is created for it and the slot moves
into the Claimed state.
As soon as this happens, the \Condor{startd} will spawn a
\Condor{starter} to manage the execution of the job.
At this point, from the perspective of the \Condor{startd}, this claim
is just like any other.
The usual policy expressions are evaluated, and if the job needs to be
suspended or evicted, it will be.
If a higher-ranked job being managed by a \Condor{schedd} is matched
with the slot, that job will preempt the fetched work.

The \Condor{starter} itself can optionally invoke additional hooks to
help manage the execution of the specific job.
There are hooks to prepare the execution environment for the job,
periodically update information about the job as it runs, notify when
the job exits, and to take special actions when the job is being evicted.

Assuming there are no interruptions, the job completes, and the
\Condor{starter} exits, the \Condor{startd} will invoke the hook to
fetch work again.
If another job is available, the existing claim will be reused and a
new \Condor{starter} is spawned.
If the hook returns that there is no more work to perform, the claim
will be evicted, and the slot will return to the Owner state.


%%%%%%%%%%%%%%%%%%%%%%%%%%%%%%%%%%%%%%%%%%%%%%%%%%%%%%%%%%%%%%%%%%%%%%
\subsubsection{\label{sec:job-hooks-hooks}
Work Fetching Hooks Invoked by Condor}
%%%%%%%%%%%%%%%%%%%%%%%%%%%%%%%%%%%%%%%%%%%%%%%%%%%%%%%%%%%%%%%%%%%%%%
\index{Job hooks!Hooks invoked by Condor} 

There are a handful of hooks invoked by Condor related to fetching
work, some of which are called by the \Condor{startd} and others by
the \Condor{starter}.
Each hook is described, including when it is invoked, what
task it is supposed to accomplish, what data is passed to the hook,
what output is expected, and, when relevant, the exit status expected.


\begin{itemize}
\index{Job hooks!Fetch Hooks!Fetch work}
\item[Hook: Fetch Work]

The hook defined by the configuration variable
\Macro{<Keyword>\_HOOK\_FETCH\_WORK} is invoked whenever the \Condor{startd}
wants to see if there is any work to fetch.
There is a related configuration variable called
\Macro{FetchWorkDelay} which determines how long the \Condor{startd}
will wait between attempts to fetch work, which is described in detail
in within section~\ref{sec:job-hooks-fetch-work-delay} on
page~\pageref{sec:job-hooks-fetch-work-delay}.
\MacroNI{<Keyword>\_HOOK\_FETCH\_WORK} is the most important hook in the whole
system, and is the only hook that must be defined for any of the other
\Condor{startd} hooks to operate.

The job ClassAd returned by the hook needs to contain enough
information for the \Condor{starter} to eventually spawn the work.
The required and optional attributes in this ClassAd are identical to
the ones described for Computing on Demand (COD) jobs in
section~\ref{sec:cod-application-attributes} 
on COD Application Attributes, 
page~\pageref{sec:cod-application-attributes}.

\begin{description}
\item[Command-line arguments passed to the hook]
  None.

\item[Standard input given to the hook]
  ClassAd of the slot that is looking for work.

\item[Expected standard output from the hook]
  ClassAd of a job that can be run.
  If there is no work, the hook should return no output.

\item[Exit status of the hook]
  Ignored.
\end{description}


\index{Job hooks!Fetch Hooks!Reply to fetched work}
\item[Hook: Reply Fetch]

The hook defined by the configuration variable
\Macro{<Keyword>\_HOOK\_REPLY\_FETCH} is invoked whenever
\Macro{<Keyword>\_HOOK\_FETCH\_WORK} returns data and the \Condor{startd}
decides if it is going to accept the fetched job or not.

The \Condor{startd} will not wait for this hook to return before
taking other actions, and it ignores all output.
The hook is simply advisory, and it has no impact on the behavior of the
\Condor{startd}.

\begin{description}
\item[Command-line arguments passed to the hook]
  Either the string \verb@accept@ or \verb@reject@.

\item[Standard input given to the hook]
  A copy of the job ClassAd and the slot ClassAd
  (separated by the string \verb@-----@ and a new line).

\item[Expected standard output from the hook]
  None.

\item[Exit status of the hook]
  Ignored.
\end{description}


\index{Job hooks!Fetch Hooks!Evict a claim}
\item[Hook: Evict Claim]

The hook defined by the configuration variable
\Macro{<Keyword>\_HOOK\_EVICT\_CLAIM} is invoked whenever the \Condor{startd}
needs to evict a claim representing fetched work.

The \Condor{startd} will not wait for this hook to return before
taking other actions, and ignores all output.
The hook is simply advisory, and has no impact on the behavior of the
\Condor{startd}.

\begin{description}
\item[Command-line arguments passed to the hook]
  None.

\item[Standard input given to the hook]
  A copy of the job ClassAd and the slot ClassAd
  (separated by the string \verb@-----@ and a new line).

\item[Expected standard output from the hook]
  None.

\item[Exit status of the hook]
  Ignored.
\end{description}


\index{Job hooks!Fetch Hooks!Prepare job}
\item[Hook: Prepare Job]

The hook defined by the configuration variable
\Macro{<Keyword>\_HOOK\_PREPARE\_JOB} is invoked by the \Condor{starter} before
a job is going to be run.
This hook provides a chance to execute commands to set up the job
environment, for example, to transfer input files.

The \Condor{starter} waits until this hook returns before
attempting to execute the job.
If the hook returns a non-zero exit status, the \Condor{starter} will
assume an error was reached while attempting to set up the job
environment and abort the job.

\begin{description}
\item[Command-line arguments passed to the hook]
  None.

\item[Standard input given to the hook]
  A copy of the job ClassAd.

\item[Expected standard output from the hook]
  A set of attributes to insert or update into the job ad.  For example,
  changing the \Attr{Cmd} attribute to a quoted string changes the executable 
  to be run.

\item[Exit status of the hook]
  0 for success preparing the job, any non-zero value on failure.
\end{description}


\index{Job hooks!Fetch Hooks! Update job info}
\item[Hook:  Update Job Info]

The hook defined by the configuration variable
\Macro{<Keyword>\_HOOK\_UPDATE\_JOB\_INFO} is invoked periodically during the
life of the job to update information about the status of the job.
When the job is first spawned, the \Condor{starter} will invoke this
hook after \Macro{STARTER\_INITIAL\_UPDATE\_INTERVAL} seconds
(defaults to 8).
Thereafter, the \Condor{starter} will invoke the hook every 
\Macro{STARTER\_UPDATE\_INTERVAL} seconds (defaults to 300,
which is 5 minutes).

The \Condor{starter} will not wait for this hook to return before
taking other actions, and ignores all output.
The hook is simply advisory, and has no impact on the behavior of the
\Condor{starter}.


\begin{description}
\item[Command-line arguments passed to the hook]
  None.

\item[Standard input given to the hook]
  A copy of the job ClassAd that has been augmented with additional
  attributes describing the current status and execution behavior of
  the job.

The additional attributes included inside the job ClassAd are:
\begin{description}
\item[\AdAttr{JobState}]
  The current state of the job.
  Can be either \AdStr{Running} or \AdStr{Suspended}.

\item[\AdAttr{JobPid}]
  The process identifier for the initial job directly spawned by the
  \Condor{starter}.

\item[\AdAttr{NumPids}]
  The number of processes that the job has currently spawned.

\item[\AdAttr{JobStartDate}]
  The epoch time when the job was first spawned by the \Condor{starter}.

\item[\AdAttr{RemoteSysCpu}]
  The total number of seconds of system CPU time (the time spent at
  system calls) the job has used.

\item[\AdAttr{RemoteUserCpu}]
  The total number of seconds of user CPU time the job has used.

\item[\AdAttr{ImageSize}]
  The memory image size of the job in Kbytes.
\end{description}

\item[Expected standard output from the hook]
  None.

\item[Exit status of the hook]
  Ignored.
\end{description}


\index{Job hooks!Fetch Hooks! Job exit}
\item[Hook:  Job Exit]

The hook defined by the configuration variable
\Macro{<Keyword>\_HOOK\_JOB\_EXIT} is invoked by the \Condor{starter}
whenever a job exits, either on its
own or when being evicted from an execution slot.

The \Condor{starter} will wait for this hook to return before
taking any other actions.
In the case of jobs that are being managed by a \Condor{shadow}, this
hook is invoked before the \Condor{starter} does its own optional file
transfer back to the submission machine, writes to the local user log
file, or notifies the \Condor{shadow} that the job has exited.

\begin{description}
\item[Command-line arguments passed to the hook]
  A string describing how the job exited:
  \begin{itemize}
    \item \verb@exit@ The job exited or died with a signal on its own.
    \item \verb@remove@ The job was removed with \Condor{rm} or as the result of
    user job policy expressions (for example, \Attr{PeriodicRemove}).
    \item \verb@hold@ The job was held with \Condor{hold} or the
    user job policy expressions (for example, \Attr{PeriodicHold}).
    \item \verb@evict@ The job was evicted from the execution slot for
    any other reason (\Macro{PREEMPT} evaluated to TRUE in the
    \Condor{startd}, \Condor{vacate}, \Condor{off}, etc).
  \end{itemize}

\item[Standard input given to the hook]
  A copy of the job ClassAd that has been augmented with additional
  attributes describing the execution behavior of the job and its
  final results.

The job ClassAd passed to this hook contains all of the extra
attributes described above for \Macro{<Keyword>\_HOOK\_UPDATE\_JOB\_INFO}, and
the following additional attributes that are only present once a job
exits:
\begin{description}
\item[\AdAttr{ExitReason}]
  A human-readable string describing why the job exited.

\item[\AdAttr{ExitBySignal}]
  A boolean indicating if the job exited due to being killed by a
  signal, or if it exited with an exit status.

\item[\AdAttr{ExitSignal}]
  If \AdAttr{ExitBySignal} is true, the signal number that killed the job.

\item[\AdAttr{ExitCode}]
  If \AdAttr{ExitBySignal} is false, the integer exit code of the job.

\item[\AdAttr{JobDuration}]
  The number of seconds that the job ran during this invocation.
\end{description}

\item[Expected standard output from the hook]
  None.

\item[Exit status of the hook]
  Ignored.
\end{description}

% \Todo
% If the job was running in the Java universe and died with a Java
% exception, the following attributes can be present:
% ATTR_EXCEPTION_HIERARCHY
% ATTR_EXCEPTION_NAME
% ATTR_EXCEPTION_TYPE

% \Todo
% Is this relevant?
% If the job is running in the Virtual Machine (VM) universe and 
% requested checkpointing by setting \AdAttr{JobVMCheckpoint} to true in
% its ClassAd, the following additional attributes might be present:
% \begin{description}
% \item[\AdAttr{VM\_CkptMac}]
%   The MAC address of the virtual machine.
% 
% \item[\AdAttr{VM\_CkptIP}]
%   The IP address of the virtual machine.
% \end{description}

\end{itemize}

%%%%%%%%%%%%%%%%%%%%%%%%%%%%%%%%%%%%%%%%%%%%%%%%%%%%%%%%%%%%%%%%%%%%%%
\subsubsection{\label{sec:job-hooks-keywords}
Keywords to Define Job Fetch Hooks in the Condor Configuration files }
%%%%%%%%%%%%%%%%%%%%%%%%%%%%%%%%%%%%%%%%%%%%%%%%%%%%%%%%%%%%%%%%%%%%%%
\index{Job hooks!keywords} 

Hooks are defined in the Condor configuration files by prefixing
the name of the hook with a keyword.
This way, a given machine can have multiple sets of hooks, each set
identified by a specific keyword.

Each slot on the machine can define a separate keyword for the set
of hooks that should be used with \Macro{SLOT<N>\_JOB\_HOOK\_KEYWORD}.
For example, on slot 1, the variable name will be
called \MacroNI{SLOT1\_JOB\_HOOK\_KEYWORD}.
If the slot-specific keyword is not defined, the \Condor{startd} will
use a global keyword as defined by \Macro{STARTD\_JOB\_HOOK\_KEYWORD}.

Once a job is fetched via \Macro{<Keyword>\_HOOK\_FETCH\_WORK}, the
\Condor{startd} will insert the keyword used to fetch that job into
the job ClassAd as \AdAttr{HookKeyword}.
This way, the same keyword will be used to select the hooks invoked by
the \Condor{starter} during the actual execution of the job.
However, the \Macro{STARTER\_JOB\_HOOK\_KEYWORD} can be defined to
force the \Condor{starter} to always use a given keyword for its own
hooks, instead of looking the job ClassAd for a \AdAttr{HookKeyword}
attribute.

For example, the following configuration defines two sets of hooks,
and on a machine with 4 slots, 3 of the slots use the global keyword
for running work from a database-driven system, and one of the slots
uses a custom keyword to handle work fetched from a web service.
\footnotesize
\begin{verbatim}
  # Most slots fetch and run work from the database system.
  STARTD_JOB_HOOK_KEYWORD = DATABASE

  # Slot4 fetches and runs work from a web service.
  SLOT4_JOB_HOOK_KEYWORD = WEB

  # The database system needs to both provide work and know the reply
  # for each attempted claim.
  DATABASE_HOOK_DIR = /usr/local/condor/fetch/database
  DATABASE_HOOK_FETCH_WORK = $(DATABASE_HOOK_DIR)/fetch_work.php
  DATABASE_HOOK_REPLY_FETCH = $(DATABASE_HOOK_DIR)/reply_fetch.php

  # The web system only needs to fetch work.
  WEB_HOOK_DIR = /usr/local/condor/fetch/web
  WEB_HOOK_FETCH_WORK = $(WEB_HOOK_DIR)/fetch_work.php
\end{verbatim}
\normalsize

The keywords \AdStr{DATABASE} and \AdStr{WEB} are completely arbitrary, so
each site is encouraged to use different (more specific) names as
appropriate for their own needs.


%%%%%%%%%%%%%%%%%%%%%%%%%%%%%%%%%%%%%%%%%%%%%%%%%%%%%%%%%%%%%%%%%%%%%%
\subsubsection{\label{sec:job-hooks-fetch-work-delay}
Defining the FetchWorkDelay Expression}
%%%%%%%%%%%%%%%%%%%%%%%%%%%%%%%%%%%%%%%%%%%%%%%%%%%%%%%%%%%%%%%%%%%%%%
\index{Job hooks!FetchWorkDelay}

There are two events that trigger the \Condor{startd} to attempt to
fetch new work:
\begin{itemize}
\item the \Condor{startd} evaluates its own state
\item the \Condor{starter} exits after completing some fetched work
\end{itemize}

Even if a given compute slot is already busy running other work, it is
possible that if it fetched new work, the \Condor{startd} would prefer
this newly fetched work (via the \AdAttr{Rank} expression) over the work it
is currently running.
However, the \Condor{startd} frequently evaluates its own state,
especially when a slot is claimed.
Therefore, administrators can define a configuration variable which controls
how long the \Condor{startd} will wait between attempts to fetch new work.
This variable is called \Macro{FetchWorkDelay}.

The \MacroNI{FetchWorkDelay} expression must evaluate to an integer,
which defines the number of seconds since the last fetch attempt
completed before the \Condor{startd} will attempt to fetch more work.
However, as a ClassAd expression (evaluated in the context of the
ClassAd of the slot considering if it should fetch more work, and the
ClassAd of the currently running job, if any), the length of the delay
can be based on the current state the slot and even the currently
running job.

For example, a common configuration would be to always wait 5
minutes (300 seconds) between attempts to fetch work, unless the slot
is Claimed/Idle, in which case the \Condor{startd} should fetch
immediately:

\footnotesize
\begin{verbatim}
FetchWorkDelay = ifThenElse(State == "Claimed" && Activity == "Idle", 0, 300) 
\end{verbatim}
\normalsize

If the \Condor{startd} wants to fetch work, but the time since the
last attempted fetch is shorter than the current value of the delay
expression, the \Condor{startd} will set a timer to fetch as soon as
the delay expires.

If this expression is not defined, the \Condor{startd} will default to
a five minute (300 second) delay between all attempts to fetch work.

%%%%%%%%%%%%%%%%%%%%%%%%%%%%%%%%%%%%%%%%%%%%%%%%%%%%%%%%%%%%%%%%%%%%%%
\subsubsection{\label{sec:job-hooks-example}
Example Hook: Specifying the Executable at Execution Time}
%%%%%%%%%%%%%%%%%%%%%%%%%%%%%%%%%%%%%%%%%%%%%%%%%%%%%%%%%%%%%%%%%%%%%%
\index{Job hooks!Java example}

The availability of multiple versions of an application leads to
the need to specify one of the versions. 
As an example, consider that 
the java universe utilizes a single, fixed JVM.
There may be multiple JVMs available, and the Condor job may
need to make the choice of JVM version.
The use of a job hook solves this problem.
The job does not use the java universe, and instead uses the
vanilla universe in combination with a 
prepare job hook to overwrite the \Attr{Cmd} attribute of the job ClassAd.
This attribute is the name of the
executable the \Condor{starter} daemon will invoke,
thereby selecting the specific JVM installation.

In the configuration of the execute machine:

\footnotesize
\begin{verbatim}
JAVA5_HOOK_PREPARE_JOB = $(LIBEXEC)/java5_prepare_hook
\end{verbatim}
\normalsize

With this configuration, a job that sets the \Attr{HookKeyword} attribute with

\begin{verbatim}
+HookKeyword = "JAVA5"
\end{verbatim}

in the submit description file causes the \Condor{starter}
will run the hook specified by \Macro{JAVA5\_HOOK\_PREPARE\_JOB}
before running this job.
Note that the double quote marks are required to correctly define
the attribute.
Any output from this hook is an update to the job ClassAd.  
Therefore, the hook that changes the executable may be

\begin{verbatim}
#!/bin/sh

# Read and discard the job ClassAd
cat > /dev/null
echo 'Cmd = "/usr/java/java5/bin/java"'
\end{verbatim}

The submit description file for this example job may be
\footnotesize
\begin{verbatim}
universe = vanilla
executable = /usr/bin/java
arguments = Hello
# match with a machine that advertises the JAVA5 hook
requirements = (JAVA5_HOOK_PREPARE_JOB =!= UNDEFINED)

should_transfer_files = always
when_to_transfer_output = on_exit
transfer_input_files = Hello.class

output = hello.out
error  = hello.err
log    = hello.log

+HookKeyword="JAVA5"
queue

\end{verbatim}
\normalsize
Note that the \SubmitCmd{requirements} command ensures that this job
matches with a machine that has \MacroNI{JAVA5\_HOOK\_PREPARE\_JOB} defined.

