Checkpointing is taking a snapshot of the current state of a program
in such a way that the program can be restarted from that state at a
later time.  Checkpointing gives the Condor scheduler the freedom to
reconsider scheduling decisions through preemptive-resume scheduling.
If the scheduler decides to no longer allocate a machine to a job (for
example, when the owner of that machine returns), it can checkpoint
the job and preempt it without losing the work the job has already
accomplished.  The job can be resumed later when the scheduler
allocates it a new machine.  Additionally, periodic checkpointing
provides fault tolerance in Condor.  Snapshots are taken periodically,
and after an interruption in service the program can continue from the
most recent snapshot.

Condor provides checkpointing services to single process jobs on a
number of Unix platforms.
To enable checkpointing, the user must link the program with the
Condor system call library (\File{libcondorsyscall.a}), using the
\Condor{compile} command.
This means that the
user must have the object files or source code of the program to use
Condor checkpointing.  However, the checkpointing services provided by
Condor are strictly optional.  So, while there are some classes of
jobs for which Condor does not provide checkpointing services, these
jobs may still be submitted to Condor to take advantage of Condor's
resource management functionality.  (See
section~\ref{sec:standard-universe} on
page~\pageref{sec:standard-universe} for a description of the
classes of jobs for which Condor does not provide checkpointing
services.)

Process checkpointing is implemented in the Condor system call library
as a signal handler.  When Condor sends a checkpoint signal to a
process linked with this library, the provided signal handler writes
the state of the process out to a file or a network socket.  This
state includes the contents of the process stack and data segments,
all shared library code and data mapped into the process's address
space, the state of all open files, and any signal handlers and
pending signals.  On restart, the process reads this state from the
file, restoring the stack, shared library and data segments, file
state, signal handlers, and pending signals.  The checkpoint signal
handler then returns to user code, which continues from where it left
off when the checkpoint signal arrived.

Condor processes for which checkpointing is enabled perform a
checkpoint when preempted from a machine.  When a suitable replacement
execution machine is found (of the same architecture and operating
system), the process is restored on this new machine from the
checkpoint, and computation is resumed from where it left off.  Jobs
that can not be checkpointed are preempted and restarted from the
beginning.

Condor's periodic checkpointing provides fault tolerance.  Condor
pools are each configured with the \Expr{PERIODIC\_CHECKPOINT}
expression which controls when and how often jobs which can be
checkpointed do periodic checkpoints (examples: never, every three
hours, etc.).  When the time for a periodic checkpoint occurs, the job
suspends processing, performs the checkpoint, and immediately
continues from where it left off.  There is also a \Condor{ckpt} command
which allows the user to request that a Condor job immediately perform
a periodic checkpoint.

In all cases, Condor jobs continue execution from the most recent
complete checkpoint.  If service is interrupted while a checkpoint is
being performed, causing that checkpoint to fail, the process will
restart from the previous checkpoint.  Condor uses a commit style
algorithm for writing checkpoints: a previous checkpoint is deleted
only after a new complete checkpoint has been written successfully.

The Condor distributions include a standalone checkpointing library,
\File{libcondorckpt.a}, which provides checkpointing for Unix
processes without Condor's remote system call functionality.
Standalone checkpointing is described in
section~\ref{sec:standalone-ckpt}.

Condor can now read and write compressed checkpoints.  This new
functionality is provided in the \File{libcondorzsyscall.a} and
\File{libcondorzckpt.a} libraries.  
If \File{/usr/lib/libz.a} exists on your workstation, \Condor{compile}
will automatically link your job with the compression-enabled version
of the checkpointing library.
Currently, compression is used only for periodic checkpoints, while we
experiment with this new functionality.

By default, a checkpoint is written to a file on the local disk of the
machine where the job was submitted.  A checkpoint server is available
to serve as a repository for checkpoints.  (See
section~\ref{sec:Ckpt-Server} on page~\pageref{sec:Ckpt-Server}.)
When a host is configured to use a checkpoint server, jobs submitted
on that machine write and read checkpoints to and from the server
rather than the local disk of the submitting machine, taking the
burden of storing checkpoint files off of the submitting machines and
placing it instead on server machines (with disk space dedicated to
the purpose of storing checkpoints).

\subsection{\label{sec:standalone-ckpt}Standalone Checkpointing}

Using the Condor checkpoint library without the remote system call
functionality and outside of the Condor system is known as standalone
mode checkpointing.

To link in standalone mode, use the \Condor{compile} utility with the
``-condor\_standalone'' option.
Once your program is re-linked with the Condor standalone
checkpointing library \File{libcondorckpt.a}, your program will
require two new command line arguments: ``\_condor\_ckpt filename''
and ``\_condor\_restart filename''.

If the command line looks like:

\begin{verbatim}
	exec_name -_condor_ckpt ckpt_filename ..
\end{verbatim}

then we set up to checkpoint to the given file name.

If the command line looks like:

\begin{verbatim}
	exec_name -_condor_restart ckpt_filename ...
\end{verbatim}

then we effect a restart from the given file name.

Any Condor command line options are removed from the head of the
command line before main() is called.

If we aren't given instructions on the command line, by default we
assume we are an original invocation, and that we should write any
checkpoints to the name by which we were invoked with a
``ckpt'' extension.

To cause a program to checkpoint and exit, send it a SIGTSTP signal.  For 
example, in C you would add the following line to your code:

\begin{verbatim}
	kill( getpid(), SIGTSTP );
\end{verbatim}

Note that most Unix shells are configured to send a TSTP signal to the
foreground process when the user enters a Ctrl-Z.  To cause a program
to write a periodic checkpoint (i.e., checkpoint and continue
running), sent it a SIGUSR2:

\begin{verbatim}
	kill( getpid(), SIGUSR2 );
\end{verbatim}

In addition to the command-line parameters interface described above,
a C interface is also provided for restarting a program from a
checkpoint file.  The prototypes are:

\begin{verbatim}
	void init_image_with_file_name( char *ckpt_name );
	void init_image_with_file_descriptor( int fd );
	void restart( );
\end{verbatim}

The init\_image\_with\_file\_name() and
init\_image\_with\_file\_descriptor() functions are used to specify
the location of the checkpoint file.  Only one of the two must be
used.  The restart() function causes the process image from the
specified file to be read and restored.
