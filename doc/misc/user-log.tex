%%%%%%%%%%%%%%%%%%%%%%%%%%%%%%%%%%%%%%%%%%%%%%%%%%%
\subsection{\label{sec:job-log-reader}The HTCondor User and Job Log Reader API}
%%%%%%%%%%%%%%%%%%%%%%%%%%%%%%%%%%%%%%%%%%%%%%%%%%%
\index{API!ReadUserLog class}
\index{User Log Reader API}
\index{Event Log Reader API}
\index{ReadUserLog class@\texttt{ReadUserLog} class}
\index{Job Log Reader API}

HTCondor has the ability to log an HTCondor job's significant events during 
its lifetime.
This is enabled in the job's submit description file with the
\SubmitCmd{Log} command.

This section describes the API defined by the C++ \Code{ReadUserLog} class,
which provides a programming interface for applications
to read and parse events,
polling for events, and saving and restoring reader state.

%%%%%%%%%%%%%%%%%%%%%%%%%%%%%%%%%%%%%%%%%%%%%%%%%%%
\subsubsection{Constants and Enumerated Types}

The following define enumerated types useful to the API.

\begin{itemize}
\item \Type{ULogEventOutcome} (defined in \File{condor\_event.h}):
  \begin{itemize}
    \item \Const{ULOG\_OK}: Event is valid
    \item \Const{ULOG\_NO\_EVENT}: No event occurred (like EOF)
    \item \Const{ULOG\_RD\_ERROR}: Error reading log file
    \item \Const{ULOG\_MISSED\_EVENT}: Missed event
    \item \Const{ULOG\_UNK\_ERROR}: Unknown Error
  \end{itemize}

\item \Type{ReadUserLog::FileStatus}
  \begin{itemize}
    \item \Const{LOG\_STATUS\_ERROR}: An error was encountered
    \item \Const{LOG\_STATUS\_NOCHANGE}: No change in file size
    \item \Const{LOG\_STATUS\_GROWN}: File has grown
    \item \Const{LOG\_STATUS\_SHRUNK}: File has shrunk
  \end{itemize}

\end{itemize}


%%%%%%%%%%%%%%%%%%%%%%%%%%%%%%%%%%%%%%%%%%%%%%%%%%%
\subsubsection{Constructors and Destructors}

All \Code{ReadUserLog} constructors invoke one of the \Code{initialize()}
methods.
Since C++ constructors cannot return errors, 
an application using any but the default constructor should call
\Code{isIinitialized()} to verify that the object initialized correctly,
and for example, had permissions to open required files.

Note that because the constructors cannot return status information,
most of these constructors will be eliminated in the future.
All constructors, except for the default constructor with no parameters,
will be removed.
The application will need to call the appropriate \Code{initialize()} method.

\begin{itemize}

\item \Constructor
  {ReadUserLog}
  {bool isEventLog}
  {default}
  \begin{itemize}
  \item \ParamDefOpt{isEventLog}{bool}{\Const{false}}
    If \Const{true}, the \Code{ReadUserLog} object is
    initialized to read the schedd-wide event log.
    \\ \Note If \Param{isEventLog} is \Const{true}, the initialization may
    silently fail, so the value of \MethodRef{ReadUserLog}{isInitialized}
    should be checked to verify that the initialization was successful.
    \\ \Note The \Param{isEventLog} parameter will be removed in the future.
  \end{itemize}

\item \Constructor
  {ReadUserLog}
  {FILE *fp, bool is\_xml, bool enable\_close}
  {of a limited functionality reader: no rotation handling, no locking}
  \begin{itemize}
  \item \ParamDef{fp}{FILE *}
    File pointer to the previously opened log file to read.
  \item \ParamDef{is\_xml}{bool}
    If \Const{true}, the file is treated as XML; otherwise, it will be
    read as an old style file.
  \item \ParamDefOpt{enable\_close}{bool}{\Const{false}}
    If \Const{true}, the reader will open the file read-only.
  \end{itemize}
  \Note The \MethodRef{ReadUserLog}{isInitialized} method
  should be invoked to verify that this constructor was initialized
  successfully.
  \\ \Note This constructor will be removed in the future.

\item \Constructor
  {ReadUserLog}
  {const char *filename, bool read\_only}
  {to read a specific log file}
  \begin{itemize}
  \item \ParamDef{filename}{const char *}
    Path to the log file to read
  \item \ParamDefOpt{read\_only}{bool}{\Const{false}}
    If \Const{true}, the reader will open the file read-only and
    disable locking.
  \end{itemize}
  \Note This constructor will be removed in the future.

\item \Constructor
  {ReadUserLog}
  {const FileState \&state, bool read\_only}
  {to continue from a persisted reader state}
  \begin{itemize}
  \item \ParamDef{state}{const FileState \&}
    Reference to the persisted state to restore from
  \item \ParamDefOpt{read\_only}{bool}{\Const{false}}
    If \Const{true}, the reader will open the file read-only and
    disable locking.
  \end{itemize}
  \Note The \MethodRef{ReadUserLog}{isInitialized}
  method should be invoked to verify that this constructor was
  initialized successfully.
  \\ \Note This constructor will be removed in the future.

\item \Destructor{ReadUserLog}

\end{itemize}


%%%%%%%%%%%%%%%%%%%%%%%%%%%%%%%%%%%%%%%%%%%%%%%%%%%
\subsubsection{Initializers}

These methods are used to perform the initialization of the
\Code{ReadUserLog} objects.  These initializers are used by all constructors
that do real work.  
Applications should never use those constructors,
should use the default constructor,
 and should instead use one of these initializer methods.

All of these functions will return \Const{false} if there are problems
such as being unable to open the log file,
or \Const{true} if successful.

\begin{itemize}

\item \MethodDef{ReadUserLog}{initialize}
  {\Type{bool}} {\Const{true}: success, \Const{false}: failed}
  {void}
  {Initialize to read the EventLog file.
  \\ \Note This method will likely be eliminated in the future, and this
  functionality will be moved to a new \Code{ReadEventLog} class.}
  \begin{itemize}\item None. \end{itemize}

\item \MethodDef{ReadUserLog}{initialize}
  {\Type{bool}} {\Const{true}: success, \Const{false}: failed}
  {const char *filename, bool handle\_rotation,
    bool check\_for\_rotated, bool read\_only}
  {Initialize to read a specific log file.}
  \begin{itemize}
  \item \ParamDef{filename}{const char *}
    Path to the log file to read
  \item \ParamDefOpt{handle\_rotation} {bool} {\Const{false}}
    If \Const{true}, enable the reader to handle rotating log files,
    which is only useful for global user logs
  \item \ParamDefOpt{check\_for\_rotated} {bool} {\Const{false}}
    If \Const{true}, try to open the rotated files
    (with file names appended with \File{.old} or \File{.1}, \File{.2}, \Dots)
    first.
  \item \ParamDefOpt{read\_only} {bool} {\Const{false}}
    If \Const{true}, the reader will open the file read-only and
    disable locking.
  \end{itemize}

\item \MethodDef{ReadUserLog}{initialize}
  {\Type{bool}} {\Const{true}: success, \Const{false}: failed}
  {const char *filename, int max\_rotation,
    bool check\_for\_rotated, bool read\_only}
  {Initialize to read a specific log file.}
  \begin{itemize}
  \item \ParamDef{filename}{const char *}
    Path to the log file to read
  \item \ParamDef{max\_rotation} {int}
    Limits what previously rotated files will be considered by the number
    given in the file name suffix.
    A value of 0 disables looking for rotated files.
    A value of 1 limits the rotated file to be that with the file name suffix
    of \File{.old}.
    As only event logs are rotated, this parameter is only useful for
    event logs.
  \item \ParamDefOpt{check\_for\_rotated} {bool} {\Const{false}}
    If \Const{true}, try to open the rotated files
    (with file names appended with \File{.old} or \File{.1}, \File{.2}, \Dots)
    first.
  \item \ParamDefOpt{read\_only} {bool} {\Const{false}}
    If \Const{true}, the reader will open the file read-only and
    disable locking.
  \end{itemize}

\item \MethodDef{ReadUserLog}{initialize}
  {\Type{bool}} {\Const{true}: success, \Const{false}: failed}
  {const FileState \&state, bool read\_only}
  {Initialize to continue from a persisted reader state.}
  \begin{itemize}
  \item \ParamDef{state}{const FileState \&}
    Reference to the persisted state to restore from
  \item \ParamDefOpt{read\_only}{bool}{\Const{false}}
    If \Const{true}, the reader will open the file read-only and
    disable locking.
  \end{itemize}

\item \MethodDef{ReadUserLog}{initialize}
  {\Type{bool}} {\Const{true}: success, \Const{false}: failed}
  {const FileState \&state, int max\_rotation, bool read\_only}
  {Initialize to continue from a persisted reader state and set the
  rotation parameters.}
  \begin{itemize}
  \item \ParamDef{state}{const FileState \&}
    Reference to the persisted state to restore from
  \item \ParamDef{max\_rotation} {int}
    Limits what previously rotated files will be considered by the number
    given in the file name suffix.
    A value of 0 disables looking for rotated files.
    A value of 1 limits the rotated file to be that with the file name suffix
    of \File{.old}.
    As only event logs are rotated, this parameter is only useful for
    event logs.
  \item \ParamDefOpt{read\_only}{bool}{\Const{false}}
    If \Const{true}, the reader will open the file read-only and
    disable locking.
  \end{itemize}

\end{itemize}

%%%%%%%%%%%%%%%%%%%%%%%%%%%%%%%%%%%%%%%%%%%%%%%%%%%
\subsubsection{Primary Methods}
\begin{itemize}

\item \MethodDef{ReadUserLog}{readEvent}
  {\Type{ULogEventOutcome}}
  {Outcome of the log read attempt. \Type{ULogEventOutcome} is an enumerated
  type.}
  {ULogEvent $*$\& event}
  {Read the next event from the log file.}
  \begin{itemize}
  \item \ParamDef{event}{\Type{ULogEvent} $*$\&}
    Pointer to an \Type{ULogEvent} that is allocated by this call to
    \MethodRef{ReadUserLog}{readEvent}.
    If no event is allocated, this pointer is
    set to \Const{NULL}.  Otherwise the event needs to be {\tt
    delete()ed} by the application.
  \end{itemize}

\item \MethodDef{ReadUserLog}{synchronize}
  {\Type{bool}} {\Const{true}: success, \Const{false}: failed}
  {void}
  {Synchronize the log file if the last event read was an error.  This
    safe guard function should be called if there is some error reading an
    event, but there are events after it in the file.
    It will skip over the
    bad event, meaning it will read up to and including the event separator,
    so that the rest of the events can be read.}
  \begin{itemize}\item None. \end{itemize}

\end{itemize}

%%%%%%%%%%%%%%%%%%%%%%%%%%%%%%%%%%%%%%%%%%%%%%%%%%%
\subsubsection{Accessors}
\begin{itemize}

\item \MethodDef{ReadUserLog}{CheckFileStatus}
  {\Type{ReadUserLog::FileStatus}}{the status of the log file, an
  enumerated type.}
  {void}
  {Check the status of the file, and whether it has grown, shrunk, etc.}
  \begin{itemize}\item None. \end{itemize}

\item \MethodDef{ReadUserLog}{CheckFileStatus}
  {\Type{ReadUserLog::FileStatus}} {the status of the log file, an
  enumerated type.}
  {bool \&is\_empty}
  {Check the status of the file, and whether it has grown, shrunk, etc.}
  \begin{itemize}
  \item \ParamDef{is\_empty}{bool \&}
    Set to \Const{true} if the file is empty, \Const{false} otherwise.
  \end{itemize}

\end{itemize}


%%%%%%%%%%%%%%%%%%%%%%%%%%%%%%%%%%%%%%%%%%%%%%%%%%%
\subsubsection{Methods for saving and restoring persistent reader state}

The \Type{ReadUserLog::FileState} structure is used to save and
restore the state of the \Type{ReadUserLog} state for persistence.  The
application should always use \Procedure{InitFileState} to initialize this
structure.

All of these methods take a reference to a state buffer
as their only parameter.

All of these methods return \Const{true} upon success.

%%%%%%%%%%%%%%%%%%%%%%%%%%%%%%%%%%%%%%%%%%%%%%%%%%%
\subsubsection{Save state to persistent storage}
To save the state, do something like this:
\footnotesize
\begin{verbatim}
  ReadUserLog                reader;
  ReadUserLog::FileState     statebuf;

  status = ReadUserLog::InitFileState( statebuf );

  status = reader.GetFileState( statebuf );
  write( fd, statebuf.buf, statebuf.size );
  ...
  status = reader.GetFileState( statebuf );
  write( fd, statebuf.buf, statebuf.size );
  ...

  status = UninitFileState( statebuf );
\end{verbatim}
\normalsize

%%%%%%%%%%%%%%%%%%%%%%%%%%%%%%%%%%%%%%%%%%%%%%%%%%%
\subsubsection{Restore state from persistent storage}
To restore the state, do something like this:
\footnotesize
\begin{verbatim}
  ReadUserLog::FileState     statebuf;
  status = ReadUserLog::InitFileState( statebuf );

  read( fd, statebuf.buf, statebuf.size );

  ReadUserLog                reader;
  status = reader.initialize( statebuf );

  status = UninitFileState( statebuf );
  ....
\end{verbatim}
\normalsize

%%%%%%%%%%%%%%%%%%%%%%%%%%%%%%%%%%%%%%%%%%%%%%%%%%%
\subsubsection{API Reference}
\begin{itemize}

\item \StaticMethodDef{ReadUserLog}{InitFileState}
  {\Type{bool}} {\Const{true} if successful, \Const{false} otherwise}
  {ReadUserLog::FileState \&state}
  {Initialize a file state buffer}
  \begin{itemize}
  \item \ParamDef{state}{ReadUserLog::FileState \&}
    The file state buffer to initialize.
  \end{itemize}

\item \StaticMethodDef{ReadUserLog}{UninitFileState}
  {\Type{bool}} {\Const{true} if successful, \Const{false} otherwise}
  {ReadUserLog::FileState \&state}
  {Clean up a file state buffer and free allocated memory}
  \begin{itemize}
  \item \ParamDef{state}{ReadUserLog::FileState \&}
    The file state buffer to un-initialize.
  \end{itemize}

\item \ConstMethodDef{ReadUserLog}{GetFileState}
  {\Type{bool}} {\Const{true} if successful, \Const{false} otherwise}
  {ReadUserLog::FileState \&state}
  {Get the current state to persist it or save it off to disk}
  \begin{itemize}
  \item \ParamDef{state}{ReadUserLog::FileState \&}
    The file state buffer to read the state into.
  \end{itemize}

\item \MethodDef{ReadUserLog}{SetFileState}
  {\Type{bool}} {\Const{true} if successful, \Const{false} otherwise}
  {const ReadUserLog::FileState \&state}
  {Use this method to set the current state, after restoring it.
    \\ \Note The state buffer is \emph{NOT} automatically updated; a call 
    \emph{MUST} be made to
    the \Procedure{GetFileState} method each time before persisting the
    buffer to disk, or however else is chosen to persist its contents.}
  \begin{itemize}
  \item \ParamDef{state}{const ReadUserLog::FileState \&}
    The file state buffer to restore from.
  \end{itemize}

\end{itemize}

%%%%%%%%%%%%%%%%%%%%%%%%%%%%%%%%%%%%%%%%%%%%%%%%%%%
\subsubsection{Access to the persistent state data}

If the application needs access to the data elements in a persistent
state, it should instantiate a \Type{ReadUserLogStateAccess} object.

\begin{itemize}
\item Constructors / Destructors

\begin{itemize}
\item \Constructor
  {ReadUserLogStateAccess}
  {const ReadUserLog::FileState \&state}
  {default}
  \begin{itemize}
  \item \ParamDef{state}{const ReadUserLog::FileState \&}
    Reference to the persistent state data to initialize from.
  \end{itemize}

\item \Destructor
  {ReadUserLogStateAccess}
\end{itemize}

\item Accessor Methods
\begin{itemize}

\item \ConstMethodDef{ReadUserLogFileState}{isInitialized}
  {\Type{bool}} {\Const{true} if successfully initialized, \Const{false} otherwise}
  {void}
  {Checks if the buffer initialized}
  \begin{itemize} \item None. \end{itemize}

\item \ConstMethodDef{ReadUserLogFileState}{isValid}
  {\Type{bool}} {\Const{true} if successful, \Const{false} otherwise}
  {void}
  {Checks if the buffer is valid for use by 
  \Procedure{ReadUserLog::initialize}}
  \begin{itemize} \item None. \end{itemize}

\item \ConstMethodDef{ReadUserLogFileState}{getFileOffset}
  {\Type{bool}} {\Const{true} if successful, \Const{false} otherwise}
  {unsigned long \&pos}
  {Get position within individual file.
    \\ \Note Can return an error if the result is too large to be
    stored in a \Type{long}.}
  \begin{itemize}
  \item \ParamDef{pos}{unsigned long \&}
    Byte position within the current log file
  \end{itemize}

\item \ConstMethodDef{ReadUserLogFileState}{getFileEventNum}
  {\Type{bool}} {\Const{true} if successful, \Const{false} otherwise}
  {unsigned long \&num}
  {Get event number in individual file.
    \\ \Note Can return an error if the result is too large to be
    stored in a \Type{long}.}
  \begin{itemize}
  \item \ParamDef{num}{unsigned long \&}
    Event number of the current event in the current log file
  \end{itemize}

\item \ConstMethodDef{ReadUserLogFileState}{getLogPosition}
  {\Type{bool}} {\Const{true} if successful, \Const{false} otherwise}
  {unsigned long \&pos}
  {Position of the start of the current file in overall log.
    \\ \Note Can return an error if the result is too large
    to be stored in a \Type{long}.}
  \begin{itemize}
  \item \ParamDef{pos}{unsigned long \&}
    Byte offset of the start of the current file in the overall
    logical log stream.
  \end{itemize}

\item \ConstMethodDef{ReadUserLogFileState}{getEventNumber}
  {bool} {\Const{true} if successful, \Const{false} otherwise}
  {unsigned long \&num}
  {Get the event number of the first event in the current file
    \\ \Note Can return an error if the result is too large
    to be stored in a \Type{long}.}
  \begin{itemize}
  \item \ParamDef{num}{unsigned long \&}
    This is the absolute event number of the first event in the
    current file in the overall logical log stream.
  \end{itemize}

\item \ConstMethodDef{ReadUserLogFileState}{getUniqId}
  {bool} {\Const{true} if successful, \Const{false} otherwise}
  {char *buf, int size}
  {Get the unique ID of the associated state file.}
  \begin{itemize}
  \item \ParamDef{buf}{char $*$}
    Buffer to fill with the unique ID of the current file.
  \item \ParamDef{size}{int}
    Size in bytes of \Param{buf}.
    \\ This is to prevent \MethodRef{ReadUserLogFileState}{getUniqId}
    from writing past the end of \Param{buf}.
  \end{itemize}

\item \ConstMethodDef{ReadUserLogFileState}{getSequenceNumber}
  {\Type{bool}} {\Const{true} if successful, \Const{false} otherwise}
  {int \&seqno}
  {Get the sequence number of the associated state file.}
  \begin{itemize}
  \item \ParamDef{seqno}{int \&}
    Sequence number of the current file
  \end{itemize}
\end{itemize}

\item Comparison Methods
\begin{itemize}

\item \ConstMethodDef{ReadUserLogFileState}{getFileOffsetDiff}
  {\Type{bool}} {\Const{true} if successful, \Const{false} otherwise}
  {const ReadUserLogStateAccess \&other, unsigned long \&pos}
  {Get the position difference of two states given by \Code{this} 
  and \Code{other}.
    \\ \Note Can return an error if the result is too large to be
    stored in a \Type{long}.}
  \begin{itemize}
  \item \ParamDef{other}{const ReadUserLogStateAccess \&}
    Reference to the state to compare to.
  \item \ParamDef{diff}{long \&}
    Difference in the positions
  \end{itemize}

\item \ConstMethodDef{ReadUserLogFileState}{getFileEventNumDiff}
  {bool} {\Const{true} if successful, \Const{false} otherwise}
  {const ReadUserLogStateAccess \&other, long \&diff}
  {Get event number in individual file.
    \\ \Note Can return an error if the result is too large to be
    stored in a \Type{long}.}
  \begin{itemize}
  \item \ParamDef{other}{const ReadUserLogStateAccess \&}
    Reference to the state to compare to.
  \item \ParamDef{diff}{long \&} Event number of the current event in
    the current log file
  \end{itemize}

\item \ConstMethodDef{ReadUserLogFileState}{getLogPosition}
  {bool} {\Const{true} if successful, \Const{false} otherwise}
  {const ReadUserLogStateAccess \&other, long \&diff}
  {Get the position difference of two states given by \Code{this} 
  and \Param{other}.
    \\ \Note Can return an error if the result is too large
    to be stored in a \Type{long}.}
  \begin{itemize}
  \item \ParamDef{other}{const ReadUserLogStateAccess \&}
    Reference to the state to compare to.
  \item \ParamDef{diff}{long \&}
    Difference between the byte offset of the start of the current
    file in the overall logical log stream and that of \Param{other}.
  \end{itemize}

\item \ConstMethodDef{ReadUserLogFileState}{getEventNumber}
  {bool} {\Const{true} if successful, \Const{false} otherwise}
  {const ReadUserLogStateAccess \&other, long \&diff}
  {Get the difference between the event number of the first event in
    two state buffers (this - other).
    \\ \Note Can return an error if the result is too large
    to be stored in a \Type{long}.}
  \begin{itemize}
  \item \ParamDef{other}{const ReadUserLogStateAccess \&}
    Reference to the state to compare to.
  \item \ParamDef{diff}{long \&}
    Difference between the absolute event number of the first event in
    the current file in the overall logical log stream and that of
    \Param{other}.
  \end{itemize}

\end{itemize}	% Comparison methods

\end{itemize}

%%%%%%%%%%%%%%%%%%%%%%%%%%%%%%%%%%%%%%%%%%%%%%%%%%%
\subsubsection{Future persistence API}
The \Type{ReadUserLog::FileState} will likely be replaced with a new
C++ \Type{ReadUserLog::NewFileState}, or a similarly named class that
will self initialize.

Additionally, the functionality of \Type{ReadUserLogStateAccess} will
be integrated into this class.
