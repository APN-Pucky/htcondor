
\index{HTCondor!FAQ|(}
\index{HTCondor!Frequently Asked Questions|(}
\index{FAQ|(}
\index{Frequently Asked Questions|(}

There are many Frequently Asked Questions maintained
on the HTCondor web page, at
\URL{http://htcondor-wiki.cs.wisc.edu/index.cgi/wiki}
and on the configuration how-to and recipes page
at
\URL{https://htcondor-wiki.cs.wisc.edu/index.cgi/wiki?p=HowToAdminRecipes}

Supported platforms are listed in section~\ref{sec:Availability}, on
page~\pageref{sec:Availability}.
There is also platform-specific information at
Chapter~\ref{platforms} on page~\pageref{platforms}.

%%%%%%%%%%%%%%%%%%%%%%%%%%%%%%%%%%%%%%%%%%%%%%%%%
%\subsection*{What do I do now? My installation of HTCondor does not work.}
%%%%%%%%%%%%%%%%%%%%%%%%%%%%%%%%%%%%%%%%%%%%%%%%%
%
%\index{port usage!FAQ on communication errors}
%What to do to get HTCondor running properly depends on what sort of
%error occurs. 
%One common error category are communication errors.
%HTCondor daemon log files report a failure to bind.
%For example:
%
%\footnotesize
%\begin{verbatim}
%(date and time) Failed to bind to command ReliSock
%\end{verbatim}
%\normalsize
%
%Or, the errors in the various log files may be of the form:
%
%\footnotesize
%\begin{verbatim}
%(date and time) Error sending update to collector(s)
%(date and time) Can't send end_of_message
%(date and time) Error sending UDP update to the collector
%
%(date and time) failed to update central manager
%
%(date and time) Can't send EOM to the collector
%\end{verbatim}
%\normalsize
%
%This problem can also be observed by running \Condor{status}.
%It will give a message of the form:
%\footnotesize
%\begin{verbatim}
%Error:  Could not fetch ads --- error communication error
%\end{verbatim}
%\normalsize
%
%To solve this problem, understand that
%HTCondor uses the first network interface it sees on the machine.
%Since machines often have more than one interface,
%this problem usually implies that the wrong network
%interface is being used.  It also may be the case that
%the system simply has the wrong IP address configured.
%
%It is incorrect to use the localhost network interface.
%This has IP address 127.0.0.1 on all machines.
%To check if this incorrect IP address is being used,
%look at the contents of the
%CollectorLog file on the pool's
%your central manager right after it is started.  
%The contents will be of the form:
%
%\footnotesize
%\begin{verbatim}
%5/25 15:39:33 ******************************************************
%5/25 15:39:33 ** condor_collector (CONDOR_COLLECTOR) STARTING UP
%5/25 15:39:33 ** $CondorVersion: 6.2.0 Mar 16 2001 $
%5/25 15:39:33 ** $CondorPlatform: INTEL-LINUX-GLIBC21 $
%5/25 15:39:33 ** PID = 18658
%5/25 15:39:33 ******************************************************
%5/25 15:39:33 DaemonCore: Command Socket at <128.105.101.15:9618>
%\end{verbatim}
%\normalsize
%
%The last line tells the IP address and port the collector has
%bound to and is listening on.
%If the IP address is 127.0.0.1, then HTCondor is definitely using the wrong
%network interface.
%
%There are two solutions to this problem.
%One solution changes the order of the network interfaces.
%The preferred solution
%sets which network interface HTCondor should use
%by adding the following parameter to the
%local HTCondor configuration file:
%
%\begin{verbatim}
%NETWORK_INTERFACE = machine-ip-address
%\end{verbatim}
%
%Where \verb@machine-ip-address@ is the IP address of the interface you wish
%HTCondor to use.
%
%%%%%%%%%%%%%%%%%%%%%%%%%%%%%%%%%%%%%%%%%%%%%%%%%
%\subsection*{After an installation of HTCondor, why do the daemons refuse to start?}
%%%%%%%%%%%%%%%%%%%%%%%%%%%%%%%%%%%%%%%%%%%%%%%%%
%
%This message appears in the log files:
%\footnotesize
%\begin{verbatim}
%ERROR "The following configuration macros appear to contain default values 
%that must be changed before Condor will run.  These macros are:
%allow_write
%(found on line 1853 of /scratch/adesmet/TRUNK/work/src/localdir/condor_config)"
%at line 217 in file condor_config.C
%\end{verbatim}
%\normalsize
%
%As of HTCondor 6.8.0, if 
%HTCondor sees the bare key word: 
%\Expr{YOU\_MUST\_CHANGE\_THIS\_INVALID\_CONDOR\_CONFIGURATION\_VALUE}
%as the value of a configuration file entry,
%HTCondor daemons will log the given error message and exit.
%
%By default, an installation of HTCondor 6.8.0 and later releases
%will have the
%configuration file entry \MacroNI{ALLOW\_WRITE} set to the above sentinel
%value. 
%The HTCondor administrator must alter this value to be the correct domain
%or IP addresses that the administrator desires.
%The wild card character (\verb@*@) may be used to define this entry,
%but that allows anyone, from anywhere,
%to submit jobs into the pool.
%A better value will be of the form \Expr{*.domainname.com}.
%
%%%%%%%%%%%%%%%%%%%%%%%%%%%%%%%%%%%%%%%%%%%%%%%%%
%\section{Grid Computing}
%%%%%%%%%%%%%%%%%%%%%%%%%%%%%%%%%%%%%%%%%%%%%%%%%
%
%\index{grid computing!FAQs}
%
%%%%%%%%%%%%%%%%%%%%%%%%%%%%%%%%%%%%%%%%%%%%%%%%%
%\subsection*{What must be installed to access grid resources?}
%%%%%%%%%%%%%%%%%%%%%%%%%%%%%%%%%%%%%%%%%%%%%%%%%
%A single machine with HTCondor installed such that jobs may
%be submitted is the minimum software necessary.
%If matchmaking is desired,
%then a single machine must not only be running HTCondor
%such that jobs may be submitted,
%but also fill the role of a central manager.
%A Personal HTCondor installation may satisfy both.
%
%%%%%%%%%%%%%%%%%%%%%%%%%%%%%%%%%%%%%%%%%%%%%%%%%
%\subsection*{I am the administrator at Physics, and I have a 64-node cluster
%running HTCondor.
%The administrator at Chemistry is also running HTCondor on her 64-node cluster.
%We would like to be able to share resources.
%How do we do this?}
%%%%%%%%%%%%%%%%%%%%%%%%%%%%%%%%%%%%%%%%%%%%%%%%%
%
%HTCondor's flocking feature
%allows multiple HTCondor pools to share resources.
%By setting configuration variables within each pool,
%jobs may be executed on either cluster.
%See the manual section on flocking, section~\ref{sec:Flocking},
%for details.
%
%%%%%%%%%%%%%%%%%%%%%%%%%%%%%%%%%%%%%%%%%%%%%%%%%
%\subsection*{Using my Globus gatekeeper to submit jobs to the HTCondor pool
%does not work.  What is wrong?}
%%%%%%%%%%%%%%%%%%%%%%%%%%%%%%%%%%%%%%%%%%%%%%%%%
%\index{Globus!gatekeeper errors}
%
%The HTCondor configuration file is in a non-standard location,
%and the Globus software does not know how to locate it,
%when you see either of the following error messages.
%
%\underline{first error message}
%\footnotesize
%\begin{verbatim}
%% globus-job-run \
%  globus-gate-keeper.example.com/jobmanager-condor /bin/date
%
%Neither the environment variable CONDOR_CONFIG, /etc/condor/,
%nor ~condor/ contain a condor_config file.  Either set
%CONDOR_CONFIG to point to a valid config file, or put a
%"condor_config" file in /etc/condor or ~condor/ Exiting.
%
%GRAM Job failed because the job failed when the job manager
%attempted to run it (error code 17)
%\end{verbatim}
%\normalsize
%
%\underline{second error message}
%\footnotesize
%\begin{verbatim}
%% globus-job-run \
%   globus-gate-keeper.example.com/jobmanager-condor /bin/date
%
%ERROR: Can't find address of local schedd GRAM Job failed
%because the job failed when the job manager attempted to run it
%(error code 17)
%\end{verbatim}
%\normalsize
%
%As described in
%section~\ref{sec:Preparing-to-Install}, 
%HTCondor searches for its configuration file using the following
%ordering.
%\begin{enumerate}
%\item File specified in the \Env{CONDOR\_CONFIG} environment variable
%\item \File{\$(HOME)/.condor/condor\_config}
%\item \File{/etc/condor/condor\_config}
%\item \File{/usr/local/etc/condor\_config}
%\item \File{\Tilde condor/condor\_config}
%\end{enumerate}
%
%Presuming the configuration file is not in a standard location,
%you will need to set the \Env{CONDOR\_CONFIG} environment variable
%\index{environment variables!CONDOR\_CONFIG}
%by hand, or set it in an initialization script.
%One of the following solutions for an initialization may be used.
%\begin{enumerate}
%\item 
%Wherever \Prog{globus-gatekeeper} is launched,
%replace it with a minimal shell script that sets
%\Env{CONDOR\_CONFIG} and then starts \Prog{globus-gatekeeper}.
%Something like the following should work:
%
%\footnotesize
%\begin{verbatim}
%  #! /bin/sh
%  CONDOR_CONFIG=/path/to/condor_config
%  export CONDOR_CONFIG
%  exec /path/to/globus/sbin/globus-gatekeeper "$@"
%\end{verbatim}
%\normalsize
%\item 
%If you are starting \Prog{globus-gatekeeper} using \Prog{inetd},
%\Prog{xinetd}, or a similar program,
%set the environment variable there.
%If you are using \Prog{inetd}, you can use the \Prog{env} program
%to set the environment.
%This example does this;
%the example is shown on multiple lines,
%but it will be all on one line in the \Prog{inetd} configuration. 
%\footnotesize
%\begin{verbatim}
%globus-gatekeeper stream tcp nowait root /usr/bin/env
%env CONDOR_CONFIG=/path/to/condor_config
%/path/to/globus/sbin/globus-gatekeeper
%-co /path/to/globus/etc/globus-gatekeeper.conf
%\end{verbatim}
%\normalsize
%If you're using \Prog{xinetd}, add an env setting
%something like the following:
%\footnotesize
%\begin{verbatim}
%service gsigatekeeper
%{
%    env = CONDOR_CONFIG=/path/to/condor_config
%    cps = 1000 1
%    disable = no
%    instances = UNLIMITED
%    max_load = 300
%    nice = 10
%    protocol = tcp
%    server = /path/to/globus/sbin/globus-gatekeeper
%    server_args = -conf /path/to/globus/etc/globus-gatekeeper.conf
%    socket_type = stream
%    user = root
%    wait = no
%}
%\end{verbatim}
%\normalsize
%
%\end{enumerate}
%
%%%%%%%%%%%%%%%%%%%%%%%%%%%%%%%%%%%%%%%%%%%%%%%%%
%\section{Troubleshooting}
%%%%%%%%%%%%%%%%%%%%%%%%%%%%%%%%%%%%%%%%%%%%%%%%%
%
%%%%%%%%%%%%%%%%%%%%%%%%%%%%%%%%%%%%%%%%%%%%%%%%%
%\subsection*{If I see \texttt{PERMISSION DENIED} in my log files,
%what does that mean?}
%%%%%%%%%%%%%%%%%%%%%%%%%%%%%%%%%%%%%%%%%%%%%%%%%
%\index{permission denied}
%
%Most likely, the HTCondor installation has been misconfigured
%and HTCondor's access control security functionality is preventing
%daemons and tools from communicating with each other.
%Other symptoms of this problem include HTCondor tools (such as
%\Condor{status} and \Condor{q}) not producing any output, or commands
%that appear to have no effect (for example, \Condor{off} or
%\Condor{on}). 
%
%The solution is to properly configure the \Macro{ALLOW\_*} and
%\Macro{DENY\_*} settings (for host/IP based authentication) or to
%configure strong authentication and set \Macro{ALLOW\_*} and
%\Macro{DENY\_*} as appropriate.
%Host-based authentication is described in
%section~\ref{sec:Host-Security} on page~\pageref{sec:Host-Security}.
%Information about other forms of authentication is provided in 
%section~\ref{sec:Config-Security} on page~\pageref{sec:Config-Security}.
%
%%%%%%%%%%%%%%%%%%%%%%%%%%%%%%%%%%%%%%%%%%%%%%%%%
%\subsection*{Why did the \Condor{schedd} daemon die and restart?}
%%%%%%%%%%%%%%%%%%%%%%%%%%%%%%%%%%%%%%%%%%%%%%%%%
%\index{condor\_schedd daemon!receiving signal 25}
%
%The \Condor{schedd} daemon receives signal 25,
%dies, and is restarted when the
%history file reaches a 2 Gbyte size limit.
%On 32-bit OSes, HTCondor cannot write log files larger than
%2 Gbytes.
%If you need to keep more than 2 Gbytes of history, you can set a
%maximum history file size of 2 Gbytes and multiple rotations of the
%file.
%For example, to keep 6 Gbytes of history, you would put these lines in
%your HTCondor configuration file:
%\begin{verbatim}
%ENABLE_HISTORY_ROTATION = True
%MAX_HISTORY_LOG = 2000000000
%MAX_HISTORY_ROTATIONS = 2
%\end{verbatim}
%
%%%%%%%%%%%%%%%%%%%%%%%%%%%%%%%%%%%%%%%%%%%%%%%%%
%\subsection*{When I ssh/telnet to a machine to check particulars of how
%HTCondor is doing something, it is always vacating or unclaimed when I
%know a job had been running there!}
%%%%%%%%%%%%%%%%%%%%%%%%%%%%%%%%%%%%%%%%%%%%%%%%%
%
%Depending on how your policy is set up, HTCondor will track \emph{any} tty
%on the machine for the purpose of determining if a job is to be vacated
%or suspended on the machine. It could be the case that after you ssh
%there, HTCondor notices activity on the tty allocated to your connection
%and then vacates the job.
%
%%%%%%%%%%%%%%%%%%%%%%%%%%%%%%%%%%%%%%%%%%%%%%%%%
%\subsection*{What is wrong? I get no output from \Condor{status}, but the HTCondor daemons are running.}
%%%%%%%%%%%%%%%%%%%%%%%%%%%%%%%%%%%%%%%%%%%%%%%%%
%
%\underline{One likely error message} within the collector log of the form
%\footnotesize
%\begin{verbatim}
%DaemonCore: PERMISSION DENIED to host <xxx.xxx.xxx.xxx> for command 0 (UPDATE_STARTD_AD)
%\end{verbatim}
%\normalsize
%indicates a permissions problem.
%The \Condor{startd} daemons do not have write permission to the
%\Condor{collector} daemon.
%This could be because
%you used domain names in your \MacroNI{ALLOW\_WRITE} and/or
%\MacroNI{DENY\_WRITE} configuration macros,
%but the domain name server (DNS) is not properly configured at your site.
%Without the proper configuration, HTCondor cannot resolve
%the IP addresses of your machines
%into fully-qualified domain names (an inverse look up).
%If this is the problem, then the solution takes one of two forms:
%\begin{enumerate}
%\item Fix the DNS so that inverse look ups (trying to get the domain name
%   from an IP address) works for your machines.  You can
%   either fix the DNS itself,
%   or use the \MacroNI{DEFAULT\_DOMAIN\_NAME} setting in your HTCondor
%         configuration file.
%\item Use numeric IP addresses in the \MacroNI{ALLOW\_WRITE} and/or
%   \MacroNI{DENY\_WRITE} configuration macros
%   instead of domain names.
%   As an example of this, assume your site has a machine such as
%   foo.your.domain.com, and it has two subnets, with IP addresses
%   129.131.133.10, and 129.131.132.10.
%   If the configuration macro is set as 
%
%\begin{verbatim}
% ALLOW_WRITE = *.your.domain.com
%\end{verbatim}
%
%   and this does not work, use
%
%\begin{verbatim}
% ALLOW_WRITE = 192.131.133.*, 192.131.132.*
%\end{verbatim}
%\end{enumerate}
%
%\underline{Alternatively}, this permissions problem
%may be caused by being too restrictive in the setting of
%your \MacroNI{ALLOW\_WRITE} and/or
%\MacroNI{DENY\_WRITE} configuration macros.
%If it is, then the solution is to change the macros,
%for example from
%\begin{verbatim}
% ALLOW_WRITE = condor.your.domain.com
%\end{verbatim}
%to
%\begin{verbatim}
% ALLOW_WRITE = *.your.domain.com
%\end{verbatim}
%or possibly
%\footnotesize
%\begin{verbatim}
% ALLOW_WRITE = condor.your.domain.com, foo.your.domain.com, \
% bar.your.domain.com 
%\end{verbatim}
%\normalsize
%
%
%\underline{Another likely error message} within the collector log of the form
%\footnotesize
%\begin{verbatim}
%DaemonCore: PERMISSION DENIED to host <xxx.xxx.xxx.xxx> for command 5 (QUERY_STARTD_ADS)
%\end{verbatim}
%\normalsize
%indicates a similar problem as above, but read permission
%is the problem (as opposed to write permission).
%Use the solutions given above.
%
%%%%%%%%%%%%%%%%%%%%%%%%%%%%%%%%%%%%%%%%%%%%%%%%%
%\subsection*{Why does HTCondor leave mail processes around?}
%%%%%%%%%%%%%%%%%%%%%%%%%%%%%%%%%%%%%%%%%%%%%%%%%
%
%Under FreeBSD and Mac OSX operating systems,
%misconfiguration of of a system's outgoing mail causes
%HTCondor to inadvertently leave paused and zombie mail
%processes around when HTCondor attempts to send notification e-mail.
%The solution to this problem is
%to correct the mailer configuration.
%
%Execute the following command as the user under which HTCondor
%daemons run to determine whether outgoing e-mail works.
%
%\begin{verbatim}
%$ uname -a | mail -v your@emailaddress.com
%\end{verbatim}
%
%If no e-mail arrives, then outgoing e-mail does not work
%correctly.
%
%Note that this problem does not manifest itself
%on non-BSD Unix platforms, such as Linux.
%
%%%%%%%%%%%%%%%%%%%%%%%%%%%%%%%%%%%%%%%%%%%%%%%%%
%\section{Other questions}
%%%%%%%%%%%%%%%%%%%%%%%%%%%%%%%%%%%%%%%%%%%%%%%%%
%
%\index{HTCondor!FAQ|)}
%\index{HTCondor!Frequently Asked Questions|)}
%\index{FAQ|)}
%\index{Frequently Asked Questions|)}
