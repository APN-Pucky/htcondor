%%%%%%%%%%%%%%%%%%%%%%%%%%%%%%%%%%%%%%%%%%%%%%%%%%%%%%%%%%%%%%%%%%%%%%%%%%%
\section{\label{sec:grids-intro}Introduction}
%%%%%%%%%%%%%%%%%%%%%%%%%%%%%%%%%%%%%%%%%%%%%%%%%%%%%%%%%%%%%%%%%%%%%%%%%%%

A goal of grid computing is to allow the utilization of resources that
span many administrative domains.
An HTCondor pool often includes
resources owned and controlled by many different people.
Yet collaborating researchers from different organizations
may not find it feasible to combine all of their computers
into a single, large HTCondor pool.
HTCondor shines in grid computing,
continuing to evolve with the field.

Due to the field's rapid evolution, HTCondor has its own native mechanisms
for grid computing as well as developing interactions 
with other grid systems.


\Term{Flocking} is a native mechanism that allows HTCondor jobs
submitted from within one pool
to execute on another, separate HTCondor pool.
Flocking is enabled by configuration within each of the pools.
An advantage to flocking is that jobs migrate from one
pool to another based on the availability of machines to
execute jobs.
When the local HTCondor pool is not able to run the job
(due to a lack of currently available machines),
the job flocks to another pool.
A second advantage to using flocking is that the user
(who submits the job) does not need to be concerned with
any aspects of the job.
The user's submit description file (and the job's \SubmitCmd{universe})
are independent of the flocking mechanism.

Other forms of grid computing are enabled by using
the \SubmitCmd{grid} \SubmitCmd{universe}
and further specified with the \SubmitCmd{grid\_type}.
For any HTCondor job, 
the job is submitted on a machine in the local HTCondor pool.
The location where it is executed is identified as the remote machine
or remote resource.
These various grid computing mechanisms offered by
HTCondor are distinguished by the software
running on the remote resource.

When HTCondor is running on the remote resource,
and the desired grid computing mechanism 
is to move the job from the local pool's job queue
to the remote pool's job queue,
it is called HTCondor-C.
The job is submitted using the 
\SubmitCmd{grid}
\SubmitCmd{universe}, 
and the \SubmitCmd{grid\_type} is \SubmitCmd{condor}.
HTCondor-C jobs have the advantage that once the job has moved
to the remote pool's job queue,
a network partition does not affect the execution of the job.
A further advantage of HTCondor-C jobs is that the \SubmitCmd{universe}
of the job at the remote resource is not restricted. 

When other middleware is running on the remote resource,
such as Globus,
HTCondor can still submit and manage jobs to be executed on
remote resources.
A \SubmitCmd{grid} \SubmitCmd{universe} job,
with a \SubmitCmd{grid\_type} of
\SubmitCmd{gt2} or \SubmitCmd{gt5}
calls on Globus software to execute the job on a remote resource.
Like HTCondor-C jobs, a network partition does not affect
the execution of the job.
The remote resource must have Globus software running.

\index{glidein}
\index{grid computing!glidein}
HTCondor permints the temporary addition of a
Globus-controlled resource to a local pool.
This is called \Term{glidein}.
Globus software is utilized to execute HTCondor daemons on the
remote resource.
The remote resource appears to have joined the local HTCondor pool.
A user submitting a job may then explicitly specify the
remote resource as the execution site of a job.

Starting with HTCondor Version 6.7.0, the \SubmitCmd{grid} universe
replaces the \SubmitCmd{globus} universe.
Further specification of a \SubmitCmd{grid} universe job is done
within the \SubmitCmd{grid\_resource} command in a submit description file.
