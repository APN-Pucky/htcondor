
%%%%%%%%%%%%%%%%%%%%%%%%%%%%%%%%%%%%%%%%%%%%%%%%%%%%%%%%%%%%%%%%%%%%%%%%%%%
\subsection{\label{sec:CREAM}The cream Grid Type }
%%%%%%%%%%%%%%%%%%%%%%%%%%%%%%%%%%%%%%%%%%%%%%%%%%%%%%%%%%%%%%%%%%%%%%%%%%%
\index{cream}
\index{grid computing!submitting jobs to cream}

CREAM is a job submission interface being developed at INFN for the
gLite software stack. 
The CREAM homepage is \URL{http://grid.pd.infn.it/cream/}.
The protocol is based on web services.

The protocol requires an X.509 proxy for the job,
so the submit description file command \SubmitCmd{x509userproxy}
will be used.

A CREAM resource specification is of the form:
\footnotesize
\begin{verbatim}
grid_resource = cream <web-services-address> <batch-system> <queue-name>
\end{verbatim}
\normalsize
The \verb@<web-services-address>@ appears the same for most servers,
differing only in the host name, as
\begin{verbatim}
<machinename[:port]>/ce-cream/services/CREAM2
\end{verbatim}
Future versions of HTCondor may require only the host name, 
filling in other aspects of the web service for the user.

The \verb@<batch-system>@ is the name of the batch system that sits behind
the CREAM server,
into which it submits the jobs.
Normal values are \verb@pbs@, \verb@lsf@, and \verb@condor@.

The \verb@<queue-name>@ identifies which queue within the batch system
should be used.
Values for this will vary by site, with no typical values.

A full example for the specification of a CREAM \SubmitCmd{grid\_resource} is
\footnotesize
\begin{verbatim}
grid_resource = cream https://cream-12.pd.infn.it:8443/ce-cream/services/CREAM2
   pbs cream_1
\end{verbatim}
\normalsize
This is a single line within the submit description file,
although it is shown here on two lines for formatting reasons.

