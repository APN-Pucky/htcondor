%%%%%%%%%%%%%%%%%%%%%%%%%%%%%%%%%%%%%%%%%%%%%%%%%%%%%%%%%%%%%%%%%%%%%%%%%%%
\subsection{\label{sec:Gce}The GCE Grid Type }
%%%%%%%%%%%%%%%%%%%%%%%%%%%%%%%%%%%%%%%%%%%%%%%%%%%%%%%%%%%%%%%%%%%%%%%%%%%
\index{Google Compute Engine}
\index{GCE grid jobs}
\index{grid computing!submitting jobs to GCE}
\index{grid type!gce}

HTCondor jobs may be submitted to the Google Compute Engine (GCE)
cloud service.
GCE is an on-line commercial service that provides
the rental of computers by the hour to run computational applications.
Its runs virtual machine images that have been uploaded to Google's
servers.
More information about Google Compute Engine is available at
\URL{http://cloud.google.com/Compute}.

%%%%%%%%%%%%%%%%%%%%%%%%%%%%%%%%%%%%%%%%%%%%%%%%%%%%%%%%%%%%%%%%%%%%%%%%%%%
\subsubsection{\label{sec:Gce-submit}GCE Job Submission}
%%%%%%%%%%%%%%%%%%%%%%%%%%%%%%%%%%%%%%%%%%%%%%%%%%%%%%%%%%%%%%%%%%%%%%%%%%%

HTCondor jobs are submitted to the GCE service
with the \SubmitCmdNI{grid} universe, setting the
\SubmitCmd{grid\_resource} command to \SubmitCmdNI{gce}, followed 
by the service's URL, your GCE project, 
and the desired GCE zone to be used.
The submit description file command will be similar to:
\begin{verbatim}
grid_resource = gce https://www.googleapis.com/compute/v1 my_proj us-central1-a
\end{verbatim}

Since the HTCondor job is a virtual machine image,
most of the submit description file commands
specifying input or output files are not applicable.
The \SubmitCmd{executable} command is still required,
but its value is ignored. 
It identifies different jobs in the output of \Condor{q}.

The VM image for the job must already reside in Google's Cloud Storage
service and be registered with GCE.
In the submit description file,
provide the identifier for the image using
the \SubmitCmd{gce\_image} command.

This grid type requires granting HTCondor permission to use your
Google account. 
The easiest way to do this is to use the \Prog{gcloud}
command-line tool distributed by Google.
Find \Prog{gcloud} and documentation for it at
\URL{https://cloud.google.com/compute/docs/gcloud-compute/}.
After installation of \Prog{gcloud}, 
run \Prog{gcloud auth login} and follow its directions.
Once done with that step,
the tool will write authorization credentials to the
file \File{.config/gcloud/credentials} under your HOME directory.

Given an authorization file, specify its location in
the submit description file using the \SubmitCmd{gce\_auth\_file} command,
as in the example:
\begin{verbatim}
gce_auth_file = /path/to/auth-file
\end{verbatim}

GCE allows the choice of different hardware configurations 
for instances to run on.
Select which configuration to use for the \SubmitCmdNI{gce} grid type
with the \SubmitCmd{gce\_machine\_type} submit description file command.
HTCondor provides no default.

Each virtual machine instance can be given a unique set of metadata,
which consists of name/value pairs, similar to the environment variables
of regular jobs.
The instance can query its metadata via a well-known address.
This makes it easy for many instances to share the same VM image,
but perform different work.
This data can be specified to HTCondor in one of two ways.
First, the data can be provided directly in the submit description file 
using the \SubmitCmd{gce\_metadata} command.
The value should be a comma-separated list of name=value settings, 
as the example:
\begin{verbatim}
gce_metadata = setting1=foo,setting2=bar
\end{verbatim}

Second, the data can be stored in a file, 
and the file name is specified with the
\SubmitCmd{gce\_metadata\_file} submit description file command.
This second option allows a wider range of characters to be used in the
metadata values.
Each name=value pair should be on its own line.
No white space is removed from the lines, except for the newline that
separates entries.

Both options can be used at the same time, 
but do not use the same metadata name in both places.

HTCondor sets the following elements when describing the instance
to the GCE server: \SubmitCmdNI{machineType}, \SubmitCmdNI{name},
\SubmitCmdNI{scheduling}, \SubmitCmdNI{disks}, \SubmitCmdNI{metadata},
and \SubmitCmdNI{networkInterfaces}.
You can provide additional elements to be included in the instance
description as a block of JSON.
Write the additional elements to a file, and specify the filename
in your submit file with the \SubmitCmd{gce\_json\_file} command.
The contents of the file are inserted into HTCondor's JSON description
of the instance, between a comma and the closing brace.

Here's a sample JSON file that sets two additional elements:

\begin{verbatim}
"canIpForward": True,
"description": "My first instance"
\end{verbatim}

%%%%%%%%%%%%%%%%%%%%%%%%%%%%%%%%%%%%%%%%%%%%%%%%%%%%%%%%%%%%%%%%%%%%%%%%%%%
\subsubsection{\label{sec:Gce-config}GCE Configuration Variables}
%%%%%%%%%%%%%%%%%%%%%%%%%%%%%%%%%%%%%%%%%%%%%%%%%%%%%%%%%%%%%%%%%%%%%%%%%%%

The following configuration parameters are specific to the \SubmitCmdNI{gce}
grid type. 
The values listed here are the defaults. 
Different values may be specified in the HTCondor configuration files.

\footnotesize
\begin{verbatim}
GCE_GAHP     = $(SBIN)/gce_gahp
GCE_GAHP_LOG = /tmp/GceGahpLog.$(USERNAME)
\end{verbatim}
\normalsize
