%%%%%%%%%%%%%%%%%%%%%%%%%%%%%%%%%%%%%%%%%%%%%%%%%%
\subsubsection{\label{sec:Condor-G-GridMonitor}The Grid Monitor}
%%%%%%%%%%%%%%%%%%%%%%%%%%%%%%%%%%%%%%%%%%%%%%%%%%
\index{Grid Monitor}
\index{grid computing!Grid Monitor}
\index{scalability!using the Grid Monitor}

Condor's Grid Monitor is designed to improve the scalability of
machines running the Globus Toolkit's GRAM2 gatekeeper.
Normally, this service runs a jobmanager process for 
every job submitted to the gatekeeper.
This includes both currently running jobs and jobs waiting in the queue.
Each jobmanager runs a Perl script at
frequent intervals (every 10 seconds) to poll the state of
its job in the local batch system.
For example, with 400 jobs submitted to a gatekeeper,
there will be 400 jobmanagers running,
each regularly starting a Perl script.
When a large number of jobs
have been submitted to a single gatekeeper,
this frequent polling can heavily load the gatekeeper.
When the gatekeeper is under heavy load,
the system can become non-responsive, and a variety of problems can occur.

Condor's Grid Monitor temporarily replaces these jobmanagers.
It is named the Grid Monitor, because it replaces the monitoring
(polling) duties previously done by jobmanagers.
When the Grid Monitor runs,
Condor attempts to start a single
process to poll all of a user's jobs at a given gatekeeper.
While a job is waiting in the queue, but not yet running,
Condor shuts down the associated jobmanager,
and instead relies on the Grid Monitor to report changes in status.
The jobmanager started to add the job to the remote
batch system queue is shut down.
The jobmanager restarts when the job begins running.

The Grid Monitor requires that the gatekeeper support the fork
jobmanager with the name \Prog{jobmanager-fork}.
If the gatekeeper does not support the fork jobmanager,
the Grid Monitor will not be used for that site.
The \Condor{gridmanager} log file reports any problems
using the Grid Monitor.

The Grid Monitor is enabled by default,
and the
configuration macro \Macro{GRID\_MONITOR} identifies
the location of the executable.

