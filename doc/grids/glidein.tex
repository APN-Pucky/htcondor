%%%%%%%%%%%%%%%%%%%%%%%%%%%%%%%%%%%%%%%%%%%%%%%%%%
\subsection{\label{sec:Glidein}Glidein}
%%%%%%%%%%%%%%%%%%%%%%%%%%%%%%%%%%%%%%%%%%%%%%%%%%
\index{universe!Globus}
\index{Globus}
\index{Condor commands!condor\_glidein}

Condor works together with Globus software to provide the capability
of submitting Condor jobs to remote computer systems that are
not part of the local Condor pool of machines.
Globus software provides mechanisms to access and utilize
remote resources.

Glidein is a mechanism by which one or more Grid resources (remote machines)
temporarily join a local Condor pool. 
The program \Condor{glidein} is used to add a machine
to a Condor pool.
During the period of time when the added resource is
part of the local pool, the resource is visible 
to users of the pool, but the resource is only available for
use by the user
that added the resource to the pool.

After glidein, the user may submit jobs for execution on the
added resource the same way that all Condor jobs are submitted.
To force a submitted job to run on the added resource, the
submit description file contains a requirement that the job run 
specifically on the added resource.


%%%%%%%%%%%%%%%%%%%%%%%%%%%%%%%%%%%%%%%%%%%%%%%%%%
\subsubsection{\Condor{glidein} Requirements}
%%%%%%%%%%%%%%%%%%%%%%%%%%%%%%%%%%%%%%%%%%%%%%%%%%

Use of \Condor{glidein} requires prior set up of two
categories of items.
The first category enable the use of Globus software.
The second category are specific to Condor.

To use Globus for the purpose of \Condor{glidein}:
\begin{enumerate}

\item Obtain an X.509 certificate issued by a certificate authority.

\item Have valid access to a Globus resource.
You must be a valid Globus user and be mapped to a valid login account by
the site's Globus administrator on every Globus resource that will be
added to the local Condor pool using \Condor{glidein}.
More information can be found at \URL{http://www.globus.org}

\item The environment variables \Env{HOME} and \Env{GLOBUS\_LOCATION}
must be set.

\end{enumerate}

In order to use \Condor{glidein} to add a Globus resource to the
local Condor pool,
there are several requirements beyond the general Globus requirements
given above.

\begin{enumerate}
\item Use Globus v1.1 or better.

\item The local Condor pool configuration file(s) must 
  give \Macro{HOSTALLOW\_WRITE} permission
  to every resource that will be added using \Condor{glidein}. 
  Wildcards are permitted in this specification.
  An example is of adding every machine at
  cs.wisc.edu by adding *.cs.wisc.edu to the
  \Macro{HOSTALLOW\_WRITE} list.
  Recall that the changes take effect when all machines
  in the local pool are sent a reconfigure command.

\item Included in the \Env{PATH} must be the common user programs
  directory \File{/bin}, globus tools, and the Condor user program
  directory.

\end{enumerate}

%%%%%%%%%%%%%%%%%%%%%%%%%%%%%%%%%%%%%%%%%%%%%%%%%%
\subsubsection{What \Condor{glidein} Does}
%%%%%%%%%%%%%%%%%%%%%%%%%%%%%%%%%%%%%%%%%%%%%%%%%%

\Condor{glidein} first checks that there is a valid proxy
and that the necessary files are available to \Condor{glidein}.

\Condor{glidein} then contacts the Globus resource and checks for the
presence of the necessary configuration files and Condor executables.
If the executables are not present for the machine architecture,
operating system version, and Condor version required, a
server running at UW is contacted to transfer the needed executables.
To gain access to the server, send email to \Email{condor-admin@cs.wisc.edu}
that includes the name of your X.509 certificate.

When the files are correctly in place,
Condor daemons are started.
\Condor{glidein} does this by creating a submit description file for
\Condor{submit}, which runs the \Condor{master} under the Globus
universe.
This implies that execution of the \Condor{master} is started on the Globus
resource.
The Condor daemons exit gracefully when no jobs run on the daemons for a
configurable period of time. The default length of time is 20 minutes.

The Condor daemons on the Globus resource contact the local pool and
attempt to join the pool.  The \Expr{START}
expression for the \Condor{startd} daemon requires that the username
of the person running \Condor{glidein} matches the username of the jobs
submitted through Condor.

After a short length of time,
the Globus resource can be seen in the local Condor pool,
as with this example.

\footnotesize
\begin{verbatim}
% condor_status | grep denal
7591386@denal IRIX65      SGI    Unclaimed  Idle       3.700  24064  0+00:06:35
\end{verbatim}
\normalsize

Once the Globus resource has been added to the local Condor
pool with \Condor{glidein},
job(s) may be submitted.
To force a job to run on the Globus resource,
specify that Globus resource as a machine requirement
in the submit description file. 
Here is an example from within the submit description file
that forces submission to the Globus resource denali.mcs.anl.gov:
\begin{verbatim}
      requirements = ( machine == "denali.mcs.anl.gov" ) \
         && FileSystemDomain != "" \
         && Arch != "" && OpSys != ""
\end{verbatim}
This example requires that the job run only on denali.mcs.anl.gov,
and it prevents Condor from inserting the filesystem domain,
architecture, and operating system attributes as requirements
in the matchmaking process.
Condor must be told not to use the submission machine's
attributes in those cases
where the Globus resource's attributes
do not match the submission machine's attributes.
