%%%%%%%%%%%%%%%%%%%%%%%%%%%%%%%%%%%%%%%%%%%%%%%%%%
\section{\label{sec:Glidein}Glidein}
%%%%%%%%%%%%%%%%%%%%%%%%%%%%%%%%%%%%%%%%%%%%%%%%%%
\index{universe!grid}
\index{Condor commands!condor\_glidein}
\index{glidein}
\index{grid computing!glidein}

Glidein is a mechanism by which one or more grid resources (remote machines)
temporarily join a local Condor pool. 
The program \Condor{glidein} is used to add a machine
to a Condor pool.
During the period of time when the added resource is
part of the local pool, the resource is visible 
to users of the pool.
But, by default, the resource is only available for
use by the user
that added the resource to the pool.

After glidein, the user may submit jobs for execution on the
added resource the same way that all Condor jobs are submitted.
To force a submitted job to run on the added resource, the
submit description file could contain a requirement that the job run 
specifically on the added resource.


%%%%%%%%%%%%%%%%%%%%%%%%%%%%%%%%%%%%%%%%%%%%%%%%%%
\subsection{What \Condor{glidein} Does}
%%%%%%%%%%%%%%%%%%%%%%%%%%%%%%%%%%%%%%%%%%%%%%%%%%

\Condor{glidein} works by installing and executing
necessary Condor daemons and configuration on the remote resource,
such that the resource reports to and joins the local pool.
\Condor{glidein} accomplishes two separate tasks towards
having a remote grid resource join the local Condor pool.
They are the set up task and the execution task.

The set up task generates necessary 
configuration files and locates proper platform-dependent
binaries for the Condor daemons.
A script is also generated that can be used during
the execution task to invoke the proper Condor daemons.
These files are copied to the remote resource as necessary.
The configuration variable \Macro{GLIDEIN\_SERVER\_URLS}
defines a list of locations from which the necessary
binaries are obtained.
Default values cause binaries to be downloaded from the 
UW site.
See 
section~\ref{param:GlideinServerURLS} 
on page~\pageref{param:GlideinServerURLS}
for a full definition of this configuration variable.

When the files are correctly in place,
the execution task starts the Condor daemons.
\Condor{glidein} does this by submitting a Condor job
to run under the grid universe.
The job runs the \Condor{master} on the remote grid resource.
The \Condor{master} invokes other daemons, which contact
the local pool's \Condor{collector} to join the pool.
The Condor daemons exit gracefully when no jobs run on the daemons for a
preset period of time.

Here is an example of how a glidein resource appears, similar to how
any other machine appears.  The name has a
slightly different form, in order to handle the possibility of
multiple instances of glidein daemons inhabiting a multi-processor
machine.

\footnotesize
\begin{verbatim}
% condor_status | grep denal
7591386@denal LINUX       INTEL  Unclaimed  Idle       3.700  24064  0+00:06:35

\end{verbatim}
\normalsize

%%%%%%%%%%%%%%%%%%%%%%%%%%%%%%%%%%%%%%%%%%%%%%%%%%
\subsection{Configuration Requirements in the Local Pool}
%%%%%%%%%%%%%%%%%%%%%%%%%%%%%%%%%%%%%%%%%%%%%%%%%%
\index{configuration!for glidein}
\index{glidein!configuration}

As remote grid resources join the local pool,
these resources must report to the local pool's \Condor{collector} daemon.
Security demands that the local pool's \Condor{collector} 
list all hosts from which they will accept communication.
Therefore, all remote grid resources accepted for glidein
must be given
\Macro{HOSTALLOW\_WRITE} permission.
An expected way to do this is to modify the empty variable
(within the sample configuration file)
\MacroNI{GLIDEIN\_SITES} to list all remote grid resources
accepted for glidein.
The list is a space or comma separated list of hosts.
This list is then given the proper permissions by an additional
redefinition of the \MacroNI{HOSTALLOW\_WRITE} configuration variable,
to also include the list of hosts
as in the following example.

\footnotesize
\begin{verbatim}
GLIDEIN_SITES = A.example.com, B.example.com, C.example.com
HOSTALLOW_WRITE = $(HOSTALLOW_WRITE) $(GLIDEIN_SITES)
\end{verbatim}
\normalsize
Recall that for configuration file changes to take effect,
\Condor{reconfig} must be run.

If this configuration change to the security settings on
the local Condor pool cannot be made,
an additional Condor pool that utilizes
personal Condor may be defined.
The single machine pool
may coexist with other instances of Condor.
\Condor{glidein} is executed to have the remote grid
resources join this personal Condor pool.

%%%%%%%%%%%%%%%%%%%%%%%%%%%%%%%%%%%%%%%%%%%%%%%%%%
\subsection{Running Jobs on the Remote Grid Resource After Glidein }
%%%%%%%%%%%%%%%%%%%%%%%%%%%%%%%%%%%%%%%%%%%%%%%%%%

Once the Globus resource has been added to the local Condor
pool with \Condor{glidein},
job(s) may be submitted.
To force a job to run on the Globus resource,
specify that Globus resource as a machine requirement
in the submit description file. 
Here is an example from within the submit description file
that forces submission to the Globus resource denali.mcs.anl.gov:
\begin{verbatim}
      requirements = ( machine == "denali.mcs.anl.gov" ) \
         && FileSystemDomain != "" \
         && Arch != "" && OpSys != ""
\end{verbatim}
This example requires that the job run only on denali.mcs.anl.gov,
and it prevents Condor from inserting the file system domain,
architecture, and operating system attributes as requirements
in the matchmaking process.
Condor must be told not to use the submission machine's
attributes in those cases
where the Globus resource's attributes
do not match the submission machine's attributes and your job
really is capable of running on the target machine.  You
may want to use Condor's file-transfer capabilities in order
to copy input and output files back and forth between the submission
and execution machine.
