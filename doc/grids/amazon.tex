%%%%%%%%%%%%%%%%%%%%%%%%%%%%%%%%%%%%%%%%%%%%%%%%%%%%%%%%%%%%%%%%%%%%%%%%%%%
\subsection{\label{sec:Amazon}The EC2 Grid Type }
%%%%%%%%%%%%%%%%%%%%%%%%%%%%%%%%%%%%%%%%%%%%%%%%%%%%%%%%%%%%%%%%%%%%%%%%%%%
\index{Amazon EC2 Query API}
\index{grid computing!submitting jobs using the EC2 Query API}
\index{grid type!ec2}

Condor jobs may be submitted to clouds supporting
Amazon's Elastic Compute Cloud (EC2) interface.
Amazon's EC2 is an on-line commercial service that allows 
the rental of computers by the hour to run computational applications.
It runs virtual machine images that have been uploaded to Amazon's
online storage service (S3 or EBS).
More information about Amazon's EC2 service is available at
\URL{http://aws.amazon.com/ec2}.

The \SubmitCmd{ec2} grid type uses the EC2 Query API,
also called the EC2 REST API.

%%%%%%%%%%%%%%%%%%%%%%%%%%%%%%%%%%%%%%%%%%%%%%%%%%%%%%%%%%%%%%%%%%%%%%%%%%%
\subsubsection{\label{sec:Amazon-submit}EC2 Job Submission}
%%%%%%%%%%%%%%%%%%%%%%%%%%%%%%%%%%%%%%%%%%%%%%%%%%%%%%%%%%%%%%%%%%%%%%%%%%%

Condor jobs are submitted to Amazon's EC2
with the \SubmitCmd{grid} universe, and setting the
\SubmitCmd{grid\_resource} command to \SubmitCmd{ec2}, followed 
by the service's URL. For example,
partial contents of the submit description file may be
\begin{verbatim}
grid_resource = ec2 https://ec2.amazonaws.com/
\end{verbatim}

Since the job is a virtual machine image,
most of the submit description file commands
specifying input or output files are not applicable.
The \SubmitCmd{executable} command is still required,
but its value is ignored. 
It can be used to identify different jobs in the output of \Condor{q}.

The VM image for the job must already reside in one of Amazon's storage
service (S3 or EBS) and be registered with EC2.
In the submit description file,
provide the identifier for the image using either \SubmitCmd{ec2\_ami\_id}.

This grid type requires access to user authentication information,
in the form of path names to files containing the appropriate keys.

The \SubmitCmd{ec2} grid type has two different authentication methods.
The first authentication method uses the EC2 API's built-in authentication.
Specify the service with expected \Expr{http://} or \Expr{https://} URL,
and set the EC2 access key and secret access key as follows:

\begin{verbatim}
ec2_access_key_id = /path/to/access.key
ec2_secret_access_key = /path/to/secret.key
\end{verbatim}

While both pairs of files may be associated with the same account, 
the credentials are not the same.

The second authentication method for the EC2 grid type is X.509.
Specify the service with an \Expr{x509://} URL, 
even if the URL was given in another form.  
Use \SubmitCmd{ec2\_access\_key\_id} to 
specify the path to the X.509 public key (certificate),
and \SubmitCmd{ec2\_secret\_access\_key} specifies the path to the X.509 
private key as in the following example:

\begin{verbatim}
grid_resource = ec2 x509://service.example
ec2_access_key_id = /path/to/x.509/public.key
ec2_secret_access_key = /path/to/x.509/private.key
\end{verbatim}

If using an X.509 proxy, specify the proxy in both places.

%Condor can use the EC2 API to create an SSH key pair that allows
%secure log in to the virtual machine once it is running.
%If the command
%\SubmitCmd{ec2\_keypair\_file}
%is set in the submit description file,
%Condor will write an SSH private key into the indicated file.
%The key can be used to log into the virtual machine.
%Note that modification will also be needed of the firewall
%rules for the job to incoming SSH connections.

An EC2 service uses a firewall to restrict network access to 
the virtual machine instances it runs.
Typically, no incoming connections are allowed.
One can define sets of firewall rules and give them names.
The EC2 API calls these security groups. 
If utilized, tell Condor what set of security
groups should be applied to each VM using the
\SubmitCmd{ec2\_security\_groups} submit description file command.
If not provided, Condor uses the security group \SubmitCmd{default}.

The EC2 API allows the choice of different hardware configurations 
for instances to run on.
Select which configuration to use for the \SubmitCmd{ec2} grid type
with the \SubmitCmd{ec2\_instance\_type} submit description file command.
Condor provides no default.

Each virtual machine instance can be given up to 16Kbytes of unique data, 
accessible by the instance by connecting to a well-known address.
This makes it easy for many instances to share the same VM image,
but perform different work.
This data can be specified to Condor in one of two ways.
First, the data can be provided directly in the submit description file 
using the \SubmitCmd{ec2\_user\_data} command.
Second, the data can be
stored in a file, and the file name is specified with the
\SubmitCmd{ec2\_user\_data\_file} submit description file command.
This second option allows the use of binary data.
If both options are used, the two blocks of
data are concatenated, with the data from \SubmitCmd{ec2\_user\_data} 
occurring first.  Condor performs the base64 encoding that EC2 expects on 
the data.

Below is a sample submission: 

\begin{verbatim}
###################################
# Note to submit an AMI as a job we need the grid universe
# To submit to a different region update the url passed to the grid_resource
# e.g. https://ec2.ap-northeast-1.amazonaws.com
# Note: The ami *must* be present in that region
# For more details see: 
# http://docs.amazonwebservices.com/AWSEC2/latest/APIReference/Welcome.html

Universe = grid
grid_resource = ec2 https://ec2.amazonaws.com/

# Executable in this context is just a label for the job
Executable  = ec2_test_job

# log for job run
Log=$(cluster).ec2.log

###################################
# The AMI ID used 
ec2_ami_id = ami-MyAmiId0
ec2_instance_type = m1.small

###################################
# User data input for the instance (optional)
# ec2_user_data = Hello EC2!
# ec2_user_data_file = /path/to/datafile

###################################
# Required credentials used to access EC2 (must be full paths)
ec2_access_key_id = /home/user/your_ec2.aid
ec2_secret_access_key = /home/user/your_ec2.key

###################################
# Location to store instance's SSH keypair (optional)
ec2_keypair_file = /home/user/ec2_gend_key.pem

###################################
# Pre-allocated elastic ip for the instance (optional)
# ec2_elastic_ip = your.elastic.ip.addr

###################################
# Security group for the instance (optional, default if not provided)
ec2_security_groups = sg-MySecGroup

###################################
# VPC subnet in which to launch the instance (optional)
# ec2_vpc_subnet = subnet-a1bc23d4

###################################
# Param to attach to an ebs volume volume:/drive_loc (optional)
# ec2_ebs_volumes = vol-abcde15f:/dev/sdb
###################################
# If a volume is specified, the zone is required
# Note: zone must be within specified or default region
# ec2_availability_zone = us-east-1b

############################
# Adding Tag (name) foo=bar (value) (optional)
# ec2_tag_names = foo
# ec2_tag_foo = bar 

queue
\end{verbatim}

%%%%%%%%%%%%%%%%%%%%%%%%%%%%%%%%%%%%%%%%%%%%%%%%%%%%%%%%%%%%%%%%%%%%%%%%%%%
\subsubsection{\label{sec:Amazon-config}EC2 Configuration Variables}
%%%%%%%%%%%%%%%%%%%%%%%%%%%%%%%%%%%%%%%%%%%%%%%%%%%%%%%%%%%%%%%%%%%%%%%%%%%

The \SubmitCmd{ec2} grid type requires these configuration variables 
to be set in the Condor configuration file:

\footnotesize
\begin{verbatim}
EC2_GAHP     = $(SBIN)/ec2_gahp
EC2_GAHP_LOG = /tmp/EC2GahpLog.$(USERNAME)
\end{verbatim}
\normalsize

The \SubmitCmd{ec2} grid type does not presently permit the explicit use 
of an HTTP proxy.
