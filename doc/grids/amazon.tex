%%%%%%%%%%%%%%%%%%%%%%%%%%%%%%%%%%%%%%%%%%%%%%%%%%%%%%%%%%%%%%%%%%%%%%%%%%%
\subsection{\label{sec:Amazon}The amazon and ec2 Grid Types }
%%%%%%%%%%%%%%%%%%%%%%%%%%%%%%%%%%%%%%%%%%%%%%%%%%%%%%%%%%%%%%%%%%%%%%%%%%%
\index{Amazon EC2 Soap API}
\index{grid computing!submitting jobs using the EC2 Soap API}
\index{grid computing!submitting jobs using the EC2 Query API}
\index{grid type!amazon}
\index{grid type!ec2}

Condor jobs may be submitted to clouds supporting
Amazon's Elastic Compute Cloud (EC2) interface.
Amazon's EC2 is an on-line commercial service that allows 
the rental of computers by the hour to run computational applications.
It runs virtual machine images that have been uploaded to Amazon's
online storage service (S3 or EBS).
More information about Amazon's EC2 service is available at
\URL{http://aws.amazon.com/ec2}.

The \SubmitCmd{amazon} grid type uses the EC2 SOAP API;
the \SubmitCmd{ec2} grid type uses the EC2 Query API,
also called the EC2 REST API.
More cloud software, and more cloud service providers, support the 
EC2 Query API.

%%%%%%%%%%%%%%%%%%%%%%%%%%%%%%%%%%%%%%%%%%%%%%%%%%%%%%%%%%%%%%%%%%%%%%%%%%%
\subsubsection{\label{sec:Amazon-submit}Amazon EC2 Job Submission}
%%%%%%%%%%%%%%%%%%%%%%%%%%%%%%%%%%%%%%%%%%%%%%%%%%%%%%%%%%%%%%%%%%%%%%%%%%%

Condor jobs are submitted to Amazon's EC2
with the \SubmitCmd{grid} universe, and setting the
\SubmitCmd{grid\_resource} command to either
\SubmitCmd{amazon}  or \SubmitCmd{ec2}, followed by the service's URL.
For example,
partial contents of the submit description file may be
\begin{verbatim}
grid_resource = amazon https://ec2.amazonaws.com/
\end{verbatim}

Since the job is a virtual machine image,
most of the submit description file commands
specifying input or output files are not applicable.
The \SubmitCmd{executable} command is still required,
but its value is ignored. 
It can be used to identify different jobs in the output of \Condor{q}.

The VM image for the job must already reside in one of Amazon's storage
service (S3 or EBS) and be registered with EC2.
In the submit description file,
provide the identifier for the image using either
\SubmitCmd{amazon\_ami\_id} or \SubmitCmd{ec2\_ami\_id},
as appropriate.

Both of these grid types require access to user authentication information,
in the form of path names to files containing the appropriate keys.
For the \SubmitCmd{amazon} grid type:
\begin{verbatim}
amazon_public_key = /path/to/x509/cert
amazon_private_key = /path/to/private/key
\end{verbatim}

For the \SubmitCmd{ec2} grid type:
\begin{verbatim}
ec2_access_key = /path/to/access.key
ec2_secret_access_key = /path/to/secret.key
\end{verbatim}

While both pairs of files may be associated with the same account, 
the credentials are not the same.

Condor can use the EC2 API to create an SSH key pair that allows
secure log in to the virtual machine once it is running.
If the command
\SubmitCmd{amazon\_keypair\_file}
or \SubmitCmd{ec2\_keypair\_file}
is set in the submit description file,
Condor will write an SSH private key into the indicated file.
The key can be used to log into the virtual machine.
Note that modification will also be needed of the firewall
rules for the job to incoming SSH connections.

An EC2 service uses a firewall to restrict network access to 
the virtual machine instances it runs.
Typically, no incoming connections are allowed.
One can define sets of firewall rules and give them names.
The EC2 API calls these security groups. 
If utilized, tell Condor what set of security
groups should be applied to each VM using the
\SubmitCmd{amazon\_security\_groups} 
or \SubmitCmd{ec2\_security\_groups} submit description file command.
If not provided, Condor uses the security group \SubmitCmd{default}.

The EC2 API allows the choice of different hardware configurations 
for instances to run on.
Select which configuration to use for the \SubmitCmd{amazon} grid type
with the \SubmitCmd{amazon\_instance\_type} submit description file command.
Condor will use a default value of
\SubmitCmd{m1.small}, if not specified.
% Other values: m1.large m1.xlarge c1.medium c1.xlarge
The \SubmitCmd{ec2} grid type uses the \SubmitCmd{ec2\_instance\_type}, 
but provides no default.

Each virtual machine instance can be given up to 16Kbytes of unique data, 
accessible by the instance by connecting to a well-known address.
This makes it easy for many instances to share the same VM image,
but perform different work.
This data can be specified to Condor in one of two ways.
First, the data can be provided directly in the submit description file 
using the \SubmitCmd{amazon\_user\_data} 
or the \SubmitCmd{ec2\_user\_data} command.
Second, the data can be
stored in a file, and the file name is specified with the
\SubmitCmd{amazon\_user\_data\_file} 
or the \SubmitCmd{ec2\_user\_data\_file} submit description file command.
This second option allows the use of binary data.
If both options are used, the two blocks of
data are concatenated, with the data from \SubmitCmd{amazon\_user\_data}
or \SubmitCmd{ec2\_user\_data} occurring first.
Condor performs the base64 encoding that EC2 expects on the data.

%%%%%%%%%%%%%%%%%%%%%%%%%%%%%%%%%%%%%%%%%%%%%%%%%%%%%%%%%%%%%%%%%%%%%%%%%%%
\subsubsection{\label{sec:Amazon-config}EC2 Configuration Variables}
%%%%%%%%%%%%%%%%%%%%%%%%%%%%%%%%%%%%%%%%%%%%%%%%%%%%%%%%%%%%%%%%%%%%%%%%%%%

The \SubmitCmd{amazon} grid type requires these configuration variables 
to be set in the Condor configuration file:

\footnotesize
\begin{verbatim}
AMAZON_GAHP     = $(SBIN)/amazon_gahp
AMAZON_GAHP_LOG = /tmp/AmazonGahpLog.$(USERNAME)
\end{verbatim}
\normalsize

If an HTTP proxy is needed to reach EC2, tell Condor to use it
via the \Macro{AMAZON\_HTTP\_PROXY} configuration variable.

The \SubmitCmd{ec2} grid type requires two configuration variables to be
set in the Condor configuration file:

\footnotesize
\begin{verbatim}
EC2_GAHP     = $(SBIN)/ec2_gahp
EC2_GAHP_LOG = /tmp/EC2GahpLog.$(USERNAME)
\end{verbatim}
\normalsize

The \SubmitCmd{ec2} grid type does not presently permit the explicit use 
of an HTTP proxy.
