%%%%%%%%%%%%%%%%%%%%%%%%%%%%%%%%%%%%%%%%%%%%%%%%%%%%%%%%%%%%%%%%%%%%%%%%%%%
\subsection{\label{sec:Amazon}The EC2 Grid Type }
%%%%%%%%%%%%%%%%%%%%%%%%%%%%%%%%%%%%%%%%%%%%%%%%%%%%%%%%%%%%%%%%%%%%%%%%%%%
\index{Amazon EC2 Query API}
\index{EC2 grid jobs}
\index{grid computing!submitting jobs using the EC2 Query API}
\index{grid type!ec2}

HTCondor jobs may be submitted to clouds supporting
Amazon's Elastic Compute Cloud (EC2) interface.
The EC2 interface permits on-line commercial services that provide
the rental of computers by the hour to run computational applications.
They run virtual machine images that have been uploaded to Amazon's
online storage service (S3 or EBS).
More information about Amazon's EC2 service is available at
\URL{http://aws.amazon.com/ec2}.

The \SubmitCmdNI{ec2} grid type uses the EC2 Query API,
also called the EC2 REST API.

%%%%%%%%%%%%%%%%%%%%%%%%%%%%%%%%%%%%%%%%%%%%%%%%%%%%%%%%%%%%%%%%%%%%%%%%%%%
\subsubsection{\label{sec:Amazon-submit}EC2 Job Submission}
%%%%%%%%%%%%%%%%%%%%%%%%%%%%%%%%%%%%%%%%%%%%%%%%%%%%%%%%%%%%%%%%%%%%%%%%%%%

HTCondor jobs are submitted to an EC2 service
with the \SubmitCmdNI{grid} universe, setting the
\SubmitCmd{grid\_resource} command to \SubmitCmdNI{ec2}, followed 
by the service's URL. For example,
partial contents of the submit description file may be
\begin{verbatim}
grid_resource = ec2 https://ec2.amazonaws.com/
\end{verbatim}

Since the job is a virtual machine image,
most of the submit description file commands
specifying input or output files are not applicable.
The \SubmitCmd{executable} command is still required,
but its value is ignored. 
It can be used to identify different jobs in the output of \Condor{q}.

The VM image for the job must already reside in one of Amazon's storage
service (S3 or EBS) and be registered with EC2.
In the submit description file,
provide the identifier for the image using \SubmitCmd{ec2\_ami\_id}.

\index{grid type!ec2!authentication methods}
This grid type requires access to user authentication information,
in the form of path names to files containing the appropriate keys.

The \SubmitCmdNI{ec2} grid type has two different authentication methods.
The first authentication method uses the EC2 API's built-in authentication.
Specify the service with expected \Expr{http://} or \Expr{https://} URL,
and set the EC2 access key and secret access key as follows:

\begin{verbatim}
ec2_access_key_id = /path/to/access.key
ec2_secret_access_key = /path/to/secret.key
\end{verbatim}

The \Expr{euca3://} and \Expr{euca3s://} protocols must use this 
authentication method.
These protocols exist to work correctly when the resources do not support
the \Param{InstanceInitiatedShutdownBehavior} parameter.

The second authentication method for the EC2 grid type is X.509.
Specify the service with an \Expr{x509://} URL, 
even if the URL was given in another form.  
Use \SubmitCmd{ec2\_access\_key\_id} to 
specify the path to the X.509 public key (certificate),
which is not the same as the built-in authentication's access key.
\SubmitCmd{ec2\_secret\_access\_key} specifies the path to the X.509 
private key,
which is not the same as the built-in authentication's secret key.
The following example illustrates the specification for X.509 authentication:

\begin{verbatim}
grid_resource = ec2 x509://service.example
ec2_access_key_id = /path/to/x.509/public.key
ec2_secret_access_key = /path/to/x.509/private.key
\end{verbatim}

If using an X.509 proxy, specify the proxy in both places.

HTCondor can use the EC2 API to create an SSH key pair that allows
secure log in to the virtual machine once it is running.
If the command
\SubmitCmd{ec2\_keypair\_file}
is set in the submit description file,
HTCondor will write an SSH private key into the indicated file.
The key can be used to log into the virtual machine.
Note that modification will also be needed of the firewall
rules for the job to incoming SSH connections.

An EC2 service uses a firewall to restrict network access to
the virtual machine instances it runs.
Typically, no incoming connections are allowed.
One can define sets of firewall rules and give them names.
The EC2 API calls these security groups.
If utilized, tell HTCondor what set of security
groups should be applied to each VM using the
\SubmitCmd{ec2\_security\_groups} submit description file command.
If not provided, HTCondor uses the security group \SubmitCmdNI{default}.
This command specifies security group names; to specify IDs, use
\SubmitCmd{ec2\_security\_ids}.  This may be necessary when specifying
a Virtual Private Cloud (VPC) instance.

To run an instance in a VPC, 
set \SubmitCmd{ec2\_vpc\_subnet} to the the desired VPC's specification string.
The instance's IP address 
may also be specified by setting \SubmitCmd{ec2\_vpc\_id}.

The EC2 API allows the choice of different hardware configurations 
for instances to run on.
Select which configuration to use for the \SubmitCmdNI{ec2} grid type
with the \SubmitCmd{ec2\_instance\_type} submit description file command.
HTCondor provides no default.

Certain instance types provide additional block devices whose names must be
mapped to kernel device names in order to be used.
The \SubmitCmd{ec2\_block\_device\_mapping} submit description file command
allows specification of these maps.  
A map is a device name followed by a
colon, followed by kernel name; 
maps are separated by a commas, and/or spaces.
For example, 
to specify that the first ephemeral device should be \Expr{/dev/sdb}
and the second \Expr{/dev/sdc}:

\begin{verbatim}
ec2_block_device_mapping = ephemeral0:/dev/sdb, ephemeral1:/dev/sdc
\end{verbatim}

Each virtual machine instance can be given up to 16 KiB of unique data, 
accessible by the instance by connecting to a well-known address.
This makes it easy for many instances to share the same VM image,
but perform different work.
This data can be specified to HTCondor in one of two ways.
First, the data can be provided directly in the submit description file 
using the \SubmitCmd{ec2\_user\_data} command.
Second, the data can be
stored in a file, and the file name is specified with the
\SubmitCmd{ec2\_user\_data\_file} submit description file command.
This second option allows the use of binary data.
If both options are used, the two blocks of
data are concatenated, with the data from \SubmitCmdNI{ec2\_user\_data} 
occurring first.  HTCondor performs the base64 encoding that EC2 expects on 
the data.

Amazon also offers an Identity and Access Management (IAM) service.
To specify an IAM (instance) profile for an EC2 job, 
use submit commands
\SubmitCmd{ec2\_iam\_profile\_name} or \SubmitCmd{ec2\_iam\_profile\_arn}.

%%%%%%%%%%%%%%%%%%%%%%%%%%%%%%%%%%%%%%%%%%%%%%%%%%%%%%%%%%%%%%%%%%%%%%%%%%%
\subsubsection{\label{sec:EC2-termination}Termination of EC2 Jobs}
%%%%%%%%%%%%%%%%%%%%%%%%%%%%%%%%%%%%%%%%%%%%%%%%%%%%%%%%%%%%%%%%%%%%%%%%%%%

A protocol defines the shutdown procedure for jobs running as
EC2 instances.
The service is told to shut down the instance,
and the service acknowledges.
The service then advances the instance to a state in which
the termination is imminent, but the job is given time to
shut down gracefully.

Once this state is reached, some services other than Amazon cannot be
relied upon to actually terminate the job.
Thus, HTCondor must check that the instance has terminated
before removing the job from the queue.
This avoids the possibility of HTCondor losing track of a job
while it is still accumulating charges on the service.  

HTCondor checks after a fixed time interval
that the job actually has terminated.
If the job has not terminated after a total of four checks,
the job is placed on hold.

%%%%%%%%%%%%%%%%%%%%%%%%%%%%%%%%%%%%%%%%%%%%%%%%%%%%%%%%%%%%%%%%%%%%%%%%%%%
\subsubsection{\label{sec:spot-instances}Using Spot Instances}
%%%%%%%%%%%%%%%%%%%%%%%%%%%%%%%%%%%%%%%%%%%%%%%%%%%%%%%%%%%%%%%%%%%%%%%%%%%

EC2 jobs may also be submitted to clouds that support spot instances.
A spot instance differs from a conventional, or dedicated, instance in two
primary ways.
First, the instance price varies according to demand.
Second,
the cloud provider may terminate the instance prematurely.
To start a spot instance,
the submitter specifies a bid,
which represents the most the submitter is willing to pay per hour
to run the VM.
\index{submit commands!ec2\_spot\_price}
Within HTCondor, the submit command \SubmitCmd{ec2\_spot\_price}
specifies this floating point value.
For example, 
to bid 1.1 cents per hour on Amazon:

\begin{verbatim}
ec2_spot_price = 0.011
\end{verbatim}

Note that the EC2 API does not specify how the cloud provider 
should interpret the bid.
Empirically, Amazon uses fractional US dollars.

Other submission details for a spot instance are identical to those
for a dedicated instance.

A spot instance will not necessarily begin immediately.
Instead, 
it will begin as soon as the price drops below the bid.
Thus, spot instance jobs
may remain in the idle state for much longer than dedicated instance jobs,
as they wait for the price to drop.
Furthermore, if the price rises above the bid, 
the cloud service will terminate the instance.

More information about Amazon's spot instances is available at
\URL{http://aws.amazon.com/ec2/spot-instances/}.

%%%%%%%%%%%%%%%%%%%%%%%%%%%%%%%%%%%%%%%%%%%%%%%%%%%%%%%%%%%%%%%%%%%%%%%%%%%
\subsubsection{\label{sec:Amazon-parameters}Advanced Usage}
%%%%%%%%%%%%%%%%%%%%%%%%%%%%%%%%%%%%%%%%%%%%%%%%%%%%%%%%%%%%%%%%%%%%%%%%%%%

Additional control of EC2 instances is available in the form
of permitting the
direct specification of instance creation parameters.  
To set an instance creation parameter, 
first list its name in the submit command \SubmitCmd{ec2\_parameter\_names},
a space or comma separated list.  
The parameter may need to be properly capitalized.  
Also tell HTCondor the parameter's value, 
by specifying it as a submit command whose name begins with
\SubmitCmdNI{ec2\_parameter\_}; dots within the parameter name must be
written as underscores in the submit command name.

For example, the submit description file commands
to set parameter \texttt{IamInstanceProfile.Name} 
to value \texttt{ExampleProfile} are

\begin{verbatim}
ec2_parameter_names = IamInstanceProfile.Name
ec2_parameter_IamInstanceProfile_Name = ExampleProfile
\end{verbatim}

%%%%%%%%%%%%%%%%%%%%%%%%%%%%%%%%%%%%%%%%%%%%%%%%%%%%%%%%%%%%%%%%%%%%%%%%%%%
\subsubsection{\label{sec:Amazon-config}EC2 Configuration Variables}
%%%%%%%%%%%%%%%%%%%%%%%%%%%%%%%%%%%%%%%%%%%%%%%%%%%%%%%%%%%%%%%%%%%%%%%%%%%

The configuration variables \MacroNI{EC2\_GAHP} and \MacroNI{EC2\_GAHP\_LOG}
must be set, and by default are equal to \verb@$(SBIN)/ec2_gahp@ and
\verb@/tmp/EC2GahpLog.$(USERNAME)@, respectively.

The configuration variable \MacroNI{EC2\_GAHP\_DEBUG} is optional and defaults
to \verb@D_PID@; we recommend you keep \verb@D_PID@ if you change the default,
to disambiguate between the logs of different resources specified by the same
user.

%%%%%%%%%%%%%%%%%%%%%%%%%%%%%%%%%%%%%%%%%%%%%%%%%%%%%%%%%%%%%%%%%%%%%%%%%%%
\subsubsection{\label{sec:Amazon-config-communication}Communicating with an EC2 Service}
%%%%%%%%%%%%%%%%%%%%%%%%%%%%%%%%%%%%%%%%%%%%%%%%%%%%%%%%%%%%%%%%%%%%%%%%%%%

The \SubmitCmdNI{ec2} grid type does not presently permit the explicit
use of an HTTP proxy.

By default, HTCondor assumes that EC2 services are reliably available.
If an attempt to contact a service during the normal course of operation fails,
HTCondor makes a special attempt to contact the service.
If this attempt fails, the service is marked as down,
and normal operation for that service is
suspended until a subsequent special attempt succeeds.
The jobs using that service do not go on hold.
To place jobs on hold when their service becomes unavailable,
set configuration variable \Macro{EC2\_RESOURCE\_TIMEOUT} to
the number of seconds to delay before placing the job on hold.
The default value of -1 for this variable implements
an infinite delay, such that the job is never placed on hold.
When setting this value, consider the value of configuration variable
\Macro{GRIDMANAGER\_RESOURCE\_PROBE\_INTERVAL},
which sets the number of seconds that HTCondor will wait after each
special contact attempt before trying again.

By default, the EC2 GAHP enforces a 100 millisecond interval between requests
to the same service.  This helps ensure reliable service.  You may configure
this interval with the configuration variable \MacroNI{EC2\_GAHP\_RATE\_LIMIT},
which must be an integer number of milliseconds.  Adjusting the interval may
result in higher or lower throughput, depending on the service.  Too short
of an interval may trigger rate-limiting by the service; while HTCondor will
react appropriately (by retrying with an exponential back-off), it may be
more efficient to configure a longer interval.

%%%%%%%%%%%%%%%%%%%%%%%%%%%%%%%%%%%%%%%%%%%%%%%%%%%%%%%%%%%%%%%%%%%%%%%%%%%
\subsubsection{\label{sec:Amazon-config-certs}Secure Communication with and EC2 Service}
%%%%%%%%%%%%%%%%%%%%%%%%%%%%%%%%%%%%%%%%%%%%%%%%%%%%%%%%%%%%%%%%%%%%%%%%%%%

The specification of a service with an \Expr{https://}, an \Expr{x509://},
or an \Expr{euca3s://} URL validates that service's certificate,
checking that a trusted certificate authority (CA) signed it.
Commercial EC2 service providers generally use certificates signed by
widely-recognized CAs.
These CAs will usually work without any additional configuration.
For other providers, a specification of trusted CAs may be needed.
Without, errors such as the following will be in the EC2 GAHP log:

\footnotesize
\begin{verbatim}
06/13/13 15:16:16 curl_easy_perform() failed (60):
'Peer certificate cannot be authenticated with given CA certificates'.
\end{verbatim}
\normalsize

Specify trusted CAs by including their certificates in a group of trusted CAs
either in an on disk directory or in a single file.
Either of these alternatives may contain multiple certificates.
Which is used will vary from system to system,
depending on the system's SSL implementation.
HTCondor uses \Prog{libcurl};
information about the \Prog{libcurl} specification of trusted CAs
is available at

\URL{http://curl.haxx.se/libcurl/c/curl\_easy\_setopt.html}

Versions of HTCondor with standard universe support ship with their
own \Prog{libcurl}, which will be linked against \Prog{OpenSSL}.

The behavior when specifying both a directory and a file is undefined,
although the EC2 GAHP allows it.

The EC2 GAHP will set the CA file to whichever variable it finds first,
checking these in the following order:

\begin{enumerate}
\item The environment variable \Env{X509\_CERT\_FILE},
  set when the \Condor{master} starts up.
\item The HTCondor configuration variable \Macro{GAHP\_SSL\_CAFILE}.
\end{enumerate}

The EC2 GAHP supplies no default value, if it does not find a CA file.

The EC2 GAHP will set the CA directory given whichever of these
variables it finds first, checking in the following order:

\begin{enumerate}
\item The HTCondor configuration variable \Macro{GSI\_DAEMON\_TRUSTED\_CA\_DIR}.
\item The environment variable \Env{X509\_CERT\_DIR},
  set when the \Condor{master} starts up.
\item The HTCondor configuration variable \Macro{GAHP\_SSL\_CADIR}.
\end{enumerate}

The EC2 GAHP supplies no default value, if it does not find a CA directory.

%%%%%%%%%%%%%%%%%%%%%%%%%%%%%%%%%%%%%%%%%%%%%%%%%%%%%%%%%%%%%%%%%%%%%%%%%%%
\subsubsection{\label{sec:Amazon-statistics}EC2 GAHP Statistics}
%%%%%%%%%%%%%%%%%%%%%%%%%%%%%%%%%%%%%%%%%%%%%%%%%%%%%%%%%%%%%%%%%%%%%%%%%%%

The EC2 GAHP tracks, and reports in the corresponding grid resource ad,
statistics related to resource's rate limit.

%% I'll skip the explanation of the difference between these and the
%% corresponding RecentX attributes until somebody actually cares.
%%
%% (The EC2 GAHP's statistics are collected on every batch status poll,
%% but bucketed according to the interval at which they are reported.
%% IIRC, there are four total buckets, but I'm not sure if the base
%% statistic only remembers those four buckets "back" or if there's
%% additional storage somewhere for the lifetime total.

\begin{description}

\index{EC2 GAHP Statistics!NumRequests}
\index{NumRequests!EC2 GAHP Statistics}
\item[\AdAttr{NumRequests}:]
The total number of requests made by HTCondor to this resource.

\index{EC2 GAHP Statistics!NumDistinctRequests}
\index{NumDistinctRequests!EC2 GAHP Statistics}
\item[\AdAttr{NumDistinctRequests}:]
The number of distinct requests made by HTCondor to this
resource.  The difference between this and NumRequests is
the total number of retries.  Retries are not unusual.

\index{EC2 GAHP Statistics!NumRequestsExceedingLimit}
\index{NumRequestsExceedingLimit!EC2 GAHP Statistics}
\item[\AdAttr{NumRequestsExceedingLimit}:]
The number of requests which exceeded the service's
rate limit.  Each such request will cause a retry, unless the maximum
number of retries is exceeded, or if the retries have already taken so
long that the signature on the original request has expired.

\index{EC2 GAHP Statistics!NumExpiredSignatures}
\index{NumExpiredSignatures!EC2 GAHP Statistics}
\item[\AdAttr{NumExpiredSignatures}:]
The number of requests which the EC2 GAHP did not even attempt
to send to the service because signature expired.  Signatures should not,
generally, expire; a request's retries will usually -- eventually -- succeed.

\end{description}
