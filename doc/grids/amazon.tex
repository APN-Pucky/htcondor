%%%%%%%%%%%%%%%%%%%%%%%%%%%%%%%%%%%%%%%%%%%%%%%%%%%%%%%%%%%%%%%%%%%%%%%%%%%
\subsection{\label{sec:Amazon}The amazon Grid Type }
%%%%%%%%%%%%%%%%%%%%%%%%%%%%%%%%%%%%%%%%%%%%%%%%%%%%%%%%%%%%%%%%%%%%%%%%%%%
\index{Amazon EC2}
\index{grid computing!submitting jobs to Amazon EC2}

Condor jobs may be submitted to Amazon's Elastic Compute Cloud (EC2)
service.
EC2 is an on-line commercial service that allows you to rent computers
by the hour to run computational applications.
EC2 runs virtual machine images that have been uploaded to Amazon's
online storage service (S3).
See the Amazon EC2 webpage at \URL{http://aws.amazon.com/ec2} for more
information about EC2.

%%%%%%%%%%%%%%%%%%%%%%%%%%%%%%%%%%%%%%%%%%%%%%%%%%%%%%%%%%%%%%%%%%%%%%%%%%%
\subsubsection{\label{sec:Amazon-submit}Amazon EC2 Job Submision}
%%%%%%%%%%%%%%%%%%%%%%%%%%%%%%%%%%%%%%%%%%%%%%%%%%%%%%%%%%%%%%%%%%%%%%%%%%%

Condor jobs are submitted to EC2
using the \SubmitCmd{grid} universe and the
\SubmitCmd{grid\_resource} command  by placing the following
into the submit description file.
\begin{verbatim}
grid_resource = amazon
\end{verbatim}

Since the job is a virtual machine image, most of the submit file
attributes specifying input or output files aren't applicable. The
\SubmitCmd{executable} attribute is still required, but its value is
ignored. You can use it to identify different jobs in the output of
\Condor{q}.

The VM image for the job must already reside in Amazon's storage
service (S3) and be registered with EC2. In the submit file, you
must provide the identifier for the image using the
\SubmitCmd{amazon\_ami\_id} attribute.

You must also provide the files containing the X509 certificate and
private key used to authenticate you with the EC2 service:

\begin{verbatim}
amazon_public_key = /path/to/x509/cert
amazon_private_key = /path/to/private/key
\end{verbatim}

Condor and EC2 can create an ssh keypair to allow you to securely log
into the virtual machine once it's running. If you set
\SubmitCmd{amazon\_keypair\_file} in the submit file, Condor will write
an ssh private key into the indicated file. The key can be used to log
into the virtual machine. Note that you'll also need to modify the firewall
rules for the job to all incoming ssh connections.

EC2 uses a firewall to restrict network access to the virtual machine
instances it runs. By default, no incoming connections are allowed.
You can define sets of firewall rules and give them names. EC2 calls
these security groups. You can then tell Condor what set of security
groups should be applied to each VM using the
\SubmitCmd{amazon\_security\_groups} attribute. If you don't provide
this attribute, Condor uses the security group \SubmitCmd{default}.

EC2 offers several hardware configurations for instances to run on, with
varying prices. You can select which configuration to use with the
\SubmitCmd{amazon\_instance\_type} attribute. The default value is
\SubmitCmd{m1.small}.
% Other values: m1.large m1.xlarge c1.medium c1.xlarge

Each virtual machine instance can be given up 16kB of unique data, 
accessible by the instance by connecting to a well-known address. This
makes it easy for many instances to share the same VM image but perform
different work. This data can be specified to Condor in one of two ways.
First, the data can be provided directly in the submit file using the
\SubmitCmd{amazon\_user\_data} attribute. Second, the data can be
stored in a file, and the filename specified with the
\SubmitCmd{amazon\_user\_data\_file} attribute. The second option allows
you to use binary data. Condor performs the base64 encoding that EC2
expects on the data.

%%%%%%%%%%%%%%%%%%%%%%%%%%%%%%%%%%%%%%%%%%%%%%%%%%%%%%%%%%%%%%%%%%%%%%%%%%%
\subsubsection{\label{sec:Amazon-config}Amazon EC2 Configuration Parameters}
%%%%%%%%%%%%%%%%%%%%%%%%%%%%%%%%%%%%%%%%%%%%%%%%%%%%%%%%%%%%%%%%%%%%%%%%%%%

The amazon grid type requires several parameters to be set in the Condor
configuration file:

\footnotesize
\begin{verbatim}
AMAZON_GAHP=$(SBIN)/amazon-gahp
AMAZON_GAHP_LOG=/tmp/AmazonGahpLog.$(USERNAME)
\end{verbatim}
\normalsize

You can alter the URL used by Condor to contact the EC2 service using
the \Macro{AMAZON\_EC2\_URL} parameter. The default value is
\Macro{https://ec2.amazonaws.com/}.

If you need to use an HTTP proxy to reach EC2, you tell Condor to use it
via the \Macro{AMAZON\_HTTP\_PROXY} paramater.
