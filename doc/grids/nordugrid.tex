
%%%%%%%%%%%%%%%%%%%%%%%%%%%%%%%%%%%%%%%%%%%%%%%%%%%%%%%%%%%%%%%%%%%%%%%%%%%
\subsection{\label{sec:NorduGrid}The nordugrid Grid Type }
%%%%%%%%%%%%%%%%%%%%%%%%%%%%%%%%%%%%%%%%%%%%%%%%%%%%%%%%%%%%%%%%%%%%%%%%%%%
\index{NorduGrid}
\index{grid computing!submitting jobs to NorduGrid}

NorduGrid is a project to develop free grid middleware named
the Advanced  Resource Connector (ARC).
See the NorduGrid web page (\URL{http://www.nordugrid.org})
for more information about NorduGrid software.

HTCondor jobs may be submitted to
NorduGrid resources using the \SubmitCmdNI{grid} universe.
The \SubmitCmd{grid\_resource} command specifies the name of the
NorduGrid resource as follows:
\begin{verbatim}
grid_resource = nordugrid ng.example.com
\end{verbatim}

NorduGrid uses X.509 credentials for authentication,
usually in the form a proxy certificate. 
\Condor{submit} looks in default locations for the proxy. 
The submit description file command \SubmitCmd{x509userproxy}
may be used to give the full path name to the directory containing the proxy,
when the proxy is not in a default location.
If this optional command is not present in the submit description file,
then the value of the environment variable
\Env{X509\_USER\_PROXY} is checked for the location of the proxy.
If this environment variable is not present, then 
the proxy in the file
\File{/tmp/x509up\_uXXXX} is used,
where the characters \verb@XXXX@ in this file name are
replaced with the Unix user id.

NorduGrid uses RSL syntax to describe jobs.
The submit description file command
\SubmitCmd{nordugrid\_rsl}
adds additional attributes to the job RSL that HTCondor
constructs. 
The format this submit description file command is
\begin{verbatim}
nordugrid_rsl = (name=value)(name=value)
\end{verbatim}
