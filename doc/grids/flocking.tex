%%%%%%%%%%%%%%%%%%%%%%%%%%%%%%%%%%%%%%%%%%%%%%%%%%%%%%%%%%%%%%%%%%%%%%
\section{\label{sec:Flocking}Connecting HTCondor Pools with Flocking}
%%%%%%%%%%%%%%%%%%%%%%%%%%%%%%%%%%%%%%%%%%%%%%%%%%%%%%%%%%%%%%%%%%%%%%
\index{flocking}
\index{HTCondor!flocking}

Flocking is HTCondor's way of allowing jobs that cannot immediately
run within the pool of machines where the job was
submitted to instead run on a different HTCondor pool. 
If a machine within HTCondor pool A can send jobs to be run on HTCondor pool B,
then we say that jobs from machine A flock to pool B.
Flocking can occur in a one way manner,
such as jobs from machine A flocking to pool B,
or it can be set up to flock in both directions. 
Configuration variables allow the
\Condor{schedd} daemon (which runs on each machine
that may submit jobs) to implement flocking.

\Note Flocking to pools which use HTCondor's high availability mechanisms
is not advised.
See section ~\ref{sec:HA-negotiator} 
for a discussion of the issues.


%%%%%%%%%%%%%%%%%%%%%%%%%%%%%%%%%%%%%%%%%%%%%%%%%%%%%%%%%%%%%%%%%%%%%%
\subsection{\label{sec:Configure-Flocking}Flocking Configuration}
%%%%%%%%%%%%%%%%%%%%%%%%%%%%%%%%%%%%%%%%%%%%%%%%%%%%%%%%%%%%%%%%%%%%%%

\index{configuration!for flocking}
The simplest flocking configuration sets
a few configuration variables.
If jobs from machine A are to flock to pool B, 
then in machine A's configuration,
set the following configuration variables:
\begin{description}
  \item[\Macro{FLOCK\_TO}] is a comma separated list of the central
  manager machines of the pools that jobs from machine A may flock to.
  \item[\Macro{FLOCK\_COLLECTOR\_HOSTS}]
  is the list of \Condor{collector} daemons within the pools
  that jobs from machine A may flock to.
  In most cases, it is the same as \MacroNI{FLOCK\_TO}, and
  it would be defined with 
  \begin{verbatim}
  FLOCK_COLLECTOR_HOSTS = $(FLOCK_TO)
  \end{verbatim}
  \item[\Macro{FLOCK\_NEGOTIATOR\_HOSTS}]
  is the list of \Condor{negotiator} daemons within the pools
  that jobs from machine A may flock to.
  In most cases, it is the same as \MacroNI{FLOCK\_TO}, and
  it would be defined with 
  \begin{verbatim}
  FLOCK_NEGOTIATOR_HOSTS = $(FLOCK_TO)
  \end{verbatim}
  \item[\Macro{HOSTALLOW\_NEGOTIATOR\_SCHEDD}]
  provides an access level and authorization list for the
  \Condor{schedd} daemon to allow negotiation (for security
  reasons) with the machines within the pools
  that jobs from machine A may flock to.
  This configuration variable will not likely need to change
  from its default value as given in the sample configuration:
  \footnotesize
  \begin{verbatim}
  ##  Now, with flocking we need to let the SCHEDD trust the other
  ##  negotiators we are flocking with as well.  You should normally
  ##  not have to change this either.
ALLOW_NEGOTIATOR_SCHEDD = $(CONDOR_HOST), $(FLOCK_NEGOTIATOR_HOSTS), $(IP_ADDRESS)
  \end{verbatim}
  \normalsize
  % NEGOTIATOR_HOST is no longer used since ports are dynamically
  % allocated, so suggest COLLECTOR_HOST instead
  This example configuration presumes that the \Condor{collector}
  and \Condor{negotiator} daemons are running on the same machine.
  See 
  section~\ref{sec:Security-Authorization} on
  page~\pageref{sec:Security-Authorization} for a discussion
  of security macros and their use.
\end{description}

The configuration macros that must be set in 
pool B are ones that authorize jobs from machine A
to flock to pool B.

The configuration variables are more easily
set by introducing a list of machines where the jobs may flock from. 
\Macro{FLOCK\_FROM} is a comma separated list of machines,
and  it is used in the default configuration setting
of the security macros that do authorization:
\footnotesize
\begin{verbatim}
ALLOW_WRITE_COLLECTOR = $(ALLOW_WRITE), $(FLOCK_FROM)
ALLOW_WRITE_STARTD    = $(ALLOW_WRITE), $(FLOCK_FROM)
ALLOW_READ_COLLECTOR  = $(ALLOW_READ), $(FLOCK_FROM)
ALLOW_READ_STARTD     = $(ALLOW_READ), $(FLOCK_FROM)
\end{verbatim}
\normalsize

Wild cards may be used when setting the \MacroNI{FLOCK\_FROM}
configuration variable.
For example, \verb@*.cs.wisc.edu@ specifies all hosts
from the \verb@cs.wisc.edu@ domain. 

Further, if using Kerberos or GSI authentication, then the setting
becomes:
\footnotesize
\begin{verbatim}
ALLOW_NEGOTIATOR = condor@$(UID_DOMAIN)/$(COLLECTOR_HOST)
\end{verbatim}
\normalsize

To enable flocking in both directions, consider each direction
separately, following the guidelines given.

%%%%%%%%%%%%%%%%%%%%%%%%%%%%%%%%%%%%%%%%%%%%%%%%%%%%%%%%%%%%%%%%%%%%%%
\subsection{\label{sec:Jobs-Flocking}Job Considerations}
%%%%%%%%%%%%%%%%%%%%%%%%%%%%%%%%%%%%%%%%%%%%%%%%%%%%%%%%%%%%%%%%%%%%%%

A particular job will only flock to another pool
when it cannot currently run in the current pool.

The submission of jobs other than standard universe jobs must consider
the location of input, output and error files.
The common case will be that machines within separate pools
do not have a shared file system.
Therefore, when submitting jobs, the user will need to enable
file transfer mechanisms.
These mechanisms are discussed in
section~\ref{sec:file-transfer} on page~\pageref{sec:file-transfer}.
