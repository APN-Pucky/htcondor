%%%%%%%%%%%%%%%%%%%%%%%%%%%%%%%%%%%%%%%%%%%%%%%%%%%%%%%%%%%%%%%%%%%%%%
\section{\label{sec:Flocking}Flocking}
%%%%%%%%%%%%%%%%%%%%%%%%%%%%%%%%%%%%%%%%%%%%%%%%%%%%%%%%%%%%%%%%%%%%%%
\index{flocking}
\index{Condor!flocking}

Flocking is Condor's way of allowing jobs that cannot immediately
run (within the pool of machines where the job was
submitted) to instead run on a different Condor pool. 
If Condor pool A can send jobs to be run on Condor pool B,
then we say that pool A flocks to pool B.
Flocking can be one way,
such as pool A flocking to pool B,
or it can be set up to flock in both directions. 
Configuration variables allow the
\Condor{schedd} daemon to implement flocking.

\index{configuration!for flocking}
The simplest flocking configuration sets
a few configuration variables.
If pool A is to flock to pool B, 
then in pool A's configuration,
set the following configuration variables:
\begin{description}
  \item[\Macro{FLOCK\_TO}] is a comma separated list of the central
  manager machines of the pools that pool A may flock to.
  \item[\Macro{FLOCK\_COLLECTOR\_HOSTS}]
  is the list of \Condor{collector} deamons within the pools
  that pool A may flock to.
  In most cases, it is the same as \MacroNI{FLOCK\_TO}, and
  it would be defined with 
  \begin{verbatim}
  FLOCK_COLLECTOR_HOSTS = $(FLOCK_TO)
  \end{verbatim}
  \item[\Macro{FLOCK\_NEGOTIATOR\_HOSTS}]
  is the list of \Condor{negotiator} deamons within the pools
  that pool A may flock to.
  In most cases, it is the same as \MacroNI{FLOCK\_TO}, and
  it would be defined with 
  \begin{verbatim}
  FLOCK_COLLECTOR_HOSTS = $(FLOCK_TO)
  \end{verbatim}
  \item[\Macro{HOSTALLOW\_NEGOTIATOR\_SCHEDD}]
  provides a host-based access level and authorization list for the
  \Condor{schedd} daemon to allow negotiation (for security
  reasons) with the machines within the pools
  that pool A may flock to.
  This configuration variable will not likely need to change
  from its default value as given in the sample configuration:
  \footnotesize
  \begin{verbatim}
  ##  Now, with flocking we need to let the SCHEDD trust the other
  ##  negotiators we are flocking with as well.  You should normally
  ##  not have to change this either.
  HOSTALLOW_NEGOTIATOR_SCHEDD = $(NEGOTIATOR_HOST), $(FLOCK_NEGOTIATOR_HOSTS)
  \end{verbatim}
  \normalsize
  See 
  section~\ref{sec:Security-Authorization} on
  page~\pageref{sec:Security-Authorization} for a discussion
  of security macros and their use.
\end{description}

The configuration macros that must be set in 
pool B are ones that authorize jobs from pool A
to flock to pool B.

The host-based configuration macros are more easily
set by introducing a list of pools where the jobs may flock from. 
\Macro{FLOCK\_FROM} is a comma separated list of machines,
and  it is used in the default configuration setting
of the security macros that do host-based authorization:
\footnotesize
\begin{verbatim}
HOSTALLOW_WRITE_COLLECTOR = $(HOSTALLOW_WRITE), $(FLOCK_FROM)
HOSTALLOW_WRITE_STARTD    = $(HOSTALLOW_WRITE), $(FLOCK_FROM)
HOSTALLOW_READ_COLLECTOR  = $(HOSTALLOW_READ), $(FLOCK_FROM)
HOSTALLOW_READ_STARTD     = $(HOSTALLOW_READ), $(FLOCK_FROM)
\end{verbatim}
\normalsize

Wildcards may be used when setting the \MacroNI{FLOCK\_FROM}
configuration variable.
For example, \verb@*.cs.wisc.edu@ specifies all hosts
from the \verb@cs.wisc.edu@ domain. 

The user-based configuration macros are used, then the default
will be:
\footnotesize
\begin{verbatim}
ALLOW_NEGOTIATIOR =  $(NEGOTIATOR_HOST), $(FLOCK_NEGOTIATOR_HOSTS)
\end{verbatim}
\normalsize

Further, if using Kerberos or GSI authentication, then the setting
becomes:
\footnotesize
\begin{verbatim}
ALLOW_NEGOTIATOR = condor@$(UID_DOMAIN)/$(NEGOTIATOR_HOST)
\end{verbatim}
\normalsize

To enable flocking in both directions, consider each direction
separately, following the guidelines given.

A particular job will only flock to another pool
when it cannot currently run in the current pool.

% no longer true:
%Jobs
%that are run in another pool can only be standard universe jobs, and
%they are run as user ``nobody''.

