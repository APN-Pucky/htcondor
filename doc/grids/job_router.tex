%%%%%%%%%%%%%%%%%%%%%%%%%%%%%%%%%%%%%%%%%%%%%%%%%%%%%%%%%%%%%%%%%%%%%%%%%%%%%%%%
\section{\label{sec:JobRouter}The Condor Job Router}
%%%%%%%%%%%%%%%%%%%%%%%%%%%%%%%%%%%%%%%%%%%%%%%%%%%%%%%%%%%%%%%%%%%%%%%%%%%%%%%%
\index{Job Router}
\index{Condor daemon!condor\_job\_router@\Condor{job\_router}}
\index{daemon!condor\_job\_router@\Condor{job\_router}}
\index{condor\_job\_router daemon}

The Condor Job Router is an add-on to the \Condor{schedd} that transforms
jobs from one type into another according to a configurable policy.
This process of transforming the jobs is called \emph{job routing}.

One example of how the Job Router can be used is for the task of sending
excess jobs to one or more remote grid sites.
The Job Router can transform the jobs such as vanilla universe jobs into grid universe
jobs that use any of the grid types supported by Condor.  The rate at
which jobs are routed can be matched roughly to the rate at which the
site is able to start running them.  This makes it possible to balance
a large work flow across multiple grid sites, a local Condor pool, and
any flocked Condor pools, without having to guess in advance how quickly
jobs will run and complete in each of the different sites.

Job Routing is most appropriate for high throughput work flows, 
where there are many more jobs than computers,
and the goal is to keep as many of the computers busy as possible.
Job Routing is less suitable when there are a small number of jobs,
and the scheduler needs to choose the best place for each job,
in order to finish them as quickly as possible.
The Job Router does not know which site will run the jobs faster,
but it can decide whether to send more jobs to a site,
based on whether jobs already submitted to that site are
sitting idle or not, 
as well as whether the site has experienced recent job failures.

%%%%%%%%%%%%%%%%%%%%%%%%%%%%%%%%%%%%%%%%%%%%%%%%%%%%%%%%%%%%%%%%%%%%%%%%%%%%%%%%
\subsection{\label{sec:RouterMechanism}Routing Mechanism}
%%%%%%%%%%%%%%%%%%%%%%%%%%%%%%%%%%%%%%%%%%%%%%%%%%%%%%%%%%%%%%%%%%%%%%%%%%%%%%%%

The \Condor{job\_router} daemon and configuration determine a policy
for which jobs may be transformed and sent to 
grid sites.
By default, a job is transformed into a grid universe job
by making a copy of the original job ClassAd, and
modifying some attributes in this copy of the job.
The copy is called the routed copy,
and it shows up in the job queue under a new job id.

Until the routed copy finishes or is removed,
the original copy of the job passively mirrors the state of the routed job.
During this time,
the original job is not available for matchmaking,
because it is tied to the routed copy.
The original jobs also does not evaluate periodic expressions,
such as \Attr{PeriodicHold}.
Periodic expressions are evaluated for the routed copy.
When the routed copy completes,
the original job ClassAd is updated such that it reflects the
final status of the job.
If the routed copy is removed,
the original job returns to the normal idle state,
and is available for matchmaking or rerouting.
If, instead, the original job is removed or goes on hold,
the routed copy is removed.

Although the default mode routes vanilla universe jobs to
grid universe jobs, the routing rules may be configured
to do some other transformation of the job.  It is also
possible to edit the job in place rather than creating
a new transformed version of the job.

The \Condor{job\_router} daemon utilizes a \Term{routing table},
in which a ClassAd describes each site to where jobs may be sent.
The routing table is given in the New ClassAd language,
as currently used by Condor internally.
A good place to learn about the syntax of New ClassAds
is the Informal Language Description in the C++ ClassAds
tutorial: \URL{http://www.cs.wisc.edu/condor/classad/c++tut.html}.

Two essential differences distinguish the New ClassAd language
from the current one.
In the New ClassAd language,
each ClassAd is surrounded by square brackets.
And, in the New ClassAd language,
each assignment statement ends with a semicolon.
When the New
ClassAd is embedded in a Condor configuration file,
it may appear all on a single line,
but the readability is often improved by inserting line continuation
characters
after each assignment statement.
This is done in the examples.
Unfortunately, this makes the insertion of comments into
the configuration file awkward,
because of the interaction between comments and line continuation
characters in configuration files.
An alternative is to use C-style comments (\Code{/* \Dots */}).
Another alternative is to read in the routing table entries
from a separate file,
rather than embedding them in the Condor configuration file.

%%%%%%%%%%%%%%%%%%%%%%%%%%%%%%%%%%%%%%%%%%%%%%%%%%%%%%%%%%%%%%%%%%%%%%%%%%%%%%%%
\subsection{\label{sec:RouterJobSubmission}Job Submission with Job Routing Capability}
%%%%%%%%%%%%%%%%%%%%%%%%%%%%%%%%%%%%%%%%%%%%%%%%%%%%%%%%%%%%%%%%%%%%%%%%%%%%%%%%

If Job Routing is set up, then the following items
ought to be considered for jobs to have the
necessary prerequisites to be considered for routing.

\begin{itemize}

\item Jobs appropriate for routing to the grid must not rely on access to
a shared file system, or other services that are only available on the
local pool.
The job will use Condor's file transfer mechanism, 
rather than relying on a shared file system
to access input files and write output files.
In the submit description file, to enable file transfer, there
will be a set of commands similar to

\begin{verbatim}
should_transfer_files = YES
when_to_transfer_output = ON_EXIT
transfer_input_files = input1, input2
transfer_output_files = output1, output2
\end{verbatim}

Vanilla universe jobs and most types of grid universe jobs differ in the
set of files transferred back when the job completes.
Vanilla universe jobs transfer back all files created or modified,
while all grid universe jobs except for Condor-C only transfer back the \SubmitCmd{output} file,
as well as those explicitly listed
with \SubmitCmd{transfer\_output\_files}.
Therefore, when routing jobs to grid universes other than Condor-C, it is
important to explicitly specify all
output files that must be transferred upon job completion.

An additional difference between the vanilla universe jobs
and \SubmitCmd{gt2} grid universe jobs
is that \SubmitCmd{gt2} jobs do not return
any information about the job's exit status.
The exit status as reported in the job ClassAd and user log are
always 0.
Therefore, jobs that may be routed to a \SubmitCmd{gt2} grid site
must not rely upon a non-zero job exit status.

\item One configuration for routed jobs requires the jobs to
identify themselves as candidates for Job Routing.
This may be accomplished by inventing a ClassAd attribute
that the configuration utilizes in setting the policy 
for job identification,
and the job defines this attribute to identify itself.
If the invented attribute is called \Attr{WantJobRouter},
then the job identifies itself as a job that may be routed
by placing in the submit description file:

\begin{verbatim}
+WantJobRouter = True
\end{verbatim}

This implementation can be taken further,
allowing the job to first be rejected within the local pool,
before being a candidate for Job Routing:

\begin{verbatim}
+WantJobRouter = LastRejMatchTime =!= UNDEFINED
\end{verbatim}

\item As appropriate to the potential grid site,
create a grid proxy, and specify it in the submit description file:

\begin{verbatim}
x509userproxy = /tmp/x509up_u275
\end{verbatim}

This is not necessary if the \Condor{job\_router} daemon is configured
to add a grid proxy on behalf of jobs.

\end{itemize}

Job submission does not change for jobs that may be routed.

\begin{verbatim}
  $ condor_submit job1.sub
\end{verbatim}

where \File{job1.sub} might contain:

\begin{verbatim}
universe = vanilla
executable = my_executable
output = job1.stdout
error = job1.stderr
log = job1.ulog
should_transfer_files = YES
when_to_transfer_output = ON_EXIT
+WantJobRouter = LastRejMatchTime =!= UNDEFINED
x509userproxy = /tmp/x509up_u275
queue
\end{verbatim}

The status of the job may be observed as with any other Condor job,
for example by looking in the job's log file.
Before the job completes,
\Condor{q} shows the job's status.
Should the job become routed,
a second job will enter the job queue.
This is the routed copy of the original job.
The command \Condor{router\_q} shows a more specialized view of routed jobs, 
as this example shows:

\begin{verbatim}
$ condor_router_q -S
   JOBS ST Route      GridResource
     40  I Site1      site1.edu/jobmanager-condor
     10  I Site2      site2.edu/jobmanager-pbs
      2  R Site3      condor submit.site3.edu condor.site3.edu
\end{verbatim}

\Condor{router\_history} summarizes the history of routed jobs,
as this example shows:

\begin{verbatim}
$ condor_router_history
Routed job history from 2007-06-27 23:38 to 2007-06-28 23:38

Site            Hours    Jobs    Runs
                      Completed Aborted
-------------------------------------------------------
Site1              10       2     0
Site2               8       2     1
Site3              40       6     0
-------------------------------------------------------
TOTAL              58      10     1
\end{verbatim}


%%%%%%%%%%%%%%%%%%%%%%%%%%%%%%%%%%%%%%%%%%%%%%%%%%%%%%%%%%%%%%%%%%%%%%%%%%%%%%%
\subsection{\label{ExampleJobRouterConfiguration} An Example Configuration}
%%%%%%%%%%%%%%%%%%%%%%%%%%%%%%%%%%%%%%%%%%%%%%%%%%%%%%%%%%%%%%%%%%%%%%%%%%%%%%%

The following sample configuration sets up potential job routing
to three routes (grid sites).
Definitions of the configuration variables specific to the Job Router
are in section~ \ref{sec:JobRouter-Config-File-Entries}.
One route is a Condor site accessed via the Globus gt2 protocol.
A second route is a PBS site, also accessed via Globus gt2.
The third site is a Condor site accessed by Condor-C.
The \Condor{job\_router} daemon
does not know which site will be best for a given job.
The policy implemented in this sample configuration 
stops sending more jobs to a site,
if ten jobs that have already been sent to that site are idle.

These configuration settings belong in the local configuration file
of the machine where jobs are submitted.
Check that the machine can successfully submit grid jobs
before setting up and using the Job Router.
Typically, the single required element that needs to be
added for GSI authentication
is an X.509 trusted certification authority directory,
in a place recognized by Condor
(for example,  \File{/etc/grid-security/certificates}).
The VDT (\URL{http://vdt.cs.wisc.edu}) project provides 
a convenient way to set up and install a trusted CAs,
if needed.

\footnotesize
\begin{verbatim}

# These settings become the default settings for all routes
JOB_ROUTER_DEFAULTS = \
  [ \
    requirements=target.WantJobRouter is True; \
    MaxIdleJobs = 10; \
    MaxJobs = 200; \
\
    /* now modify routed job attributes */ \
    /* remove routed job if it goes on hold or stays idle for over 6 hours */ \
    set_PeriodicRemove = JobStatus == 5 || \
                        (JobStatus == 1 && (CurrentTime - QDate) > 3600*6); \
    delete_WantJobRouter = true; \
    set_requirements = true; \
  ]

# This could be made an attribute of the job, rather than being hard-coded
ROUTED_JOB_MAX_TIME = 1440

# Now we define each of the routes to send jobs on
JOB_ROUTER_ENTRIES = \
   [ GridResource = "gt2 site1.edu/jobmanager-condor"; \
     name = "Site 1"; \
   ] \
   [ GridResource = "gt2 site2.edu/jobmanager-pbs"; \
     name = "Site 2"; \
     set_GlobusRSL = "(maxwalltime=$(ROUTED_JOB_MAX_TIME))(jobType=single)"; \
   ] \
   [ GridResource = "condor submit.site3.edu condor.site3.edu"; \
     name = "Site 3"; \
     set_remote_jobuniverse = 5; \
   ]


# Reminder: you must restart Condor for changes to DAEMON_LIST to take effect.
DAEMON_LIST = $(DAEMON_LIST) JOB_ROUTER

# For testing, set this to a small value to speed things up.
# Once you are running at large scale, set it to a higher value
# to prevent the JobRouter from using too much cpu.
JOB_ROUTER_POLLING_PERIOD = 10

#It is good to save lots of schedd queue history
#for use with the router_history command.
MAX_HISTORY_ROTATIONS = 20
\end{verbatim}
\normalsize


%Some questions you may have after reading the above policy: Can the
%routing table be dynamically generated from grid information systems?
%Do users have to have their own grid credentials or can the \Condor{job\_router} daemon
%insert service credentials for them?  What's up with the syntax of the
%routing table: C-style comments, strange ClassAd expressions, escaped
%end of lines?  The next section covers the specifics of the \Condor{job\_router} daemon
%configuration.  Read on!

%%%%%%%%%%%%%%%%%%%%%%%%%%%%%%%%%%%%%%%%%%%%%%%%%%%%%%%%%%%%%%%%%%%%%%%%%%%%%%%%
\subsection{\label{RoutingTableAttributes} Routing Table Entry ClassAd Attributes}
%%%%%%%%%%%%%%%%%%%%%%%%%%%%%%%%%%%%%%%%%%%%%%%%%%%%%%%%%%%%%%%%%%%%%%%%%%%%%%%%

The conversion of a job to a routed copy may require the
job ClassAd to be modified.
The Routing Table specifies attributes of the different possible
routes and it may specify specific modifications that should be made
to the job when it is sent along a specific route.  In addition
to this mechanism for transforming the job, external programs may be
invoked to transform the job.  For more information, see
section~\ref{sec:job-hooks-JR}.

The following attributes and instructions for modifying job attributes
may appear in a Routing Table entry.

\begin{description}

\index{Job Router Routing Table ClassAd attribute!GridResource}
\item[GridResource] Specifies the value for the \Attr{GridResource}
attribute that will be inserted into the routed copy of the job's ClassAd.

\index{Job Router Routing Table ClassAd attribute!Name}
\item[Name] An optional identifier that will be used in log
messages concerning this route.  If no name is specified, the default
used will be the value of \Attr{GridResource}.
The \Condor{job\_router} distinguishes routes and advertises
statistics based on this attribute's value.

\index{Job Router Routing Table ClassAd attribute!Requirements}
\item[Requirements] A \Attr{Requirements} expression
that identifies jobs that may be matched to the route.  Note
that, as with all settings, requirements specified in
the configuration variable
\MacroNI{JOB\_ROUTER\_ENTRIES} override the setting of
\MacroNI{JOB\_ROUTER\_DEFAULTS}.  To specify global requirements that
are not overridden by \MacroNI{JOB\_ROUTER\_ENTRIES}, use
\MacroNI{JOB\_ROUTER\_SOURCE\_JOB\_CONSTRAINT}.

\index{Job Router Routing Table ClassAd attribute!MaxJobs}
\item[MaxJobs] An integer maximum number of jobs permitted on the route at
one time. The default is 100.

\index{Job Router Routing Table ClassAd attribute!MaxIdleJobs}
\item[MaxIdleJobs] An integer maximum number of routed jobs in the
idle state.  At or above this value, no more jobs will be sent
to this site.
This is intended to prevent too many jobs from being sent to sites
which are too busy to run them.
If the value set for this attribute is too small,
the rate of job submission to the site will slow,
because the \Condor{job\_router} daemon will submit jobs up to this limit,
wait to see some of the jobs enter the running state,
and then submit more.
The disadvantage of setting this attribute's value too high
is that a lot of jobs may be sent
to a site, only to site idle for hours or days.
The default value is 50.

\index{Job Router Routing Table ClassAd attribute!FailureRateThreshold}
\item[FailureRateThreshold] A maximum tolerated rate of job failures.
Failure is determined by the expression sets for 
the attribute \Attr{JobFailureTest} expression.
The default threshold is 0.03 jobs/second.
If the threshold is exceeded,
submission of new jobs is throttled until jobs begin succeeding,
such that the failure rate is less than the threshold.
This attribute implements \Term{black hole throttling},
such that a site at which jobs are sent only to fail (a black hole)
receives fewer jobs.

\index{Job Router Routing Table ClassAd attribute!JobFailureTest}
\item[JobFailureTest] An expression
evaluated for each job that finishes,
to determine whether it was a failure.
The default value if no expression is defined
assumes all jobs are successful.
Routed jobs that are removed are considered to be failures.
An example expression to treat all jobs running for less than 30 minutes as
failures is \Expr{target.RemoteWallClockTime < 1800}.  A more flexible
expression might reference a property or expression of the job that
specifies a failure condition specific to the type of job.

\index{Job Router Routing Table ClassAd attribute!TargetUniverse}
\item[TargetUniverse] An integer value specifying the desired
universe for the routed copy of the job.  The default value is 9, 
which is the \SubmitCmd{grid} universe.

\index{Job Router Routing Table ClassAd attribute!UseSharedX509UserProxy}
\item[UseSharedX509UserProxy] A boolean expression
that when \Expr{True} causes the value of \Attr{SharedX509UserProxy}
to be the X.509 user proxy for the routed job.
Note that if the \Condor{job\_router} daemon is running as root,
the copy of this file that is given to the job
will have its ownership set to that of the user running the job.
This requires the trust of the user.
It is therefore recommended to avoid this mechanism when possible.
Instead,
require users to submit jobs with \Attr{X509UserProxy}
set in the submit description  file.
If this feature is needed,
use the boolean expression to only allow specific values of \Expr{target.Owner}
to use this shared proxy file.
The shared proxy file should be owned by the \Login{condor} user.
Currently, to use a shared proxy, the job must also
turn on sandboxing by having the attribute \Attr{JobShouldBeSandboxed}.

\index{Job Router Routing Table ClassAd attribute!SharedX509UserProxy}
\item[SharedX509UserProxy]
A string representing file containing the X.509 user proxy for the routed job.

\index{Job Router Routing Table ClassAd attribute!JobShouldBeSandboxed}
\item[JobShouldBeSandboxed] A boolean expression
that when \Expr{True} causes the created copy of the job to be sandboxed.
A copy of the input files will be placed in the
\Condor{schedd} daemon's spool area for the target job,
and when the job runs,
the output will be staged back into the spool area.
Once all of the output has been successfully staged back,
it will be copied again,
this time from the spool area of the sandboxed job back to the
original job's output locations.
By default, sandboxing is turned off.
Only to turn it on if using a shared X.509
user proxy or if direct staging of remote output files
back to the final output locations is not desired.

\index{Job Router Routing Table ClassAd attribute!OverrideRoutingEntry}
\item[OverrideRoutingEntry] A boolean value that when \Expr{True},
indicates that this entry
in the routing table replaces any previous entry in the table
with the same name.
When \Expr{False}, it indicates that if there is a
previous entry by the same name, the previous entry should be retained
and this entry should be ignored.
The default value is \Expr{True}.

\index{Job Router Routing Table ClassAd attribute!Set\_<ATTR>}
\item[Set\_<ATTR>] Sets the value of \Attr{<ATTR>} in the routed
job ClassAd to the specified value.  An example of
an attribute that might be set is \Attr{PeriodicRemove}.
For example, if the routed job goes on hold or stays idle for too long,
remove it and return the original copy of the job to a normal state.

\index{Job Router Routing Table ClassAd attribute!Eval\_Set\_<ATTR>}
\item[Eval\_Set\_<ATTR>] Defines an expression.
The expression is evaluated, and the resulting value
sets the value of the routed copy's job ClassAd attribute \Attr{<ATTR>}.

\index{Job Router Routing Table ClassAd attribute!Copy\_<ATTR>}
\item[Copy\_<ATTR>] Defined with the name of a routed copy ClassAd
attribute. Copies the value of \Attr{<ATTR>} from the
original job ClassAd into the specified attribute named of the routed copy.
Useful to save the value of an
expression, before replacing it with something else that references the
original expression.

\index{Job Router Routing Table ClassAd attribute!Delete\_<ATTR>}
\item[Delete\_<ATTR>] Deletes \Attr{<ATTR>} from the routed copy
ClassAd.  A value assigned to this attribute in the routing table
entry is ignored.

\index{Job Router Routing Table ClassAd attribute!EditJobInPlace}
\item[EditJobInPlace] A boolean expression that, when \Expr{True},
causes the original job to be transformed in place rather than creating a new
transformed version (a routed copy) of the job.  
In this mode, the Job Router Hook
\Macro{<Keyword>\_HOOK\_TRANSLATE\_JOB} and transformation
rules in the routing table are applied during the job
transformation.  The routing table attribute \Attr{GridResource} is
ignored, and there is no default transformation of the job from a
vanilla job to a grid universe job as there is otherwise.  Once
transformed, the job is still a candidate for matching routing rules,
so it is up to the routing logic to control whether the job may be
transformed multiple times or not.  For example, to transform the job
only once, an attribute could be set in the job ClassAd to prevent it from
matching the same routing rule in the future.  To transform the job
multiple times with limited frequency, a timestamp could be inserted
into the job ClassAd marking the time of the last transformation, and the
routing entry could require that this timestamp either be undefined
or older than some limit.

\end{description}

%%%%%%%%%%%%%%%%%%%%%%%%%%%%%%%%%%%%%%%%%%%%%%%%%%%%%%%%%%%%%%%%%%%%%%%%%%%%%%%%
\subsection{\label{JobRouterReSSExample}Example: constructing the routing table from ReSS}
%%%%%%%%%%%%%%%%%%%%%%%%%%%%%%%%%%%%%%%%%%%%%%%%%%%%%%%%%%%%%%%%%%%%%%%%%%%%%%%%

The Open Science Grid has a service called ReSS (Resource Selection
Service).  It presents grid sites as ClassAds in a Condor collector.
This example builds a routing table from the site ClassAds in the ReSS
collector.

Using \Macro{JOB\_ROUTER\_ENTRIES\_CMD}, we tell the \Condor{job\_router} daemon to call a
simple script which queries the collector and outputs a routing table.
The script, called \verb|osg_ress_routing_table.sh|, is just this:

\footnotesize
\begin{verbatim}
#!/bin/sh

# you _MUST_ change this:
export condor_status=/path/to/condor_status
# if no command line arguments specify -pool, use this:
export _CONDOR_COLLECTOR_HOST=osg-ress-1.fnal.gov

$condor_status -format '[ ' BeginAd \
              -format 'GridResource = "gt2 %s"; ' GlueCEInfoContactString \
	      -format ']\n' EndAd "$@" | uniq
\end{verbatim}
\normalsize

Save this script to a file and make sure the permissions on the file
mark it as executable.  Test this script by calling it by hand before
trying to use it with the \Condor{job\_router} daemon.  You may supply additional arguments
such as \Opt{-constraint} to limit the sites which are returned.

Once you are satisfied that the routing table constructed by the
script is what you want, configure the \Condor{job\_router} daemon to use it:

\footnotesize
\begin{verbatim}
# command to build the routing table
JOB_ROUTER_ENTRIES_CMD = /path/to/osg_ress_routing_table.sh <extra arguments>

# how often to rebuild the routing table:
JOB_ROUTER_ENTRIES_REFRESH = 3600
\end{verbatim}
\normalsize

Using the example configuration, use the
above settings to replace \Macro{JOB\_ROUTER\_ENTRIES}.  Or,
leave \Macro{JOB\_ROUTER\_ENTRIES} there and have a routing table
containing entries from both sources.  When you restart or reconfigure
the \Condor{job\_router} daemon,
you should see messages in the Job Router's log indicating that it
is adding more routes to the table.
