%%%%%%%%%%%%%%%%%%%%%%%%%%%%%%%%%%%%%%%%%%%%%%%%%%%%%%%%%%%%%%%%%%%%%%%%%%%%%%%%
\section{\label{sec:JobRouter}The Condor Job Router}
%%%%%%%%%%%%%%%%%%%%%%%%%%%%%%%%%%%%%%%%%%%%%%%%%%%%%%%%%%%%%%%%%%%%%%%%%%%%%%%%
\index{Job Router}
\index{Condor daemon!condor\_job\_router@\Condor{job\_router}}
\index{daemon!condor\_job\_router@\Condor{job\_router}}
\index{condor\_job\_router daemon}

The \Condor{job\_router} daemon allows you to specify a policy for routing vanilla universe
jobs to one or more grid sites, through any of the grid protocols
supported by Condor (e.g. gt2, gt4, Condor-C). The idea is to do as
little scheduling in advance as possible and to only feed jobs to the
sites as they consume them. Meanwhile, the jobs waiting to be routed
are ordinary vanilla universe jobs, so they may run in the local
Condor pool or in other pools via flocking. Except for having your
excess jobs queue up in the vanilla universe job queue, you can get a
similar effect by submitting all of your jobs as grid universe jobs
and using Condor-G matchmaking.  However, the router adds some
additional convenience features: tracking of aggregate job states for
use in routing policy, MaxIdleJobs, and blackhole throttling.

The \Condor{job\_router} daemon is most appropriate for high throughput work flows, where you
have many more jobs than computers and you just want to keep as many
of the computers busy as possible.  It is less suitable for a
situation where you have a small number of jobs and you need a
scheduler to choose the best place to run the jobs in order to finish
them as quickly as possible.  The JobRouter doesn't know which site
will run your jobs faster, but it can decide whether to send more jobs
to a site based on whether jobs already submitted to that site are
sitting idle or not and whether you have experienced a lot of recent
job failures at that site.

%%%%%%%%%%%%%%%%%%%%%%%%%%%%%%%%%%%%%%%%%%%%%%%%%%%%%%%%%%%%%%%%%%%%%%%%%%%%%%%%
\subsection{What does it mean to route a job?}
%%%%%%%%%%%%%%%%%%%%%%%%%%%%%%%%%%%%%%%%%%%%%%%%%%%%%%%%%%%%%%%%%%%%%%%%%%%%%%%%

What is meant by \textit{job routing} is this: a vanilla universe job is
transformed into a grid universe job by making a copy of the job
ClassAd and modifying some attributes of the job as described by a
configurable policy.  The copy of the ClassAd shows up in the job
queue as a new job id.  It is referred to as the routed copy of the
job.

Until the routed job finishes or is removed, the original copy of the
job passively mirrors the state of the routed job.  During this time,
it is not available for matchmaking as a normal vanilla universe job,
because it is tied to the routed job.  It also does not evaluate
periodic expressions, such as \Attr{PeriodicHold}.  Only the periodic
expressions of the routed job are evaluated.  When the routed job
completes, the original job ClassAd is updated so that it reflects the
final status of the job.  If the routed job is removed, the original
job returns to the normal idle state and may be available for
matchmaking or rerouting as usual.  If instead the original job is
removed or goes on hold, the routed job is removed.

Not all jobs are suitable for routing.  The following section gives a
more specific example of how job routing can be used and what types of
jobs are suitable.

%%%%%%%%%%%%%%%%%%%%%%%%%%%%%%%%%%%%%%%%%%%%%%%%%%%%%%%%%%%%%%%%%%%%%%%%%%%%%%%%
\subsection{Usage Case: sending excess jobs to the grid}
%%%%%%%%%%%%%%%%%%%%%%%%%%%%%%%%%%%%%%%%%%%%%%%%%%%%%%%%%%%%%%%%%%%%%%%%%%%%%%%%

Suppose you submit jobs to a Condor pool and you would like to send
excess jobs to other available sites, such as resources on the Open
Science Grid.  Here is how you could use JobRouter to make this work:

\begin{enumerate}

\item Jobs appropriate for routing to the grid must not rely on access to
a shared file system or other services that are only available at your
\textit{home} site.  For example, rather than relying on a shared file system
to access input files and write output files, you could use Condor's file
transfer mechanism.  In the Condor submit file:

\begin{verbatim}
should_transfer_files = yes
when_to_transfer_output = ON_EXIT
input_files = input1,input2
output_files = output1,output2
\end{verbatim}

Note that unlike in the vanilla universe, if your job is
transformed into a globus job and you have not explicitly listed
output files, files produced in the working directory of your job will
not be automatically transferred back when the job completes.  Only
files your explicitly list will be returned.

An additional difference between the vanilla universe and running the
job via globus (in this case just gt2) is that globus does not return
any information about the job exit status.  The exit status as
reported in the job ClassAd and user log are therefore always 0 when
the job ran via globus gt2.

\item Your job must satisfy the requirements of the JobRouter.  The example
routing configuration that will be described below requires that your job
define an expression WantJobRouter that evaluates to True.  Example of
what you could put in a submit file:

\begin{verbatim}
+WantJobRouter = True
\end{verbatim}

You could make this expression fancier.  For example, suppose you want
jobs to first be rejected by your local Condor matchmaker before being
candidates for routing to the grid:

\begin{verbatim}
+WantJobRouter = LastRejMatchTime =!= UNDEFINED
\end{verbatim}

\item If necessary, create a grid proxy and specify it in your submit
file.  Example:

\begin{verbatim}
x509userproxy = /tmp/x509up_u275
\end{verbatim}

This is not necessary if the \Condor{job\_router} daemon is configured to add a grid proxy
to your job for you.

\item Submit your job as usual.  Example:

\begin{verbatim}
$ condor_submit job1.sub
\end{verbatim}

where \verb|job1.sub| might look like this:

\begin{verbatim}
universe = vanilla
executable = my_executable
output = job1.stdout
error = job1.stderr
log = job1.ulog
should_transfer_files = true
when_to_transfer_output = on_exit
+WantJobRouter = LastRejMatchTime =!= UNDEFINED
x509userproxy = /tmp/x509up_u275
queue
\end{verbatim}

\item Watch the status of the job.  As usual, the job \textit{user log} will
contain the final completion status of the job.  Before it finishes,
you can see the status of it by looking in the Condor job queue.  Once
it is picked up by the \Condor{job\_router} daemon, a second job will enter the job queue.
This is the transformed copy of the original job, turned into a grid
universe job.

To see the full job queue, use \Condor{q} as usual.  To
see a more specialized view of the routed jobs, use
\Condor{router\_q}.  Example:

\begin{verbatim}
$ condor_router_q -S
   JOBS ST Route      GridResource
     40  I Site1      site1.edu/jobmanager-condor
     10  I Site2      site2.edu/jobmanager-pbs
      2  R Site3      condor submit.site3.edu condor.site3.edu
\end{verbatim}

To see history of routed jobs, use \Condor{router\_history}.  Example:

\begin{verbatim}
$ condor_router_history
Routed job history from 2007-06-27 23:38 to 2007-06-28 23:38

Site            Hours    Jobs    Runs
                      Completed Aborted
-------------------------------------------------------
Site1              10       2     0
Site2               8       2     1
Site3              40       6     0
-------------------------------------------------------
TOTAL              58      10     1
\end{verbatim}

\end{enumerate}


%%%%%%%%%%%%%%%%%%%%%%%%%%%%%%%%%%%%%%%%%%%%%%%%%%%%%%%%%%%%%%%%%%%%%%%%%%%%%%%%
\subsection{Example Configuration}\label{ExampleJobRouterConfiguration}
%%%%%%%%%%%%%%%%%%%%%%%%%%%%%%%%%%%%%%%%%%%%%%%%%%%%%%%%%%%%%%%%%%%%%%%%%%%%%%%%

This is a specific example of how you could configure the \Condor{job\_router} daemon to send
jobs to grid sites.  A general discussion of configuration options
will follow.

This example sets up three routes for jobs.  One is a Condor site
accessed via the Globus gt2 protocol.  Another is a PBS site also
accessed via Globus gt2.  The third site is a Condor site accessed by
schedd-to-schedd job submission, a.k.a Condor-C.  The \Condor{job\_router} daemon
doesn't know which site would be best for a given job, but, as
specified in the following policy, it will stop sending more jobs to a
site if ten jobs that have already been sent there are idle.

These configuration settings should be made in the local config file
of the submit machine.  If you have not already successfully submitted
grid jobs from this machine, it is a good idea to get that working
before you attempt to use JobRouter.  Typically, the only thing you
need to add (supposing you are using GSI authentication for the grid)
is an X509 trusted certification authority directory in a place
recognized by Condor (e.g. /etc/grid-security/certificates).  The VDT
(\URL{http://vdt.cs.wisc.edu}) provides a convenient way to setup and
install trusted CAs if you are using one of the common CAs in your
grid.

\begin{verbatim}

# These settings become the default settings for all routes
JOB_ROUTER_DEFAULTS = \
  [ \
    requirements=target.WantJobRouter is True; \
    MaxIdleJobs = 10; \
    MaxJobs = 200; \
\
    /* now modify routed job attributes */ \
    /* remove routed job if it goes on hold or stays idle for over 6 hours */ \
    set_PeriodicRemove = JobStatus == 5 || \
                        (JobStatus == 1 && (CurrentTime - QDate) > 3600*6); \
    delete_WantJobRouter = true; \
    set_requirements = true; \
  ]

# This could be made an attribute of the job, rather than being hard-coded
ROUTED_JOB_MAX_TIME = 1440

# Now we define each of the routes to send jobs on
JOB_ROUTER_ENTRIES = \
   [ GridResource = "gt2 site1.edu/jobmanager-condor"; \
     name = "Site 1"; \
   ] \
   [ GridResource = "gt2 site2.edu/jobmanager-pbs"; \
     name = "Site 2"; \
     set_GlobusRSL = "(maxwalltime=$(ROUTED_JOB_MAX_TIME))(jobType=single)"; \
   ] \
   [ GridResource = "condor submit.site3.edu condor.site3.edu"; \
     name = "Site 3"; \
     set_remote_jobuniverse = 5; \
   ]


# Reminder: you must restart Condor for changes to DAEMON_LIST to take effect.
DAEMON_LIST = $(DAEMON_LIST) JOB_ROUTER

# For testing, set this to a small value to speed things up.
# Once you are running at large scale, set it to a higher value
# to prevent the JobRouter from using too much cpu.
JOB_ROUTER_POLLING_PERIOD = 10

#It is good to save lots of schedd queue history
#for use with the router_history command.
MAX_HISTORY_ROTATIONS = 20
\end{verbatim}


Some questions you may have after reading the above policy: Can the
routing table be dynamically generated from grid information systems?
Do users have to have their own grid credentials or can the \Condor{job\_router} daemon
insert service credentials for them?  What's up with the syntax of the
routing table: C-style comments, strange ClassAd expressions, escaped
end of lines?  The next section covers the specifics of the \Condor{job\_router} daemon
configuration.  Read on!

%%%%%%%%%%%%%%%%%%%%%%%%%%%%%%%%%%%%%%%%%%%%%%%%%%%%%%%%%%%%%%%%%%%%%%%%%%%%%%%%
\subsection{The \Condor{job\_router} Daemon Configuration Details}
%%%%%%%%%%%%%%%%%%%%%%%%%%%%%%%%%%%%%%%%%%%%%%%%%%%%%%%%%%%%%%%%%%%%%%%%%%%%%%%%

The \Condor{job\_router} daemon is configured with a \textit{routing table}, which is a list
of ClassAds describing each site where jobs may be sent.  The ClassAd
syntax is slightly different from much of the rest of Condor, because
it uses \emph{New ClassAds}, a re-implementation of ClassAds that
Condor is gradually adopting and which may one day completely replace
the old implementation.  A good place to learn about the syntax of New
ClassAds is the Informal Language Description in the C++ ClassAds
tutorial: \URL{http://www.cs.wisc.edu/condor/classad/c++tut.html}.
For the most part, everything in the old ClassAds language is
supported by New ClassAds, with the exception of a number of ClassAd
functions that have not yet been added to New ClassAds.  So if job
ClassAds make use of ClassAd functions, they cannot currently be
routed.

Since the \Condor{job\_router} daemon is configured with New ClassAds but is operating on
Old ClassAds stored in the job queue, it may be confusing at first to
understand which ClassAd expressions are evaluated as New ClassAds and
which are evaluated as Old ClassAds.  For example, the
\Attr{requirements} expression of routes in the routing table are
evaluated by the \Condor{job\_router} daemon, so they may use New ClassAds features,
whereas the \Attr{PeriodicHold} expression in the routed job is
evaluated by \Condor{schedd}, so it may only use features of Old
ClassAds.  As long as the expressions you use in the routing table are
compatible with both implementations of ClassAds, you do not need to
be concerned about this detail.  In case you do need to use special
features, the expressions that are evaluated (as New ClassAds) by the
\Condor{job\_router} daemon will be identified in the reference below.

The most basic thing to know about New ClassAd syntax is simply that
each ClassAd is surrounded by square brackets, and each assignment
statement in the ClassAd should end with a semicolon.  When the
ClassAd is embedded in a Condor configuration file, it could all
appear on a single line, but the readability of the ClassAd is often
improved by inserting line continuations (i.e. backslashes followed by
newlines) after each assignment statement in the ClassAd, as in the
examples above.  Unfortunately, this makes it a little awkward to
insert configuration comments in the ClassAd, because of the way line
continuations and the Condor configuration comment character `\#' work.
One alternative is to use C-style comments \verb|/* ... */| as in the
examples above.  Another option is to read in the \Condor{job\_router} daemon entries
from a separate file, rather than embedding them in the Condor
configuration file.

The \Condor{job\_router} daemon configuration parameters are listed below:

\begin{description}

\item[\Macro{JOB\_ROUTER\_DEFAULTS}] \label{param:JobRouterDefaults}
This is a single New ClassAd that is used to provide default values
for all routes in the \Condor{job\_router} daemon's routing table.  The meaning of the
attributes in this ad is identical to the attributes of
\MacroNI{JOB\_ROUTER\_ENTRIES}.  If a route ClassAd sets an
attribute by the same name, this overrides the attribute from the
defaults.

\item[\Macro{JOB\_ROUTER\_ENTRIES}] \label{param:JobRouterEntries}
This specifies the job routing table.  It is a list of New ClassAds
(each surrounded by []'s and separated from each other by spaces).
Each entry in the routing table describes a site you want jobs to be
routed to.

You may modify the routing table and issue \Condor{reconfig} to have
the \Condor{job\_router} daemon rebuild the routing table.  If you do so and you change the
name of a route, then the count of currently routed jobs on that route
will not be accurate until the existing jobs finish.  Another option
if you want dynamic routing is to read the routing table from an
external source via \Macro{JOB\_ROUTER\_ENTRIES\_FILE} or
\Macro{JOB\_ROUTER\_ENTRIES\_CMD}.

The meaning of each attribute in the routing entries is listed below.


{\Large Job Route ClassAd Attributes}

\begin{description}

\item[GridResource] This specifies the GridResource attribute that
will be inserted into the routed job ClassAd.

\item[name] This is an optional identifier that will be used in log
messages concerning this route.  If no name is specified, the default
is the value of \Attr{GridResource}.

\item[requirements] This is a requirements expression (in New ClassAd
syntax) that specifies which jobs may be matched to the route.  Note
that, as with all settings, if you specify the requirements in
\MacroNI{JOB\_ROUTER\_ENTRIES}, it overrides the setting in
\MacroNI{JOB\_ROUTER\_DEFAULTS}.  To specify global requirements, that
are not overridden by \MacroNI{JOB\_ROUTER\_ENTRIES}, use
\MacroNI{JOB\_ROUTER\_SOURCE\_JOB\_CONSTRAINT} instead.

\item[MaxJobs] This is the maximum number of jobs to allow on the route at
one time. The default is 100.

\item[MaxIdleJobs] This is the maximum number of routed jobs in the
idle state.  At or above this level, no more jobs will be routed.
This is intended to prevent too many jobs from being sent to sites
which are too busy to run them.  The disadvantage of setting this too
small is that it will slow down the rate of job submission to the
site, because the \Condor{job\_router} daemon will submit jobs up to this limit, wait to see
some of them enter the running state, and then submit more.  The
disadvantage of setting it too high is that a lot of jobs may be sent
to a site only to site idle for hours or days.  The default is 50.

\item[FailureRateThreshold] Maximum tolerated rate of job failures.
Failure is determined by the \Attr{JobFailureTest} expression.  The
default threshold is 0.03 jobs/sec.  If the failure rate is exceeded,
submission of new jobs is throttled until jobs begin succeeding and
failures are less than the threshold.

\item[JobFailureTest] This is an expression (in New ClassAds syntax),
evaluated for each job that finishes to determine whether it was a
failure.  The default is to assume the job was a success.  An example
expression to treat all jobs running for less than 30 minutes as
failures: other.RemoteWallClockTime < 1800.  A more flexible
expression could reference a property or expression of the job that
specifies a failure condition specific to the type of job.  Routed
jobs that are removed (e.g. by \Attr{PeriodicRemove}) are considered
to be failures.

\item[TargetUniverse] This is an integer specifying the desired
universe for the routed copy of the job.  The default is 9, which
means the grid universe.

\item[UseSharedX509UserProxy] If this (New ClassAds) expression
evaluates to true, then the value of \Attr{SharedX509UserProxy} will
be used as the x509 user proxy for the routed job.  NOTE: if the \Condor{job\_router} daemon
is running as root, the copy of this file that is given to the job
will have its ownership set to that of the user running the job.  This
means you must trust the user to have access to the proxy file.  It is
therefore recommended to avoid this mechanism when possible.  (For
example, simply require users to submit jobs with \Attr{X509UserProxy}
set in their submit file.)  If you do need this feature, use the
boolean expression to only allow specific values of \Expr{other.Owner}
to use this shared proxy file.  The shared proxy file should be owned
by the \Login{condor} user.  Currently, to use a shared proxy, you must also
turn on sandboxing with \Attr{JobShouldBeSandboxed}.

\item[SharedX509UserProxy] See \Attr{UseSharedX509UserProxy}.

\item[JobShouldBeSandboxed] If this (New ClassAds) expression
evaluates to true, the copy of the job that is created will be
sandboxed.  This means a copy of the input files will be placed in the
schedd's spool area for the target job, and when the job runs, the
output will be staged back into the spool area.  Once all of the
output has been successfully staged back, then it will be copied
again, this time from the spool area of the sandboxed job back to the
original job's output locations.  By default, sandboxing is turned
off.  You would only want to turn it on if you are using a shared x509
user proxy or if you do not want direct staging of remote output files
back to the final output locations.

\item[OverrideRoutingEntry] When true, this indicates that this entry
in the routing table should replace any previous entry in the table
with the same name.  When false, it indicates that if there is a
previous entry by the same name, the previous entry should be retained
and this entry should be ignored.  The default is true.

\item[set\_ATTRIBUTE] Sets the value of \Attr{ATTRIBUTE} in the routed
job ClassAd to the specified value.  The \Attr{PeriodicRemove}
expression is an example of something you might want to set in the
routed job.  (For example, if the routed job goes on hold or stays
idle for too long, remove it and return the original copy of the job to
a normal state.)

\item[eval\_set\_ATTRIBUTE] Sets the value of \Attr{ATTRIBUTE} in the
routed job ClassAd to the result gotten by evaluating the specified
expression.  Note that the expression is evaluated using New ClassAds
syntax.  One reason to force evaluation by the \Condor{job\_router} daemon is when you need
to use New ClassAds features in the expression.  Otherwise, the
unevaluated expression is inserted into the ClassAd and if it is
incompatible with Old ClassAds, the ClassAd conversion will fail when
the routed job is submitted to \Condor{schedd}.

\item[copy\_ATTRIBUTE] Copies the value of \Attr{ATTRIBUTE} from the
original job ClassAd into the specified attribute of the routed job
ClassAd.  This may be useful if you need to save the value of an
expression before replacing it with something else that references the
original expression.

\item[delete\_ATTRIBUTE] Deletes \Attr{ATTRIBUTE} from the routed job
ClassAd.  The value that you assign to this attribute in the routing
entry is ignored.

\end{description}

\item[\Macro{JOB\_ROUTER\_ENTRIES\_FILE}] \label{param:JobRouterEntriesFile}
This specifies a file that contains the ClassAds describing the
routing table.  The format of this file is the same as
\Macro{JOB\_ROUTER\_ENTRIES}, except that it is interpreted directly
by the ClassAd parser, rather than first by the Condor configuration
system.  The biggest difference therefore is that you do not need to
use line continuations (backslashes before newlines).  The specified
file is periodically reread to check for new information.  This
happens every \Macro{JOB\_ROUTER\_ENTRIES\_REFRESH} seconds.

\item[\Macro{JOB\_ROUTER\_ENTRIES\_CMD}]
\label{param:JobRouterEntriesCmd} This specifies an external program
(plus arguments) to run.  The output of the program should be ClassAds
describing the routing table.  The format of this file is the same as
\Macro{JOB\_ROUTER\_ENTRIES}, except that it is interpreted directly
by the ClassAd parser, rather than first by the Condor configuration
system.  The biggest difference therefore is that you do not need to
use line continuations (backslashes before newlines).  The specified
command is periodically rerun to check for new information.  This
happens every \Macro{JOB\_ROUTER\_ENTRIES\_REFRESH} seconds.  If you
need to enter spaces in any arguments or in the command name itself,
surround the whole configuration value with double quotes and use
single quotes around individual arguments that contain spaces, just
like when dealing with spaces in job arguments in a Condor submit
file.  For an example of \MacroNI{JOB\_ROUTER\_ENTRIES\_CMD}, see
page~\pageref{JobRouterReSSExample}.

\item[\Macro{JOB\_ROUTER\_ENTRIES\_REFRESH}]
\label{param:JobRouterEntriesRefresh} This specifies the number of
seconds between checks for updates to the routing table described by
\Macro{JOB\_ROUTER\_ENTRIES\_FILE} or
\Macro{JOB\_ROUTER\_ENTRIES\_CMD}.  The default is 0, meaning updates
will only be read whenever \Condor{reconfig} is run or
\Condor{job\_router} restarts.

\item[\Macro{JOB\_ROUTER\_SOURCE\_JOB\_CONSTRAINT}]
\label{JobRouterSourceJobConstraint} This specifies global
requirements that must be met by all routed jobs in addition to
requirements specified within the routing table.

\item[\Macro{JOB\_ROUTER\_MAX\_JOBS}] \label{JobRouterMaxJobs} This
specifies the maximum number of jobs that may be routed summed over
all routes.  The default is -1, which means unlimited.

\item[\Macro{MAX\_JOB\_MIRROR\_UPDATE\_LAG}]
\label{MaxJobMirrorUpdateLag} This specifies the maximum number of
seconds the \Condor{job\_router} daemon should wait before giving up when waiting to see
routed jobs appear in the \Condor{schedd}'s job queue log file.  The
default is 600.  We do not expect that you will need to adjust this.

\item[\Macro{JOB\_ROUTER\_POLLING\_PERIOD}]
\label{JobRouterPollingPeriod} This specifies the number of seconds
between cycles in the \Condor{job\_router} daemon's task loop.  The default is 10
seconds.  A small value makes the \Condor{job\_router} daemon quick to see new candidate
jobs for routing.  A large value makes the \Condor{job\_router} daemon generate less
overhead at the cost of being slower to see new candidates for
routing.  If you have very large job queues and a few minutes of
routing latency is no problem for you, increasing this to a few
hundred seconds would be reasonable.

\item[\Macro{JOB\_ROUTER\_NAME}] \label{JobRouterName} This is only
useful if you wish to run multiple instances of the \Condor{job\_router} daemon in the same
instance of Condor on a given machine.  Each one must have a different
name (or all but the first to start up will refuse to run).

\end{description}

%%%%%%%%%%%%%%%%%%%%%%%%%%%%%%%%%%%%%%%%%%%%%%%%%%%%%%%%%%%%%%%%%%%%%%%%%%%%%%%%
\subsection{Example: constructing the routing table from ReSS}
\label{JobRouterReSSExample}
%%%%%%%%%%%%%%%%%%%%%%%%%%%%%%%%%%%%%%%%%%%%%%%%%%%%%%%%%%%%%%%%%%%%%%%%%%%%%%%%

The Open Science Grid has a service called ReSS (Resource Selection
Service).  It presents grid sites as ClassAds in a Condor collector.
This example builds a routing table from the site ClassAds in the ReSS
collector.

Using \Macro{JOB\_ROUTER\_ENTRIES\_CMD}, we tell the \Condor{job\_router} daemon to call a
simple script which queries the collector and outputs a routing table.
The script, called \verb|osg_ress_routing_table.sh|, is just this:

\begin{verbatim}
#!/bin/sh

# you _MUST_ change this:
export condor_status=/path/to/condor_status
# if no command line arguments specify -pool, use this:
export _CONDOR_COLLECTOR_HOST=osg-ress-1.fnal.gov

$condor_status -format '[ ' BeginAd \
              -format 'GridResource = "gt2 %s"; ' GlueCEInfoContactString \
	      -format ']\n' EndAd "$@" | uniq
\end{verbatim}

Save this script to a file and make sure the permissions on the file
mark it as executable.  Test this script by calling it by hand before
trying to use it with the \Condor{job\_router} daemon.  You may supply additional arguments
such as \Opt{-constraint} to limit the sites which are returned.

Once you are satisfied that the routing table constructed by the
script is what you want, configure the \Condor{job\_router} daemon to use it:

\begin{verbatim}
# command to build the routing table
JOB_ROUTER_ENTRIES_CMD = /path/to/osg_ress_routing_table.sh <extra arguments>

# how often to rebuild the routing table:
JOB_ROUTER_ENTRIES_REFRESH = 3600
\end{verbatim}

Using the previous example of the \Condor{job\_router} daemon configuration on
page~\pageref{ExampleJobRouterConfiguration}, you may simply use the
above settings to replace \Macro{JOB\_ROUTER\_ENTRIES}.  (Or you may
leave \Macro{JOB\_ROUTER\_ENTRIES} there and have a routing table
containing entries from both sources.)  When you restart or reconfig
the \Condor{job\_router} daemon, you should see messages in JobRouterLog indicating that it
is adding more routes to the table.
