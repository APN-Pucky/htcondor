%%%%%%%%%%%%%%%%%%%%%%%%%%%%%%%%%%%%%%%%%%%%%%%%%%%%%%%%%%%%%%%%%%%%%%%%%%%%%%%%
\section{\label{sec:Deployment}Dynamic Deployment}
%%%%%%%%%%%%%%%%%%%%%%%%%%%%%%%%%%%%%%%%%%%%%%%%%%%%%%%%%%%%%%%%%%%%%%%%%%%%%%%%
\index{dynamic deployment!relevance to grid computing}

See section~\ref{sec:Dynamic-Deployment}
for a complete description of HTCondor's dynamic deployment tools.

HTCondor's dynamic deployment tool,
\Condor{cold\_start},
allows new pools of resources to be  incorporated on the fly.
While HTCondor is able to manage compute jobs remotely 
through Globus and other grid-computing protocols,
dynamic deployment of HTCondor makes it possible to go one step further.
HTCondor remotely installs and runs portions of itself.
This process inhabits computing resources
on demand.  
It leverages the lowest common denominator of grid middleware systems,
simple program  execution,
to bind together resources in a heterogeneous computing grid,
with different management policies and different job execution methods,
into a full-fledged HTCondor system. 

The mobility of HTCondor services also benefits from
the development  of HTCondor-C,
which provides a richer tool set for interlinking HTCondor-managed computers.
HTCondor-C is a protocol that allows one HTCondor scheduler
to delegate jobs to another HTCondor scheduler.
The second scheduler could be at a remote site and/or an entry point
into a restricted network.
Delegating details of managing a job
achieves greater flexibility with respect to network architecture,
as well as fault tolerance and scalability.
In the  context of deployments,
the beach-head for each compute site is a dynamically deployed
HTCondor scheduler which then serves as a target for HTCondor-C traffic.

In general,
the mobility of the HTCondor scheduler and job execution agents,
and the flexibility in how these are interconnected
provide a uniform and feature-rich platform
that can expand onto diverse resources and environments
when the user requires it.


