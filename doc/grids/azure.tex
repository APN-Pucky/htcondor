%%%%%%%%%%%%%%%%%%%%%%%%%%%%%%%%%%%%%%%%%%%%%%%%%%%%%%%%%%%%%%%%%%%%%%%%%%%
\subsection{\label{sec:Azure}The Azure Grid Type }
%%%%%%%%%%%%%%%%%%%%%%%%%%%%%%%%%%%%%%%%%%%%%%%%%%%%%%%%%%%%%%%%%%%%%%%%%%%
\index{Azure}
\index{Azure grid jobs}
\index{grid computing!submitting jobs to Azure}
\index{grid type!azure}

HTCondor jobs may be submitted to the Microsoft Azure
cloud service.
Azure is an on-line commercial service that provides
the rental of computers by the hour to run computational applications.
It runs virtual machine images that have been uploaded to Azure's
servers.
More information about Azure is available at
\URL{https://azure.microsoft.com}.

%%%%%%%%%%%%%%%%%%%%%%%%%%%%%%%%%%%%%%%%%%%%%%%%%%%%%%%%%%%%%%%%%%%%%%%%%%%
\subsubsection{\label{sec:Azure-submit}Azure Job Submission}
%%%%%%%%%%%%%%%%%%%%%%%%%%%%%%%%%%%%%%%%%%%%%%%%%%%%%%%%%%%%%%%%%%%%%%%%%%%

HTCondor jobs are submitted to the Azyre service
with the \SubmitCmdNI{grid} universe, setting the
\SubmitCmd{grid\_resource} command to \SubmitCmdNI{azure}, followed
by your Azure subscription id.
The submit description file command will be similar to:
\begin{verbatim}
grid_resource = azure 4843bfe3-1ebe-423e-a6ea-c777e57700a9
\end{verbatim}

Since the HTCondor job is a virtual machine image,
most of the submit description file commands
specifying input or output files are not applicable.
The \SubmitCmd{executable} command is still required,
but its value is ignored.
It identifies different jobs in the output of \Condor{q}.

The VM image for the job must already be registered a virtual
machine image in Azure.
In the submit description file,
provide the identifier for the image using
the \SubmitCmd{azure\_image} command.

This grid type requires granting HTCondor permission to use your
Azure account.
The easiest way to do this is to use the \Prog{az}
command-line tool distributed by Microsoft.
Find \Prog{az} and documentation for it at
\URL{https://docs.microsoft.com/en-us/cli/azure/?view=azure-cli-latest}.
After installation of \Prog{az},
run \Prog{az login} and follow its directions.
Once done with that step,
the tool will write authorization credentials in a file
under your HOME directory.
HTCondor will use these credentials to communicate with Azure.
% TODO better description of need for azure CLI and how to configure

You can also set up a service account in Azure for HTCondor to use.
This lets you limit the level of acccess HTCondor has to your Azure
account.
Instructions for creating a service account can be found here:
\URL{http://research.cs.wisc.edu/htcondor/gahp/AzureGAHPSetup.docx}.
% TODO Better description of creating auth file

Once you have created a file containing the service account credentials,
you can specify its location in
the submit description file using the \SubmitCmd{azure\_auth\_file} command,
as in the example:
\begin{verbatim}
azure_auth_file = /path/to/auth-file
\end{verbatim}

Azure allows the choice of different hardware configurations
for instances to run on.
Select which configuration to use for the \SubmitCmdNI{azure} grid type
with the \SubmitCmd{azure\_size} submit description file command.
HTCondor provides no default.

Azure has many locations where instances can be run (i.e. multiple
data centers distributed throughout the world).
You can select which location to use with the \SubmitCmd{azure\_location}
submit description file command.

Azure creates an administrator account within each instance, which you
can log into remote via SSH.
You can select the name of the account with the
\SubmitCmd{azure\_admin\_username} command.
You can supply the name of a file containing an SSH public key that
will allow access to the administrator account with the
\SubmitCmd{azure\_admin\_key} command.

%TODO add text about azure_image
